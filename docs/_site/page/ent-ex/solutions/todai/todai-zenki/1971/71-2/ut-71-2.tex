\documentclass[a4j]{jarticle}
\usepackage{amsmath}
\usepackage{ascmac}
\usepackage{amssymb}
\usepackage{enumerate}
\usepackage{multicol}
\usepackage{framed}
\usepackage{fancyhdr}
\usepackage{latexsym}
\usepackage{indent}
\usepackage{cases}
\usepackage[dvips]{graphicx}
\usepackage{color}
\usepackage{emath}
\usepackage{emathPp}
\allowdisplaybreaks
\pagestyle{fancy}
\lhead{}
\chead{}
\rhead{東京大学前期$1971$年$2$番}
\begin{document}
%分数関係


\def\tfrac#1#2{{\textstyle\frac{#1}{#2}}} %数式中で文中表示の分数を使う時


%Σ関係

\def\dsum#1#2{{\displaystyle\sum_{#1}^{#2}}} %文中で数式表示のΣを使う時


%ベクトル関係


\def\vector#1{\overrightarrow{#1}}  %ベクトルを表現したいとき(aベクトルを表現するときは\ver
\def\norm#1{|\overrightarrow{#1}|} %ベクトルの絶対値
\def\vtwo#1#2{ \left(%
      \begin{array}{c}%
      #1 \\%
      #2 \\%
      \end{array}%
      \right) }                        %2次元ベクトル成分表示
      
      \def\vthree#1#2#3{ \left(
      \begin{array}{c}
      #1 \\
      #2 \\
      #3 \\
      \end{array}
      \right) }                        %3次元ベクトル成分表示



%数列関係


\def\an#1{\verb|{|$#1$\verb|}|}


%極限関係

\def\limit#1#2{\stackrel{#1 \to #2}{\longrightarrow}}   %等式変形からの極限
\def\dlim#1#2{{\displaystyle \lim_{#1\to#2}}} %文中で数式表示の極限を使う



%積分関係

\def\dint#1#2{{\displaystyle \int_{#1}^{#2}}} %文中で数式表示の積分を使う時

\def\ne{\nearrow}
\def\se{\searrow}
\def\nw{\nwarrow}
\def\ne{\nearrow}


%便利なやつ

\def\case#1#2{%
 \[\left\{%
 \begin{array}{l}%
 #1 \\%
 #2%
 \end{array}%
 \right.\] }                           %場合分け
 
\def\1{$\cos\theta=c$,$\sin\theta=s$とおく.}  %cs表示を与える前書きシータ
\def\2{$\cos t=c$,$\sin t=s$とおく.}     %cs表示を与える前書きt
\def\3{$\cos x=c$,$\sin x=s$とおく.}                %cs表示を与える前書きx

\def\fig#1#2#3 {%
\begin{wrapfigure}[#1]{r}{#2 zw}%
\vspace*{-1zh}%
\input{#3}%
\end{wrapfigure} }           %絵の挿入


\def\a{\alpha}   %ギリシャ文字
\def\b{\beta}
\def\g{\gamma}

%問題番号のためのマクロ

\newcounter{nombre} %必須
\renewcommand{\thenombre}{\arabic{nombre}} %任意
\setcounter{nombre}{2} %任意
\newcounter{nombresub}[nombre] %親子関係を定義
\renewcommand{\thenombresub}{\arabic{nombresub}} %任意
\setcounter{nombresub}{0} %任意
\newcommand{\prob}[1][]{\refstepcounter{nombre}#1[問題 \thenombre]}
\newcommand{\probsub}[1][]{\refstepcounter{nombresub}#1(\thenombresub)}


%1-1みたいなカウンタ(todaiとtodaia)
\newcounter{todai}
\setcounter{todai}{0}
\newcounter{todaisub}[todai] 
\setcounter{todaisub}{0} 
\newcommand{\todai}[1][]{\refstepcounter{todai}#1 \thetodai-\thetodaisub}
\newcommand{\todaib}[1][]{\refstepcounter{todai}#1\refstepcounter{todaisub}#1 {\bf [問題 \thetodai.\thetodaisub]}}
\newcommand{\todaia}[1][]{\refstepcounter{todaisub}#1 {\bf [問題 \thetodai.\thetodaisub]}}


     \begin{oframed}
     正数$x$を与えて,
          \[2a_1=x,2a_2=a_1^2+1,\cdots,2a_{n+1}=a_n^2+1,\cdots\]
     のように数列$\{a_n\}$を定める時,
          \begin{enumerate}[(1)]
          \item $x\not=2$ならば,$a_1<a_2<\cdots<a_n<\cdots$となることを証明せよ.
          \item $x<2$ならば,$a_n<1$となることを証明せよ.このとき,正数$\epsilon$を$1-x/2$より
          小となるようにとって,$a_1,a_2,\cdots,a_n$までが$1-\epsilon$以下となったとすれば,
          個数$n$について次の不等式が成り立つことを証明せよ.
               \[2-x>n\epsilon^2\]
          \end{enumerate}
     \end{oframed}

\setlength{\columnseprule}{0.4pt}
\begin{multicols}{2}
{\bf[解]} 
\begin{enumerate}[(1)]
\item 与えられた漸化式より,
     \begin{align}
     &a_n<a_{n+1}\nonumber\\
     \Longleftrightarrow&2a_n<a_n^2+1\nonumber\\
     \Longleftrightarrow&(a_n-1)^2>0\nonumber\\
     \Longleftrightarrow&a_n\not=1\label{1}
     \end{align}
である.$x\not=2$ならば,$a_1\not=1$だから,漸化式より帰納的に
$a_n\not=1$である.これと\eqref{1}より題意は示された.$\Box$     
  
 \item 漸化式および$x>0$から,$a_n>0$であることに注意すると,$a_{n+1}$は$a_n$の単調増加関数である.
$x<2$ならば$a_1<1$である.そこで以下$k\in\mathbb{N}$に対して$a_k<1$が成立すると仮定すると
     \begin{align*}
     a_{k+1}=\frac{a_k^2+1}{2}<\frac{1+1}{2}=1
     \end{align*}
故に帰納法により$\forall n,a_n<1$となる.$\Box$

次に,後半部分を考える.題意より
     \begin{align}
          \left\{\begin{array}{ll}
          1-\dfrac{x}{2}>\epsilon\\
          a_k<1-\epsilon&k=1,2,\cdots,n
          \end{array}\right.\label{2}
      \end{align}
である.第一式から,$a_1<1-\epsilon$が保証される.

ここで,漸化式より,$k\le n$のとき,
     \begin{align*}
     &2(a_{k+1}-a_k)=(a_k-1)^2>\epsilon^2&(\because\eqref{2})
     \end{align*}     
となる.$k$について和をとって,
     \begin{align*}
     2(a_{n+1}-a_1)=\sum_{k=1}^n (a_k-1)^2>n\epsilon^2 \\
     a_{n+1}-\frac{x}{2}>\frac{n\epsilon^2}{2}
     \end{align*}
前半部分から$a_{n+1}<1$だから,    
     \begin{align*}
     1-\frac{x}{2}>\frac{n\epsilon^2}{2} \\
     2-x>n\epsilon^2
     \end{align*}
となって,題意の不等式を得る.$\Box$
\end{enumerate}
\newpage
\end{multicols}
\end{document}