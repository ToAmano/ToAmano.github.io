\documentclass[a4j]{jarticle}
\usepackage{amsmath}
\usepackage{ascmac}
\usepackage{amssymb}
\usepackage{enumerate}
\usepackage{multicol}
\usepackage{framed}
\usepackage{latexsym}
\begin{document}
%分数関係


\def\tfrac#1#2{{\textstyle\frac{#1}{#2}}} %数式中で文中表示の分数を使う時


%Σ関係

\def\dsum#1#2{{\displaystyle\sum_{#1}^{#2}}} %文中で数式表示のΣを使う時


%ベクトル関係


\def\vector#1{\overrightarrow{#1}}  %ベクトルを表現したいとき(aベクトルを表現するときは\ver
\def\norm#1{|\overrightarrow{#1}|} %ベクトルの絶対値
\def\vtwo#1#2{ \left(%
      \begin{array}{c}%
      #1 \\%
      #2 \\%
      \end{array}%
      \right) }                        %2次元ベクトル成分表示
      
      \def\vthree#1#2#3{ \left(
      \begin{array}{c}
      #1 \\
      #2 \\
      #3 \\
      \end{array}
      \right) }                        %3次元ベクトル成分表示



%数列関係


\def\an#1{\verb|{|$#1$\verb|}|}


%極限関係

\def\limit#1#2{\stackrel{#1 \to #2}{\longrightarrow}}   %等式変形からの極限
\def\dlim#1#2{{\displaystyle \lim_{#1\to#2}}} %文中で数式表示の極限を使う



%積分関係

\def\dint#1#2{{\displaystyle \int_{#1}^{#2}}} %文中で数式表示の積分を使う時

\def\ne{\nearrow}
\def\se{\searrow}
\def\nw{\nwarrow}
\def\ne{\nearrow}


%便利なやつ

\def\case#1#2{%
 \[\left\{%
 \begin{array}{l}%
 #1 \\%
 #2%
 \end{array}%
 \right.\] }                           %場合分け
 
\def\1{$\cos\theta=c$,$\sin\theta=s$とおく.}  %cs表示を与える前書きシータ
\def\2{$\cos t=c$,$\sin t=s$とおく.}     %cs表示を与える前書きt
\def\3{$\cos x=c$,$\sin x=s$とおく.}                %cs表示を与える前書きx

\def\fig#1#2#3 {%
\begin{wrapfigure}[#1]{r}{#2 zw}%
\vspace*{-1zh}%
\input{#3}%
\end{wrapfigure} }           %絵の挿入


\def\a{\alpha}   %ギリシャ文字
\def\b{\beta}
\def\g{\gamma}

%問題番号のためのマクロ

\newcounter{nombre} %必須
\renewcommand{\thenombre}{\arabic{nombre}} %任意
\setcounter{nombre}{2} %任意
\newcounter{nombresub}[nombre] %親子関係を定義
\renewcommand{\thenombresub}{\arabic{nombresub}} %任意
\setcounter{nombresub}{0} %任意
\newcommand{\prob}[1][]{\refstepcounter{nombre}#1[問題 \thenombre]}
\newcommand{\probsub}[1][]{\refstepcounter{nombresub}#1(\thenombresub)}


%1-1みたいなカウンタ(todaiとtodaia)
\newcounter{todai}
\setcounter{todai}{0}
\newcounter{todaisub}[todai] 
\setcounter{todaisub}{0} 
\newcommand{\todai}[1][]{\refstepcounter{todai}#1 \thetodai-\thetodaisub}
\newcommand{\todaib}[1][]{\refstepcounter{todai}#1\refstepcounter{todaisub}#1 {\bf [問題 \thetodai.\thetodaisub]}}
\newcommand{\todaia}[1][]{\refstepcounter{todaisub}#1 {\bf [問題 \thetodai.\thetodaisub]}}


\begin{oframed}
サイコロを$n$回投げて,$xy$平面上の点$P_0$,$P_1$,$\dots$,$P_n$を次の規則(a),(b)によって定める.
     \begin{enumerate}[(a)]
     \item $P_0=(0,0)$ 
     \item $1\le k\le n$のとき,$k$回目に出た目の数が$1,2,3,4$のときには,$P_{k-1}$をそれぞれ
     東,北,西,南に$\left(\dfrac{1}{2}\right)^k$だけ動かした点を$P_k$とする.また$k$回目に出た
     目の数が$5,6$のときには$P_k=P_{k-1}$とする.ただし$y$軸の正の向きを北と定める.
     \end{enumerate}
このとき以下の問いに答えよ.
     \begin{enumerate}[(1)]
     \item $P_n$が$x$軸上にあれば$P_0$,$P_1$,$\dots$,$P_{n-1}$もすべて$x$軸上にあること
     を示せ.
     \item $P_n$が第一象限$\{(x,y)|x>0,y>0\}$にある確率を$n$で示せ.
     \end{enumerate}
\end{oframed}

\setlength{\columnseprule}{0.4pt}
\begin{multicols}{2}
{\bf[解]}
     \begin{enumerate}[(1)]
     \item $l$回目$(a\le l\le n-1,l\in\mathbb{N})$にはじめて南北方向に動いたとすると,移動量は
     $\left(\frac{1}{2}\right)^l$である.ここで$l+1$回目から$n$回目まで全て$l$回目と逆向きに動い
     たとしても,その合計の移動量は
          \begin{align*}
          \left(\frac{1}{2}\right)^{l+1}+ \dots+\left(\frac{1}{2}\right)^n \\
          =\left(\frac{1}{2}\right)^l-\left(\frac{1}{2}\right)^n<\left(\frac{1}{2}\right)^l
          \end{align*}
     だから,$P_n$は$x$軸上には存在しない.以上から題意の対偶が示された.$\Box$
     \item $P_n$が$x$軸上にあるという事象を$X$,$y$軸上にあるという事象を$Y$とする.
     求める確率$a_n$とすると,対称性から他の象限にある確率も$a_n$であるから,
          \begin{align}
          4a_n+P(X\cup Y)=1 \label{1}
          \end{align}
     が成立する.前問の結果から$P_n$が軸上に有るとき,
     $P_k(1\le k\le n)$も軸上にある$x$軸上にあるのは常にサイコロが$2,4,5,6$のとき,
     $y$軸上にあるのは常にサイコロが$1,3,5,6$のときである.したがって
          \begin{align}
           P(X)=P(Y)=\left(\frac{2}{3}\right)^n  \label{2}    
          \end{align}
     また,原点にあるのは,(1)と同様に考えれば常にサイコロが$5,6$であるときだから,
          \begin{align}
          P(X\cap Y)=\left(\frac{1}{3}\right)^n \label{3}
          \end{align}
     である.
          \[P(X\cup Y)=P(X)+P(Y)-P(X\cap Y) \]
     に注意して     
     \eqref{2},\eqref{3}を\eqref{1}に代入すれば
          \begin{align*}
          a_n=\frac{1}{4}\left(1-2\left(\frac{2}{3}\right)^n+\left(\frac{1}{3}\right)^n\right)\cdots\text{(答)}
          \end{align*}
     となる.     
     \end{enumerate}
\newpage
\end{multicols}
\end{document}