\documentclass[a4j]{jarticle}
\usepackage{amsmath}
\usepackage{ascmac}
\usepackage{amssymb}
\usepackage{enumerate}
\usepackage{multicol}
\usepackage{framed}
\usepackage{fancyhdr}
\usepackage{latexsym}
\usepackage{indent}
\usepackage{cases}
\allowdisplaybreaks
\pagestyle{fancy}
\lhead{}
\chead{}
\rhead{東京大学前期$1989$年$1$番}
\begin{document}
%分数関係


\def\tfrac#1#2{{\textstyle\frac{#1}{#2}}} %数式中で文中表示の分数を使う時


%Σ関係

\def\dsum#1#2{{\displaystyle\sum_{#1}^{#2}}} %文中で数式表示のΣを使う時


%ベクトル関係


\def\vector#1{\overrightarrow{#1}}  %ベクトルを表現したいとき(aベクトルを表現するときは\ver
\def\norm#1{|\overrightarrow{#1}|} %ベクトルの絶対値
\def\vtwo#1#2{ \left(%
      \begin{array}{c}%
      #1 \\%
      #2 \\%
      \end{array}%
      \right) }                        %2次元ベクトル成分表示
      
      \def\vthree#1#2#3{ \left(
      \begin{array}{c}
      #1 \\
      #2 \\
      #3 \\
      \end{array}
      \right) }                        %3次元ベクトル成分表示



%数列関係


\def\an#1{\verb|{|$#1$\verb|}|}


%極限関係

\def\limit#1#2{\stackrel{#1 \to #2}{\longrightarrow}}   %等式変形からの極限
\def\dlim#1#2{{\displaystyle \lim_{#1\to#2}}} %文中で数式表示の極限を使う



%積分関係

\def\dint#1#2{{\displaystyle \int_{#1}^{#2}}} %文中で数式表示の積分を使う時

\def\ne{\nearrow}
\def\se{\searrow}
\def\nw{\nwarrow}
\def\ne{\nearrow}


%便利なやつ

\def\case#1#2{%
 \[\left\{%
 \begin{array}{l}%
 #1 \\%
 #2%
 \end{array}%
 \right.\] }                           %場合分け
 
\def\1{$\cos\theta=c$,$\sin\theta=s$とおく.}  %cs表示を与える前書きシータ
\def\2{$\cos t=c$,$\sin t=s$とおく.}     %cs表示を与える前書きt
\def\3{$\cos x=c$,$\sin x=s$とおく.}                %cs表示を与える前書きx

\def\fig#1#2#3 {%
\begin{wrapfigure}[#1]{r}{#2 zw}%
\vspace*{-1zh}%
\input{#3}%
\end{wrapfigure} }           %絵の挿入


\def\a{\alpha}   %ギリシャ文字
\def\b{\beta}
\def\g{\gamma}

%問題番号のためのマクロ

\newcounter{nombre} %必須
\renewcommand{\thenombre}{\arabic{nombre}} %任意
\setcounter{nombre}{2} %任意
\newcounter{nombresub}[nombre] %親子関係を定義
\renewcommand{\thenombresub}{\arabic{nombresub}} %任意
\setcounter{nombresub}{0} %任意
\newcommand{\prob}[1][]{\refstepcounter{nombre}#1[問題 \thenombre]}
\newcommand{\probsub}[1][]{\refstepcounter{nombresub}#1(\thenombresub)}


%1-1みたいなカウンタ(todaiとtodaia)
\newcounter{todai}
\setcounter{todai}{0}
\newcounter{todaisub}[todai] 
\setcounter{todaisub}{0} 
\newcommand{\todai}[1][]{\refstepcounter{todai}#1 \thetodai-\thetodaisub}
\newcommand{\todaib}[1][]{\refstepcounter{todai}#1\refstepcounter{todaisub}#1 {\bf [問題 \thetodai.\thetodaisub]}}
\newcommand{\todaia}[1][]{\refstepcounter{todaisub}#1 {\bf [問題 \thetodai.\thetodaisub]}}


     \begin{oframed}
     $k>0$とする.$xy$平面上の二曲線$y=k(x-x^3)$,$x=k(y-y^3)$が第一象限に$\a\not=\b$
     なる交点$(\a,\b)$をもつような$k$の範囲を求めよ.
     \end{oframed}

\setlength{\columnseprule}{0.4pt}
\begin{multicols}{2}
{\bf[解]} 
     \begin{subnumcases}
     {\exists\a\exists\b}
     \b=k(\a-\a^3) \label{1}\\
     \a=k(\b-\b^3) \label{2}\\
     \a\not=\b \ \ \ \ 0<\a,\b \label{3}
     \end{subnumcases}
なる条件を調べればよい.
     \begin{align*}
     \eqref{1}\land\eqref{2}\Longleftrightarrow\eqref{1}-\eqref{2}\land\eqref{1}+\eqref{2}
     \end{align*}
より同値変形して    
     \begin{align*}
     &\exists\a\exists\b
     \left\{
     \begin{array}{l}
     \a+\b=k\left\{(\a+\b)-(\a^3+\b^3)\right\} \\
     \b-\a=k\left\{(\a-\b)-(\a^3-\b^3)\right\} \\
     \a\not=\b \ \ \ \ 0<\a,\b
     \end{array}
     \right. \\
     \Longleftrightarrow
     &\exists\a\exists\b
     \left\{
     \begin{array}{l}
     1=k\left\{1-(\a^2-\a\b+\b^2)\right\} \\
     -1=k\left\{1-(\a^2+\a\b+\b^2)\right\} \\
     \a\not=\b \ \ \ \ 0<\a,\b
     \end{array}
     \right. \\
     \Longleftrightarrow 
     &\exists\a\exists\b
     \left\{
     \begin{array}{l}
     0=2k\left\{1-(\a^2+\b^2)\right\} \\
     1=k\a\b \\
     \a\not=\b \ \ \ \ 0<\a,\b
     \end{array}
     \right.
     \end{align*}     
途中の変形に$\a+\b\not=0$,$\a-\b\not=0$を用いた. 然るに$k>0$だから
      \begin{subnumcases}
      {\exists\a\exists\b}
      1-(\a^2+\b^2)=0 \label{4}\\
      1=k\a\b  \label{5}\\
      \a\not=\b \ \ \ \ 0<\a,\b \label{6}
      \end{subnumcases}
このような$k_{>0}$の条件を調べる.そこで\1 ($0<\theta<\pi/2 , \theta\not=\pi/4$)
\eqref{4},\eqref{6}から$\a=c,\b=s$と置ける.\eqref{5}に代入して
      \[k=\frac{1}{sc}=\frac{2}{\sin 2\theta}\]
であって,$\theta$の範囲から$0<\sin 2\theta<1$であるから,求める$k$の範囲は
     \[2<k\]
である.$\cdots$(答)
  \\
  \\
{\bf[別解1]}\eqref{4}以下,$\a\b$平面上に図示する方法も考えられる.この時は\eqref{5}と原点の距離
の二乗$L$が
     \begin{align*}
     L&=\a^2+\b^2 \\ 
     &=  \a^2+\left(\frac{1}{k\a}\right)^2 \\
     &\ge 2\sqrt{\frac{1}{k^2}} \tag{$\a,k>0$からAM-GM} \\
     &=\frac{2}{k}
     \end{align*}
で与えられること,及び答号成立が$\a=\b$であって,グラフが連続であることから,
     \begin{align*}
     \frac{2}{k}<1\Longleftrightarrow 2<k
     \end{align*}
となる.$\cdots$(答)
 \\
 \\
 {\bf[別解2]}\eqref{4}以下,$\a$,$\b$を解とする$2$次方程式を考えてもよい.$a=\a+\b$,$b=\a\b$
 とおけば
       \begin{subnumcases}
      {\exists a\exists b}
      1=a^2-2b \label{7}\\
      1=kb  \label{8}\\
      \a\not=\b \ \ \ \ 0<\a,\b \label{9}
      \end{subnumcases}
 を考えればよい.考える方程式は$x^2-ax+b=0$であって,\eqref{9}からこれが正の異$2$実解を持てばよいので,判別式$D$として
      \begin{align*}
      &D>0 &a,b>0 \\
      \Longleftrightarrow
      &a^2-4b>0 & a,b>0
      \end{align*}
これに\eqref{7},\eqref{8}を代入して$a$,$b$を消去する.$k>0$から$a,b>0$は自動的に満たされ,
     \begin{align*}
     \left(1+\frac{2}{k}\right)-\frac{4}{k}>0
     \Longleftrightarrow
     2<k
     \end{align*} 
となる.$\cdots$(答)    
\newpage
\end{multicols}
\end{document}