\documentclass[a4j]{jarticle}
\usepackage{amsmath}
\usepackage{ascmac}
\usepackage{amssymb}
\usepackage{enumerate}
\usepackage{multicol}
\usepackage{framed}

\title{}
\begin{document}
%分数関係


\def\tfrac#1#2{{\textstyle\frac{#1}{#2}}} %数式中で文中表示の分数を使う時


%Σ関係

\def\dsum#1#2{{\displaystyle\sum_{#1}^{#2}}} %文中で数式表示のΣを使う時


%ベクトル関係


\def\vector#1{\overrightarrow{#1}}  %ベクトルを表現したいとき(aベクトルを表現するときは\ver
\def\norm#1{|\overrightarrow{#1}|} %ベクトルの絶対値
\def\vtwo#1#2{ \left(%
      \begin{array}{c}%
      #1 \\%
      #2 \\%
      \end{array}%
      \right) }                        %2次元ベクトル成分表示
      
      \def\vthree#1#2#3{ \left(
      \begin{array}{c}
      #1 \\
      #2 \\
      #3 \\
      \end{array}
      \right) }                        %3次元ベクトル成分表示



%数列関係


\def\an#1{\verb|{|$#1$\verb|}|}


%極限関係

\def\limit#1#2{\stackrel{#1 \to #2}{\longrightarrow}}   %等式変形からの極限
\def\dlim#1#2{{\displaystyle \lim_{#1\to#2}}} %文中で数式表示の極限を使う



%積分関係

\def\dint#1#2{{\displaystyle \int_{#1}^{#2}}} %文中で数式表示の積分を使う時

\def\ne{\nearrow}
\def\se{\searrow}
\def\nw{\nwarrow}
\def\ne{\nearrow}


%便利なやつ

\def\case#1#2{%
 \[\left\{%
 \begin{array}{l}%
 #1 \\%
 #2%
 \end{array}%
 \right.\] }                           %場合分け
 
\def\1{$\cos\theta=c$,$\sin\theta=s$とおく.}  %cs表示を与える前書きシータ
\def\2{$\cos t=c$,$\sin t=s$とおく.}     %cs表示を与える前書きt
\def\3{$\cos x=c$,$\sin x=s$とおく.}                %cs表示を与える前書きx

\def\fig#1#2#3 {%
\begin{wrapfigure}[#1]{r}{#2 zw}%
\vspace*{-1zh}%
\input{#3}%
\end{wrapfigure} }           %絵の挿入


\def\a{\alpha}   %ギリシャ文字
\def\b{\beta}
\def\g{\gamma}

%問題番号のためのマクロ

\newcounter{nombre} %必須
\renewcommand{\thenombre}{\arabic{nombre}} %任意
\setcounter{nombre}{2} %任意
\newcounter{nombresub}[nombre] %親子関係を定義
\renewcommand{\thenombresub}{\arabic{nombresub}} %任意
\setcounter{nombresub}{0} %任意
\newcommand{\prob}[1][]{\refstepcounter{nombre}#1[問題 \thenombre]}
\newcommand{\probsub}[1][]{\refstepcounter{nombresub}#1(\thenombresub)}


%1-1みたいなカウンタ(todaiとtodaia)
\newcounter{todai}
\setcounter{todai}{0}
\newcounter{todaisub}[todai] 
\setcounter{todaisub}{0} 
\newcommand{\todai}[1][]{\refstepcounter{todai}#1 \thetodai-\thetodaisub}
\newcommand{\todaib}[1][]{\refstepcounter{todai}#1\refstepcounter{todaisub}#1 {\bf [問題 \thetodai.\thetodaisub]}}
\newcommand{\todaia}[1][]{\refstepcounter{todaisub}#1 {\bf [問題 \thetodai.\thetodaisub]}}


\begin{oframed}
点$O$で$60^\circ$の角をなす半直線$OX$,$OY$と$\angle XOY$の$2$等分線$Z$があり,
$OX$,$OY$上に$O$から$1\,\mathrm{cm}$の距離にそれぞれ点$A$,$B$がある.いま動点
$P$,$Q$,$R$がそれぞれ$A$$O$$B$から同時に出発して半直線$OX$,$OZ$,$OY$
上をそれぞれ毎秒$1\,\mathrm{cm}$,$\sqrt{3}\,\mathrm{cm}$,$2\,\mathrm{cm}$の速さで$O$から遠ざかる.
     \begin{enumerate}[(i)]
     \item $3$点$P$,$Q$,$R$が一直線上に来るまでの時間 \\
     および
     \item $\triangle PQR$の面積が$\triangle AOB$の面積に等しくなるまでの時間
     \end{enumerate}
を求めよ.     
\end{oframed}

\setlength{\columnseprule}{0.4pt}
\begin{multicols}{2}

{\bf[解]} まず,$O$を原点とし,半直線$OX$が$x$軸正方向になるように$xy$座標を定める.このとき$OY$が第$1$象限にあるようにする.さらに動点が時刻$0$に各点を出発したとする.
すると時刻$t$での動点の位置は以下のように表せる.
     \begin{align*}
     &P(t+1,0) \\
     &Q(\dfrac{3}{2}t,\dfrac{\sqrt{3}}{2}t) \\
     & R(t+\dfrac{1}{2},\sqrt{3}t+\dfrac{\sqrt{3}}{2})
     \end{align*}
これを用いて問に答える.
     \begin{enumerate}[(1)]
     \item $\vector{PQ}\parallel\vector{PR}$となればよい.故に$k\in\mathbb{R}$として
          \begin{align*}
          \vector{PQ}=&k\vector{PR} \\
          \vtwo{\dfrac{1}{2}t-1}{\frac{\sqrt{3}}{2}t}=&k\vtwo{-\frac{1}{2}}{\sqrt{3}t+\frac{\sqrt{3}}{2}}
          \end{align*}
     このような$k$の存在条件を求めればよい.ゆえに$k$を消去して
          \begin{align*}
          \frac{\sqrt{3}}{2}t=&(2-t)(\sqrt{3}t+\frac{\sqrt{3}}{2}) \\
          t=&\frac{1+\sqrt{5}}{2}
          \end{align*}
     よってもとめる値は$t=\dfrac{1+\sqrt{5}}{2}\cdots$(答) である.

     \item $\triangle AOB$の面積は$\dfrac{\sqrt3}{4}\,\mathrm{cm^2}$であるから,時刻$t$での
     $\triangle PQR$の面積がこれに等しい時,
          \begin{align*}
          \frac{\sqrt3}{4}=&\frac{1}{2}\left|\left(\frac{1}{2}t-1\right)\sqrt{3}\left(t+\frac{1}{2}\right)%
          +\frac{1}{2}\frac{\sqrt{3}}{2}t\right| \\
          = &\frac{\sqrt3}{4}\left|t^2-t-1\right| 
          \end{align*}
          これを解いて
          \begin{align*}
          &\left|t^2-t-1\right|=1 \\
          &\therefore  t=1,2 \cdots\text{(答)}
          \end{align*}
          となる.
     \end{enumerate}
     
\newpage
\end{multicols}
\end{document}