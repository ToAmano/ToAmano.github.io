\documentclass[a4j]{jarticle}
\usepackage{amsmath}
\usepackage{ascmac}
\usepackage{amssymb}
\usepackage{enumerate}
\usepackage{multicol}
\usepackage{framed}
\usepackage{fancyhdr}
\usepackage{latexsym}
\usepackage{indent}
\usepackage{cases}
\usepackage[dvips]{graphicx}
\usepackage{color}
\allowdisplaybreaks
\pagestyle{fancy}
\lhead{}
\chead{}
\rhead{東京大学前期$1961$年$5$番}
\begin{document}
%分数関係


\def\tfrac#1#2{{\textstyle\frac{#1}{#2}}} %数式中で文中表示の分数を使う時


%Σ関係

\def\dsum#1#2{{\displaystyle\sum_{#1}^{#2}}} %文中で数式表示のΣを使う時


%ベクトル関係


\def\vector#1{\overrightarrow{#1}}  %ベクトルを表現したいとき(aベクトルを表現するときは\ver
\def\norm#1{|\overrightarrow{#1}|} %ベクトルの絶対値
\def\vtwo#1#2{ \left(%
      \begin{array}{c}%
      #1 \\%
      #2 \\%
      \end{array}%
      \right) }                        %2次元ベクトル成分表示
      
      \def\vthree#1#2#3{ \left(
      \begin{array}{c}
      #1 \\
      #2 \\
      #3 \\
      \end{array}
      \right) }                        %3次元ベクトル成分表示



%数列関係


\def\an#1{\verb|{|$#1$\verb|}|}


%極限関係

\def\limit#1#2{\stackrel{#1 \to #2}{\longrightarrow}}   %等式変形からの極限
\def\dlim#1#2{{\displaystyle \lim_{#1\to#2}}} %文中で数式表示の極限を使う



%積分関係

\def\dint#1#2{{\displaystyle \int_{#1}^{#2}}} %文中で数式表示の積分を使う時

\def\ne{\nearrow}
\def\se{\searrow}
\def\nw{\nwarrow}
\def\ne{\nearrow}


%便利なやつ

\def\case#1#2{%
 \[\left\{%
 \begin{array}{l}%
 #1 \\%
 #2%
 \end{array}%
 \right.\] }                           %場合分け
 
\def\1{$\cos\theta=c$,$\sin\theta=s$とおく.}  %cs表示を与える前書きシータ
\def\2{$\cos t=c$,$\sin t=s$とおく.}     %cs表示を与える前書きt
\def\3{$\cos x=c$,$\sin x=s$とおく.}                %cs表示を与える前書きx

\def\fig#1#2#3 {%
\begin{wrapfigure}[#1]{r}{#2 zw}%
\vspace*{-1zh}%
\input{#3}%
\end{wrapfigure} }           %絵の挿入


\def\a{\alpha}   %ギリシャ文字
\def\b{\beta}
\def\g{\gamma}

%問題番号のためのマクロ

\newcounter{nombre} %必須
\renewcommand{\thenombre}{\arabic{nombre}} %任意
\setcounter{nombre}{2} %任意
\newcounter{nombresub}[nombre] %親子関係を定義
\renewcommand{\thenombresub}{\arabic{nombresub}} %任意
\setcounter{nombresub}{0} %任意
\newcommand{\prob}[1][]{\refstepcounter{nombre}#1[問題 \thenombre]}
\newcommand{\probsub}[1][]{\refstepcounter{nombresub}#1(\thenombresub)}


%1-1みたいなカウンタ(todaiとtodaia)
\newcounter{todai}
\setcounter{todai}{0}
\newcounter{todaisub}[todai] 
\setcounter{todaisub}{0} 
\newcommand{\todai}[1][]{\refstepcounter{todai}#1 \thetodai-\thetodaisub}
\newcommand{\todaib}[1][]{\refstepcounter{todai}#1\refstepcounter{todaisub}#1 {\bf [問題 \thetodai.\thetodaisub]}}
\newcommand{\todaia}[1][]{\refstepcounter{todaisub}#1 {\bf [問題 \thetodai.\thetodaisub]}}


     \begin{oframed}
     $t$がすべての実数の範囲を動く時$x=t^2+1$,$y=t^2+t-2$を座標とする点$(x,y)$は一つの曲線を描く.この曲線と$x$軸とによって
     囲まれる部分の面積を求めよ.
     \end{oframed}

\setlength{\columnseprule}{0.4pt}
\begin{multicols}{2}
{\bf[解]} $x'=2t$,$y'=2t+1$であるから,下表を得る.
     \begin{align*}
          \begin{array}{|c|c|c|c|c|c|} \hline
          t      &    &-1/2          &   & 0     &  \\ \hline
          x'     & -  & -              &- &0       &+ \\ \hline
          y'     & -  & 0             &+&+       & + \\ \hline
          (x,y)&\se&(5/4,-9/4) &\se&(1,-2)&\ne \\ \hline      
          \end{array}
     \end{align*}
また,極限値は,$t\to\pm\infty$の時,$x,y\to\infty$である.ゆえにグラフの概形は下図.
     \begin{center}
     \scalebox{1}{%WinTpicVersion4.32a
{\unitlength 0.1in%
\begin{picture}(26.2000,18.0000)(3.8000,-26.0000)%
% STR 2 0 3 0 Black White  
% 4 590 1197 590 1210 4 400 0 0
% O
\put(5.9000,-12.1000){\makebox(0,0)[rt]{O}}%
% STR 2 0 3 0 Black White  
% 4 560 787 560 800 4 400 0 0
% $y$
\put(5.6000,-8.0000){\makebox(0,0)[rt]{$y$}}%
% STR 2 0 3 0 Black White  
% 4 3000 1227 3000 1240 4 400 0 0
% $x$
\put(30.0000,-12.4000){\makebox(0,0)[rt]{$x$}}%
% VECTOR 2 0 3 0 Black White  
% 2 600 2600 600 800
% 
\special{pn 8}%
\special{pa 600 2600}%
\special{pa 600 800}%
\special{fp}%
\special{sh 1}%
\special{pa 600 800}%
\special{pa 580 867}%
\special{pa 600 853}%
\special{pa 620 867}%
\special{pa 600 800}%
\special{fp}%
% VECTOR 2 0 3 0 Black White  
% 2 400 1200 3000 1200
% 
\special{pn 8}%
\special{pa 400 1200}%
\special{pa 3000 1200}%
\special{fp}%
\special{sh 1}%
\special{pa 3000 1200}%
\special{pa 2933 1180}%
\special{pa 2947 1200}%
\special{pa 2933 1220}%
\special{pa 3000 1200}%
\special{fp}%
% FUNC 2 0 3 0 Black White  
% 10 400 800 3000 2600 600 1200 1000 1200 600 800 400 800 3000 2600 50 2 0 2 1 0
% t^2+1///t^2+t-2///-10///10
\special{pn 8}%
\special{pa 3000 895}%
\special{pa 2929 949}%
\special{pa 2859 1003}%
\special{pa 2791 1055}%
\special{pa 2724 1106}%
\special{pa 2658 1156}%
\special{pa 2594 1205}%
\special{pa 2530 1252}%
\special{pa 2468 1298}%
\special{pa 2408 1343}%
\special{pa 2348 1386}%
\special{pa 2290 1428}%
\special{pa 2233 1469}%
\special{pa 2178 1509}%
\special{pa 2124 1547}%
\special{pa 2071 1584}%
\special{pa 2019 1620}%
\special{pa 1968 1654}%
\special{pa 1919 1687}%
\special{pa 1871 1719}%
\special{pa 1825 1750}%
\special{pa 1780 1779}%
\special{pa 1735 1807}%
\special{pa 1693 1834}%
\special{pa 1651 1859}%
\special{pa 1611 1883}%
\special{pa 1572 1906}%
\special{pa 1535 1928}%
\special{pa 1498 1948}%
\special{pa 1463 1967}%
\special{pa 1429 1985}%
\special{pa 1397 2002}%
\special{pa 1366 2017}%
\special{pa 1336 2031}%
\special{pa 1307 2043}%
\special{pa 1280 2055}%
\special{pa 1253 2065}%
\special{pa 1229 2074}%
\special{pa 1205 2081}%
\special{pa 1183 2088}%
\special{pa 1162 2093}%
\special{pa 1142 2096}%
\special{pa 1124 2099}%
\special{pa 1107 2100}%
\special{pa 1091 2100}%
\special{pa 1076 2098}%
\special{pa 1063 2096}%
\special{pa 1051 2092}%
\special{pa 1040 2086}%
\special{pa 1030 2080}%
\special{pa 1022 2072}%
\special{pa 1015 2063}%
\special{pa 1010 2053}%
\special{pa 1005 2041}%
\special{pa 1002 2028}%
\special{pa 1001 2014}%
\special{pa 1000 1998}%
\special{pa 1001 1982}%
\special{pa 1003 1964}%
\special{pa 1006 1944}%
\special{pa 1011 1924}%
\special{pa 1017 1902}%
\special{pa 1024 1879}%
\special{pa 1032 1854}%
\special{pa 1042 1828}%
\special{pa 1053 1801}%
\special{pa 1065 1773}%
\special{pa 1079 1744}%
\special{pa 1094 1713}%
\special{pa 1127 1647}%
\special{pa 1146 1612}%
\special{pa 1166 1577}%
\special{pa 1187 1539}%
\special{pa 1210 1501}%
\special{pa 1233 1461}%
\special{pa 1259 1420}%
\special{pa 1285 1377}%
\special{pa 1313 1334}%
\special{pa 1342 1289}%
\special{pa 1372 1243}%
\special{pa 1403 1195}%
\special{pa 1436 1146}%
\special{pa 1470 1096}%
\special{pa 1505 1045}%
\special{pa 1542 992}%
\special{pa 1580 939}%
\special{pa 1619 883}%
\special{pa 1659 827}%
\special{pa 1662 824}%
\special{pa 1662 823}%
\special{pa 1664 821}%
\special{pa 1664 820}%
\special{pa 1665 820}%
\special{pa 1666 817}%
\special{pa 1667 817}%
\special{pa 1667 816}%
\special{pa 1669 814}%
\special{pa 1669 813}%
\special{pa 1670 813}%
\special{pa 1671 810}%
\special{pa 1672 810}%
\special{pa 1672 809}%
\special{pa 1673 809}%
\special{pa 1674 806}%
\special{pa 1675 806}%
\special{pa 1675 805}%
\special{pa 1676 805}%
\special{pa 1676 803}%
\special{pa 1677 803}%
\special{pa 1677 802}%
\special{pa 1678 802}%
\special{pa 1678 801}%
\special{fp}%
% LINE 2 2 3 0 Black White  
% 2 1010 1960 1010 1200
% 
\special{pn 8}%
\special{pa 1010 1960}%
\special{pa 1010 1200}%
\special{dt 0.045}%
% STR 2 0 3 0 Black White  
% 4 1000 1100 1000 1200 2 0 1 0
% $1$
\put(10.0000,-12.0000){\makebox(0,0)[lb]{{\colorbox[named]{White}{$1$}}}}%
% STR 2 0 3 0 Black White  
% 4 1400 1100 1400 1200 2 0 1 0
% $2$
\put(14.0000,-12.0000){\makebox(0,0)[lb]{{\colorbox[named]{White}{$2$}}}}%
% STR 2 0 3 0 Black White  
% 4 2600 1100 2600 1200 2 0 1 0
% $5$
\put(26.0000,-12.0000){\makebox(0,0)[lb]{{\colorbox[named]{White}{$5$}}}}%
% STR 2 0 3 0 Black White  
% 4 2600 1100 2600 1200 3 0 1 0
% $t=-2$
\put(26.0000,-12.0000){\makebox(0,0)[rb]{{\colorbox[named]{White}{$t=-2$}}}}%
% STR 2 0 3 0 Black White  
% 4 1400 1100 1400 1200 4 0 1 0
% $t=1$
\put(14.0000,-12.0000){\makebox(0,0)[rt]{{\colorbox[named]{White}{$t=1$}}}}%
% STR 2 0 3 0 Black White  
% 4 1000 1900 1000 2000 2 0 1 0
% $t=0$
\put(10.0000,-20.0000){\makebox(0,0)[lb]{{\colorbox[named]{White}{$t=0$}}}}%
\end{picture}}%
}
     \end{center}

ここでグラフの下側を$y_-$,上側を$y_+$とすると,求める面積$S$は,
     \begin{align*}
     S&=\int_1^2(y_+-y_-)dx-\int_2^5y_-dx \\
     &=\int_0^1y_+\frac{dx}{dt}dt-\int_0^{-1}y_-\frac{dx}{dt}dt-\int_{-1}^{-2}y_-\frac{dx}{dt}dt \\
     &=\int_{-2}^1y\frac{dx}{dt}dt \\
     &=\int_{-2}^1(t^2+t-2)2tdt \\
     &=2\left[\frac{1}{4}t^4+\frac{1}{3}t^3-t^2\right]_{-2}^1=\frac{9}{2}
     \end{align*}
である.$\cdots$(答)
  \\ \\
{\bf[別解]}$xy'-x'y=-t^2+6t+1$より,ガウスグリーンの定理から,
     \begin{align*}
     T=\frac{1}{2}\int_{-2}^1-(-t^2+6t+1)dt=\frac{9}{2}
     \end{align*}     
である.$\cdots$(答)     
\newpage
\end{multicols}
\end{document}