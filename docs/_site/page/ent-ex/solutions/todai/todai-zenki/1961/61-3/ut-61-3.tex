\documentclass[a4j]{jarticle}
\usepackage{amsmath}
\usepackage{ascmac}
\usepackage{amssymb}
\usepackage{enumerate}
\usepackage{multicol}
\usepackage{framed}
\usepackage{fancyhdr}
\usepackage{latexsym}
\usepackage{indent}
\usepackage{cases}
\usepackage[dvips]{graphicx}
\usepackage{color}
\allowdisplaybreaks
\pagestyle{fancy}
\lhead{}
\chead{}
\rhead{東京大学前期$1961$年$3$番}
\begin{document}
%分数関係


\def\tfrac#1#2{{\textstyle\frac{#1}{#2}}} %数式中で文中表示の分数を使う時


%Σ関係

\def\dsum#1#2{{\displaystyle\sum_{#1}^{#2}}} %文中で数式表示のΣを使う時


%ベクトル関係


\def\vector#1{\overrightarrow{#1}}  %ベクトルを表現したいとき(aベクトルを表現するときは\ver
\def\norm#1{|\overrightarrow{#1}|} %ベクトルの絶対値
\def\vtwo#1#2{ \left(%
      \begin{array}{c}%
      #1 \\%
      #2 \\%
      \end{array}%
      \right) }                        %2次元ベクトル成分表示
      
      \def\vthree#1#2#3{ \left(
      \begin{array}{c}
      #1 \\
      #2 \\
      #3 \\
      \end{array}
      \right) }                        %3次元ベクトル成分表示



%数列関係


\def\an#1{\verb|{|$#1$\verb|}|}


%極限関係

\def\limit#1#2{\stackrel{#1 \to #2}{\longrightarrow}}   %等式変形からの極限
\def\dlim#1#2{{\displaystyle \lim_{#1\to#2}}} %文中で数式表示の極限を使う



%積分関係

\def\dint#1#2{{\displaystyle \int_{#1}^{#2}}} %文中で数式表示の積分を使う時

\def\ne{\nearrow}
\def\se{\searrow}
\def\nw{\nwarrow}
\def\ne{\nearrow}


%便利なやつ

\def\case#1#2{%
 \[\left\{%
 \begin{array}{l}%
 #1 \\%
 #2%
 \end{array}%
 \right.\] }                           %場合分け
 
\def\1{$\cos\theta=c$,$\sin\theta=s$とおく.}  %cs表示を与える前書きシータ
\def\2{$\cos t=c$,$\sin t=s$とおく.}     %cs表示を与える前書きt
\def\3{$\cos x=c$,$\sin x=s$とおく.}                %cs表示を与える前書きx

\def\fig#1#2#3 {%
\begin{wrapfigure}[#1]{r}{#2 zw}%
\vspace*{-1zh}%
\input{#3}%
\end{wrapfigure} }           %絵の挿入


\def\a{\alpha}   %ギリシャ文字
\def\b{\beta}
\def\g{\gamma}

%問題番号のためのマクロ

\newcounter{nombre} %必須
\renewcommand{\thenombre}{\arabic{nombre}} %任意
\setcounter{nombre}{2} %任意
\newcounter{nombresub}[nombre] %親子関係を定義
\renewcommand{\thenombresub}{\arabic{nombresub}} %任意
\setcounter{nombresub}{0} %任意
\newcommand{\prob}[1][]{\refstepcounter{nombre}#1[問題 \thenombre]}
\newcommand{\probsub}[1][]{\refstepcounter{nombresub}#1(\thenombresub)}


%1-1みたいなカウンタ(todaiとtodaia)
\newcounter{todai}
\setcounter{todai}{0}
\newcounter{todaisub}[todai] 
\setcounter{todaisub}{0} 
\newcommand{\todai}[1][]{\refstepcounter{todai}#1 \thetodai-\thetodaisub}
\newcommand{\todaib}[1][]{\refstepcounter{todai}#1\refstepcounter{todaisub}#1 {\bf [問題 \thetodai.\thetodaisub]}}
\newcommand{\todaia}[1][]{\refstepcounter{todaisub}#1 {\bf [問題 \thetodai.\thetodaisub]}}


     \begin{oframed}
     あたえられた半径$a$の半球に外接する直円錐をつくり,その全表面積(側面積と底面積の和)をもっとも小さくするには,その高さをな
     にほどにすればよいか.ただし直円錐の底面は半球の底面とおなじ平面上にあるものとする.
     \end{oframed}

\setlength{\columnseprule}{0.4pt}
\begin{multicols}{2}
{\bf[解]} 直円錐の軸を含む平面で切って,下図のように点,図形量を置く.また,\2ただし$0<t<\pi/2$である.
     \begin{center}
     \scalebox{1}{%WinTpicVersion4.32a
{\unitlength 0.1in%
\begin{picture}(20.1500,14.0000)(11.8500,-20.0000)%
% FUNC 2 0 3 0 Black White  
% 10 1200 600 3200 2000 2200 1800 2800 1800 2200 1200 1200 600 3200 2000 50 4 0 2 0 0
% cos(t)///sin(t)///00///pi
\special{pn 8}%
\special{pa 2800 1800}%
\special{pa 2800 1777}%
\special{pa 2799 1774}%
\special{pa 2799 1759}%
\special{pa 2798 1755}%
\special{pa 2798 1747}%
\special{pa 2797 1743}%
\special{pa 2797 1736}%
\special{pa 2796 1732}%
\special{pa 2796 1729}%
\special{pa 2795 1725}%
\special{pa 2795 1721}%
\special{pa 2794 1717}%
\special{pa 2794 1714}%
\special{pa 2793 1710}%
\special{pa 2793 1706}%
\special{pa 2792 1702}%
\special{pa 2791 1699}%
\special{pa 2791 1695}%
\special{pa 2790 1691}%
\special{pa 2789 1688}%
\special{pa 2789 1684}%
\special{pa 2787 1676}%
\special{pa 2786 1673}%
\special{pa 2786 1669}%
\special{pa 2785 1665}%
\special{pa 2784 1662}%
\special{pa 2782 1654}%
\special{pa 2781 1651}%
\special{pa 2779 1643}%
\special{pa 2778 1640}%
\special{pa 2777 1636}%
\special{pa 2776 1633}%
\special{pa 2774 1625}%
\special{pa 2773 1622}%
\special{pa 2772 1618}%
\special{pa 2771 1615}%
\special{pa 2769 1611}%
\special{pa 2768 1607}%
\special{pa 2767 1604}%
\special{pa 2766 1600}%
\special{pa 2765 1597}%
\special{pa 2763 1593}%
\special{pa 2762 1590}%
\special{pa 2761 1586}%
\special{pa 2759 1583}%
\special{pa 2758 1579}%
\special{pa 2756 1576}%
\special{pa 2755 1572}%
\special{pa 2754 1569}%
\special{pa 2752 1565}%
\special{pa 2751 1562}%
\special{pa 2749 1558}%
\special{pa 2748 1555}%
\special{pa 2746 1551}%
\special{pa 2744 1548}%
\special{pa 2743 1544}%
\special{pa 2741 1541}%
\special{pa 2740 1538}%
\special{pa 2738 1534}%
\special{pa 2736 1531}%
\special{pa 2735 1528}%
\special{pa 2733 1524}%
\special{pa 2729 1518}%
\special{pa 2728 1514}%
\special{pa 2724 1508}%
\special{pa 2722 1504}%
\special{pa 2716 1495}%
\special{pa 2714 1491}%
\special{pa 2713 1488}%
\special{pa 2709 1482}%
\special{pa 2707 1478}%
\special{pa 2705 1475}%
\special{pa 2702 1472}%
\special{pa 2692 1457}%
\special{pa 2690 1453}%
\special{pa 2688 1450}%
\special{pa 2685 1447}%
\special{pa 2679 1438}%
\special{pa 2676 1435}%
\special{pa 2672 1429}%
\special{pa 2669 1426}%
\special{pa 2665 1420}%
\special{pa 2662 1418}%
\special{pa 2660 1415}%
\special{pa 2657 1412}%
\special{pa 2653 1406}%
\special{pa 2650 1403}%
\special{pa 2648 1400}%
\special{pa 2645 1398}%
\special{pa 2642 1395}%
\special{pa 2640 1392}%
\special{pa 2632 1384}%
\special{pa 2630 1381}%
\special{pa 2627 1378}%
\special{pa 2624 1376}%
\special{pa 2622 1373}%
\special{pa 2619 1370}%
\special{pa 2616 1368}%
\special{pa 2605 1357}%
\special{pa 2602 1355}%
\special{pa 2600 1352}%
\special{pa 2597 1350}%
\special{pa 2594 1347}%
\special{pa 2588 1343}%
\special{pa 2585 1340}%
\special{pa 2582 1338}%
\special{pa 2580 1335}%
\special{pa 2574 1331}%
\special{pa 2571 1328}%
\special{pa 2565 1324}%
\special{pa 2562 1321}%
\special{pa 2553 1315}%
\special{pa 2550 1312}%
\special{pa 2547 1310}%
\special{pa 2543 1308}%
\special{pa 2531 1300}%
\special{pa 2528 1297}%
\special{pa 2525 1295}%
\special{pa 2521 1293}%
\special{pa 2509 1285}%
\special{pa 2505 1284}%
\special{pa 2496 1278}%
\special{pa 2492 1276}%
\special{pa 2486 1272}%
\special{pa 2482 1271}%
\special{pa 2476 1267}%
\special{pa 2472 1265}%
\special{pa 2469 1264}%
\special{pa 2466 1262}%
\special{pa 2462 1260}%
\special{pa 2459 1259}%
\special{pa 2455 1257}%
\special{pa 2452 1255}%
\special{pa 2449 1254}%
\special{pa 2445 1252}%
\special{pa 2442 1251}%
\special{pa 2438 1249}%
\special{pa 2435 1248}%
\special{pa 2431 1246}%
\special{pa 2428 1245}%
\special{pa 2424 1244}%
\special{pa 2421 1242}%
\special{pa 2417 1241}%
\special{pa 2414 1239}%
\special{pa 2410 1238}%
\special{pa 2407 1237}%
\special{pa 2403 1235}%
\special{pa 2400 1234}%
\special{pa 2396 1233}%
\special{pa 2393 1232}%
\special{pa 2389 1231}%
\special{pa 2385 1229}%
\special{pa 2382 1228}%
\special{pa 2378 1227}%
\special{pa 2375 1226}%
\special{pa 2367 1224}%
\special{pa 2364 1223}%
\special{pa 2356 1221}%
\special{pa 2353 1220}%
\special{pa 2349 1219}%
\special{pa 2346 1218}%
\special{pa 2338 1216}%
\special{pa 2335 1215}%
\special{pa 2331 1214}%
\special{pa 2327 1214}%
\special{pa 2323 1213}%
\special{pa 2320 1212}%
\special{pa 2316 1211}%
\special{pa 2312 1211}%
\special{pa 2309 1210}%
\special{pa 2305 1209}%
\special{pa 2301 1209}%
\special{pa 2298 1208}%
\special{pa 2294 1207}%
\special{pa 2290 1207}%
\special{pa 2286 1206}%
\special{pa 2283 1206}%
\special{pa 2279 1205}%
\special{pa 2275 1205}%
\special{pa 2271 1204}%
\special{pa 2268 1204}%
\special{pa 2264 1203}%
\special{pa 2256 1203}%
\special{pa 2253 1202}%
\special{pa 2245 1202}%
\special{pa 2241 1201}%
\special{pa 2226 1201}%
\special{pa 2223 1200}%
\special{pa 2177 1200}%
\special{pa 2174 1201}%
\special{pa 2159 1201}%
\special{pa 2155 1202}%
\special{pa 2147 1202}%
\special{pa 2143 1203}%
\special{pa 2136 1203}%
\special{pa 2132 1204}%
\special{pa 2129 1204}%
\special{pa 2125 1205}%
\special{pa 2121 1205}%
\special{pa 2117 1206}%
\special{pa 2114 1206}%
\special{pa 2110 1207}%
\special{pa 2106 1207}%
\special{pa 2102 1208}%
\special{pa 2099 1209}%
\special{pa 2095 1209}%
\special{pa 2091 1210}%
\special{pa 2088 1211}%
\special{pa 2084 1211}%
\special{pa 2076 1213}%
\special{pa 2073 1214}%
\special{pa 2069 1214}%
\special{pa 2065 1215}%
\special{pa 2062 1216}%
\special{pa 2054 1218}%
\special{pa 2051 1219}%
\special{pa 2043 1221}%
\special{pa 2040 1222}%
\special{pa 2036 1223}%
\special{pa 2033 1224}%
\special{pa 2025 1226}%
\special{pa 2022 1227}%
\special{pa 2018 1228}%
\special{pa 2015 1229}%
\special{pa 2011 1231}%
\special{pa 2007 1232}%
\special{pa 2004 1233}%
\special{pa 2000 1234}%
\special{pa 1997 1235}%
\special{pa 1993 1237}%
\special{pa 1990 1238}%
\special{pa 1986 1239}%
\special{pa 1983 1241}%
\special{pa 1979 1242}%
\special{pa 1976 1244}%
\special{pa 1972 1245}%
\special{pa 1969 1246}%
\special{pa 1965 1248}%
\special{pa 1962 1249}%
\special{pa 1958 1251}%
\special{pa 1955 1252}%
\special{pa 1951 1254}%
\special{pa 1948 1256}%
\special{pa 1944 1257}%
\special{pa 1941 1259}%
\special{pa 1938 1260}%
\special{pa 1934 1262}%
\special{pa 1931 1264}%
\special{pa 1928 1265}%
\special{pa 1924 1267}%
\special{pa 1918 1271}%
\special{pa 1914 1272}%
\special{pa 1908 1276}%
\special{pa 1904 1278}%
\special{pa 1895 1284}%
\special{pa 1891 1286}%
\special{pa 1888 1287}%
\special{pa 1882 1291}%
\special{pa 1878 1293}%
\special{pa 1875 1295}%
\special{pa 1872 1298}%
\special{pa 1857 1308}%
\special{pa 1853 1310}%
\special{pa 1850 1312}%
\special{pa 1847 1315}%
\special{pa 1838 1321}%
\special{pa 1835 1324}%
\special{pa 1829 1328}%
\special{pa 1826 1331}%
\special{pa 1820 1335}%
\special{pa 1818 1338}%
\special{pa 1815 1340}%
\special{pa 1812 1343}%
\special{pa 1806 1347}%
\special{pa 1803 1350}%
\special{pa 1800 1352}%
\special{pa 1798 1355}%
\special{pa 1795 1358}%
\special{pa 1792 1360}%
\special{pa 1784 1368}%
\special{pa 1781 1370}%
\special{pa 1778 1373}%
\special{pa 1776 1376}%
\special{pa 1773 1378}%
\special{pa 1770 1381}%
\special{pa 1768 1384}%
\special{pa 1757 1395}%
\special{pa 1755 1398}%
\special{pa 1752 1400}%
\special{pa 1750 1403}%
\special{pa 1747 1406}%
\special{pa 1743 1412}%
\special{pa 1740 1415}%
\special{pa 1738 1418}%
\special{pa 1735 1420}%
\special{pa 1731 1426}%
\special{pa 1728 1429}%
\special{pa 1724 1435}%
\special{pa 1721 1438}%
\special{pa 1715 1447}%
\special{pa 1712 1450}%
\special{pa 1710 1453}%
\special{pa 1708 1457}%
\special{pa 1700 1469}%
\special{pa 1697 1472}%
\special{pa 1695 1475}%
\special{pa 1693 1479}%
\special{pa 1685 1491}%
\special{pa 1684 1495}%
\special{pa 1678 1504}%
\special{pa 1676 1508}%
\special{pa 1672 1514}%
\special{pa 1671 1518}%
\special{pa 1667 1524}%
\special{pa 1665 1528}%
\special{pa 1664 1531}%
\special{pa 1662 1534}%
\special{pa 1660 1538}%
\special{pa 1659 1541}%
\special{pa 1657 1545}%
\special{pa 1655 1548}%
\special{pa 1654 1551}%
\special{pa 1652 1555}%
\special{pa 1651 1558}%
\special{pa 1649 1562}%
\special{pa 1648 1565}%
\special{pa 1646 1569}%
\special{pa 1645 1572}%
\special{pa 1644 1576}%
\special{pa 1642 1579}%
\special{pa 1641 1583}%
\special{pa 1639 1586}%
\special{pa 1638 1590}%
\special{pa 1637 1593}%
\special{pa 1635 1597}%
\special{pa 1634 1600}%
\special{pa 1633 1604}%
\special{pa 1632 1607}%
\special{pa 1631 1611}%
\special{pa 1629 1615}%
\special{pa 1628 1618}%
\special{pa 1627 1622}%
\special{pa 1626 1625}%
\special{pa 1624 1633}%
\special{pa 1623 1636}%
\special{pa 1621 1644}%
\special{pa 1620 1647}%
\special{pa 1619 1651}%
\special{pa 1618 1654}%
\special{pa 1616 1662}%
\special{pa 1615 1665}%
\special{pa 1614 1669}%
\special{pa 1614 1673}%
\special{pa 1613 1677}%
\special{pa 1612 1680}%
\special{pa 1611 1684}%
\special{pa 1611 1688}%
\special{pa 1610 1691}%
\special{pa 1609 1695}%
\special{pa 1609 1699}%
\special{pa 1608 1702}%
\special{pa 1607 1706}%
\special{pa 1607 1710}%
\special{pa 1606 1714}%
\special{pa 1606 1717}%
\special{pa 1605 1721}%
\special{pa 1605 1725}%
\special{pa 1604 1729}%
\special{pa 1604 1732}%
\special{pa 1603 1736}%
\special{pa 1603 1744}%
\special{pa 1602 1747}%
\special{pa 1602 1755}%
\special{pa 1601 1759}%
\special{pa 1601 1774}%
\special{pa 1600 1777}%
\special{pa 1600 1800}%
\special{fp}%
% FUNC 2 0 3 0 Black White  
% 10 1200 600 3200 2000 2200 1800 2800 1800 2200 1200 2200 600 3049 2000 0 4 0 0 0 0
% -x+sqrt(2)
\special{pn 8}%
\special{pn 8}%
\special{pa 1849 600}%
\special{pa 1855 606}%
\special{ip}%
\special{pa 1881 632}%
\special{pa 1887 638}%
\special{ip}%
\special{pa 1913 664}%
\special{pa 1918 669}%
\special{ip}%
\special{pa 1945 696}%
\special{pa 1950 701}%
\special{ip}%
\special{pa 1977 728}%
\special{pa 1982 733}%
\special{ip}%
\special{pa 2009 760}%
\special{pa 2014 765}%
\special{ip}%
\special{pa 2040 791}%
\special{pa 2046 797}%
\special{ip}%
\special{pa 2072 823}%
\special{pa 2078 829}%
\special{ip}%
\special{pa 2104 855}%
\special{pa 2110 861}%
\special{ip}%
\special{pa 2136 887}%
\special{pa 2142 893}%
\special{ip}%
\special{pa 2168 919}%
\special{pa 2174 925}%
\special{ip}%
\special{pa 2200 951}%
\special{pa 3045 1796}%
\special{fp}%
\special{pn 8}%
\special{pa 3052 1803}%
\special{pa 3084 1835}%
\special{ip}%
\special{pa 3091 1842}%
\special{pa 3122 1874}%
\special{ip}%
\special{pa 3129 1880}%
\special{pa 3161 1912}%
\special{ip}%
\special{pa 3168 1919}%
\special{pa 3200 1951}%
\special{ip}%
% FUNC 2 0 3 0 Black White  
% 10 1200 600 3200 2000 2200 1800 2800 1800 2200 1200 1351 600 2200 2000 0 4 0 0 0 0
% x+sqrt(2)
\special{pn 8}%
\special{pn 8}%
\special{pa 1200 1951}%
\special{pa 1207 1944}%
\special{ip}%
\special{pa 1238 1914}%
\special{pa 1244 1907}%
\special{ip}%
\special{pa 1275 1876}%
\special{pa 1282 1869}%
\special{ip}%
\special{pa 1313 1838}%
\special{pa 1319 1832}%
\special{ip}%
\special{pa 1350 1801}%
\special{pa 2200 951}%
\special{fp}%
\special{pn 8}%
\special{pa 2206 945}%
\special{pa 2232 919}%
\special{ip}%
\special{pa 2238 913}%
\special{pa 2264 887}%
\special{ip}%
\special{pa 2269 882}%
\special{pa 2296 855}%
\special{ip}%
\special{pa 2301 850}%
\special{pa 2328 823}%
\special{ip}%
\special{pa 2333 818}%
\special{pa 2360 791}%
\special{ip}%
\special{pa 2365 786}%
\special{pa 2391 760}%
\special{ip}%
\special{pa 2397 754}%
\special{pa 2423 728}%
\special{ip}%
\special{pa 2429 722}%
\special{pa 2455 696}%
\special{ip}%
\special{pa 2461 690}%
\special{pa 2487 664}%
\special{ip}%
\special{pa 2493 658}%
\special{pa 2519 632}%
\special{ip}%
\special{pa 2525 626}%
\special{pa 2551 600}%
\special{ip}%
% LINE 2 0 3 0 Black White  
% 2 1350 1800 3050 1800
% 
\special{pn 8}%
\special{pa 1350 1800}%
\special{pa 3050 1800}%
\special{fp}%
% LINE 2 2 3 0 Black White  
% 4 2200 970 2200 1800 2200 1800 1790 1390
% 
\special{pn 8}%
\special{pa 2200 970}%
\special{pa 2200 1800}%
\special{dt 0.045}%
\special{pa 2200 1800}%
\special{pa 1790 1390}%
\special{dt 0.045}%
% STR 2 0 3 0 Black White  
% 4 2190 1800 2190 1900 5 0 1 0
% $O$
\put(21.9000,-19.0000){\makebox(0,0){{\colorbox[named]{White}{$O$}}}}%
% STR 2 0 3 0 Black White  
% 4 2190 790 2190 890 5 0 0 0
% $H$
\put(21.9000,-8.9000){\makebox(0,0){$H$}}%
% STR 2 0 3 0 Black White  
% 4 1280 1790 1280 1890 5 0 1 0
% $A$
\put(12.8000,-18.9000){\makebox(0,0){{\colorbox[named]{White}{$A$}}}}%
% STR 2 0 3 0 Black White  
% 4 1560 1610 1560 1710 5 0 1 0
% $t$
\put(15.6000,-17.1000){\makebox(0,0){{\colorbox[named]{White}{$t$}}}}%
% STR 2 0 3 0 Black White  
% 4 2020 1390 2020 1490 5 0 1 0
% $a$
\put(20.2000,-14.9000){\makebox(0,0){{\colorbox[named]{White}{$a$}}}}%
\end{picture}}%
}
     \end{center}
上図から,
     \begin{align*}
     &|AO|=\frac{a}{c}&|AH|=\frac{a}{cs} \\
     &|OH|=\frac{a}{c}
     \end{align*}
であるから,直円錐の表面積$T$は,
     \begin{align*}
     T&=\text{底面積}+\text{側面積} \\
     &=\pi|AO|^2+\pi|AO||AH| \\
     &=a^2\pi\left(\frac{1}{s^2}+\frac{1}{cs^2}\right) \\     
     &=a^2\pi\frac{1}{c(1-c)}\ge 4a^2\pi &\\
     \end{align*}
等号成立は$c=1/2$の時,つまり$t=\pi/3$の時である.この時,
     \[|OH|=2a\]
となる.$\cdots$(答)
\newpage
\end{multicols}
\end{document}