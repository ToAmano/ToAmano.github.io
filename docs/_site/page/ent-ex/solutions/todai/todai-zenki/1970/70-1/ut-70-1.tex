\documentclass[a4j]{jarticle}
\usepackage{amsmath}
\usepackage{ascmac}
\usepackage{amssymb}
\usepackage{enumerate}
\usepackage{multicol}
\usepackage{framed}
\usepackage{fancyhdr}
\usepackage{latexsym}
\usepackage{indent}
\usepackage{cases}
\allowdisplaybreaks
\pagestyle{fancy}
\lhead{}
\chead{}
\rhead{東京大学前期$$年$$番}
\begin{document}
%分数関係


\def\tfrac#1#2{{\textstyle\frac{#1}{#2}}} %数式中で文中表示の分数を使う時


%Σ関係

\def\dsum#1#2{{\displaystyle\sum_{#1}^{#2}}} %文中で数式表示のΣを使う時


%ベクトル関係


\def\vector#1{\overrightarrow{#1}}  %ベクトルを表現したいとき(aベクトルを表現するときは\ver
\def\norm#1{|\overrightarrow{#1}|} %ベクトルの絶対値
\def\vtwo#1#2{ \left(%
      \begin{array}{c}%
      #1 \\%
      #2 \\%
      \end{array}%
      \right) }                        %2次元ベクトル成分表示
      
      \def\vthree#1#2#3{ \left(
      \begin{array}{c}
      #1 \\
      #2 \\
      #3 \\
      \end{array}
      \right) }                        %3次元ベクトル成分表示



%数列関係


\def\an#1{\verb|{|$#1$\verb|}|}


%極限関係

\def\limit#1#2{\stackrel{#1 \to #2}{\longrightarrow}}   %等式変形からの極限
\def\dlim#1#2{{\displaystyle \lim_{#1\to#2}}} %文中で数式表示の極限を使う



%積分関係

\def\dint#1#2{{\displaystyle \int_{#1}^{#2}}} %文中で数式表示の積分を使う時

\def\ne{\nearrow}
\def\se{\searrow}
\def\nw{\nwarrow}
\def\ne{\nearrow}


%便利なやつ

\def\case#1#2{%
 \[\left\{%
 \begin{array}{l}%
 #1 \\%
 #2%
 \end{array}%
 \right.\] }                           %場合分け
 
\def\1{$\cos\theta=c$,$\sin\theta=s$とおく.}  %cs表示を与える前書きシータ
\def\2{$\cos t=c$,$\sin t=s$とおく.}     %cs表示を与える前書きt
\def\3{$\cos x=c$,$\sin x=s$とおく.}                %cs表示を与える前書きx

\def\fig#1#2#3 {%
\begin{wrapfigure}[#1]{r}{#2 zw}%
\vspace*{-1zh}%
\input{#3}%
\end{wrapfigure} }           %絵の挿入


\def\a{\alpha}   %ギリシャ文字
\def\b{\beta}
\def\g{\gamma}

%問題番号のためのマクロ

\newcounter{nombre} %必須
\renewcommand{\thenombre}{\arabic{nombre}} %任意
\setcounter{nombre}{2} %任意
\newcounter{nombresub}[nombre] %親子関係を定義
\renewcommand{\thenombresub}{\arabic{nombresub}} %任意
\setcounter{nombresub}{0} %任意
\newcommand{\prob}[1][]{\refstepcounter{nombre}#1[問題 \thenombre]}
\newcommand{\probsub}[1][]{\refstepcounter{nombresub}#1(\thenombresub)}


%1-1みたいなカウンタ(todaiとtodaia)
\newcounter{todai}
\setcounter{todai}{0}
\newcounter{todaisub}[todai] 
\setcounter{todaisub}{0} 
\newcommand{\todai}[1][]{\refstepcounter{todai}#1 \thetodai-\thetodaisub}
\newcommand{\todaib}[1][]{\refstepcounter{todai}#1\refstepcounter{todaisub}#1 {\bf [問題 \thetodai.\thetodaisub]}}
\newcommand{\todaia}[1][]{\refstepcounter{todaisub}#1 {\bf [問題 \thetodai.\thetodaisub]}}


     \begin{oframed}
     $i$を虚数単位とし$a=\cos\frac{\pi}{3}+i\sin\frac{\pi}{3}$とおく.また$n$はすべての自然数にわた
     って動くとする.このとき,
         \begin{enumerate}[(1)]
         \item $a^n$は何個の異なる値をとり得るか.
         \item $\frac{(1-a^n)(1-a^{2n})(1-a^{3n})(1-a^{4n})(1-a^{5n})}{(1-a)(1-a^2)(1-a^3)(1-a^4)(1-a^5)}$
         の値を求めよ
         \end{enumerate}
     \end{oframed}

\setlength{\columnseprule}{0.4pt}
\begin{multicols}{2}
{\bf[解]}ドモアブルの定理から$a^n=1$だから,$k=0,1,2,\dots,5$に対して
     \begin{align}
     a^{6n+k}=a^k\label{1}
     \end{align}
となる.
     \begin{enumerate}[(1)]
     \item \eqref{1}から$a^k$についてのみ考えればよいが,このとき$0\le i<j\le5$に対して
     $a^i=a^j$と仮定すると$a^{j-i}=1$となって矛盾.故に$a^i\not=a^j$だから$a^k$はすべて
     異なり,求めるのは$6$個である.
     \item 合同式の法を$6$とする.
          \begin{indentation}{2zw}{0pt}
          \noindent\underline{(i)$n\equiv \pm1$の時} 
          \[\{n,2n,3n,4n,5n\}\equiv\{1,2,3,4,5\}\]
         であるから,\eqref{1}から(与式)$=1$である.
         
         \underline{(ii)otherwise} \\
         $\{n,2n,3n,4n,5n\}$の中に合同式で$0$に等しいものがあるので,(与式)$=0$となる.
         \end{indentation}
    以上から
         \begin{align*}
         \text{(与式)}=\left\{
              \begin{array}{ll}
              1 &  (n\equiv \pm1)     \\
              0 &  (otherwise)
              \end{array}
         \right.\tag{答}
         \end{align*}
    となる.     
    \end{enumerate}
\newpage
\end{multicols}
\end{document}