\documentclass[a4j]{jarticle}
\usepackage{amsmath}
\usepackage{ascmac}
\usepackage{amssymb}
\usepackage{enumerate}
\usepackage{multicol}
\usepackage{framed}
\usepackage{fancyhdr}
\usepackage{latexsym}
\usepackage{indent}
\usepackage{cases}
\usepackage[dvips]{graphicx}
\usepackage{color}
\allowdisplaybreaks
\pagestyle{fancy}
\lhead{}
\chead{}
\rhead{東京大学前期$1970$年$3$番}
\begin{document}
%分数関係


\def\tfrac#1#2{{\textstyle\frac{#1}{#2}}} %数式中で文中表示の分数を使う時


%Σ関係

\def\dsum#1#2{{\displaystyle\sum_{#1}^{#2}}} %文中で数式表示のΣを使う時


%ベクトル関係


\def\vector#1{\overrightarrow{#1}}  %ベクトルを表現したいとき(aベクトルを表現するときは\ver
\def\norm#1{|\overrightarrow{#1}|} %ベクトルの絶対値
\def\vtwo#1#2{ \left(%
      \begin{array}{c}%
      #1 \\%
      #2 \\%
      \end{array}%
      \right) }                        %2次元ベクトル成分表示
      
      \def\vthree#1#2#3{ \left(
      \begin{array}{c}
      #1 \\
      #2 \\
      #3 \\
      \end{array}
      \right) }                        %3次元ベクトル成分表示



%数列関係


\def\an#1{\verb|{|$#1$\verb|}|}


%極限関係

\def\limit#1#2{\stackrel{#1 \to #2}{\longrightarrow}}   %等式変形からの極限
\def\dlim#1#2{{\displaystyle \lim_{#1\to#2}}} %文中で数式表示の極限を使う



%積分関係

\def\dint#1#2{{\displaystyle \int_{#1}^{#2}}} %文中で数式表示の積分を使う時

\def\ne{\nearrow}
\def\se{\searrow}
\def\nw{\nwarrow}
\def\ne{\nearrow}


%便利なやつ

\def\case#1#2{%
 \[\left\{%
 \begin{array}{l}%
 #1 \\%
 #2%
 \end{array}%
 \right.\] }                           %場合分け
 
\def\1{$\cos\theta=c$,$\sin\theta=s$とおく.}  %cs表示を与える前書きシータ
\def\2{$\cos t=c$,$\sin t=s$とおく.}     %cs表示を与える前書きt
\def\3{$\cos x=c$,$\sin x=s$とおく.}                %cs表示を与える前書きx

\def\fig#1#2#3 {%
\begin{wrapfigure}[#1]{r}{#2 zw}%
\vspace*{-1zh}%
\input{#3}%
\end{wrapfigure} }           %絵の挿入


\def\a{\alpha}   %ギリシャ文字
\def\b{\beta}
\def\g{\gamma}

%問題番号のためのマクロ

\newcounter{nombre} %必須
\renewcommand{\thenombre}{\arabic{nombre}} %任意
\setcounter{nombre}{2} %任意
\newcounter{nombresub}[nombre] %親子関係を定義
\renewcommand{\thenombresub}{\arabic{nombresub}} %任意
\setcounter{nombresub}{0} %任意
\newcommand{\prob}[1][]{\refstepcounter{nombre}#1[問題 \thenombre]}
\newcommand{\probsub}[1][]{\refstepcounter{nombresub}#1(\thenombresub)}


%1-1みたいなカウンタ(todaiとtodaia)
\newcounter{todai}
\setcounter{todai}{0}
\newcounter{todaisub}[todai] 
\setcounter{todaisub}{0} 
\newcommand{\todai}[1][]{\refstepcounter{todai}#1 \thetodai-\thetodaisub}
\newcommand{\todaib}[1][]{\refstepcounter{todai}#1\refstepcounter{todaisub}#1 {\bf [問題 \thetodai.\thetodaisub]}}
\newcommand{\todaia}[1][]{\refstepcounter{todaisub}#1 {\bf [問題 \thetodai.\thetodaisub]}}


     \begin{oframed}
     $25\,\ \mathrm{m}$隔てて二地点$P$,$Q$がある.いま$A$,$B$二人がそれぞれ$P$$Q$に立ち,同時に向かい合って走り出す.
     走り出してから$t$秒後の$A$,$B$の速度を,$P$から$Q$に向かう方向を正の向きとしてそれぞれ$u \,\ \mathrm{(m/s)}$,
     $v\,\ \mathrm{(m/s)}$とすれば,$u$は一定で,$v=3t^2/4-3t$である.
     
     このとき,$B$が$Q$に帰るまでに$A$が$B$に出会うかまたは追いつくためには,$u$が少なくともどれほどの大きさでなければ
     ならないか.
     \end{oframed}

\setlength{\columnseprule}{0.4pt}
\begin{multicols}{2}
{\bf[解]}$A$$B$の$P$からの距離をそれぞれ$A(t)$,$B(t)$とすると,
     \begin{align*}
     A(t)=\int_0^tudt+A(0)=ut \\
     B(t)=\int_0^tvdt+B(0)=\frac{1}{4}t^3-\frac{3}{2}t^2+25
     \end{align*}
である.$B(t)\le25\Longleftrightarrow0\le t\le6$であるから,$A$が$B$に追いつくには,$y=A(t)$と$y=B(t)$が$0<t<6$で少なくとも一回
交わればよい.$y$を消去して
     \begin{align}
     &A(t)=B(t) \nonumber \\
     \Longleftrightarrow &u=\frac{1}{4}t^2-\frac{3}{2}t+\frac{25}{t}\equiv f(t)\label{1}
     \end{align}
である.
      \[f'(t)=\frac{1}{2}t-\frac{3}{2}-\frac{25}{t^2}=\frac{(t-5)(t^2+2t+10)}{2t^2}\] 
から,下表を得る.
     \begin{align*}
          \begin{array}{|c|c|c|c|c|c|}\hline
          t  & 0 &     & 5     &     &6  \\ \hline
          f' &    & -   & 0     &+    &    \\  \hline
          f  &    &\se&15/4 &\ne &25/6  \\ \hline  
          \end{array}
     \end{align*}
よって,$f(t)\to\infty(t\to0)$と合わせて,グラフは下図.
     \begin{center}
     \scalebox{1}{a}
     \end{center}
従って\eqref{1}が$0<t<6$に解を持つ条件は
     \[\frac{15}{4}\le u\]     
であるから,求める最小値は$15/4 \,\ \mathrm{(m/s)}$である.$\cdots$(答)          
\newpage
\end{multicols}
\end{document}