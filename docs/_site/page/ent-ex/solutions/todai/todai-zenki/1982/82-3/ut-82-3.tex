\documentclass[a4j]{jarticle}
\usepackage{amsmath}
\usepackage{ascmac}
\usepackage{amssymb}
\usepackage{enumerate}
\usepackage{multicol}
\usepackage{framed}
\usepackage{fancyhdr}
\usepackage{latexsym}
\usepackage{indent}
\usepackage{cases}
\allowdisplaybreaks
\pagestyle{fancy}
\lhead{}
\chead{}
\rhead{東京大学前期$1982$年$3$番}
\begin{document}
%分数関係


\def\tfrac#1#2{{\textstyle\frac{#1}{#2}}} %数式中で文中表示の分数を使う時


%Σ関係

\def\dsum#1#2{{\displaystyle\sum_{#1}^{#2}}} %文中で数式表示のΣを使う時


%ベクトル関係


\def\vector#1{\overrightarrow{#1}}  %ベクトルを表現したいとき(aベクトルを表現するときは\ver
\def\norm#1{|\overrightarrow{#1}|} %ベクトルの絶対値
\def\vtwo#1#2{ \left(%
      \begin{array}{c}%
      #1 \\%
      #2 \\%
      \end{array}%
      \right) }                        %2次元ベクトル成分表示
      
      \def\vthree#1#2#3{ \left(
      \begin{array}{c}
      #1 \\
      #2 \\
      #3 \\
      \end{array}
      \right) }                        %3次元ベクトル成分表示



%数列関係


\def\an#1{\verb|{|$#1$\verb|}|}


%極限関係

\def\limit#1#2{\stackrel{#1 \to #2}{\longrightarrow}}   %等式変形からの極限
\def\dlim#1#2{{\displaystyle \lim_{#1\to#2}}} %文中で数式表示の極限を使う



%積分関係

\def\dint#1#2{{\displaystyle \int_{#1}^{#2}}} %文中で数式表示の積分を使う時

\def\ne{\nearrow}
\def\se{\searrow}
\def\nw{\nwarrow}
\def\ne{\nearrow}


%便利なやつ

\def\case#1#2{%
 \[\left\{%
 \begin{array}{l}%
 #1 \\%
 #2%
 \end{array}%
 \right.\] }                           %場合分け
 
\def\1{$\cos\theta=c$,$\sin\theta=s$とおく.}  %cs表示を与える前書きシータ
\def\2{$\cos t=c$,$\sin t=s$とおく.}     %cs表示を与える前書きt
\def\3{$\cos x=c$,$\sin x=s$とおく.}                %cs表示を与える前書きx

\def\fig#1#2#3 {%
\begin{wrapfigure}[#1]{r}{#2 zw}%
\vspace*{-1zh}%
\input{#3}%
\end{wrapfigure} }           %絵の挿入


\def\a{\alpha}   %ギリシャ文字
\def\b{\beta}
\def\g{\gamma}

%問題番号のためのマクロ

\newcounter{nombre} %必須
\renewcommand{\thenombre}{\arabic{nombre}} %任意
\setcounter{nombre}{2} %任意
\newcounter{nombresub}[nombre] %親子関係を定義
\renewcommand{\thenombresub}{\arabic{nombresub}} %任意
\setcounter{nombresub}{0} %任意
\newcommand{\prob}[1][]{\refstepcounter{nombre}#1[問題 \thenombre]}
\newcommand{\probsub}[1][]{\refstepcounter{nombresub}#1(\thenombresub)}


%1-1みたいなカウンタ(todaiとtodaia)
\newcounter{todai}
\setcounter{todai}{0}
\newcounter{todaisub}[todai] 
\setcounter{todaisub}{0} 
\newcommand{\todai}[1][]{\refstepcounter{todai}#1 \thetodai-\thetodaisub}
\newcommand{\todaib}[1][]{\refstepcounter{todai}#1\refstepcounter{todaisub}#1 {\bf [問題 \thetodai.\thetodaisub]}}
\newcommand{\todaia}[1][]{\refstepcounter{todaisub}#1 {\bf [問題 \thetodai.\thetodaisub]}}


     \begin{oframed}
     $xy$平面において,点$A$は原点$O$を中心とする半径$1$の円周の第一象限にある部分を動き,
     点$B$は$x$軸上を動く.ただし線分$AB$の長さは$1$であり,線分$AB$は両端$A$,$B$以外の
     点$C$で円周と交わるものとする.
          \begin{enumerate}[(1)]
          \item $\theta=\angle AOB$の取りうる値の範囲を求めよ.
          \item $BC$の長さを$\theta$で表せ.
          \item 線分$OB$の中点を$M$とするとき,線分$CM$の長さの範囲を求めよ.
          \end{enumerate}
     \end{oframed}

\setlength{\columnseprule}{0.4pt}
\begin{multicols}{2}
{\bf[解]}
\begin{enumerate}[(1)]
\item
 \1 ただし,題意から$0<\theta<\pi/2$である.また$A(c,s)$である.
この時$\triangle ABO$は$OA=AB=1$の二等辺三角形だから$B(2c,0)$となる.故に線分$AB$の方程式は,$c\not=0$に注意して,
      \begin{align}
      &(2c-c)(y-s)-(0-s)(x-s)=0     \nonumber\\
      &sx+cy-2sc=0                     \nonumber\\
      &y=\frac{s}{c}(-x+2c) \,\,\,\,\ (c\le x\le2c) \label{3}
      \end{align}
である.これが円周と交わるので,$y$を消した
     \begin{align}
     &x^2+\left(\frac{s}{c}\right)^2(-x+2c)^2=1  \nonumber\\
     &c^2x^2+s^2(-x+2c)^2=c^2 \nonumber\\
     &x^2-4s^2cx+c^2(4s^2-1)=0  \nonumber\\
     &(x-c)\left(x-c(4s^2-1)\right)=0  \label{1}
     \end{align}
が$c\le x\le2c$に解を持つ.従って$0<\theta<\pi/2$に注意して
     \begin{align}
     c<c(4s^2-1)<2c \nonumber\\
     1<4s^2-1<2 \nonumber\\
     \frac{1}{2}<s^2<\frac{3}{4} \label{4}\\
     \frac{\pi}{4}<\theta<\frac{\pi}{3}\label{2}
     \end{align}
である.$\cdots$(答)

     \item\eqref{1}から$C$の$x$座標は$c(4s^2-1)$である.さらに$AB$の傾きは\eqref{3}から
     $\dfrac{-s}{c}$であるから,
          \[|BC|=\sqrt{1+\left(\frac{s}{c}\right)^2}(2c-c(4s^2-1))=3-4s^2\]
     である.$\cdots$(答)
     
     \item $M(c,0)$である.故に$|MB|=c$.これと前問の結果,及び二等辺三角形の性質より
     $\angle ABO=\angle AOB=\theta$であることから,
     $\triangle BCM$に余弦定理を用いて
          \begin{align*}
          &|MC|^2 \\
          =&|MB|^2+|BC|^2-2|MB||BC|\cos\angle MBC  \\
          =&c^2+(3-4s^2)^2-2c(3-4s^2)\cos\angle ABO \\
          =&c^2+(3-4s^2)^2-2c(3-4s^2)c \\
          =&c^2(-5+8s^2)+(3-4s^2)^2 \\
          =&(1-p)(8p-5)+(4p-3)^2   \tag{$p=s^2$}\\ 
          =&8p^2+11p+4  \\
          =&8\left(p-\frac{11}{16}\right)^2+\frac{7}{32}
          \end{align*}     
     である.\eqref{4}から$\dfrac{1}{2}<p<\dfrac{3}{4}$であることを考慮すると,
          \begin{align*}
          \left\{
               \begin{array}{l}
               p=11/16\text{で}\min7/32 \\
               p\to 1/2\text{で}\max 1/2
               \end{array}
          \right.
          \end{align*}
     である.$|MC|>0$から,
          \begin{align*}
          &\sqrt{\frac{7}{32}}\le|MC|<\sqrt{\frac{1}{2}} \\
          \therefore \ &\frac{\sqrt{14}}{8}\le|MC|<\frac{\sqrt{2}}{2}
          \end{align*}
     である.$\cdots$(答)
     \end{enumerate}

{\bf[別解]}(1)について,$A(c,s)$での円の接線
     \[cx+sy=1\]
の$x$切片は$1/c$であるから,
円周と$AB$が交わるための条件は
      \[1<2c<\frac{1}{c}\Longleftrightarrow \frac{1}{2}<c<\frac{\sqrt{2}}{2}\]
であり,これを解いて
      \[\frac{\pi}{4}<\theta<\frac{\pi}{3}\]
となる.$\cdots$(答)
\newpage
\end{multicols}
\end{document}