\documentclass[a4j]{jarticle}
\usepackage{amsmath}
\usepackage{ascmac}
\usepackage{amssymb}
\usepackage{enumerate}
\usepackage{multicol}
\usepackage{framed}
\usepackage{fancyhdr}
\usepackage{latexsym}
\usepackage{indent}
\usepackage{cases}
\usepackage[dvips]{graphicx}
\usepackage{color}
\usepackage{emath}
\usepackage{emathPp}
\allowdisplaybreaks
\pagestyle{fancy}
\lhead{}
\chead{}
\rhead{東京大学前期$1977$年$1$番}
\begin{document}
%分数関係


\def\tfrac#1#2{{\textstyle\frac{#1}{#2}}} %数式中で文中表示の分数を使う時


%Σ関係

\def\dsum#1#2{{\displaystyle\sum_{#1}^{#2}}} %文中で数式表示のΣを使う時


%ベクトル関係


\def\vector#1{\overrightarrow{#1}}  %ベクトルを表現したいとき(aベクトルを表現するときは\ver
\def\norm#1{|\overrightarrow{#1}|} %ベクトルの絶対値
\def\vtwo#1#2{ \left(%
      \begin{array}{c}%
      #1 \\%
      #2 \\%
      \end{array}%
      \right) }                        %2次元ベクトル成分表示
      
      \def\vthree#1#2#3{ \left(
      \begin{array}{c}
      #1 \\
      #2 \\
      #3 \\
      \end{array}
      \right) }                        %3次元ベクトル成分表示



%数列関係


\def\an#1{\verb|{|$#1$\verb|}|}


%極限関係

\def\limit#1#2{\stackrel{#1 \to #2}{\longrightarrow}}   %等式変形からの極限
\def\dlim#1#2{{\displaystyle \lim_{#1\to#2}}} %文中で数式表示の極限を使う



%積分関係

\def\dint#1#2{{\displaystyle \int_{#1}^{#2}}} %文中で数式表示の積分を使う時

\def\ne{\nearrow}
\def\se{\searrow}
\def\nw{\nwarrow}
\def\ne{\nearrow}


%便利なやつ

\def\case#1#2{%
 \[\left\{%
 \begin{array}{l}%
 #1 \\%
 #2%
 \end{array}%
 \right.\] }                           %場合分け
 
\def\1{$\cos\theta=c$,$\sin\theta=s$とおく.}  %cs表示を与える前書きシータ
\def\2{$\cos t=c$,$\sin t=s$とおく.}     %cs表示を与える前書きt
\def\3{$\cos x=c$,$\sin x=s$とおく.}                %cs表示を与える前書きx

\def\fig#1#2#3 {%
\begin{wrapfigure}[#1]{r}{#2 zw}%
\vspace*{-1zh}%
\input{#3}%
\end{wrapfigure} }           %絵の挿入


\def\a{\alpha}   %ギリシャ文字
\def\b{\beta}
\def\g{\gamma}

%問題番号のためのマクロ

\newcounter{nombre} %必須
\renewcommand{\thenombre}{\arabic{nombre}} %任意
\setcounter{nombre}{2} %任意
\newcounter{nombresub}[nombre] %親子関係を定義
\renewcommand{\thenombresub}{\arabic{nombresub}} %任意
\setcounter{nombresub}{0} %任意
\newcommand{\prob}[1][]{\refstepcounter{nombre}#1[問題 \thenombre]}
\newcommand{\probsub}[1][]{\refstepcounter{nombresub}#1(\thenombresub)}


%1-1みたいなカウンタ(todaiとtodaia)
\newcounter{todai}
\setcounter{todai}{0}
\newcounter{todaisub}[todai] 
\setcounter{todaisub}{0} 
\newcommand{\todai}[1][]{\refstepcounter{todai}#1 \thetodai-\thetodaisub}
\newcommand{\todaib}[1][]{\refstepcounter{todai}#1\refstepcounter{todaisub}#1 {\bf [問題 \thetodai.\thetodaisub]}}
\newcommand{\todaia}[1][]{\refstepcounter{todaisub}#1 {\bf [問題 \thetodai.\thetodaisub]}}


     \begin{oframed}
     $k$を実数の定数とするとき,$x$の関数$f(x)$が$-1\le x\le1$の範囲でとる最大値を$M(k)$で表す.$k$が実数
     全体を動くとき,$M(k)$が最小となる$k$の値および$M(k)$の最小値を求めよ.
     \end{oframed}

\setlength{\columnseprule}{0.4pt}
\begin{multicols}{2}
{\bf[解]} $g(x)=x^3-3kx$とおく.$g(x)$は奇関数だから,$f(x)$は偶関数.故に$0\le x$で考える.
     \[g'(x)=3(x^2-k)\]
であるから,$k$の値によって場合分けする.\\

     \begin{indentation}{2zw}{0pt}
     \noindent\underline{(i)$k\le0$の時} \\
     $g'(x)\ge0$だから$g(x)$は単調増加で
          \begin{align*}
          &g(0)=0&g(1)=|1-3k|\ge0
          \end{align*}
     に注意して
          \begin{align*}
          M(k)=g(1)=|1-3k|
          \end{align*}
     である.\\
     
     \noindent\underline{(ii)$0<k\le 1$の時}\\
     下表を得る.
          \[
          \begin{array}{|c|c|c|c|c|c|}\hline
          x & 0 &    &\sqrt{k}&     & 1 \\   \hline
          g'&    & -  & 0        & +  &    \\ \hline
          g&    &\se&           &\ne&    \\ \hline
          \end{array}
          \]
     故に(i)と同様にして($f(0)$は最大値の候補から除外できて)
          \begin{align*}
          M(k)&=\max\{f(0),f(\sqrt{k}),f(1)\} \\
          &=\max\{|1-3k|,2k\sqrt{k}\}
          \end{align*}
     である.\\
     
     \noindent\underline{(iii)$1\le k$の時}\\
     下表を得る.
          \[
          \begin{array}{|c|c|c|c|}\hline
          x & 0 &     & 1 \\ \hline
          g'&    & -   &    \\ \hline
          g&    &\se &    \\ \hline
          \end{array}
          \]     
      故に(i)と同様にして,
           \begin{align*}
           M(k)=f(1)=|1-3k|
           \end{align*}
      \end{indentation}
      
 以上をまとめて
      \begin{align}
      M(k)=
           \begin{cases}
           |1-3k|&k\le0,1\le k \\
           \max\{|1-3k|,2k\sqrt{k}\}&0<k\le1
           \end{cases}
      \end{align}
 であるから,図示して下図.\\
 
      \begin{zahyou}[ul=20mm,yokozikukigou=$k$](-1,2)(-0.5,3)
      \def\Fx{1-3*X}
      \def\Hx{-1+3*X}
      \def\Gx{2*X*sqrt(X)}
      \tenretu*{A(1,2);B(0.25,0.25);}
      \YGurafu*[migix=0.25]\Fx
      \YGurafu\Gx{0.25}{1}
      \YGurafu\Hx{1}{\xmax}
      \YGurafu(0.05)(0.02)\Fx{0.25}{1/3}
      \YGurafu(0.05)(0.02)\Gx{0}{0.25}
      \YGurafu(0.05)(0.02)\Hx{1/3}{1}
      \Put\A[syaei=xy,xlabel=1,ylabel=2]{}
      \Put\B[syaei=xy,xlabel=\frac{1}{4},ylabel=\frac{1}{4}]{}
      \Put\A[e]{$y=3k-1$}
      \Put{(0,1)}[w]{$y=1-3k$}
      \Put{(0.2,1,5)}{$y=2k^{\frac{3}{2}}$}
      \end{zahyou}
      
故に求める最小値は
     \[M\left(\frac{1}{4}\right)=\frac{1}{4}\]      
である.$\cdots$(答)
      
\newpage
\end{multicols}
\end{document}