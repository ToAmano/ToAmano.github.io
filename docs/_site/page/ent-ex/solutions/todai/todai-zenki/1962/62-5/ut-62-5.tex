\documentclass[a4j]{jarticle}
\usepackage{amsmath}
\usepackage{ascmac}
\usepackage{amssymb}
\usepackage{enumerate}
\usepackage{multicol}
\usepackage{framed}
\usepackage{fancyhdr}
\usepackage{latexsym}
\usepackage{indent}
\usepackage{cases}
\usepackage[dvips]{graphicx}
\usepackage{color}
\allowdisplaybreaks
\pagestyle{fancy}
\lhead{}
\chead{}
\rhead{東京大学前期$1962$年$5$番}
\begin{document}
%分数関係


\def\tfrac#1#2{{\textstyle\frac{#1}{#2}}} %数式中で文中表示の分数を使う時


%Σ関係

\def\dsum#1#2{{\displaystyle\sum_{#1}^{#2}}} %文中で数式表示のΣを使う時


%ベクトル関係


\def\vector#1{\overrightarrow{#1}}  %ベクトルを表現したいとき(aベクトルを表現するときは\ver
\def\norm#1{|\overrightarrow{#1}|} %ベクトルの絶対値
\def\vtwo#1#2{ \left(%
      \begin{array}{c}%
      #1 \\%
      #2 \\%
      \end{array}%
      \right) }                        %2次元ベクトル成分表示
      
      \def\vthree#1#2#3{ \left(
      \begin{array}{c}
      #1 \\
      #2 \\
      #3 \\
      \end{array}
      \right) }                        %3次元ベクトル成分表示



%数列関係


\def\an#1{\verb|{|$#1$\verb|}|}


%極限関係

\def\limit#1#2{\stackrel{#1 \to #2}{\longrightarrow}}   %等式変形からの極限
\def\dlim#1#2{{\displaystyle \lim_{#1\to#2}}} %文中で数式表示の極限を使う



%積分関係

\def\dint#1#2{{\displaystyle \int_{#1}^{#2}}} %文中で数式表示の積分を使う時

\def\ne{\nearrow}
\def\se{\searrow}
\def\nw{\nwarrow}
\def\ne{\nearrow}


%便利なやつ

\def\case#1#2{%
 \[\left\{%
 \begin{array}{l}%
 #1 \\%
 #2%
 \end{array}%
 \right.\] }                           %場合分け
 
\def\1{$\cos\theta=c$,$\sin\theta=s$とおく.}  %cs表示を与える前書きシータ
\def\2{$\cos t=c$,$\sin t=s$とおく.}     %cs表示を与える前書きt
\def\3{$\cos x=c$,$\sin x=s$とおく.}                %cs表示を与える前書きx

\def\fig#1#2#3 {%
\begin{wrapfigure}[#1]{r}{#2 zw}%
\vspace*{-1zh}%
\input{#3}%
\end{wrapfigure} }           %絵の挿入


\def\a{\alpha}   %ギリシャ文字
\def\b{\beta}
\def\g{\gamma}

%問題番号のためのマクロ

\newcounter{nombre} %必須
\renewcommand{\thenombre}{\arabic{nombre}} %任意
\setcounter{nombre}{2} %任意
\newcounter{nombresub}[nombre] %親子関係を定義
\renewcommand{\thenombresub}{\arabic{nombresub}} %任意
\setcounter{nombresub}{0} %任意
\newcommand{\prob}[1][]{\refstepcounter{nombre}#1[問題 \thenombre]}
\newcommand{\probsub}[1][]{\refstepcounter{nombresub}#1(\thenombresub)}


%1-1みたいなカウンタ(todaiとtodaia)
\newcounter{todai}
\setcounter{todai}{0}
\newcounter{todaisub}[todai] 
\setcounter{todaisub}{0} 
\newcommand{\todai}[1][]{\refstepcounter{todai}#1 \thetodai-\thetodaisub}
\newcommand{\todaib}[1][]{\refstepcounter{todai}#1\refstepcounter{todaisub}#1 {\bf [問題 \thetodai.\thetodaisub]}}
\newcommand{\todaia}[1][]{\refstepcounter{todaisub}#1 {\bf [問題 \thetodai.\thetodaisub]}}


     \begin{oframed}
     $1$つの頂点から出る$3$辺の長さが$x$,$y$,$z$であるような直方体において,
     $x$,$y$,$z$の和が$6$,全表面積が$18$であるとき,
          \begin{enumerate}[(i)]
          \item $x$のとりうる値の範囲を求めよ.
          \item このような直方体の体積の最大値を求めよ.
          \end{enumerate}
     \end{oframed}

\setlength{\columnseprule}{0.4pt}
\begin{multicols}{2}
{\bf[解]}まず題意から
     \begin{subnumcases}
     {}
     x,y,z>0 \label{1} \\
     x+y+z=6  \label{2}\\
     yz+zx+xy=9  \label{3}
     \end{subnumcases}
となる.     
     \begin{enumerate}[(i)]
     \item  まず$s=y+z$,$t=yz$として,$y$,$z$の存在条件を考える.\eqref{2},\eqref{3}
     から
          \begin{align}
          \left\{
               \begin{array}{l}
               s=6-x \\
               t=x^2-6x+9
               \end{array}
          \right.\label{4}
          \end{align}
     となる.
     
     さて,\eqref{1},\eqref{4}から$y$,$z$は$p$の$2$次方程式
          \begin{align*}
          &p^2-sp+t=0\\ 
          \Longleftrightarrow &p^2-(6-x)p+(x^2-6x+9)=0 
          \end{align*}
     の正$2$実解であるから,判別式$D$として     
          \begin{align*}
          &\left\{
               \begin{array}{l}
               D\ge0  \\
               s,t>0
               \end{array}
          \right.  \\
          \Longleftrightarrow&\left\{
               \begin{array}{l}
               (6-x)^2-4(x^2-6x+9)\ge0  \\
               6-x>0  \\
               x^2-6x+9>0
               \end{array}
          \right.\\
          \Longleftrightarrow&\left\{
               \begin{array}{l}
               (6-x)^2-4(x^2-6x+9)\ge0  \\
               6-x>0  \\
               x^2-6x+9>0
               \end{array}
          \right.\\
          \Longleftrightarrow&\left\{
               \begin{array}{l}
               0\le x\le 4  \\
               x<6  \\
               x\not=3
               \end{array}
          \right. \\
          \Longleftrightarrow&
               0\le x<3,3<x\le 4  \tag{答}
          \end{align*} 
となる.     
     \item $a=x+y+z$,$b=yz+zx+xy$,$c=xyz$とおく.すると$x$,$y$,$z$は$p$の$3$次方程式
          \begin{align}
          &p^3-ap^2+bp-c=0  \nonumber\\
          \Longleftrightarrow&p^3-6p^2+9p-c=0 \label{5}
          \end{align}
     の正実解である.故にこれが正の$3$実解(重解含む)をもつ条件を求める.
          \begin{center}          
          \eqref{5}が正の$3$実解を持つ.\\
          $\Longleftrightarrow$$y=f(p)=p^3-6p^2+9p$と$y=c$が$0<p$に$3$つの共有点(重解含む)を持
          つ.
          \end{center}
     である.
          \begin{align*}
          f'(p)&=3p^2-12p+9  \\
          &=3(p-1)(p-3)
          \end{align*}
     であるから下表を得る.
          \begin{align*}
               \begin{array}{|c|c|c|c|c|c|}  \hline
               p &    &  1  &      &  3  &     \\ \hline
               f' &+  &  0  & -    &  0  & +  \\ \hline     
               f &\ne&  4  &\se &  0  &\ne  \\ \hline
               \end{array}
          \end{align*}
     従って,$y=f(p)$のグラフは下のようになり,故に$c$の値域は$0\le c\le4$である.
          \begin{center}
          \scalebox{1}{%WinTpicVersion4.32a
{\unitlength 0.1in%
\begin{picture}(22.2000,20.4000)(1.8000,-24.4000)%
% STR 2 0 3 0 Black White  
% 4 390 2397 390 2410 4 400 0 0
% O
\put(3.9000,-24.1000){\makebox(0,0)[rt]{O}}%
% STR 2 0 3 0 Black White  
% 4 360 387 360 400 4 400 0 0
% $y$
\put(3.6000,-4.0000){\makebox(0,0)[rt]{$y$}}%
% STR 2 0 3 0 Black White  
% 4 2400 2427 2400 2440 4 400 0 0
% $x$
\put(24.0000,-24.4000){\makebox(0,0)[rt]{$x$}}%
% VECTOR 2 0 3 0 Black White  
% 2 400 2400 400 400
% 
\special{pn 8}%
\special{pa 400 2400}%
\special{pa 400 400}%
\special{fp}%
\special{sh 1}%
\special{pa 400 400}%
\special{pa 380 467}%
\special{pa 400 453}%
\special{pa 420 467}%
\special{pa 400 400}%
\special{fp}%
% VECTOR 2 0 3 0 Black White  
% 2 400 2400 2400 2400
% 
\special{pn 8}%
\special{pa 400 2400}%
\special{pa 2400 2400}%
\special{fp}%
\special{sh 1}%
\special{pa 2400 2400}%
\special{pa 2333 2380}%
\special{pa 2347 2400}%
\special{pa 2333 2420}%
\special{pa 2400 2400}%
\special{fp}%
% FUNC 2 0 3 0 Black White  
% 10 400 400 2400 2400 400 2400 800 2400 400 2000 400 400 2400 2400 0 2 0 0 1 0
% x^3-6x^2+9x
\special{pn 8}%
\special{pa 400 2400}%
\special{pa 405 2355}%
\special{pa 410 2311}%
\special{pa 415 2268}%
\special{pa 425 2184}%
\special{pa 430 2143}%
\special{pa 435 2103}%
\special{pa 445 2025}%
\special{pa 455 1949}%
\special{pa 470 1841}%
\special{pa 480 1773}%
\special{pa 490 1707}%
\special{pa 495 1675}%
\special{pa 505 1613}%
\special{pa 510 1583}%
\special{pa 520 1525}%
\special{pa 525 1497}%
\special{pa 535 1443}%
\special{pa 545 1391}%
\special{pa 550 1366}%
\special{pa 560 1318}%
\special{pa 565 1295}%
\special{pa 575 1251}%
\special{pa 585 1209}%
\special{pa 595 1169}%
\special{pa 600 1150}%
\special{pa 615 1096}%
\special{pa 620 1079}%
\special{pa 630 1047}%
\special{pa 635 1032}%
\special{pa 640 1018}%
\special{pa 645 1003}%
\special{pa 660 964}%
\special{pa 670 940}%
\special{pa 675 929}%
\special{pa 690 899}%
\special{pa 700 881}%
\special{pa 710 865}%
\special{pa 720 851}%
\special{pa 735 833}%
\special{pa 740 828}%
\special{pa 760 812}%
\special{pa 765 809}%
\special{pa 780 803}%
\special{pa 795 800}%
\special{pa 805 800}%
\special{pa 820 803}%
\special{pa 835 809}%
\special{pa 850 818}%
\special{pa 865 830}%
\special{pa 885 850}%
\special{pa 895 862}%
\special{pa 900 869}%
\special{pa 905 875}%
\special{pa 910 882}%
\special{pa 915 890}%
\special{pa 920 897}%
\special{pa 935 921}%
\special{pa 960 966}%
\special{pa 975 996}%
\special{pa 980 1007}%
\special{pa 985 1017}%
\special{pa 1005 1061}%
\special{pa 1015 1085}%
\special{pa 1020 1096}%
\special{pa 1025 1108}%
\special{pa 1030 1121}%
\special{pa 1035 1133}%
\special{pa 1040 1146}%
\special{pa 1045 1158}%
\special{pa 1065 1210}%
\special{pa 1070 1224}%
\special{pa 1075 1237}%
\special{pa 1085 1265}%
\special{pa 1090 1278}%
\special{pa 1105 1320}%
\special{pa 1110 1335}%
\special{pa 1120 1363}%
\special{pa 1125 1378}%
\special{pa 1130 1392}%
\special{pa 1135 1407}%
\special{pa 1140 1421}%
\special{pa 1155 1466}%
\special{pa 1160 1480}%
\special{pa 1240 1720}%
\special{pa 1245 1734}%
\special{pa 1260 1779}%
\special{pa 1265 1793}%
\special{pa 1270 1808}%
\special{pa 1275 1822}%
\special{pa 1280 1837}%
\special{pa 1290 1865}%
\special{pa 1295 1880}%
\special{pa 1310 1922}%
\special{pa 1315 1935}%
\special{pa 1325 1963}%
\special{pa 1330 1976}%
\special{pa 1335 1990}%
\special{pa 1355 2042}%
\special{pa 1360 2054}%
\special{pa 1365 2067}%
\special{pa 1370 2079}%
\special{pa 1375 2092}%
\special{pa 1380 2104}%
\special{pa 1385 2115}%
\special{pa 1395 2139}%
\special{pa 1415 2183}%
\special{pa 1420 2193}%
\special{pa 1425 2204}%
\special{pa 1440 2234}%
\special{pa 1465 2279}%
\special{pa 1480 2303}%
\special{pa 1485 2310}%
\special{pa 1490 2318}%
\special{pa 1495 2325}%
\special{pa 1500 2331}%
\special{pa 1505 2338}%
\special{pa 1515 2350}%
\special{pa 1535 2370}%
\special{pa 1550 2382}%
\special{pa 1565 2391}%
\special{pa 1580 2397}%
\special{pa 1595 2400}%
\special{pa 1605 2400}%
\special{pa 1620 2397}%
\special{pa 1635 2391}%
\special{pa 1640 2388}%
\special{pa 1660 2372}%
\special{pa 1665 2367}%
\special{pa 1680 2349}%
\special{pa 1690 2335}%
\special{pa 1700 2319}%
\special{pa 1710 2301}%
\special{pa 1725 2271}%
\special{pa 1730 2260}%
\special{pa 1740 2236}%
\special{pa 1755 2197}%
\special{pa 1760 2182}%
\special{pa 1765 2168}%
\special{pa 1770 2153}%
\special{pa 1780 2121}%
\special{pa 1785 2104}%
\special{pa 1800 2050}%
\special{pa 1805 2031}%
\special{pa 1815 1991}%
\special{pa 1825 1949}%
\special{pa 1835 1905}%
\special{pa 1840 1882}%
\special{pa 1850 1834}%
\special{pa 1855 1809}%
\special{pa 1865 1757}%
\special{pa 1875 1703}%
\special{pa 1880 1675}%
\special{pa 1890 1617}%
\special{pa 1895 1587}%
\special{pa 1905 1525}%
\special{pa 1910 1493}%
\special{pa 1920 1427}%
\special{pa 1930 1359}%
\special{pa 1945 1251}%
\special{pa 1955 1175}%
\special{pa 1965 1097}%
\special{pa 1970 1057}%
\special{pa 1975 1016}%
\special{pa 1985 932}%
\special{pa 1990 889}%
\special{pa 1995 845}%
\special{pa 2005 755}%
\special{pa 2010 708}%
\special{pa 2015 662}%
\special{pa 2025 566}%
\special{pa 2040 416}%
\special{pa 2042 400}%
\special{fp}%
% LINE 2 2 3 0 Black White  
% 6 2000 2400 2000 800 400 800 2000 800 800 800 800 2400
% 
\special{pn 8}%
\special{pa 2000 2400}%
\special{pa 2000 800}%
\special{dt 0.045}%
\special{pa 400 800}%
\special{pa 2000 800}%
\special{dt 0.045}%
\special{pa 800 800}%
\special{pa 800 2400}%
\special{dt 0.045}%
% STR 2 0 3 0 Black White  
% 4 800 2300 800 2400 2 0 1 0
% $1$
\put(8.0000,-24.0000){\makebox(0,0)[lb]{{\colorbox[named]{White}{$1$}}}}%
% STR 2 0 3 0 Black White  
% 4 1600 2310 1600 2410 2 0 1 0
% $2$
\put(16.0000,-24.1000){\makebox(0,0)[lb]{{\colorbox[named]{White}{$2$}}}}%
% STR 2 0 3 0 Black White  
% 4 2000 2300 2000 2400 2 0 1 0
% $4$
\put(20.0000,-24.0000){\makebox(0,0)[lb]{{\colorbox[named]{White}{$4$}}}}%
% STR 2 0 3 0 Black White  
% 4 400 700 400 800 2 0 1 0
% $4$
\put(4.0000,-8.0000){\makebox(0,0)[lb]{{\colorbox[named]{White}{$4$}}}}%
\end{picture}}%
}
          \end{center}
     さて,直方体の体積は$c$に等しいから$\max c=4\cdots$(答)である.     
     \end{enumerate}
     
\vspace{3zh}
{\bf[別解]} 

[解]と同様に,$x$,$y$,$z$は$p$の$3$次方程式$p^3-6p^2+9p-c=0$の正の$3$実解であるから,$y=f(p)$と$y=c$の共有点の
$p$座標である.ゆえにグラフから,$x$の値域は$0<x\le4$である.$\cdots$(答)    
\newpage
\end{multicols}
\end{document}