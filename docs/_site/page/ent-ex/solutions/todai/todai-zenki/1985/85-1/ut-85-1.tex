\documentclass[a4j]{jarticle}
\usepackage{amsmath}
\usepackage{ascmac}
\usepackage{amssymb}
\usepackage{enumerate}
\usepackage{multicol}
\usepackage{framed}
\usepackage{fancyhdr}
\usepackage{latexsym}
\usepackage{indent}
\usepackage[dvips]{graphicx}
\usepackage{color}
\usepackage{emath}
\usepackage{emathPp}
\usepackage{cases}
\allowdisplaybreaks
\pagestyle{fancy}
\lhead{}
\chead{}
\rhead{東京大学前期$1985$年$1$番}
\begin{document}
%分数関係


\def\tfrac#1#2{{\textstyle\frac{#1}{#2}}} %数式中で文中表示の分数を使う時


%Σ関係

\def\dsum#1#2{{\displaystyle\sum_{#1}^{#2}}} %文中で数式表示のΣを使う時


%ベクトル関係


\def\vector#1{\overrightarrow{#1}}  %ベクトルを表現したいとき(aベクトルを表現するときは\ver
\def\norm#1{|\overrightarrow{#1}|} %ベクトルの絶対値
\def\vtwo#1#2{ \left(%
      \begin{array}{c}%
      #1 \\%
      #2 \\%
      \end{array}%
      \right) }                        %2次元ベクトル成分表示
      
      \def\vthree#1#2#3{ \left(
      \begin{array}{c}
      #1 \\
      #2 \\
      #3 \\
      \end{array}
      \right) }                        %3次元ベクトル成分表示



%数列関係


\def\an#1{\verb|{|$#1$\verb|}|}


%極限関係

\def\limit#1#2{\stackrel{#1 \to #2}{\longrightarrow}}   %等式変形からの極限
\def\dlim#1#2{{\displaystyle \lim_{#1\to#2}}} %文中で数式表示の極限を使う



%積分関係

\def\dint#1#2{{\displaystyle \int_{#1}^{#2}}} %文中で数式表示の積分を使う時

\def\ne{\nearrow}
\def\se{\searrow}
\def\nw{\nwarrow}
\def\ne{\nearrow}


%便利なやつ

\def\case#1#2{%
 \[\left\{%
 \begin{array}{l}%
 #1 \\%
 #2%
 \end{array}%
 \right.\] }                           %場合分け
 
\def\1{$\cos\theta=c$,$\sin\theta=s$とおく.}  %cs表示を与える前書きシータ
\def\2{$\cos t=c$,$\sin t=s$とおく.}     %cs表示を与える前書きt
\def\3{$\cos x=c$,$\sin x=s$とおく.}                %cs表示を与える前書きx

\def\fig#1#2#3 {%
\begin{wrapfigure}[#1]{r}{#2 zw}%
\vspace*{-1zh}%
\input{#3}%
\end{wrapfigure} }           %絵の挿入


\def\a{\alpha}   %ギリシャ文字
\def\b{\beta}
\def\g{\gamma}

%問題番号のためのマクロ

\newcounter{nombre} %必須
\renewcommand{\thenombre}{\arabic{nombre}} %任意
\setcounter{nombre}{2} %任意
\newcounter{nombresub}[nombre] %親子関係を定義
\renewcommand{\thenombresub}{\arabic{nombresub}} %任意
\setcounter{nombresub}{0} %任意
\newcommand{\prob}[1][]{\refstepcounter{nombre}#1[問題 \thenombre]}
\newcommand{\probsub}[1][]{\refstepcounter{nombresub}#1(\thenombresub)}


%1-1みたいなカウンタ(todaiとtodaia)
\newcounter{todai}
\setcounter{todai}{0}
\newcounter{todaisub}[todai] 
\setcounter{todaisub}{0} 
\newcommand{\todai}[1][]{\refstepcounter{todai}#1 \thetodai-\thetodaisub}
\newcommand{\todaib}[1][]{\refstepcounter{todai}#1\refstepcounter{todaisub}#1 {\bf [問題 \thetodai.\thetodaisub]}}
\newcommand{\todaia}[1][]{\refstepcounter{todaisub}#1 {\bf [問題 \thetodai.\thetodaisub]}}


\preEqlabel{$\cdots$}
     \begin{oframed}
     $a\ge 1$とする.$xy$平面において,不等式$0\le x\le\pi/2$,$1\le y\le a\sin x$によって定められる領域
     の面積を$S_1$,不等式$0\le x\le\pi/2$,$0\le y\le a\sin x$,$0\le y\le1$によって定められる領域の面積
     を$S_2$とする.$S_2-S_1$を最大にするような$a$の値と,$S_2-S_1$の最大値を求めよ.
     \end{oframed}

\setlength{\columnseprule}{0.4pt}
\begin{multicols}{2}
{\bf[解]} グラフの概形は下図である.
     
     \begin{zahyou}[ul=10mm,gentenhaiti={[se]}](-0.5,4)(-0.5,2.5)
     \def\Fx{2*sin(X/2)}
     \def\Gx{1}
     \calcval{$pi}\A
     \calcval{$pi/3}\B
     \calcval{3*$pi/4}\C
     \YGurafu*\Fx
     \YGurafu*\Gx
     \Put{(\A,2)}[syaei=xy,xlabel=\frac{\pi}{2},ylabel=a]{}
     \Put{(\B,1)}[syaei=x,xlabel=t,]{}
     \Put{(\C,1.2)}[n]{$S_1$}
     \Put{(\C,0.8)}[s]{$S_2$}
     \end{zahyou}

$y=a\sin x$と$y=1$の交点の$x$座標$t$として
     \begin{align}
     &a\sin t=1&0<t\le\frac{\pi}{2}\label{1}
     \end{align}
である.以下\2 

まず,
     \[S=\int_0^{\pi/2}a\sin x\,dx=a\]
とおく.題意から
     \[S_1=\int_t^{\pi/2}(a\sin x-1)\,dx\]
 である.$f(a)=S_2-S_1$は
     \begin{align*}
     f(a)&=S-2S_1 \\
     &=a-2ac+\pi-2t \\
     &=\frac{1}{s}-\frac{2c}{s}-2t+\pi&(\because\eqref{1})
     \end{align*}
である.これを$g(t)$とおく.
     \begin{align*}
     g'(t)&=\frac{-c}{s^2}+\frac{2}{s^2}-2 \\
     &=\frac{2-c}{s^2}-2 \\
     &=\frac{c(2c-1)}{1-c^2}
     \end{align*}
ゆえ,下表を得る.
     \begin{align*}
          \begin{array}{|c|c|c|c|c|c|}\hline
          t  &0 &     &\pi/3&     &\pi/2 \\ \hline
          f' &   &+   &0     &-    & 0     \\ \hline
          f &    &\ne&\pi/3&\se&       \\ \hline 
          \end{array}
     \end{align*}
従って,求める最大値は$t=\pi/3$の時の$\pi/3$である.このとき\eqref{1}から
     \[a=\frac{2\sqrt{3}}{3}\]
である.$\cdots$(答)
\newpage
\end{multicols}
\end{document}