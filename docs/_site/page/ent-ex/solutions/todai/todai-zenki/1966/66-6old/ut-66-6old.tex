\documentclass[a4j]{jarticle}
\usepackage{amsmath}
\usepackage{ascmac}
\usepackage{amssymb}
\usepackage{enumerate}
\usepackage{multicol}
\usepackage{framed}
\usepackage{fancyhdr}
\usepackage{latexsym}
\usepackage{indent}
\usepackage{cases}
\allowdisplaybreaks
\pagestyle{fancy}
\lhead{}
\chead{}
\rhead{東京大学前期$1966$年$6old$番}
\begin{document}
%分数関係


\def\tfrac#1#2{{\textstyle\frac{#1}{#2}}} %数式中で文中表示の分数を使う時


%Σ関係

\def\dsum#1#2{{\displaystyle\sum_{#1}^{#2}}} %文中で数式表示のΣを使う時


%ベクトル関係


\def\vector#1{\overrightarrow{#1}}  %ベクトルを表現したいとき(aベクトルを表現するときは\ver
\def\norm#1{|\overrightarrow{#1}|} %ベクトルの絶対値
\def\vtwo#1#2{ \left(%
      \begin{array}{c}%
      #1 \\%
      #2 \\%
      \end{array}%
      \right) }                        %2次元ベクトル成分表示
      
      \def\vthree#1#2#3{ \left(
      \begin{array}{c}
      #1 \\
      #2 \\
      #3 \\
      \end{array}
      \right) }                        %3次元ベクトル成分表示



%数列関係


\def\an#1{\verb|{|$#1$\verb|}|}


%極限関係

\def\limit#1#2{\stackrel{#1 \to #2}{\longrightarrow}}   %等式変形からの極限
\def\dlim#1#2{{\displaystyle \lim_{#1\to#2}}} %文中で数式表示の極限を使う



%積分関係

\def\dint#1#2{{\displaystyle \int_{#1}^{#2}}} %文中で数式表示の積分を使う時

\def\ne{\nearrow}
\def\se{\searrow}
\def\nw{\nwarrow}
\def\ne{\nearrow}


%便利なやつ

\def\case#1#2{%
 \[\left\{%
 \begin{array}{l}%
 #1 \\%
 #2%
 \end{array}%
 \right.\] }                           %場合分け
 
\def\1{$\cos\theta=c$,$\sin\theta=s$とおく.}  %cs表示を与える前書きシータ
\def\2{$\cos t=c$,$\sin t=s$とおく.}     %cs表示を与える前書きt
\def\3{$\cos x=c$,$\sin x=s$とおく.}                %cs表示を与える前書きx

\def\fig#1#2#3 {%
\begin{wrapfigure}[#1]{r}{#2 zw}%
\vspace*{-1zh}%
\input{#3}%
\end{wrapfigure} }           %絵の挿入


\def\a{\alpha}   %ギリシャ文字
\def\b{\beta}
\def\g{\gamma}

%問題番号のためのマクロ

\newcounter{nombre} %必須
\renewcommand{\thenombre}{\arabic{nombre}} %任意
\setcounter{nombre}{2} %任意
\newcounter{nombresub}[nombre] %親子関係を定義
\renewcommand{\thenombresub}{\arabic{nombresub}} %任意
\setcounter{nombresub}{0} %任意
\newcommand{\prob}[1][]{\refstepcounter{nombre}#1[問題 \thenombre]}
\newcommand{\probsub}[1][]{\refstepcounter{nombresub}#1(\thenombresub)}


%1-1みたいなカウンタ(todaiとtodaia)
\newcounter{todai}
\setcounter{todai}{0}
\newcounter{todaisub}[todai] 
\setcounter{todaisub}{0} 
\newcommand{\todai}[1][]{\refstepcounter{todai}#1 \thetodai-\thetodaisub}
\newcommand{\todaib}[1][]{\refstepcounter{todai}#1\refstepcounter{todaisub}#1 {\bf [問題 \thetodai.\thetodaisub]}}
\newcommand{\todaia}[1][]{\refstepcounter{todaisub}#1 {\bf [問題 \thetodai.\thetodaisub]}}


     \begin{oframed}
     平面上で,曲線$x+y^2-5=0$を,$x$軸に平行なある直線$l_1$に関して折り返し,さらに別の直線
     $l_2$に関して折り返せば,曲線$x^2-y+1=0$に重なるという.直線$l_1$,$l_2$の方程式を求め
     よ.
     \end{oframed}

\setlength{\columnseprule}{0.4pt}
\begin{multicols}{2}
{\bf[解]}題意より$l_1:y=k$とおける.$C_2:x^2-y+1=0$とする.$C:x+y^2-5=0$を$l_1$に関して折り返したグラフを$C_1$とすれば$C$上の点$(x,y)$は$C_1$上の点$(x,2k-y)$に移るので
     \begin{align*}
     C_1:x+(y-2k)^2-5=0
     \end{align*}
である.次に$l_2$について,$C_1$の軸は$y=2k$,$C_2$の軸は$x=0$だから,$l_2$はこれらの
垂直$2$等分線で,傾き$\pm1$で,$(x,y)=(0,2k)$を通るので$l_2:y=\pm x+2k$である.又,
頂点は頂点に移るので$A(0,1)$が$B(5,2k)$に移る.故に以下複合同順として
     \begin{align*}
     &\left\{
          \begin{array}{l}
          \text{線分$AB$の中点$M$が$l_2$上} \\
          \vector{PQ}\perp l_2
          \end{array}
     \right. \\
     \Longleftrightarrow
     &\left\{
          \begin{array}{l}
          \dfrac{2k+1}{2}=\pm\dfrac{5}{2}+2k \\
          \vtwo{5}{2k-1}\perp \vtwo{1}{\pm1}
          \end{array}
     \right. \\    
     \Longleftrightarrow &\text{(複合負)} ,k=3
     \end{align*}  


以下十分条件を求める.

$C_1$上の点$P(x,y)$が$C_2$上の点$Q(x+2X,y+2Y)$に移るとすると,$(X,Y)$は
     \begin{align*}
     &\left\{
          \begin{array}{l}
          \text{線分$PQ$の中点$M$が$l_2$上} \\
          \vector{PQ}\perp l_2
          \end{array}
     \right. \\
     \Longleftrightarrow
     &\left\{
          \begin{array}{l}
          (y+Y)=-(x+X)+6 \\
          \vtwo{X}{Y}\perp \vtwo{1}{-1}
          \end{array}
     \right.     \\
     \Longleftrightarrow
     &2X=2Y=-x-y+6
     \end{align*}
である.この時$Q$が$C_2$上にある条件から     
     \begin{align*}
     (x+2X)^2-(y+2Y)+1=0  \\
     (6-y)^2-(6-x)+1=0  \\
     x+(y-6)^2-5=0
     \end{align*}
 これは$Q$が$l_2$の折り返しにより$P$に移動することを示し,十分である.以上から
      \begin{align*}
     \left\{
          \begin{array}{l}
       l_1:y=6\\
          l_2:y=-x+6
          \end{array}
     \right.
     \end{align*}
である.$\cdots$(答)           
\newpage
\end{multicols}
\end{document}