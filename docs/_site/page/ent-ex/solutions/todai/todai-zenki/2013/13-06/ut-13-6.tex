\documentclass[a4j]{jarticle}
\usepackage{amsmath}
\usepackage{ascmac}
\usepackage{amssymb}
\usepackage{enumerate}
\usepackage{multicol}
\usepackage{framed}

\title{}
\begin{document}
\maketitle

\begin{oframed}
 $xyz$空間において$A(-1,1,0)$,$B(1,1,0)$,$C(1,-1,0)$,$D(-1,-1,0)$を頂点とする正方形$S$とする.正方形$S$を,直線$BD$を軸として回転させて出来る立体を$V_1$,直線$AC$を軸として回転させて出来る立体を$V_2$とする.\\
 (1)$0\le t<1$を満たす実数$t$に対し,平面$x=t$による$V_1$の切り口を表す式を求め図示せよ.\\
 (2) $V_1$,$V_2$の共通部分の体積を求めよ.
\end{oframed}

\setlength{\columnseprule}{0.4pt}
\begin{multicols}{2}
     \begin{enumerate}[(1)]
     \item 座標軸を,$z$軸を回転軸として$\frac{\pi}{4}$回転して得られる新しい座標系を$XYz$空間とする.この変換によって, 
     $A(-1,1,0)$,$B(1,1,0)$,$C(1,-1,0)$,$D(-1,-1,0)$となる.以下この座標系で考える.$V_1$を$X=p(0\le p<\sqrt2)$で切断した断面は
     
     \end{enumerate}

\end{multicols}
\end{document}