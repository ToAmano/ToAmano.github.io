\documentclass[a4j]{jarticle}
\usepackage{amsmath}
\usepackage{ascmac}
\usepackage{amssymb}
\usepackage{enumerate}
\usepackage{multicol}
\usepackage{framed}
\usepackage{fancyhdr}
\usepackage{latexsym}
\usepackage{indent}
\usepackage{cases}
\allowdisplaybreaks
\pagestyle{fancy}
\lhead{}
\chead{}
\rhead{東京大学前期$1974$年$1$番}
\begin{document}
%分数関係


\def\tfrac#1#2{{\textstyle\frac{#1}{#2}}} %数式中で文中表示の分数を使う時


%Σ関係

\def\dsum#1#2{{\displaystyle\sum_{#1}^{#2}}} %文中で数式表示のΣを使う時


%ベクトル関係


\def\vector#1{\overrightarrow{#1}}  %ベクトルを表現したいとき(aベクトルを表現するときは\ver
\def\norm#1{|\overrightarrow{#1}|} %ベクトルの絶対値
\def\vtwo#1#2{ \left(%
      \begin{array}{c}%
      #1 \\%
      #2 \\%
      \end{array}%
      \right) }                        %2次元ベクトル成分表示
      
      \def\vthree#1#2#3{ \left(
      \begin{array}{c}
      #1 \\
      #2 \\
      #3 \\
      \end{array}
      \right) }                        %3次元ベクトル成分表示



%数列関係


\def\an#1{\verb|{|$#1$\verb|}|}


%極限関係

\def\limit#1#2{\stackrel{#1 \to #2}{\longrightarrow}}   %等式変形からの極限
\def\dlim#1#2{{\displaystyle \lim_{#1\to#2}}} %文中で数式表示の極限を使う



%積分関係

\def\dint#1#2{{\displaystyle \int_{#1}^{#2}}} %文中で数式表示の積分を使う時

\def\ne{\nearrow}
\def\se{\searrow}
\def\nw{\nwarrow}
\def\ne{\nearrow}


%便利なやつ

\def\case#1#2{%
 \[\left\{%
 \begin{array}{l}%
 #1 \\%
 #2%
 \end{array}%
 \right.\] }                           %場合分け
 
\def\1{$\cos\theta=c$,$\sin\theta=s$とおく.}  %cs表示を与える前書きシータ
\def\2{$\cos t=c$,$\sin t=s$とおく.}     %cs表示を与える前書きt
\def\3{$\cos x=c$,$\sin x=s$とおく.}                %cs表示を与える前書きx

\def\fig#1#2#3 {%
\begin{wrapfigure}[#1]{r}{#2 zw}%
\vspace*{-1zh}%
\input{#3}%
\end{wrapfigure} }           %絵の挿入


\def\a{\alpha}   %ギリシャ文字
\def\b{\beta}
\def\g{\gamma}

%問題番号のためのマクロ

\newcounter{nombre} %必須
\renewcommand{\thenombre}{\arabic{nombre}} %任意
\setcounter{nombre}{2} %任意
\newcounter{nombresub}[nombre] %親子関係を定義
\renewcommand{\thenombresub}{\arabic{nombresub}} %任意
\setcounter{nombresub}{0} %任意
\newcommand{\prob}[1][]{\refstepcounter{nombre}#1[問題 \thenombre]}
\newcommand{\probsub}[1][]{\refstepcounter{nombresub}#1(\thenombresub)}


%1-1みたいなカウンタ(todaiとtodaia)
\newcounter{todai}
\setcounter{todai}{0}
\newcounter{todaisub}[todai] 
\setcounter{todaisub}{0} 
\newcommand{\todai}[1][]{\refstepcounter{todai}#1 \thetodai-\thetodaisub}
\newcommand{\todaib}[1][]{\refstepcounter{todai}#1\refstepcounter{todaisub}#1 {\bf [問題 \thetodai.\thetodaisub]}}
\newcommand{\todaia}[1][]{\refstepcounter{todaisub}#1 {\bf [問題 \thetodai.\thetodaisub]}}


     \begin{oframed}
     自然数$n$,$p$に対し,$n^p$を十進法で書いた時の$1$の位の数を$f_p(n)$で表す.ただし,
     自然数とは,$1,2,3,\cdots$のことである.
          \begin{enumerate}[(1)]
          \item $n$が自然数の全体を動く時,$f_2(n)$のとる値を全部求めよ.
          \item あらゆる自然数$n$に対して,$f_5(n)=f_1(n)$が成り立つことを証明せよ.
          \item $n$が自然数の全体を動く時,$f_{100}(n)$のとる値を全部求めよ.
          \end{enumerate}
     \end{oframed}

\setlength{\columnseprule}{0.4pt}
\begin{multicols}{2}
{\bf[解]} $n=0$に対しても同様に$f_p(n)$を定義できるので,以下$n\ge0$で考える.
合同式の法を$10$とすると,
     \begin{align*}
     n^p\equiv f_p(n)
     \end{align*}
である.故に$k=0,1,2,\cdots,9$および$i\in\mathbb{Z}_{\ge0}$に対して
     \begin{align*}
     f_p(10i+k)=f_p(k)
     \end{align*}
が成り立つ.よって以下$f_p(0),\cdots,f_p(9)$についてのみ考えればよい.
     \begin{enumerate}[(1)]
     \item $(10-k)^2\equiv k^2$ ゆえ,$f_2(k)=f_2(10-k)$である.故に$k=0,1,2,3,4,5$についてのみ調べ
     ればよい.
          \begin{align*}
          &f_2(0)=0 &f_2(1)=1 \\
          &f_2(2)=4 &f_2(3)=9  \\
          &f_2(4)=6 & f_2(5)=5
          \end{align*}
     だから,求める値は
           \[f_2(n)=0,1,4,5,6,9\]
     である.$\cdots$(答)
     
     \item $n^5-n=n(n-1)(n+1)(n^2+1)$は$2$連続整数の積を因数に含むから$2$の倍数である.
     
     又,ここでのみ合同式の法を$5$とすると,$\forall n,n^5-n\equiv0$
     (実際に$n\equiv0,\pm1,\pm2,3$を代入すれば明らか)である.故に$5$の倍数である.
     
     以上から,$n^5-n$は$2$かつ$5$の倍数,つまり$10$の倍数.よって 
          \[n^5-n\equiv 0\Longleftrightarrow f_5(n)=f_1(n)\]
     である.$\Box$
     \item (2)の結果から
          \begin{align*}
          n^{100}\equiv (n^4)^{25}\equiv n^4
          \end{align*}
     であるから,$f_4{n}$の取り得る値だけを求めればよい.(1)と同様にして,$k=0,1,2,3,4,5$のみ調
     べれば十分である.
          \begin{align*}
          &f_4(0)=0 &f_4(1)=1 \\
          &f_4(2)=6 &f_4(3)=1  \\
          &f_4(4)=6 & f_4(5)=5
          \end{align*}
     であるから,求める値は
          \[f_4(n)=0,1,5,6\]
     である.$\cdots$(答)
     \end{enumerate}
     
\newpage
\end{multicols}
\end{document}