\documentclass[a4j]{jarticle}
\usepackage{amsmath}
\usepackage{ascmac}
\usepackage{amssymb}
\usepackage{enumerate}
\usepackage{multicol}
\usepackage{framed}
\usepackage{latexsym}
\usepackage{indent}
\title{}
\begin{document}
%分数関係


\def\tfrac#1#2{{\textstyle\frac{#1}{#2}}} %数式中で文中表示の分数を使う時


%Σ関係

\def\dsum#1#2{{\displaystyle\sum_{#1}^{#2}}} %文中で数式表示のΣを使う時


%ベクトル関係


\def\vector#1{\overrightarrow{#1}}  %ベクトルを表現したいとき(aベクトルを表現するときは\ver
\def\norm#1{|\overrightarrow{#1}|} %ベクトルの絶対値
\def\vtwo#1#2{ \left(%
      \begin{array}{c}%
      #1 \\%
      #2 \\%
      \end{array}%
      \right) }                        %2次元ベクトル成分表示
      
      \def\vthree#1#2#3{ \left(
      \begin{array}{c}
      #1 \\
      #2 \\
      #3 \\
      \end{array}
      \right) }                        %3次元ベクトル成分表示



%数列関係


\def\an#1{\verb|{|$#1$\verb|}|}


%極限関係

\def\limit#1#2{\stackrel{#1 \to #2}{\longrightarrow}}   %等式変形からの極限
\def\dlim#1#2{{\displaystyle \lim_{#1\to#2}}} %文中で数式表示の極限を使う



%積分関係

\def\dint#1#2{{\displaystyle \int_{#1}^{#2}}} %文中で数式表示の積分を使う時

\def\ne{\nearrow}
\def\se{\searrow}
\def\nw{\nwarrow}
\def\ne{\nearrow}


%便利なやつ

\def\case#1#2{%
 \[\left\{%
 \begin{array}{l}%
 #1 \\%
 #2%
 \end{array}%
 \right.\] }                           %場合分け
 
\def\1{$\cos\theta=c$,$\sin\theta=s$とおく.}  %cs表示を与える前書きシータ
\def\2{$\cos t=c$,$\sin t=s$とおく.}     %cs表示を与える前書きt
\def\3{$\cos x=c$,$\sin x=s$とおく.}                %cs表示を与える前書きx

\def\fig#1#2#3 {%
\begin{wrapfigure}[#1]{r}{#2 zw}%
\vspace*{-1zh}%
\input{#3}%
\end{wrapfigure} }           %絵の挿入


\def\a{\alpha}   %ギリシャ文字
\def\b{\beta}
\def\g{\gamma}

%問題番号のためのマクロ

\newcounter{nombre} %必須
\renewcommand{\thenombre}{\arabic{nombre}} %任意
\setcounter{nombre}{2} %任意
\newcounter{nombresub}[nombre] %親子関係を定義
\renewcommand{\thenombresub}{\arabic{nombresub}} %任意
\setcounter{nombresub}{0} %任意
\newcommand{\prob}[1][]{\refstepcounter{nombre}#1[問題 \thenombre]}
\newcommand{\probsub}[1][]{\refstepcounter{nombresub}#1(\thenombresub)}


%1-1みたいなカウンタ(todaiとtodaia)
\newcounter{todai}
\setcounter{todai}{0}
\newcounter{todaisub}[todai] 
\setcounter{todaisub}{0} 
\newcommand{\todai}[1][]{\refstepcounter{todai}#1 \thetodai-\thetodaisub}
\newcommand{\todaib}[1][]{\refstepcounter{todai}#1\refstepcounter{todaisub}#1 {\bf [問題 \thetodai.\thetodaisub]}}
\newcommand{\todaia}[1][]{\refstepcounter{todaisub}#1 {\bf [問題 \thetodai.\thetodaisub]}}


\begin{oframed}
     \begin{align*}
     f(x)=x^4+x^3+\frac{1}{2}x^2+\frac{1}{6}x+\frac{1}{24} , 
     g(x)=x^5+x^4+\frac{1}{2}x^3+\frac{1}{6}x^2+\frac{1}{24}x+\frac{1}{120}
     \end{align*}
とする.このとき以下のことが成り立つことを示せ.
     \begin{enumerate}[(1)]
     \item 任意の実数$x$に対し,$f(x)>0$である.
     \item 方程式$g(x)=0$はただひとつの実数解$\alpha$をもち,$-1<\alpha<0$となる.
     \end{enumerate}     
\end{oframed}

\setlength{\columnseprule}{0.4pt}
\begin{multicols}{2}
{\bf[解]}
     \begin{enumerate}[(1)]
     \item 微分計算により
          \begin{align*}
          f'(x)&=4x^3+3x^2+x+\frac{1}{6} \\
          f''(x)&=12(x+\frac{1}{4})^2+\frac{1}{4}>0 
          \end{align*}
     であるから,$f'(x)$は単調増加.故に$f'(x)$が$3$次式であることと併せて,$f'(t)=0$
     なる$t$がただひとつ存在する.したがって
           \begin{align*}
           f(t)&=(\frac{1}{4}x+\frac{1}{16})f'(t)+\frac{1}{32}(2t^2+2t+1)  \\
           &=\frac{1}{32}(2t^2+2t+1) \\
           &=\frac{1}{32}\left\{2\left(t+\frac{1}{2}\right)^2+\frac{1}{2}\right\} >0
           \end{align*}
      となって,下表をうる.
           \begin{align*}
                \begin{array}{|c|c|c|c|} \hline
                x  &      &    t            &        \\ \hline
                f'   &  -  &   0            &  +   \\ \hline
                f    &\se&   \text{正} &  \ne \\ \hline
               \end{array}
          \end{align*}       
      ゆえに題意は示された.$\Box$      
      
      \item 微分計算から
           \begin{align*}
           g'(x)&=5x^4+4x^3+\frac{3}{2}x^2+\frac{1}{3}x+\frac{1}{24} \\
           g''(x)&=20x^3+12x^2+3x+\frac{1}{3}  \\
           g'''(x)&=60\left(x+\frac{1}{5}\right)^2+\frac{3}{5}>0
           \end{align*}
      だから,前問と同様に$g''(t)=0$なる実数$t$がただひとつ存在し,下表を得る.    
           \begin{align*}
                \begin{array}{|c|c|c|c|} \hline
                x     &      &    t       &        \\ \hline
                g''    &  -  &   0       &  +   \\ \hline
                g'    &\se&             &  \ne \\ \hline
               \end{array}
          \end{align*}
      したがって
           \begin{align*}
           g'(x)&\ge g'(t) \\
           &=\left(\frac{1}{4}x+\frac{1}{20}\right)g''(t)+%
           40\left\{6\left(t+\frac{1}{3}\right)^2+\frac{1}{3}\right\} \\
           &=40\left\{6\left(t+\frac{1}{3}\right)^2+\frac{1}{3}\right\}>0
           \end{align*}  
      から$g'(x)>0$であるので$g(x)$は単調増加で,$g(x)$が$5$次であること,
      および$g(0)>0$,$g(-1)=\dfrac{-11}{30}$から題意は示された.$\Box$     
      \end{enumerate}
{\bf[別解]}
     \begin{enumerate}[(1)]
     \item 与式を変形する.
          \begin{align*}
          f(x)=\left(x^2+\frac{1}{2}x\right)^2+\frac{1}{4}\left(x+\frac{1}{3}\right)^2+\frac{1}{72}
          \end{align*}
     で,確かに$f(x)>0$である.よって示された.$\Box$
     
     \item 前問と同様に
          \begin{align*}
          g'(x)&=5x^4+4x^3+\frac{3}{2}x^2+\frac{1}{3}x+\frac{1}{24}     \\
          &=5(x^2+\frac{2}{5}x)^2+\frac{7}{10}(x+\frac{5}{21})^2+\frac{3}{1512} 
          \end{align*}
     ゆえ$g'(x)>0$である.これと$g(-1)<0$および$g(0)>0$から示すべきことは明らか.$\Box$     
     \end{enumerate}      
\newpage
\end{multicols}
\end{document}