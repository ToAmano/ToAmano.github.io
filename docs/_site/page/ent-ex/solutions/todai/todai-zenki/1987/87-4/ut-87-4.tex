\documentclass[a4j]{jarticle}
\usepackage{amsmath}
\usepackage{ascmac}
\usepackage{amssymb}
\usepackage{enumerate}
\usepackage{multicol}
\usepackage{framed}
\usepackage{latexsym}
\usepackage{indent}
\title{}
\begin{document}
%分数関係


\def\tfrac#1#2{{\textstyle\frac{#1}{#2}}} %数式中で文中表示の分数を使う時


%Σ関係

\def\dsum#1#2{{\displaystyle\sum_{#1}^{#2}}} %文中で数式表示のΣを使う時


%ベクトル関係


\def\vector#1{\overrightarrow{#1}}  %ベクトルを表現したいとき(aベクトルを表現するときは\ver
\def\norm#1{|\overrightarrow{#1}|} %ベクトルの絶対値
\def\vtwo#1#2{ \left(%
      \begin{array}{c}%
      #1 \\%
      #2 \\%
      \end{array}%
      \right) }                        %2次元ベクトル成分表示
      
      \def\vthree#1#2#3{ \left(
      \begin{array}{c}
      #1 \\
      #2 \\
      #3 \\
      \end{array}
      \right) }                        %3次元ベクトル成分表示



%数列関係


\def\an#1{\verb|{|$#1$\verb|}|}


%極限関係

\def\limit#1#2{\stackrel{#1 \to #2}{\longrightarrow}}   %等式変形からの極限
\def\dlim#1#2{{\displaystyle \lim_{#1\to#2}}} %文中で数式表示の極限を使う



%積分関係

\def\dint#1#2{{\displaystyle \int_{#1}^{#2}}} %文中で数式表示の積分を使う時

\def\ne{\nearrow}
\def\se{\searrow}
\def\nw{\nwarrow}
\def\ne{\nearrow}


%便利なやつ

\def\case#1#2{%
 \[\left\{%
 \begin{array}{l}%
 #1 \\%
 #2%
 \end{array}%
 \right.\] }                           %場合分け
 
\def\1{$\cos\theta=c$,$\sin\theta=s$とおく.}  %cs表示を与える前書きシータ
\def\2{$\cos t=c$,$\sin t=s$とおく.}     %cs表示を与える前書きt
\def\3{$\cos x=c$,$\sin x=s$とおく.}                %cs表示を与える前書きx

\def\fig#1#2#3 {%
\begin{wrapfigure}[#1]{r}{#2 zw}%
\vspace*{-1zh}%
\input{#3}%
\end{wrapfigure} }           %絵の挿入


\def\a{\alpha}   %ギリシャ文字
\def\b{\beta}
\def\g{\gamma}

%問題番号のためのマクロ

\newcounter{nombre} %必須
\renewcommand{\thenombre}{\arabic{nombre}} %任意
\setcounter{nombre}{2} %任意
\newcounter{nombresub}[nombre] %親子関係を定義
\renewcommand{\thenombresub}{\arabic{nombresub}} %任意
\setcounter{nombresub}{0} %任意
\newcommand{\prob}[1][]{\refstepcounter{nombre}#1[問題 \thenombre]}
\newcommand{\probsub}[1][]{\refstepcounter{nombresub}#1(\thenombresub)}


%1-1みたいなカウンタ(todaiとtodaia)
\newcounter{todai}
\setcounter{todai}{0}
\newcounter{todaisub}[todai] 
\setcounter{todaisub}{0} 
\newcommand{\todai}[1][]{\refstepcounter{todai}#1 \thetodai-\thetodaisub}
\newcommand{\todaib}[1][]{\refstepcounter{todai}#1\refstepcounter{todaisub}#1 {\bf [問題 \thetodai.\thetodaisub]}}
\newcommand{\todaia}[1][]{\refstepcounter{todaisub}#1 {\bf [問題 \thetodai.\thetodaisub]}}


\begin{oframed}
$xyz$空間において,点$P$は$yz$平面上の放物線$z=1-y^2$上にあるとする.点$A(1,0,1)$と
$P$を結ぶ直線を$x$軸のまわりに回転して得られる曲線と二平面$x=0$,$x=1$とによって囲まれる
部分の体積を$V$とする.$V$を$P$の$y$座標で表せ.また$V$の最小値を求めよ.
\end{oframed}

\setlength{\columnseprule}{0.4pt}
\begin{multicols}{2}
{\bf[解]}$P(0,t,1-t^2)$とおく.また,$\vec{a}=(1,0,1)$とする.このとき,題意の直線$l$の方向ベクトル$\vec{l}$は$\vec{l}=(-1,t,-t^2)$であるから,
     \begin{align*}
     l:\vector{OX}=\vec{a}+p\vec{l}
     \end{align*}
である.$x=k(0\le k\le 1)$で切断する.この平面と$l$の交点は$p=1-k$に対応し,
$\left(k,(1-k)t,1-(1-k)t^2\right)$である.したがって,この平面での断面積$S(k)$は
     \begin{align*}
     S(k)&=\pi\left\{\left((1-k)t\right)^2+\left(1-(1-k)t^2\right)^2\right\} \\
     &=\pi\left\{1-2t^2(1-k)+(t^4+t^2)(1-k)^2\right\} \\
     \end{align*}     
となる.よって
     \begin{align*}
     V&=\int_0^1S(k)dk \\
     &=\pi\left[k+t^2(1-k)^2-\frac{1}{3}(t^4+t^2)(1-k)^3\right]_0^1 \\
     &=\pi(1-t^2+\frac{1}{3}(t^4+t^2)) \\
     &=\pi\left(\frac{1}{3}t^4-\frac{2}{3}t^2+1\right)\cdots\text{(答)} \\
     &=\frac{\pi}{3}\left(t^2-1\right)^2+\frac{2\pi}{3}
     \end{align*}     
よって求める最小値は$t=\pm1$のときの
     \begin{align*}
     \min V=\dfrac{2}{3}\pi\cdots\text{(答)}
     \end{align*}
である.      
          \newpage
\end{multicols}
\end{document}