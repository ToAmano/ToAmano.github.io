\documentclass[a4j]{jarticle}
\usepackage{amsmath}
\usepackage{ascmac}
\usepackage{amssymb}
\usepackage{enumerate}
\usepackage{multicol}
\usepackage{framed}
\usepackage{fancyhdr}
\usepackage{latexsym}
\usepackage{indent}
\usepackage{cases}
\usepackage[dvips]{graphicx}
\usepackage{color}
\allowdisplaybreaks
\pagestyle{fancy}
\lhead{}
\chead{}
\rhead{東京大学前期$2005$年$6$番}
\begin{document}
%分数関係


\def\tfrac#1#2{{\textstyle\frac{#1}{#2}}} %数式中で文中表示の分数を使う時


%Σ関係

\def\dsum#1#2{{\displaystyle\sum_{#1}^{#2}}} %文中で数式表示のΣを使う時


%ベクトル関係


\def\vector#1{\overrightarrow{#1}}  %ベクトルを表現したいとき(aベクトルを表現するときは\ver
\def\norm#1{|\overrightarrow{#1}|} %ベクトルの絶対値
\def\vtwo#1#2{ \left(%
      \begin{array}{c}%
      #1 \\%
      #2 \\%
      \end{array}%
      \right) }                        %2次元ベクトル成分表示
      
      \def\vthree#1#2#3{ \left(
      \begin{array}{c}
      #1 \\
      #2 \\
      #3 \\
      \end{array}
      \right) }                        %3次元ベクトル成分表示



%数列関係


\def\an#1{\verb|{|$#1$\verb|}|}


%極限関係

\def\limit#1#2{\stackrel{#1 \to #2}{\longrightarrow}}   %等式変形からの極限
\def\dlim#1#2{{\displaystyle \lim_{#1\to#2}}} %文中で数式表示の極限を使う



%積分関係

\def\dint#1#2{{\displaystyle \int_{#1}^{#2}}} %文中で数式表示の積分を使う時

\def\ne{\nearrow}
\def\se{\searrow}
\def\nw{\nwarrow}
\def\ne{\nearrow}


%便利なやつ

\def\case#1#2{%
 \[\left\{%
 \begin{array}{l}%
 #1 \\%
 #2%
 \end{array}%
 \right.\] }                           %場合分け
 
\def\1{$\cos\theta=c$,$\sin\theta=s$とおく.}  %cs表示を与える前書きシータ
\def\2{$\cos t=c$,$\sin t=s$とおく.}     %cs表示を与える前書きt
\def\3{$\cos x=c$,$\sin x=s$とおく.}                %cs表示を与える前書きx

\def\fig#1#2#3 {%
\begin{wrapfigure}[#1]{r}{#2 zw}%
\vspace*{-1zh}%
\input{#3}%
\end{wrapfigure} }           %絵の挿入


\def\a{\alpha}   %ギリシャ文字
\def\b{\beta}
\def\g{\gamma}

%問題番号のためのマクロ

\newcounter{nombre} %必須
\renewcommand{\thenombre}{\arabic{nombre}} %任意
\setcounter{nombre}{2} %任意
\newcounter{nombresub}[nombre] %親子関係を定義
\renewcommand{\thenombresub}{\arabic{nombresub}} %任意
\setcounter{nombresub}{0} %任意
\newcommand{\prob}[1][]{\refstepcounter{nombre}#1[問題 \thenombre]}
\newcommand{\probsub}[1][]{\refstepcounter{nombresub}#1(\thenombresub)}


%1-1みたいなカウンタ(todaiとtodaia)
\newcounter{todai}
\setcounter{todai}{0}
\newcounter{todaisub}[todai] 
\setcounter{todaisub}{0} 
\newcommand{\todai}[1][]{\refstepcounter{todai}#1 \thetodai-\thetodaisub}
\newcommand{\todaib}[1][]{\refstepcounter{todai}#1\refstepcounter{todaisub}#1 {\bf [問題 \thetodai.\thetodaisub]}}
\newcommand{\todaia}[1][]{\refstepcounter{todaisub}#1 {\bf [問題 \thetodai.\thetodaisub]}}


     \begin{oframed}
     $r$を正の実数とする.$xyz$空間において,
          \begin{align*}
          &x^2+y^2\le r^2 \\
          &y^2+z^2\ge r^2 \\
          &z^2+x^2\le r^2 
          \end{align*}
     を満たす点全体からなる立体の体積を求めよ.
     \end{oframed}

\setlength{\columnseprule}{0.4pt}
\begin{multicols}{2}
{\bf[解]} 各軸を$1/r$倍して考える.題意の体積を$V$とする.
          \begin{align*}
               \begin{cases}
               &x^2+y^2\le1 \\
               &y^2+z^2\ge1 \\
               &z^2+x^2\le1 
               \end{cases}
          \end{align*}
を満たす立体の体積を$V'$とすると,
     \begin{align}
     V=r^3V'\label{1}
     \end{align}
である.対称性から,さらに$x\ge0$,$0\le y\le z$を満たす部分の体積$v$とすると,     
     \begin{align}
     V'=16v\label{2}
     \end{align}
である.

\1 ただし,$0\le\theta\le\pi/4$である.$x=s$での切断面は,
          \begin{align*}
               \begin{cases}
               &y^2\le1-s^2=c^2 \\ 
               &y^2+z^2\ge1 \\
               &z^2\le1-s^2=c^2 
               \end{cases}
          \end{align*}     
であって,下図である.
     \begin{center}
     \scalebox{1}{%WinTpicVersion4.32a
{\unitlength 0.1in%
\begin{picture}(22.1000,18.0000)(3.9000,-22.0000)%
% STR 2 0 3 0 Black White  
% 4 590 1997 590 2010 4 400 0 0
% O
\put(5.9000,-20.1000){\makebox(0,0)[rt]{O}}%
% STR 2 0 3 0 Black White  
% 4 560 387 560 400 4 0 0 0
% $z$
\put(5.6000,-4.0000){\makebox(0,0)[rt]{$z$}}%
% STR 2 0 3 0 Black White  
% 4 2600 2027 2600 2040 4 0 0 0
% $y$
\put(26.0000,-20.4000){\makebox(0,0)[rt]{$y$}}%
% VECTOR 2 0 3 0 Black White  
% 2 600 2200 600 400
% 
\special{pn 8}%
\special{pa 600 2200}%
\special{pa 600 400}%
\special{fp}%
\special{sh 1}%
\special{pa 600 400}%
\special{pa 580 467}%
\special{pa 600 453}%
\special{pa 620 467}%
\special{pa 600 400}%
\special{fp}%
% VECTOR 2 0 3 0 Black White  
% 2 400 2000 2600 2000
% 
\special{pn 8}%
\special{pa 400 2000}%
\special{pa 2600 2000}%
\special{fp}%
\special{sh 1}%
\special{pa 2600 2000}%
\special{pa 2533 1980}%
\special{pa 2547 2000}%
\special{pa 2533 2020}%
\special{pa 2600 2000}%
\special{fp}%
% FUNC 2 0 3 0 Black White  
% 9 400 400 2600 2200 600 2000 2000 2000 600 600 400 400 2600 2200 50 4 0 2
% sin(t)///cos(t)///0///pi/2
\special{pn 8}%
\special{pa 600 600}%
\special{pa 635 600}%
\special{pa 640 601}%
\special{pa 662 601}%
\special{pa 666 602}%
\special{pa 684 602}%
\special{pa 688 603}%
\special{pa 697 603}%
\special{pa 701 604}%
\special{pa 710 604}%
\special{pa 714 605}%
\special{pa 723 605}%
\special{pa 727 606}%
\special{pa 732 606}%
\special{pa 736 607}%
\special{pa 741 607}%
\special{pa 745 608}%
\special{pa 754 608}%
\special{pa 758 609}%
\special{pa 762 609}%
\special{pa 767 610}%
\special{pa 771 611}%
\special{pa 776 611}%
\special{pa 780 612}%
\special{pa 784 612}%
\special{pa 789 613}%
\special{pa 793 613}%
\special{pa 797 614}%
\special{pa 802 615}%
\special{pa 806 615}%
\special{pa 810 616}%
\special{pa 815 617}%
\special{pa 819 617}%
\special{pa 823 618}%
\special{pa 828 619}%
\special{pa 832 619}%
\special{pa 836 620}%
\special{pa 841 621}%
\special{pa 845 622}%
\special{pa 849 622}%
\special{pa 854 623}%
\special{pa 862 625}%
\special{pa 867 626}%
\special{pa 871 626}%
\special{pa 875 627}%
\special{pa 880 628}%
\special{pa 888 630}%
\special{pa 893 631}%
\special{pa 905 634}%
\special{pa 910 635}%
\special{pa 918 637}%
\special{pa 923 638}%
\special{pa 935 641}%
\special{pa 940 642}%
\special{pa 952 645}%
\special{pa 957 646}%
\special{pa 965 648}%
\special{pa 969 650}%
\special{pa 974 651}%
\special{pa 986 654}%
\special{pa 991 656}%
\special{pa 1003 659}%
\special{pa 1007 661}%
\special{pa 1012 662}%
\special{pa 1016 663}%
\special{pa 1020 665}%
\special{pa 1028 667}%
\special{pa 1033 669}%
\special{pa 1041 671}%
\special{pa 1045 673}%
\special{pa 1049 674}%
\special{pa 1054 675}%
\special{pa 1058 677}%
\special{pa 1062 678}%
\special{pa 1066 680}%
\special{pa 1070 681}%
\special{pa 1074 683}%
\special{pa 1078 684}%
\special{pa 1083 686}%
\special{pa 1087 687}%
\special{pa 1091 689}%
\special{pa 1095 690}%
\special{pa 1103 694}%
\special{pa 1107 695}%
\special{pa 1111 697}%
\special{pa 1115 698}%
\special{pa 1120 700}%
\special{pa 1124 702}%
\special{pa 1128 703}%
\special{pa 1136 707}%
\special{pa 1140 708}%
\special{pa 1148 712}%
\special{pa 1152 713}%
\special{pa 1164 719}%
\special{pa 1168 720}%
\special{pa 1188 730}%
\special{pa 1192 731}%
\special{pa 1240 755}%
\special{pa 1243 757}%
\special{pa 1271 771}%
\special{pa 1274 773}%
\special{pa 1282 777}%
\special{pa 1286 780}%
\special{pa 1294 784}%
\special{pa 1297 786}%
\special{pa 1301 788}%
\special{pa 1305 791}%
\special{pa 1313 795}%
\special{pa 1316 797}%
\special{pa 1320 799}%
\special{pa 1324 802}%
\special{pa 1332 806}%
\special{pa 1335 809}%
\special{pa 1343 813}%
\special{pa 1346 816}%
\special{pa 1354 820}%
\special{pa 1358 823}%
\special{pa 1361 825}%
\special{pa 1365 827}%
\special{pa 1369 830}%
\special{pa 1372 832}%
\special{pa 1376 835}%
\special{pa 1380 837}%
\special{pa 1383 840}%
\special{pa 1387 842}%
\special{pa 1391 845}%
\special{pa 1394 847}%
\special{pa 1398 850}%
\special{pa 1401 852}%
\special{pa 1405 855}%
\special{pa 1409 857}%
\special{pa 1412 860}%
\special{pa 1416 862}%
\special{pa 1419 865}%
\special{pa 1423 867}%
\special{pa 1426 870}%
\special{pa 1430 873}%
\special{pa 1434 875}%
\special{pa 1437 878}%
\special{pa 1441 880}%
\special{pa 1444 883}%
\special{pa 1448 886}%
\special{pa 1451 888}%
\special{pa 1455 891}%
\special{pa 1458 894}%
\special{pa 1462 897}%
\special{pa 1465 899}%
\special{pa 1468 902}%
\special{pa 1472 905}%
\special{pa 1475 907}%
\special{pa 1479 910}%
\special{pa 1482 913}%
\special{pa 1486 916}%
\special{pa 1489 919}%
\special{pa 1492 921}%
\special{pa 1496 924}%
\special{pa 1499 927}%
\special{pa 1503 930}%
\special{pa 1506 933}%
\special{pa 1509 935}%
\special{pa 1513 938}%
\special{pa 1519 944}%
\special{pa 1523 947}%
\special{pa 1532 956}%
\special{pa 1536 959}%
\special{pa 1545 968}%
\special{pa 1549 970}%
\special{pa 1558 979}%
\special{pa 1562 982}%
\special{pa 1565 986}%
\special{pa 1577 998}%
\special{pa 1581 1001}%
\special{pa 1599 1019}%
\special{pa 1602 1023}%
\special{pa 1608 1029}%
\special{pa 1612 1032}%
\special{pa 1618 1038}%
\special{pa 1621 1042}%
\special{pa 1630 1051}%
\special{pa 1633 1055}%
\special{pa 1636 1058}%
\special{pa 1638 1061}%
\special{pa 1641 1064}%
\special{pa 1644 1068}%
\special{pa 1650 1074}%
\special{pa 1653 1078}%
\special{pa 1662 1087}%
\special{pa 1665 1091}%
\special{pa 1667 1094}%
\special{pa 1670 1098}%
\special{pa 1676 1104}%
\special{pa 1679 1108}%
\special{pa 1682 1111}%
\special{pa 1684 1114}%
\special{pa 1687 1118}%
\special{pa 1690 1121}%
\special{pa 1693 1125}%
\special{pa 1695 1128}%
\special{pa 1698 1132}%
\special{pa 1704 1138}%
\special{pa 1706 1142}%
\special{pa 1709 1145}%
\special{pa 1712 1149}%
\special{pa 1714 1152}%
\special{pa 1717 1156}%
\special{pa 1720 1159}%
\special{pa 1722 1163}%
\special{pa 1725 1166}%
\special{pa 1727 1170}%
\special{pa 1730 1174}%
\special{pa 1733 1177}%
\special{pa 1735 1181}%
\special{pa 1738 1184}%
\special{pa 1740 1188}%
\special{pa 1743 1191}%
\special{pa 1745 1195}%
\special{pa 1748 1199}%
\special{pa 1750 1202}%
\special{pa 1753 1206}%
\special{pa 1755 1209}%
\special{pa 1758 1213}%
\special{pa 1760 1217}%
\special{pa 1763 1220}%
\special{pa 1765 1224}%
\special{pa 1768 1228}%
\special{pa 1770 1231}%
\special{pa 1773 1235}%
\special{pa 1775 1239}%
\special{pa 1777 1242}%
\special{pa 1780 1246}%
\special{pa 1784 1254}%
\special{pa 1787 1257}%
\special{pa 1791 1265}%
\special{pa 1794 1269}%
\special{pa 1796 1272}%
\special{pa 1798 1276}%
\special{pa 1801 1280}%
\special{pa 1803 1284}%
\special{pa 1805 1287}%
\special{pa 1807 1291}%
\special{pa 1810 1295}%
\special{pa 1814 1303}%
\special{pa 1816 1306}%
\special{pa 1820 1314}%
\special{pa 1823 1318}%
\special{pa 1827 1326}%
\special{pa 1829 1329}%
\special{pa 1845 1361}%
\special{pa 1847 1364}%
\special{pa 1869 1408}%
\special{pa 1870 1412}%
\special{pa 1880 1432}%
\special{pa 1881 1436}%
\special{pa 1887 1448}%
\special{pa 1888 1452}%
\special{pa 1892 1460}%
\special{pa 1893 1464}%
\special{pa 1897 1472}%
\special{pa 1898 1476}%
\special{pa 1900 1481}%
\special{pa 1902 1485}%
\special{pa 1903 1489}%
\special{pa 1907 1497}%
\special{pa 1908 1501}%
\special{pa 1910 1505}%
\special{pa 1911 1509}%
\special{pa 1913 1513}%
\special{pa 1914 1518}%
\special{pa 1916 1522}%
\special{pa 1917 1526}%
\special{pa 1919 1530}%
\special{pa 1920 1534}%
\special{pa 1922 1538}%
\special{pa 1923 1542}%
\special{pa 1925 1547}%
\special{pa 1927 1555}%
\special{pa 1929 1559}%
\special{pa 1931 1567}%
\special{pa 1933 1572}%
\special{pa 1934 1576}%
\special{pa 1936 1580}%
\special{pa 1938 1588}%
\special{pa 1939 1593}%
\special{pa 1941 1597}%
\special{pa 1944 1609}%
\special{pa 1946 1614}%
\special{pa 1949 1626}%
\special{pa 1950 1631}%
\special{pa 1952 1635}%
\special{pa 1954 1643}%
\special{pa 1955 1648}%
\special{pa 1958 1660}%
\special{pa 1959 1665}%
\special{pa 1962 1677}%
\special{pa 1963 1682}%
\special{pa 1965 1690}%
\special{pa 1966 1695}%
\special{pa 1968 1703}%
\special{pa 1969 1708}%
\special{pa 1972 1720}%
\special{pa 1973 1725}%
\special{pa 1974 1729}%
\special{pa 1974 1733}%
\special{pa 1975 1738}%
\special{pa 1977 1746}%
\special{pa 1978 1751}%
\special{pa 1978 1755}%
\special{pa 1979 1759}%
\special{pa 1980 1764}%
\special{pa 1981 1768}%
\special{pa 1981 1772}%
\special{pa 1982 1777}%
\special{pa 1983 1781}%
\special{pa 1983 1785}%
\special{pa 1984 1790}%
\special{pa 1985 1794}%
\special{pa 1985 1798}%
\special{pa 1986 1803}%
\special{pa 1987 1807}%
\special{pa 1987 1811}%
\special{pa 1988 1816}%
\special{pa 1988 1820}%
\special{pa 1989 1825}%
\special{pa 1990 1829}%
\special{pa 1990 1833}%
\special{pa 1991 1838}%
\special{pa 1991 1842}%
\special{pa 1992 1846}%
\special{pa 1992 1855}%
\special{pa 1993 1860}%
\special{pa 1993 1864}%
\special{pa 1994 1868}%
\special{pa 1994 1873}%
\special{pa 1995 1877}%
\special{pa 1995 1886}%
\special{pa 1996 1890}%
\special{pa 1996 1899}%
\special{pa 1997 1903}%
\special{pa 1997 1912}%
\special{pa 1998 1917}%
\special{pa 1998 1934}%
\special{pa 1999 1938}%
\special{pa 1999 1960}%
\special{pa 2000 1965}%
\special{pa 2000 2000}%
\special{fp}%
% FUNC 2 0 3 0 Black White  
% 10 400 400 2600 2200 600 2000 2000 2000 600 600 400 788 2600 2000 0 4 0 1 0 0
% sqrt(3)/2
\special{pn 8}%
\special{pn 8}%
\special{pa 1812 400}%
\special{pa 1812 409}%
\special{ip}%
\special{pa 1812 448}%
\special{pa 1812 457}%
\special{ip}%
\special{pa 1812 496}%
\special{pa 1812 505}%
\special{ip}%
\special{pa 1812 544}%
\special{pa 1812 553}%
\special{ip}%
\special{pa 1812 592}%
\special{pa 1812 601}%
\special{ip}%
\special{pa 1812 641}%
\special{pa 1812 649}%
\special{ip}%
\special{pa 1812 689}%
\special{pa 1812 697}%
\special{ip}%
\special{pa 1812 737}%
\special{pa 1812 745}%
\special{ip}%
\special{pa 1812 785}%
\special{pa 1812 785}%
\special{ip}%
\special{pa 1812 785}%
\special{pa 1812 2000}%
\special{fp}%
\special{pn 8}%
\special{pa 1812 2009}%
\special{pa 1812 2050}%
\special{ip}%
\special{pa 1812 2059}%
\special{pa 1812 2100}%
\special{ip}%
\special{pa 1812 2109}%
\special{pa 1812 2150}%
\special{ip}%
\special{pa 1812 2159}%
\special{pa 1812 2200}%
\special{ip}%
% FUNC 2 0 3 0 Black White  
% 10 400 400 2600 2200 600 2000 2000 2000 600 600 600 400 1812 2200 0 4 0 0 0 0
% sqrt(3)/2
\special{pn 8}%
\special{pn 8}%
\special{pa 400 788}%
\special{pa 409 788}%
\special{ip}%
\special{pa 450 788}%
\special{pa 459 788}%
\special{ip}%
\special{pa 500 788}%
\special{pa 509 788}%
\special{ip}%
\special{pa 550 788}%
\special{pa 559 788}%
\special{ip}%
\special{ip}%
\special{pa 600 788}%
\special{pa 1810 788}%
\special{fp}%
\special{pn 8}%
\special{pa 1818 788}%
\special{pa 1856 788}%
\special{ip}%
\special{pa 1865 788}%
\special{pa 1903 788}%
\special{ip}%
\special{pa 1911 788}%
\special{pa 1949 788}%
\special{ip}%
\special{pa 1958 788}%
\special{pa 1996 788}%
\special{ip}%
\special{pa 2004 788}%
\special{pa 2042 788}%
\special{ip}%
\special{pa 2051 788}%
\special{pa 2089 788}%
\special{ip}%
\special{pa 2097 788}%
\special{pa 2135 788}%
\special{ip}%
\special{pa 2144 788}%
\special{pa 2182 788}%
\special{ip}%
\special{pa 2190 788}%
\special{pa 2228 788}%
\special{ip}%
\special{pa 2236 788}%
\special{pa 2275 788}%
\special{ip}%
\special{pa 2283 788}%
\special{pa 2321 788}%
\special{ip}%
\special{pa 2329 788}%
\special{pa 2368 788}%
\special{ip}%
\special{pa 2376 788}%
\special{pa 2414 788}%
\special{ip}%
\special{pa 2422 788}%
\special{pa 2461 788}%
\special{ip}%
\special{pa 2469 788}%
\special{pa 2507 788}%
\special{ip}%
\special{pa 2515 788}%
\special{pa 2554 788}%
\special{ip}%
\special{pa 2562 788}%
\special{pa 2600 788}%
\special{ip}%
% FUNC 2 0 3 0 Black White  
% 10 400 400 2600 2200 600 2000 2000 2000 600 600 600 400 1812 2200 0 4 0 0 0 0
% x
\special{pn 8}%
\special{pn 8}%
\special{pa 400 2200}%
\special{pa 406 2194}%
\special{ip}%
\special{pa 433 2167}%
\special{pa 439 2161}%
\special{ip}%
\special{pa 467 2133}%
\special{pa 473 2127}%
\special{ip}%
\special{pa 500 2100}%
\special{pa 506 2094}%
\special{ip}%
\special{pa 533 2067}%
\special{pa 539 2061}%
\special{ip}%
\special{pa 567 2033}%
\special{pa 573 2027}%
\special{ip}%
\special{ip}%
\special{pa 600 2000}%
\special{pa 1810 790}%
\special{fp}%
\special{pn 8}%
\special{pa 1816 784}%
\special{pa 1842 758}%
\special{ip}%
\special{pa 1848 752}%
\special{pa 1875 725}%
\special{ip}%
\special{pa 1881 719}%
\special{pa 1908 692}%
\special{ip}%
\special{pa 1913 687}%
\special{pa 1940 660}%
\special{ip}%
\special{pa 1946 654}%
\special{pa 1972 628}%
\special{ip}%
\special{pa 1978 622}%
\special{pa 2005 595}%
\special{ip}%
\special{pa 2011 589}%
\special{pa 2037 563}%
\special{ip}%
\special{pa 2043 557}%
\special{pa 2070 530}%
\special{ip}%
\special{pa 2076 524}%
\special{pa 2102 498}%
\special{ip}%
\special{pa 2108 492}%
\special{pa 2135 465}%
\special{ip}%
\special{pa 2141 459}%
\special{pa 2167 433}%
\special{ip}%
\special{pa 2173 427}%
\special{pa 2200 400}%
\special{ip}%
% LINE 2 2 3 0 Black White  
% 2 1800 1310 600 1310
% 
\special{pn 8}%
\special{pa 1800 1310}%
\special{pa 600 1310}%
\special{dt 0.045}%
% LINE 3 0 3 0 Black White  
% 16 1810 1010 1700 1120 1810 1070 1720 1160 1810 1130 1750 1190 1810 1190 1770 1230 1810 1250 1790 1270 1810 950 1670 1090 1810 890 1640 1060 1810 830 1610 1030
% 
\special{pn 4}%
\special{pa 1810 1010}%
\special{pa 1700 1120}%
\special{fp}%
\special{pa 1810 1070}%
\special{pa 1720 1160}%
\special{fp}%
\special{pa 1810 1130}%
\special{pa 1750 1190}%
\special{fp}%
\special{pa 1810 1190}%
\special{pa 1770 1230}%
\special{fp}%
\special{pa 1810 1250}%
\special{pa 1790 1270}%
\special{fp}%
\special{pa 1810 950}%
\special{pa 1670 1090}%
\special{fp}%
\special{pa 1810 890}%
\special{pa 1640 1060}%
\special{fp}%
\special{pa 1810 830}%
\special{pa 1610 1030}%
\special{fp}%
% STR 2 0 3 0 Black White  
% 4 1810 1900 1810 2000 2 0 1 0
% $c$
\put(18.1000,-20.0000){\makebox(0,0)[lb]{{\colorbox[named]{White}{$c$}}}}%
% STR 2 0 3 0 Black White  
% 4 2010 1900 2010 2000 2 0 1 0
% $1$
\put(20.1000,-20.0000){\makebox(0,0)[lb]{{\colorbox[named]{White}{$1$}}}}%
% STR 2 0 3 0 Black White  
% 4 600 680 600 780 2 0 1 0
% $c$
\put(6.0000,-7.8000){\makebox(0,0)[lb]{{\colorbox[named]{White}{$c$}}}}%
% STR 2 0 3 0 Black White  
% 4 600 1210 600 1310 2 0 1 0
% $s$
\put(6.0000,-13.1000){\makebox(0,0)[lb]{{\colorbox[named]{White}{$s$}}}}%
% LINE 2 2 3 0 Black White  
% 2 600 2000 1800 1310
% 
\special{pn 8}%
\special{pa 600 2000}%
\special{pa 1800 1310}%
\special{dt 0.045}%
% STR 2 0 3 0 Black White  
% 4 1080 1810 1080 1910 2 0 1 0
% $\theta$
\put(10.8000,-19.1000){\makebox(0,0)[lb]{{\colorbox[named]{White}{$\theta$}}}}%
% STR 2 0 3 0 Black White  
% 4 1160 1490 1160 1590 5 0 1 0
% $\pi/4-\theta$
\put(11.6000,-15.9000){\makebox(0,0){{\colorbox[named]{White}{$\pi/4-\theta$}}}}%
% CIRCLE 2 2 3 0 Black White  
% 4 600 2000 730 1925 2800 2000 1800 1310
% 
\special{pn 8}%
\special{pn 8}%
\special{pa 730 1925}%
\special{pa 736 1936}%
\special{fp}%
\special{pa 749 1988}%
\special{pa 750 2000}%
\special{fp}%
\end{picture}}%
}
     \end{center}
この平面での$v$の面積$S(\theta)$として,
     \begin{align*}
     S(\theta)=\frac{1}{2}(c-s)c-\frac{1}{2}\left(\frac{\pi}{4}-\theta\right)
     \end{align*}
であるから,
     \begin{align*}
     v&=\int_0^{\sqrt{2}/2}S(\theta)ds \\
     &=\int_0^{\pi/4}S(\theta)\frac{ds}{d\theta}d\theta \\
     &=\frac{1}{2}\int_0^{\pi/4}\left\{(c-s)c^2-c\left(\frac{\pi}{4}-\theta\right)\right\}d\theta
     \end{align*}
各項計算して,
     \begin{align*}
     &\int_0^{\pi/4}c^3d\theta=\left[s-\frac{1}{3}s^3\right]_0^{\pi/4} \\
     &=\frac{5\sqrt{2}}{12} \\
     &\int_0^{\pi/4}sc^2d\theta=\frac{-1}{3}[c^3]_0^{\pi/4} \\
     &=\frac{1}{3}\left(1-\frac{\sqrt{2}}{4}\right) \\
     &\int_0^{\pi/4}cd\theta=\frac{\sqrt{2}}{2}  \\
     &\int_0^{\pi/4}c\theta d\theta=[s\theta+c]_0^{\pi/4}=\frac{\sqrt{2}\pi}{8}+\frac{\sqrt{2}}{2}-1 
     \end{align*}
であるから,代入して,
     \begin{align*}
     v&=\frac{1}{2}\left[\frac{5\sqrt{2}}{12}-\frac{1}{3}\left(1-\frac{\sqrt{2}}{4}\right)-\frac{\sqrt{2}\pi}{8}+\frac{\sqrt{2}\pi}{8}+%
     \frac{\sqrt{2}}{2}-1 \right]    \\
     &=\frac{1}{2}\left(\sqrt{2}-\frac{4}{3}\right)
     \end{align*}
これを\eqref{1},\eqref{2}に代入して,
     \[V=8\left(\sqrt{2}-\frac{4}{3}\right)r^3\]
である.$\cdots$(答)
\newpage
\end{multicols}
\end{document}