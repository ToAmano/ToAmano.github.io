\documentclass[a4j]{jarticle}
\usepackage{amsmath}
\usepackage{ascmac}
\usepackage{amssymb}
\usepackage{enumerate}
\usepackage{multicol}
\usepackage{framed}
\usepackage{fancyhdr}
\usepackage{latexsym}
\usepackage{indent}
\usepackage[dvips]{graphicx}
\usepackage{color}
\usepackage{emath}
\usepackage{emathPp}
\usepackage{cases}
\allowdisplaybreaks
\pagestyle{fancy}
\lhead{}
\chead{}
\rhead{東京大学前期$1984$年$2$番}
\begin{document}
%分数関係


\def\tfrac#1#2{{\textstyle\frac{#1}{#2}}} %数式中で文中表示の分数を使う時


%Σ関係

\def\dsum#1#2{{\displaystyle\sum_{#1}^{#2}}} %文中で数式表示のΣを使う時


%ベクトル関係


\def\vector#1{\overrightarrow{#1}}  %ベクトルを表現したいとき(aベクトルを表現するときは\ver
\def\norm#1{|\overrightarrow{#1}|} %ベクトルの絶対値
\def\vtwo#1#2{ \left(%
      \begin{array}{c}%
      #1 \\%
      #2 \\%
      \end{array}%
      \right) }                        %2次元ベクトル成分表示
      
      \def\vthree#1#2#3{ \left(
      \begin{array}{c}
      #1 \\
      #2 \\
      #3 \\
      \end{array}
      \right) }                        %3次元ベクトル成分表示



%数列関係


\def\an#1{\verb|{|$#1$\verb|}|}


%極限関係

\def\limit#1#2{\stackrel{#1 \to #2}{\longrightarrow}}   %等式変形からの極限
\def\dlim#1#2{{\displaystyle \lim_{#1\to#2}}} %文中で数式表示の極限を使う



%積分関係

\def\dint#1#2{{\displaystyle \int_{#1}^{#2}}} %文中で数式表示の積分を使う時

\def\ne{\nearrow}
\def\se{\searrow}
\def\nw{\nwarrow}
\def\ne{\nearrow}


%便利なやつ

\def\case#1#2{%
 \[\left\{%
 \begin{array}{l}%
 #1 \\%
 #2%
 \end{array}%
 \right.\] }                           %場合分け
 
\def\1{$\cos\theta=c$,$\sin\theta=s$とおく.}  %cs表示を与える前書きシータ
\def\2{$\cos t=c$,$\sin t=s$とおく.}     %cs表示を与える前書きt
\def\3{$\cos x=c$,$\sin x=s$とおく.}                %cs表示を与える前書きx

\def\fig#1#2#3 {%
\begin{wrapfigure}[#1]{r}{#2 zw}%
\vspace*{-1zh}%
\input{#3}%
\end{wrapfigure} }           %絵の挿入


\def\a{\alpha}   %ギリシャ文字
\def\b{\beta}
\def\g{\gamma}

%問題番号のためのマクロ

\newcounter{nombre} %必須
\renewcommand{\thenombre}{\arabic{nombre}} %任意
\setcounter{nombre}{2} %任意
\newcounter{nombresub}[nombre] %親子関係を定義
\renewcommand{\thenombresub}{\arabic{nombresub}} %任意
\setcounter{nombresub}{0} %任意
\newcommand{\prob}[1][]{\refstepcounter{nombre}#1[問題 \thenombre]}
\newcommand{\probsub}[1][]{\refstepcounter{nombresub}#1(\thenombresub)}


%1-1みたいなカウンタ(todaiとtodaia)
\newcounter{todai}
\setcounter{todai}{0}
\newcounter{todaisub}[todai] 
\setcounter{todaisub}{0} 
\newcommand{\todai}[1][]{\refstepcounter{todai}#1 \thetodai-\thetodaisub}
\newcommand{\todaib}[1][]{\refstepcounter{todai}#1\refstepcounter{todaisub}#1 {\bf [問題 \thetodai.\thetodaisub]}}
\newcommand{\todaia}[1][]{\refstepcounter{todaisub}#1 {\bf [問題 \thetodai.\thetodaisub]}}


\preEqlabel{$\cdots$}
     \begin{oframed}
     $xy$平面において,直線$x=0$を$L$とし,曲線$y=\log x$を$C$とする.さらに,$L$上,または
     $C$上,または$L$と$C$に挟まれた部分にある点全体の集合を$A$とする.$A$に含まれ,直線
     $L$に接し,かつ曲線$C$と点$(t,\log t)\quad (0<t)$において共通の接線を持つ 円の中心を
     $\mathrm{P_t}$とする.
     
     $\mathrm{P_t}$の$x$座標,$y$座標を$t$の関数として$x=f(t)$,$y=g(t)$と表した時,次の極限値は
     どのような数となるか.
          \begin{enumerate}[i)]
          \item $\dlim{t}{0}\dfrac{f(t)}{g(t)}$ \\
          \item $\dlim{t}{+\infty}\dfrac{f(t)}{g(t)}$
          \end{enumerate}
     \end{oframed}

\setlength{\columnseprule}{0.4pt}
\begin{multicols}{2}
{\bf[解]} $f(t)=f$などと略記する.グラフの概形は下のようになる.

     \begin{zahyou}[ul=10mm,Migiyohaku=45pt](-0.5,4)(-2,2)
     \calcval{sqrt(2)/(1+sqrt(2))}\f
     \calcval{1/(1+sqrt(2))}\g
     \def\Fx{log (X)}
     \def\Gx{X-1}
     \def\Ft{\f+\f*cos(T)}
     \def\Gt{\g+\f*sin(T)}
     \YGurafu*[hidarix=0]\Fx
     \YGurafu*\Gx
     \BGurafu\Ft\Gt{0}{2*$pi}
     \Put{(1,0)}[se]{$t$}
     \tenretu*<perl>{P(\f,\g);Q(2,log(2))}
     \Put\P[s]{$\mathrm{P_t}$}
     \Put\migiT[se]{$l$}
     \Put\Q[s]{$C$}
     \kuromaru\P
     \end{zahyou}

$C$のQ$(t,\log t)$での接線$l$は,
     \begin{align*}
     l:y=\frac{1}{t}x+\log t-1
     \end{align*}
である.円が$L$と接することから,半径$r$として
     \begin{align}
     r=f
     \end{align}
である.故に円の方程式は
     \begin{align*}
     (x-f)^2+(y-g)^2=f^2
     \end{align*}
さらに,円がQを通る条件から
     \begin{align}
     (t-f)^2+(\log t-g)^2=f^2\label{1}
     \end{align}
Qでの接線が$l$に一致することから
     \begin{align*}
     &l \perp\mathrm{P_tQ} \\
     &\vtwo{t}{1}\cdot\vtwo{t-f}{\log t-g}=0 \\
     &t(t-f)+(\log t-g)=0\atag\label{2}
     \end{align*}
である.\eqref{2}を\eqref{1}に代入して$g$を消去して
     \begin{align*}
     &(t-f)^2+t^2(t-f)^2=f^2 \\
     &(1+t^2)(t-f)^2=f^2
     \end{align*}
ここで,円が$A$に含まれることから
     \begin{align*}
     &t-f>0&f>0
     \end{align*}
だから
     \begin{align*}
     \sqrt{1+t^2}(t-f)=f \\
     \therefore \,\, f=\frac{t\sqrt{1+t^2}}{1+\sqrt{1+t^2}}\atag\label{3}
     \end{align*}
である.\eqref{2}に代入して
     \begin{align}
     g(t)=\frac{tf}{\sqrt{1+t^2}}+\log t\label{4}
     \end{align}

さて,$h(t)=f(t)/g(t)$とおく.\eqref{4}の両辺$f(>0)$でわると
     \begin{align*}
     \frac{1}{h}&=\frac{t}{\sqrt{1+t^2}}+\frac{\log t}{f} \\
     &=\frac{t}{\sqrt{1+t^2}}+\frac{1+\sqrt{1+t^2}}{t\sqrt{1+t^2}}\log t &(\because \eqref{3}) \\
     &=\frac{t^2+(1+t^2)\log t}{t\sqrt{1+t^2}} \\
     \therefore\,\, h&=\frac{t\sqrt{1+t^2}}{t^2+(1+t^2)\log t}
     \end{align*}
だから,
     \[h\limit{t}{+0}0\]
および
     \begin{align*}
     h=\frac{1}{\frac{t}{1\sqrt{1+t^2}}+\frac{\log t}{t}\frac{1+\sqrt{1+t^2}}{\sqrt{1+t^2}}}\limit{t}{\infty}1
     \end{align*}
である.$\cdots$(答)
\newpage
\end{multicols}
\end{document}