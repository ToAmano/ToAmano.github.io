\documentclass[a4j]{jarticle}
\usepackage{amsmath}
\usepackage{ascmac}
\usepackage{amssymb}
\usepackage{enumerate}
\usepackage{multicol}
\usepackage{framed}
\usepackage{fancyhdr}
\usepackage{latexsym}
\usepackage{indent}
\usepackage{cases}
\allowdisplaybreaks
\pagestyle{fancy}
\lhead{}
\chead{}
\rhead{東京大学前期$1968$年$5$番}
\begin{document}
%分数関係


\def\tfrac#1#2{{\textstyle\frac{#1}{#2}}} %数式中で文中表示の分数を使う時


%Σ関係

\def\dsum#1#2{{\displaystyle\sum_{#1}^{#2}}} %文中で数式表示のΣを使う時


%ベクトル関係


\def\vector#1{\overrightarrow{#1}}  %ベクトルを表現したいとき(aベクトルを表現するときは\ver
\def\norm#1{|\overrightarrow{#1}|} %ベクトルの絶対値
\def\vtwo#1#2{ \left(%
      \begin{array}{c}%
      #1 \\%
      #2 \\%
      \end{array}%
      \right) }                        %2次元ベクトル成分表示
      
      \def\vthree#1#2#3{ \left(
      \begin{array}{c}
      #1 \\
      #2 \\
      #3 \\
      \end{array}
      \right) }                        %3次元ベクトル成分表示



%数列関係


\def\an#1{\verb|{|$#1$\verb|}|}


%極限関係

\def\limit#1#2{\stackrel{#1 \to #2}{\longrightarrow}}   %等式変形からの極限
\def\dlim#1#2{{\displaystyle \lim_{#1\to#2}}} %文中で数式表示の極限を使う



%積分関係

\def\dint#1#2{{\displaystyle \int_{#1}^{#2}}} %文中で数式表示の積分を使う時

\def\ne{\nearrow}
\def\se{\searrow}
\def\nw{\nwarrow}
\def\ne{\nearrow}


%便利なやつ

\def\case#1#2{%
 \[\left\{%
 \begin{array}{l}%
 #1 \\%
 #2%
 \end{array}%
 \right.\] }                           %場合分け
 
\def\1{$\cos\theta=c$,$\sin\theta=s$とおく.}  %cs表示を与える前書きシータ
\def\2{$\cos t=c$,$\sin t=s$とおく.}     %cs表示を与える前書きt
\def\3{$\cos x=c$,$\sin x=s$とおく.}                %cs表示を与える前書きx

\def\fig#1#2#3 {%
\begin{wrapfigure}[#1]{r}{#2 zw}%
\vspace*{-1zh}%
\input{#3}%
\end{wrapfigure} }           %絵の挿入


\def\a{\alpha}   %ギリシャ文字
\def\b{\beta}
\def\g{\gamma}

%問題番号のためのマクロ

\newcounter{nombre} %必須
\renewcommand{\thenombre}{\arabic{nombre}} %任意
\setcounter{nombre}{2} %任意
\newcounter{nombresub}[nombre] %親子関係を定義
\renewcommand{\thenombresub}{\arabic{nombresub}} %任意
\setcounter{nombresub}{0} %任意
\newcommand{\prob}[1][]{\refstepcounter{nombre}#1[問題 \thenombre]}
\newcommand{\probsub}[1][]{\refstepcounter{nombresub}#1(\thenombresub)}


%1-1みたいなカウンタ(todaiとtodaia)
\newcounter{todai}
\setcounter{todai}{0}
\newcounter{todaisub}[todai] 
\setcounter{todaisub}{0} 
\newcommand{\todai}[1][]{\refstepcounter{todai}#1 \thetodai-\thetodaisub}
\newcommand{\todaib}[1][]{\refstepcounter{todai}#1\refstepcounter{todaisub}#1 {\bf [問題 \thetodai.\thetodaisub]}}
\newcommand{\todaia}[1][]{\refstepcounter{todaisub}#1 {\bf [問題 \thetodai.\thetodaisub]}}


     \begin{oframed}
     整数を係数とする次数$3$の多項式$P(x)$で,次の条件を満たすものを求めよ.
          \begin{enumerate}[(1)]
          \item $P(x)$のグラフは原点に関して対称である.
          \item $P(x)=0$は重根を持たない.
          \item $P(x)$は極大値も極小値も持たない.
          \item $P\left(\dfrac{1}{2}\right)$は整数である.
          \item $0<P(1)<6$である.
          \end{enumerate}
     \end{oframed}

\setlength{\columnseprule}{0.4pt}
\begin{multicols}{2}
{\bf[解]}条件(1)から,$P(x)$は奇関数であるから,$a,b\in\mathbb{R}$として
$P(x)=ax^3+bx$とおける.ただし$a\not=0$である.さらに条件(2)から,$b\not=0$が従う.
($b=0$なら重根$x=0$を持つ.)さらに条件(3)から$P'(x)=3ax^2+b$の符号が入れ替わらなければよい.言い換えれば常に非負か,常に非正であればよい.然るに$P'(0)=b$及び$x\to\infty$の極限を考えることで,$a,b$の符号が一致していなければならない($\because ab\not=0$).つまり$ab>0$である.

次に(4)から
      \begin{align}
      &P\left(\frac{1}{2}\right)=\frac{a+4b}{8}\in\mathbb{Z} \nonumber\\
      \Longleftrightarrow&a+4b=8k(k\in\mathbb{Z})\label{1}
      \end{align}
となる.以上に加えて条件(5)から$0<a+b<6$である.$a,b\in\mathbb{Z}$及び$ab>0$からあり得る
$(a,b)$の候補は
     \begin{align*}
     (1,1),(1,2),(1,3),(1,4),(2,1)  \\
     (2,2),(2,3),(3,1),(3,2),(4,1)
     \end{align*}
のみであるが,このうち
\eqref{1}を満たすのは$(a,b)=(4,1)$のみである.故に求めるのは$P(x)=4x^3+x\cdots$(答)である.  
\newpage
\end{multicols}
\end{document}