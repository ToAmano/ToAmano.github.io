\documentclass[a4j]{jarticle}
\usepackage{amsmath}
\usepackage{ascmac}
\usepackage{amssymb}
\usepackage{enumerate}
\usepackage{multicol}
\usepackage{framed}
\usepackage{fancyhdr}
\usepackage{latexsym}
\usepackage{indent}
\usepackage{cases}
\usepackage[dvips]{graphicx}
\usepackage{color}
\usepackage{emath}
\usepackage{emathPp}
\allowdisplaybreaks
\pagestyle{fancy}
\lhead{}
\chead{}
\rhead{東京大学前期$1978$年$2$番}
\begin{document}
%分数関係


\def\tfrac#1#2{{\textstyle\frac{#1}{#2}}} %数式中で文中表示の分数を使う時


%Σ関係

\def\dsum#1#2{{\displaystyle\sum_{#1}^{#2}}} %文中で数式表示のΣを使う時


%ベクトル関係


\def\vector#1{\overrightarrow{#1}}  %ベクトルを表現したいとき(aベクトルを表現するときは\ver
\def\norm#1{|\overrightarrow{#1}|} %ベクトルの絶対値
\def\vtwo#1#2{ \left(%
      \begin{array}{c}%
      #1 \\%
      #2 \\%
      \end{array}%
      \right) }                        %2次元ベクトル成分表示
      
      \def\vthree#1#2#3{ \left(
      \begin{array}{c}
      #1 \\
      #2 \\
      #3 \\
      \end{array}
      \right) }                        %3次元ベクトル成分表示



%数列関係


\def\an#1{\verb|{|$#1$\verb|}|}


%極限関係

\def\limit#1#2{\stackrel{#1 \to #2}{\longrightarrow}}   %等式変形からの極限
\def\dlim#1#2{{\displaystyle \lim_{#1\to#2}}} %文中で数式表示の極限を使う



%積分関係

\def\dint#1#2{{\displaystyle \int_{#1}^{#2}}} %文中で数式表示の積分を使う時

\def\ne{\nearrow}
\def\se{\searrow}
\def\nw{\nwarrow}
\def\ne{\nearrow}


%便利なやつ

\def\case#1#2{%
 \[\left\{%
 \begin{array}{l}%
 #1 \\%
 #2%
 \end{array}%
 \right.\] }                           %場合分け
 
\def\1{$\cos\theta=c$,$\sin\theta=s$とおく.}  %cs表示を与える前書きシータ
\def\2{$\cos t=c$,$\sin t=s$とおく.}     %cs表示を与える前書きt
\def\3{$\cos x=c$,$\sin x=s$とおく.}                %cs表示を与える前書きx

\def\fig#1#2#3 {%
\begin{wrapfigure}[#1]{r}{#2 zw}%
\vspace*{-1zh}%
\input{#3}%
\end{wrapfigure} }           %絵の挿入


\def\a{\alpha}   %ギリシャ文字
\def\b{\beta}
\def\g{\gamma}

%問題番号のためのマクロ

\newcounter{nombre} %必須
\renewcommand{\thenombre}{\arabic{nombre}} %任意
\setcounter{nombre}{2} %任意
\newcounter{nombresub}[nombre] %親子関係を定義
\renewcommand{\thenombresub}{\arabic{nombresub}} %任意
\setcounter{nombresub}{0} %任意
\newcommand{\prob}[1][]{\refstepcounter{nombre}#1[問題 \thenombre]}
\newcommand{\probsub}[1][]{\refstepcounter{nombresub}#1(\thenombresub)}


%1-1みたいなカウンタ(todaiとtodaia)
\newcounter{todai}
\setcounter{todai}{0}
\newcounter{todaisub}[todai] 
\setcounter{todaisub}{0} 
\newcommand{\todai}[1][]{\refstepcounter{todai}#1 \thetodai-\thetodaisub}
\newcommand{\todaib}[1][]{\refstepcounter{todai}#1\refstepcounter{todaisub}#1 {\bf [問題 \thetodai.\thetodaisub]}}
\newcommand{\todaia}[1][]{\refstepcounter{todaisub}#1 {\bf [問題 \thetodai.\thetodaisub]}}


     \begin{oframed}
     $x$の関数$f(x)=(x^2-4)(x^2-9)$の,$t\le x\le t+1$という範囲における最大値を$g(t)$とする.$t$が$-3\le t\le3$の範囲を
     動く時,関数$s=g(t)$を求め,そのグラフを描け.
     \end{oframed}

\setlength{\columnseprule}{0.4pt}
\begin{multicols}{2}
{\bf[解]} まず,
     \[f'(x)=2x(2x^2-13)\]
である.故に$\a=\sqrt{13/2}$とおいて下表を得る.
     \begin{align*}
          \begin{array}{|c|c|c|c|c|c|c|c|} \hline
          t&     &-\a              &    &  0&     &\b        &        \\ \hline
           f'&-   &0               &+   & 0 & -  &  0       &+      \\ \hline
          f&\se&                 &\ne&    &\se&            &\ne   \\ \hline
          \end{array}
    \end{align*}
また,
     \begin{align*}
     &f(t)=f(t+1) \\
     \Longleftrightarrow
     &(t+3)(t-2)(4t+2)=0
     \end{align*}     
である.
故に$y=f(x)$のグラフは下図.\\

     \begin{zahyou}[ul=8mm,gentenhaiti={[ne]}](-4,4)(-2,6)
     \def\Fx{X**4/10-13*X**2/10+36/10}
     \def\Gx{X**4/10+4*X**3/10-7*X**2/10-22*X/10+24/10}
     \def\aval{0.5}
     \def\bval{-0.5}
     \YGurafu*(*)\Fx
     \Put{(-3,0)}[sw]{$-3$}
     \Put{(3,0)}[se]{$3$}
     \Put{(-2,0)}[se]{$-2$}
     \Put{(2,0)}[sw]{$2$}
     \Put{(0,3.6)}[n]{$36$}
     \YTen\Fx\aval\A
     \YTen\Fx\bval\B
     \Put\A[syaei=xy,xlabel=\frac{1}{2},ylabel=]{}
     \Put\B[syaei=xy,xlabel=\frac{-1}{2},ylabel=]{}
     \end{zahyou}

従って,求める関数$g(x)$は以下のようになる.
     \begin{align*}
     g(x)=\left\{
          \begin{array}{ll}
          f(t+1)&(-3\le t\le -1) \\
          36    &(-1\le t\le 0) \\
          f(t)    &(0\le t\le 2) \\
          f(t+1)&(2\le t\le 3)
          \end{array}
     \right.
     \end{align*}
$f(x+1)=x^4+4x^3-7x^2-22x+24$に注意して,このグラフは右上図.$\cdots$(答)
     
     \begin{zahyou}[ul=8mm](-4,4)(-2,10)
     \def\Fx{X**4/10-13*X**2/10+36/10}
     \def\Gx{X**4/10+4*X**3/10-7*X**2/10-22*X/10+24/10}
     \def\Hx{3.6}
     \def\cval{-1}
     \def\dval{3}
     \YGurafu(*)\Fx{0}{2}
     \YGurafu(*)\Gx{-3}{-1}
     \YGurafu(*)\Gx{2}{3}
     \YGurafu\Hx{-1}{0}
     \Put{(-3,0)}[sw]{$-3$}
     \Put{(2,0)}[sw]{$2$}
     \Put{(0,3.6)}[n]{$36$}
     \YTen\Gx\cval\C
     \YTen\Gx\dval\D
     \Put\C[syaei=x,xlabel=-1]{}
     \Put\D[syaei=x,xlabel=3]{}
     \end{zahyou}
     
\newpage
\end{multicols}
\end{document}