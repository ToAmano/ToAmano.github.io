\documentclass[a4j]{jarticle}
\usepackage{amsmath}
\usepackage{ascmac}
\usepackage{amssymb}
\usepackage{enumerate}
\usepackage{multicol}
\usepackage{framed}
\usepackage{fancyhdr}
\usepackage{latexsym}
\usepackage{indent}
\usepackage{cases}
\usepackage[dvips]{graphicx}
\usepackage{color}
\usepackage{emath}
\usepackage{emathPp}
\allowdisplaybreaks
\pagestyle{fancy}
\lhead{}
\chead{}
\rhead{東京大学前期$1978$年$3$番}
\begin{document}
%分数関係


\def\tfrac#1#2{{\textstyle\frac{#1}{#2}}} %数式中で文中表示の分数を使う時


%Σ関係

\def\dsum#1#2{{\displaystyle\sum_{#1}^{#2}}} %文中で数式表示のΣを使う時


%ベクトル関係


\def\vector#1{\overrightarrow{#1}}  %ベクトルを表現したいとき(aベクトルを表現するときは\ver
\def\norm#1{|\overrightarrow{#1}|} %ベクトルの絶対値
\def\vtwo#1#2{ \left(%
      \begin{array}{c}%
      #1 \\%
      #2 \\%
      \end{array}%
      \right) }                        %2次元ベクトル成分表示
      
      \def\vthree#1#2#3{ \left(
      \begin{array}{c}
      #1 \\
      #2 \\
      #3 \\
      \end{array}
      \right) }                        %3次元ベクトル成分表示



%数列関係


\def\an#1{\verb|{|$#1$\verb|}|}


%極限関係

\def\limit#1#2{\stackrel{#1 \to #2}{\longrightarrow}}   %等式変形からの極限
\def\dlim#1#2{{\displaystyle \lim_{#1\to#2}}} %文中で数式表示の極限を使う



%積分関係

\def\dint#1#2{{\displaystyle \int_{#1}^{#2}}} %文中で数式表示の積分を使う時

\def\ne{\nearrow}
\def\se{\searrow}
\def\nw{\nwarrow}
\def\ne{\nearrow}


%便利なやつ

\def\case#1#2{%
 \[\left\{%
 \begin{array}{l}%
 #1 \\%
 #2%
 \end{array}%
 \right.\] }                           %場合分け
 
\def\1{$\cos\theta=c$,$\sin\theta=s$とおく.}  %cs表示を与える前書きシータ
\def\2{$\cos t=c$,$\sin t=s$とおく.}     %cs表示を与える前書きt
\def\3{$\cos x=c$,$\sin x=s$とおく.}                %cs表示を与える前書きx

\def\fig#1#2#3 {%
\begin{wrapfigure}[#1]{r}{#2 zw}%
\vspace*{-1zh}%
\input{#3}%
\end{wrapfigure} }           %絵の挿入


\def\a{\alpha}   %ギリシャ文字
\def\b{\beta}
\def\g{\gamma}

%問題番号のためのマクロ

\newcounter{nombre} %必須
\renewcommand{\thenombre}{\arabic{nombre}} %任意
\setcounter{nombre}{2} %任意
\newcounter{nombresub}[nombre] %親子関係を定義
\renewcommand{\thenombresub}{\arabic{nombresub}} %任意
\setcounter{nombresub}{0} %任意
\newcommand{\prob}[1][]{\refstepcounter{nombre}#1[問題 \thenombre]}
\newcommand{\probsub}[1][]{\refstepcounter{nombresub}#1(\thenombresub)}


%1-1みたいなカウンタ(todaiとtodaia)
\newcounter{todai}
\setcounter{todai}{0}
\newcounter{todaisub}[todai] 
\setcounter{todaisub}{0} 
\newcommand{\todai}[1][]{\refstepcounter{todai}#1 \thetodai-\thetodaisub}
\newcommand{\todaib}[1][]{\refstepcounter{todai}#1\refstepcounter{todaisub}#1 {\bf [問題 \thetodai.\thetodaisub]}}
\newcommand{\todaia}[1][]{\refstepcounter{todaisub}#1 {\bf [問題 \thetodai.\thetodaisub]}}


     \begin{oframed}
     $C$を放物線$y=3x^2/2-1/3$とする.$C$上の点Q$(t,3t^2/2-1/3)$を通り,Qにおける$C$の接線と垂直な直線を,
     Qにおける$C$の法線とする.
          \begin{enumerate}[(1)]
          \item $xy$平面上の点P$(x,y)$でPを通る$C$の法線が一本だけ引けるようなものの存在範囲を求め,図示せよ.
          \item (1)で求めた範囲と放物線の内部(不等式$y>3x^2/2-1/3$の定める範囲)の共通部分の面積を求めよ.
          \end{enumerate}
     \end{oframed}

\setlength{\columnseprule}{0.4pt}
\begin{multicols}{2}
{\bf[解]} $y'=3x$だから,$C$における法線は
     \begin{align}
     (x-t)+3t\left(y-\frac{3}{2}t^2+\frac{1}{3}\right)=0 \nonumber\\
     9t^3-6yt-2x=0 \label{1}
      \end{align}
である.放物線では接点が異なれば法線が異なるから,点Pにたいし\eqref{1}を満たす$t$がただ一つ存在すればよい.
すなわち,求める領域を$D$とすると,
     \begin{align*}
     D=\{(x,y)\mid \eqref{1}\text{を満たす$t$が唯一つ存在する.}\}
     \end{align*}
\eqref{1}の左辺$g(t)$とおく.$g'(t)=27t^2-6y$より,$y$の値によって以下のようになる.\\

     \begin{indentation}{2zw}{0pt}
     \noindent\underline{(i)$y\le0$の時}\\
     $g'(t)\ge0$だから,$g(t)$は単調増加.これと$g(t)$は$3$次関数だから,$g(t)=0$は唯一つ解を持つ.
     故に$D$に含まれる.\\
     
     \noindent\underline{(ii)$y>0$の時}\\
     下表を得る.ただし$\a=\sqrt{2y}/3$である.
          \begin{align*}
               \begin{array}{|c|c|c|c|c|c|}\hline
               t  &     &-\a&     &\a &     \\\hline
               g' &+  &0  & -   & 0 & +  \\\hline
               g &\ne&    &\se&    &\ne \\ \hline
               \end{array}
          \end{align*}
     従って,グラフの概形は下図のようになる.
     
     
     $D$のようになる条件は,
          \begin{align}
          g(-\a)g(\a)>0\label{2}
          \end{align}
     である.ここで\eqref{1}から
          \begin{align*}
          g(\pm\a)&=\pm 9\a^3\mp 6y\a -2x \\
          &=\mp \frac{4\sqrt{2}}{3}y^{3/2}-2x
          \end{align*}
     であるから,\eqref{2}に代入して
          \begin{align}
          \left(\frac{2\sqrt{2}}{3}y^{3/2}-x\right)\left(-\frac{2\sqrt{2}}{3}y^{3/2}-x\right)>0\label{3}
          \end{align}
     \end{indentation}
     
以上から,求める領域$D$は,
     \begin{align}
     y\le0 \lor(y>0\land \eqref{3})
     \end{align}
     
で,図示して下図斜線部(境界含む).$\cdots$((1)の答)

     \begin{zahyou}[ul=10mm](-3,3)(-1,3)
     \def\Fx{X**(2/3)}
     \def\Gx{(-1*X)**(2/3)}
     \def\Hx{-1}
     \YGurafu\Fx{0}{\xmax}
     \YGurafu\Gx{\xmin}{0}
     \YNurii*\Fx\Hx{0}{\xmax}
     \YNurii*\Gx\Hx{\xmin}{0}
     \end{zahyou}


この領域の対称性から$D$のうち$x\ge0$の部分の面積$S_1$とすると,求める面積$S$との関係は
     \begin{align}
     S=2S_1\label{4}
     \end{align}
である.$y=f(x)$と$x=\dfrac{2\sqrt{2}}{3}y^{3/2}$の$y>0$での交点は,
     \begin{align*}
     &y=\frac{3}{2}\left(\frac{2\sqrt{2}}{3}y^{3/2}\right)^2-\frac{1}{3} \\
     &4y^3-3y-1=0 \\
     &(y-1)(4y^2+4y+1)=0 \\
     &y=1&(\because y>0)
     \end{align*}
である.このとき$x=\dfrac{2\sqrt{2}}{3}(\equiv \b)$である.グラフの概形は下図.\\

     \begin{zahyou}[ul=10mm](-1,3)(-1.5,2)
     \def\Fx{X**(2/3)}
     \def\Mx{1.5*X**2-1}
     \YGurafu\Fx{0}{\xmax}
     \YGurafu\Mx{0}{\xmax}
     \YKouten\Fx\Mx{0}{}\tempi\P
     \YNurii*\Fx\Mx{0}{\tempi}
     \Put\P[syaei=xy,xlabel=\b,ylabel=1]{}
     \end{zahyou}

さて,$x>0$のとき
     \begin{align*}
     &x=\frac{2\sqrt{2}}{3}y^{3/2} \\
     \therefore \,&y=\frac{\sqrt[3]{9}}{2}x^{2/3}
     \end{align*}
に注意して
     \begin{align*}
     S_1&=\int_0^\b \left(f(x)-\frac{\sqrt[3]{9}}{2}x^{2/3}\right)\,dx \\
     &=\teisekibun{\frac{1}{2}x^3-\frac{1}{3}x}{0}{\b}-\frac{\sqrt[3]{9}}{2}\teisekibun{\frac{3}{5}x^{5/3}}{0}{\b} \\
     &=\frac{44\sqrt{2}}{135}
     \end{align*}
であるから,\eqref{4}に代入して
     \[S=\frac{88\sqrt{2}}{135}\]
である.$\cdots$((2)の答)
\newpage
\end{multicols}
\end{document}