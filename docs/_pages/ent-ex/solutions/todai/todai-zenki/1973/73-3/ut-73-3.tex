\documentclass[a4j]{jarticle}
\usepackage{amsmath}
\usepackage{ascmac}
\usepackage{amssymb}
\usepackage{enumerate}
\usepackage{multicol}
\usepackage{framed}
\usepackage{fancyhdr}
\usepackage{latexsym}
\usepackage{indent}
\usepackage{cases}
\usepackage[dvips]{graphicx}
\usepackage{color}
\usepackage{emath}
\usepackage{emathPp}
\allowdisplaybreaks
\pagestyle{fancy}
\lhead{}
\chead{}
\rhead{東京大学前期$1973$年$3$番}
\begin{document}
%分数関係


\def\tfrac#1#2{{\textstyle\frac{#1}{#2}}} %数式中で文中表示の分数を使う時


%Σ関係

\def\dsum#1#2{{\displaystyle\sum_{#1}^{#2}}} %文中で数式表示のΣを使う時


%ベクトル関係


\def\vector#1{\overrightarrow{#1}}  %ベクトルを表現したいとき(aベクトルを表現するときは\ver
\def\norm#1{|\overrightarrow{#1}|} %ベクトルの絶対値
\def\vtwo#1#2{ \left(%
      \begin{array}{c}%
      #1 \\%
      #2 \\%
      \end{array}%
      \right) }                        %2次元ベクトル成分表示
      
      \def\vthree#1#2#3{ \left(
      \begin{array}{c}
      #1 \\
      #2 \\
      #3 \\
      \end{array}
      \right) }                        %3次元ベクトル成分表示



%数列関係


\def\an#1{\verb|{|$#1$\verb|}|}


%極限関係

\def\limit#1#2{\stackrel{#1 \to #2}{\longrightarrow}}   %等式変形からの極限
\def\dlim#1#2{{\displaystyle \lim_{#1\to#2}}} %文中で数式表示の極限を使う



%積分関係

\def\dint#1#2{{\displaystyle \int_{#1}^{#2}}} %文中で数式表示の積分を使う時

\def\ne{\nearrow}
\def\se{\searrow}
\def\nw{\nwarrow}
\def\ne{\nearrow}


%便利なやつ

\def\case#1#2{%
 \[\left\{%
 \begin{array}{l}%
 #1 \\%
 #2%
 \end{array}%
 \right.\] }                           %場合分け
 
\def\1{$\cos\theta=c$,$\sin\theta=s$とおく.}  %cs表示を与える前書きシータ
\def\2{$\cos t=c$,$\sin t=s$とおく.}     %cs表示を与える前書きt
\def\3{$\cos x=c$,$\sin x=s$とおく.}                %cs表示を与える前書きx

\def\fig#1#2#3 {%
\begin{wrapfigure}[#1]{r}{#2 zw}%
\vspace*{-1zh}%
\input{#3}%
\end{wrapfigure} }           %絵の挿入


\def\a{\alpha}   %ギリシャ文字
\def\b{\beta}
\def\g{\gamma}

%問題番号のためのマクロ

\newcounter{nombre} %必須
\renewcommand{\thenombre}{\arabic{nombre}} %任意
\setcounter{nombre}{2} %任意
\newcounter{nombresub}[nombre] %親子関係を定義
\renewcommand{\thenombresub}{\arabic{nombresub}} %任意
\setcounter{nombresub}{0} %任意
\newcommand{\prob}[1][]{\refstepcounter{nombre}#1[問題 \thenombre]}
\newcommand{\probsub}[1][]{\refstepcounter{nombresub}#1(\thenombresub)}


%1-1みたいなカウンタ(todaiとtodaia)
\newcounter{todai}
\setcounter{todai}{0}
\newcounter{todaisub}[todai] 
\setcounter{todaisub}{0} 
\newcommand{\todai}[1][]{\refstepcounter{todai}#1 \thetodai-\thetodaisub}
\newcommand{\todaib}[1][]{\refstepcounter{todai}#1\refstepcounter{todaisub}#1 {\bf [問題 \thetodai.\thetodaisub]}}
\newcommand{\todaia}[1][]{\refstepcounter{todaisub}#1 {\bf [問題 \thetodai.\thetodaisub]}}


     \begin{oframed}
     区間$1\le x\le 3$において次のように定義された関数$f(x)$がある.
          \begin{align*}
          f(x)=
               \begin{cases}
               1&(1\le x\le 2) \\
               x-1&(2\le x\le3)
               \end{cases}
          \end{align*}
     いま実数$a$に対して,区間$1\le x\le 3$における関数$f(x)-ax$の最大値から最小値を
     引いた値を$V(a)$とおく.このとき次の問いに答えよ.
         \begin{enumerate}[(1)]
         \item $a$がすべての実数に渡って動くとき,$V(a)$の最小値を求めよ.
         \item $V(a)$の最小値を与えるような$a$の値を求めよ.
         \end{enumerate}
     \end{oframed}

\setlength{\columnseprule}{0.4pt}
\begin{multicols}{2}
{\bf[解]} $g(x)=f(x)-ax$とおく.
          \begin{align*}
          g(x)=
               \begin{cases}
               -ax+1&(1\le x\le 2) \\
               (1-a)x-1&(2\le x\le3)
               \end{cases}
          \end{align*}     
である($g(x)$は連続).$a$の値によって以下のようになる.\\

     \begin{indentation}{2zw}{0pt}
     \noindent\underline{(i)$a\le0$の時}\\
     $g(x)$は増加関数だから,
          \begin{align*}
          V(a)=g(3)-g(1)=1-2a
          \end{align*}
          
     \noindent\underline{(ii)$0\le a\le 1$の時}\\
     $g(x)$のグラフの概形は下図.
     
          \begin{zahyou}[ul=10mm](-0.5,3.5)(-0.5,2.5)
          \def\Fx{1-X/4}
          \def\Gx{3*X/4-1}
          \def\aval{1}
          \def\bval{2}
          \def\cval{3}
          \YGurafu\Fx{1}{2}
          \YGurafu\Gx{2}{3}
          \YTen\Fx\aval\A
          \YTen\Fx\bval\B
          \YTen\Gx\cval\C
          \Put\A[syaei=xy,xlabel=1,ylabel=1-a]{}
          \Put\B[syaei=xy,xlabel=2,ylabel=1-2a]{}
          \Put\C[syaei=xy,xlabel=3,ylabel=2-3a]{}
          \end{zahyou}
          
     また
          \begin{align*}
               \begin{cases}
               g(3)\ge g(1)&(0\le a\le 1/2) \\
               g(3)\le g(1) &(1/2\le a\le 1)
               \end{cases}
          \end{align*}
     であるから,
          \begin{align*}
          V(a)=&
               \begin{cases}
               g(3)-g(2)&(0\le a\le 1/2)\\
               g(1)-g(2)&(1/2\le a\le 1)
               \end{cases} \\
               =&
               \begin{cases}
               1-a&(0\le a\le 1/2)\\
               a&(1/2\le a\le 1)
               \end{cases}
          \end{align*}
     \noindent\underline{(iii)$1\le a$の時}\\
     $g(x)$は減少関数だから,
          \begin{align*}
          V(a)=g(1)-g(3)=2a-1
          \end{align*}
     \end{indentation}
     
以上から,$\min V(a)=V(1/2)=1/2$である.$\cdots$(答)
\newpage
\end{multicols}
\end{document}