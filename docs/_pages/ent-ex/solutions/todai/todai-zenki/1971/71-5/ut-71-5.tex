\documentclass[a4j]{jarticle}
\usepackage{amsmath}
\usepackage{ascmac}
\usepackage{amssymb}
\usepackage{enumerate}
\usepackage{multicol}
\usepackage{framed}
\usepackage{latexsym}
\usepackage{indent}
\usepackage{cases}
\title{}
\begin{document}
%分数関係


\def\tfrac#1#2{{\textstyle\frac{#1}{#2}}} %数式中で文中表示の分数を使う時


%Σ関係

\def\dsum#1#2{{\displaystyle\sum_{#1}^{#2}}} %文中で数式表示のΣを使う時


%ベクトル関係


\def\vector#1{\overrightarrow{#1}}  %ベクトルを表現したいとき(aベクトルを表現するときは\ver
\def\norm#1{|\overrightarrow{#1}|} %ベクトルの絶対値
\def\vtwo#1#2{ \left(%
      \begin{array}{c}%
      #1 \\%
      #2 \\%
      \end{array}%
      \right) }                        %2次元ベクトル成分表示
      
      \def\vthree#1#2#3{ \left(
      \begin{array}{c}
      #1 \\
      #2 \\
      #3 \\
      \end{array}
      \right) }                        %3次元ベクトル成分表示



%数列関係


\def\an#1{\verb|{|$#1$\verb|}|}


%極限関係

\def\limit#1#2{\stackrel{#1 \to #2}{\longrightarrow}}   %等式変形からの極限
\def\dlim#1#2{{\displaystyle \lim_{#1\to#2}}} %文中で数式表示の極限を使う



%積分関係

\def\dint#1#2{{\displaystyle \int_{#1}^{#2}}} %文中で数式表示の積分を使う時

\def\ne{\nearrow}
\def\se{\searrow}
\def\nw{\nwarrow}
\def\ne{\nearrow}


%便利なやつ

\def\case#1#2{%
 \[\left\{%
 \begin{array}{l}%
 #1 \\%
 #2%
 \end{array}%
 \right.\] }                           %場合分け
 
\def\1{$\cos\theta=c$,$\sin\theta=s$とおく.}  %cs表示を与える前書きシータ
\def\2{$\cos t=c$,$\sin t=s$とおく.}     %cs表示を与える前書きt
\def\3{$\cos x=c$,$\sin x=s$とおく.}                %cs表示を与える前書きx

\def\fig#1#2#3 {%
\begin{wrapfigure}[#1]{r}{#2 zw}%
\vspace*{-1zh}%
\input{#3}%
\end{wrapfigure} }           %絵の挿入


\def\a{\alpha}   %ギリシャ文字
\def\b{\beta}
\def\g{\gamma}

%問題番号のためのマクロ

\newcounter{nombre} %必須
\renewcommand{\thenombre}{\arabic{nombre}} %任意
\setcounter{nombre}{2} %任意
\newcounter{nombresub}[nombre] %親子関係を定義
\renewcommand{\thenombresub}{\arabic{nombresub}} %任意
\setcounter{nombresub}{0} %任意
\newcommand{\prob}[1][]{\refstepcounter{nombre}#1[問題 \thenombre]}
\newcommand{\probsub}[1][]{\refstepcounter{nombresub}#1(\thenombresub)}


%1-1みたいなカウンタ(todaiとtodaia)
\newcounter{todai}
\setcounter{todai}{0}
\newcounter{todaisub}[todai] 
\setcounter{todaisub}{0} 
\newcommand{\todai}[1][]{\refstepcounter{todai}#1 \thetodai-\thetodaisub}
\newcommand{\todaib}[1][]{\refstepcounter{todai}#1\refstepcounter{todaisub}#1 {\bf [問題 \thetodai.\thetodaisub]}}
\newcommand{\todaia}[1][]{\refstepcounter{todaisub}#1 {\bf [問題 \thetodai.\thetodaisub]}}


\begin{oframed}
$n$を正の整数とし,$\left(\cos \dfrac{2\pi}{n}k,\sin\dfrac{2\pi}{n}k\right)$を座標とする点を$Q_n$で表す.このとき$n$個の点$Q_0,Q_1,\cdots,Q_{n-1}$によって円周$x^2+y^2=1$は$n$等分される.

平面上の点$P$の座標を$(a,b)$とし,
$s_n=\dfrac{1}{n}\left(\overline{PQ_0}^2+\overline{PQ_1}^2+\dots+\overline{PQ_{n-1}}^2\right)$とするとき,$\lambda_p=\dlim{n}{\infty}s_n$の値を$a$,$b$を用いてあらわせ.また,$P$がどこにあれば$\lambda_p$の値は最小となるか.
\end{oframed}

\setlength{\columnseprule}{0.4pt}
\begin{multicols}{2}
{\bf[解]}$L_n=\overline{PQ_n}^2$とおくと,
     \begin{align*}
     L_n&=\left(\cos\frac{2\pi}{n}k-a\right)^2+\left(\sin\frac{2\pi}{n}k-b\right)^2  \\
     &=1+a^2+b^2-2\left(a\cos\frac{2\pi}{n}k+b\sin\frac{2\pi}{n}k\right) 
     \end{align*}
となる.$a_k=\left(a\cos\dfrac{2\pi}{n}k+b\sin\dfrac{2\pi}{n}k\right)$とおけば区分求積により,
     \begin{align*}
     \frac{1}{n}\sum_{k=0}^{n-1}a_k
     &\limit{n}{\infty}\int_0^1\left(a\cos2\pi x+b\sin2\pi x\right)dx \\
     &=0
     \end{align*}
より,
     \begin{align*}
     s_n&=\frac{1}{n}\sum_{k=0}^{n-1}L_k \\
     &=\frac{1}{n}\left\{1+a^2+b^2-2\sum_{k=0}^{n-1}a_k\right\} \\
     &\limit{n}{\infty}1+a^2+b^2
     \end{align*}
となって$\lambda_p=\dlim{n}{\infty}s_n=1+a^2+b^2\cdots$(答)で,これは$P(0,0)$の時に最小であること明らかである.$\cdots$(答)   
 \\
 \\
  \\
{\bf[解注]}$\dsum{k=0}{n-1}a_k=0$は極限を取る前から成立していること明白だが,減点される恐れありと判断してあえて区分求積を用いた.      
\newpage
\end{multicols}
\end{document}