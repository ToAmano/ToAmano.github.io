\documentclass[a4j]{jarticle}
\usepackage{amsmath}
\usepackage{ascmac}
\usepackage{amssymb}
\usepackage{enumerate}
\usepackage{multicol}
\usepackage{framed}
\usepackage{fancyhdr}
\usepackage{latexsym}
\usepackage{indent}
\usepackage{cases}
\allowdisplaybreaks
\pagestyle{fancy}
\lhead{}
\chead{}
\rhead{東京大学前期$1971$年$6$番}
\begin{document}
%分数関係


\def\tfrac#1#2{{\textstyle\frac{#1}{#2}}} %数式中で文中表示の分数を使う時


%Σ関係

\def\dsum#1#2{{\displaystyle\sum_{#1}^{#2}}} %文中で数式表示のΣを使う時


%ベクトル関係


\def\vector#1{\overrightarrow{#1}}  %ベクトルを表現したいとき(aベクトルを表現するときは\ver
\def\norm#1{|\overrightarrow{#1}|} %ベクトルの絶対値
\def\vtwo#1#2{ \left(%
      \begin{array}{c}%
      #1 \\%
      #2 \\%
      \end{array}%
      \right) }                        %2次元ベクトル成分表示
      
      \def\vthree#1#2#3{ \left(
      \begin{array}{c}
      #1 \\
      #2 \\
      #3 \\
      \end{array}
      \right) }                        %3次元ベクトル成分表示



%数列関係


\def\an#1{\verb|{|$#1$\verb|}|}


%極限関係

\def\limit#1#2{\stackrel{#1 \to #2}{\longrightarrow}}   %等式変形からの極限
\def\dlim#1#2{{\displaystyle \lim_{#1\to#2}}} %文中で数式表示の極限を使う



%積分関係

\def\dint#1#2{{\displaystyle \int_{#1}^{#2}}} %文中で数式表示の積分を使う時

\def\ne{\nearrow}
\def\se{\searrow}
\def\nw{\nwarrow}
\def\ne{\nearrow}


%便利なやつ

\def\case#1#2{%
 \[\left\{%
 \begin{array}{l}%
 #1 \\%
 #2%
 \end{array}%
 \right.\] }                           %場合分け
 
\def\1{$\cos\theta=c$,$\sin\theta=s$とおく.}  %cs表示を与える前書きシータ
\def\2{$\cos t=c$,$\sin t=s$とおく.}     %cs表示を与える前書きt
\def\3{$\cos x=c$,$\sin x=s$とおく.}                %cs表示を与える前書きx

\def\fig#1#2#3 {%
\begin{wrapfigure}[#1]{r}{#2 zw}%
\vspace*{-1zh}%
\input{#3}%
\end{wrapfigure} }           %絵の挿入


\def\a{\alpha}   %ギリシャ文字
\def\b{\beta}
\def\g{\gamma}

%問題番号のためのマクロ

\newcounter{nombre} %必須
\renewcommand{\thenombre}{\arabic{nombre}} %任意
\setcounter{nombre}{2} %任意
\newcounter{nombresub}[nombre] %親子関係を定義
\renewcommand{\thenombresub}{\arabic{nombresub}} %任意
\setcounter{nombresub}{0} %任意
\newcommand{\prob}[1][]{\refstepcounter{nombre}#1[問題 \thenombre]}
\newcommand{\probsub}[1][]{\refstepcounter{nombresub}#1(\thenombresub)}


%1-1みたいなカウンタ(todaiとtodaia)
\newcounter{todai}
\setcounter{todai}{0}
\newcounter{todaisub}[todai] 
\setcounter{todaisub}{0} 
\newcommand{\todai}[1][]{\refstepcounter{todai}#1 \thetodai-\thetodaisub}
\newcommand{\todaib}[1][]{\refstepcounter{todai}#1\refstepcounter{todaisub}#1 {\bf [問題 \thetodai.\thetodaisub]}}
\newcommand{\todaia}[1][]{\refstepcounter{todaisub}#1 {\bf [問題 \thetodai.\thetodaisub]}}


     \begin{oframed}
     $3$人で'ジャンケン'をして勝者を決めることにする.例えば,$1$人が'紙'を出し,他の$2$人
     が'石'を出せば,ただ一回で丁度一人の勝者が決まることになる.$3$人で'ジャンケン'をして,
     負けた人は次の解に参加しないことにして丁度$1$人の勝者がきまるまで'ジャンケン'を繰り
     返すことにする.この,$k$回目に,はじめて丁度一人の勝者が決まる確率を求めよ.
     \end{oframed}

\setlength{\columnseprule}{0.4pt}
\begin{multicols}{2}
{\bf[解]}$n$回目に$3$人,$2$人でジャンケンする確率をそれぞれ$a_n$,$b_n$とおく.
また,求める確率を$c_n$とおく.ジャンケンの推移する確率は以下の通り.
     \begin{center}
          \begin{tabular}{cc}
          3人$\to$3人  & $1/3$\\
          3人$\to$2人  & $1/3$\\
          3人$\to$1人  &$1/3$ \\
          2人$\to$2人  &$1/3$ \\
          2人$\to$1人  &$2/3$
          \end{tabular}
     \end{center}   
よって,ジャンケンの推移から以下の漸化式を得る.
     \begin{subnumcases}
     {}
     c_n=\frac{1}{3}a_n+\frac{2}{3}b_n \label{1}\\
     a_{n+1}=\frac{1}{3}a_n  \label{2}\\
     b_{n+1}=\frac{1}{3}a_n+\frac{1}{3}b_n \label{3}
     \end{subnumcases}
さらに,初期条件は$a_1=1$,$b_1=0$である.従って\eqref{2}から
     \begin{align}
     a_n=\left(\frac{1}{3}\right)^{n-1} \label{4}
     \end{align}
となる.さらにこれを\eqref{3}に代入して
     \begin{align*}
     b_{n+1}&=\frac{1}{3}b_n+\left(\frac{1}{3}\right)^n \\
     d_{n+1}&=d_n+1          \\
     d_n&=n-1+d_1\\
     &=n-1+3b_1 \\
     &=n-1  \tag{$\because b_1=0$}\\
     \therefore  b_n&=(n-1)\left(\frac{1}{3}\right)^{n-1}
     \end{align*}
を得る.ただし$d_n=3^nb_n$である.これを\eqref{1}に代入すれば,求める確率は     
     \begin{align*}
     c_k=(2k-1)\left(\frac{1}{3}\right)^n\tag{答}
     \end{align*}
である.
 \\
 \\
 
{\bf[解$2$]}まず$k\ge2$とする.事象$A$,$B$を
     \begin{center}
          \begin{tabular}{ll}
          $A\cdots$ & $3$人でジャンケンする. \\
          $B\cdots$ & $2$人でジャンケンする.
          \end{tabular}
     \end{center}
と定める.$k$回目に一人の勝者が決まるのは以下のいずれかである.
     \begin{enumerate}[(i)]
     \item $A\to \dots\to \stackrel{a回目}{A}\to B\to \dots \to \stackrel{k回目}{B}$
     \item $A\to\dots\to \stackrel{k回目}{A}$
     \end{enumerate}
(i)で$A$を行う回数$a$,$B$を行う回数を$b$とすると,
     \begin{align}
     a+b=k\label{1}
     \end{align}      
が成り立つ.ただし$a,b>0$である.[解] でのジャンケンの推移の確率から,(i)となる確率は
     \begin{align*}
     p(a)=\left(\frac{1}{3}\right)^{a-1}\frac{1}{3}\left(\frac{1}{3}\right)^{b-1}\frac{2}{3}%
     =2\left(\frac{1}{3}\right)^k\tag{$\because\eqref{1}$}
     \end{align*}   
同様に(ii)となる確率は
     \begin{align*}
      q=\left(\frac{1}{3}\right)^{k-1}\frac{1}{3}=\left(\frac{1}{3}\right)^k
      \end{align*}
 となる.故に求める確率は
      \begin{align*}
      \sum_{a=0}^{k-1}p(a)+q=(2k-1)\left(\frac{1}{3}\right)^n\tag{答}
      \end{align*}
 となる.     
\newpage
\end{multicols}
\end{document}