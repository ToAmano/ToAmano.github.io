\documentclass[a4j]{jarticle}
\usepackage{amsmath}
\usepackage{ascmac}
\usepackage{amssymb}
\usepackage{enumerate}
\usepackage{multicol}
\usepackage{framed}
\usepackage{fancyhdr}
\usepackage{latexsym}
\usepackage{indent}
\usepackage{cases}
\usepackage[dvips]{graphicx}
\usepackage{color}
\allowdisplaybreaks
\pagestyle{fancy}
\lhead{}
\chead{}
\rhead{東京大学前期$1994$年$3$番}
\begin{document}
%分数関係


\def\tfrac#1#2{{\textstyle\frac{#1}{#2}}} %数式中で文中表示の分数を使う時


%Σ関係

\def\dsum#1#2{{\displaystyle\sum_{#1}^{#2}}} %文中で数式表示のΣを使う時


%ベクトル関係


\def\vector#1{\overrightarrow{#1}}  %ベクトルを表現したいとき(aベクトルを表現するときは\ver
\def\norm#1{|\overrightarrow{#1}|} %ベクトルの絶対値
\def\vtwo#1#2{ \left(%
      \begin{array}{c}%
      #1 \\%
      #2 \\%
      \end{array}%
      \right) }                        %2次元ベクトル成分表示
      
      \def\vthree#1#2#3{ \left(
      \begin{array}{c}
      #1 \\
      #2 \\
      #3 \\
      \end{array}
      \right) }                        %3次元ベクトル成分表示



%数列関係


\def\an#1{\verb|{|$#1$\verb|}|}


%極限関係

\def\limit#1#2{\stackrel{#1 \to #2}{\longrightarrow}}   %等式変形からの極限
\def\dlim#1#2{{\displaystyle \lim_{#1\to#2}}} %文中で数式表示の極限を使う



%積分関係

\def\dint#1#2{{\displaystyle \int_{#1}^{#2}}} %文中で数式表示の積分を使う時

\def\ne{\nearrow}
\def\se{\searrow}
\def\nw{\nwarrow}
\def\ne{\nearrow}


%便利なやつ

\def\case#1#2{%
 \[\left\{%
 \begin{array}{l}%
 #1 \\%
 #2%
 \end{array}%
 \right.\] }                           %場合分け
 
\def\1{$\cos\theta=c$,$\sin\theta=s$とおく.}  %cs表示を与える前書きシータ
\def\2{$\cos t=c$,$\sin t=s$とおく.}     %cs表示を与える前書きt
\def\3{$\cos x=c$,$\sin x=s$とおく.}                %cs表示を与える前書きx

\def\fig#1#2#3 {%
\begin{wrapfigure}[#1]{r}{#2 zw}%
\vspace*{-1zh}%
\input{#3}%
\end{wrapfigure} }           %絵の挿入


\def\a{\alpha}   %ギリシャ文字
\def\b{\beta}
\def\g{\gamma}

%問題番号のためのマクロ

\newcounter{nombre} %必須
\renewcommand{\thenombre}{\arabic{nombre}} %任意
\setcounter{nombre}{2} %任意
\newcounter{nombresub}[nombre] %親子関係を定義
\renewcommand{\thenombresub}{\arabic{nombresub}} %任意
\setcounter{nombresub}{0} %任意
\newcommand{\prob}[1][]{\refstepcounter{nombre}#1[問題 \thenombre]}
\newcommand{\probsub}[1][]{\refstepcounter{nombresub}#1(\thenombresub)}


%1-1みたいなカウンタ(todaiとtodaia)
\newcounter{todai}
\setcounter{todai}{0}
\newcounter{todaisub}[todai] 
\setcounter{todaisub}{0} 
\newcommand{\todai}[1][]{\refstepcounter{todai}#1 \thetodai-\thetodaisub}
\newcommand{\todaib}[1][]{\refstepcounter{todai}#1\refstepcounter{todaisub}#1 {\bf [問題 \thetodai.\thetodaisub]}}
\newcommand{\todaia}[1][]{\refstepcounter{todaisub}#1 {\bf [問題 \thetodai.\thetodaisub]}}


     \begin{oframed}
     $xyz$空間において,条件
          \begin{align*}
               \begin{array}{lll}
               x^2+y^2\le z^2&z^2\le x&0\le z\le1
               \end{array}
          \end{align*}
     を満たす点$P(x,y,z)$の全体からなる立体を考える.この立体の体積を$V$とし,$0\le k\le 1$に対し,$z$軸と直交する平面
     $z=k$による切り口の面積を$S(k)$とする.
          \begin{enumerate}[(1)]
          \item $k=\cos\theta$とおくとき$S(k)$を$\theta$で表せ.ただし$0\le\theta\pi/2$とする.
          \item $V$の値を求めよ.
          \end{enumerate}
     \end{oframed}

\setlength{\columnseprule}{0.4pt}
\begin{multicols}{2}
{\bf[解]} \1 $z=c$での切断面は,
     \begin{align*}
          \begin{cases}
          x^2+y^2\le c^2 \\
          c^2\le x
          \end{cases}
     \end{align*}
である.従って図示して下図.
     \begin{center}
     \scalebox{.7}{%WinTpicVersion4.32a
{\unitlength 0.1in%
\begin{picture}(16.2000,28.0000)(7.8000,-32.0000)%
% STR 2 0 3 0 Black White  
% 4 990 1797 990 1810 4 800 0 0
% O
\put(9.9000,-18.1000){\makebox(0,0)[rt]{O}}%
% STR 2 0 3 0 Black White  
% 4 960 387 960 400 4 800 0 0
% $y$
\put(9.6000,-4.0000){\makebox(0,0)[rt]{$y$}}%
% STR 2 0 3 0 Black White  
% 4 2400 1827 2400 1840 4 800 0 0
% $x$
\put(24.0000,-18.4000){\makebox(0,0)[rt]{$x$}}%
% VECTOR 2 0 3 0 Black White  
% 2 1000 3200 1000 400
% 
\special{pn 8}%
\special{pa 1000 3200}%
\special{pa 1000 400}%
\special{fp}%
\special{sh 1}%
\special{pa 1000 400}%
\special{pa 980 467}%
\special{pa 1000 453}%
\special{pa 1020 467}%
\special{pa 1000 400}%
\special{fp}%
% VECTOR 2 0 3 0 Black White  
% 2 800 1800 2400 1800
% 
\special{pn 8}%
\special{pa 800 1800}%
\special{pa 2400 1800}%
\special{fp}%
\special{sh 1}%
\special{pa 2400 1800}%
\special{pa 2333 1780}%
\special{pa 2347 1800}%
\special{pa 2333 1820}%
\special{pa 2400 1800}%
\special{fp}%
% FUNC 2 0 3 0 Black White  
% 9 800 400 2400 3200 1000 1800 2200 1800 1000 600 800 400 2400 3200 50 4 0 2
% cos(t)///sin(t)///-pi/2///pi/2
\special{pn 8}%
\special{pa 1000 3000}%
\special{pa 1030 3000}%
\special{pa 1038 2999}%
\special{pa 1053 2999}%
\special{pa 1060 2998}%
\special{pa 1075 2998}%
\special{pa 1083 2997}%
\special{pa 1090 2997}%
\special{pa 1098 2996}%
\special{pa 1105 2995}%
\special{pa 1113 2995}%
\special{pa 1121 2994}%
\special{pa 1128 2993}%
\special{pa 1136 2992}%
\special{pa 1143 2991}%
\special{pa 1150 2991}%
\special{pa 1158 2990}%
\special{pa 1165 2989}%
\special{pa 1173 2987}%
\special{pa 1180 2986}%
\special{pa 1188 2985}%
\special{pa 1195 2984}%
\special{pa 1203 2983}%
\special{pa 1210 2981}%
\special{pa 1218 2980}%
\special{pa 1225 2979}%
\special{pa 1232 2977}%
\special{pa 1240 2976}%
\special{pa 1247 2974}%
\special{pa 1254 2973}%
\special{pa 1262 2971}%
\special{pa 1269 2969}%
\special{pa 1277 2968}%
\special{pa 1291 2964}%
\special{pa 1299 2962}%
\special{pa 1320 2956}%
\special{pa 1328 2954}%
\special{pa 1349 2948}%
\special{pa 1357 2946}%
\special{pa 1364 2944}%
\special{pa 1371 2941}%
\special{pa 1378 2939}%
\special{pa 1385 2936}%
\special{pa 1399 2932}%
\special{pa 1407 2929}%
\special{pa 1414 2926}%
\special{pa 1421 2924}%
\special{pa 1435 2918}%
\special{pa 1442 2916}%
\special{pa 1484 2898}%
\special{pa 1490 2895}%
\special{pa 1518 2883}%
\special{pa 1525 2879}%
\special{pa 1531 2876}%
\special{pa 1538 2873}%
\special{pa 1545 2869}%
\special{pa 1552 2866}%
\special{pa 1558 2862}%
\special{pa 1565 2859}%
\special{pa 1572 2855}%
\special{pa 1578 2852}%
\special{pa 1585 2848}%
\special{pa 1591 2844}%
\special{pa 1598 2840}%
\special{pa 1604 2837}%
\special{pa 1611 2833}%
\special{pa 1617 2829}%
\special{pa 1624 2825}%
\special{pa 1630 2821}%
\special{pa 1637 2817}%
\special{pa 1649 2809}%
\special{pa 1656 2805}%
\special{pa 1668 2797}%
\special{pa 1675 2792}%
\special{pa 1693 2780}%
\special{pa 1699 2775}%
\special{pa 1705 2771}%
\special{pa 1711 2766}%
\special{pa 1718 2762}%
\special{pa 1724 2757}%
\special{pa 1730 2753}%
\special{pa 1736 2748}%
\special{pa 1741 2743}%
\special{pa 1747 2739}%
\special{pa 1759 2729}%
\special{pa 1765 2725}%
\special{pa 1777 2715}%
\special{pa 1782 2710}%
\special{pa 1794 2700}%
\special{pa 1799 2695}%
\special{pa 1805 2690}%
\special{pa 1810 2685}%
\special{pa 1822 2675}%
\special{pa 1827 2670}%
\special{pa 1832 2664}%
\special{pa 1838 2659}%
\special{pa 1859 2638}%
\special{pa 1864 2632}%
\special{pa 1870 2627}%
\special{pa 1875 2621}%
\special{pa 1880 2616}%
\special{pa 1885 2610}%
\special{pa 1890 2605}%
\special{pa 1895 2599}%
\special{pa 1900 2594}%
\special{pa 1915 2576}%
\special{pa 1920 2571}%
\special{pa 1925 2565}%
\special{pa 1929 2559}%
\special{pa 1944 2541}%
\special{pa 1948 2535}%
\special{pa 1953 2529}%
\special{pa 1957 2523}%
\special{pa 1962 2517}%
\special{pa 1966 2511}%
\special{pa 1971 2505}%
\special{pa 1975 2499}%
\special{pa 1980 2493}%
\special{pa 1988 2481}%
\special{pa 1993 2474}%
\special{pa 2005 2456}%
\special{pa 2009 2449}%
\special{pa 2017 2437}%
\special{pa 2021 2430}%
\special{pa 2025 2424}%
\special{pa 2029 2417}%
\special{pa 2033 2411}%
\special{pa 2037 2404}%
\special{pa 2041 2398}%
\special{pa 2044 2391}%
\special{pa 2048 2385}%
\special{pa 2052 2378}%
\special{pa 2055 2371}%
\special{pa 2059 2365}%
\special{pa 2062 2358}%
\special{pa 2066 2351}%
\special{pa 2069 2345}%
\special{pa 2073 2338}%
\special{pa 2079 2324}%
\special{pa 2083 2318}%
\special{pa 2098 2283}%
\special{pa 2101 2277}%
\special{pa 2119 2235}%
\special{pa 2121 2228}%
\special{pa 2127 2214}%
\special{pa 2129 2206}%
\special{pa 2132 2199}%
\special{pa 2134 2192}%
\special{pa 2137 2185}%
\special{pa 2141 2171}%
\special{pa 2144 2164}%
\special{pa 2146 2156}%
\special{pa 2152 2135}%
\special{pa 2154 2127}%
\special{pa 2160 2106}%
\special{pa 2162 2098}%
\special{pa 2166 2084}%
\special{pa 2168 2076}%
\special{pa 2169 2069}%
\special{pa 2171 2062}%
\special{pa 2173 2054}%
\special{pa 2174 2047}%
\special{pa 2176 2040}%
\special{pa 2177 2032}%
\special{pa 2179 2025}%
\special{pa 2180 2017}%
\special{pa 2181 2010}%
\special{pa 2183 2003}%
\special{pa 2184 1995}%
\special{pa 2185 1988}%
\special{pa 2186 1980}%
\special{pa 2188 1973}%
\special{pa 2189 1965}%
\special{pa 2190 1958}%
\special{pa 2191 1950}%
\special{pa 2191 1943}%
\special{pa 2192 1935}%
\special{pa 2193 1928}%
\special{pa 2194 1920}%
\special{pa 2195 1913}%
\special{pa 2195 1905}%
\special{pa 2196 1898}%
\special{pa 2197 1890}%
\special{pa 2197 1883}%
\special{pa 2198 1875}%
\special{pa 2198 1860}%
\special{pa 2199 1853}%
\special{pa 2199 1838}%
\special{pa 2200 1830}%
\special{pa 2200 1770}%
\special{pa 2199 1762}%
\special{pa 2199 1747}%
\special{pa 2198 1740}%
\special{pa 2198 1725}%
\special{pa 2197 1717}%
\special{pa 2197 1710}%
\special{pa 2196 1702}%
\special{pa 2195 1695}%
\special{pa 2195 1687}%
\special{pa 2194 1679}%
\special{pa 2193 1672}%
\special{pa 2192 1664}%
\special{pa 2191 1657}%
\special{pa 2191 1650}%
\special{pa 2190 1642}%
\special{pa 2189 1635}%
\special{pa 2187 1627}%
\special{pa 2186 1620}%
\special{pa 2185 1612}%
\special{pa 2184 1605}%
\special{pa 2183 1597}%
\special{pa 2181 1590}%
\special{pa 2180 1582}%
\special{pa 2179 1575}%
\special{pa 2177 1568}%
\special{pa 2176 1560}%
\special{pa 2174 1553}%
\special{pa 2173 1546}%
\special{pa 2171 1538}%
\special{pa 2169 1531}%
\special{pa 2168 1523}%
\special{pa 2164 1509}%
\special{pa 2162 1501}%
\special{pa 2156 1480}%
\special{pa 2154 1472}%
\special{pa 2148 1451}%
\special{pa 2146 1443}%
\special{pa 2144 1436}%
\special{pa 2141 1429}%
\special{pa 2139 1422}%
\special{pa 2136 1415}%
\special{pa 2132 1401}%
\special{pa 2129 1393}%
\special{pa 2126 1386}%
\special{pa 2124 1379}%
\special{pa 2118 1365}%
\special{pa 2116 1358}%
\special{pa 2098 1316}%
\special{pa 2095 1310}%
\special{pa 2083 1282}%
\special{pa 2079 1275}%
\special{pa 2076 1269}%
\special{pa 2073 1262}%
\special{pa 2069 1255}%
\special{pa 2066 1248}%
\special{pa 2062 1242}%
\special{pa 2059 1235}%
\special{pa 2055 1228}%
\special{pa 2052 1222}%
\special{pa 2048 1215}%
\special{pa 2044 1209}%
\special{pa 2040 1202}%
\special{pa 2037 1196}%
\special{pa 2033 1189}%
\special{pa 2029 1183}%
\special{pa 2025 1176}%
\special{pa 2021 1170}%
\special{pa 2017 1163}%
\special{pa 2009 1151}%
\special{pa 2005 1144}%
\special{pa 1997 1132}%
\special{pa 1992 1125}%
\special{pa 1980 1107}%
\special{pa 1975 1101}%
\special{pa 1971 1095}%
\special{pa 1966 1089}%
\special{pa 1962 1082}%
\special{pa 1957 1076}%
\special{pa 1953 1070}%
\special{pa 1948 1064}%
\special{pa 1943 1059}%
\special{pa 1939 1053}%
\special{pa 1929 1041}%
\special{pa 1925 1035}%
\special{pa 1915 1023}%
\special{pa 1910 1018}%
\special{pa 1900 1006}%
\special{pa 1895 1001}%
\special{pa 1890 995}%
\special{pa 1885 990}%
\special{pa 1875 978}%
\special{pa 1870 973}%
\special{pa 1864 968}%
\special{pa 1859 962}%
\special{pa 1838 941}%
\special{pa 1832 936}%
\special{pa 1827 930}%
\special{pa 1821 925}%
\special{pa 1816 920}%
\special{pa 1810 915}%
\special{pa 1805 910}%
\special{pa 1799 905}%
\special{pa 1794 900}%
\special{pa 1776 885}%
\special{pa 1771 880}%
\special{pa 1765 875}%
\special{pa 1759 871}%
\special{pa 1741 856}%
\special{pa 1735 852}%
\special{pa 1729 847}%
\special{pa 1723 843}%
\special{pa 1717 838}%
\special{pa 1711 834}%
\special{pa 1705 829}%
\special{pa 1699 825}%
\special{pa 1693 820}%
\special{pa 1681 812}%
\special{pa 1674 807}%
\special{pa 1656 795}%
\special{pa 1649 791}%
\special{pa 1637 783}%
\special{pa 1630 779}%
\special{pa 1624 775}%
\special{pa 1617 771}%
\special{pa 1611 767}%
\special{pa 1604 763}%
\special{pa 1598 759}%
\special{pa 1591 756}%
\special{pa 1585 752}%
\special{pa 1578 748}%
\special{pa 1571 745}%
\special{pa 1565 741}%
\special{pa 1558 738}%
\special{pa 1551 734}%
\special{pa 1545 731}%
\special{pa 1538 727}%
\special{pa 1524 721}%
\special{pa 1518 717}%
\special{pa 1483 702}%
\special{pa 1477 699}%
\special{pa 1435 681}%
\special{pa 1428 679}%
\special{pa 1414 673}%
\special{pa 1406 671}%
\special{pa 1399 668}%
\special{pa 1392 666}%
\special{pa 1385 663}%
\special{pa 1371 659}%
\special{pa 1364 656}%
\special{pa 1356 654}%
\special{pa 1335 648}%
\special{pa 1327 646}%
\special{pa 1306 640}%
\special{pa 1298 638}%
\special{pa 1284 634}%
\special{pa 1276 632}%
\special{pa 1269 631}%
\special{pa 1262 629}%
\special{pa 1254 627}%
\special{pa 1247 626}%
\special{pa 1240 624}%
\special{pa 1232 623}%
\special{pa 1225 621}%
\special{pa 1217 620}%
\special{pa 1210 619}%
\special{pa 1203 617}%
\special{pa 1195 616}%
\special{pa 1188 615}%
\special{pa 1180 614}%
\special{pa 1173 612}%
\special{pa 1165 611}%
\special{pa 1158 610}%
\special{pa 1150 609}%
\special{pa 1143 609}%
\special{pa 1135 608}%
\special{pa 1128 607}%
\special{pa 1120 606}%
\special{pa 1113 605}%
\special{pa 1105 605}%
\special{pa 1098 604}%
\special{pa 1090 603}%
\special{pa 1083 603}%
\special{pa 1075 602}%
\special{pa 1060 602}%
\special{pa 1053 601}%
\special{pa 1038 601}%
\special{pa 1030 600}%
\special{pa 1000 600}%
\special{fp}%
% FUNC 2 0 3 0 Black White  
% 9 800 400 2400 3200 1000 1800 2200 1800 1000 600 800 401 2400 3199 0 2 0 1
% 0.5
\special{pn 8}%
\special{pa 1600 400}%
\special{pa 1600 3195}%
\special{fp}%
\special{pn 8}%
% LINE 2 2 3 0 Black White  
% 2 1000 1800 1600 780
% 
\special{pn 8}%
\special{pa 1000 1800}%
\special{pa 1600 780}%
\special{dt 0.045}%
% LINE 3 0 3 0 Black White  
% 54 2040 1200 1600 1640 2060 1240 1600 1700 2080 1280 1600 1760 2100 1320 1620 1800 2120 1360 1680 1800 2130 1410 1740 1800 2150 1450 1800 1800 2160 1500 1860 1800 2170 1550 1920 1800 2180 1600 1980 1800 2190 1650 2040 1800 2200 1700 2100 1800 2200 1760 2160 1800 2010 1170 1600 1580 1990 1130 1600 1520 1960 1100 1600 1460 1940 1060 1600 1400 1910 1030 1600 1340 1890 990 1600 1280 1860 960 1600 1220 1830 930 1600 1160 1790 910 1600 1100 1760 880 1600 1040 1730 850 1600 980 1690 830 1600 920 1660 800 1600 860 1620 780 1600 800
% 
\special{pn 4}%
\special{pa 2040 1200}%
\special{pa 1600 1640}%
\special{fp}%
\special{pa 2060 1240}%
\special{pa 1600 1700}%
\special{fp}%
\special{pa 2080 1280}%
\special{pa 1600 1760}%
\special{fp}%
\special{pa 2100 1320}%
\special{pa 1620 1800}%
\special{fp}%
\special{pa 2120 1360}%
\special{pa 1680 1800}%
\special{fp}%
\special{pa 2130 1410}%
\special{pa 1740 1800}%
\special{fp}%
\special{pa 2150 1450}%
\special{pa 1800 1800}%
\special{fp}%
\special{pa 2160 1500}%
\special{pa 1860 1800}%
\special{fp}%
\special{pa 2170 1550}%
\special{pa 1920 1800}%
\special{fp}%
\special{pa 2180 1600}%
\special{pa 1980 1800}%
\special{fp}%
\special{pa 2190 1650}%
\special{pa 2040 1800}%
\special{fp}%
\special{pa 2200 1700}%
\special{pa 2100 1800}%
\special{fp}%
\special{pa 2200 1760}%
\special{pa 2160 1800}%
\special{fp}%
\special{pa 2010 1170}%
\special{pa 1600 1580}%
\special{fp}%
\special{pa 1990 1130}%
\special{pa 1600 1520}%
\special{fp}%
\special{pa 1960 1100}%
\special{pa 1600 1460}%
\special{fp}%
\special{pa 1940 1060}%
\special{pa 1600 1400}%
\special{fp}%
\special{pa 1910 1030}%
\special{pa 1600 1340}%
\special{fp}%
\special{pa 1890 990}%
\special{pa 1600 1280}%
\special{fp}%
\special{pa 1860 960}%
\special{pa 1600 1220}%
\special{fp}%
\special{pa 1830 930}%
\special{pa 1600 1160}%
\special{fp}%
\special{pa 1790 910}%
\special{pa 1600 1100}%
\special{fp}%
\special{pa 1760 880}%
\special{pa 1600 1040}%
\special{fp}%
\special{pa 1730 850}%
\special{pa 1600 980}%
\special{fp}%
\special{pa 1690 830}%
\special{pa 1600 920}%
\special{fp}%
\special{pa 1660 800}%
\special{pa 1600 860}%
\special{fp}%
\special{pa 1620 780}%
\special{pa 1600 800}%
\special{fp}%
% LINE 3 0 3 0 Black White  
% 34 2160 1800 1600 2360 2200 1820 1600 2420 2200 1880 1600 2480 2190 1950 1600 2540 2180 2020 1600 2600 2160 2100 1600 2660 2140 2180 1600 2720 2100 2280 1600 2780 2020 2420 1620 2820 2100 1800 1600 2300 2040 1800 1600 2240 1980 1800 1600 2180 1920 1800 1600 2120 1860 1800 1600 2060 1800 1800 1600 2000 1740 1800 1600 1940 1680 1800 1600 1880
% 
\special{pn 4}%
\special{pa 2160 1800}%
\special{pa 1600 2360}%
\special{fp}%
\special{pa 2200 1820}%
\special{pa 1600 2420}%
\special{fp}%
\special{pa 2200 1880}%
\special{pa 1600 2480}%
\special{fp}%
\special{pa 2190 1950}%
\special{pa 1600 2540}%
\special{fp}%
\special{pa 2180 2020}%
\special{pa 1600 2600}%
\special{fp}%
\special{pa 2160 2100}%
\special{pa 1600 2660}%
\special{fp}%
\special{pa 2140 2180}%
\special{pa 1600 2720}%
\special{fp}%
\special{pa 2100 2280}%
\special{pa 1600 2780}%
\special{fp}%
\special{pa 2020 2420}%
\special{pa 1620 2820}%
\special{fp}%
\special{pa 2100 1800}%
\special{pa 1600 2300}%
\special{fp}%
\special{pa 2040 1800}%
\special{pa 1600 2240}%
\special{fp}%
\special{pa 1980 1800}%
\special{pa 1600 2180}%
\special{fp}%
\special{pa 1920 1800}%
\special{pa 1600 2120}%
\special{fp}%
\special{pa 1860 1800}%
\special{pa 1600 2060}%
\special{fp}%
\special{pa 1800 1800}%
\special{pa 1600 2000}%
\special{fp}%
\special{pa 1740 1800}%
\special{pa 1600 1940}%
\special{fp}%
\special{pa 1680 1800}%
\special{pa 1600 1880}%
\special{fp}%
% STR 2 0 3 0 Black White  
% 4 2200 1700 2200 1800 2 0 1 0
% $c$
\put(22.0000,-18.0000){\makebox(0,0)[lb]{{\colorbox[named]{White}{$c$}}}}%
% STR 2 0 3 0 Black White  
% 4 1600 1700 1600 1800 2 0 1 0
% $c^2$
\put(16.0000,-18.0000){\makebox(0,0)[lb]{{\colorbox[named]{White}{$c^2$}}}}%
% STR 2 0 3 0 Black White  
% 4 1200 1610 1200 1710 5 0 1 0
% $\theta$
\put(12.0000,-17.1000){\makebox(0,0){{\colorbox[named]{White}{$\theta$}}}}%
\end{picture}}%
}
     \end{center}

$S(k)$は斜線部の面積で
     \begin{align*}
     S(k)=\theta c^2-c^3s
     \end{align*}
である.$\cdots$((1)の答) \\
従って,求める体積$V$は
     \begin{align*}
     V&=\int_0^1S(k)dk \\ 
     &=\int_{\pi/2}^0(\theta c^2-c^3s)\frac{dk}{d\theta}d\theta\\
     &=\int_0^{\pi/2}(c^2s\theta-c^3s^2)d\theta\\
     &=-\left[\frac{1}{3}c^3\theta \right]_0^{\pi/2}+\frac{1}{3}\int_0^{\pi/2}c^3d\theta-\int_0^{\pi/2}c^3s^2d\theta \\
     &=\int_0^{\pi/2}c(1-s^2)\left(\frac{1}{3}-s^2\right)d\theta \\
     &=\int_0^1(1-x^2)\left(\frac{1}{3}-x^2\right)dx \\
     &=\left[\frac{1}{5}x^5-\frac{4}{9}x^3+\frac{1}{3}x\right]_0^1 \\
     &=\frac{4}{45}
     \end{align*}
である.$\cdots$((2)の答)
  
\newpage
\end{multicols}
\end{document}