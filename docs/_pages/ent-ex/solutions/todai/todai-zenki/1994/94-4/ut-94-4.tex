\documentclass[a4j]{jarticle}
\usepackage{amsmath}
\usepackage{ascmac}
\usepackage{amssymb}
\usepackage{enumerate}
\usepackage{multicol}
\usepackage{framed}
\usepackage{latexsym}
\usepackage{indent}
\usepackage{cases}
\title{}
\begin{document}
%分数関係


\def\tfrac#1#2{{\textstyle\frac{#1}{#2}}} %数式中で文中表示の分数を使う時


%Σ関係

\def\dsum#1#2{{\displaystyle\sum_{#1}^{#2}}} %文中で数式表示のΣを使う時


%ベクトル関係


\def\vector#1{\overrightarrow{#1}}  %ベクトルを表現したいとき(aベクトルを表現するときは\ver
\def\norm#1{|\overrightarrow{#1}|} %ベクトルの絶対値
\def\vtwo#1#2{ \left(%
      \begin{array}{c}%
      #1 \\%
      #2 \\%
      \end{array}%
      \right) }                        %2次元ベクトル成分表示
      
      \def\vthree#1#2#3{ \left(
      \begin{array}{c}
      #1 \\
      #2 \\
      #3 \\
      \end{array}
      \right) }                        %3次元ベクトル成分表示



%数列関係


\def\an#1{\verb|{|$#1$\verb|}|}


%極限関係

\def\limit#1#2{\stackrel{#1 \to #2}{\longrightarrow}}   %等式変形からの極限
\def\dlim#1#2{{\displaystyle \lim_{#1\to#2}}} %文中で数式表示の極限を使う



%積分関係

\def\dint#1#2{{\displaystyle \int_{#1}^{#2}}} %文中で数式表示の積分を使う時

\def\ne{\nearrow}
\def\se{\searrow}
\def\nw{\nwarrow}
\def\ne{\nearrow}


%便利なやつ

\def\case#1#2{%
 \[\left\{%
 \begin{array}{l}%
 #1 \\%
 #2%
 \end{array}%
 \right.\] }                           %場合分け
 
\def\1{$\cos\theta=c$,$\sin\theta=s$とおく.}  %cs表示を与える前書きシータ
\def\2{$\cos t=c$,$\sin t=s$とおく.}     %cs表示を与える前書きt
\def\3{$\cos x=c$,$\sin x=s$とおく.}                %cs表示を与える前書きx

\def\fig#1#2#3 {%
\begin{wrapfigure}[#1]{r}{#2 zw}%
\vspace*{-1zh}%
\input{#3}%
\end{wrapfigure} }           %絵の挿入


\def\a{\alpha}   %ギリシャ文字
\def\b{\beta}
\def\g{\gamma}

%問題番号のためのマクロ

\newcounter{nombre} %必須
\renewcommand{\thenombre}{\arabic{nombre}} %任意
\setcounter{nombre}{2} %任意
\newcounter{nombresub}[nombre] %親子関係を定義
\renewcommand{\thenombresub}{\arabic{nombresub}} %任意
\setcounter{nombresub}{0} %任意
\newcommand{\prob}[1][]{\refstepcounter{nombre}#1[問題 \thenombre]}
\newcommand{\probsub}[1][]{\refstepcounter{nombresub}#1(\thenombresub)}


%1-1みたいなカウンタ(todaiとtodaia)
\newcounter{todai}
\setcounter{todai}{0}
\newcounter{todaisub}[todai] 
\setcounter{todaisub}{0} 
\newcommand{\todai}[1][]{\refstepcounter{todai}#1 \thetodai-\thetodaisub}
\newcommand{\todaib}[1][]{\refstepcounter{todai}#1\refstepcounter{todaisub}#1 {\bf [問題 \thetodai.\thetodaisub]}}
\newcommand{\todaia}[1][]{\refstepcounter{todaisub}#1 {\bf [問題 \thetodai.\thetodaisub]}}


\begin{oframed}
$0<c<1$とする.$0\le x<1$において連続な関数$f(x)$に対して
     \begin{align*}
     f_1(x)=f(x)+\int_0^cf(t)dt , f_2(x)=f(x)+\int_0^cf_1(t)dt
     \end{align*}
とおく.以下,関数$f_3(x),f_4(x),\cdots$を順次
     \begin{align*}
     f_n(x)=f(x)+\int_0^cf_{n-1}(t)dt \ \ \ \ \ (n=3,4,\cdots)
     \end{align*}
により定める.また,
     \begin{align*}
     g(c)=\int_0^cf(t)dt
     \end{align*}
とし,$n=1,2,3,\cdots$に対し          
     \begin{align*}
     g_n(c)=\int_0^cf_n(t)dt
     \end{align*}
とおく.このとき,$0<x<1$を満たす任意の$x$に対し$xf(x)=g(x)+x\dlim{n}{\infty}g_n(x)$が成り立ち,さらに$f(0)=1$となるような$f(x)$を定めよ.     
\end{oframed}

\setlength{\columnseprule}{0.4pt}
\begin{multicols}{2}
{\bf[解]}題意より$n\in\mathbb{N}$に対し,
     \[f_{n+1}(x)=f(x)+g_n(c)\]
であるから,漸化式に代入して
     \begin{align}
     g_{n+1}(c)&=\int_0^c\{f(x)+g_n(c)\}dt\nonumber \\
     &=cg_n(c)+g(c) \nonumber \\
     \therefore g_{n+1}(c)-\frac{g(c)}{1-c}&=c\left\{ g_{n}(c)-\frac{g(c)}{1-c}\right\} \label{1}
     \end{align}
これは数列$\{ g_n(c)-\dfrac{g(c)}{1-c}\}$が公比$c$の等比数列であることをあらわす.
また,初期条件について,漸化式より
     \begin{align}
     g_1(c)&=\int_0^c\left\{f(t)+g(c)\right\}dt\nonumber \\
     &=(1+c)g(c) \label{2}
     \end{align}
であるから,\eqref{1},\eqref{2}から
     \begin{align}
     g_n(c)&=c^{n-1}\left\{(1+c)g(c)-\frac{g(c)}{1-c}\right\}+\frac{g(c)}{1-c} \nonumber \\
     &\limit{n}{\infty} \frac{g(c)}{1-c}
     \end{align}     
を得る$(\because0<c<1)$.故に
     \begin{subnumcases}
     {}
     xf(x)=g(x)+x\frac{g(x)}{1-x}  \label{4}\\
     f(0)=1 \label{5}
     \end{subnumcases}
をみたす$f(x)$をみつければよい.\eqref{4}で$0<x<1$だから,分母を払って
     \begin{align}
     (1-x)xf(x)=g(x) \label{6}
     \end{align}   
$g(x)=y$とおけば$f(x)=dy/dx$である.$y=0$なる$x$があれば$f(x)\equiv0$となって
\eqref{5}に反することから$y\not=0$である.
従って\eqref{6}を書き換えると
     \begin{align}
     \frac{dy}{y}&=\frac{dx}{x(1-x)}\nonumber \\
     y&=C\frac{x}{1-x}  \nonumber\\
     \therefore f(x)&=C\frac{1}{(1-x)^2} \label{7}
     \end{align}       
であり,\eqref{5}および$f(x)$の連続性から$\dlim{x}{+0}f(x)=1$であるので
     \begin{align*}
     \dlim{x}{+0}f(x)=C=1
     \end{align*}
となる.これを\eqref{7}に代入して
     \begin{align*}
     f(x)=\frac{1}{(1-x)^2}\cdots\text{(答)}
     \end{align*}
である.    
 \\
 \\
{\bf[解注]}問題文から$g$を関数と見たければ(すなわち$g_n(x)$が出現するということは)
$c$は与えられた定数ではなく,$0<c<1$で動く変数と見るべきである.

$f_n(x)$自体に$c$が含まれているが,こちらでは変数は$x$のみであり,$c$は定数扱いなのに対し,$g_n(c)$ではその$c$を変数扱いにしたいので,文字を変えたものと思われる.     
\newpage
\end{multicols}
\end{document}