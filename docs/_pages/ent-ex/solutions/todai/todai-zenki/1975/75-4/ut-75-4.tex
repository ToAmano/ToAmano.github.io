\documentclass[a4j]{jarticle}
\usepackage{amsmath}
\usepackage{ascmac}
\usepackage{amssymb}
\usepackage{enumerate}
\usepackage{multicol}
\usepackage{framed}
\usepackage{fancyhdr}
\usepackage{latexsym}
\usepackage{indent}
\usepackage{cases}
\allowdisplaybreaks
\pagestyle{fancy}
\lhead{}
\chead{}
\rhead{東京大学前期$1975$年$4$番}
\begin{document}
%分数関係


\def\tfrac#1#2{{\textstyle\frac{#1}{#2}}} %数式中で文中表示の分数を使う時


%Σ関係

\def\dsum#1#2{{\displaystyle\sum_{#1}^{#2}}} %文中で数式表示のΣを使う時


%ベクトル関係


\def\vector#1{\overrightarrow{#1}}  %ベクトルを表現したいとき(aベクトルを表現するときは\ver
\def\norm#1{|\overrightarrow{#1}|} %ベクトルの絶対値
\def\vtwo#1#2{ \left(%
      \begin{array}{c}%
      #1 \\%
      #2 \\%
      \end{array}%
      \right) }                        %2次元ベクトル成分表示
      
      \def\vthree#1#2#3{ \left(
      \begin{array}{c}
      #1 \\
      #2 \\
      #3 \\
      \end{array}
      \right) }                        %3次元ベクトル成分表示



%数列関係


\def\an#1{\verb|{|$#1$\verb|}|}


%極限関係

\def\limit#1#2{\stackrel{#1 \to #2}{\longrightarrow}}   %等式変形からの極限
\def\dlim#1#2{{\displaystyle \lim_{#1\to#2}}} %文中で数式表示の極限を使う



%積分関係

\def\dint#1#2{{\displaystyle \int_{#1}^{#2}}} %文中で数式表示の積分を使う時

\def\ne{\nearrow}
\def\se{\searrow}
\def\nw{\nwarrow}
\def\ne{\nearrow}


%便利なやつ

\def\case#1#2{%
 \[\left\{%
 \begin{array}{l}%
 #1 \\%
 #2%
 \end{array}%
 \right.\] }                           %場合分け
 
\def\1{$\cos\theta=c$,$\sin\theta=s$とおく.}  %cs表示を与える前書きシータ
\def\2{$\cos t=c$,$\sin t=s$とおく.}     %cs表示を与える前書きt
\def\3{$\cos x=c$,$\sin x=s$とおく.}                %cs表示を与える前書きx

\def\fig#1#2#3 {%
\begin{wrapfigure}[#1]{r}{#2 zw}%
\vspace*{-1zh}%
\input{#3}%
\end{wrapfigure} }           %絵の挿入


\def\a{\alpha}   %ギリシャ文字
\def\b{\beta}
\def\g{\gamma}

%問題番号のためのマクロ

\newcounter{nombre} %必須
\renewcommand{\thenombre}{\arabic{nombre}} %任意
\setcounter{nombre}{2} %任意
\newcounter{nombresub}[nombre] %親子関係を定義
\renewcommand{\thenombresub}{\arabic{nombresub}} %任意
\setcounter{nombresub}{0} %任意
\newcommand{\prob}[1][]{\refstepcounter{nombre}#1[問題 \thenombre]}
\newcommand{\probsub}[1][]{\refstepcounter{nombresub}#1(\thenombresub)}


%1-1みたいなカウンタ(todaiとtodaia)
\newcounter{todai}
\setcounter{todai}{0}
\newcounter{todaisub}[todai] 
\setcounter{todaisub}{0} 
\newcommand{\todai}[1][]{\refstepcounter{todai}#1 \thetodai-\thetodaisub}
\newcommand{\todaib}[1][]{\refstepcounter{todai}#1\refstepcounter{todaisub}#1 {\bf [問題 \thetodai.\thetodaisub]}}
\newcommand{\todaia}[1][]{\refstepcounter{todaisub}#1 {\bf [問題 \thetodai.\thetodaisub]}}


\begin{oframed}
数列$\{a_n\}$の項が$a_1=\sqrt{2}$,$a_{n+1}=\sqrt{2+a_n}$によって与えられているものとする.
このとき$a_n=2\sin \theta_n$,$0<\theta_n<\dfrac{\pi}{2}$を満たす$\theta_n$を見いだせ.また
$\lim_{n\to\infty}\theta_n$をもとめよ.
\end{oframed}

\setlength{\columnseprule}{0.4pt}
\begin{multicols}{2}
{\bf[解]}漸化式から
     \begin{align*}
     2\sin \theta_{n+1}&=\sqrt{2+2\sin \theta_n} \\
     &=\sqrt{2+2\cos \left(\frac{\pi}{2}-\theta_n\right)} \\
     &=\sqrt{4\cos^2\left(\frac{\pi}{4}-\frac{\theta_n}{2}\right)} \\
     &=2\cos\left(\frac{\pi}{4}-\frac{\theta_n}{2}\right) \\ 
     &=2\sin\left(\frac{\pi}{4}+\frac{\theta_n}{2}\right)
     \end{align*}
ただし,変形途中で,$0<\theta_n<\dfrac{\pi}{2}$により
$\cos\left(\dfrac{\pi}{4}-\dfrac{\theta_n}{2}\right)>0$であることに注意した.
だから$0<\theta_n<\dfrac{\pi}{2}$とあわせて
\[\theta_{n+1}=\frac{\pi}{4}+\frac{\theta_n}{2}\]
である.変形して
\[\theta_{n+1}-\frac{\pi}{2}=\frac{1}{2}\left(\theta_n-\frac{\pi}{2}\right) \]
だから,繰り返し用いて,初期条件$\theta_1=\dfrac{\pi}{4}$より
\[\theta_n=-\left(\frac{1}{2}\right)^{n-1}\frac{\pi}{2}+\frac{\pi}{2} \cdots\text{(答)} \]
故に求める極限値は$\lim_{n\to\infty}\theta_n=\dfrac{\pi}{2}\cdots$(答)である.


   

\newpage
\end{multicols}
\end{document}