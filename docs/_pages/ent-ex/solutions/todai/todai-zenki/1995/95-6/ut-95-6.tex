\documentclass[a4j]{jarticle}
\usepackage{amsmath}
\usepackage{ascmac}
\usepackage{amssymb}
\usepackage{enumerate}
\usepackage{multicol}
\usepackage{framed}
\usepackage{latexsym}
\begin{document}
%分数関係


\def\tfrac#1#2{{\textstyle\frac{#1}{#2}}} %数式中で文中表示の分数を使う時


%Σ関係

\def\dsum#1#2{{\displaystyle\sum_{#1}^{#2}}} %文中で数式表示のΣを使う時


%ベクトル関係


\def\vector#1{\overrightarrow{#1}}  %ベクトルを表現したいとき(aベクトルを表現するときは\ver
\def\norm#1{|\overrightarrow{#1}|} %ベクトルの絶対値
\def\vtwo#1#2{ \left(%
      \begin{array}{c}%
      #1 \\%
      #2 \\%
      \end{array}%
      \right) }                        %2次元ベクトル成分表示
      
      \def\vthree#1#2#3{ \left(
      \begin{array}{c}
      #1 \\
      #2 \\
      #3 \\
      \end{array}
      \right) }                        %3次元ベクトル成分表示



%数列関係


\def\an#1{\verb|{|$#1$\verb|}|}


%極限関係

\def\limit#1#2{\stackrel{#1 \to #2}{\longrightarrow}}   %等式変形からの極限
\def\dlim#1#2{{\displaystyle \lim_{#1\to#2}}} %文中で数式表示の極限を使う



%積分関係

\def\dint#1#2{{\displaystyle \int_{#1}^{#2}}} %文中で数式表示の積分を使う時

\def\ne{\nearrow}
\def\se{\searrow}
\def\nw{\nwarrow}
\def\ne{\nearrow}


%便利なやつ

\def\case#1#2{%
 \[\left\{%
 \begin{array}{l}%
 #1 \\%
 #2%
 \end{array}%
 \right.\] }                           %場合分け
 
\def\1{$\cos\theta=c$,$\sin\theta=s$とおく.}  %cs表示を与える前書きシータ
\def\2{$\cos t=c$,$\sin t=s$とおく.}     %cs表示を与える前書きt
\def\3{$\cos x=c$,$\sin x=s$とおく.}                %cs表示を与える前書きx

\def\fig#1#2#3 {%
\begin{wrapfigure}[#1]{r}{#2 zw}%
\vspace*{-1zh}%
\input{#3}%
\end{wrapfigure} }           %絵の挿入


\def\a{\alpha}   %ギリシャ文字
\def\b{\beta}
\def\g{\gamma}

%問題番号のためのマクロ

\newcounter{nombre} %必須
\renewcommand{\thenombre}{\arabic{nombre}} %任意
\setcounter{nombre}{2} %任意
\newcounter{nombresub}[nombre] %親子関係を定義
\renewcommand{\thenombresub}{\arabic{nombresub}} %任意
\setcounter{nombresub}{0} %任意
\newcommand{\prob}[1][]{\refstepcounter{nombre}#1[問題 \thenombre]}
\newcommand{\probsub}[1][]{\refstepcounter{nombresub}#1(\thenombresub)}


%1-1みたいなカウンタ(todaiとtodaia)
\newcounter{todai}
\setcounter{todai}{0}
\newcounter{todaisub}[todai] 
\setcounter{todaisub}{0} 
\newcommand{\todai}[1][]{\refstepcounter{todai}#1 \thetodai-\thetodaisub}
\newcommand{\todaib}[1][]{\refstepcounter{todai}#1\refstepcounter{todaisub}#1 {\bf [問題 \thetodai.\thetodaisub]}}
\newcommand{\todaia}[1][]{\refstepcounter{todaisub}#1 {\bf [問題 \thetodai.\thetodaisub]}}


\begin{oframed}
原点を$O$とする$xy$平面上の双曲線
     \[\frac{x^2}{a^2}-\frac{y^2}{b^2}=1 \ \ \ \ \ (a>0,b>0) \]
上の点$P$における接線と$2$つの漸近線との交点を$Q$,$R$とする.このとき以下の問いに答えよ.
     \begin{enumerate}[(1)]
     \item 三角形$QOR$の面積$S$は,点$P$の取り方にはよらず,$a$,$b$によって定まることを
     示せ.
     \item $a=5e^{2t}+e^{-t}$,$b=e^{2t}+e^{-t}$として実数$t$を変化させるときの$S$の最小値を求め
     よ.
     \end{enumerate}
\end{oframed}

\setlength{\columnseprule}{0.4pt}
\begin{multicols}{2}
{\bf[解]}
     \begin{enumerate}[(1)]
     \item $2$本の漸近線は$y=\pm\dfrac{b}{a}x$であり,$P(X,Y)$での双曲線の接線$l$は
          \[\frac{Xx}{a^2}-\frac{Yy}{b^2}=1 \]
     である.故にこれらの交点は
          \[A=\dfrac{X}{a}+\dfrac{Y}{b} , B=\dfrac{X}{a}-\dfrac{Y}{b} \]
     として
          \[ \left(\frac{a}{A},\frac{-b}{A}\right),\left(\frac{a}{B},\frac{b}{B}\right)\]
     である.したがってサラスの公式から
          \begin{align}
          S&=\frac{1}{2}\left|\frac{ab}{AB}+\frac{ab}{AB}\right|  \nonumber\\   
          &=\frac{ab}{AB}  \nonumber\\
          &=ab \ \ \ \left(\because AB=\frac{X^2}{a^2}-\frac{Y^2}{b^2}=1\right)  \label{1}
          \end{align}
     となる.これは$X$,$Y$によらない定数である.$\Box$     
     \item $p=e^t$とする.$t$が任意実数だから$p>0$である.\eqref{1}に値を代入して
          \begin{align}
          S&=(5p^2+p^{-1})(p^2+p^{-1})\nonumber \\
          &=5p^4+6p+p^{-2} \label{2}
          \end{align}
     であるから,
          \begin{align*}
          \frac{dS}{dp}&=20p^3+6=\frac{2}{p^3} \\
          &=\frac{2}{p^3}(5p^3-1)(2p^3+1) 
          \end{align*}
     となって,下表をうる.          
          \begin{align*}
               \begin{array}{|c|c|c|c|c|} \hline
               p & 0 &   & \left(\dfrac{1}{5}\right)^{1/3} &   \\ \hline
               S' &   & - & 0                          & + \\ \hline
               S  &   & \se&                          &\ne \\ \hline
               \end{array}
          \end{align*}
     したがって\eqref{2}とあわせて
          \begin{align*}
          \min S&=5\left(\frac{1}{5}\right)^{4/3}+6\left(\frac{1}{5}\right)^{1/3}+5^{2/3} \\
          &=7\left(\frac{1}{5}\right)^{1/3}+5^{2/3}\cdots\text{(答)}
          \end{align*}
     となる.     
     
          
               
     \end{enumerate}
\newpage
\end{multicols}
\end{document}