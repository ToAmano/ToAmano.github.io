\documentclass[a4j]{jarticle}
\usepackage{amsmath}
\usepackage{ascmac}
\usepackage{amssymb}
\usepackage{enumerate}
\usepackage{multicol}
\usepackage{framed}
\usepackage{fancyhdr}
\usepackage{latexsym}
\usepackage{indent}
\usepackage{cases}
\usepackage[dvips]{graphicx}
\usepackage{color}
\allowdisplaybreaks
\pagestyle{fancy}
\lhead{}
\chead{}
\rhead{東京大学前期$1991$年$6$番}
\begin{document}
%分数関係


\def\tfrac#1#2{{\textstyle\frac{#1}{#2}}} %数式中で文中表示の分数を使う時


%Σ関係

\def\dsum#1#2{{\displaystyle\sum_{#1}^{#2}}} %文中で数式表示のΣを使う時


%ベクトル関係


\def\vector#1{\overrightarrow{#1}}  %ベクトルを表現したいとき(aベクトルを表現するときは\ver
\def\norm#1{|\overrightarrow{#1}|} %ベクトルの絶対値
\def\vtwo#1#2{ \left(%
      \begin{array}{c}%
      #1 \\%
      #2 \\%
      \end{array}%
      \right) }                        %2次元ベクトル成分表示
      
      \def\vthree#1#2#3{ \left(
      \begin{array}{c}
      #1 \\
      #2 \\
      #3 \\
      \end{array}
      \right) }                        %3次元ベクトル成分表示



%数列関係


\def\an#1{\verb|{|$#1$\verb|}|}


%極限関係

\def\limit#1#2{\stackrel{#1 \to #2}{\longrightarrow}}   %等式変形からの極限
\def\dlim#1#2{{\displaystyle \lim_{#1\to#2}}} %文中で数式表示の極限を使う



%積分関係

\def\dint#1#2{{\displaystyle \int_{#1}^{#2}}} %文中で数式表示の積分を使う時

\def\ne{\nearrow}
\def\se{\searrow}
\def\nw{\nwarrow}
\def\ne{\nearrow}


%便利なやつ

\def\case#1#2{%
 \[\left\{%
 \begin{array}{l}%
 #1 \\%
 #2%
 \end{array}%
 \right.\] }                           %場合分け
 
\def\1{$\cos\theta=c$,$\sin\theta=s$とおく.}  %cs表示を与える前書きシータ
\def\2{$\cos t=c$,$\sin t=s$とおく.}     %cs表示を与える前書きt
\def\3{$\cos x=c$,$\sin x=s$とおく.}                %cs表示を与える前書きx

\def\fig#1#2#3 {%
\begin{wrapfigure}[#1]{r}{#2 zw}%
\vspace*{-1zh}%
\input{#3}%
\end{wrapfigure} }           %絵の挿入


\def\a{\alpha}   %ギリシャ文字
\def\b{\beta}
\def\g{\gamma}

%問題番号のためのマクロ

\newcounter{nombre} %必須
\renewcommand{\thenombre}{\arabic{nombre}} %任意
\setcounter{nombre}{2} %任意
\newcounter{nombresub}[nombre] %親子関係を定義
\renewcommand{\thenombresub}{\arabic{nombresub}} %任意
\setcounter{nombresub}{0} %任意
\newcommand{\prob}[1][]{\refstepcounter{nombre}#1[問題 \thenombre]}
\newcommand{\probsub}[1][]{\refstepcounter{nombresub}#1(\thenombresub)}


%1-1みたいなカウンタ(todaiとtodaia)
\newcounter{todai}
\setcounter{todai}{0}
\newcounter{todaisub}[todai] 
\setcounter{todaisub}{0} 
\newcommand{\todai}[1][]{\refstepcounter{todai}#1 \thetodai-\thetodaisub}
\newcommand{\todaib}[1][]{\refstepcounter{todai}#1\refstepcounter{todaisub}#1 {\bf [問題 \thetodai.\thetodaisub]}}
\newcommand{\todaia}[1][]{\refstepcounter{todaisub}#1 {\bf [問題 \thetodai.\thetodaisub]}}


     \begin{oframed}
     $f(x)$は$x>0$で定義された連続な関数で,$0<x_1<x_2$ならば,つねに$f(x_1)>f(x_2)>0$であるものとし,
     $S(x)=\dint{x}{2x}f(t)dt$とおく.このとき,$S(1)=1$であり,さらに任意の$a>0$に対して,原点と点$(a,f(a))$,原点と点
     $(2a,f(2a))$を結ぶ$2$直線と曲線$y=f(x)$とで囲まれる部分の面積は$3S(x)$に等しいものとする.
          \begin{enumerate}[(1)]
          \item $S(x)$,$f(x)-2f(2x)$をそれぞれ$x$の関数として表せ.
          \item $x>0$に対して,$a(x)=\dlim{n}{\infty}2^nf(2^nx)$とおく.積分$\dint{x}{2x}a(t)dt$を求めよ.
          \item 関数$f(x)$を決定せよ.
          \end{enumerate}
     \end{oframed}

\setlength{\columnseprule}{0.4pt}
\begin{multicols}{2}
{\bf[解]} 
     \begin{enumerate}[(1)]
     \item 題意から,$f(x)$は区間内で単調減少である.また,
          \begin{align}
          S(x)=\int_x^{2x}f(t)dt\label{0}
          \end{align} 
     である.グラフの概形は下図.
          \begin{center}
          \scalebox{0.7}{%WinTpicVersion4.32a
{\unitlength 0.1in%
\begin{picture}(28.2000,28.0000)(29.8000,-30.0000)%
% STR 2 0 3 0 Black White  
% 4 3190 2797 3190 2810 4 3000 0 0
% O
\put(31.9000,-28.1000){\makebox(0,0)[rt]{O}}%
% STR 2 0 3 0 Black White  
% 4 3160 187 3160 200 4 3000 0 0
% $y$
\put(31.6000,-2.0000){\makebox(0,0)[rt]{$y$}}%
% STR 2 0 3 0 Black White  
% 4 5800 2827 5800 2840 4 3000 0 0
% $x$
\put(58.0000,-28.4000){\makebox(0,0)[rt]{$x$}}%
% VECTOR 2 0 3 0 Black White  
% 2 3200 3000 3200 200
% 
\special{pn 8}%
\special{pa 3200 3000}%
\special{pa 3200 200}%
\special{fp}%
\special{sh 1}%
\special{pa 3200 200}%
\special{pa 3180 267}%
\special{pa 3200 253}%
\special{pa 3220 267}%
\special{pa 3200 200}%
\special{fp}%
% VECTOR 2 0 3 0 Black White  
% 2 3000 2800 5800 2800
% 
\special{pn 8}%
\special{pa 3000 2800}%
\special{pa 5800 2800}%
\special{fp}%
\special{sh 1}%
\special{pa 5800 2800}%
\special{pa 5733 2780}%
\special{pa 5747 2800}%
\special{pa 5733 2820}%
\special{pa 5800 2800}%
\special{fp}%
% FUNC 2 0 3 0 Black White  
% 9 3000 200 5800 3000 3200 2800 3600 2800 3200 2600 3000 200 5800 3000 0 2 0 0
% 4/x
\special{pn 8}%
\special{pa 3323 200}%
\special{pa 3325 240}%
\special{pa 3330 338}%
\special{pa 3335 430}%
\special{pa 3340 514}%
\special{pa 3345 593}%
\special{pa 3350 667}%
\special{pa 3355 735}%
\special{pa 3360 800}%
\special{pa 3365 861}%
\special{pa 3370 918}%
\special{pa 3375 971}%
\special{pa 3380 1022}%
\special{pa 3385 1070}%
\special{pa 3390 1116}%
\special{pa 3395 1159}%
\special{pa 3400 1200}%
\special{pa 3405 1239}%
\special{pa 3410 1276}%
\special{pa 3415 1312}%
\special{pa 3425 1378}%
\special{pa 3430 1409}%
\special{pa 3440 1467}%
\special{pa 3445 1494}%
\special{pa 3450 1520}%
\special{pa 3455 1545}%
\special{pa 3460 1569}%
\special{pa 3470 1615}%
\special{pa 3480 1657}%
\special{pa 3490 1697}%
\special{pa 3505 1751}%
\special{pa 3510 1768}%
\special{pa 3520 1800}%
\special{pa 3535 1845}%
\special{pa 3540 1859}%
\special{pa 3545 1872}%
\special{pa 3550 1886}%
\special{pa 3555 1899}%
\special{pa 3575 1947}%
\special{pa 3585 1969}%
\special{pa 3590 1979}%
\special{pa 3595 1990}%
\special{pa 3610 2020}%
\special{pa 3630 2056}%
\special{pa 3635 2064}%
\special{pa 3640 2073}%
\special{pa 3655 2097}%
\special{pa 3660 2104}%
\special{pa 3665 2112}%
\special{pa 3695 2154}%
\special{pa 3705 2166}%
\special{pa 3710 2173}%
\special{pa 3720 2185}%
\special{pa 3725 2190}%
\special{pa 3735 2202}%
\special{pa 3740 2207}%
\special{pa 3745 2213}%
\special{pa 3755 2223}%
\special{pa 3760 2229}%
\special{pa 3770 2239}%
\special{pa 3775 2243}%
\special{pa 3790 2258}%
\special{pa 3795 2262}%
\special{pa 3800 2267}%
\special{pa 3810 2275}%
\special{pa 3815 2280}%
\special{pa 3850 2308}%
\special{pa 3855 2311}%
\special{pa 3865 2319}%
\special{pa 3870 2322}%
\special{pa 3875 2326}%
\special{pa 3880 2329}%
\special{pa 3885 2333}%
\special{pa 3890 2336}%
\special{pa 3895 2340}%
\special{pa 3915 2352}%
\special{pa 3920 2356}%
\special{pa 3940 2368}%
\special{pa 3945 2370}%
\special{pa 3965 2382}%
\special{pa 3970 2384}%
\special{pa 3980 2390}%
\special{pa 3985 2392}%
\special{pa 3990 2395}%
\special{pa 3995 2397}%
\special{pa 4000 2400}%
\special{pa 4005 2402}%
\special{pa 4010 2405}%
\special{pa 4015 2407}%
\special{pa 4020 2410}%
\special{pa 4030 2414}%
\special{pa 4035 2417}%
\special{pa 4045 2421}%
\special{pa 4050 2424}%
\special{pa 4140 2460}%
\special{pa 4145 2461}%
\special{pa 4160 2467}%
\special{pa 4165 2468}%
\special{pa 4175 2472}%
\special{pa 4180 2473}%
\special{pa 4190 2477}%
\special{pa 4195 2478}%
\special{pa 4205 2482}%
\special{pa 4210 2483}%
\special{pa 4215 2485}%
\special{pa 4220 2486}%
\special{pa 4225 2488}%
\special{pa 4230 2489}%
\special{pa 4235 2491}%
\special{pa 4240 2492}%
\special{pa 4245 2494}%
\special{pa 4250 2495}%
\special{pa 4255 2497}%
\special{pa 4260 2498}%
\special{pa 4265 2500}%
\special{pa 4275 2502}%
\special{pa 4280 2504}%
\special{pa 4290 2506}%
\special{pa 4295 2508}%
\special{pa 4305 2510}%
\special{pa 4310 2512}%
\special{pa 4320 2514}%
\special{pa 4325 2516}%
\special{pa 4340 2519}%
\special{pa 4345 2521}%
\special{pa 4370 2526}%
\special{pa 4375 2528}%
\special{pa 4405 2534}%
\special{pa 4410 2536}%
\special{pa 4520 2558}%
\special{pa 4525 2558}%
\special{pa 4565 2566}%
\special{pa 4570 2566}%
\special{pa 4595 2571}%
\special{pa 4600 2571}%
\special{pa 4620 2575}%
\special{pa 4625 2575}%
\special{pa 4645 2579}%
\special{pa 4650 2579}%
\special{pa 4665 2582}%
\special{pa 4670 2582}%
\special{pa 4685 2585}%
\special{pa 4690 2585}%
\special{pa 4700 2587}%
\special{pa 4705 2587}%
\special{pa 4715 2589}%
\special{pa 4720 2589}%
\special{pa 4735 2592}%
\special{pa 4740 2592}%
\special{pa 4750 2594}%
\special{pa 4755 2594}%
\special{pa 4765 2596}%
\special{pa 4770 2596}%
\special{pa 4775 2597}%
\special{pa 4780 2597}%
\special{pa 4790 2599}%
\special{pa 4795 2599}%
\special{pa 4805 2601}%
\special{pa 4810 2601}%
\special{pa 4815 2602}%
\special{pa 4820 2602}%
\special{pa 4830 2604}%
\special{pa 4835 2604}%
\special{pa 4840 2605}%
\special{pa 4845 2605}%
\special{pa 4855 2607}%
\special{pa 4860 2607}%
\special{pa 4865 2608}%
\special{pa 4870 2608}%
\special{pa 4880 2610}%
\special{pa 4885 2610}%
\special{pa 4890 2611}%
\special{pa 4895 2611}%
\special{pa 4900 2612}%
\special{pa 4905 2612}%
\special{pa 4910 2613}%
\special{pa 4915 2613}%
\special{pa 4920 2614}%
\special{pa 4925 2614}%
\special{pa 4935 2616}%
\special{pa 4940 2616}%
\special{pa 4945 2617}%
\special{pa 4950 2617}%
\special{pa 4955 2618}%
\special{pa 4960 2618}%
\special{pa 4965 2619}%
\special{pa 4970 2619}%
\special{pa 4975 2620}%
\special{pa 4980 2620}%
\special{pa 4985 2621}%
\special{pa 4990 2621}%
\special{pa 4995 2622}%
\special{pa 5000 2622}%
\special{pa 5005 2623}%
\special{pa 5010 2623}%
\special{pa 5015 2624}%
\special{pa 5020 2624}%
\special{pa 5025 2625}%
\special{pa 5030 2625}%
\special{pa 5035 2626}%
\special{pa 5040 2626}%
\special{pa 5045 2627}%
\special{pa 5055 2627}%
\special{pa 5060 2628}%
\special{pa 5065 2628}%
\special{pa 5070 2629}%
\special{pa 5075 2629}%
\special{pa 5080 2630}%
\special{pa 5085 2630}%
\special{pa 5090 2631}%
\special{pa 5095 2631}%
\special{pa 5100 2632}%
\special{pa 5110 2632}%
\special{pa 5115 2633}%
\special{pa 5120 2633}%
\special{pa 5125 2634}%
\special{pa 5130 2634}%
\special{pa 5135 2635}%
\special{pa 5145 2635}%
\special{pa 5150 2636}%
\special{pa 5155 2636}%
\special{pa 5160 2637}%
\special{pa 5165 2637}%
\special{pa 5170 2638}%
\special{pa 5180 2638}%
\special{pa 5185 2639}%
\special{pa 5190 2639}%
\special{pa 5195 2640}%
\special{pa 5205 2640}%
\special{pa 5210 2641}%
\special{pa 5215 2641}%
\special{pa 5220 2642}%
\special{pa 5230 2642}%
\special{pa 5235 2643}%
\special{pa 5240 2643}%
\special{pa 5245 2644}%
\special{pa 5255 2644}%
\special{pa 5260 2645}%
\special{pa 5270 2645}%
\special{pa 5275 2646}%
\special{pa 5280 2646}%
\special{pa 5285 2647}%
\special{pa 5295 2647}%
\special{pa 5300 2648}%
\special{pa 5310 2648}%
\special{pa 5315 2649}%
\special{pa 5325 2649}%
\special{pa 5330 2650}%
\special{pa 5340 2650}%
\special{pa 5345 2651}%
\special{pa 5350 2651}%
\special{pa 5355 2652}%
\special{pa 5365 2652}%
\special{pa 5370 2653}%
\special{pa 5380 2653}%
\special{pa 5385 2654}%
\special{pa 5395 2654}%
\special{pa 5400 2655}%
\special{pa 5410 2655}%
\special{pa 5415 2656}%
\special{pa 5425 2656}%
\special{pa 5430 2657}%
\special{pa 5445 2657}%
\special{pa 5450 2658}%
\special{pa 5460 2658}%
\special{pa 5465 2659}%
\special{pa 5475 2659}%
\special{pa 5480 2660}%
\special{pa 5490 2660}%
\special{pa 5495 2661}%
\special{pa 5510 2661}%
\special{pa 5515 2662}%
\special{pa 5525 2662}%
\special{pa 5530 2663}%
\special{pa 5540 2663}%
\special{pa 5545 2664}%
\special{pa 5560 2664}%
\special{pa 5565 2665}%
\special{pa 5575 2665}%
\special{pa 5580 2666}%
\special{pa 5595 2666}%
\special{pa 5600 2667}%
\special{pa 5615 2667}%
\special{pa 5620 2668}%
\special{pa 5630 2668}%
\special{pa 5635 2669}%
\special{pa 5650 2669}%
\special{pa 5655 2670}%
\special{pa 5670 2670}%
\special{pa 5675 2671}%
\special{pa 5690 2671}%
\special{pa 5695 2672}%
\special{pa 5705 2672}%
\special{pa 5710 2673}%
\special{pa 5725 2673}%
\special{pa 5730 2674}%
\special{pa 5745 2674}%
\special{pa 5750 2675}%
\special{pa 5770 2675}%
\special{pa 5775 2676}%
\special{pa 5790 2676}%
\special{pa 5795 2677}%
\special{pa 5800 2677}%
\special{fp}%
% LINE 2 0 3 0 Black White  
% 4 3200 2800 3600 2000 3200 2800 4400 2530
% 
\special{pn 8}%
\special{pa 3200 2800}%
\special{pa 3600 2000}%
\special{fp}%
\special{pa 3200 2800}%
\special{pa 4400 2530}%
\special{fp}%
% LINE 2 2 3 0 Black White  
% 6 3200 2530 4400 2530 3600 2000 3200 2000 3600 2000 3600 2800
% 
\special{pn 8}%
\special{pa 3200 2530}%
\special{pa 4400 2530}%
\special{dt 0.045}%
\special{pa 3600 2000}%
\special{pa 3200 2000}%
\special{dt 0.045}%
\special{pa 3600 2000}%
\special{pa 3600 2800}%
\special{dt 0.045}%
% LINE 2 2 3 0 Black White  
% 2 4400 2800 4400 2530
% 
\special{pn 8}%
\special{pa 4400 2800}%
\special{pa 4400 2530}%
\special{dt 0.045}%
% STR 2 0 3 0 Black White  
% 4 4400 2740 4400 2840 5 0 1 0
% 2x
\put(44.0000,-28.4000){\makebox(0,0){{\colorbox[named]{White}{2x}}}}%
% STR 2 0 3 0 Black White  
% 4 3600 2740 3600 2840 5 0 0 0
% x
\put(36.0000,-28.4000){\makebox(0,0){x}}%
% STR 2 0 3 0 Black White  
% 4 3600 1340 3600 1440 5 0 0 0
% $f(x)$
\put(36.0000,-14.4000){\makebox(0,0){$f(x)$}}%
% LINE 3 0 3 0 Black White  
% 36 3860 2320 3650 2530 3830 2290 3600 2520 3800 2260 3600 2460 3770 2230 3600 2400 3740 2200 3600 2340 3710 2170 3600 2280 3680 2140 3600 2220 3660 2100 3600 2160 3630 2070 3600 2100 3900 2340 3710 2530 3940 2360 3770 2530 3970 2390 3830 2530 4010 2410 3890 2530 4060 2420 3950 2530 4100 2440 4010 2530 4140 2460 4070 2530 4190 2470 4130 2530 4230 2490 4190 2530
% 
\special{pn 4}%
\special{pa 3860 2320}%
\special{pa 3650 2530}%
\special{fp}%
\special{pa 3830 2290}%
\special{pa 3600 2520}%
\special{fp}%
\special{pa 3800 2260}%
\special{pa 3600 2460}%
\special{fp}%
\special{pa 3770 2230}%
\special{pa 3600 2400}%
\special{fp}%
\special{pa 3740 2200}%
\special{pa 3600 2340}%
\special{fp}%
\special{pa 3710 2170}%
\special{pa 3600 2280}%
\special{fp}%
\special{pa 3680 2140}%
\special{pa 3600 2220}%
\special{fp}%
\special{pa 3660 2100}%
\special{pa 3600 2160}%
\special{fp}%
\special{pa 3630 2070}%
\special{pa 3600 2100}%
\special{fp}%
\special{pa 3900 2340}%
\special{pa 3710 2530}%
\special{fp}%
\special{pa 3940 2360}%
\special{pa 3770 2530}%
\special{fp}%
\special{pa 3970 2390}%
\special{pa 3830 2530}%
\special{fp}%
\special{pa 4010 2410}%
\special{pa 3890 2530}%
\special{fp}%
\special{pa 4060 2420}%
\special{pa 3950 2530}%
\special{fp}%
\special{pa 4100 2440}%
\special{pa 4010 2530}%
\special{fp}%
\special{pa 4140 2460}%
\special{pa 4070 2530}%
\special{fp}%
\special{pa 4190 2470}%
\special{pa 4130 2530}%
\special{fp}%
\special{pa 4230 2490}%
\special{pa 4190 2530}%
\special{fp}%
% LINE 3 0 3 0 Black White  
% 22 3770 2530 3600 2700 3710 2530 3600 2640 3650 2530 3600 2580 3830 2530 3670 2690 3890 2530 3740 2680 3950 2530 3820 2660 4010 2530 3900 2640 4070 2530 3980 2620 4130 2530 4050 2610 4190 2530 4130 2590 4250 2530 4210 2570
% 
\special{pn 4}%
\special{pa 3770 2530}%
\special{pa 3600 2700}%
\special{fp}%
\special{pa 3710 2530}%
\special{pa 3600 2640}%
\special{fp}%
\special{pa 3650 2530}%
\special{pa 3600 2580}%
\special{fp}%
\special{pa 3830 2530}%
\special{pa 3670 2690}%
\special{fp}%
\special{pa 3890 2530}%
\special{pa 3740 2680}%
\special{fp}%
\special{pa 3950 2530}%
\special{pa 3820 2660}%
\special{fp}%
\special{pa 4010 2530}%
\special{pa 3900 2640}%
\special{fp}%
\special{pa 4070 2530}%
\special{pa 3980 2620}%
\special{fp}%
\special{pa 4130 2530}%
\special{pa 4050 2610}%
\special{fp}%
\special{pa 4190 2530}%
\special{pa 4130 2590}%
\special{fp}%
\special{pa 4250 2530}%
\special{pa 4210 2570}%
\special{fp}%
% LINE 3 0 3 0 Black White  
% 12 3530 2530 3280 2780 3470 2530 3210 2790 3410 2530 3260 2680 3590 2530 3360 2760 3600 2580 3430 2750 3600 2640 3510 2730
% 
\special{pn 4}%
\special{pa 3530 2530}%
\special{pa 3280 2780}%
\special{fp}%
\special{pa 3470 2530}%
\special{pa 3210 2790}%
\special{fp}%
\special{pa 3410 2530}%
\special{pa 3260 2680}%
\special{fp}%
\special{pa 3590 2530}%
\special{pa 3360 2760}%
\special{fp}%
\special{pa 3600 2580}%
\special{pa 3430 2750}%
\special{fp}%
\special{pa 3600 2640}%
\special{pa 3510 2730}%
\special{fp}%
% LINE 3 0 3 0 Black White  
% 16 3600 2400 3470 2530 3600 2340 3410 2530 3600 2280 3350 2530 3600 2220 3380 2440 3600 2160 3440 2320 3600 2100 3500 2200 3600 2040 3560 2080 3600 2460 3530 2530
% 
\special{pn 4}%
\special{pa 3600 2400}%
\special{pa 3470 2530}%
\special{fp}%
\special{pa 3600 2340}%
\special{pa 3410 2530}%
\special{fp}%
\special{pa 3600 2280}%
\special{pa 3350 2530}%
\special{fp}%
\special{pa 3600 2220}%
\special{pa 3380 2440}%
\special{fp}%
\special{pa 3600 2160}%
\special{pa 3440 2320}%
\special{fp}%
\special{pa 3600 2100}%
\special{pa 3500 2200}%
\special{fp}%
\special{pa 3600 2040}%
\special{pa 3560 2080}%
\special{fp}%
\special{pa 3600 2460}%
\special{pa 3530 2530}%
\special{fp}%
% STR 2 0 3 0 Black White  
% 4 3740 2330 3740 2430 5 0 1 0
% $3S(x)$
\put(37.4000,-24.3000){\makebox(0,0){{\colorbox[named]{White}{$3S(x)$}}}}%
\end{picture}}%
}
          \end{center}
     故に,題意から
          \begin{align}
          3S(x)&=\int_x^{2x}f(t)dt+\frac{1}{2}xf(x)-\frac{1}{2}2xf(2x) \nonumber\\
          &=S(x)+\frac{1}{2}xf(x)-xf(2x) \nonumber\\
          S(x)&=\frac{1}{4}x(f(x)-2f(2x)) \label{1}
          \end{align}
     である.$g(x)=f(x)-2f(2x)$とおくと,\eqref{0},\eqref{1}から
          \begin{align}
               \begin{array}{l}
               S'(x)=2f(2x)-f(x)=-g(x) \\
               S(x)=\dfrac{1}{4}xg(x) 
               \end{array}\label{2}
          \end{align}
     $g(x)$を消して,$S(x)=y$とすれば,
          \[4y=-x\frac{dy}{dx} \]
     となる.$0<x$で$0<f(x)$であるから,$y>0$となるので,変形して
          \[\frac{-4x}{dx}=\frac{y}{dy}\]
     積分して,$S(1)=1$より,
          \[y=S(x)=x^{-4}\]
     である.従って\eqref{2}から,$g(x)=4x^{-5}$である.$\cdots$(答)
     
     \item $a_n(x)=2^nf(2^nx)$とおく.前問の結果で$x$に$2^nx$を代入して,
          \begin{align*}
          2f(2^{n+1}x)&=f(2^nx)-4(2^nx)^{-5} \\
          2^{n+1}f(2^{n+1}x)&=2^nf(2^nx)-2^{n+2}(2^nx)^{-5} \\
          a_{n+1}(x)&=a_n(x)-2^{2-4n}x^{-5}
          \end{align*}
     である.繰り返し用いて,$n\ge1$のとき,
          \begin{align*}
          a_n(x)=a_0(x)-\sum_{k=0}^{n-1}2^{2-4n}x^{-5} \\
          a_n(x)=f(x)-4x^{-5}\frac{1-2^{-4n}}{1-2^4} \\
          \limit{n}{\infty}f(x)-\frac{64}{15}x^{-5}=a(x)
          \end{align*}
     であるから,求める積分値は
           \begin{align*}
           \dint{x}{2x}a(t)dt&=S(x)+\left[\frac{16}{15}t^{-4}\right]_x^{2x} \\
           &=x^{-4}+\frac{16}{15}\left(\frac{1}{16}-1\right)x^{-4} \\
           &=0
           \end{align*}
      である.
      
      \item $f(x)>0$から,$a(x)>0$である.これと前問の積分計算から,被積分が恒等的に$0$である.
           \[a(x)\equiv0\Longleftrightarrow f(x)=\frac{64}{15}x^{-5}\]
      これは$x>0$で定義された,連続な単調減少関数であり,十分である.以上から,
           \[f(x)=\frac{64}{15}x^{-5}\]
      である.$\cdots$(答)
      \end{enumerate}
\newpage
\end{multicols}
\end{document}