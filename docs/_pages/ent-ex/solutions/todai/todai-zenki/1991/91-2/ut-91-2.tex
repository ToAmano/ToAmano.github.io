\documentclass[a4j]{jarticle}
\usepackage{amsmath}
\usepackage{ascmac}
\usepackage{amssymb}
\usepackage{enumerate}
\usepackage{multicol}
\usepackage{framed}
\title{東大91-2}
\begin{document}
\begin{oframed}
$a$,$b$,$c$を正の実数とする.$xyz$空間において,$|x|\ge a$,$|y|\ge b$,$z=c$を満たす点
$(x,y,z)$からなる板$R$を考える.点光源$P$が平面$z=c+1$上の楕円
$\frac{x^2}{a^2}+\frac{y^2}{b^2}=1,z=c+1$の上を一周するとき,光が板$R$に遮られて$xy$平面にできる影の通過する部分の図を描き,その面積を求めよ
\end{oframed}
\setlength{\columnseprule}{0.4pt}
\begin{multicols}{2}
\noindent\textgt{[解]} $4$頂点$A(a,b)$,$B(a,-b)$,$C(-a,-b)$,$D(-a,b)$とする.
$0\le\theta<2\pi$とする.$t=\cos\theta$,$s=\sin\theta$とおけば,$P(at,bs,c+1)$と書ける.この時,
頂点$X$が点光源により$xy$平面に射影される点$X'$とすれば
\begin{align*}
A'()
\end{align*}
となる.故に光が$R$で遮られてできる影は以下の不等式で与えられる長方形領域である.
\begin{align}
\le x\le
\le y\le
\end{align}
$\theta$を動かした時に\eqref{eq:eq1}の動く領域を求めればよい.
 \newpage

\end{multicols}
\end{document}