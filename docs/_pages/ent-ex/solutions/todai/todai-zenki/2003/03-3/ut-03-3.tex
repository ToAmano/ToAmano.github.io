\documentclass[a4j]{jarticle}
\usepackage{amsmath}
\usepackage{ascmac}
\usepackage{amssymb}
\usepackage{enumerate}
\usepackage{multicol}
\usepackage{framed}
\usepackage{fancyhdr}
\usepackage{latexsym}
\usepackage{indent}
\usepackage{cases}
\usepackage[dvips]{graphicx}
\usepackage{color}
\allowdisplaybreaks
\pagestyle{fancy}
\lhead{}
\chead{}
\rhead{東京大学前期$2003$年$3$番}
\begin{document}
%分数関係


\def\tfrac#1#2{{\textstyle\frac{#1}{#2}}} %数式中で文中表示の分数を使う時


%Σ関係

\def\dsum#1#2{{\displaystyle\sum_{#1}^{#2}}} %文中で数式表示のΣを使う時


%ベクトル関係


\def\vector#1{\overrightarrow{#1}}  %ベクトルを表現したいとき(aベクトルを表現するときは\ver
\def\norm#1{|\overrightarrow{#1}|} %ベクトルの絶対値
\def\vtwo#1#2{ \left(%
      \begin{array}{c}%
      #1 \\%
      #2 \\%
      \end{array}%
      \right) }                        %2次元ベクトル成分表示
      
      \def\vthree#1#2#3{ \left(
      \begin{array}{c}
      #1 \\
      #2 \\
      #3 \\
      \end{array}
      \right) }                        %3次元ベクトル成分表示



%数列関係


\def\an#1{\verb|{|$#1$\verb|}|}


%極限関係

\def\limit#1#2{\stackrel{#1 \to #2}{\longrightarrow}}   %等式変形からの極限
\def\dlim#1#2{{\displaystyle \lim_{#1\to#2}}} %文中で数式表示の極限を使う



%積分関係

\def\dint#1#2{{\displaystyle \int_{#1}^{#2}}} %文中で数式表示の積分を使う時

\def\ne{\nearrow}
\def\se{\searrow}
\def\nw{\nwarrow}
\def\ne{\nearrow}


%便利なやつ

\def\case#1#2{%
 \[\left\{%
 \begin{array}{l}%
 #1 \\%
 #2%
 \end{array}%
 \right.\] }                           %場合分け
 
\def\1{$\cos\theta=c$,$\sin\theta=s$とおく.}  %cs表示を与える前書きシータ
\def\2{$\cos t=c$,$\sin t=s$とおく.}     %cs表示を与える前書きt
\def\3{$\cos x=c$,$\sin x=s$とおく.}                %cs表示を与える前書きx

\def\fig#1#2#3 {%
\begin{wrapfigure}[#1]{r}{#2 zw}%
\vspace*{-1zh}%
\input{#3}%
\end{wrapfigure} }           %絵の挿入


\def\a{\alpha}   %ギリシャ文字
\def\b{\beta}
\def\g{\gamma}

%問題番号のためのマクロ

\newcounter{nombre} %必須
\renewcommand{\thenombre}{\arabic{nombre}} %任意
\setcounter{nombre}{2} %任意
\newcounter{nombresub}[nombre] %親子関係を定義
\renewcommand{\thenombresub}{\arabic{nombresub}} %任意
\setcounter{nombresub}{0} %任意
\newcommand{\prob}[1][]{\refstepcounter{nombre}#1[問題 \thenombre]}
\newcommand{\probsub}[1][]{\refstepcounter{nombresub}#1(\thenombresub)}


%1-1みたいなカウンタ(todaiとtodaia)
\newcounter{todai}
\setcounter{todai}{0}
\newcounter{todaisub}[todai] 
\setcounter{todaisub}{0} 
\newcommand{\todai}[1][]{\refstepcounter{todai}#1 \thetodai-\thetodaisub}
\newcommand{\todaib}[1][]{\refstepcounter{todai}#1\refstepcounter{todaisub}#1 {\bf [問題 \thetodai.\thetodaisub]}}
\newcommand{\todaia}[1][]{\refstepcounter{todaisub}#1 {\bf [問題 \thetodai.\thetodaisub]}}


     \begin{oframed}
     $xyz$空間において,平面$z=0$上の原点を中心とする半径$2$の円を底面とし,点$(0,0,1)$を頂点とする円錐を$A$とする.
     
     次に平面$z=0$上の点$(1,0,0)$を中心とする半径$1$の円を$H$,平面$z=1$上の点$(1,0,1)$を中心とする半径$1$の円を$K$とする.
     $H$と$K$を$2$つの底面とする円柱を$B$とする.
     
     円錐$A$と円柱$B$の共通部分を$C$とする.
     
     $0\le t\le1$を満たす実数$t$に対し,平面$z=t$による$C$の切り口の面積を$S(t)$とおく.
          \begin{enumerate}[(1)]
          \item $0\le\theta\le\pi/2$とする.$t=1-\cos\theta$のとき,$S(t)$を$\theta$で表せ.
          \item $C$の体積$\dint{0}{1}S(t)dt$を求めよ.
          \end{enumerate}     
     \end{oframed}

\setlength{\columnseprule}{0.4pt}
\begin{multicols}{2}
{\bf[解]} \1 円錐$A$の側面の方程式は,$x^2+y^2=(1-z)^2$,円柱$B$の側面は$(x-1)^2+y^2=1$であるから,
$C$の$z=1-c$での切り口は,
     \begin{align*}
          \begin{cases}
          x^2+y^2\ge(2c)^2 \\
          (x-1)^2+y^2\le1 
          \end{cases}
     \end{align*}
である.これを図示すると下図.ここで,断面の$2$円の交点は$(2c^2,\pm\sin2\theta)$であるから,円の中心と交点を結ぶ直線
と$x$軸のなす角は,順に$\theta$,$\pi-2\theta$である.
     \begin{center}
     \scalebox{.6}{%WinTpicVersion4.32a
{\unitlength 0.1in%
\begin{picture}(36.2000,36.0000)(3.8000,-40.0000)%
% STR 2 0 3 0 Black White  
% 4 590 2197 590 2210 4 400 0 0
% O
\put(5.9000,-22.1000){\makebox(0,0)[rt]{O}}%
% STR 2 0 3 0 Black White  
% 4 560 387 560 400 4 400 0 0
% $y$
\put(5.6000,-4.0000){\makebox(0,0)[rt]{$y$}}%
% STR 2 0 3 0 Black White  
% 4 4000 2227 4000 2240 4 400 0 0
% $x$
\put(40.0000,-22.4000){\makebox(0,0)[rt]{$x$}}%
% VECTOR 2 0 3 0 Black White  
% 2 600 4000 600 400
% 
\special{pn 8}%
\special{pa 600 4000}%
\special{pa 600 400}%
\special{fp}%
\special{sh 1}%
\special{pa 600 400}%
\special{pa 580 467}%
\special{pa 600 453}%
\special{pa 620 467}%
\special{pa 600 400}%
\special{fp}%
% VECTOR 2 0 3 0 Black White  
% 2 400 2200 4000 2200
% 
\special{pn 8}%
\special{pa 400 2200}%
\special{pa 4000 2200}%
\special{fp}%
\special{sh 1}%
\special{pa 4000 2200}%
\special{pa 3933 2180}%
\special{pa 3947 2200}%
\special{pa 3933 2220}%
\special{pa 4000 2200}%
\special{fp}%
% FUNC 2 0 3 0 Black White  
% 9 400 400 4000 4000 600 2200 2200 2200 600 600 400 400 4000 4000 50 2 0 2
% 1+cos(t)///sin(t)///0///10
\special{pn 8}%
\special{pa 3800 2200}%
\special{pa 3800 2168}%
\special{pa 3799 2136}%
\special{pa 3795 2072}%
\special{pa 3792 2040}%
\special{pa 3784 1976}%
\special{pa 3780 1945}%
\special{pa 3774 1913}%
\special{pa 3768 1882}%
\special{pa 3761 1851}%
\special{pa 3754 1819}%
\special{pa 3746 1788}%
\special{pa 3738 1758}%
\special{pa 3728 1727}%
\special{pa 3719 1696}%
\special{pa 3686 1606}%
\special{pa 3674 1577}%
\special{pa 3661 1547}%
\special{pa 3647 1518}%
\special{pa 3634 1489}%
\special{pa 3604 1433}%
\special{pa 3572 1377}%
\special{pa 3538 1323}%
\special{pa 3520 1296}%
\special{pa 3502 1270}%
\special{pa 3483 1244}%
\special{pa 3464 1219}%
\special{pa 3424 1169}%
\special{pa 3403 1145}%
\special{pa 3381 1121}%
\special{pa 3360 1097}%
\special{pa 3337 1075}%
\special{pa 3315 1052}%
\special{pa 3291 1030}%
\special{pa 3268 1008}%
\special{pa 3244 987}%
\special{pa 3219 967}%
\special{pa 3194 946}%
\special{pa 3169 927}%
\special{pa 3117 889}%
\special{pa 3091 871}%
\special{pa 3064 853}%
\special{pa 3037 836}%
\special{pa 3010 820}%
\special{pa 2982 804}%
\special{pa 2954 789}%
\special{pa 2925 774}%
\special{pa 2897 760}%
\special{pa 2868 746}%
\special{pa 2839 733}%
\special{pa 2779 709}%
\special{pa 2750 697}%
\special{pa 2719 687}%
\special{pa 2689 677}%
\special{pa 2658 667}%
\special{pa 2628 658}%
\special{pa 2566 642}%
\special{pa 2534 635}%
\special{pa 2472 623}%
\special{pa 2440 618}%
\special{pa 2408 614}%
\special{pa 2377 610}%
\special{pa 2313 604}%
\special{pa 2281 602}%
\special{pa 2217 600}%
\special{pa 2185 600}%
\special{pa 2121 602}%
\special{pa 2057 606}%
\special{pa 2025 610}%
\special{pa 1994 613}%
\special{pa 1930 623}%
\special{pa 1899 629}%
\special{pa 1867 635}%
\special{pa 1836 642}%
\special{pa 1774 658}%
\special{pa 1743 667}%
\special{pa 1713 676}%
\special{pa 1682 686}%
\special{pa 1622 708}%
\special{pa 1593 720}%
\special{pa 1563 732}%
\special{pa 1534 745}%
\special{pa 1476 773}%
\special{pa 1420 803}%
\special{pa 1392 819}%
\special{pa 1365 835}%
\special{pa 1337 852}%
\special{pa 1311 870}%
\special{pa 1284 888}%
\special{pa 1232 926}%
\special{pa 1207 945}%
\special{pa 1182 965}%
\special{pa 1134 1007}%
\special{pa 1110 1029}%
\special{pa 1064 1073}%
\special{pa 1020 1119}%
\special{pa 999 1143}%
\special{pa 978 1168}%
\special{pa 957 1192}%
\special{pa 937 1217}%
\special{pa 899 1269}%
\special{pa 863 1321}%
\special{pa 829 1375}%
\special{pa 797 1431}%
\special{pa 782 1459}%
\special{pa 767 1488}%
\special{pa 753 1516}%
\special{pa 740 1546}%
\special{pa 727 1575}%
\special{pa 715 1604}%
\special{pa 703 1634}%
\special{pa 692 1664}%
\special{pa 682 1695}%
\special{pa 672 1725}%
\special{pa 654 1787}%
\special{pa 646 1818}%
\special{pa 632 1880}%
\special{pa 626 1911}%
\special{pa 616 1975}%
\special{pa 612 2006}%
\special{pa 608 2038}%
\special{pa 605 2070}%
\special{pa 601 2134}%
\special{pa 600 2166}%
\special{pa 600 2230}%
\special{pa 601 2262}%
\special{pa 605 2326}%
\special{pa 608 2358}%
\special{pa 611 2389}%
\special{pa 615 2421}%
\special{pa 620 2453}%
\special{pa 625 2484}%
\special{pa 631 2516}%
\special{pa 645 2578}%
\special{pa 653 2609}%
\special{pa 671 2671}%
\special{pa 691 2731}%
\special{pa 702 2762}%
\special{pa 713 2791}%
\special{pa 725 2821}%
\special{pa 751 2879}%
\special{pa 765 2908}%
\special{pa 780 2937}%
\special{pa 810 2993}%
\special{pa 827 3021}%
\special{pa 843 3048}%
\special{pa 860 3075}%
\special{pa 878 3102}%
\special{pa 897 3128}%
\special{pa 915 3154}%
\special{pa 935 3179}%
\special{pa 954 3204}%
\special{pa 975 3229}%
\special{pa 1017 3277}%
\special{pa 1039 3301}%
\special{pa 1061 3324}%
\special{pa 1130 3390}%
\special{pa 1154 3411}%
\special{pa 1179 3432}%
\special{pa 1229 3472}%
\special{pa 1255 3491}%
\special{pa 1281 3509}%
\special{pa 1307 3528}%
\special{pa 1388 3579}%
\special{pa 1416 3595}%
\special{pa 1472 3625}%
\special{pa 1530 3653}%
\special{pa 1588 3679}%
\special{pa 1618 3690}%
\special{pa 1648 3702}%
\special{pa 1678 3713}%
\special{pa 1709 3723}%
\special{pa 1739 3732}%
\special{pa 1770 3741}%
\special{pa 1832 3757}%
\special{pa 1894 3771}%
\special{pa 1926 3776}%
\special{pa 1957 3782}%
\special{pa 2021 3790}%
\special{pa 2085 3796}%
\special{pa 2117 3798}%
\special{pa 2149 3799}%
\special{pa 2180 3800}%
\special{pa 2212 3800}%
\special{pa 2276 3798}%
\special{pa 2340 3794}%
\special{pa 2372 3791}%
\special{pa 2436 3783}%
\special{pa 2467 3778}%
\special{pa 2499 3772}%
\special{pa 2530 3766}%
\special{pa 2561 3759}%
\special{pa 2623 3743}%
\special{pa 2685 3725}%
\special{pa 2715 3715}%
\special{pa 2775 3693}%
\special{pa 2835 3669}%
\special{pa 2864 3656}%
\special{pa 2893 3642}%
\special{pa 2921 3628}%
\special{pa 2950 3613}%
\special{pa 2978 3598}%
\special{pa 3006 3582}%
\special{pa 3033 3566}%
\special{pa 3060 3549}%
\special{pa 3114 3513}%
\special{pa 3140 3495}%
\special{pa 3166 3476}%
\special{pa 3216 3436}%
\special{pa 3240 3416}%
\special{pa 3264 3395}%
\special{pa 3288 3373}%
\special{pa 3334 3329}%
\special{pa 3356 3306}%
\special{pa 3378 3282}%
\special{pa 3400 3259}%
\special{pa 3421 3234}%
\special{pa 3441 3210}%
\special{pa 3461 3185}%
\special{pa 3480 3159}%
\special{pa 3499 3134}%
\special{pa 3518 3107}%
\special{pa 3536 3081}%
\special{pa 3570 3027}%
\special{pa 3602 2971}%
\special{pa 3632 2915}%
\special{pa 3646 2886}%
\special{pa 3659 2857}%
\special{pa 3672 2827}%
\special{pa 3684 2798}%
\special{pa 3696 2768}%
\special{pa 3707 2738}%
\special{pa 3717 2708}%
\special{pa 3727 2677}%
\special{pa 3736 2647}%
\special{pa 3745 2616}%
\special{pa 3753 2585}%
\special{pa 3767 2523}%
\special{pa 3773 2491}%
\special{pa 3779 2460}%
\special{pa 3784 2428}%
\special{pa 3792 2364}%
\special{pa 3794 2333}%
\special{pa 3797 2301}%
\special{pa 3799 2269}%
\special{pa 3800 2237}%
\special{pa 3800 2173}%
\special{pa 3799 2141}%
\special{pa 3795 2077}%
\special{pa 3789 2013}%
\special{pa 3785 1981}%
\special{pa 3780 1950}%
\special{pa 3775 1918}%
\special{pa 3769 1887}%
\special{pa 3762 1855}%
\special{pa 3755 1824}%
\special{pa 3739 1762}%
\special{pa 3730 1732}%
\special{pa 3720 1701}%
\special{pa 3710 1671}%
\special{pa 3688 1611}%
\special{pa 3676 1581}%
\special{pa 3650 1523}%
\special{pa 3636 1494}%
\special{pa 3621 1465}%
\special{pa 3591 1409}%
\special{pa 3575 1381}%
\special{pa 3541 1327}%
\special{pa 3523 1301}%
\special{pa 3505 1274}%
\special{pa 3486 1248}%
\special{pa 3467 1223}%
\special{pa 3427 1173}%
\special{pa 3385 1125}%
\special{pa 3363 1101}%
\special{pa 3341 1078}%
\special{pa 3318 1056}%
\special{pa 3295 1033}%
\special{pa 3271 1012}%
\special{pa 3248 991}%
\special{pa 3223 970}%
\special{pa 3173 930}%
\special{pa 3148 911}%
\special{pa 3122 892}%
\special{pa 3068 856}%
\special{pa 3041 839}%
\special{pa 3014 823}%
\special{pa 2958 791}%
\special{pa 2930 776}%
\special{pa 2872 748}%
\special{pa 2814 722}%
\special{pa 2784 710}%
\special{pa 2724 688}%
\special{pa 2694 678}%
\special{pa 2663 669}%
\special{pa 2633 660}%
\special{pa 2602 651}%
\special{pa 2571 644}%
\special{pa 2539 636}%
\special{pa 2477 624}%
\special{pa 2413 614}%
\special{pa 2382 610}%
\special{pa 2318 604}%
\special{pa 2286 602}%
\special{pa 2222 600}%
\special{pa 2190 600}%
\special{pa 2126 602}%
\special{pa 2062 606}%
\special{pa 2030 609}%
\special{pa 1999 613}%
\special{pa 1967 617}%
\special{pa 1935 622}%
\special{pa 1904 628}%
\special{pa 1872 634}%
\special{pa 1810 648}%
\special{pa 1779 656}%
\special{pa 1748 665}%
\special{pa 1718 674}%
\special{pa 1687 684}%
\special{pa 1627 706}%
\special{pa 1597 718}%
\special{pa 1568 730}%
\special{pa 1539 743}%
\special{pa 1452 785}%
\special{pa 1424 801}%
\special{pa 1396 816}%
\special{pa 1315 867}%
\special{pa 1288 885}%
\special{pa 1262 904}%
\special{pa 1237 923}%
\special{pa 1211 942}%
\special{pa 1186 962}%
\special{pa 1114 1025}%
\special{pa 1091 1047}%
\special{pa 1068 1070}%
\special{pa 1045 1092}%
\special{pa 1023 1116}%
\special{pa 1002 1139}%
\special{pa 981 1164}%
\special{pa 961 1188}%
\special{pa 941 1213}%
\special{pa 921 1239}%
\special{pa 902 1264}%
\special{pa 884 1291}%
\special{pa 866 1317}%
\special{pa 848 1344}%
\special{pa 831 1371}%
\special{pa 815 1399}%
\special{pa 799 1426}%
\special{pa 784 1455}%
\special{pa 770 1483}%
\special{pa 742 1541}%
\special{pa 729 1570}%
\special{pa 705 1630}%
\special{pa 694 1660}%
\special{pa 674 1720}%
\special{pa 664 1751}%
\special{pa 640 1844}%
\special{pa 633 1875}%
\special{pa 627 1906}%
\special{pa 622 1938}%
\special{pa 617 1969}%
\special{pa 612 2001}%
\special{pa 603 2097}%
\special{pa 602 2129}%
\special{pa 600 2161}%
\special{pa 600 2225}%
\special{pa 602 2289}%
\special{pa 605 2321}%
\special{pa 607 2352}%
\special{pa 619 2448}%
\special{pa 625 2479}%
\special{pa 630 2511}%
\special{pa 644 2573}%
\special{pa 660 2635}%
\special{pa 669 2666}%
\special{pa 679 2696}%
\special{pa 689 2727}%
\special{pa 711 2787}%
\special{pa 723 2816}%
\special{pa 736 2846}%
\special{pa 749 2875}%
\special{pa 763 2904}%
\special{pa 777 2932}%
\special{pa 792 2961}%
\special{pa 808 2989}%
\special{pa 824 3016}%
\special{pa 841 3044}%
\special{pa 857 3070}%
\special{fp}%
% FUNC 2 0 3 0 Black White  
% 10 400 400 4000 4000 600 2200 2200 2200 600 600 400 400 4000 4000 50 2 0 2 0 0
% cos(t)/2///sin(t)/2///0///10
\special{pn 8}%
\special{pa 1400 2200}%
\special{pa 1400 2184}%
\special{pa 1399 2168}%
\special{pa 1399 2152}%
\special{pa 1397 2136}%
\special{pa 1396 2120}%
\special{pa 1390 2072}%
\special{pa 1387 2057}%
\special{pa 1381 2025}%
\special{pa 1377 2010}%
\special{pa 1373 1994}%
\special{pa 1369 1979}%
\special{pa 1364 1963}%
\special{pa 1349 1918}%
\special{pa 1337 1888}%
\special{pa 1330 1874}%
\special{pa 1324 1859}%
\special{pa 1317 1845}%
\special{pa 1310 1830}%
\special{pa 1294 1802}%
\special{pa 1286 1789}%
\special{pa 1278 1775}%
\special{pa 1269 1761}%
\special{pa 1242 1722}%
\special{pa 1232 1709}%
\special{pa 1212 1685}%
\special{pa 1201 1672}%
\special{pa 1191 1660}%
\special{pa 1180 1649}%
\special{pa 1169 1637}%
\special{pa 1157 1626}%
\special{pa 1146 1615}%
\special{pa 1134 1604}%
\special{pa 1122 1594}%
\special{pa 1110 1583}%
\special{pa 1097 1573}%
\special{pa 1085 1563}%
\special{pa 1059 1545}%
\special{pa 1045 1536}%
\special{pa 1019 1518}%
\special{pa 977 1494}%
\special{pa 963 1487}%
\special{pa 948 1480}%
\special{pa 934 1473}%
\special{pa 919 1466}%
\special{pa 905 1460}%
\special{pa 890 1454}%
\special{pa 875 1449}%
\special{pa 860 1443}%
\special{pa 845 1438}%
\special{pa 829 1434}%
\special{pa 814 1429}%
\special{pa 798 1425}%
\special{pa 783 1421}%
\special{pa 767 1418}%
\special{pa 752 1414}%
\special{pa 736 1412}%
\special{pa 720 1409}%
\special{pa 672 1403}%
\special{pa 624 1400}%
\special{pa 576 1400}%
\special{pa 560 1401}%
\special{pa 545 1402}%
\special{pa 529 1403}%
\special{pa 465 1411}%
\special{pa 449 1414}%
\special{pa 434 1417}%
\special{pa 418 1421}%
\special{pa 403 1425}%
\special{pa 400 1425}%
\special{fp}%
\special{pa 400 2975}%
\special{pa 416 2978}%
\special{pa 431 2982}%
\special{pa 463 2988}%
\special{pa 478 2991}%
\special{pa 526 2997}%
\special{pa 574 3000}%
\special{pa 622 3000}%
\special{pa 670 2997}%
\special{pa 686 2995}%
\special{pa 702 2994}%
\special{pa 718 2991}%
\special{pa 733 2989}%
\special{pa 765 2983}%
\special{pa 780 2979}%
\special{pa 796 2976}%
\special{pa 811 2972}%
\special{pa 827 2967}%
\special{pa 842 2962}%
\special{pa 857 2958}%
\special{pa 872 2952}%
\special{pa 887 2947}%
\special{pa 917 2935}%
\special{pa 932 2928}%
\special{pa 960 2914}%
\special{pa 975 2907}%
\special{pa 1003 2891}%
\special{pa 1016 2883}%
\special{pa 1030 2875}%
\special{pa 1043 2866}%
\special{pa 1057 2857}%
\special{pa 1070 2848}%
\special{pa 1082 2838}%
\special{pa 1108 2818}%
\special{pa 1132 2798}%
\special{pa 1144 2787}%
\special{pa 1155 2776}%
\special{pa 1167 2765}%
\special{pa 1189 2741}%
\special{pa 1200 2730}%
\special{pa 1210 2717}%
\special{pa 1230 2693}%
\special{pa 1250 2667}%
\special{pa 1268 2641}%
\special{pa 1276 2627}%
\special{pa 1285 2614}%
\special{pa 1301 2586}%
\special{pa 1308 2572}%
\special{pa 1316 2558}%
\special{pa 1323 2543}%
\special{pa 1329 2529}%
\special{pa 1336 2514}%
\special{pa 1348 2484}%
\special{pa 1353 2469}%
\special{pa 1359 2454}%
\special{pa 1363 2439}%
\special{pa 1368 2424}%
\special{pa 1372 2408}%
\special{pa 1376 2393}%
\special{pa 1380 2377}%
\special{pa 1384 2362}%
\special{pa 1387 2346}%
\special{pa 1389 2330}%
\special{pa 1392 2314}%
\special{pa 1394 2298}%
\special{pa 1396 2283}%
\special{pa 1400 2219}%
\special{pa 1400 2187}%
\special{pa 1399 2171}%
\special{pa 1399 2155}%
\special{pa 1398 2139}%
\special{pa 1396 2123}%
\special{pa 1395 2107}%
\special{pa 1393 2091}%
\special{pa 1390 2075}%
\special{pa 1388 2059}%
\special{pa 1385 2044}%
\special{pa 1381 2028}%
\special{pa 1378 2012}%
\special{pa 1370 1982}%
\special{pa 1365 1966}%
\special{pa 1350 1921}%
\special{pa 1332 1876}%
\special{pa 1325 1862}%
\special{pa 1318 1847}%
\special{pa 1311 1833}%
\special{pa 1303 1819}%
\special{pa 1296 1805}%
\special{pa 1288 1791}%
\special{pa 1279 1777}%
\special{pa 1271 1764}%
\special{pa 1262 1751}%
\special{pa 1253 1737}%
\special{pa 1243 1724}%
\special{pa 1234 1712}%
\special{pa 1224 1699}%
\special{pa 1214 1687}%
\special{pa 1203 1675}%
\special{pa 1193 1663}%
\special{pa 1171 1639}%
\special{pa 1159 1628}%
\special{pa 1148 1617}%
\special{pa 1124 1595}%
\special{pa 1112 1585}%
\special{pa 1099 1575}%
\special{pa 1087 1565}%
\special{pa 1074 1556}%
\special{pa 1061 1546}%
\special{pa 1035 1528}%
\special{pa 1021 1520}%
\special{pa 1007 1511}%
\special{pa 993 1503}%
\special{pa 979 1496}%
\special{pa 965 1488}%
\special{pa 937 1474}%
\special{pa 922 1468}%
\special{pa 907 1461}%
\special{pa 892 1455}%
\special{pa 877 1450}%
\special{pa 862 1444}%
\special{pa 832 1434}%
\special{pa 817 1430}%
\special{pa 801 1426}%
\special{pa 786 1422}%
\special{pa 770 1418}%
\special{pa 754 1415}%
\special{pa 739 1412}%
\special{pa 723 1409}%
\special{pa 691 1405}%
\special{pa 675 1404}%
\special{pa 659 1402}%
\special{pa 627 1400}%
\special{pa 579 1400}%
\special{pa 531 1403}%
\special{pa 516 1404}%
\special{pa 484 1408}%
\special{pa 452 1414}%
\special{pa 437 1417}%
\special{pa 421 1420}%
\special{pa 405 1424}%
\special{pa 404 1424}%
\special{pa 403 1425}%
\special{pa 400 1425}%
\special{fp}%
% LINE 2 2 3 0 Black White  
% 4 600 2200 800 1420 800 1420 2200 2200
% 
\special{pn 8}%
\special{pa 600 2200}%
\special{pa 800 1420}%
\special{dt 0.045}%
\special{pa 800 1420}%
\special{pa 2200 2200}%
\special{dt 0.045}%
% LINE 2 2 3 0 Black White  
% 2 800 2200 800 1430
% 
\special{pn 8}%
\special{pa 800 2200}%
\special{pa 800 1430}%
\special{dt 0.045}%
% STR 2 0 3 0 Black White  
% 4 1400 2100 1400 2200 2 0 1 0
% $2c$
\put(14.0000,-22.0000){\makebox(0,0)[lb]{{\colorbox[named]{White}{$2c$}}}}%
% STR 2 0 3 0 Black White  
% 4 2200 2100 2200 2200 2 0 1 0
% $1$
\put(22.0000,-22.0000){\makebox(0,0)[lb]{{\colorbox[named]{White}{$1$}}}}%
% STR 2 0 3 0 Black White  
% 4 3800 2100 3800 2200 2 0 1 0
% $2$
\put(38.0000,-22.0000){\makebox(0,0)[lb]{{\colorbox[named]{White}{$2$}}}}%
% STR 2 0 3 0 Black White  
% 4 800 2100 800 2200 2 0 1 0
% $2c^2$
\put(8.0000,-22.0000){\makebox(0,0)[lb]{{\colorbox[named]{White}{$2c^2$}}}}%
% STR 2 0 3 0 Black White  
% 4 1840 1990 1840 2090 2 0 1 0
% $\pi-2\theta$
\put(18.4000,-20.9000){\makebox(0,0)[lb]{{\colorbox[named]{White}{$\pi-2\theta$}}}}%
% STR 2 0 3 0 Black White  
% 4 730 1850 730 1950 2 0 1 0
% $\theta$
\put(7.3000,-19.5000){\makebox(0,0)[lb]{{\colorbox[named]{White}{$\theta$}}}}%
\end{picture}}%
}
     \end{center}
この面積$S(t)$は,下の斜線部の面積$S_1$と$S_2$の合計である.すなわち,
     \begin{align}
     S(t)=S_1+S_2\label{1}
     \end{align}
     
     \begin{minipage}{0.5\hsize}
          \begin{center}
          \scalebox{.7}{%WinTpicVersion4.32a
{\unitlength 0.1in%
\begin{picture}(22.0000,16.0000)(6.0000,-26.0000)%
% FUNC 2 0 3 0 Black White  
% 9 600 1000 2800 2600 800 2400 1800 2400 800 1400 600 1000 2800 2600 50 4 0 2
% cos(t)///sin(t)///0///pi/3
\special{pn 8}%
\special{pa 1800 2400}%
\special{pa 1800 2369}%
\special{pa 1799 2366}%
\special{pa 1799 2346}%
\special{pa 1798 2343}%
\special{pa 1798 2331}%
\special{pa 1797 2329}%
\special{pa 1797 2318}%
\special{pa 1796 2316}%
\special{pa 1796 2306}%
\special{pa 1795 2304}%
\special{pa 1795 2295}%
\special{pa 1794 2293}%
\special{pa 1794 2287}%
\special{pa 1793 2285}%
\special{pa 1793 2279}%
\special{pa 1792 2277}%
\special{pa 1792 2270}%
\special{pa 1791 2268}%
\special{pa 1791 2264}%
\special{pa 1790 2262}%
\special{pa 1790 2256}%
\special{pa 1789 2254}%
\special{pa 1789 2250}%
\special{pa 1788 2248}%
\special{pa 1788 2244}%
\special{pa 1787 2241}%
\special{pa 1787 2237}%
\special{pa 1786 2235}%
\special{pa 1786 2231}%
\special{pa 1785 2229}%
\special{pa 1785 2225}%
\special{pa 1784 2223}%
\special{pa 1784 2221}%
\special{pa 1783 2219}%
\special{pa 1783 2215}%
\special{pa 1782 2213}%
\special{pa 1782 2211}%
\special{pa 1781 2208}%
\special{pa 1781 2204}%
\special{pa 1780 2202}%
\special{pa 1780 2200}%
\special{pa 1779 2198}%
\special{pa 1779 2194}%
\special{pa 1778 2192}%
\special{pa 1778 2190}%
\special{pa 1777 2188}%
\special{pa 1777 2186}%
\special{pa 1776 2184}%
\special{pa 1776 2182}%
\special{pa 1775 2180}%
\special{pa 1775 2176}%
\special{pa 1774 2174}%
\special{pa 1774 2172}%
\special{pa 1773 2170}%
\special{pa 1773 2168}%
\special{pa 1772 2166}%
\special{pa 1772 2163}%
\special{pa 1771 2161}%
\special{pa 1771 2159}%
\special{pa 1770 2157}%
\special{pa 1770 2155}%
\special{pa 1769 2153}%
\special{pa 1769 2151}%
\special{pa 1768 2149}%
\special{pa 1768 2147}%
\special{pa 1766 2143}%
\special{pa 1766 2141}%
\special{pa 1765 2139}%
\special{pa 1765 2137}%
\special{pa 1764 2135}%
\special{pa 1764 2133}%
\special{pa 1763 2131}%
\special{pa 1763 2129}%
\special{pa 1761 2125}%
\special{pa 1761 2123}%
\special{pa 1760 2121}%
\special{pa 1760 2119}%
\special{pa 1759 2117}%
\special{pa 1759 2115}%
\special{pa 1757 2111}%
\special{pa 1757 2109}%
\special{pa 1755 2105}%
\special{pa 1755 2103}%
\special{pa 1754 2101}%
\special{pa 1754 2099}%
\special{pa 1752 2095}%
\special{pa 1752 2093}%
\special{pa 1750 2089}%
\special{pa 1750 2087}%
\special{pa 1748 2083}%
\special{pa 1748 2081}%
\special{pa 1746 2077}%
\special{pa 1746 2075}%
\special{pa 1744 2071}%
\special{pa 1744 2069}%
\special{pa 1742 2065}%
\special{pa 1742 2063}%
\special{pa 1739 2057}%
\special{pa 1739 2055}%
\special{pa 1737 2051}%
\special{pa 1737 2049}%
\special{pa 1734 2044}%
\special{pa 1734 2042}%
\special{pa 1731 2036}%
\special{pa 1731 2034}%
\special{pa 1727 2026}%
\special{pa 1727 2024}%
\special{pa 1723 2016}%
\special{pa 1723 2014}%
\special{pa 1722 2012}%
\special{pa 1721 2011}%
\special{pa 1719 2007}%
\special{pa 1719 2005}%
\special{pa 1714 1995}%
\special{pa 1714 1993}%
\special{pa 1712 1989}%
\special{pa 1711 1988}%
\special{pa 1707 1980}%
\special{pa 1707 1978}%
\special{pa 1698 1961}%
\special{pa 1698 1959}%
\special{pa 1695 1953}%
\special{pa 1694 1952}%
\special{pa 1681 1927}%
\special{pa 1680 1926}%
\special{pa 1669 1905}%
\special{pa 1668 1904}%
\special{pa 1659 1887}%
\special{pa 1656 1884}%
\special{pa 1653 1878}%
\special{pa 1652 1877}%
\special{pa 1648 1869}%
\special{pa 1645 1866}%
\special{pa 1643 1862}%
\special{pa 1642 1861}%
\special{pa 1639 1855}%
\special{pa 1636 1852}%
\special{pa 1631 1843}%
\special{pa 1628 1840}%
\special{pa 1625 1834}%
\special{pa 1622 1831}%
\special{pa 1621 1829}%
\special{pa 1620 1828}%
\special{pa 1618 1824}%
\special{pa 1615 1821}%
\special{pa 1613 1817}%
\special{pa 1611 1816}%
\special{pa 1609 1812}%
\special{pa 1608 1811}%
\special{pa 1607 1809}%
\special{pa 1605 1807}%
\special{pa 1602 1802}%
\special{pa 1599 1799}%
\special{pa 1597 1795}%
\special{pa 1595 1794}%
\special{pa 1593 1790}%
\special{pa 1591 1789}%
\special{pa 1589 1785}%
\special{pa 1586 1782}%
\special{pa 1585 1780}%
\special{pa 1581 1776}%
\special{pa 1580 1774}%
\special{pa 1577 1771}%
\special{pa 1576 1769}%
\special{pa 1573 1766}%
\special{pa 1572 1764}%
\special{pa 1570 1763}%
\special{pa 1568 1759}%
\special{pa 1566 1758}%
\special{pa 1565 1756}%
\special{pa 1562 1753}%
\special{pa 1561 1751}%
\special{pa 1557 1747}%
\special{pa 1556 1745}%
\special{pa 1550 1739}%
\special{pa 1549 1737}%
\special{pa 1547 1736}%
\special{pa 1545 1732}%
\special{pa 1543 1731}%
\special{pa 1542 1729}%
\special{pa 1540 1728}%
\special{pa 1539 1726}%
\special{pa 1537 1725}%
\special{pa 1536 1723}%
\special{pa 1533 1720}%
\special{pa 1532 1718}%
\special{pa 1530 1717}%
\special{pa 1529 1715}%
\special{pa 1513 1699}%
\special{pa 1512 1697}%
\special{pa 1510 1696}%
\special{pa 1509 1694}%
\special{pa 1507 1693}%
\special{pa 1506 1691}%
\special{pa 1504 1690}%
\special{pa 1503 1688}%
\special{pa 1501 1687}%
\special{pa 1485 1671}%
\special{pa 1483 1670}%
\special{pa 1474 1661}%
\special{pa 1472 1660}%
\special{pa 1471 1658}%
\special{pa 1469 1657}%
\special{pa 1468 1655}%
\special{pa 1464 1653}%
\special{pa 1463 1651}%
\special{pa 1461 1650}%
\special{pa 1460 1648}%
\special{pa 1458 1647}%
\special{pa 1455 1644}%
\special{pa 1453 1643}%
\special{pa 1449 1639}%
\special{pa 1447 1638}%
\special{pa 1441 1632}%
\special{pa 1439 1631}%
\special{pa 1436 1628}%
\special{pa 1434 1627}%
\special{pa 1433 1625}%
\special{pa 1429 1623}%
\special{pa 1426 1620}%
\special{pa 1424 1619}%
\special{pa 1421 1616}%
\special{pa 1419 1615}%
\special{pa 1415 1611}%
\special{pa 1411 1609}%
\special{pa 1410 1607}%
\special{pa 1406 1605}%
\special{pa 1405 1603}%
\special{pa 1401 1601}%
\special{pa 1398 1598}%
\special{pa 1393 1595}%
\special{pa 1391 1593}%
\special{pa 1389 1592}%
\special{pa 1388 1591}%
\special{pa 1384 1589}%
\special{pa 1383 1587}%
\special{pa 1379 1585}%
\special{pa 1376 1582}%
\special{pa 1372 1580}%
\special{pa 1371 1579}%
\special{pa 1369 1578}%
\special{pa 1366 1575}%
\special{pa 1360 1572}%
\special{pa 1357 1569}%
\special{pa 1348 1564}%
\special{pa 1345 1561}%
\special{pa 1339 1558}%
\special{pa 1338 1557}%
\special{pa 1334 1555}%
\special{pa 1332 1553}%
\special{pa 1330 1552}%
\special{pa 1329 1551}%
\special{pa 1316 1544}%
\special{pa 1313 1541}%
\special{pa 1305 1537}%
\special{pa 1304 1536}%
\special{pa 1300 1534}%
\special{fp}%
% FUNC 2 0 3 0 Black White  
% 10 600 1000 2800 2600 800 2400 1800 2400 800 1400 800 1000 1800 2600 0 4 0 0 0 0
% 0
\special{pn 8}%
\special{pn 8}%
\special{pa 600 2400}%
\special{pa 609 2400}%
\special{ip}%
\special{pa 650 2400}%
\special{pa 659 2400}%
\special{ip}%
\special{pa 700 2400}%
\special{pa 709 2400}%
\special{ip}%
\special{pa 750 2400}%
\special{pa 759 2400}%
\special{ip}%
\special{ip}%
\special{pa 800 2400}%
\special{pa 1800 2400}%
\special{fp}%
\special{pn 8}%
\special{pa 1808 2400}%
\special{pa 1845 2400}%
\special{ip}%
\special{pa 1854 2400}%
\special{pa 1891 2400}%
\special{ip}%
\special{pa 1899 2400}%
\special{pa 1936 2400}%
\special{ip}%
\special{pa 1944 2400}%
\special{pa 1982 2400}%
\special{ip}%
\special{pa 1990 2400}%
\special{pa 2027 2400}%
\special{ip}%
\special{pa 2035 2400}%
\special{pa 2073 2400}%
\special{ip}%
\special{pa 2081 2400}%
\special{pa 2118 2400}%
\special{ip}%
\special{pa 2126 2400}%
\special{pa 2164 2400}%
\special{ip}%
\special{pa 2172 2400}%
\special{pa 2209 2400}%
\special{ip}%
\special{pa 2217 2400}%
\special{pa 2255 2400}%
\special{ip}%
\special{pa 2263 2400}%
\special{pa 2300 2400}%
\special{ip}%
\special{pa 2308 2400}%
\special{pa 2345 2400}%
\special{ip}%
\special{pa 2354 2400}%
\special{pa 2391 2400}%
\special{ip}%
\special{pa 2399 2400}%
\special{pa 2436 2400}%
\special{ip}%
\special{pa 2444 2400}%
\special{pa 2482 2400}%
\special{ip}%
\special{pa 2490 2400}%
\special{pa 2527 2400}%
\special{ip}%
\special{pa 2535 2400}%
\special{pa 2573 2400}%
\special{ip}%
\special{pa 2581 2400}%
\special{pa 2618 2400}%
\special{ip}%
\special{pa 2626 2400}%
\special{pa 2664 2400}%
\special{ip}%
\special{pa 2672 2400}%
\special{pa 2709 2400}%
\special{ip}%
\special{pa 2717 2400}%
\special{pa 2755 2400}%
\special{ip}%
\special{pa 2763 2400}%
\special{pa 2800 2400}%
\special{ip}%
% LINE 2 0 3 0 Black White  
% 4 800 2400 1290 1530 1290 1530 1290 2400
% 
\special{pn 8}%
\special{pa 800 2400}%
\special{pa 1290 1530}%
\special{fp}%
\special{pa 1290 1530}%
\special{pa 1290 2400}%
\special{fp}%
% LINE 3 0 3 0 Black White  
% 44 1680 1920 1290 2310 1700 1960 1290 2370 1710 2010 1320 2400 1730 2050 1380 2400 1750 2090 1440 2400 1760 2140 1500 2400 1770 2190 1560 2400 1790 2230 1620 2400 1790 2290 1680 2400 1800 2340 1740 2400 1660 1880 1290 2250 1630 1850 1290 2190 1610 1810 1290 2130 1580 1780 1290 2070 1550 1750 1290 2010 1520 1720 1290 1950 1490 1690 1290 1890 1460 1660 1290 1830 1430 1630 1290 1770 1400 1600 1290 1710 1360 1580 1290 1650 1330 1550 1290 1590
% 
\special{pn 4}%
\special{pa 1680 1920}%
\special{pa 1290 2310}%
\special{fp}%
\special{pa 1700 1960}%
\special{pa 1290 2370}%
\special{fp}%
\special{pa 1710 2010}%
\special{pa 1320 2400}%
\special{fp}%
\special{pa 1730 2050}%
\special{pa 1380 2400}%
\special{fp}%
\special{pa 1750 2090}%
\special{pa 1440 2400}%
\special{fp}%
\special{pa 1760 2140}%
\special{pa 1500 2400}%
\special{fp}%
\special{pa 1770 2190}%
\special{pa 1560 2400}%
\special{fp}%
\special{pa 1790 2230}%
\special{pa 1620 2400}%
\special{fp}%
\special{pa 1790 2290}%
\special{pa 1680 2400}%
\special{fp}%
\special{pa 1800 2340}%
\special{pa 1740 2400}%
\special{fp}%
\special{pa 1660 1880}%
\special{pa 1290 2250}%
\special{fp}%
\special{pa 1630 1850}%
\special{pa 1290 2190}%
\special{fp}%
\special{pa 1610 1810}%
\special{pa 1290 2130}%
\special{fp}%
\special{pa 1580 1780}%
\special{pa 1290 2070}%
\special{fp}%
\special{pa 1550 1750}%
\special{pa 1290 2010}%
\special{fp}%
\special{pa 1520 1720}%
\special{pa 1290 1950}%
\special{fp}%
\special{pa 1490 1690}%
\special{pa 1290 1890}%
\special{fp}%
\special{pa 1460 1660}%
\special{pa 1290 1830}%
\special{fp}%
\special{pa 1430 1630}%
\special{pa 1290 1770}%
\special{fp}%
\special{pa 1400 1600}%
\special{pa 1290 1710}%
\special{fp}%
\special{pa 1360 1580}%
\special{pa 1290 1650}%
\special{fp}%
\special{pa 1330 1550}%
\special{pa 1290 1590}%
\special{fp}%
% STR 2 0 3 0 Black White  
% 4 1830 2370 1830 2470 5 0 1 0
% $2c$
\put(18.3000,-24.7000){\makebox(0,0){{\colorbox[named]{White}{$2c$}}}}%
% STR 2 0 3 0 Black White  
% 4 1100 2160 1100 2260 5 0 1 0
% $\theta$
\put(11.0000,-22.6000){\makebox(0,0){{\colorbox[named]{White}{$\theta$}}}}%
% STR 2 0 3 0 Black White  
% 4 1540 2000 1540 2100 5 0 1 0
% $S_1/2$
\put(15.4000,-21.0000){\makebox(0,0){{\colorbox[named]{White}{$S_1/2$}}}}%
\end{picture}}%
}
          \end{center}
     \end{minipage}
     \begin{minipage}{0.5\hsize}
           \begin{center}
           \scalebox{.7}{%WinTpicVersion4.32a
{\unitlength 0.1in%
\begin{picture}(20.0000,16.0000)(4.0000,-30.0000)%
% FUNC 2 0 3 0 Black White  
% 9 400 1400 2400 3000 600 2800 1600 2800 600 1800 400 1400 2400 3000 50 4 0 2
% 1+cos(t)///sin(t)///2pi/3///pi
\special{pn 8}%
\special{pa 1100 1934}%
\special{pa 1096 1936}%
\special{pa 1095 1937}%
\special{pa 1087 1941}%
\special{pa 1084 1944}%
\special{pa 1078 1947}%
\special{pa 1077 1948}%
\special{pa 1069 1952}%
\special{pa 1066 1955}%
\special{pa 1062 1957}%
\special{pa 1061 1958}%
\special{pa 1055 1961}%
\special{pa 1052 1964}%
\special{pa 1043 1969}%
\special{pa 1040 1972}%
\special{pa 1034 1975}%
\special{pa 1031 1978}%
\special{pa 1029 1979}%
\special{pa 1028 1980}%
\special{pa 1024 1982}%
\special{pa 1021 1985}%
\special{pa 1017 1987}%
\special{pa 1016 1989}%
\special{pa 1012 1991}%
\special{pa 1011 1992}%
\special{pa 1009 1993}%
\special{pa 1007 1995}%
\special{pa 1002 1998}%
\special{pa 999 2001}%
\special{pa 995 2003}%
\special{pa 994 2005}%
\special{pa 990 2007}%
\special{pa 989 2009}%
\special{pa 985 2011}%
\special{pa 982 2014}%
\special{pa 980 2015}%
\special{pa 976 2019}%
\special{pa 974 2020}%
\special{pa 971 2023}%
\special{pa 967 2025}%
\special{pa 966 2027}%
\special{pa 964 2028}%
\special{pa 961 2031}%
\special{pa 959 2032}%
\special{pa 958 2034}%
\special{pa 956 2035}%
\special{pa 953 2038}%
\special{pa 951 2039}%
\special{pa 947 2043}%
\special{pa 945 2044}%
\special{pa 939 2050}%
\special{pa 937 2051}%
\special{pa 936 2053}%
\special{pa 932 2055}%
\special{pa 931 2057}%
\special{pa 929 2058}%
\special{pa 928 2060}%
\special{pa 926 2061}%
\special{pa 917 2070}%
\special{pa 915 2071}%
\special{pa 899 2087}%
\special{pa 897 2088}%
\special{pa 896 2090}%
\special{pa 894 2091}%
\special{pa 893 2093}%
\special{pa 891 2094}%
\special{pa 890 2096}%
\special{pa 888 2097}%
\special{pa 887 2099}%
\special{pa 871 2115}%
\special{pa 870 2117}%
\special{pa 861 2126}%
\special{pa 860 2128}%
\special{pa 858 2129}%
\special{pa 857 2131}%
\special{pa 855 2132}%
\special{pa 853 2136}%
\special{pa 851 2137}%
\special{pa 850 2139}%
\special{pa 844 2145}%
\special{pa 843 2147}%
\special{pa 839 2151}%
\special{pa 838 2153}%
\special{pa 835 2156}%
\special{pa 834 2158}%
\special{pa 832 2159}%
\special{pa 831 2161}%
\special{pa 828 2164}%
\special{pa 827 2166}%
\special{pa 825 2167}%
\special{pa 823 2171}%
\special{pa 820 2174}%
\special{pa 819 2176}%
\special{pa 815 2180}%
\special{pa 814 2182}%
\special{pa 811 2185}%
\special{pa 809 2189}%
\special{pa 807 2190}%
\special{pa 805 2194}%
\special{pa 803 2195}%
\special{pa 801 2199}%
\special{pa 798 2202}%
\special{pa 795 2207}%
\special{pa 793 2209}%
\special{pa 792 2211}%
\special{pa 791 2212}%
\special{pa 789 2216}%
\special{pa 787 2217}%
\special{pa 785 2221}%
\special{pa 782 2224}%
\special{pa 780 2228}%
\special{pa 779 2229}%
\special{pa 778 2231}%
\special{pa 775 2234}%
\special{pa 772 2240}%
\special{pa 769 2243}%
\special{pa 764 2252}%
\special{pa 761 2255}%
\special{pa 758 2261}%
\special{pa 757 2262}%
\special{pa 755 2266}%
\special{pa 752 2269}%
\special{pa 748 2277}%
\special{pa 747 2278}%
\special{pa 744 2284}%
\special{pa 741 2287}%
\special{pa 732 2304}%
\special{pa 731 2305}%
\special{pa 720 2326}%
\special{pa 719 2327}%
\special{pa 706 2352}%
\special{pa 705 2353}%
\special{pa 702 2359}%
\special{pa 702 2361}%
\special{pa 693 2378}%
\special{pa 693 2380}%
\special{pa 689 2388}%
\special{pa 688 2389}%
\special{pa 686 2393}%
\special{pa 686 2395}%
\special{pa 681 2405}%
\special{pa 681 2407}%
\special{pa 679 2411}%
\special{pa 678 2412}%
\special{pa 677 2414}%
\special{pa 677 2416}%
\special{pa 673 2424}%
\special{pa 673 2426}%
\special{pa 669 2434}%
\special{pa 669 2436}%
\special{pa 666 2442}%
\special{pa 666 2444}%
\special{pa 664 2448}%
\special{pa 663 2449}%
\special{pa 663 2451}%
\special{pa 661 2455}%
\special{pa 661 2457}%
\special{pa 658 2463}%
\special{pa 658 2465}%
\special{pa 656 2469}%
\special{pa 656 2471}%
\special{pa 654 2475}%
\special{pa 654 2477}%
\special{pa 652 2481}%
\special{pa 652 2483}%
\special{pa 650 2487}%
\special{pa 650 2489}%
\special{pa 648 2493}%
\special{pa 648 2495}%
\special{pa 646 2499}%
\special{pa 646 2501}%
\special{pa 645 2503}%
\special{pa 645 2505}%
\special{pa 643 2509}%
\special{pa 643 2511}%
\special{pa 641 2515}%
\special{pa 641 2517}%
\special{pa 640 2519}%
\special{pa 640 2521}%
\special{pa 639 2523}%
\special{pa 639 2525}%
\special{pa 637 2529}%
\special{pa 637 2531}%
\special{pa 636 2533}%
\special{pa 636 2535}%
\special{pa 635 2537}%
\special{pa 635 2539}%
\special{pa 634 2541}%
\special{pa 634 2543}%
\special{pa 632 2547}%
\special{pa 632 2549}%
\special{pa 631 2551}%
\special{pa 631 2553}%
\special{pa 630 2555}%
\special{pa 630 2557}%
\special{pa 629 2559}%
\special{pa 629 2561}%
\special{pa 628 2564}%
\special{pa 628 2566}%
\special{pa 627 2568}%
\special{pa 627 2570}%
\special{pa 626 2572}%
\special{pa 626 2574}%
\special{pa 625 2576}%
\special{pa 625 2580}%
\special{pa 624 2582}%
\special{pa 624 2584}%
\special{pa 623 2586}%
\special{pa 623 2588}%
\special{pa 622 2590}%
\special{pa 622 2592}%
\special{pa 621 2594}%
\special{pa 621 2598}%
\special{pa 620 2600}%
\special{pa 620 2602}%
\special{pa 619 2604}%
\special{pa 619 2609}%
\special{pa 618 2611}%
\special{pa 618 2613}%
\special{pa 617 2615}%
\special{pa 617 2619}%
\special{pa 616 2621}%
\special{pa 616 2623}%
\special{pa 615 2625}%
\special{pa 615 2629}%
\special{pa 614 2631}%
\special{pa 614 2635}%
\special{pa 613 2637}%
\special{pa 613 2641}%
\special{pa 612 2644}%
\special{pa 612 2648}%
\special{pa 611 2650}%
\special{pa 611 2654}%
\special{pa 610 2656}%
\special{pa 610 2662}%
\special{pa 609 2664}%
\special{pa 609 2668}%
\special{pa 608 2671}%
\special{pa 608 2677}%
\special{pa 607 2679}%
\special{pa 607 2685}%
\special{pa 606 2687}%
\special{pa 606 2693}%
\special{pa 605 2695}%
\special{pa 605 2704}%
\special{pa 604 2706}%
\special{pa 604 2716}%
\special{pa 603 2718}%
\special{pa 603 2729}%
\special{pa 602 2731}%
\special{pa 602 2743}%
\special{pa 601 2746}%
\special{pa 601 2766}%
\special{pa 600 2769}%
\special{pa 600 2800}%
\special{fp}%
% FUNC 2 0 3 0 Black White  
% 10 400 1400 2400 3000 600 2800 1600 2800 600 1800 600 1400 1600 3000 0 4 0 0 0 0
% 0
\special{pn 8}%
\special{pn 8}%
\special{pa 400 2800}%
\special{pa 409 2800}%
\special{ip}%
\special{pa 450 2800}%
\special{pa 459 2800}%
\special{ip}%
\special{pa 500 2800}%
\special{pa 509 2800}%
\special{ip}%
\special{pa 550 2800}%
\special{pa 559 2800}%
\special{ip}%
\special{ip}%
\special{pa 600 2800}%
\special{pa 1600 2800}%
\special{fp}%
\special{pn 8}%
\special{pa 1608 2800}%
\special{pa 1647 2800}%
\special{ip}%
\special{pa 1655 2800}%
\special{pa 1694 2800}%
\special{ip}%
\special{pa 1702 2800}%
\special{pa 1741 2800}%
\special{ip}%
\special{pa 1750 2800}%
\special{pa 1788 2800}%
\special{ip}%
\special{pa 1797 2800}%
\special{pa 1835 2800}%
\special{ip}%
\special{pa 1844 2800}%
\special{pa 1882 2800}%
\special{ip}%
\special{pa 1891 2800}%
\special{pa 1929 2800}%
\special{ip}%
\special{pa 1938 2800}%
\special{pa 1976 2800}%
\special{ip}%
\special{pa 1985 2800}%
\special{pa 2024 2800}%
\special{ip}%
\special{pa 2032 2800}%
\special{pa 2071 2800}%
\special{ip}%
\special{pa 2079 2800}%
\special{pa 2118 2800}%
\special{ip}%
\special{pa 2126 2800}%
\special{pa 2165 2800}%
\special{ip}%
\special{pa 2173 2800}%
\special{pa 2212 2800}%
\special{ip}%
\special{pa 2220 2800}%
\special{pa 2259 2800}%
\special{ip}%
\special{pa 2267 2800}%
\special{pa 2306 2800}%
\special{ip}%
\special{pa 2314 2800}%
\special{pa 2353 2800}%
\special{ip}%
\special{pa 2361 2800}%
\special{pa 2400 2800}%
\special{ip}%
% LINE 2 0 3 0 Black White  
% 4 1600 2800 1110 1930 1110 1930 1110 2800
% 
\special{pn 8}%
\special{pa 1600 2800}%
\special{pa 1110 1930}%
\special{fp}%
\special{pa 1110 1930}%
\special{pa 1110 2800}%
\special{fp}%
% LINE 3 0 3 0 Black White  
% 28 1110 2310 620 2800 1110 2250 600 2760 1110 2190 600 2700 1110 2130 610 2630 1110 2070 630 2550 1110 2010 660 2460 1110 1950 700 2360 1110 2370 680 2800 1110 2430 740 2800 1110 2490 800 2800 1110 2550 860 2800 1110 2610 920 2800 1110 2670 980 2800 1110 2730 1040 2800
% 
\special{pn 4}%
\special{pa 1110 2310}%
\special{pa 620 2800}%
\special{fp}%
\special{pa 1110 2250}%
\special{pa 600 2760}%
\special{fp}%
\special{pa 1110 2190}%
\special{pa 600 2700}%
\special{fp}%
\special{pa 1110 2130}%
\special{pa 610 2630}%
\special{fp}%
\special{pa 1110 2070}%
\special{pa 630 2550}%
\special{fp}%
\special{pa 1110 2010}%
\special{pa 660 2460}%
\special{fp}%
\special{pa 1110 1950}%
\special{pa 700 2360}%
\special{fp}%
\special{pa 1110 2370}%
\special{pa 680 2800}%
\special{fp}%
\special{pa 1110 2430}%
\special{pa 740 2800}%
\special{fp}%
\special{pa 1110 2490}%
\special{pa 800 2800}%
\special{fp}%
\special{pa 1110 2550}%
\special{pa 860 2800}%
\special{fp}%
\special{pa 1110 2610}%
\special{pa 920 2800}%
\special{fp}%
\special{pa 1110 2670}%
\special{pa 980 2800}%
\special{fp}%
\special{pa 1110 2730}%
\special{pa 1040 2800}%
\special{fp}%
% STR 2 0 3 0 Black White  
% 4 1420 2560 1420 2660 5 0 1 0
% $\pi-2\theta$
\put(14.2000,-26.6000){\makebox(0,0){{\colorbox[named]{White}{$\pi-2\theta$}}}}%
% STR 2 0 3 0 Black White  
% 4 930 2290 930 2390 5 0 1 0
% $S_2/2$
\put(9.3000,-23.9000){\makebox(0,0){{\colorbox[named]{White}{$S_2/2$}}}}%
\end{picture}}%
}
           \end{center}
     \end{minipage}
各々計算すると,
     \begin{align*}
     S_1&=\theta(2c)^2-(2c)^2cs=4c^2\theta-4c^3s \\
     S_2&=(\pi-2\theta)-\sin(\pi-2\theta)\cos(\pi-2\theta) \\
     &=\pi-2\theta+2cs(1+2c^2) \\
     &=\pi-2\theta+2cs+4c^3s
     \end{align*}
であるから,\eqref{1}に代入して,     
     \begin{align}
     S&=(4c^2-2)\theta-4c^3s+\pi+2cs+4c^3s \nonumber\\
     &=2\theta(1-2s^2)+\pi-2cs \label{2}
     \end{align}
である.$\cdots$((1)の答) 

従って求める体積$V$として,
     \begin{align*}
     V&=\int_0^1S(t)dt \\
     &=\int_0^{\pi/2}S(t)\frac{dt}{d\theta}d\theta \\
     &=\int_0^{\pi/2}S(t)sd\theta \\
     &=\int_0^{\pi/2}(2s\theta-4s^3\theta+\pi s-2cs^2)d\theta &\left(\because\eqref{2}\right)
     \end{align*}
である.各項計算する.
     \begin{align*}
     &A=\int_0^{\pi/2}s\theta d\theta=2\left[-c\theta+s\right]_0^{\pi/2}=1 \\
     &4\int_0^{\pi/2}s^3\theta d\theta=\int_0^{\pi/2}(3s-\sin3\theta)\theta d\theta \\
     &=3A-\left[\frac{-1}{3}\theta\cos3\theta+\frac{1}{9}\sin3\theta\right]_0^{\pi/2} \\
     &=3+\frac{1}{9}=\frac{28}{9} \\
     &\int_0^{\pi/2}sd\theta=1 \\
     &\int_0^{\pi/2}cs^2d\theta=\left[\frac{1}{3}s^3\right]_0^{\pi/2}=\frac{1}{3}
     \end{align*}
これを代入して
     \begin{align*}
     V=2-\frac{28}{9}+\pi-\frac{2}{3}=\pi-\frac{16}{9}
     \end{align*}     
である.$\cdots$((2)の答)     
         
\newpage
\end{multicols}
\end{document}