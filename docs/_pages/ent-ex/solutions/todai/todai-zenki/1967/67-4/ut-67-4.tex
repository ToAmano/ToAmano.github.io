\documentclass[a4j]{jarticle}
\usepackage{amsmath}
\usepackage{ascmac}
\usepackage{amssymb}
\usepackage{enumerate}
\usepackage{multicol}
\usepackage{framed}
\usepackage{fancyhdr}
\usepackage{latexsym}
\usepackage{indent}
\usepackage{cases}
\usepackage[dvips]{graphicx}
\usepackage{color}
\allowdisplaybreaks
\pagestyle{fancy}
\lhead{}
\chead{}
\rhead{東京大学前期$1967$年$4$番}
\begin{document}
%分数関係


\def\tfrac#1#2{{\textstyle\frac{#1}{#2}}} %数式中で文中表示の分数を使う時


%Σ関係

\def\dsum#1#2{{\displaystyle\sum_{#1}^{#2}}} %文中で数式表示のΣを使う時


%ベクトル関係


\def\vector#1{\overrightarrow{#1}}  %ベクトルを表現したいとき(aベクトルを表現するときは\ver
\def\norm#1{|\overrightarrow{#1}|} %ベクトルの絶対値
\def\vtwo#1#2{ \left(%
      \begin{array}{c}%
      #1 \\%
      #2 \\%
      \end{array}%
      \right) }                        %2次元ベクトル成分表示
      
      \def\vthree#1#2#3{ \left(
      \begin{array}{c}
      #1 \\
      #2 \\
      #3 \\
      \end{array}
      \right) }                        %3次元ベクトル成分表示



%数列関係


\def\an#1{\verb|{|$#1$\verb|}|}


%極限関係

\def\limit#1#2{\stackrel{#1 \to #2}{\longrightarrow}}   %等式変形からの極限
\def\dlim#1#2{{\displaystyle \lim_{#1\to#2}}} %文中で数式表示の極限を使う



%積分関係

\def\dint#1#2{{\displaystyle \int_{#1}^{#2}}} %文中で数式表示の積分を使う時

\def\ne{\nearrow}
\def\se{\searrow}
\def\nw{\nwarrow}
\def\ne{\nearrow}


%便利なやつ

\def\case#1#2{%
 \[\left\{%
 \begin{array}{l}%
 #1 \\%
 #2%
 \end{array}%
 \right.\] }                           %場合分け
 
\def\1{$\cos\theta=c$,$\sin\theta=s$とおく.}  %cs表示を与える前書きシータ
\def\2{$\cos t=c$,$\sin t=s$とおく.}     %cs表示を与える前書きt
\def\3{$\cos x=c$,$\sin x=s$とおく.}                %cs表示を与える前書きx

\def\fig#1#2#3 {%
\begin{wrapfigure}[#1]{r}{#2 zw}%
\vspace*{-1zh}%
\input{#3}%
\end{wrapfigure} }           %絵の挿入


\def\a{\alpha}   %ギリシャ文字
\def\b{\beta}
\def\g{\gamma}

%問題番号のためのマクロ

\newcounter{nombre} %必須
\renewcommand{\thenombre}{\arabic{nombre}} %任意
\setcounter{nombre}{2} %任意
\newcounter{nombresub}[nombre] %親子関係を定義
\renewcommand{\thenombresub}{\arabic{nombresub}} %任意
\setcounter{nombresub}{0} %任意
\newcommand{\prob}[1][]{\refstepcounter{nombre}#1[問題 \thenombre]}
\newcommand{\probsub}[1][]{\refstepcounter{nombresub}#1(\thenombresub)}


%1-1みたいなカウンタ(todaiとtodaia)
\newcounter{todai}
\setcounter{todai}{0}
\newcounter{todaisub}[todai] 
\setcounter{todaisub}{0} 
\newcommand{\todai}[1][]{\refstepcounter{todai}#1 \thetodai-\thetodaisub}
\newcommand{\todaib}[1][]{\refstepcounter{todai}#1\refstepcounter{todaisub}#1 {\bf [問題 \thetodai.\thetodaisub]}}
\newcommand{\todaia}[1][]{\refstepcounter{todaisub}#1 {\bf [問題 \thetodai.\thetodaisub]}}


     \begin{oframed}
     方程式$x^2-xy+y^2=3$の表す曲線の略図をえがき,その第一象限にある部分が$x$軸,$y$軸と囲む図形の面積を求めよ.
     \end{oframed}

\setlength{\columnseprule}{0.4pt}
\begin{multicols}{2}
{\bf[解]} 題意の曲線$C$とし,$C$を負方向に$\pi/4$だけ回転した図形$C'$とする.$C(X,Y)$が$C'(x,y)$に移るとする.
 $e(\theta)=\cos\theta+i\sin\theta$として,
      \begin{align*}
      X+iY&=e\left(\frac{\pi}{4}\right)(x+iy) \\
      &=\frac{\sqrt{2}}{2}(1+i)(x+iy) 
      \end{align*}
であるから,
     \begin{align*}
      &X=\dfrac{\sqrt{2}}{2}(x-y) &
      Y=\dfrac{\sqrt{2}}{2}(x+y)
      \end{align*}
これを$C$の方程式に代入して
     \begin{align*}
     &(x-y)^2-(x+y)(x-y)+(x+y)^2=6 \\
     \therefore \,\ &x^2+3y^2=6
     \end{align*}
故に$C'$は短半径$\sqrt{2}$,長半径$\sqrt{6}$の楕円である.これを正方向に$\pi/4$回転させたものが$C$であるから,$C$の概形は以下.
     \begin{center}
     \scalebox{.5}{%WinTpicVersion4.32a
{\unitlength 0.1in%
\begin{picture}(24.0000,24.0000)(4.0000,-28.0000)%
% STR 2 0 3 0 Black White  
% 4 1590 1597 1590 1610 4 400 0 0
% O
\put(15.9000,-16.1000){\makebox(0,0)[rt]{O}}%
% STR 2 0 3 0 Black White  
% 4 1560 387 1560 400 4 400 0 0
% $y$
\put(15.6000,-4.0000){\makebox(0,0)[rt]{$y$}}%
% STR 2 0 3 0 Black White  
% 4 2800 1627 2800 1640 4 400 0 0
% $x$
\put(28.0000,-16.4000){\makebox(0,0)[rt]{$x$}}%
% VECTOR 2 0 3 0 Black White  
% 2 1600 2800 1600 400
% 
\special{pn 8}%
\special{pa 1600 2800}%
\special{pa 1600 400}%
\special{fp}%
\special{sh 1}%
\special{pa 1600 400}%
\special{pa 1580 467}%
\special{pa 1600 453}%
\special{pa 1620 467}%
\special{pa 1600 400}%
\special{fp}%
% VECTOR 2 0 3 0 Black White  
% 2 400 1600 2800 1600
% 
\special{pn 8}%
\special{pa 400 1600}%
\special{pa 2800 1600}%
\special{fp}%
\special{sh 1}%
\special{pa 2800 1600}%
\special{pa 2733 1580}%
\special{pa 2747 1600}%
\special{pa 2733 1620}%
\special{pa 2800 1600}%
\special{fp}%
% FUNC 2 0 3 0 Black White  
% 10 400 400 2800 2800 1600 1600 2000 1600 1600 1200 0 0 0 0 10 2 0 4 1 0
% ///x^2-xy+y^2=6///-3///3
\special{pn 8}%
\special{pa 2700 1279}%
\special{pa 2700 1278}%
\special{pa 2701 1277}%
\special{pa 2701 1273}%
\special{pa 2702 1272}%
\special{pa 2703 1265}%
\special{pa 2704 1264}%
\special{pa 2704 1261}%
\special{pa 2705 1260}%
\special{pa 2705 1256}%
\special{pa 2706 1255}%
\special{pa 2706 1252}%
\special{pa 2707 1251}%
\special{pa 2708 1242}%
\special{pa 2709 1241}%
\special{pa 2709 1237}%
\special{pa 2710 1236}%
\special{pa 2710 1233}%
\special{pa 2711 1232}%
\special{pa 2711 1228}%
\special{pa 2712 1227}%
\special{pa 2712 1222}%
\special{pa 2713 1221}%
\special{pa 2714 1212}%
\special{pa 2715 1211}%
\special{pa 2715 1206}%
\special{pa 2716 1205}%
\special{pa 2716 1201}%
\special{pa 2717 1200}%
\special{pa 2718 1189}%
\special{pa 2719 1188}%
\special{pa 2719 1182}%
\special{pa 2720 1181}%
\special{pa 2720 1176}%
\special{pa 2721 1175}%
\special{pa 2722 1162}%
\special{pa 2723 1161}%
\special{pa 2724 1146}%
\special{pa 2725 1145}%
\special{pa 2726 1128}%
\special{pa 2727 1127}%
\special{pa 2727 1118}%
\special{pa 2728 1117}%
\special{pa 2728 1106}%
\special{pa 2729 1105}%
\special{pa 2729 1092}%
\special{pa 2730 1091}%
\special{pa 2730 1074}%
\special{pa 2731 1073}%
\special{pa 2731 997}%
\special{pa 2730 996}%
\special{pa 2730 979}%
\special{pa 2729 978}%
\special{pa 2729 966}%
\special{pa 2728 965}%
\special{pa 2727 946}%
\special{pa 2726 945}%
\special{pa 2725 930}%
\special{pa 2724 929}%
\special{pa 2724 923}%
\special{pa 2723 922}%
\special{pa 2720 899}%
\special{pa 2719 898}%
\special{pa 2716 879}%
\special{pa 2715 878}%
\special{pa 2715 875}%
\special{pa 2714 874}%
\special{pa 2714 870}%
\special{pa 2713 869}%
\special{pa 2710 854}%
\special{pa 2709 853}%
\special{pa 2709 850}%
\special{pa 2708 849}%
\special{pa 2708 847}%
\special{pa 2707 846}%
\special{pa 2706 839}%
\special{pa 2705 838}%
\special{pa 2704 833}%
\special{pa 2703 832}%
\special{pa 2703 829}%
\special{pa 2702 828}%
\special{pa 2699 817}%
\special{pa 2698 816}%
\special{pa 2695 805}%
\special{pa 2694 804}%
\special{pa 2694 802}%
\special{pa 2693 801}%
\special{pa 2693 800}%
\special{pa 2692 799}%
\special{pa 2691 794}%
\special{pa 2690 793}%
\special{pa 2690 792}%
\special{pa 2689 791}%
\special{pa 2689 789}%
\special{pa 2688 788}%
\special{pa 2688 787}%
\special{pa 2687 786}%
\special{pa 2687 784}%
\special{pa 2686 783}%
\special{pa 2686 782}%
\special{pa 2685 781}%
\special{pa 2685 779}%
\special{pa 2684 778}%
\special{pa 2684 777}%
\special{pa 2683 776}%
\special{pa 2683 774}%
\special{pa 2682 773}%
\special{pa 2681 770}%
\special{pa 2680 769}%
\special{pa 2680 767}%
\special{pa 2679 766}%
\special{pa 2676 759}%
\special{pa 2675 758}%
\special{pa 2675 757}%
\special{pa 2674 756}%
\special{pa 2674 754}%
\special{pa 2673 753}%
\special{pa 2670 746}%
\special{pa 2669 745}%
\special{pa 2666 738}%
\special{pa 2665 737}%
\special{pa 2664 734}%
\special{pa 2662 732}%
\special{pa 2659 725}%
\special{pa 2657 723}%
\special{pa 2656 720}%
\special{pa 2655 719}%
\special{pa 2655 718}%
\special{pa 2653 716}%
\special{pa 2652 713}%
\special{pa 2651 712}%
\special{pa 2651 711}%
\special{pa 2649 709}%
\special{pa 2649 708}%
\special{pa 2647 706}%
\special{pa 2646 703}%
\special{pa 2644 701}%
\special{pa 2644 700}%
\special{pa 2642 698}%
\special{pa 2642 697}%
\special{pa 2640 695}%
\special{pa 2639 692}%
\special{pa 2636 689}%
\special{pa 2636 688}%
\special{pa 2634 686}%
\special{pa 2634 685}%
\special{pa 2632 683}%
\special{pa 2632 682}%
\special{pa 2630 680}%
\special{pa 2630 679}%
\special{pa 2627 676}%
\special{pa 2627 675}%
\special{pa 2624 672}%
\special{pa 2624 671}%
\special{pa 2621 668}%
\special{pa 2621 667}%
\special{pa 2618 664}%
\special{pa 2618 663}%
\special{pa 2614 659}%
\special{pa 2614 658}%
\special{pa 2610 654}%
\special{pa 2610 653}%
\special{pa 2605 648}%
\special{pa 2605 647}%
\special{pa 2598 640}%
\special{pa 2598 639}%
\special{pa 2585 626}%
\special{pa 2585 625}%
\special{pa 2575 615}%
\special{pa 2574 615}%
\special{pa 2561 602}%
\special{pa 2560 602}%
\special{pa 2553 595}%
\special{pa 2552 595}%
\special{pa 2547 590}%
\special{pa 2546 590}%
\special{pa 2542 586}%
\special{pa 2541 586}%
\special{pa 2537 582}%
\special{pa 2536 582}%
\special{pa 2533 579}%
\special{pa 2532 579}%
\special{pa 2529 576}%
\special{pa 2528 576}%
\special{pa 2525 573}%
\special{pa 2524 573}%
\special{pa 2521 570}%
\special{pa 2520 570}%
\special{pa 2518 568}%
\special{pa 2517 568}%
\special{pa 2515 566}%
\special{pa 2514 566}%
\special{pa 2512 564}%
\special{pa 2511 564}%
\special{pa 2508 561}%
\special{pa 2505 560}%
\special{pa 2503 558}%
\special{pa 2502 558}%
\special{pa 2500 556}%
\special{pa 2499 556}%
\special{pa 2497 554}%
\special{pa 2494 553}%
\special{pa 2492 551}%
\special{pa 2491 551}%
\special{pa 2489 549}%
\special{pa 2486 548}%
\special{pa 2485 547}%
\special{pa 2484 547}%
\special{pa 2482 545}%
\special{pa 2479 544}%
\special{pa 2478 543}%
\special{pa 2477 543}%
\special{pa 2475 541}%
\special{pa 2468 538}%
\special{pa 2466 536}%
\special{pa 2459 533}%
\special{pa 2458 532}%
\special{pa 2451 529}%
\special{pa 2450 528}%
\special{pa 2447 527}%
\special{pa 2446 526}%
\special{pa 2444 526}%
\special{pa 2443 525}%
\special{pa 2436 522}%
\special{pa 2435 521}%
\special{pa 2434 521}%
\special{pa 2433 520}%
\special{pa 2431 520}%
\special{pa 2430 519}%
\special{pa 2427 518}%
\special{pa 2426 517}%
\special{pa 2424 517}%
\special{pa 2423 516}%
\special{pa 2422 516}%
\special{pa 2421 515}%
\special{pa 2419 515}%
\special{pa 2418 514}%
\special{pa 2417 514}%
\special{pa 2416 513}%
\special{pa 2414 513}%
\special{pa 2413 512}%
\special{pa 2412 512}%
\special{pa 2411 511}%
\special{pa 2409 511}%
\special{pa 2408 510}%
\special{pa 2407 510}%
\special{pa 2406 509}%
\special{pa 2401 508}%
\special{pa 2400 507}%
\special{pa 2399 507}%
\special{pa 2398 506}%
\special{pa 2387 503}%
\special{pa 2386 502}%
\special{pa 2375 499}%
\special{pa 2374 498}%
\special{pa 2372 498}%
\special{pa 2371 497}%
\special{pa 2368 497}%
\special{pa 2367 496}%
\special{pa 2362 495}%
\special{pa 2361 494}%
\special{pa 2354 493}%
\special{pa 2353 492}%
\special{pa 2351 492}%
\special{pa 2350 491}%
\special{pa 2335 488}%
\special{pa 2334 487}%
\special{pa 2331 487}%
\special{pa 2330 486}%
\special{pa 2326 486}%
\special{pa 2325 485}%
\special{pa 2322 485}%
\special{pa 2321 484}%
\special{pa 2302 481}%
\special{pa 2301 480}%
\special{pa 2278 477}%
\special{pa 2277 476}%
\special{pa 2271 476}%
\special{pa 2270 475}%
\special{pa 2255 474}%
\special{pa 2254 473}%
\special{pa 2235 472}%
\special{pa 2234 471}%
\special{pa 2222 471}%
\special{pa 2221 470}%
\special{pa 2204 470}%
\special{pa 2203 469}%
\special{pa 2127 469}%
\special{pa 2126 470}%
\special{pa 2109 470}%
\special{pa 2108 471}%
\special{pa 2095 471}%
\special{pa 2094 472}%
\special{pa 2083 472}%
\special{pa 2082 473}%
\special{pa 2073 473}%
\special{pa 2072 474}%
\special{pa 2055 475}%
\special{pa 2054 476}%
\special{pa 2039 477}%
\special{pa 2038 478}%
\special{pa 2025 479}%
\special{pa 2024 480}%
\special{pa 2019 480}%
\special{pa 2018 481}%
\special{pa 2012 481}%
\special{pa 2011 482}%
\special{pa 2000 483}%
\special{pa 1999 484}%
\special{pa 1995 484}%
\special{pa 1994 485}%
\special{pa 1989 485}%
\special{pa 1988 486}%
\special{pa 1979 487}%
\special{pa 1978 488}%
\special{pa 1973 488}%
\special{pa 1972 489}%
\special{pa 1968 489}%
\special{pa 1967 490}%
\special{pa 1964 490}%
\special{pa 1963 491}%
\special{pa 1954 492}%
\special{pa 1953 493}%
\special{pa 1949 493}%
\special{pa 1948 494}%
\special{pa 1945 494}%
\special{pa 1944 495}%
\special{pa 1940 495}%
\special{pa 1939 496}%
\special{pa 1932 497}%
\special{pa 1931 498}%
\special{pa 1928 498}%
\special{pa 1927 499}%
\special{pa 1923 499}%
\special{pa 1922 500}%
\special{pa 1907 503}%
\special{pa 1906 504}%
\special{pa 1903 504}%
\special{pa 1902 505}%
\special{pa 1900 505}%
\special{pa 1899 506}%
\special{pa 1892 507}%
\special{pa 1891 508}%
\special{pa 1888 508}%
\special{pa 1887 509}%
\special{pa 1885 509}%
\special{pa 1884 510}%
\special{pa 1881 510}%
\special{pa 1880 511}%
\special{pa 1878 511}%
\special{pa 1877 512}%
\special{pa 1874 512}%
\special{pa 1873 513}%
\special{pa 1871 513}%
\special{pa 1870 514}%
\special{pa 1867 514}%
\special{pa 1866 515}%
\special{pa 1864 515}%
\special{pa 1863 516}%
\special{pa 1860 516}%
\special{pa 1859 517}%
\special{pa 1854 518}%
\special{pa 1853 519}%
\special{pa 1850 519}%
\special{pa 1849 520}%
\special{pa 1844 521}%
\special{pa 1843 522}%
\special{pa 1841 522}%
\special{pa 1840 523}%
\special{pa 1837 523}%
\special{pa 1836 524}%
\special{pa 1825 527}%
\special{pa 1824 528}%
\special{pa 1813 531}%
\special{pa 1812 532}%
\special{pa 1801 535}%
\special{pa 1800 536}%
\special{pa 1789 539}%
\special{pa 1788 540}%
\special{pa 1787 540}%
\special{pa 1786 541}%
\special{pa 1775 544}%
\special{pa 1774 545}%
\special{pa 1773 545}%
\special{pa 1772 546}%
\special{pa 1767 547}%
\special{pa 1766 548}%
\special{pa 1765 548}%
\special{pa 1764 549}%
\special{pa 1759 550}%
\special{pa 1758 551}%
\special{pa 1757 551}%
\special{pa 1756 552}%
\special{pa 1751 553}%
\special{pa 1750 554}%
\special{pa 1749 554}%
\special{pa 1748 555}%
\special{pa 1746 555}%
\special{pa 1745 556}%
\special{pa 1744 556}%
\special{pa 1743 557}%
\special{pa 1741 557}%
\special{pa 1740 558}%
\special{pa 1739 558}%
\special{pa 1738 559}%
\special{pa 1733 560}%
\special{pa 1732 561}%
\special{pa 1731 561}%
\special{pa 1730 562}%
\special{pa 1728 562}%
\special{pa 1727 563}%
\special{pa 1724 564}%
\special{pa 1723 565}%
\special{pa 1721 565}%
\special{pa 1720 566}%
\special{pa 1719 566}%
\special{pa 1718 567}%
\special{pa 1716 567}%
\special{pa 1715 568}%
\special{pa 1714 568}%
\special{pa 1713 569}%
\special{pa 1711 569}%
\special{pa 1710 570}%
\special{pa 1707 571}%
\special{pa 1706 572}%
\special{pa 1704 572}%
\special{pa 1703 573}%
\special{pa 1702 573}%
\special{pa 1701 574}%
\special{pa 1699 574}%
\special{pa 1698 575}%
\special{pa 1695 576}%
\special{pa 1694 577}%
\special{pa 1692 577}%
\special{pa 1691 578}%
\special{pa 1688 579}%
\special{pa 1687 580}%
\special{pa 1685 580}%
\special{pa 1684 581}%
\special{pa 1681 582}%
\special{pa 1680 583}%
\special{pa 1679 583}%
\special{pa 1678 584}%
\special{pa 1676 584}%
\special{pa 1675 585}%
\special{pa 1672 586}%
\special{pa 1671 587}%
\special{pa 1670 587}%
\special{pa 1669 588}%
\special{pa 1667 588}%
\special{pa 1666 589}%
\special{pa 1663 590}%
\special{pa 1662 591}%
\special{pa 1661 591}%
\special{pa 1660 592}%
\special{pa 1658 592}%
\special{pa 1657 593}%
\special{pa 1650 596}%
\special{pa 1649 597}%
\special{pa 1648 597}%
\special{pa 1647 598}%
\special{pa 1645 598}%
\special{pa 1644 599}%
\special{pa 1637 602}%
\special{pa 1636 603}%
\special{pa 1629 606}%
\special{pa 1628 607}%
\special{pa 1626 607}%
\special{pa 1625 608}%
\special{pa 1618 611}%
\special{pa 1617 612}%
\special{pa 1610 615}%
\special{pa 1609 616}%
\special{pa 1602 619}%
\special{pa 1601 620}%
\special{pa 1594 623}%
\special{pa 1593 624}%
\special{pa 1586 627}%
\special{pa 1585 628}%
\special{pa 1578 631}%
\special{pa 1577 632}%
\special{pa 1574 633}%
\special{pa 1572 635}%
\special{pa 1565 638}%
\special{pa 1564 639}%
\special{pa 1557 642}%
\special{pa 1556 643}%
\special{pa 1555 643}%
\special{pa 1553 645}%
\special{pa 1546 648}%
\special{pa 1545 649}%
\special{pa 1542 650}%
\special{pa 1540 652}%
\special{pa 1533 655}%
\special{pa 1531 657}%
\special{pa 1524 660}%
\special{pa 1522 662}%
\special{pa 1515 665}%
\special{pa 1513 667}%
\special{pa 1510 668}%
\special{pa 1509 669}%
\special{pa 1508 669}%
\special{pa 1506 671}%
\special{pa 1503 672}%
\special{pa 1502 673}%
\special{pa 1501 673}%
\special{pa 1499 675}%
\special{pa 1496 676}%
\special{pa 1494 678}%
\special{pa 1491 679}%
\special{pa 1490 680}%
\special{pa 1489 680}%
\special{pa 1487 682}%
\special{pa 1484 683}%
\special{pa 1482 685}%
\special{pa 1479 686}%
\special{pa 1477 688}%
\special{pa 1474 689}%
\special{pa 1473 690}%
\special{pa 1472 690}%
\special{pa 1470 692}%
\special{pa 1467 693}%
\special{pa 1465 695}%
\special{pa 1464 695}%
\special{pa 1462 697}%
\special{pa 1459 698}%
\special{pa 1457 700}%
\special{pa 1454 701}%
\special{pa 1452 703}%
\special{pa 1449 704}%
\special{pa 1447 706}%
\special{pa 1446 706}%
\special{pa 1444 708}%
\special{pa 1441 709}%
\special{pa 1439 711}%
\special{pa 1438 711}%
\special{pa 1436 713}%
\special{pa 1433 714}%
\special{pa 1431 716}%
\special{pa 1430 716}%
\special{pa 1428 718}%
\special{pa 1425 719}%
\special{pa 1423 721}%
\special{pa 1422 721}%
\special{pa 1420 723}%
\special{pa 1419 723}%
\special{pa 1417 725}%
\special{pa 1416 725}%
\special{pa 1414 727}%
\special{pa 1411 728}%
\special{pa 1409 730}%
\special{pa 1408 730}%
\special{pa 1406 732}%
\special{pa 1405 732}%
\special{pa 1403 734}%
\special{pa 1402 734}%
\special{pa 1400 736}%
\special{pa 1399 736}%
\special{pa 1397 738}%
\special{pa 1396 738}%
\special{pa 1394 740}%
\special{pa 1393 740}%
\special{pa 1391 742}%
\special{pa 1390 742}%
\special{pa 1388 744}%
\special{pa 1387 744}%
\special{pa 1385 746}%
\special{pa 1384 746}%
\special{pa 1382 748}%
\special{pa 1381 748}%
\special{pa 1379 750}%
\special{pa 1378 750}%
\special{pa 1376 752}%
\special{pa 1375 752}%
\special{pa 1373 754}%
\special{pa 1372 754}%
\special{pa 1370 756}%
\special{pa 1369 756}%
\special{pa 1367 758}%
\special{pa 1366 758}%
\special{pa 1364 760}%
\special{pa 1363 760}%
\special{pa 1360 763}%
\special{pa 1359 763}%
\special{pa 1357 765}%
\special{pa 1356 765}%
\special{pa 1354 767}%
\special{pa 1353 767}%
\special{pa 1351 769}%
\special{pa 1350 769}%
\special{pa 1347 772}%
\special{pa 1346 772}%
\special{pa 1344 774}%
\special{pa 1343 774}%
\special{pa 1341 776}%
\special{pa 1340 776}%
\special{pa 1337 779}%
\special{pa 1336 779}%
\special{pa 1334 781}%
\special{pa 1333 781}%
\special{pa 1330 784}%
\special{pa 1329 784}%
\special{pa 1327 786}%
\special{pa 1326 786}%
\special{pa 1323 789}%
\special{pa 1322 789}%
\special{pa 1320 791}%
\special{pa 1319 791}%
\special{pa 1316 794}%
\special{pa 1315 794}%
\special{pa 1313 796}%
\special{pa 1312 796}%
\special{pa 1309 799}%
\special{pa 1308 799}%
\special{pa 1305 802}%
\special{pa 1304 802}%
\special{pa 1301 805}%
\special{pa 1300 805}%
\special{pa 1298 807}%
\special{pa 1297 807}%
\special{pa 1294 810}%
\special{pa 1293 810}%
\special{pa 1290 813}%
\special{pa 1289 813}%
\special{pa 1286 816}%
\special{pa 1285 816}%
\special{pa 1282 819}%
\special{pa 1281 819}%
\special{pa 1278 822}%
\special{pa 1277 822}%
\special{pa 1273 826}%
\special{pa 1272 826}%
\special{pa 1269 829}%
\special{pa 1268 829}%
\special{pa 1265 832}%
\special{pa 1264 832}%
\special{pa 1261 835}%
\special{pa 1260 835}%
\special{pa 1256 839}%
\special{pa 1255 839}%
\special{pa 1252 842}%
\special{pa 1251 842}%
\special{pa 1247 846}%
\special{pa 1246 846}%
\special{pa 1242 850}%
\special{pa 1241 850}%
\special{pa 1237 854}%
\special{pa 1236 854}%
\special{pa 1233 857}%
\special{pa 1232 857}%
\special{pa 1228 861}%
\special{pa 1227 861}%
\special{pa 1222 866}%
\special{pa 1221 866}%
\special{pa 1217 870}%
\special{pa 1216 870}%
\special{pa 1212 874}%
\special{pa 1211 874}%
\special{pa 1206 879}%
\special{pa 1205 879}%
\special{pa 1201 883}%
\special{pa 1200 883}%
\special{pa 1195 888}%
\special{pa 1194 888}%
\special{pa 1189 893}%
\special{pa 1188 893}%
\special{pa 1182 899}%
\special{pa 1181 899}%
\special{pa 1176 904}%
\special{pa 1175 904}%
\special{pa 1169 910}%
\special{pa 1168 910}%
\special{pa 1162 916}%
\special{pa 1161 916}%
\special{pa 1154 923}%
\special{pa 1153 923}%
\special{pa 1146 930}%
\special{pa 1145 930}%
\special{pa 1137 938}%
\special{pa 1136 938}%
\special{pa 1128 946}%
\special{pa 1127 946}%
\special{pa 1118 955}%
\special{pa 1117 955}%
\special{pa 1106 966}%
\special{pa 1105 966}%
\special{pa 1092 979}%
\special{pa 1091 979}%
\special{pa 1074 996}%
\special{pa 1073 996}%
\special{pa 996 1073}%
\special{pa 996 1074}%
\special{pa 979 1091}%
\special{pa 979 1092}%
\special{pa 966 1105}%
\special{pa 966 1106}%
\special{pa 955 1117}%
\special{pa 955 1118}%
\special{pa 946 1127}%
\special{pa 946 1128}%
\special{pa 938 1136}%
\special{pa 938 1137}%
\special{pa 930 1145}%
\special{pa 930 1146}%
\special{pa 923 1153}%
\special{pa 923 1154}%
\special{pa 916 1161}%
\special{pa 916 1162}%
\special{pa 910 1168}%
\special{pa 910 1169}%
\special{pa 904 1175}%
\special{pa 904 1176}%
\special{pa 899 1181}%
\special{pa 899 1182}%
\special{pa 893 1188}%
\special{pa 893 1189}%
\special{pa 888 1194}%
\special{pa 888 1195}%
\special{pa 883 1200}%
\special{pa 883 1201}%
\special{pa 879 1205}%
\special{pa 879 1206}%
\special{pa 874 1211}%
\special{pa 874 1212}%
\special{pa 870 1216}%
\special{pa 870 1217}%
\special{pa 866 1221}%
\special{pa 866 1222}%
\special{pa 861 1227}%
\special{pa 861 1228}%
\special{pa 857 1232}%
\special{pa 857 1233}%
\special{pa 854 1236}%
\special{pa 854 1237}%
\special{pa 850 1241}%
\special{pa 850 1242}%
\special{pa 846 1246}%
\special{pa 846 1247}%
\special{pa 842 1251}%
\special{pa 842 1252}%
\special{pa 839 1255}%
\special{pa 839 1256}%
\special{pa 835 1260}%
\special{pa 835 1261}%
\special{pa 832 1264}%
\special{pa 832 1265}%
\special{pa 829 1268}%
\special{pa 829 1269}%
\special{pa 826 1272}%
\special{pa 826 1273}%
\special{pa 822 1277}%
\special{pa 822 1278}%
\special{pa 819 1281}%
\special{pa 819 1282}%
\special{pa 816 1285}%
\special{pa 816 1286}%
\special{pa 813 1289}%
\special{pa 813 1290}%
\special{pa 810 1293}%
\special{pa 810 1294}%
\special{pa 807 1297}%
\special{pa 807 1298}%
\special{pa 805 1300}%
\special{pa 805 1301}%
\special{pa 802 1304}%
\special{pa 802 1305}%
\special{pa 799 1308}%
\special{pa 799 1309}%
\special{pa 796 1312}%
\special{pa 796 1313}%
\special{pa 794 1315}%
\special{pa 794 1316}%
\special{pa 791 1319}%
\special{pa 791 1320}%
\special{pa 789 1322}%
\special{pa 789 1323}%
\special{pa 786 1326}%
\special{pa 786 1327}%
\special{pa 784 1329}%
\special{pa 784 1330}%
\special{pa 781 1333}%
\special{pa 781 1334}%
\special{pa 779 1336}%
\special{pa 779 1337}%
\special{pa 776 1340}%
\special{pa 776 1341}%
\special{pa 774 1343}%
\special{pa 774 1344}%
\special{pa 772 1346}%
\special{pa 772 1347}%
\special{pa 769 1350}%
\special{pa 769 1351}%
\special{pa 767 1353}%
\special{pa 767 1354}%
\special{pa 765 1356}%
\special{pa 765 1357}%
\special{pa 763 1359}%
\special{pa 763 1360}%
\special{pa 760 1363}%
\special{pa 760 1364}%
\special{pa 758 1366}%
\special{pa 758 1367}%
\special{pa 756 1369}%
\special{pa 756 1370}%
\special{pa 754 1372}%
\special{pa 754 1373}%
\special{pa 752 1375}%
\special{pa 752 1376}%
\special{pa 750 1378}%
\special{pa 750 1379}%
\special{pa 748 1381}%
\special{pa 748 1382}%
\special{pa 746 1384}%
\special{pa 746 1385}%
\special{pa 744 1387}%
\special{pa 744 1388}%
\special{pa 742 1390}%
\special{pa 742 1391}%
\special{pa 740 1393}%
\special{pa 740 1394}%
\special{pa 738 1396}%
\special{pa 738 1397}%
\special{pa 736 1399}%
\special{pa 736 1400}%
\special{pa 734 1402}%
\special{pa 734 1403}%
\special{pa 732 1405}%
\special{pa 732 1406}%
\special{pa 730 1408}%
\special{pa 730 1409}%
\special{pa 728 1411}%
\special{pa 727 1414}%
\special{pa 725 1416}%
\special{pa 725 1417}%
\special{pa 723 1419}%
\special{pa 723 1420}%
\special{pa 721 1422}%
\special{pa 721 1423}%
\special{pa 719 1425}%
\special{pa 718 1428}%
\special{pa 716 1430}%
\special{pa 716 1431}%
\special{pa 714 1433}%
\special{pa 713 1436}%
\special{pa 711 1438}%
\special{pa 711 1439}%
\special{pa 709 1441}%
\special{pa 708 1444}%
\special{pa 706 1446}%
\special{pa 706 1447}%
\special{pa 704 1449}%
\special{pa 703 1452}%
\special{pa 701 1454}%
\special{pa 700 1457}%
\special{pa 698 1459}%
\special{pa 697 1462}%
\special{pa 695 1464}%
\special{pa 695 1465}%
\special{pa 693 1467}%
\special{pa 692 1470}%
\special{pa 690 1472}%
\special{pa 689 1475}%
\special{pa 688 1476}%
\special{pa 688 1477}%
\special{pa 686 1479}%
\special{pa 685 1482}%
\special{pa 683 1484}%
\special{pa 682 1487}%
\special{pa 680 1489}%
\special{pa 679 1492}%
\special{pa 678 1493}%
\special{pa 678 1494}%
\special{pa 676 1496}%
\special{pa 675 1499}%
\special{pa 673 1501}%
\special{pa 672 1504}%
\special{pa 671 1505}%
\special{pa 671 1506}%
\special{pa 669 1508}%
\special{pa 668 1511}%
\special{pa 667 1512}%
\special{pa 667 1513}%
\special{pa 665 1515}%
\special{pa 662 1522}%
\special{pa 660 1524}%
\special{pa 657 1531}%
\special{pa 655 1533}%
\special{pa 652 1540}%
\special{pa 650 1542}%
\special{pa 647 1549}%
\special{pa 646 1550}%
\special{pa 645 1553}%
\special{pa 643 1555}%
\special{pa 640 1562}%
\special{pa 639 1563}%
\special{pa 636 1570}%
\special{pa 635 1571}%
\special{pa 635 1572}%
\special{pa 633 1574}%
\special{pa 630 1581}%
\special{pa 629 1582}%
\special{pa 626 1589}%
\special{pa 625 1590}%
\special{pa 622 1597}%
\special{pa 621 1598}%
\special{pa 618 1605}%
\special{pa 617 1606}%
\special{pa 614 1613}%
\special{pa 613 1614}%
\special{pa 610 1621}%
\special{pa 609 1622}%
\special{pa 608 1625}%
\special{pa 607 1626}%
\special{pa 607 1628}%
\special{pa 606 1629}%
\special{pa 603 1636}%
\special{pa 602 1637}%
\special{pa 599 1644}%
\special{pa 598 1645}%
\special{pa 598 1647}%
\special{pa 597 1648}%
\special{pa 594 1655}%
\special{pa 593 1656}%
\special{pa 593 1657}%
\special{pa 592 1658}%
\special{pa 592 1660}%
\special{pa 591 1661}%
\special{pa 590 1664}%
\special{pa 589 1665}%
\special{pa 589 1666}%
\special{pa 588 1667}%
\special{pa 588 1669}%
\special{pa 587 1670}%
\special{pa 586 1673}%
\special{pa 585 1674}%
\special{pa 585 1675}%
\special{pa 584 1676}%
\special{pa 584 1678}%
\special{pa 583 1679}%
\special{pa 582 1682}%
\special{pa 581 1683}%
\special{pa 581 1684}%
\special{pa 580 1685}%
\special{pa 580 1687}%
\special{pa 579 1688}%
\special{pa 578 1691}%
\special{pa 577 1692}%
\special{pa 577 1694}%
\special{pa 576 1695}%
\special{pa 575 1698}%
\special{pa 574 1699}%
\special{pa 574 1701}%
\special{pa 573 1702}%
\special{pa 573 1703}%
\special{pa 572 1704}%
\special{pa 572 1706}%
\special{pa 571 1707}%
\special{pa 570 1710}%
\special{pa 569 1711}%
\special{pa 569 1713}%
\special{pa 568 1714}%
\special{pa 568 1715}%
\special{pa 567 1716}%
\special{pa 567 1718}%
\special{pa 566 1719}%
\special{pa 566 1720}%
\special{pa 565 1721}%
\special{pa 565 1723}%
\special{pa 564 1724}%
\special{pa 563 1727}%
\special{pa 562 1728}%
\special{pa 562 1730}%
\special{pa 561 1731}%
\special{pa 561 1732}%
\special{pa 560 1733}%
\special{pa 559 1738}%
\special{pa 558 1739}%
\special{pa 558 1740}%
\special{pa 557 1741}%
\special{pa 557 1743}%
\special{pa 556 1744}%
\special{pa 556 1745}%
\special{pa 555 1746}%
\special{pa 555 1748}%
\special{pa 554 1749}%
\special{pa 554 1750}%
\special{pa 553 1751}%
\special{pa 552 1756}%
\special{pa 551 1757}%
\special{pa 551 1758}%
\special{pa 550 1759}%
\special{pa 549 1764}%
\special{pa 548 1765}%
\special{pa 548 1766}%
\special{pa 547 1767}%
\special{pa 546 1772}%
\special{pa 545 1773}%
\special{pa 545 1774}%
\special{pa 544 1775}%
\special{pa 541 1786}%
\special{pa 540 1787}%
\special{pa 540 1788}%
\special{pa 539 1789}%
\special{pa 536 1800}%
\special{pa 535 1801}%
\special{pa 532 1812}%
\special{pa 531 1813}%
\special{pa 528 1824}%
\special{pa 527 1825}%
\special{pa 524 1836}%
\special{pa 523 1837}%
\special{pa 523 1840}%
\special{pa 522 1841}%
\special{pa 521 1846}%
\special{pa 520 1847}%
\special{pa 520 1849}%
\special{pa 519 1850}%
\special{pa 519 1853}%
\special{pa 518 1854}%
\special{pa 517 1859}%
\special{pa 516 1860}%
\special{pa 516 1863}%
\special{pa 515 1864}%
\special{pa 515 1866}%
\special{pa 514 1867}%
\special{pa 514 1870}%
\special{pa 513 1871}%
\special{pa 513 1873}%
\special{pa 512 1874}%
\special{pa 512 1877}%
\special{pa 511 1878}%
\special{pa 511 1880}%
\special{pa 510 1881}%
\special{pa 510 1884}%
\special{pa 509 1885}%
\special{pa 509 1887}%
\special{pa 508 1888}%
\special{pa 507 1895}%
\special{pa 506 1896}%
\special{pa 506 1899}%
\special{pa 505 1900}%
\special{pa 505 1902}%
\special{pa 504 1903}%
\special{pa 501 1918}%
\special{pa 500 1919}%
\special{pa 500 1922}%
\special{pa 499 1923}%
\special{pa 499 1927}%
\special{pa 498 1928}%
\special{pa 497 1935}%
\special{pa 496 1936}%
\special{pa 496 1939}%
\special{pa 495 1940}%
\special{pa 495 1944}%
\special{pa 494 1945}%
\special{pa 494 1948}%
\special{pa 493 1949}%
\special{pa 492 1958}%
\special{pa 491 1959}%
\special{pa 491 1963}%
\special{pa 490 1964}%
\special{pa 490 1967}%
\special{pa 489 1968}%
\special{pa 489 1972}%
\special{pa 488 1973}%
\special{pa 488 1978}%
\special{pa 487 1979}%
\special{pa 486 1988}%
\special{pa 485 1989}%
\special{pa 485 1994}%
\special{pa 484 1995}%
\special{pa 484 1999}%
\special{pa 483 2000}%
\special{pa 482 2011}%
\special{pa 481 2012}%
\special{pa 481 2018}%
\special{pa 480 2019}%
\special{pa 480 2024}%
\special{pa 479 2025}%
\special{pa 478 2038}%
\special{pa 477 2039}%
\special{pa 476 2054}%
\special{pa 475 2055}%
\special{pa 474 2072}%
\special{pa 473 2073}%
\special{pa 473 2082}%
\special{pa 472 2083}%
\special{pa 472 2094}%
\special{pa 471 2095}%
\special{pa 471 2108}%
\special{pa 470 2109}%
\special{pa 470 2126}%
\special{pa 469 2127}%
\special{pa 469 2203}%
\special{pa 470 2204}%
\special{pa 470 2221}%
\special{pa 471 2222}%
\special{pa 471 2234}%
\special{pa 472 2235}%
\special{pa 473 2254}%
\special{pa 474 2255}%
\special{pa 475 2270}%
\special{pa 476 2271}%
\special{pa 476 2277}%
\special{pa 477 2278}%
\special{pa 480 2301}%
\special{pa 481 2302}%
\special{pa 484 2321}%
\special{pa 485 2322}%
\special{pa 485 2325}%
\special{pa 486 2326}%
\special{pa 486 2330}%
\special{pa 487 2331}%
\special{pa 490 2346}%
\special{pa 491 2347}%
\special{pa 491 2350}%
\special{pa 492 2351}%
\special{pa 492 2353}%
\special{pa 493 2354}%
\special{pa 494 2361}%
\special{pa 495 2362}%
\special{pa 496 2367}%
\special{pa 497 2368}%
\special{pa 497 2371}%
\special{pa 498 2372}%
\special{pa 501 2383}%
\special{pa 502 2384}%
\special{pa 505 2395}%
\special{pa 506 2396}%
\special{pa 506 2398}%
\special{pa 507 2399}%
\special{pa 507 2400}%
\special{pa 508 2401}%
\special{pa 509 2406}%
\special{pa 510 2407}%
\special{pa 510 2408}%
\special{pa 511 2409}%
\special{pa 511 2411}%
\special{pa 512 2412}%
\special{pa 512 2413}%
\special{pa 513 2414}%
\special{pa 513 2416}%
\special{pa 514 2417}%
\special{pa 514 2418}%
\special{pa 515 2419}%
\special{pa 515 2421}%
\special{pa 516 2422}%
\special{pa 516 2423}%
\special{pa 517 2424}%
\special{pa 517 2426}%
\special{pa 518 2427}%
\special{pa 519 2430}%
\special{pa 520 2431}%
\special{pa 520 2433}%
\special{pa 521 2434}%
\special{pa 524 2441}%
\special{pa 525 2442}%
\special{pa 525 2443}%
\special{pa 526 2444}%
\special{pa 526 2446}%
\special{pa 527 2447}%
\special{pa 530 2454}%
\special{pa 531 2455}%
\special{pa 534 2462}%
\special{pa 535 2463}%
\special{pa 536 2466}%
\special{pa 538 2468}%
\special{pa 541 2475}%
\special{pa 543 2477}%
\special{pa 544 2480}%
\special{pa 545 2481}%
\special{pa 545 2482}%
\special{pa 547 2484}%
\special{pa 548 2487}%
\special{pa 549 2488}%
\special{pa 549 2489}%
\special{pa 551 2491}%
\special{pa 551 2492}%
\special{pa 553 2494}%
\special{pa 554 2497}%
\special{pa 556 2499}%
\special{pa 556 2500}%
\special{pa 558 2502}%
\special{pa 558 2503}%
\special{pa 560 2505}%
\special{pa 561 2508}%
\special{pa 564 2511}%
\special{pa 564 2512}%
\special{pa 566 2514}%
\special{pa 566 2515}%
\special{pa 568 2517}%
\special{pa 568 2518}%
\special{pa 570 2520}%
\special{pa 570 2521}%
\special{pa 573 2524}%
\special{pa 573 2525}%
\special{pa 576 2528}%
\special{pa 576 2529}%
\special{pa 579 2532}%
\special{pa 579 2533}%
\special{pa 582 2536}%
\special{pa 582 2537}%
\special{pa 586 2541}%
\special{pa 586 2542}%
\special{pa 590 2546}%
\special{pa 590 2547}%
\special{pa 595 2552}%
\special{pa 595 2553}%
\special{pa 602 2560}%
\special{pa 602 2561}%
\special{pa 615 2574}%
\special{pa 615 2575}%
\special{pa 625 2585}%
\special{pa 626 2585}%
\special{pa 639 2598}%
\special{pa 640 2598}%
\special{pa 647 2605}%
\special{pa 648 2605}%
\special{pa 653 2610}%
\special{pa 654 2610}%
\special{pa 658 2614}%
\special{pa 659 2614}%
\special{pa 663 2618}%
\special{pa 664 2618}%
\special{pa 667 2621}%
\special{pa 668 2621}%
\special{pa 671 2624}%
\special{pa 672 2624}%
\special{pa 675 2627}%
\special{pa 676 2627}%
\special{pa 679 2630}%
\special{pa 680 2630}%
\special{pa 682 2632}%
\special{pa 683 2632}%
\special{pa 685 2634}%
\special{pa 686 2634}%
\special{pa 688 2636}%
\special{pa 689 2636}%
\special{pa 692 2639}%
\special{pa 695 2640}%
\special{pa 697 2642}%
\special{pa 698 2642}%
\special{pa 700 2644}%
\special{pa 701 2644}%
\special{pa 703 2646}%
\special{pa 706 2647}%
\special{pa 708 2649}%
\special{pa 709 2649}%
\special{pa 711 2651}%
\special{pa 714 2652}%
\special{pa 715 2653}%
\special{pa 716 2653}%
\special{pa 718 2655}%
\special{pa 721 2656}%
\special{pa 722 2657}%
\special{pa 723 2657}%
\special{pa 725 2659}%
\special{pa 732 2662}%
\special{pa 734 2664}%
\special{pa 741 2667}%
\special{pa 742 2668}%
\special{pa 749 2671}%
\special{pa 750 2672}%
\special{pa 753 2673}%
\special{pa 754 2674}%
\special{pa 756 2674}%
\special{pa 757 2675}%
\special{pa 764 2678}%
\special{pa 765 2679}%
\special{pa 766 2679}%
\special{pa 767 2680}%
\special{pa 769 2680}%
\special{pa 770 2681}%
\special{pa 773 2682}%
\special{pa 774 2683}%
\special{pa 776 2683}%
\special{pa 777 2684}%
\special{pa 778 2684}%
\special{pa 779 2685}%
\special{pa 781 2685}%
\special{pa 782 2686}%
\special{pa 783 2686}%
\special{pa 784 2687}%
\special{pa 786 2687}%
\special{pa 787 2688}%
\special{pa 788 2688}%
\special{pa 789 2689}%
\special{pa 791 2689}%
\special{pa 792 2690}%
\special{pa 793 2690}%
\special{pa 794 2691}%
\special{pa 799 2692}%
\special{pa 800 2693}%
\special{pa 801 2693}%
\special{pa 802 2694}%
\special{pa 813 2697}%
\special{pa 814 2698}%
\special{pa 825 2701}%
\special{pa 826 2702}%
\special{pa 828 2702}%
\special{pa 829 2703}%
\special{pa 832 2703}%
\special{pa 833 2704}%
\special{pa 838 2705}%
\special{pa 839 2706}%
\special{pa 846 2707}%
\special{pa 847 2708}%
\special{pa 849 2708}%
\special{pa 850 2709}%
\special{pa 865 2712}%
\special{pa 866 2713}%
\special{pa 869 2713}%
\special{pa 870 2714}%
\special{pa 874 2714}%
\special{pa 875 2715}%
\special{pa 878 2715}%
\special{pa 879 2716}%
\special{pa 898 2719}%
\special{pa 899 2720}%
\special{pa 922 2723}%
\special{pa 923 2724}%
\special{pa 929 2724}%
\special{pa 930 2725}%
\special{pa 945 2726}%
\special{pa 946 2727}%
\special{pa 965 2728}%
\special{pa 966 2729}%
\special{pa 978 2729}%
\special{pa 979 2730}%
\special{pa 996 2730}%
\special{pa 997 2731}%
\special{pa 1073 2731}%
\special{pa 1074 2730}%
\special{pa 1091 2730}%
\special{pa 1092 2729}%
\special{pa 1105 2729}%
\special{pa 1106 2728}%
\special{pa 1117 2728}%
\special{pa 1118 2727}%
\special{pa 1127 2727}%
\special{pa 1128 2726}%
\special{pa 1145 2725}%
\special{pa 1146 2724}%
\special{pa 1161 2723}%
\special{pa 1162 2722}%
\special{pa 1175 2721}%
\special{pa 1176 2720}%
\special{pa 1181 2720}%
\special{pa 1182 2719}%
\special{pa 1188 2719}%
\special{pa 1189 2718}%
\special{pa 1200 2717}%
\special{pa 1201 2716}%
\special{pa 1205 2716}%
\special{pa 1206 2715}%
\special{pa 1211 2715}%
\special{pa 1212 2714}%
\special{pa 1221 2713}%
\special{pa 1222 2712}%
\special{pa 1227 2712}%
\special{pa 1228 2711}%
\special{pa 1232 2711}%
\special{pa 1233 2710}%
\special{pa 1236 2710}%
\special{pa 1237 2709}%
\special{pa 1246 2708}%
\special{pa 1247 2707}%
\special{pa 1251 2707}%
\special{pa 1252 2706}%
\special{pa 1255 2706}%
\special{pa 1256 2705}%
\special{pa 1260 2705}%
\special{pa 1261 2704}%
\special{pa 1268 2703}%
\special{pa 1269 2702}%
\special{pa 1272 2702}%
\special{pa 1273 2701}%
\special{pa 1277 2701}%
\special{pa 1278 2700}%
\special{pa 1293 2697}%
\special{pa 1294 2696}%
\special{pa 1297 2696}%
\special{pa 1298 2695}%
\special{pa 1300 2695}%
\special{pa 1301 2694}%
\special{pa 1308 2693}%
\special{pa 1309 2692}%
\special{pa 1312 2692}%
\special{pa 1313 2691}%
\special{pa 1315 2691}%
\special{pa 1316 2690}%
\special{pa 1319 2690}%
\special{pa 1320 2689}%
\special{pa 1322 2689}%
\special{pa 1323 2688}%
\special{pa 1326 2688}%
\special{pa 1327 2687}%
\special{pa 1329 2687}%
\special{pa 1330 2686}%
\special{pa 1333 2686}%
\special{pa 1334 2685}%
\special{pa 1336 2685}%
\special{pa 1337 2684}%
\special{pa 1340 2684}%
\special{pa 1341 2683}%
\special{pa 1346 2682}%
\special{pa 1347 2681}%
\special{pa 1350 2681}%
\special{pa 1351 2680}%
\special{pa 1356 2679}%
\special{pa 1357 2678}%
\special{pa 1359 2678}%
\special{pa 1360 2677}%
\special{pa 1363 2677}%
\special{pa 1364 2676}%
\special{pa 1375 2673}%
\special{pa 1376 2672}%
\special{pa 1387 2669}%
\special{pa 1388 2668}%
\special{pa 1399 2665}%
\special{pa 1400 2664}%
\special{pa 1411 2661}%
\special{pa 1412 2660}%
\special{pa 1413 2660}%
\special{pa 1414 2659}%
\special{pa 1425 2656}%
\special{pa 1426 2655}%
\special{pa 1427 2655}%
\special{pa 1428 2654}%
\special{pa 1433 2653}%
\special{pa 1434 2652}%
\special{pa 1435 2652}%
\special{pa 1436 2651}%
\special{pa 1441 2650}%
\special{pa 1442 2649}%
\special{pa 1443 2649}%
\special{pa 1444 2648}%
\special{pa 1449 2647}%
\special{pa 1450 2646}%
\special{pa 1451 2646}%
\special{pa 1452 2645}%
\special{pa 1454 2645}%
\special{pa 1455 2644}%
\special{pa 1456 2644}%
\special{pa 1457 2643}%
\special{pa 1459 2643}%
\special{pa 1460 2642}%
\special{pa 1461 2642}%
\special{pa 1462 2641}%
\special{pa 1467 2640}%
\special{pa 1468 2639}%
\special{pa 1469 2639}%
\special{pa 1470 2638}%
\special{pa 1472 2638}%
\special{pa 1473 2637}%
\special{pa 1476 2636}%
\special{pa 1477 2635}%
\special{pa 1479 2635}%
\special{pa 1480 2634}%
\special{pa 1481 2634}%
\special{pa 1482 2633}%
\special{pa 1484 2633}%
\special{pa 1485 2632}%
\special{pa 1486 2632}%
\special{pa 1487 2631}%
\special{pa 1489 2631}%
\special{pa 1490 2630}%
\special{pa 1493 2629}%
\special{pa 1494 2628}%
\special{pa 1496 2628}%
\special{pa 1497 2627}%
\special{pa 1498 2627}%
\special{pa 1499 2626}%
\special{pa 1501 2626}%
\special{pa 1502 2625}%
\special{pa 1505 2624}%
\special{pa 1506 2623}%
\special{pa 1508 2623}%
\special{pa 1509 2622}%
\special{pa 1512 2621}%
\special{pa 1513 2620}%
\special{pa 1515 2620}%
\special{pa 1516 2619}%
\special{pa 1519 2618}%
\special{pa 1520 2617}%
\special{pa 1521 2617}%
\special{pa 1522 2616}%
\special{pa 1524 2616}%
\special{pa 1525 2615}%
\special{pa 1528 2614}%
\special{pa 1529 2613}%
\special{pa 1530 2613}%
\special{pa 1531 2612}%
\special{pa 1533 2612}%
\special{pa 1534 2611}%
\special{pa 1537 2610}%
\special{pa 1538 2609}%
\special{pa 1539 2609}%
\special{pa 1540 2608}%
\special{pa 1542 2608}%
\special{pa 1543 2607}%
\special{pa 1550 2604}%
\special{pa 1551 2603}%
\special{pa 1552 2603}%
\special{pa 1553 2602}%
\special{pa 1555 2602}%
\special{pa 1556 2601}%
\special{pa 1563 2598}%
\special{pa 1564 2597}%
\special{pa 1571 2594}%
\special{pa 1572 2593}%
\special{pa 1574 2593}%
\special{pa 1575 2592}%
\special{pa 1582 2589}%
\special{pa 1583 2588}%
\special{pa 1590 2585}%
\special{pa 1591 2584}%
\special{pa 1598 2581}%
\special{pa 1599 2580}%
\special{pa 1606 2577}%
\special{pa 1607 2576}%
\special{pa 1614 2573}%
\special{pa 1615 2572}%
\special{pa 1622 2569}%
\special{pa 1623 2568}%
\special{pa 1626 2567}%
\special{pa 1628 2565}%
\special{pa 1635 2562}%
\special{pa 1636 2561}%
\special{pa 1643 2558}%
\special{pa 1644 2557}%
\special{pa 1645 2557}%
\special{pa 1647 2555}%
\special{pa 1654 2552}%
\special{pa 1655 2551}%
\special{pa 1658 2550}%
\special{pa 1660 2548}%
\special{pa 1667 2545}%
\special{pa 1669 2543}%
\special{pa 1676 2540}%
\special{pa 1678 2538}%
\special{pa 1685 2535}%
\special{pa 1687 2533}%
\special{pa 1690 2532}%
\special{pa 1691 2531}%
\special{pa 1692 2531}%
\special{pa 1694 2529}%
\special{pa 1697 2528}%
\special{pa 1698 2527}%
\special{pa 1699 2527}%
\special{pa 1701 2525}%
\special{pa 1704 2524}%
\special{pa 1706 2522}%
\special{pa 1709 2521}%
\special{pa 1710 2520}%
\special{pa 1711 2520}%
\special{pa 1713 2518}%
\special{pa 1716 2517}%
\special{pa 1718 2515}%
\special{pa 1721 2514}%
\special{pa 1723 2512}%
\special{pa 1726 2511}%
\special{pa 1727 2510}%
\special{pa 1728 2510}%
\special{pa 1730 2508}%
\special{pa 1733 2507}%
\special{pa 1735 2505}%
\special{pa 1736 2505}%
\special{pa 1738 2503}%
\special{pa 1741 2502}%
\special{pa 1743 2500}%
\special{pa 1746 2499}%
\special{pa 1748 2497}%
\special{pa 1751 2496}%
\special{pa 1753 2494}%
\special{pa 1754 2494}%
\special{pa 1756 2492}%
\special{pa 1759 2491}%
\special{pa 1761 2489}%
\special{pa 1762 2489}%
\special{pa 1764 2487}%
\special{pa 1767 2486}%
\special{pa 1769 2484}%
\special{pa 1770 2484}%
\special{pa 1772 2482}%
\special{pa 1775 2481}%
\special{pa 1777 2479}%
\special{pa 1778 2479}%
\special{pa 1780 2477}%
\special{pa 1781 2477}%
\special{pa 1783 2475}%
\special{pa 1784 2475}%
\special{pa 1786 2473}%
\special{pa 1789 2472}%
\special{pa 1791 2470}%
\special{pa 1792 2470}%
\special{pa 1794 2468}%
\special{pa 1795 2468}%
\special{pa 1797 2466}%
\special{pa 1798 2466}%
\special{pa 1800 2464}%
\special{pa 1801 2464}%
\special{pa 1803 2462}%
\special{pa 1804 2462}%
\special{pa 1806 2460}%
\special{pa 1807 2460}%
\special{pa 1809 2458}%
\special{pa 1810 2458}%
\special{pa 1812 2456}%
\special{pa 1813 2456}%
\special{pa 1815 2454}%
\special{pa 1816 2454}%
\special{pa 1818 2452}%
\special{pa 1819 2452}%
\special{pa 1821 2450}%
\special{pa 1822 2450}%
\special{pa 1824 2448}%
\special{pa 1825 2448}%
\special{pa 1827 2446}%
\special{pa 1828 2446}%
\special{pa 1830 2444}%
\special{pa 1831 2444}%
\special{pa 1833 2442}%
\special{pa 1834 2442}%
\special{pa 1836 2440}%
\special{pa 1837 2440}%
\special{pa 1840 2437}%
\special{pa 1841 2437}%
\special{pa 1843 2435}%
\special{pa 1844 2435}%
\special{pa 1846 2433}%
\special{pa 1847 2433}%
\special{pa 1849 2431}%
\special{pa 1850 2431}%
\special{pa 1853 2428}%
\special{pa 1854 2428}%
\special{pa 1856 2426}%
\special{pa 1857 2426}%
\special{pa 1859 2424}%
\special{pa 1860 2424}%
\special{pa 1863 2421}%
\special{pa 1864 2421}%
\special{pa 1866 2419}%
\special{pa 1867 2419}%
\special{pa 1870 2416}%
\special{pa 1871 2416}%
\special{pa 1873 2414}%
\special{pa 1874 2414}%
\special{pa 1877 2411}%
\special{pa 1878 2411}%
\special{pa 1880 2409}%
\special{pa 1881 2409}%
\special{pa 1884 2406}%
\special{pa 1885 2406}%
\special{pa 1887 2404}%
\special{pa 1888 2404}%
\special{pa 1891 2401}%
\special{pa 1892 2401}%
\special{pa 1895 2398}%
\special{pa 1896 2398}%
\special{pa 1899 2395}%
\special{pa 1900 2395}%
\special{pa 1902 2393}%
\special{pa 1903 2393}%
\special{pa 1906 2390}%
\special{pa 1907 2390}%
\special{pa 1910 2387}%
\special{pa 1911 2387}%
\special{pa 1914 2384}%
\special{pa 1915 2384}%
\special{pa 1918 2381}%
\special{pa 1919 2381}%
\special{pa 1922 2378}%
\special{pa 1923 2378}%
\special{pa 1927 2374}%
\special{pa 1928 2374}%
\special{pa 1931 2371}%
\special{pa 1932 2371}%
\special{pa 1935 2368}%
\special{pa 1936 2368}%
\special{pa 1939 2365}%
\special{pa 1940 2365}%
\special{pa 1944 2361}%
\special{pa 1945 2361}%
\special{pa 1948 2358}%
\special{pa 1949 2358}%
\special{pa 1953 2354}%
\special{pa 1954 2354}%
\special{pa 1958 2350}%
\special{pa 1959 2350}%
\special{pa 1963 2346}%
\special{pa 1964 2346}%
\special{pa 1967 2343}%
\special{pa 1968 2343}%
\special{pa 1972 2339}%
\special{pa 1973 2339}%
\special{pa 1978 2334}%
\special{pa 1979 2334}%
\special{pa 1983 2330}%
\special{pa 1984 2330}%
\special{pa 1988 2326}%
\special{pa 1989 2326}%
\special{pa 1994 2321}%
\special{pa 1995 2321}%
\special{pa 1999 2317}%
\special{pa 2000 2317}%
\special{pa 2005 2312}%
\special{pa 2006 2312}%
\special{pa 2011 2307}%
\special{pa 2012 2307}%
\special{pa 2018 2301}%
\special{pa 2019 2301}%
\special{pa 2024 2296}%
\special{pa 2025 2296}%
\special{pa 2031 2290}%
\special{pa 2032 2290}%
\special{pa 2038 2284}%
\special{pa 2039 2284}%
\special{pa 2046 2277}%
\special{pa 2047 2277}%
\special{pa 2054 2270}%
\special{pa 2055 2270}%
\special{pa 2063 2262}%
\special{pa 2064 2262}%
\special{pa 2072 2254}%
\special{pa 2073 2254}%
\special{pa 2082 2245}%
\special{pa 2083 2245}%
\special{pa 2094 2234}%
\special{pa 2095 2234}%
\special{pa 2108 2221}%
\special{pa 2109 2221}%
\special{pa 2126 2204}%
\special{pa 2127 2204}%
\special{pa 2204 2127}%
\special{pa 2204 2126}%
\special{pa 2221 2109}%
\special{pa 2221 2108}%
\special{pa 2234 2095}%
\special{pa 2234 2094}%
\special{pa 2245 2083}%
\special{pa 2245 2082}%
\special{pa 2254 2073}%
\special{pa 2254 2072}%
\special{pa 2262 2064}%
\special{pa 2262 2063}%
\special{pa 2270 2055}%
\special{pa 2270 2054}%
\special{pa 2277 2047}%
\special{pa 2277 2046}%
\special{pa 2284 2039}%
\special{pa 2284 2038}%
\special{pa 2290 2032}%
\special{pa 2290 2031}%
\special{pa 2296 2025}%
\special{pa 2296 2024}%
\special{pa 2301 2019}%
\special{pa 2301 2018}%
\special{pa 2307 2012}%
\special{pa 2307 2011}%
\special{pa 2312 2006}%
\special{pa 2312 2005}%
\special{pa 2317 2000}%
\special{pa 2317 1999}%
\special{pa 2321 1995}%
\special{pa 2321 1994}%
\special{pa 2326 1989}%
\special{pa 2326 1988}%
\special{pa 2330 1984}%
\special{pa 2330 1983}%
\special{pa 2334 1979}%
\special{pa 2334 1978}%
\special{pa 2339 1973}%
\special{pa 2339 1972}%
\special{pa 2343 1968}%
\special{pa 2343 1967}%
\special{pa 2346 1964}%
\special{pa 2346 1963}%
\special{pa 2350 1959}%
\special{pa 2350 1958}%
\special{pa 2354 1954}%
\special{pa 2354 1953}%
\special{pa 2358 1949}%
\special{pa 2358 1948}%
\special{pa 2361 1945}%
\special{pa 2361 1944}%
\special{pa 2365 1940}%
\special{pa 2365 1939}%
\special{pa 2368 1936}%
\special{pa 2368 1935}%
\special{pa 2371 1932}%
\special{pa 2371 1931}%
\special{pa 2374 1928}%
\special{pa 2374 1927}%
\special{pa 2378 1923}%
\special{pa 2378 1922}%
\special{pa 2381 1919}%
\special{pa 2381 1918}%
\special{pa 2384 1915}%
\special{pa 2384 1914}%
\special{pa 2387 1911}%
\special{pa 2387 1910}%
\special{pa 2390 1907}%
\special{pa 2390 1906}%
\special{pa 2393 1903}%
\special{pa 2393 1902}%
\special{pa 2395 1900}%
\special{pa 2395 1899}%
\special{pa 2398 1896}%
\special{pa 2398 1895}%
\special{pa 2401 1892}%
\special{pa 2401 1891}%
\special{pa 2404 1888}%
\special{pa 2404 1887}%
\special{pa 2406 1885}%
\special{pa 2406 1884}%
\special{pa 2409 1881}%
\special{pa 2409 1880}%
\special{pa 2411 1878}%
\special{pa 2411 1877}%
\special{pa 2414 1874}%
\special{pa 2414 1873}%
\special{pa 2416 1871}%
\special{pa 2416 1870}%
\special{pa 2419 1867}%
\special{pa 2419 1866}%
\special{pa 2421 1864}%
\special{pa 2421 1863}%
\special{pa 2424 1860}%
\special{pa 2424 1859}%
\special{pa 2426 1857}%
\special{pa 2426 1856}%
\special{pa 2428 1854}%
\special{pa 2428 1853}%
\special{pa 2431 1850}%
\special{pa 2431 1849}%
\special{pa 2433 1847}%
\special{pa 2433 1846}%
\special{pa 2435 1844}%
\special{pa 2435 1843}%
\special{pa 2437 1841}%
\special{pa 2437 1840}%
\special{pa 2440 1837}%
\special{pa 2440 1836}%
\special{pa 2442 1834}%
\special{pa 2442 1833}%
\special{pa 2444 1831}%
\special{pa 2444 1830}%
\special{pa 2446 1828}%
\special{pa 2446 1827}%
\special{pa 2448 1825}%
\special{pa 2448 1824}%
\special{pa 2450 1822}%
\special{pa 2450 1821}%
\special{pa 2452 1819}%
\special{pa 2452 1818}%
\special{pa 2454 1816}%
\special{pa 2454 1815}%
\special{pa 2456 1813}%
\special{pa 2456 1812}%
\special{pa 2458 1810}%
\special{pa 2458 1809}%
\special{pa 2460 1807}%
\special{pa 2460 1806}%
\special{pa 2462 1804}%
\special{pa 2462 1803}%
\special{pa 2464 1801}%
\special{pa 2464 1800}%
\special{pa 2466 1798}%
\special{pa 2466 1797}%
\special{pa 2468 1795}%
\special{pa 2468 1794}%
\special{pa 2470 1792}%
\special{pa 2470 1791}%
\special{pa 2472 1789}%
\special{pa 2473 1786}%
\special{pa 2475 1784}%
\special{pa 2475 1783}%
\special{pa 2477 1781}%
\special{pa 2477 1780}%
\special{pa 2479 1778}%
\special{pa 2479 1777}%
\special{pa 2481 1775}%
\special{pa 2482 1772}%
\special{pa 2484 1770}%
\special{pa 2484 1769}%
\special{pa 2486 1767}%
\special{pa 2487 1764}%
\special{pa 2489 1762}%
\special{pa 2489 1761}%
\special{pa 2491 1759}%
\special{pa 2492 1756}%
\special{pa 2494 1754}%
\special{pa 2494 1753}%
\special{pa 2496 1751}%
\special{pa 2497 1748}%
\special{pa 2499 1746}%
\special{pa 2500 1743}%
\special{pa 2502 1741}%
\special{pa 2503 1738}%
\special{pa 2505 1736}%
\special{pa 2505 1735}%
\special{pa 2507 1733}%
\special{pa 2508 1730}%
\special{pa 2510 1728}%
\special{pa 2511 1725}%
\special{pa 2512 1724}%
\special{pa 2512 1723}%
\special{pa 2514 1721}%
\special{pa 2515 1718}%
\special{pa 2517 1716}%
\special{pa 2518 1713}%
\special{pa 2520 1711}%
\special{pa 2521 1708}%
\special{pa 2522 1707}%
\special{pa 2522 1706}%
\special{pa 2524 1704}%
\special{pa 2525 1701}%
\special{pa 2527 1699}%
\special{pa 2528 1696}%
\special{pa 2529 1695}%
\special{pa 2529 1694}%
\special{pa 2531 1692}%
\special{pa 2532 1689}%
\special{pa 2533 1688}%
\special{pa 2533 1687}%
\special{pa 2535 1685}%
\special{pa 2538 1678}%
\special{pa 2540 1676}%
\special{pa 2543 1669}%
\special{pa 2545 1667}%
\special{pa 2548 1660}%
\special{pa 2550 1658}%
\special{pa 2553 1651}%
\special{pa 2554 1650}%
\special{pa 2555 1647}%
\special{pa 2557 1645}%
\special{pa 2560 1638}%
\special{pa 2561 1637}%
\special{pa 2564 1630}%
\special{pa 2565 1629}%
\special{pa 2565 1628}%
\special{pa 2567 1626}%
\special{pa 2570 1619}%
\special{pa 2571 1618}%
\special{pa 2574 1611}%
\special{pa 2575 1610}%
\special{pa 2578 1603}%
\special{pa 2579 1602}%
\special{pa 2582 1595}%
\special{pa 2583 1594}%
\special{pa 2586 1587}%
\special{pa 2587 1586}%
\special{pa 2590 1579}%
\special{pa 2591 1578}%
\special{pa 2592 1575}%
\special{pa 2593 1574}%
\special{pa 2593 1572}%
\special{pa 2594 1571}%
\special{pa 2597 1564}%
\special{pa 2598 1563}%
\special{pa 2601 1556}%
\special{pa 2602 1555}%
\special{pa 2602 1553}%
\special{pa 2603 1552}%
\special{pa 2606 1545}%
\special{pa 2607 1544}%
\special{pa 2607 1543}%
\special{pa 2608 1542}%
\special{pa 2608 1540}%
\special{pa 2609 1539}%
\special{pa 2610 1536}%
\special{pa 2611 1535}%
\special{pa 2611 1534}%
\special{pa 2612 1533}%
\special{pa 2612 1531}%
\special{pa 2613 1530}%
\special{pa 2614 1527}%
\special{pa 2615 1526}%
\special{pa 2615 1525}%
\special{pa 2616 1524}%
\special{pa 2616 1522}%
\special{pa 2617 1521}%
\special{pa 2618 1518}%
\special{pa 2619 1517}%
\special{pa 2619 1516}%
\special{pa 2620 1515}%
\special{pa 2620 1513}%
\special{pa 2621 1512}%
\special{pa 2622 1509}%
\special{pa 2623 1508}%
\special{pa 2623 1506}%
\special{pa 2624 1505}%
\special{pa 2625 1502}%
\special{pa 2626 1501}%
\special{pa 2626 1499}%
\special{pa 2627 1498}%
\special{pa 2627 1497}%
\special{pa 2628 1496}%
\special{pa 2628 1494}%
\special{pa 2629 1493}%
\special{pa 2630 1490}%
\special{pa 2631 1489}%
\special{pa 2631 1487}%
\special{pa 2632 1486}%
\special{pa 2632 1485}%
\special{pa 2633 1484}%
\special{pa 2633 1482}%
\special{pa 2634 1481}%
\special{pa 2634 1480}%
\special{pa 2635 1479}%
\special{pa 2635 1477}%
\special{pa 2636 1476}%
\special{pa 2637 1473}%
\special{pa 2638 1472}%
\special{pa 2638 1470}%
\special{pa 2639 1469}%
\special{pa 2639 1468}%
\special{pa 2640 1467}%
\special{pa 2641 1462}%
\special{pa 2642 1461}%
\special{pa 2642 1460}%
\special{pa 2643 1459}%
\special{pa 2643 1457}%
\special{pa 2644 1456}%
\special{pa 2644 1455}%
\special{pa 2645 1454}%
\special{pa 2645 1452}%
\special{pa 2646 1451}%
\special{pa 2646 1450}%
\special{pa 2647 1449}%
\special{pa 2648 1444}%
\special{pa 2649 1443}%
\special{pa 2649 1442}%
\special{pa 2650 1441}%
\special{pa 2651 1436}%
\special{pa 2652 1435}%
\special{pa 2652 1434}%
\special{pa 2653 1433}%
\special{pa 2654 1428}%
\special{pa 2655 1427}%
\special{pa 2655 1426}%
\special{pa 2656 1425}%
\special{pa 2659 1414}%
\special{pa 2660 1413}%
\special{pa 2660 1412}%
\special{pa 2661 1411}%
\special{pa 2664 1400}%
\special{pa 2665 1399}%
\special{pa 2668 1388}%
\special{pa 2669 1387}%
\special{pa 2672 1376}%
\special{pa 2673 1375}%
\special{pa 2676 1364}%
\special{pa 2677 1363}%
\special{pa 2677 1360}%
\special{pa 2678 1359}%
\special{pa 2679 1354}%
\special{pa 2680 1353}%
\special{pa 2680 1351}%
\special{pa 2681 1350}%
\special{pa 2681 1347}%
\special{pa 2682 1346}%
\special{pa 2683 1341}%
\special{pa 2684 1340}%
\special{pa 2684 1337}%
\special{pa 2685 1336}%
\special{pa 2685 1334}%
\special{pa 2686 1333}%
\special{pa 2686 1330}%
\special{pa 2687 1329}%
\special{pa 2687 1327}%
\special{pa 2688 1326}%
\special{pa 2688 1323}%
\special{pa 2689 1322}%
\special{pa 2689 1320}%
\special{pa 2690 1319}%
\special{pa 2690 1316}%
\special{pa 2691 1315}%
\special{pa 2691 1313}%
\special{pa 2692 1312}%
\special{pa 2693 1305}%
\special{pa 2694 1304}%
\special{pa 2694 1301}%
\special{pa 2695 1300}%
\special{pa 2695 1298}%
\special{pa 2696 1297}%
\special{pa 2699 1282}%
\special{pa 2700 1281}%
\special{pa 2700 1280}%
\special{fp}%
% FUNC 2 0 3 0 Black White  
% 10 400 400 2800 2800 1600 1600 2000 1600 1600 1200 1600 400 1600 2800 0 0 0 0 0 0
% x
\special{pn 8}%
\special{pn 8}%
\special{pa 400 2800}%
\special{pa 406 2794}%
\special{fp}%
\special{pa 432 2768}%
\special{pa 438 2762}%
\special{fp}%
\special{pa 465 2735}%
\special{pa 471 2729}%
\special{fp}%
\special{pa 497 2703}%
\special{pa 503 2697}%
\special{fp}%
\special{pa 530 2670}%
\special{pa 535 2665}%
\special{fp}%
\special{pa 562 2638}%
\special{pa 568 2632}%
\special{fp}%
\special{pa 595 2605}%
\special{pa 600 2600}%
\special{fp}%
\special{pa 627 2573}%
\special{pa 633 2567}%
\special{fp}%
\special{pa 659 2541}%
\special{pa 665 2535}%
\special{fp}%
\special{pa 692 2508}%
\special{pa 698 2502}%
\special{fp}%
\special{pa 724 2476}%
\special{pa 730 2470}%
\special{fp}%
\special{pa 757 2443}%
\special{pa 763 2437}%
\special{fp}%
\special{pa 789 2411}%
\special{pa 795 2405}%
\special{fp}%
\special{pa 822 2378}%
\special{pa 827 2373}%
\special{fp}%
\special{pa 854 2346}%
\special{pa 860 2340}%
\special{fp}%
\special{pa 886 2314}%
\special{pa 892 2308}%
\special{fp}%
\special{pa 919 2281}%
\special{pa 925 2275}%
\special{fp}%
\special{pa 951 2249}%
\special{pa 957 2243}%
\special{fp}%
\special{pa 984 2216}%
\special{pa 990 2210}%
\special{fp}%
\special{pa 1016 2184}%
\special{pa 1022 2178}%
\special{fp}%
\special{pa 1049 2151}%
\special{pa 1054 2146}%
\special{fp}%
\special{pa 1081 2119}%
\special{pa 1087 2113}%
\special{fp}%
\special{pa 1114 2086}%
\special{pa 1119 2081}%
\special{fp}%
\special{pa 1146 2054}%
\special{pa 1152 2048}%
\special{fp}%
\special{pa 1178 2022}%
\special{pa 1184 2016}%
\special{fp}%
\special{pa 1211 1989}%
\special{pa 1217 1983}%
\special{fp}%
\special{pa 1243 1957}%
\special{pa 1249 1951}%
\special{fp}%
\special{pa 1276 1924}%
\special{pa 1281 1919}%
\special{fp}%
\special{pa 1308 1892}%
\special{pa 1314 1886}%
\special{fp}%
\special{pa 1341 1859}%
\special{pa 1346 1854}%
\special{fp}%
\special{pa 1373 1827}%
\special{pa 1379 1821}%
\special{fp}%
\special{pa 1405 1795}%
\special{pa 1411 1789}%
\special{fp}%
\special{pa 1438 1762}%
\special{pa 1444 1756}%
\special{fp}%
\special{pa 1470 1730}%
\special{pa 1476 1724}%
\special{fp}%
\special{pa 1503 1697}%
\special{pa 1508 1692}%
\special{fp}%
\special{pa 1535 1665}%
\special{pa 1541 1659}%
\special{fp}%
\special{pa 1568 1632}%
\special{pa 1573 1627}%
\special{fp}%
\special{pn 8}%
\special{pa 1606 1594}%
\special{pa 1632 1568}%
\special{fp}%
\special{pa 1638 1562}%
\special{pa 1665 1535}%
\special{fp}%
\special{pa 1671 1529}%
\special{pa 1697 1503}%
\special{fp}%
\special{pa 1703 1497}%
\special{pa 1730 1470}%
\special{fp}%
\special{pa 1735 1465}%
\special{pa 1762 1438}%
\special{fp}%
\special{pa 1768 1432}%
\special{pa 1795 1405}%
\special{fp}%
\special{pa 1800 1400}%
\special{pa 1827 1373}%
\special{fp}%
\special{pa 1833 1367}%
\special{pa 1859 1341}%
\special{fp}%
\special{pa 1865 1335}%
\special{pa 1892 1308}%
\special{fp}%
\special{pa 1898 1302}%
\special{pa 1924 1276}%
\special{fp}%
\special{pa 1930 1270}%
\special{pa 1957 1243}%
\special{fp}%
\special{pa 1963 1237}%
\special{pa 1989 1211}%
\special{fp}%
\special{pa 1995 1205}%
\special{pa 2022 1178}%
\special{fp}%
\special{pa 2027 1173}%
\special{pa 2054 1146}%
\special{fp}%
\special{pa 2060 1140}%
\special{pa 2086 1114}%
\special{fp}%
\special{pa 2092 1108}%
\special{pa 2119 1081}%
\special{fp}%
\special{pa 2125 1075}%
\special{pa 2151 1049}%
\special{fp}%
\special{pa 2157 1043}%
\special{pa 2184 1016}%
\special{fp}%
\special{pa 2190 1010}%
\special{pa 2216 984}%
\special{fp}%
\special{pa 2222 978}%
\special{pa 2249 951}%
\special{fp}%
\special{pa 2254 946}%
\special{pa 2281 919}%
\special{fp}%
\special{pa 2287 913}%
\special{pa 2314 886}%
\special{fp}%
\special{pa 2319 881}%
\special{pa 2346 854}%
\special{fp}%
\special{pa 2352 848}%
\special{pa 2378 822}%
\special{fp}%
\special{pa 2384 816}%
\special{pa 2411 789}%
\special{fp}%
\special{pa 2417 783}%
\special{pa 2443 757}%
\special{fp}%
\special{pa 2449 751}%
\special{pa 2476 724}%
\special{fp}%
\special{pa 2481 719}%
\special{pa 2508 692}%
\special{fp}%
\special{pa 2514 686}%
\special{pa 2541 659}%
\special{fp}%
\special{pa 2546 654}%
\special{pa 2573 627}%
\special{fp}%
\special{pa 2579 621}%
\special{pa 2605 595}%
\special{fp}%
\special{pa 2611 589}%
\special{pa 2638 562}%
\special{fp}%
\special{pa 2644 556}%
\special{pa 2670 530}%
\special{fp}%
\special{pa 2676 524}%
\special{pa 2703 497}%
\special{fp}%
\special{pa 2708 492}%
\special{pa 2735 465}%
\special{fp}%
\special{pa 2741 459}%
\special{pa 2768 432}%
\special{fp}%
\special{pa 2773 427}%
\special{pa 2800 400}%
\special{fp}%
% FUNC 2 0 3 0 Black White  
% 10 400 400 2800 2800 1600 1600 2000 1600 1600 1200 1600 400 1600 2800 0 0 0 0 0 0
% -x
\special{pn 8}%
\special{pn 8}%
\special{pa 400 400}%
\special{pa 406 406}%
\special{fp}%
\special{pa 432 432}%
\special{pa 438 438}%
\special{fp}%
\special{pa 465 465}%
\special{pa 471 471}%
\special{fp}%
\special{pa 497 497}%
\special{pa 503 503}%
\special{fp}%
\special{pa 530 530}%
\special{pa 535 535}%
\special{fp}%
\special{pa 562 562}%
\special{pa 568 568}%
\special{fp}%
\special{pa 595 595}%
\special{pa 600 600}%
\special{fp}%
\special{pa 627 627}%
\special{pa 633 633}%
\special{fp}%
\special{pa 659 659}%
\special{pa 665 665}%
\special{fp}%
\special{pa 692 692}%
\special{pa 698 698}%
\special{fp}%
\special{pa 724 724}%
\special{pa 730 730}%
\special{fp}%
\special{pa 757 757}%
\special{pa 763 763}%
\special{fp}%
\special{pa 789 789}%
\special{pa 795 795}%
\special{fp}%
\special{pa 822 822}%
\special{pa 827 827}%
\special{fp}%
\special{pa 854 854}%
\special{pa 860 860}%
\special{fp}%
\special{pa 886 886}%
\special{pa 892 892}%
\special{fp}%
\special{pa 919 919}%
\special{pa 925 925}%
\special{fp}%
\special{pa 951 951}%
\special{pa 957 957}%
\special{fp}%
\special{pa 984 984}%
\special{pa 990 990}%
\special{fp}%
\special{pa 1016 1016}%
\special{pa 1022 1022}%
\special{fp}%
\special{pa 1049 1049}%
\special{pa 1054 1054}%
\special{fp}%
\special{pa 1081 1081}%
\special{pa 1087 1087}%
\special{fp}%
\special{pa 1114 1114}%
\special{pa 1119 1119}%
\special{fp}%
\special{pa 1146 1146}%
\special{pa 1152 1152}%
\special{fp}%
\special{pa 1178 1178}%
\special{pa 1184 1184}%
\special{fp}%
\special{pa 1211 1211}%
\special{pa 1217 1217}%
\special{fp}%
\special{pa 1243 1243}%
\special{pa 1249 1249}%
\special{fp}%
\special{pa 1276 1276}%
\special{pa 1281 1281}%
\special{fp}%
\special{pa 1308 1308}%
\special{pa 1314 1314}%
\special{fp}%
\special{pa 1341 1341}%
\special{pa 1346 1346}%
\special{fp}%
\special{pa 1373 1373}%
\special{pa 1379 1379}%
\special{fp}%
\special{pa 1405 1405}%
\special{pa 1411 1411}%
\special{fp}%
\special{pa 1438 1438}%
\special{pa 1444 1444}%
\special{fp}%
\special{pa 1470 1470}%
\special{pa 1476 1476}%
\special{fp}%
\special{pa 1503 1503}%
\special{pa 1508 1508}%
\special{fp}%
\special{pa 1535 1535}%
\special{pa 1541 1541}%
\special{fp}%
\special{pa 1568 1568}%
\special{pa 1573 1573}%
\special{fp}%
\special{pn 8}%
\special{pa 1606 1606}%
\special{pa 1632 1632}%
\special{fp}%
\special{pa 1638 1638}%
\special{pa 1665 1665}%
\special{fp}%
\special{pa 1671 1671}%
\special{pa 1697 1697}%
\special{fp}%
\special{pa 1703 1703}%
\special{pa 1730 1730}%
\special{fp}%
\special{pa 1735 1735}%
\special{pa 1762 1762}%
\special{fp}%
\special{pa 1768 1768}%
\special{pa 1795 1795}%
\special{fp}%
\special{pa 1800 1800}%
\special{pa 1827 1827}%
\special{fp}%
\special{pa 1833 1833}%
\special{pa 1859 1859}%
\special{fp}%
\special{pa 1865 1865}%
\special{pa 1892 1892}%
\special{fp}%
\special{pa 1898 1898}%
\special{pa 1924 1924}%
\special{fp}%
\special{pa 1930 1930}%
\special{pa 1957 1957}%
\special{fp}%
\special{pa 1963 1963}%
\special{pa 1989 1989}%
\special{fp}%
\special{pa 1995 1995}%
\special{pa 2022 2022}%
\special{fp}%
\special{pa 2027 2027}%
\special{pa 2054 2054}%
\special{fp}%
\special{pa 2060 2060}%
\special{pa 2086 2086}%
\special{fp}%
\special{pa 2092 2092}%
\special{pa 2119 2119}%
\special{fp}%
\special{pa 2125 2125}%
\special{pa 2151 2151}%
\special{fp}%
\special{pa 2157 2157}%
\special{pa 2184 2184}%
\special{fp}%
\special{pa 2190 2190}%
\special{pa 2216 2216}%
\special{fp}%
\special{pa 2222 2222}%
\special{pa 2249 2249}%
\special{fp}%
\special{pa 2254 2254}%
\special{pa 2281 2281}%
\special{fp}%
\special{pa 2287 2287}%
\special{pa 2314 2314}%
\special{fp}%
\special{pa 2319 2319}%
\special{pa 2346 2346}%
\special{fp}%
\special{pa 2352 2352}%
\special{pa 2378 2378}%
\special{fp}%
\special{pa 2384 2384}%
\special{pa 2411 2411}%
\special{fp}%
\special{pa 2417 2417}%
\special{pa 2443 2443}%
\special{fp}%
\special{pa 2449 2449}%
\special{pa 2476 2476}%
\special{fp}%
\special{pa 2481 2481}%
\special{pa 2508 2508}%
\special{fp}%
\special{pa 2514 2514}%
\special{pa 2541 2541}%
\special{fp}%
\special{pa 2546 2546}%
\special{pa 2573 2573}%
\special{fp}%
\special{pa 2579 2579}%
\special{pa 2605 2605}%
\special{fp}%
\special{pa 2611 2611}%
\special{pa 2638 2638}%
\special{fp}%
\special{pa 2644 2644}%
\special{pa 2670 2670}%
\special{fp}%
\special{pa 2676 2676}%
\special{pa 2703 2703}%
\special{fp}%
\special{pa 2708 2708}%
\special{pa 2735 2735}%
\special{fp}%
\special{pa 2741 2741}%
\special{pa 2768 2768}%
\special{fp}%
\special{pa 2773 2773}%
\special{pa 2800 2800}%
\special{fp}%
% STR 2 0 3 0 Black White  
% 4 1970 750 1970 850 5 0 1 0
% $\sqrt(6)$
\put(19.7000,-8.5000){\makebox(0,0){{\colorbox[named]{White}{$\sqrt(6)$}}}}%
% STR 2 0 3 0 Black White  
% 4 2040 1750 2040 1850 5 0 1 0
% $\sqrt(2)$
\put(20.4000,-18.5000){\makebox(0,0){{\colorbox[named]{White}{$\sqrt(2)$}}}}%
\end{picture}}%
}
     \end{center}           
また,求める面積$S$は,$C'$側で考えれば下図斜線部である.($x=0$,$y=0$を$\pi/4$回転させると$y=\pm x$に成る.) \\ 

     \begin{minipage}{0.5\hsize}
          \begin{center}
          \scalebox{.5}{%WinTpicVersion4.32a
{\unitlength 0.1in%
\begin{picture}(28.2000,24.0000)(3.8000,-28.0000)%
% STR 2 0 3 0 Black White  
% 4 590 1597 590 1610 4 400 0 0
% O
\put(5.9000,-16.1000){\makebox(0,0)[rt]{O}}%
% STR 2 0 3 0 Black White  
% 4 560 387 560 400 4 400 0 0
% $y$
\put(5.6000,-4.0000){\makebox(0,0)[rt]{$y$}}%
% STR 2 0 3 0 Black White  
% 4 3200 1627 3200 1640 4 400 0 0
% $x$
\put(32.0000,-16.4000){\makebox(0,0)[rt]{$x$}}%
% VECTOR 2 0 3 0 Black White  
% 2 600 2800 600 400
% 
\special{pn 8}%
\special{pa 600 2800}%
\special{pa 600 400}%
\special{fp}%
\special{sh 1}%
\special{pa 600 400}%
\special{pa 580 467}%
\special{pa 600 453}%
\special{pa 620 467}%
\special{pa 600 400}%
\special{fp}%
% VECTOR 2 0 3 0 Black White  
% 2 400 1600 3200 1600
% 
\special{pn 8}%
\special{pa 400 1600}%
\special{pa 3200 1600}%
\special{fp}%
\special{sh 1}%
\special{pa 3200 1600}%
\special{pa 3133 1580}%
\special{pa 3147 1600}%
\special{pa 3133 1620}%
\special{pa 3200 1600}%
\special{fp}%
% FUNC 2 0 3 0 Black White  
% 10 400 400 3200 2800 600 1600 1200 1600 600 1000 0 0 0 0 10 2 0 4 1 0
% ///x^2+3y^2=6///-3///3
\special{pn 8}%
\special{pa 2000 1858}%
\special{pa 2001 1857}%
\special{pa 2004 1850}%
\special{pa 2006 1848}%
\special{pa 2009 1841}%
\special{pa 2010 1840}%
\special{pa 2013 1833}%
\special{pa 2014 1832}%
\special{pa 2017 1825}%
\special{pa 2018 1824}%
\special{pa 2018 1823}%
\special{pa 2019 1822}%
\special{pa 2019 1820}%
\special{pa 2020 1819}%
\special{pa 2023 1812}%
\special{pa 2024 1811}%
\special{pa 2024 1809}%
\special{pa 2025 1808}%
\special{pa 2026 1805}%
\special{pa 2027 1804}%
\special{pa 2027 1802}%
\special{pa 2028 1801}%
\special{pa 2029 1798}%
\special{pa 2030 1797}%
\special{pa 2030 1795}%
\special{pa 2031 1794}%
\special{pa 2031 1793}%
\special{pa 2032 1792}%
\special{pa 2032 1790}%
\special{pa 2033 1789}%
\special{pa 2033 1788}%
\special{pa 2034 1787}%
\special{pa 2035 1782}%
\special{pa 2036 1781}%
\special{pa 2036 1780}%
\special{pa 2037 1779}%
\special{pa 2038 1774}%
\special{pa 2039 1773}%
\special{pa 2039 1772}%
\special{pa 2040 1771}%
\special{pa 2043 1760}%
\special{pa 2044 1759}%
\special{pa 2045 1754}%
\special{pa 2046 1753}%
\special{pa 2046 1751}%
\special{pa 2047 1750}%
\special{pa 2047 1747}%
\special{pa 2048 1746}%
\special{pa 2049 1741}%
\special{pa 2050 1740}%
\special{pa 2050 1737}%
\special{pa 2051 1736}%
\special{pa 2051 1734}%
\special{pa 2052 1733}%
\special{pa 2055 1718}%
\special{pa 2056 1717}%
\special{pa 2057 1710}%
\special{pa 2058 1709}%
\special{pa 2061 1690}%
\special{pa 2062 1689}%
\special{pa 2063 1678}%
\special{pa 2064 1677}%
\special{pa 2064 1672}%
\special{pa 2065 1671}%
\special{pa 2065 1665}%
\special{pa 2066 1664}%
\special{pa 2067 1647}%
\special{pa 2068 1646}%
\special{pa 2068 1635}%
\special{pa 2069 1634}%
\special{pa 2069 1614}%
\special{pa 2070 1613}%
\special{pa 2070 1587}%
\special{pa 2069 1586}%
\special{pa 2069 1566}%
\special{pa 2068 1565}%
\special{pa 2068 1554}%
\special{pa 2067 1553}%
\special{pa 2066 1536}%
\special{pa 2065 1535}%
\special{pa 2065 1529}%
\special{pa 2064 1528}%
\special{pa 2063 1517}%
\special{pa 2062 1516}%
\special{pa 2062 1511}%
\special{pa 2061 1510}%
\special{pa 2058 1491}%
\special{pa 2057 1490}%
\special{pa 2054 1475}%
\special{pa 2053 1474}%
\special{pa 2052 1467}%
\special{pa 2051 1466}%
\special{pa 2051 1464}%
\special{pa 2050 1463}%
\special{pa 2050 1460}%
\special{pa 2049 1459}%
\special{pa 2048 1454}%
\special{pa 2047 1453}%
\special{pa 2047 1450}%
\special{pa 2046 1449}%
\special{pa 2043 1438}%
\special{pa 2042 1437}%
\special{pa 2041 1432}%
\special{pa 2040 1431}%
\special{pa 2040 1429}%
\special{pa 2039 1428}%
\special{pa 2039 1427}%
\special{pa 2038 1426}%
\special{pa 2037 1421}%
\special{pa 2036 1420}%
\special{pa 2036 1419}%
\special{pa 2035 1418}%
\special{pa 2034 1413}%
\special{pa 2033 1412}%
\special{pa 2033 1411}%
\special{pa 2032 1410}%
\special{pa 2032 1408}%
\special{pa 2031 1407}%
\special{pa 2031 1406}%
\special{pa 2030 1405}%
\special{pa 2030 1403}%
\special{pa 2029 1402}%
\special{pa 2028 1399}%
\special{pa 2027 1398}%
\special{pa 2027 1396}%
\special{pa 2026 1395}%
\special{pa 2025 1392}%
\special{pa 2024 1391}%
\special{pa 2024 1389}%
\special{pa 2023 1388}%
\special{pa 2020 1381}%
\special{pa 2019 1380}%
\special{pa 2019 1378}%
\special{pa 2018 1377}%
\special{pa 2015 1370}%
\special{pa 2014 1369}%
\special{pa 2011 1362}%
\special{pa 2010 1361}%
\special{pa 2007 1354}%
\special{pa 2006 1353}%
\special{pa 2006 1352}%
\special{pa 2004 1350}%
\special{pa 2001 1343}%
\special{pa 2000 1342}%
\special{pa 2000 1341}%
\special{pa 1998 1339}%
\special{pa 1997 1336}%
\special{pa 1996 1335}%
\special{pa 1996 1334}%
\special{pa 1994 1332}%
\special{pa 1993 1329}%
\special{pa 1992 1328}%
\special{pa 1992 1327}%
\special{pa 1990 1325}%
\special{pa 1989 1322}%
\special{pa 1987 1320}%
\special{pa 1986 1317}%
\special{pa 1984 1315}%
\special{pa 1984 1314}%
\special{pa 1982 1312}%
\special{pa 1981 1309}%
\special{pa 1979 1307}%
\special{pa 1979 1306}%
\special{pa 1977 1304}%
\special{pa 1977 1303}%
\special{pa 1975 1301}%
\special{pa 1975 1300}%
\special{pa 1973 1298}%
\special{pa 1973 1297}%
\special{pa 1971 1295}%
\special{pa 1971 1294}%
\special{pa 1969 1292}%
\special{pa 1969 1291}%
\special{pa 1967 1289}%
\special{pa 1967 1288}%
\special{pa 1965 1286}%
\special{pa 1965 1285}%
\special{pa 1963 1283}%
\special{pa 1963 1282}%
\special{pa 1960 1279}%
\special{pa 1960 1278}%
\special{pa 1958 1276}%
\special{pa 1958 1275}%
\special{pa 1955 1272}%
\special{pa 1955 1271}%
\special{pa 1953 1269}%
\special{pa 1953 1268}%
\special{pa 1950 1265}%
\special{pa 1950 1264}%
\special{pa 1947 1261}%
\special{pa 1947 1260}%
\special{pa 1943 1256}%
\special{pa 1943 1255}%
\special{pa 1940 1252}%
\special{pa 1940 1251}%
\special{pa 1936 1247}%
\special{pa 1936 1246}%
\special{pa 1932 1242}%
\special{pa 1932 1241}%
\special{pa 1928 1237}%
\special{pa 1928 1236}%
\special{pa 1923 1231}%
\special{pa 1923 1230}%
\special{pa 1918 1225}%
\special{pa 1918 1224}%
\special{pa 1911 1217}%
\special{pa 1911 1216}%
\special{pa 1903 1208}%
\special{pa 1903 1207}%
\special{pa 1891 1195}%
\special{pa 1891 1194}%
\special{pa 1854 1157}%
\special{pa 1853 1157}%
\special{pa 1841 1145}%
\special{pa 1840 1145}%
\special{pa 1831 1136}%
\special{pa 1830 1136}%
\special{pa 1823 1129}%
\special{pa 1822 1129}%
\special{pa 1816 1123}%
\special{pa 1815 1123}%
\special{pa 1810 1118}%
\special{pa 1809 1118}%
\special{pa 1804 1113}%
\special{pa 1803 1113}%
\special{pa 1798 1108}%
\special{pa 1797 1108}%
\special{pa 1793 1104}%
\special{pa 1792 1104}%
\special{pa 1788 1100}%
\special{pa 1787 1100}%
\special{pa 1784 1097}%
\special{pa 1783 1097}%
\special{pa 1779 1093}%
\special{pa 1778 1093}%
\special{pa 1775 1090}%
\special{pa 1774 1090}%
\special{pa 1770 1086}%
\special{pa 1769 1086}%
\special{pa 1766 1083}%
\special{pa 1765 1083}%
\special{pa 1762 1080}%
\special{pa 1761 1080}%
\special{pa 1758 1077}%
\special{pa 1757 1077}%
\special{pa 1755 1075}%
\special{pa 1754 1075}%
\special{pa 1751 1072}%
\special{pa 1750 1072}%
\special{pa 1747 1069}%
\special{pa 1746 1069}%
\special{pa 1744 1067}%
\special{pa 1743 1067}%
\special{pa 1740 1064}%
\special{pa 1739 1064}%
\special{pa 1737 1062}%
\special{pa 1736 1062}%
\special{pa 1734 1060}%
\special{pa 1733 1060}%
\special{pa 1730 1057}%
\special{pa 1729 1057}%
\special{pa 1727 1055}%
\special{pa 1726 1055}%
\special{pa 1724 1053}%
\special{pa 1723 1053}%
\special{pa 1721 1051}%
\special{pa 1720 1051}%
\special{pa 1718 1049}%
\special{pa 1717 1049}%
\special{pa 1715 1047}%
\special{pa 1714 1047}%
\special{pa 1712 1045}%
\special{pa 1711 1045}%
\special{pa 1709 1043}%
\special{pa 1708 1043}%
\special{pa 1706 1041}%
\special{pa 1705 1041}%
\special{pa 1703 1039}%
\special{pa 1702 1039}%
\special{pa 1700 1037}%
\special{pa 1699 1037}%
\special{pa 1697 1035}%
\special{pa 1696 1035}%
\special{pa 1694 1033}%
\special{pa 1693 1033}%
\special{pa 1691 1031}%
\special{pa 1688 1030}%
\special{pa 1686 1028}%
\special{pa 1685 1028}%
\special{pa 1683 1026}%
\special{pa 1680 1025}%
\special{pa 1678 1023}%
\special{pa 1677 1023}%
\special{pa 1675 1021}%
\special{pa 1672 1020}%
\special{pa 1670 1018}%
\special{pa 1669 1018}%
\special{pa 1667 1016}%
\special{pa 1664 1015}%
\special{pa 1662 1013}%
\special{pa 1659 1012}%
\special{pa 1657 1010}%
\special{pa 1654 1009}%
\special{pa 1652 1007}%
\special{pa 1649 1006}%
\special{pa 1648 1005}%
\special{pa 1647 1005}%
\special{pa 1645 1003}%
\special{pa 1642 1002}%
\special{pa 1640 1000}%
\special{pa 1637 999}%
\special{pa 1636 998}%
\special{pa 1635 998}%
\special{pa 1633 996}%
\special{pa 1630 995}%
\special{pa 1629 994}%
\special{pa 1628 994}%
\special{pa 1626 992}%
\special{pa 1619 989}%
\special{pa 1617 987}%
\special{pa 1610 984}%
\special{pa 1608 982}%
\special{pa 1601 979}%
\special{pa 1600 978}%
\special{pa 1597 977}%
\special{pa 1596 976}%
\special{pa 1595 976}%
\special{pa 1593 974}%
\special{pa 1586 971}%
\special{pa 1585 970}%
\special{pa 1578 967}%
\special{pa 1577 966}%
\special{pa 1570 963}%
\special{pa 1569 962}%
\special{pa 1562 959}%
\special{pa 1561 958}%
\special{pa 1554 955}%
\special{pa 1553 954}%
\special{pa 1546 951}%
\special{pa 1545 950}%
\special{pa 1538 947}%
\special{pa 1537 946}%
\special{pa 1530 943}%
\special{pa 1529 942}%
\special{pa 1527 942}%
\special{pa 1526 941}%
\special{pa 1519 938}%
\special{pa 1518 937}%
\special{pa 1517 937}%
\special{pa 1516 936}%
\special{pa 1514 936}%
\special{pa 1513 935}%
\special{pa 1506 932}%
\special{pa 1505 931}%
\special{pa 1503 931}%
\special{pa 1502 930}%
\special{pa 1499 929}%
\special{pa 1498 928}%
\special{pa 1497 928}%
\special{pa 1496 927}%
\special{pa 1494 927}%
\special{pa 1493 926}%
\special{pa 1490 925}%
\special{pa 1489 924}%
\special{pa 1488 924}%
\special{pa 1487 923}%
\special{pa 1485 923}%
\special{pa 1484 922}%
\special{pa 1481 921}%
\special{pa 1480 920}%
\special{pa 1478 920}%
\special{pa 1477 919}%
\special{pa 1474 918}%
\special{pa 1473 917}%
\special{pa 1471 917}%
\special{pa 1470 916}%
\special{pa 1467 915}%
\special{pa 1466 914}%
\special{pa 1464 914}%
\special{pa 1463 913}%
\special{pa 1460 912}%
\special{pa 1459 911}%
\special{pa 1457 911}%
\special{pa 1456 910}%
\special{pa 1455 910}%
\special{pa 1454 909}%
\special{pa 1452 909}%
\special{pa 1451 908}%
\special{pa 1450 908}%
\special{pa 1449 907}%
\special{pa 1447 907}%
\special{pa 1446 906}%
\special{pa 1445 906}%
\special{pa 1444 905}%
\special{pa 1442 905}%
\special{pa 1441 904}%
\special{pa 1440 904}%
\special{pa 1439 903}%
\special{pa 1437 903}%
\special{pa 1436 902}%
\special{pa 1435 902}%
\special{pa 1434 901}%
\special{pa 1432 901}%
\special{pa 1431 900}%
\special{pa 1430 900}%
\special{pa 1429 899}%
\special{pa 1427 899}%
\special{pa 1426 898}%
\special{pa 1425 898}%
\special{pa 1424 897}%
\special{pa 1422 897}%
\special{pa 1421 896}%
\special{pa 1420 896}%
\special{pa 1419 895}%
\special{pa 1414 894}%
\special{pa 1413 893}%
\special{pa 1412 893}%
\special{pa 1411 892}%
\special{pa 1409 892}%
\special{pa 1408 891}%
\special{pa 1407 891}%
\special{pa 1406 890}%
\special{pa 1401 889}%
\special{pa 1400 888}%
\special{pa 1399 888}%
\special{pa 1398 887}%
\special{pa 1393 886}%
\special{pa 1392 885}%
\special{pa 1391 885}%
\special{pa 1390 884}%
\special{pa 1385 883}%
\special{pa 1384 882}%
\special{pa 1382 882}%
\special{pa 1381 881}%
\special{pa 1380 881}%
\special{pa 1379 880}%
\special{pa 1368 877}%
\special{pa 1367 876}%
\special{pa 1366 876}%
\special{pa 1365 875}%
\special{pa 1354 872}%
\special{pa 1353 871}%
\special{pa 1342 868}%
\special{pa 1341 867}%
\special{pa 1330 864}%
\special{pa 1329 863}%
\special{pa 1318 860}%
\special{pa 1317 859}%
\special{pa 1306 856}%
\special{pa 1305 855}%
\special{pa 1302 855}%
\special{pa 1301 854}%
\special{pa 1290 851}%
\special{pa 1289 850}%
\special{pa 1286 850}%
\special{pa 1285 849}%
\special{pa 1280 848}%
\special{pa 1279 847}%
\special{pa 1276 847}%
\special{pa 1275 846}%
\special{pa 1270 845}%
\special{pa 1269 844}%
\special{pa 1266 844}%
\special{pa 1265 843}%
\special{pa 1263 843}%
\special{pa 1262 842}%
\special{pa 1259 842}%
\special{pa 1258 841}%
\special{pa 1256 841}%
\special{pa 1255 840}%
\special{pa 1252 840}%
\special{pa 1251 839}%
\special{pa 1249 839}%
\special{pa 1248 838}%
\special{pa 1245 838}%
\special{pa 1244 837}%
\special{pa 1242 837}%
\special{pa 1241 836}%
\special{pa 1238 836}%
\special{pa 1237 835}%
\special{pa 1235 835}%
\special{pa 1234 834}%
\special{pa 1227 833}%
\special{pa 1226 832}%
\special{pa 1224 832}%
\special{pa 1223 831}%
\special{pa 1216 830}%
\special{pa 1215 829}%
\special{pa 1212 829}%
\special{pa 1211 828}%
\special{pa 1209 828}%
\special{pa 1208 827}%
\special{pa 1193 824}%
\special{pa 1192 823}%
\special{pa 1177 820}%
\special{pa 1176 819}%
\special{pa 1161 816}%
\special{pa 1160 815}%
\special{pa 1156 815}%
\special{pa 1155 814}%
\special{pa 1148 813}%
\special{pa 1147 812}%
\special{pa 1143 812}%
\special{pa 1142 811}%
\special{pa 1135 810}%
\special{pa 1134 809}%
\special{pa 1130 809}%
\special{pa 1129 808}%
\special{pa 1126 808}%
\special{pa 1125 807}%
\special{pa 1121 807}%
\special{pa 1120 806}%
\special{pa 1117 806}%
\special{pa 1116 805}%
\special{pa 1107 804}%
\special{pa 1106 803}%
\special{pa 1102 803}%
\special{pa 1101 802}%
\special{pa 1098 802}%
\special{pa 1097 801}%
\special{pa 1078 798}%
\special{pa 1077 797}%
\special{pa 1068 796}%
\special{pa 1067 795}%
\special{pa 1063 795}%
\special{pa 1062 794}%
\special{pa 1057 794}%
\special{pa 1056 793}%
\special{pa 1047 792}%
\special{pa 1046 791}%
\special{pa 1041 791}%
\special{pa 1040 790}%
\special{pa 1036 790}%
\special{pa 1035 789}%
\special{pa 1024 788}%
\special{pa 1023 787}%
\special{pa 1018 787}%
\special{pa 1017 786}%
\special{pa 1013 786}%
\special{pa 1012 785}%
\special{pa 1007 785}%
\special{pa 1006 784}%
\special{pa 1000 784}%
\special{pa 999 783}%
\special{pa 988 782}%
\special{pa 987 781}%
\special{pa 982 781}%
\special{pa 981 780}%
\special{pa 968 779}%
\special{pa 967 778}%
\special{pa 962 778}%
\special{pa 961 777}%
\special{pa 948 776}%
\special{pa 947 775}%
\special{pa 941 775}%
\special{pa 940 774}%
\special{pa 933 774}%
\special{pa 932 773}%
\special{pa 926 773}%
\special{pa 925 772}%
\special{pa 910 771}%
\special{pa 909 770}%
\special{pa 902 770}%
\special{pa 901 769}%
\special{pa 893 769}%
\special{pa 892 768}%
\special{pa 885 768}%
\special{pa 884 767}%
\special{pa 867 766}%
\special{pa 866 765}%
\special{pa 847 764}%
\special{pa 846 763}%
\special{pa 837 763}%
\special{pa 836 762}%
\special{pa 826 762}%
\special{pa 825 761}%
\special{pa 802 760}%
\special{pa 801 759}%
\special{pa 789 759}%
\special{pa 788 758}%
\special{pa 775 758}%
\special{pa 774 757}%
\special{pa 760 757}%
\special{pa 759 756}%
\special{pa 744 756}%
\special{pa 743 755}%
\special{pa 725 755}%
\special{pa 724 754}%
\special{pa 702 754}%
\special{pa 701 753}%
\special{pa 673 753}%
\special{pa 672 752}%
\special{pa 612 752}%
\special{pa 611 751}%
\special{pa 589 751}%
\special{pa 588 752}%
\special{pa 528 752}%
\special{pa 527 753}%
\special{pa 499 753}%
\special{pa 498 754}%
\special{pa 476 754}%
\special{pa 475 755}%
\special{pa 457 755}%
\special{pa 456 756}%
\special{pa 441 756}%
\special{pa 440 757}%
\special{pa 426 757}%
\special{pa 425 758}%
\special{pa 412 758}%
\special{pa 411 759}%
\special{pa 400 759}%
\special{fp}%
\special{pa 2000 1858}%
\special{pa 2000 1859}%
\special{pa 1998 1861}%
\special{pa 1997 1864}%
\special{pa 1996 1865}%
\special{pa 1996 1866}%
\special{pa 1994 1868}%
\special{pa 1993 1871}%
\special{pa 1992 1872}%
\special{pa 1992 1873}%
\special{pa 1990 1875}%
\special{pa 1989 1878}%
\special{pa 1987 1880}%
\special{pa 1986 1883}%
\special{pa 1984 1885}%
\special{pa 1984 1886}%
\special{pa 1982 1888}%
\special{pa 1981 1891}%
\special{pa 1979 1893}%
\special{pa 1979 1894}%
\special{pa 1977 1896}%
\special{pa 1977 1897}%
\special{pa 1975 1899}%
\special{pa 1975 1900}%
\special{pa 1973 1902}%
\special{pa 1973 1903}%
\special{pa 1971 1905}%
\special{pa 1971 1906}%
\special{pa 1969 1908}%
\special{pa 1969 1909}%
\special{pa 1967 1911}%
\special{pa 1967 1912}%
\special{pa 1965 1914}%
\special{pa 1965 1915}%
\special{pa 1963 1917}%
\special{pa 1963 1918}%
\special{pa 1960 1921}%
\special{pa 1960 1922}%
\special{pa 1958 1924}%
\special{pa 1958 1925}%
\special{pa 1955 1928}%
\special{pa 1955 1929}%
\special{pa 1953 1931}%
\special{pa 1953 1932}%
\special{pa 1950 1935}%
\special{pa 1950 1936}%
\special{pa 1947 1939}%
\special{pa 1947 1940}%
\special{pa 1943 1944}%
\special{pa 1943 1945}%
\special{pa 1940 1948}%
\special{pa 1940 1949}%
\special{pa 1936 1953}%
\special{pa 1936 1954}%
\special{pa 1932 1958}%
\special{pa 1932 1959}%
\special{pa 1928 1963}%
\special{pa 1928 1964}%
\special{pa 1923 1969}%
\special{pa 1923 1970}%
\special{pa 1918 1975}%
\special{pa 1918 1976}%
\special{pa 1911 1983}%
\special{pa 1911 1984}%
\special{pa 1903 1992}%
\special{pa 1903 1993}%
\special{pa 1891 2005}%
\special{pa 1891 2006}%
\special{pa 1854 2043}%
\special{pa 1853 2043}%
\special{pa 1841 2055}%
\special{pa 1840 2055}%
\special{pa 1831 2064}%
\special{pa 1830 2064}%
\special{pa 1823 2071}%
\special{pa 1822 2071}%
\special{pa 1816 2077}%
\special{pa 1815 2077}%
\special{pa 1810 2082}%
\special{pa 1809 2082}%
\special{pa 1804 2087}%
\special{pa 1803 2087}%
\special{pa 1798 2092}%
\special{pa 1797 2092}%
\special{pa 1793 2096}%
\special{pa 1792 2096}%
\special{pa 1788 2100}%
\special{pa 1787 2100}%
\special{pa 1784 2103}%
\special{pa 1783 2103}%
\special{pa 1779 2107}%
\special{pa 1778 2107}%
\special{pa 1775 2110}%
\special{pa 1774 2110}%
\special{pa 1770 2114}%
\special{pa 1769 2114}%
\special{pa 1766 2117}%
\special{pa 1765 2117}%
\special{pa 1762 2120}%
\special{pa 1761 2120}%
\special{pa 1758 2123}%
\special{pa 1757 2123}%
\special{pa 1755 2125}%
\special{pa 1754 2125}%
\special{pa 1751 2128}%
\special{pa 1750 2128}%
\special{pa 1747 2131}%
\special{pa 1746 2131}%
\special{pa 1744 2133}%
\special{pa 1743 2133}%
\special{pa 1740 2136}%
\special{pa 1739 2136}%
\special{pa 1737 2138}%
\special{pa 1736 2138}%
\special{pa 1734 2140}%
\special{pa 1733 2140}%
\special{pa 1730 2143}%
\special{pa 1729 2143}%
\special{pa 1727 2145}%
\special{pa 1726 2145}%
\special{pa 1724 2147}%
\special{pa 1723 2147}%
\special{pa 1721 2149}%
\special{pa 1720 2149}%
\special{pa 1718 2151}%
\special{pa 1717 2151}%
\special{pa 1715 2153}%
\special{pa 1714 2153}%
\special{pa 1712 2155}%
\special{pa 1711 2155}%
\special{pa 1709 2157}%
\special{pa 1708 2157}%
\special{pa 1706 2159}%
\special{pa 1705 2159}%
\special{pa 1703 2161}%
\special{pa 1702 2161}%
\special{pa 1700 2163}%
\special{pa 1699 2163}%
\special{pa 1697 2165}%
\special{pa 1696 2165}%
\special{pa 1694 2167}%
\special{pa 1693 2167}%
\special{pa 1691 2169}%
\special{pa 1688 2170}%
\special{pa 1686 2172}%
\special{pa 1685 2172}%
\special{pa 1683 2174}%
\special{pa 1680 2175}%
\special{pa 1678 2177}%
\special{pa 1677 2177}%
\special{pa 1675 2179}%
\special{pa 1672 2180}%
\special{pa 1670 2182}%
\special{pa 1669 2182}%
\special{pa 1667 2184}%
\special{pa 1664 2185}%
\special{pa 1662 2187}%
\special{pa 1659 2188}%
\special{pa 1657 2190}%
\special{pa 1654 2191}%
\special{pa 1652 2193}%
\special{pa 1649 2194}%
\special{pa 1648 2195}%
\special{pa 1647 2195}%
\special{pa 1645 2197}%
\special{pa 1642 2198}%
\special{pa 1640 2200}%
\special{pa 1637 2201}%
\special{pa 1636 2202}%
\special{pa 1635 2202}%
\special{pa 1633 2204}%
\special{pa 1630 2205}%
\special{pa 1629 2206}%
\special{pa 1628 2206}%
\special{pa 1626 2208}%
\special{pa 1619 2211}%
\special{pa 1617 2213}%
\special{pa 1610 2216}%
\special{pa 1608 2218}%
\special{pa 1601 2221}%
\special{pa 1600 2222}%
\special{pa 1597 2223}%
\special{pa 1596 2224}%
\special{pa 1595 2224}%
\special{pa 1593 2226}%
\special{pa 1586 2229}%
\special{pa 1585 2230}%
\special{pa 1578 2233}%
\special{pa 1577 2234}%
\special{pa 1570 2237}%
\special{pa 1569 2238}%
\special{pa 1562 2241}%
\special{pa 1561 2242}%
\special{pa 1554 2245}%
\special{pa 1553 2246}%
\special{pa 1546 2249}%
\special{pa 1545 2250}%
\special{pa 1538 2253}%
\special{pa 1537 2254}%
\special{pa 1530 2257}%
\special{pa 1529 2258}%
\special{pa 1527 2258}%
\special{pa 1526 2259}%
\special{pa 1519 2262}%
\special{pa 1518 2263}%
\special{pa 1517 2263}%
\special{pa 1516 2264}%
\special{pa 1514 2264}%
\special{pa 1513 2265}%
\special{pa 1506 2268}%
\special{pa 1505 2269}%
\special{pa 1503 2269}%
\special{pa 1502 2270}%
\special{pa 1499 2271}%
\special{pa 1498 2272}%
\special{pa 1497 2272}%
\special{pa 1496 2273}%
\special{pa 1494 2273}%
\special{pa 1493 2274}%
\special{pa 1490 2275}%
\special{pa 1489 2276}%
\special{pa 1488 2276}%
\special{pa 1487 2277}%
\special{pa 1485 2277}%
\special{pa 1484 2278}%
\special{pa 1481 2279}%
\special{pa 1480 2280}%
\special{pa 1478 2280}%
\special{pa 1477 2281}%
\special{pa 1474 2282}%
\special{pa 1473 2283}%
\special{pa 1471 2283}%
\special{pa 1470 2284}%
\special{pa 1467 2285}%
\special{pa 1466 2286}%
\special{pa 1464 2286}%
\special{pa 1463 2287}%
\special{pa 1460 2288}%
\special{pa 1459 2289}%
\special{pa 1457 2289}%
\special{pa 1456 2290}%
\special{pa 1455 2290}%
\special{pa 1454 2291}%
\special{pa 1452 2291}%
\special{pa 1451 2292}%
\special{pa 1450 2292}%
\special{pa 1449 2293}%
\special{pa 1447 2293}%
\special{pa 1446 2294}%
\special{pa 1445 2294}%
\special{pa 1444 2295}%
\special{pa 1442 2295}%
\special{pa 1441 2296}%
\special{pa 1440 2296}%
\special{pa 1439 2297}%
\special{pa 1437 2297}%
\special{pa 1436 2298}%
\special{pa 1435 2298}%
\special{pa 1434 2299}%
\special{pa 1432 2299}%
\special{pa 1431 2300}%
\special{pa 1430 2300}%
\special{pa 1429 2301}%
\special{pa 1427 2301}%
\special{pa 1426 2302}%
\special{pa 1425 2302}%
\special{pa 1424 2303}%
\special{pa 1422 2303}%
\special{pa 1421 2304}%
\special{pa 1420 2304}%
\special{pa 1419 2305}%
\special{pa 1414 2306}%
\special{pa 1413 2307}%
\special{pa 1412 2307}%
\special{pa 1411 2308}%
\special{pa 1409 2308}%
\special{pa 1408 2309}%
\special{pa 1407 2309}%
\special{pa 1406 2310}%
\special{pa 1401 2311}%
\special{pa 1400 2312}%
\special{pa 1399 2312}%
\special{pa 1398 2313}%
\special{pa 1393 2314}%
\special{pa 1392 2315}%
\special{pa 1391 2315}%
\special{pa 1390 2316}%
\special{pa 1385 2317}%
\special{pa 1384 2318}%
\special{pa 1382 2318}%
\special{pa 1381 2319}%
\special{pa 1380 2319}%
\special{pa 1379 2320}%
\special{pa 1368 2323}%
\special{pa 1367 2324}%
\special{pa 1366 2324}%
\special{pa 1365 2325}%
\special{pa 1354 2328}%
\special{pa 1353 2329}%
\special{pa 1342 2332}%
\special{pa 1341 2333}%
\special{pa 1330 2336}%
\special{pa 1329 2337}%
\special{pa 1318 2340}%
\special{pa 1317 2341}%
\special{pa 1306 2344}%
\special{pa 1305 2345}%
\special{pa 1302 2345}%
\special{pa 1301 2346}%
\special{pa 1290 2349}%
\special{pa 1289 2350}%
\special{pa 1286 2350}%
\special{pa 1285 2351}%
\special{pa 1280 2352}%
\special{pa 1279 2353}%
\special{pa 1276 2353}%
\special{pa 1275 2354}%
\special{pa 1270 2355}%
\special{pa 1269 2356}%
\special{pa 1266 2356}%
\special{pa 1265 2357}%
\special{pa 1263 2357}%
\special{pa 1262 2358}%
\special{pa 1259 2358}%
\special{pa 1258 2359}%
\special{pa 1256 2359}%
\special{pa 1255 2360}%
\special{pa 1252 2360}%
\special{pa 1251 2361}%
\special{pa 1249 2361}%
\special{pa 1248 2362}%
\special{pa 1245 2362}%
\special{pa 1244 2363}%
\special{pa 1242 2363}%
\special{pa 1241 2364}%
\special{pa 1238 2364}%
\special{pa 1237 2365}%
\special{pa 1235 2365}%
\special{pa 1234 2366}%
\special{pa 1227 2367}%
\special{pa 1226 2368}%
\special{pa 1224 2368}%
\special{pa 1223 2369}%
\special{pa 1216 2370}%
\special{pa 1215 2371}%
\special{pa 1212 2371}%
\special{pa 1211 2372}%
\special{pa 1209 2372}%
\special{pa 1208 2373}%
\special{pa 1193 2376}%
\special{pa 1192 2377}%
\special{pa 1177 2380}%
\special{pa 1176 2381}%
\special{pa 1161 2384}%
\special{pa 1160 2385}%
\special{pa 1156 2385}%
\special{pa 1155 2386}%
\special{pa 1148 2387}%
\special{pa 1147 2388}%
\special{pa 1143 2388}%
\special{pa 1142 2389}%
\special{pa 1135 2390}%
\special{pa 1134 2391}%
\special{pa 1130 2391}%
\special{pa 1129 2392}%
\special{pa 1126 2392}%
\special{pa 1125 2393}%
\special{pa 1121 2393}%
\special{pa 1120 2394}%
\special{pa 1117 2394}%
\special{pa 1116 2395}%
\special{pa 1107 2396}%
\special{pa 1106 2397}%
\special{pa 1102 2397}%
\special{pa 1101 2398}%
\special{pa 1098 2398}%
\special{pa 1097 2399}%
\special{pa 1078 2402}%
\special{pa 1077 2403}%
\special{pa 1068 2404}%
\special{pa 1067 2405}%
\special{pa 1063 2405}%
\special{pa 1062 2406}%
\special{pa 1057 2406}%
\special{pa 1056 2407}%
\special{pa 1047 2408}%
\special{pa 1046 2409}%
\special{pa 1041 2409}%
\special{pa 1040 2410}%
\special{pa 1036 2410}%
\special{pa 1035 2411}%
\special{pa 1024 2412}%
\special{pa 1023 2413}%
\special{pa 1018 2413}%
\special{pa 1017 2414}%
\special{pa 1013 2414}%
\special{pa 1012 2415}%
\special{pa 1007 2415}%
\special{pa 1006 2416}%
\special{pa 1000 2416}%
\special{pa 999 2417}%
\special{pa 988 2418}%
\special{pa 987 2419}%
\special{pa 982 2419}%
\special{pa 981 2420}%
\special{pa 968 2421}%
\special{pa 967 2422}%
\special{pa 962 2422}%
\special{pa 961 2423}%
\special{pa 948 2424}%
\special{pa 947 2425}%
\special{pa 941 2425}%
\special{pa 940 2426}%
\special{pa 933 2426}%
\special{pa 932 2427}%
\special{pa 926 2427}%
\special{pa 925 2428}%
\special{pa 910 2429}%
\special{pa 909 2430}%
\special{pa 902 2430}%
\special{pa 901 2431}%
\special{pa 893 2431}%
\special{pa 892 2432}%
\special{pa 885 2432}%
\special{pa 884 2433}%
\special{pa 867 2434}%
\special{pa 866 2435}%
\special{pa 847 2436}%
\special{pa 846 2437}%
\special{pa 837 2437}%
\special{pa 836 2438}%
\special{pa 826 2438}%
\special{pa 825 2439}%
\special{pa 802 2440}%
\special{pa 801 2441}%
\special{pa 789 2441}%
\special{pa 788 2442}%
\special{pa 775 2442}%
\special{pa 774 2443}%
\special{pa 760 2443}%
\special{pa 759 2444}%
\special{pa 744 2444}%
\special{pa 743 2445}%
\special{pa 725 2445}%
\special{pa 724 2446}%
\special{pa 702 2446}%
\special{pa 701 2447}%
\special{pa 673 2447}%
\special{pa 672 2448}%
\special{pa 612 2448}%
\special{pa 611 2449}%
\special{pa 589 2449}%
\special{pa 588 2448}%
\special{pa 528 2448}%
\special{pa 527 2447}%
\special{pa 499 2447}%
\special{pa 498 2446}%
\special{pa 476 2446}%
\special{pa 475 2445}%
\special{pa 457 2445}%
\special{pa 456 2444}%
\special{pa 441 2444}%
\special{pa 440 2443}%
\special{pa 426 2443}%
\special{pa 425 2442}%
\special{pa 412 2442}%
\special{pa 411 2441}%
\special{pa 400 2441}%
\special{fp}%
% FUNC 2 0 3 0 Black White  
% 9 400 400 3200 2800 600 1600 1200 1600 600 1000 400 400 3200 2800 0 0 0 0
% x
\special{pn 8}%
\special{pa 400 1800}%
\special{pa 1800 400}%
\special{fp}%
% FUNC 2 0 3 0 Black White  
% 9 400 400 3200 2800 600 1600 1200 1600 600 1000 400 400 3200 2800 0 0 0 0
% -x
\special{pn 8}%
\special{pa 400 1400}%
\special{pa 1800 2800}%
\special{fp}%
% LINE 3 0 3 0 Black White  
% 48 1680 1020 1100 1600 1640 1000 1040 1600 1600 980 980 1600 1560 960 920 1600 1520 940 860 1600 1480 920 800 1600 1440 900 740 1600 1390 890 680 1600 1350 870 620 1600 1710 1050 1160 1600 1750 1070 1220 1600 1780 1100 1280 1600 1810 1130 1340 1600 1850 1150 1400 1600 1880 1180 1460 1600 1910 1210 1520 1600 1930 1250 1580 1600 1960 1280 1640 1600 1980 1320 1700 1600 2000 1360 1760 1600 2020 1400 1820 1600 2040 1440 1880 1600 2050 1490 1940 1600 2060 1540 2000 1600
% 
\special{pn 4}%
\special{pa 1680 1020}%
\special{pa 1100 1600}%
\special{fp}%
\special{pa 1640 1000}%
\special{pa 1040 1600}%
\special{fp}%
\special{pa 1600 980}%
\special{pa 980 1600}%
\special{fp}%
\special{pa 1560 960}%
\special{pa 920 1600}%
\special{fp}%
\special{pa 1520 940}%
\special{pa 860 1600}%
\special{fp}%
\special{pa 1480 920}%
\special{pa 800 1600}%
\special{fp}%
\special{pa 1440 900}%
\special{pa 740 1600}%
\special{fp}%
\special{pa 1390 890}%
\special{pa 680 1600}%
\special{fp}%
\special{pa 1350 870}%
\special{pa 620 1600}%
\special{fp}%
\special{pa 1710 1050}%
\special{pa 1160 1600}%
\special{fp}%
\special{pa 1750 1070}%
\special{pa 1220 1600}%
\special{fp}%
\special{pa 1780 1100}%
\special{pa 1280 1600}%
\special{fp}%
\special{pa 1810 1130}%
\special{pa 1340 1600}%
\special{fp}%
\special{pa 1850 1150}%
\special{pa 1400 1600}%
\special{fp}%
\special{pa 1880 1180}%
\special{pa 1460 1600}%
\special{fp}%
\special{pa 1910 1210}%
\special{pa 1520 1600}%
\special{fp}%
\special{pa 1930 1250}%
\special{pa 1580 1600}%
\special{fp}%
\special{pa 1960 1280}%
\special{pa 1640 1600}%
\special{fp}%
\special{pa 1980 1320}%
\special{pa 1700 1600}%
\special{fp}%
\special{pa 2000 1360}%
\special{pa 1760 1600}%
\special{fp}%
\special{pa 2020 1400}%
\special{pa 1820 1600}%
\special{fp}%
\special{pa 2040 1440}%
\special{pa 1880 1600}%
\special{fp}%
\special{pa 2050 1490}%
\special{pa 1940 1600}%
\special{fp}%
\special{pa 2060 1540}%
\special{pa 2000 1600}%
\special{fp}%
% LINE 3 0 3 0 Black White  
% 54 1520 1600 1060 2060 1580 1600 1090 2090 1640 1600 1120 2120 1700 1600 1150 2150 1760 1600 1180 2180 1820 1600 1210 2210 1880 1600 1240 2240 1940 1600 1270 2270 2000 1600 1300 2300 2060 1600 1330 2330 2070 1650 1420 2300 2050 1730 1520 2260 2010 1830 1650 2190 1460 1600 1030 2030 1400 1600 1000 2000 1340 1600 970 1970 1280 1600 940 1940 1220 1600 910 1910 1160 1600 880 1880 1100 1600 850 1850 1040 1600 820 1820 980 1600 790 1790 920 1600 760 1760 860 1600 730 1730 800 1600 700 1700 740 1600 670 1670 680 1600 640 1640
% 
\special{pn 4}%
\special{pa 1520 1600}%
\special{pa 1060 2060}%
\special{fp}%
\special{pa 1580 1600}%
\special{pa 1090 2090}%
\special{fp}%
\special{pa 1640 1600}%
\special{pa 1120 2120}%
\special{fp}%
\special{pa 1700 1600}%
\special{pa 1150 2150}%
\special{fp}%
\special{pa 1760 1600}%
\special{pa 1180 2180}%
\special{fp}%
\special{pa 1820 1600}%
\special{pa 1210 2210}%
\special{fp}%
\special{pa 1880 1600}%
\special{pa 1240 2240}%
\special{fp}%
\special{pa 1940 1600}%
\special{pa 1270 2270}%
\special{fp}%
\special{pa 2000 1600}%
\special{pa 1300 2300}%
\special{fp}%
\special{pa 2060 1600}%
\special{pa 1330 2330}%
\special{fp}%
\special{pa 2070 1650}%
\special{pa 1420 2300}%
\special{fp}%
\special{pa 2050 1730}%
\special{pa 1520 2260}%
\special{fp}%
\special{pa 2010 1830}%
\special{pa 1650 2190}%
\special{fp}%
\special{pa 1460 1600}%
\special{pa 1030 2030}%
\special{fp}%
\special{pa 1400 1600}%
\special{pa 1000 2000}%
\special{fp}%
\special{pa 1340 1600}%
\special{pa 970 1970}%
\special{fp}%
\special{pa 1280 1600}%
\special{pa 940 1940}%
\special{fp}%
\special{pa 1220 1600}%
\special{pa 910 1910}%
\special{fp}%
\special{pa 1160 1600}%
\special{pa 880 1880}%
\special{fp}%
\special{pa 1100 1600}%
\special{pa 850 1850}%
\special{fp}%
\special{pa 1040 1600}%
\special{pa 820 1820}%
\special{fp}%
\special{pa 980 1600}%
\special{pa 790 1790}%
\special{fp}%
\special{pa 920 1600}%
\special{pa 760 1760}%
\special{fp}%
\special{pa 860 1600}%
\special{pa 730 1730}%
\special{fp}%
\special{pa 800 1600}%
\special{pa 700 1700}%
\special{fp}%
\special{pa 740 1600}%
\special{pa 670 1670}%
\special{fp}%
\special{pa 680 1600}%
\special{pa 640 1640}%
\special{fp}%
% STR 2 0 3 0 Black White  
% 4 1610 490 1610 590 5 0 1 0
% $y=x$
\put(16.1000,-5.9000){\makebox(0,0){{\colorbox[named]{White}{$y=x$}}}}%
% STR 2 0 3 0 Black White  
% 4 1630 2520 1630 2620 5 0 1 0
% $y=-x$
\put(16.3000,-26.2000){\makebox(0,0){{\colorbox[named]{White}{$y=-x$}}}}%
% STR 2 0 3 0 Black White  
% 4 2070 1500 2070 1600 2 0 1 0
% $\sqrt(6)$
\put(20.7000,-16.0000){\makebox(0,0)[lb]{{\colorbox[named]{White}{$\sqrt(6)$}}}}%
% STR 2 0 3 0 Black White  
% 4 600 650 600 750 2 0 1 0
% $\sqrt(2)$
\put(6.0000,-7.5000){\makebox(0,0)[lb]{{\colorbox[named]{White}{$\sqrt(2)$}}}}%
\end{picture}}%
}
          \end{center}
     \end{minipage}
     \begin{minipage}{0.5\hsize}
          \begin{center}
          \scalebox{.5}{%WinTpicVersion4.32a
{\unitlength 0.1in%
\begin{picture}(16.2000,28.0000)(3.8000,-32.0000)%
% STR 2 0 3 0 Black White  
% 4 590 1797 590 1810 4 400 0 0
% O
\put(5.9000,-18.1000){\makebox(0,0)[rt]{O}}%
% STR 2 0 3 0 Black White  
% 4 560 387 560 400 4 0 0 0
% $Y$
\put(5.6000,-4.0000){\makebox(0,0)[rt]{$Y$}}%
% STR 2 0 3 0 Black White  
% 4 2000 1827 2000 1840 4 0 0 0
% $X$
\put(20.0000,-18.4000){\makebox(0,0)[rt]{$X$}}%
% VECTOR 2 0 3 0 Black White  
% 2 600 3200 600 400
% 
\special{pn 8}%
\special{pa 600 3200}%
\special{pa 600 400}%
\special{fp}%
\special{sh 1}%
\special{pa 600 400}%
\special{pa 580 467}%
\special{pa 600 453}%
\special{pa 620 467}%
\special{pa 600 400}%
\special{fp}%
% VECTOR 2 0 3 0 Black White  
% 2 400 1800 2000 1800
% 
\special{pn 8}%
\special{pa 400 1800}%
\special{pa 2000 1800}%
\special{fp}%
\special{sh 1}%
\special{pa 2000 1800}%
\special{pa 1933 1780}%
\special{pa 1947 1800}%
\special{pa 1933 1820}%
\special{pa 2000 1800}%
\special{fp}%
% FUNC 2 0 3 0 Black White  
% 9 400 400 2000 3200 600 1800 1800 1800 600 600 400 400 2000 3200 50 4 0 2
% cos(t)///sin(t)///-pi/2///pi/2
\special{pn 8}%
\special{pa 600 3000}%
\special{pa 630 3000}%
\special{pa 638 2999}%
\special{pa 653 2999}%
\special{pa 660 2998}%
\special{pa 675 2998}%
\special{pa 683 2997}%
\special{pa 690 2997}%
\special{pa 698 2996}%
\special{pa 705 2995}%
\special{pa 713 2995}%
\special{pa 721 2994}%
\special{pa 728 2993}%
\special{pa 736 2992}%
\special{pa 743 2991}%
\special{pa 750 2991}%
\special{pa 758 2990}%
\special{pa 765 2989}%
\special{pa 773 2987}%
\special{pa 780 2986}%
\special{pa 788 2985}%
\special{pa 795 2984}%
\special{pa 803 2983}%
\special{pa 810 2981}%
\special{pa 818 2980}%
\special{pa 825 2979}%
\special{pa 832 2977}%
\special{pa 840 2976}%
\special{pa 847 2974}%
\special{pa 854 2973}%
\special{pa 862 2971}%
\special{pa 869 2969}%
\special{pa 877 2968}%
\special{pa 891 2964}%
\special{pa 899 2962}%
\special{pa 920 2956}%
\special{pa 928 2954}%
\special{pa 949 2948}%
\special{pa 957 2946}%
\special{pa 964 2944}%
\special{pa 971 2941}%
\special{pa 978 2939}%
\special{pa 985 2936}%
\special{pa 999 2932}%
\special{pa 1007 2929}%
\special{pa 1014 2926}%
\special{pa 1021 2924}%
\special{pa 1035 2918}%
\special{pa 1042 2916}%
\special{pa 1084 2898}%
\special{pa 1090 2895}%
\special{pa 1118 2883}%
\special{pa 1125 2879}%
\special{pa 1131 2876}%
\special{pa 1138 2873}%
\special{pa 1145 2869}%
\special{pa 1152 2866}%
\special{pa 1158 2862}%
\special{pa 1165 2859}%
\special{pa 1172 2855}%
\special{pa 1178 2852}%
\special{pa 1185 2848}%
\special{pa 1191 2844}%
\special{pa 1198 2840}%
\special{pa 1204 2837}%
\special{pa 1211 2833}%
\special{pa 1217 2829}%
\special{pa 1224 2825}%
\special{pa 1230 2821}%
\special{pa 1237 2817}%
\special{pa 1249 2809}%
\special{pa 1256 2805}%
\special{pa 1268 2797}%
\special{pa 1275 2792}%
\special{pa 1293 2780}%
\special{pa 1299 2775}%
\special{pa 1305 2771}%
\special{pa 1311 2766}%
\special{pa 1318 2762}%
\special{pa 1324 2757}%
\special{pa 1330 2753}%
\special{pa 1336 2748}%
\special{pa 1341 2743}%
\special{pa 1347 2739}%
\special{pa 1359 2729}%
\special{pa 1365 2725}%
\special{pa 1377 2715}%
\special{pa 1382 2710}%
\special{pa 1394 2700}%
\special{pa 1399 2695}%
\special{pa 1405 2690}%
\special{pa 1410 2685}%
\special{pa 1422 2675}%
\special{pa 1427 2670}%
\special{pa 1432 2664}%
\special{pa 1438 2659}%
\special{pa 1459 2638}%
\special{pa 1464 2632}%
\special{pa 1470 2627}%
\special{pa 1475 2621}%
\special{pa 1480 2616}%
\special{pa 1485 2610}%
\special{pa 1490 2605}%
\special{pa 1495 2599}%
\special{pa 1500 2594}%
\special{pa 1515 2576}%
\special{pa 1520 2571}%
\special{pa 1525 2565}%
\special{pa 1529 2559}%
\special{pa 1544 2541}%
\special{pa 1548 2535}%
\special{pa 1553 2529}%
\special{pa 1557 2523}%
\special{pa 1562 2517}%
\special{pa 1566 2511}%
\special{pa 1571 2505}%
\special{pa 1575 2499}%
\special{pa 1580 2493}%
\special{pa 1588 2481}%
\special{pa 1593 2474}%
\special{pa 1605 2456}%
\special{pa 1609 2449}%
\special{pa 1617 2437}%
\special{pa 1621 2430}%
\special{pa 1625 2424}%
\special{pa 1629 2417}%
\special{pa 1633 2411}%
\special{pa 1637 2404}%
\special{pa 1641 2398}%
\special{pa 1644 2391}%
\special{pa 1648 2385}%
\special{pa 1652 2378}%
\special{pa 1655 2371}%
\special{pa 1659 2365}%
\special{pa 1662 2358}%
\special{pa 1666 2351}%
\special{pa 1669 2345}%
\special{pa 1673 2338}%
\special{pa 1679 2324}%
\special{pa 1683 2318}%
\special{pa 1698 2283}%
\special{pa 1701 2277}%
\special{pa 1719 2235}%
\special{pa 1721 2228}%
\special{pa 1727 2214}%
\special{pa 1729 2206}%
\special{pa 1732 2199}%
\special{pa 1734 2192}%
\special{pa 1737 2185}%
\special{pa 1741 2171}%
\special{pa 1744 2164}%
\special{pa 1746 2156}%
\special{pa 1752 2135}%
\special{pa 1754 2127}%
\special{pa 1760 2106}%
\special{pa 1762 2098}%
\special{pa 1766 2084}%
\special{pa 1768 2076}%
\special{pa 1769 2069}%
\special{pa 1771 2062}%
\special{pa 1773 2054}%
\special{pa 1774 2047}%
\special{pa 1776 2040}%
\special{pa 1777 2032}%
\special{pa 1779 2025}%
\special{pa 1780 2017}%
\special{pa 1781 2010}%
\special{pa 1783 2003}%
\special{pa 1784 1995}%
\special{pa 1785 1988}%
\special{pa 1786 1980}%
\special{pa 1788 1973}%
\special{pa 1789 1965}%
\special{pa 1790 1958}%
\special{pa 1791 1950}%
\special{pa 1791 1943}%
\special{pa 1792 1935}%
\special{pa 1793 1928}%
\special{pa 1794 1920}%
\special{pa 1795 1913}%
\special{pa 1795 1905}%
\special{pa 1796 1898}%
\special{pa 1797 1890}%
\special{pa 1797 1883}%
\special{pa 1798 1875}%
\special{pa 1798 1860}%
\special{pa 1799 1853}%
\special{pa 1799 1838}%
\special{pa 1800 1830}%
\special{pa 1800 1770}%
\special{pa 1799 1762}%
\special{pa 1799 1747}%
\special{pa 1798 1740}%
\special{pa 1798 1725}%
\special{pa 1797 1717}%
\special{pa 1797 1710}%
\special{pa 1796 1702}%
\special{pa 1795 1695}%
\special{pa 1795 1687}%
\special{pa 1794 1679}%
\special{pa 1793 1672}%
\special{pa 1792 1664}%
\special{pa 1791 1657}%
\special{pa 1791 1650}%
\special{pa 1790 1642}%
\special{pa 1789 1635}%
\special{pa 1787 1627}%
\special{pa 1786 1620}%
\special{pa 1785 1612}%
\special{pa 1784 1605}%
\special{pa 1783 1597}%
\special{pa 1781 1590}%
\special{pa 1780 1582}%
\special{pa 1779 1575}%
\special{pa 1777 1568}%
\special{pa 1776 1560}%
\special{pa 1774 1553}%
\special{pa 1773 1546}%
\special{pa 1771 1538}%
\special{pa 1769 1531}%
\special{pa 1768 1523}%
\special{pa 1764 1509}%
\special{pa 1762 1501}%
\special{pa 1756 1480}%
\special{pa 1754 1472}%
\special{pa 1748 1451}%
\special{pa 1746 1443}%
\special{pa 1744 1436}%
\special{pa 1741 1429}%
\special{pa 1739 1422}%
\special{pa 1736 1415}%
\special{pa 1732 1401}%
\special{pa 1729 1393}%
\special{pa 1726 1386}%
\special{pa 1724 1379}%
\special{pa 1718 1365}%
\special{pa 1716 1358}%
\special{pa 1698 1316}%
\special{pa 1695 1310}%
\special{pa 1683 1282}%
\special{pa 1679 1275}%
\special{pa 1676 1269}%
\special{pa 1673 1262}%
\special{pa 1669 1255}%
\special{pa 1666 1248}%
\special{pa 1662 1242}%
\special{pa 1659 1235}%
\special{pa 1655 1228}%
\special{pa 1652 1222}%
\special{pa 1648 1215}%
\special{pa 1644 1209}%
\special{pa 1640 1202}%
\special{pa 1637 1196}%
\special{pa 1633 1189}%
\special{pa 1629 1183}%
\special{pa 1625 1176}%
\special{pa 1621 1170}%
\special{pa 1617 1163}%
\special{pa 1609 1151}%
\special{pa 1605 1144}%
\special{pa 1597 1132}%
\special{pa 1592 1125}%
\special{pa 1580 1107}%
\special{pa 1575 1101}%
\special{pa 1571 1095}%
\special{pa 1566 1089}%
\special{pa 1562 1082}%
\special{pa 1557 1076}%
\special{pa 1553 1070}%
\special{pa 1548 1064}%
\special{pa 1543 1059}%
\special{pa 1539 1053}%
\special{pa 1529 1041}%
\special{pa 1525 1035}%
\special{pa 1515 1023}%
\special{pa 1510 1018}%
\special{pa 1500 1006}%
\special{pa 1495 1001}%
\special{pa 1490 995}%
\special{pa 1485 990}%
\special{pa 1475 978}%
\special{pa 1470 973}%
\special{pa 1464 968}%
\special{pa 1459 962}%
\special{pa 1438 941}%
\special{pa 1432 936}%
\special{pa 1427 930}%
\special{pa 1421 925}%
\special{pa 1416 920}%
\special{pa 1410 915}%
\special{pa 1405 910}%
\special{pa 1399 905}%
\special{pa 1394 900}%
\special{pa 1376 885}%
\special{pa 1371 880}%
\special{pa 1365 875}%
\special{pa 1359 871}%
\special{pa 1341 856}%
\special{pa 1335 852}%
\special{pa 1329 847}%
\special{pa 1323 843}%
\special{pa 1317 838}%
\special{pa 1311 834}%
\special{pa 1305 829}%
\special{pa 1299 825}%
\special{pa 1293 820}%
\special{pa 1281 812}%
\special{pa 1274 807}%
\special{pa 1256 795}%
\special{pa 1249 791}%
\special{pa 1237 783}%
\special{pa 1230 779}%
\special{pa 1224 775}%
\special{pa 1217 771}%
\special{pa 1211 767}%
\special{pa 1204 763}%
\special{pa 1198 759}%
\special{pa 1191 756}%
\special{pa 1185 752}%
\special{pa 1178 748}%
\special{pa 1171 745}%
\special{pa 1165 741}%
\special{pa 1158 738}%
\special{pa 1151 734}%
\special{pa 1145 731}%
\special{pa 1138 727}%
\special{pa 1124 721}%
\special{pa 1118 717}%
\special{pa 1083 702}%
\special{pa 1077 699}%
\special{pa 1035 681}%
\special{pa 1028 679}%
\special{pa 1014 673}%
\special{pa 1006 671}%
\special{pa 999 668}%
\special{pa 992 666}%
\special{pa 985 663}%
\special{pa 971 659}%
\special{pa 964 656}%
\special{pa 956 654}%
\special{pa 935 648}%
\special{pa 927 646}%
\special{pa 906 640}%
\special{pa 898 638}%
\special{pa 884 634}%
\special{pa 876 632}%
\special{pa 869 631}%
\special{pa 862 629}%
\special{pa 854 627}%
\special{pa 847 626}%
\special{pa 840 624}%
\special{pa 832 623}%
\special{pa 825 621}%
\special{pa 817 620}%
\special{pa 810 619}%
\special{pa 803 617}%
\special{pa 795 616}%
\special{pa 788 615}%
\special{pa 780 614}%
\special{pa 773 612}%
\special{pa 765 611}%
\special{pa 758 610}%
\special{pa 750 609}%
\special{pa 743 609}%
\special{pa 735 608}%
\special{pa 728 607}%
\special{pa 720 606}%
\special{pa 713 605}%
\special{pa 705 605}%
\special{pa 698 604}%
\special{pa 690 603}%
\special{pa 683 603}%
\special{pa 675 602}%
\special{pa 660 602}%
\special{pa 653 601}%
\special{pa 638 601}%
\special{pa 630 600}%
\special{pa 600 600}%
\special{fp}%
% FUNC 2 0 3 0 Black White  
% 10 400 400 2000 3200 600 1800 1800 1800 600 600 400 400 2000 3200 0 4 0 0 0 0
% sqrt(3)x
\special{pn 8}%
\special{pa 400 2146}%
\special{pa 405 2138}%
\special{pa 415 2120}%
\special{pa 420 2112}%
\special{pa 430 2094}%
\special{pa 435 2086}%
\special{pa 445 2068}%
\special{pa 450 2060}%
\special{pa 460 2042}%
\special{pa 465 2034}%
\special{pa 470 2025}%
\special{pa 475 2017}%
\special{pa 485 1999}%
\special{pa 490 1991}%
\special{pa 500 1973}%
\special{pa 505 1965}%
\special{pa 515 1947}%
\special{pa 520 1939}%
\special{pa 530 1921}%
\special{pa 535 1913}%
\special{pa 545 1895}%
\special{pa 550 1887}%
\special{pa 560 1869}%
\special{pa 565 1861}%
\special{pa 575 1843}%
\special{pa 580 1835}%
\special{pa 590 1817}%
\special{pa 595 1809}%
\special{pa 605 1791}%
\special{pa 610 1783}%
\special{pa 620 1765}%
\special{pa 625 1757}%
\special{pa 635 1739}%
\special{pa 640 1731}%
\special{pa 650 1713}%
\special{pa 655 1705}%
\special{pa 665 1687}%
\special{pa 670 1679}%
\special{pa 680 1661}%
\special{pa 685 1653}%
\special{pa 695 1635}%
\special{pa 700 1627}%
\special{pa 710 1609}%
\special{pa 715 1601}%
\special{pa 725 1583}%
\special{pa 730 1575}%
\special{pa 735 1566}%
\special{pa 740 1558}%
\special{pa 750 1540}%
\special{pa 755 1532}%
\special{pa 765 1514}%
\special{pa 770 1506}%
\special{pa 780 1488}%
\special{pa 785 1480}%
\special{pa 795 1462}%
\special{pa 800 1454}%
\special{pa 810 1436}%
\special{pa 815 1428}%
\special{pa 825 1410}%
\special{pa 830 1402}%
\special{pa 840 1384}%
\special{pa 845 1376}%
\special{pa 855 1358}%
\special{pa 860 1350}%
\special{pa 870 1332}%
\special{pa 875 1324}%
\special{pa 885 1306}%
\special{pa 890 1298}%
\special{pa 900 1280}%
\special{pa 905 1272}%
\special{pa 915 1254}%
\special{pa 920 1246}%
\special{pa 930 1228}%
\special{pa 935 1220}%
\special{pa 945 1202}%
\special{pa 950 1194}%
\special{pa 960 1176}%
\special{pa 965 1168}%
\special{pa 975 1150}%
\special{pa 980 1142}%
\special{pa 985 1133}%
\special{pa 990 1125}%
\special{pa 1000 1107}%
\special{pa 1005 1099}%
\special{pa 1015 1081}%
\special{pa 1020 1073}%
\special{pa 1030 1055}%
\special{pa 1035 1047}%
\special{pa 1045 1029}%
\special{pa 1050 1021}%
\special{pa 1060 1003}%
\special{pa 1065 995}%
\special{pa 1075 977}%
\special{pa 1080 969}%
\special{pa 1090 951}%
\special{pa 1095 943}%
\special{pa 1105 925}%
\special{pa 1110 917}%
\special{pa 1120 899}%
\special{pa 1125 891}%
\special{pa 1135 873}%
\special{pa 1140 865}%
\special{pa 1150 847}%
\special{pa 1155 839}%
\special{pa 1165 821}%
\special{pa 1170 813}%
\special{pa 1180 795}%
\special{pa 1185 787}%
\special{pa 1195 769}%
\special{pa 1200 761}%
\special{pa 1210 743}%
\special{pa 1215 735}%
\special{pa 1225 717}%
\special{pa 1230 709}%
\special{pa 1240 691}%
\special{pa 1245 683}%
\special{pa 1250 674}%
\special{pa 1255 666}%
\special{pa 1265 648}%
\special{pa 1270 640}%
\special{pa 1280 622}%
\special{pa 1285 614}%
\special{pa 1295 596}%
\special{pa 1300 588}%
\special{pa 1310 570}%
\special{pa 1315 562}%
\special{pa 1325 544}%
\special{pa 1330 536}%
\special{pa 1340 518}%
\special{pa 1345 510}%
\special{pa 1355 492}%
\special{pa 1360 484}%
\special{pa 1370 466}%
\special{pa 1375 458}%
\special{pa 1385 440}%
\special{pa 1390 432}%
\special{pa 1400 414}%
\special{pa 1405 406}%
\special{pa 1408 400}%
\special{fp}%
% FUNC 2 0 3 0 Black White  
% 10 400 400 2000 3200 600 1800 1800 1800 600 600 400 400 2000 3200 0 4 0 0 0 0
% -sqrt(3)x
\special{pn 8}%
\special{pa 400 1454}%
\special{pa 405 1462}%
\special{pa 415 1480}%
\special{pa 420 1488}%
\special{pa 430 1506}%
\special{pa 435 1514}%
\special{pa 445 1532}%
\special{pa 450 1540}%
\special{pa 460 1558}%
\special{pa 465 1566}%
\special{pa 470 1575}%
\special{pa 475 1583}%
\special{pa 485 1601}%
\special{pa 490 1609}%
\special{pa 500 1627}%
\special{pa 505 1635}%
\special{pa 515 1653}%
\special{pa 520 1661}%
\special{pa 530 1679}%
\special{pa 535 1687}%
\special{pa 545 1705}%
\special{pa 550 1713}%
\special{pa 560 1731}%
\special{pa 565 1739}%
\special{pa 575 1757}%
\special{pa 580 1765}%
\special{pa 590 1783}%
\special{pa 595 1791}%
\special{pa 605 1809}%
\special{pa 610 1817}%
\special{pa 620 1835}%
\special{pa 625 1843}%
\special{pa 635 1861}%
\special{pa 640 1869}%
\special{pa 650 1887}%
\special{pa 655 1895}%
\special{pa 665 1913}%
\special{pa 670 1921}%
\special{pa 680 1939}%
\special{pa 685 1947}%
\special{pa 695 1965}%
\special{pa 700 1973}%
\special{pa 710 1991}%
\special{pa 715 1999}%
\special{pa 725 2017}%
\special{pa 730 2025}%
\special{pa 735 2034}%
\special{pa 740 2042}%
\special{pa 750 2060}%
\special{pa 755 2068}%
\special{pa 765 2086}%
\special{pa 770 2094}%
\special{pa 780 2112}%
\special{pa 785 2120}%
\special{pa 795 2138}%
\special{pa 800 2146}%
\special{pa 810 2164}%
\special{pa 815 2172}%
\special{pa 825 2190}%
\special{pa 830 2198}%
\special{pa 840 2216}%
\special{pa 845 2224}%
\special{pa 855 2242}%
\special{pa 860 2250}%
\special{pa 870 2268}%
\special{pa 875 2276}%
\special{pa 885 2294}%
\special{pa 890 2302}%
\special{pa 900 2320}%
\special{pa 905 2328}%
\special{pa 915 2346}%
\special{pa 920 2354}%
\special{pa 930 2372}%
\special{pa 935 2380}%
\special{pa 945 2398}%
\special{pa 950 2406}%
\special{pa 960 2424}%
\special{pa 965 2432}%
\special{pa 975 2450}%
\special{pa 980 2458}%
\special{pa 985 2467}%
\special{pa 990 2475}%
\special{pa 1000 2493}%
\special{pa 1005 2501}%
\special{pa 1015 2519}%
\special{pa 1020 2527}%
\special{pa 1030 2545}%
\special{pa 1035 2553}%
\special{pa 1045 2571}%
\special{pa 1050 2579}%
\special{pa 1060 2597}%
\special{pa 1065 2605}%
\special{pa 1075 2623}%
\special{pa 1080 2631}%
\special{pa 1090 2649}%
\special{pa 1095 2657}%
\special{pa 1105 2675}%
\special{pa 1110 2683}%
\special{pa 1120 2701}%
\special{pa 1125 2709}%
\special{pa 1135 2727}%
\special{pa 1140 2735}%
\special{pa 1150 2753}%
\special{pa 1155 2761}%
\special{pa 1165 2779}%
\special{pa 1170 2787}%
\special{pa 1180 2805}%
\special{pa 1185 2813}%
\special{pa 1195 2831}%
\special{pa 1200 2839}%
\special{pa 1210 2857}%
\special{pa 1215 2865}%
\special{pa 1225 2883}%
\special{pa 1230 2891}%
\special{pa 1240 2909}%
\special{pa 1245 2917}%
\special{pa 1250 2926}%
\special{pa 1255 2934}%
\special{pa 1265 2952}%
\special{pa 1270 2960}%
\special{pa 1280 2978}%
\special{pa 1285 2986}%
\special{pa 1295 3004}%
\special{pa 1300 3012}%
\special{pa 1310 3030}%
\special{pa 1315 3038}%
\special{pa 1325 3056}%
\special{pa 1330 3064}%
\special{pa 1340 3082}%
\special{pa 1345 3090}%
\special{pa 1355 3108}%
\special{pa 1360 3116}%
\special{pa 1370 3134}%
\special{pa 1375 3142}%
\special{pa 1385 3160}%
\special{pa 1390 3168}%
\special{pa 1400 3186}%
\special{pa 1405 3194}%
\special{pa 1408 3200}%
\special{fp}%
% LINE 3 0 3 0 Black White  
% 54 1670 1270 1140 1800 1650 1230 1080 1800 1630 1190 1020 1800 1600 1160 960 1800 1580 1120 900 1800 1550 1090 840 1800 1530 1050 780 1800 1500 1020 720 1800 1480 980 660 1800 1450 950 610 1790 1420 920 680 1660 1380 900 760 1520 1350 870 840 1380 1320 840 920 1240 1280 820 1010 1090 1240 800 1090 950 1210 770 1170 810 1690 1310 1200 1800 1710 1350 1260 1800 1730 1390 1320 1800 1740 1440 1380 1800 1750 1490 1440 1800 1770 1530 1500 1800 1780 1580 1560 1800 1790 1630 1620 1800 1800 1680 1680 1800 1800 1740 1740 1800
% 
\special{pn 4}%
\special{pa 1670 1270}%
\special{pa 1140 1800}%
\special{fp}%
\special{pa 1650 1230}%
\special{pa 1080 1800}%
\special{fp}%
\special{pa 1630 1190}%
\special{pa 1020 1800}%
\special{fp}%
\special{pa 1600 1160}%
\special{pa 960 1800}%
\special{fp}%
\special{pa 1580 1120}%
\special{pa 900 1800}%
\special{fp}%
\special{pa 1550 1090}%
\special{pa 840 1800}%
\special{fp}%
\special{pa 1530 1050}%
\special{pa 780 1800}%
\special{fp}%
\special{pa 1500 1020}%
\special{pa 720 1800}%
\special{fp}%
\special{pa 1480 980}%
\special{pa 660 1800}%
\special{fp}%
\special{pa 1450 950}%
\special{pa 610 1790}%
\special{fp}%
\special{pa 1420 920}%
\special{pa 680 1660}%
\special{fp}%
\special{pa 1380 900}%
\special{pa 760 1520}%
\special{fp}%
\special{pa 1350 870}%
\special{pa 840 1380}%
\special{fp}%
\special{pa 1320 840}%
\special{pa 920 1240}%
\special{fp}%
\special{pa 1280 820}%
\special{pa 1010 1090}%
\special{fp}%
\special{pa 1240 800}%
\special{pa 1090 950}%
\special{fp}%
\special{pa 1210 770}%
\special{pa 1170 810}%
\special{fp}%
\special{pa 1690 1310}%
\special{pa 1200 1800}%
\special{fp}%
\special{pa 1710 1350}%
\special{pa 1260 1800}%
\special{fp}%
\special{pa 1730 1390}%
\special{pa 1320 1800}%
\special{fp}%
\special{pa 1740 1440}%
\special{pa 1380 1800}%
\special{fp}%
\special{pa 1750 1490}%
\special{pa 1440 1800}%
\special{fp}%
\special{pa 1770 1530}%
\special{pa 1500 1800}%
\special{fp}%
\special{pa 1780 1580}%
\special{pa 1560 1800}%
\special{fp}%
\special{pa 1790 1630}%
\special{pa 1620 1800}%
\special{fp}%
\special{pa 1800 1680}%
\special{pa 1680 1800}%
\special{fp}%
\special{pa 1800 1740}%
\special{pa 1740 1800}%
\special{fp}%
% LINE 3 0 3 0 Black White  
% 56 1790 1810 1040 2560 1800 1860 1060 2600 1790 1930 1090 2630 1780 2000 1110 2670 1760 2080 1130 2710 1740 2160 1150 2750 1720 2240 1170 2790 1660 2360 1200 2820 1510 2570 1370 2710 1740 1800 1020 2520 1680 1800 1000 2480 1620 1800 980 2440 1560 1800 950 2410 1500 1800 930 2370 1440 1800 910 2330 1380 1800 890 2290 1320 1800 870 2250 1260 1800 840 2220 1200 1800 820 2180 1140 1800 800 2140 1080 1800 780 2100 1020 1800 760 2060 960 1800 730 2030 900 1800 710 1990 840 1800 690 1950 780 1800 670 1910 720 1800 650 1870 660 1800 620 1840
% 
\special{pn 4}%
\special{pa 1790 1810}%
\special{pa 1040 2560}%
\special{fp}%
\special{pa 1800 1860}%
\special{pa 1060 2600}%
\special{fp}%
\special{pa 1790 1930}%
\special{pa 1090 2630}%
\special{fp}%
\special{pa 1780 2000}%
\special{pa 1110 2670}%
\special{fp}%
\special{pa 1760 2080}%
\special{pa 1130 2710}%
\special{fp}%
\special{pa 1740 2160}%
\special{pa 1150 2750}%
\special{fp}%
\special{pa 1720 2240}%
\special{pa 1170 2790}%
\special{fp}%
\special{pa 1660 2360}%
\special{pa 1200 2820}%
\special{fp}%
\special{pa 1510 2570}%
\special{pa 1370 2710}%
\special{fp}%
\special{pa 1740 1800}%
\special{pa 1020 2520}%
\special{fp}%
\special{pa 1680 1800}%
\special{pa 1000 2480}%
\special{fp}%
\special{pa 1620 1800}%
\special{pa 980 2440}%
\special{fp}%
\special{pa 1560 1800}%
\special{pa 950 2410}%
\special{fp}%
\special{pa 1500 1800}%
\special{pa 930 2370}%
\special{fp}%
\special{pa 1440 1800}%
\special{pa 910 2330}%
\special{fp}%
\special{pa 1380 1800}%
\special{pa 890 2290}%
\special{fp}%
\special{pa 1320 1800}%
\special{pa 870 2250}%
\special{fp}%
\special{pa 1260 1800}%
\special{pa 840 2220}%
\special{fp}%
\special{pa 1200 1800}%
\special{pa 820 2180}%
\special{fp}%
\special{pa 1140 1800}%
\special{pa 800 2140}%
\special{fp}%
\special{pa 1080 1800}%
\special{pa 780 2100}%
\special{fp}%
\special{pa 1020 1800}%
\special{pa 760 2060}%
\special{fp}%
\special{pa 960 1800}%
\special{pa 730 2030}%
\special{fp}%
\special{pa 900 1800}%
\special{pa 710 1990}%
\special{fp}%
\special{pa 840 1800}%
\special{pa 690 1950}%
\special{fp}%
\special{pa 780 1800}%
\special{pa 670 1910}%
\special{fp}%
\special{pa 720 1800}%
\special{pa 650 1870}%
\special{fp}%
\special{pa 660 1800}%
\special{pa 620 1840}%
\special{fp}%
% STR 2 0 3 0 Black White  
% 4 960 1550 960 1650 5 0 1 0
% $\pi/3$
\put(9.6000,-16.5000){\makebox(0,0){{\colorbox[named]{White}{$\pi/3$}}}}%
% STR 2 0 3 0 Black White  
% 4 1330 460 1330 560 5 0 1 0
% $Y=\sqrt(3)X$
\put(13.3000,-5.6000){\makebox(0,0){{\colorbox[named]{White}{$Y=\sqrt(3)X$}}}}%
% STR 2 0 3 0 Black White  
% 4 1330 2920 1330 3020 5 0 1 0
% $Y=-\sqrt(3)X$
\put(13.3000,-30.2000){\makebox(0,0){{\colorbox[named]{White}{$Y=-\sqrt(3)X$}}}}%
% STR 2 0 3 0 Black White  
% 4 1800 1700 1800 1800 2 0 1 0
% $1$
\put(18.0000,-18.0000){\makebox(0,0)[lb]{{\colorbox[named]{White}{$1$}}}}%
\end{picture}}%
}
          \end{center}
 \end{minipage} \\
 
ここで$(X,Y)=(x/\sqrt{6},y/\sqrt{2})$なる座標変換をすると,右上図のようになるから,
     \begin{align*}
     \frac{1}{\sqrt{6}}\frac{1}{\sqrt{2}}S=\frac{1}{2}\frac{2\pi}{3} \\
     S=\frac{2\sqrt{3}\pi}{3}
     \end{align*}
である.$\cdots$(答)
           


\newpage
\end{multicols}
\end{document}