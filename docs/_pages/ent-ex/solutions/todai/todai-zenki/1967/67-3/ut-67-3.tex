\documentclass[a4j]{jarticle}
\usepackage{amsmath}
\usepackage{ascmac}
\usepackage{amssymb}
\usepackage{enumerate}
\usepackage{multicol}
\usepackage{framed}
\usepackage{fancyhdr}
\usepackage{latexsym}
\usepackage{indent}
\usepackage{cases}
\allowdisplaybreaks
\pagestyle{fancy}
\lhead{}
\chead{}
\rhead{東京大学前期$1967$年$3$番}
\begin{document}
%分数関係


\def\tfrac#1#2{{\textstyle\frac{#1}{#2}}} %数式中で文中表示の分数を使う時


%Σ関係

\def\dsum#1#2{{\displaystyle\sum_{#1}^{#2}}} %文中で数式表示のΣを使う時


%ベクトル関係


\def\vector#1{\overrightarrow{#1}}  %ベクトルを表現したいとき(aベクトルを表現するときは\ver
\def\norm#1{|\overrightarrow{#1}|} %ベクトルの絶対値
\def\vtwo#1#2{ \left(%
      \begin{array}{c}%
      #1 \\%
      #2 \\%
      \end{array}%
      \right) }                        %2次元ベクトル成分表示
      
      \def\vthree#1#2#3{ \left(
      \begin{array}{c}
      #1 \\
      #2 \\
      #3 \\
      \end{array}
      \right) }                        %3次元ベクトル成分表示



%数列関係


\def\an#1{\verb|{|$#1$\verb|}|}


%極限関係

\def\limit#1#2{\stackrel{#1 \to #2}{\longrightarrow}}   %等式変形からの極限
\def\dlim#1#2{{\displaystyle \lim_{#1\to#2}}} %文中で数式表示の極限を使う



%積分関係

\def\dint#1#2{{\displaystyle \int_{#1}^{#2}}} %文中で数式表示の積分を使う時

\def\ne{\nearrow}
\def\se{\searrow}
\def\nw{\nwarrow}
\def\ne{\nearrow}


%便利なやつ

\def\case#1#2{%
 \[\left\{%
 \begin{array}{l}%
 #1 \\%
 #2%
 \end{array}%
 \right.\] }                           %場合分け
 
\def\1{$\cos\theta=c$,$\sin\theta=s$とおく.}  %cs表示を与える前書きシータ
\def\2{$\cos t=c$,$\sin t=s$とおく.}     %cs表示を与える前書きt
\def\3{$\cos x=c$,$\sin x=s$とおく.}                %cs表示を与える前書きx

\def\fig#1#2#3 {%
\begin{wrapfigure}[#1]{r}{#2 zw}%
\vspace*{-1zh}%
\input{#3}%
\end{wrapfigure} }           %絵の挿入


\def\a{\alpha}   %ギリシャ文字
\def\b{\beta}
\def\g{\gamma}

%問題番号のためのマクロ

\newcounter{nombre} %必須
\renewcommand{\thenombre}{\arabic{nombre}} %任意
\setcounter{nombre}{2} %任意
\newcounter{nombresub}[nombre] %親子関係を定義
\renewcommand{\thenombresub}{\arabic{nombresub}} %任意
\setcounter{nombresub}{0} %任意
\newcommand{\prob}[1][]{\refstepcounter{nombre}#1[問題 \thenombre]}
\newcommand{\probsub}[1][]{\refstepcounter{nombresub}#1(\thenombresub)}


%1-1みたいなカウンタ(todaiとtodaia)
\newcounter{todai}
\setcounter{todai}{0}
\newcounter{todaisub}[todai] 
\setcounter{todaisub}{0} 
\newcommand{\todai}[1][]{\refstepcounter{todai}#1 \thetodai-\thetodaisub}
\newcommand{\todaib}[1][]{\refstepcounter{todai}#1\refstepcounter{todaisub}#1 {\bf [問題 \thetodai.\thetodaisub]}}
\newcommand{\todaia}[1][]{\refstepcounter{todaisub}#1 {\bf [問題 \thetodai.\thetodaisub]}}


     \begin{oframed}
     辺の長さ$2$の正方形$A$が,その中心を円$x^2+y^2=1$の周上におきながら,かつその辺を座標   
     軸に平行に保ち真柄動く.一方,同じ大きさの正方形$B$が固定されていて,辺が座標軸に平行
     でありその中心が点$(1,2)$にある.このとき,二つの正方形$A$,$B$の共通部分の面積の最大値
     を求めよ.
     
     注.正方形の中心とは,その二つの対角線の交点をいう.
     \end{oframed}

\setlength{\columnseprule}{0.4pt}
\begin{multicols}{2}
{\bf[解]}\1 ただし$0\le\theta<2\pi$とする.$A$の中心$(c,s)$とすると,$A$の周及び内部の点は不等式
     \begin{align}
     \left\{
          \begin{array}{l}
          c-1\le x\le c+1 \\
          s-1\le y\le s+1
          \end{array}
     \right.\label{1}
     \end{align}
となる.一方$B$の周及び内部は,不等式
     \begin{align}
     \left\{
          \begin{array}{l}
          0\le x\le2 \\
          1\le y\le3
          \end{array}
     \right.\label{2}
     \end{align}
で表される.\eqref{1},\eqref{2}が共通部分を持つための条件は$s\ge0$つまり$0\le\theta\le\pi$で,このもとで共通部分は
     \begin{align}
     \left\{
          \begin{array}{l}
          0\le x\le c+1\\
          1\le y\le s+1
          \end{array}
     \right.\label{3}
     \end{align}    
で表される長方形である.故にこの面積$S(\theta)$は 
     \begin{align*}
     S(\theta)=s(c+1)
     \end{align*}
である.このとき$S'(\theta)=(2c-1)(c+1)$より下表を得る.
     \begin{align*}
          \begin{array}{|c|c|c|c|c|c|} \hline
          \theta & 0 &     &\pi/3 &      & \pi \\ \hline
          c        & 1 &     &  1/2 &      & -1  \\ \hline      
          S'       &    &+   &0     &  -   & 0   \\ \hline
          S        &    &\ne&       & \se&      \\ \hline
           \end{array}
     \end{align*}
従って$\max S(\theta)=S(\pi/3)=\dfrac{3\sqrt{3}}{4}$である.$\cdots$(答)     
\newpage
\end{multicols}
\end{document}