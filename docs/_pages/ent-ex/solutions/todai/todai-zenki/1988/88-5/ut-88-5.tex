\documentclass[a4j]{jarticle}
\usepackage{amsmath}
\usepackage{ascmac}
\usepackage{amssymb}
\usepackage{enumerate}
\usepackage{multicol}
\usepackage{framed}
\usepackage{fancyhdr}
\usepackage{latexsym}
\usepackage{indent}
\usepackage{cases}
\usepackage[dvips]{graphicx}
\usepackage{color}
\allowdisplaybreaks
\pagestyle{fancy}
\lhead{}
\chead{}
\rhead{東京大学前期$1988$年$5$番}
\begin{document}
%分数関係


\def\tfrac#1#2{{\textstyle\frac{#1}{#2}}} %数式中で文中表示の分数を使う時


%Σ関係

\def\dsum#1#2{{\displaystyle\sum_{#1}^{#2}}} %文中で数式表示のΣを使う時


%ベクトル関係


\def\vector#1{\overrightarrow{#1}}  %ベクトルを表現したいとき(aベクトルを表現するときは\ver
\def\norm#1{|\overrightarrow{#1}|} %ベクトルの絶対値
\def\vtwo#1#2{ \left(%
      \begin{array}{c}%
      #1 \\%
      #2 \\%
      \end{array}%
      \right) }                        %2次元ベクトル成分表示
      
      \def\vthree#1#2#3{ \left(
      \begin{array}{c}
      #1 \\
      #2 \\
      #3 \\
      \end{array}
      \right) }                        %3次元ベクトル成分表示



%数列関係


\def\an#1{\verb|{|$#1$\verb|}|}


%極限関係

\def\limit#1#2{\stackrel{#1 \to #2}{\longrightarrow}}   %等式変形からの極限
\def\dlim#1#2{{\displaystyle \lim_{#1\to#2}}} %文中で数式表示の極限を使う



%積分関係

\def\dint#1#2{{\displaystyle \int_{#1}^{#2}}} %文中で数式表示の積分を使う時

\def\ne{\nearrow}
\def\se{\searrow}
\def\nw{\nwarrow}
\def\ne{\nearrow}


%便利なやつ

\def\case#1#2{%
 \[\left\{%
 \begin{array}{l}%
 #1 \\%
 #2%
 \end{array}%
 \right.\] }                           %場合分け
 
\def\1{$\cos\theta=c$,$\sin\theta=s$とおく.}  %cs表示を与える前書きシータ
\def\2{$\cos t=c$,$\sin t=s$とおく.}     %cs表示を与える前書きt
\def\3{$\cos x=c$,$\sin x=s$とおく.}                %cs表示を与える前書きx

\def\fig#1#2#3 {%
\begin{wrapfigure}[#1]{r}{#2 zw}%
\vspace*{-1zh}%
\input{#3}%
\end{wrapfigure} }           %絵の挿入


\def\a{\alpha}   %ギリシャ文字
\def\b{\beta}
\def\g{\gamma}

%問題番号のためのマクロ

\newcounter{nombre} %必須
\renewcommand{\thenombre}{\arabic{nombre}} %任意
\setcounter{nombre}{2} %任意
\newcounter{nombresub}[nombre] %親子関係を定義
\renewcommand{\thenombresub}{\arabic{nombresub}} %任意
\setcounter{nombresub}{0} %任意
\newcommand{\prob}[1][]{\refstepcounter{nombre}#1[問題 \thenombre]}
\newcommand{\probsub}[1][]{\refstepcounter{nombresub}#1(\thenombresub)}


%1-1みたいなカウンタ(todaiとtodaia)
\newcounter{todai}
\setcounter{todai}{0}
\newcounter{todaisub}[todai] 
\setcounter{todaisub}{0} 
\newcommand{\todai}[1][]{\refstepcounter{todai}#1 \thetodai-\thetodaisub}
\newcommand{\todaib}[1][]{\refstepcounter{todai}#1\refstepcounter{todaisub}#1 {\bf [問題 \thetodai.\thetodaisub]}}
\newcommand{\todaia}[1][]{\refstepcounter{todaisub}#1 {\bf [問題 \thetodai.\thetodaisub]}}


     \begin{oframed}
     $xyz$空間において,$xz$平面上の$0\le z\le 2-x^2$で表される図形を$z$軸の周りに回転して得られる不透明な立体を$V$とする.
     $V$の表面上$z$座標$1$のところにひとつの点光源$P$がある.
     
     $xy$平面上の原点を中心とする円$C$の,$P$からの光が当たっている部分の長さが$2\pi$であるとき,$C$の影の部分の長さを求め
     よ.
     \end{oframed}

\setlength{\columnseprule}{0.4pt}
\begin{multicols}{2}
{\bf[解]}題意の図形は$z$軸に関して対称だから,$V:0\le z\le 2-(x^2+y^2)$である.対称性から$P(1,0,1)$としてよい.$P$での$V$の
接平面は$(2-x^2)'=-2x$ゆえ,
     \[z=-2x+3\]
である.これと$xy$平面の交線は,
     \[x=\frac{3}{2}\]
である.然るに光が当たる部分は$3/2\le x$である.$C$の半径$r_{>0}$とし,下図のように$t$($0<t<\pi/2$)をおく.
     \begin{center}
     \scalebox{0.7}{%WinTpicVersion4.32a
{\unitlength 0.1in%
\begin{picture}(12.2000,18.0000)(3.8000,-22.0000)%
% STR 2 0 3 0 Black White  
% 4 590 1197 590 1210 4 400 0 0
% O
\put(5.9000,-12.1000){\makebox(0,0)[rt]{O}}%
% STR 2 0 3 0 Black White  
% 4 560 387 560 400 4 400 0 0
% $y$
\put(5.6000,-4.0000){\makebox(0,0)[rt]{$y$}}%
% STR 2 0 3 0 Black White  
% 4 1600 1227 1600 1240 4 400 0 0
% $x$
\put(16.0000,-12.4000){\makebox(0,0)[rt]{$x$}}%
% VECTOR 2 0 3 0 Black White  
% 2 600 2200 600 400
% 
\special{pn 8}%
\special{pa 600 2200}%
\special{pa 600 400}%
\special{fp}%
\special{sh 1}%
\special{pa 600 400}%
\special{pa 580 467}%
\special{pa 600 453}%
\special{pa 620 467}%
\special{pa 600 400}%
\special{fp}%
% VECTOR 2 0 3 0 Black White  
% 2 400 1200 1600 1200
% 
\special{pn 8}%
\special{pa 400 1200}%
\special{pa 1600 1200}%
\special{fp}%
\special{sh 1}%
\special{pa 1600 1200}%
\special{pa 1533 1180}%
\special{pa 1547 1200}%
\special{pa 1533 1220}%
\special{pa 1600 1200}%
\special{fp}%
% FUNC 2 0 3 0 Black White  
% 10 400 400 1600 2200 600 1200 800 1200 600 1000 400 400 1600 2200 50 2 0 2 1 0
% 3cos(t)///3sin(t)///0///10
\special{pn 8}%
\special{pa 1200 1200}%
\special{pa 1200 1176}%
\special{pa 1196 1128}%
\special{pa 1190 1092}%
\special{pa 1188 1081}%
\special{pa 1186 1069}%
\special{pa 1183 1057}%
\special{pa 1180 1046}%
\special{pa 1177 1034}%
\special{pa 1173 1023}%
\special{pa 1170 1011}%
\special{pa 1166 1000}%
\special{pa 1161 989}%
\special{pa 1157 977}%
\special{pa 1153 966}%
\special{pa 1143 944}%
\special{pa 1138 934}%
\special{pa 1126 912}%
\special{pa 1121 902}%
\special{pa 1115 891}%
\special{pa 1108 881}%
\special{pa 1102 871}%
\special{pa 1088 851}%
\special{pa 1081 842}%
\special{pa 1074 832}%
\special{pa 1066 823}%
\special{pa 1059 813}%
\special{pa 1043 795}%
\special{pa 1009 761}%
\special{pa 982 737}%
\special{pa 973 730}%
\special{pa 963 723}%
\special{pa 954 715}%
\special{pa 944 708}%
\special{pa 934 702}%
\special{pa 924 695}%
\special{pa 914 689}%
\special{pa 904 682}%
\special{pa 893 677}%
\special{pa 883 671}%
\special{pa 872 665}%
\special{pa 828 645}%
\special{pa 817 641}%
\special{pa 806 636}%
\special{pa 795 632}%
\special{pa 783 629}%
\special{pa 772 625}%
\special{pa 760 622}%
\special{pa 749 619}%
\special{pa 725 613}%
\special{pa 714 611}%
\special{pa 678 605}%
\special{pa 666 604}%
\special{pa 654 602}%
\special{pa 642 601}%
\special{pa 630 601}%
\special{pa 618 600}%
\special{pa 582 600}%
\special{pa 570 601}%
\special{pa 558 601}%
\special{pa 546 602}%
\special{pa 534 604}%
\special{pa 523 605}%
\special{pa 475 613}%
\special{pa 464 616}%
\special{pa 440 622}%
\special{pa 429 625}%
\special{pa 417 628}%
\special{pa 406 632}%
\special{pa 405 632}%
\special{pa 405 633}%
\special{pa 402 633}%
\special{pa 402 634}%
\special{pa 400 634}%
\special{fp}%
\special{pa 400 1766}%
\special{pa 411 1770}%
\special{pa 423 1773}%
\special{pa 434 1777}%
\special{pa 458 1783}%
\special{pa 469 1786}%
\special{pa 505 1792}%
\special{pa 516 1794}%
\special{pa 528 1796}%
\special{pa 576 1800}%
\special{pa 624 1800}%
\special{pa 672 1796}%
\special{pa 708 1790}%
\special{pa 719 1788}%
\special{pa 731 1786}%
\special{pa 743 1783}%
\special{pa 754 1780}%
\special{pa 766 1777}%
\special{pa 777 1773}%
\special{pa 789 1770}%
\special{pa 800 1766}%
\special{pa 811 1761}%
\special{pa 823 1757}%
\special{pa 834 1753}%
\special{pa 856 1743}%
\special{pa 866 1738}%
\special{pa 888 1726}%
\special{pa 898 1721}%
\special{pa 909 1715}%
\special{pa 919 1708}%
\special{pa 929 1702}%
\special{pa 949 1688}%
\special{pa 958 1681}%
\special{pa 968 1674}%
\special{pa 977 1666}%
\special{pa 987 1659}%
\special{pa 1005 1643}%
\special{pa 1022 1626}%
\special{pa 1031 1618}%
\special{pa 1063 1582}%
\special{pa 1070 1573}%
\special{pa 1077 1563}%
\special{pa 1085 1554}%
\special{pa 1092 1544}%
\special{pa 1098 1534}%
\special{pa 1105 1524}%
\special{pa 1111 1514}%
\special{pa 1118 1504}%
\special{pa 1123 1493}%
\special{pa 1129 1483}%
\special{pa 1135 1472}%
\special{pa 1155 1428}%
\special{pa 1159 1417}%
\special{pa 1164 1406}%
\special{pa 1168 1395}%
\special{pa 1171 1383}%
\special{pa 1175 1372}%
\special{pa 1178 1360}%
\special{pa 1181 1349}%
\special{pa 1187 1325}%
\special{pa 1189 1314}%
\special{pa 1195 1278}%
\special{pa 1196 1266}%
\special{pa 1198 1254}%
\special{pa 1199 1242}%
\special{pa 1199 1230}%
\special{pa 1200 1218}%
\special{pa 1200 1182}%
\special{pa 1199 1170}%
\special{pa 1199 1158}%
\special{pa 1198 1146}%
\special{pa 1196 1134}%
\special{pa 1195 1123}%
\special{pa 1187 1075}%
\special{pa 1184 1064}%
\special{pa 1178 1040}%
\special{pa 1175 1029}%
\special{pa 1172 1017}%
\special{pa 1164 995}%
\special{pa 1160 983}%
\special{pa 1150 961}%
\special{pa 1146 950}%
\special{pa 1140 939}%
\special{pa 1135 929}%
\special{pa 1130 918}%
\special{pa 1124 907}%
\special{pa 1112 887}%
\special{pa 1105 877}%
\special{pa 1099 866}%
\special{pa 1092 857}%
\special{pa 1078 837}%
\special{pa 1071 828}%
\special{pa 1063 818}%
\special{pa 1039 791}%
\special{pa 1031 783}%
\special{pa 1023 774}%
\special{pa 1014 766}%
\special{pa 1005 757}%
\special{pa 996 749}%
\special{pa 987 742}%
\special{pa 978 734}%
\special{pa 969 727}%
\special{pa 959 719}%
\special{pa 939 705}%
\special{pa 930 699}%
\special{pa 919 692}%
\special{pa 899 680}%
\special{pa 888 674}%
\special{pa 878 668}%
\special{pa 856 658}%
\special{pa 845 652}%
\special{pa 834 648}%
\special{pa 823 643}%
\special{pa 790 631}%
\special{pa 778 627}%
\special{pa 767 624}%
\special{pa 755 620}%
\special{pa 743 617}%
\special{pa 732 615}%
\special{pa 720 612}%
\special{pa 696 608}%
\special{pa 685 606}%
\special{pa 673 604}%
\special{pa 637 601}%
\special{pa 625 601}%
\special{pa 613 600}%
\special{pa 577 600}%
\special{pa 529 604}%
\special{pa 493 610}%
\special{pa 482 612}%
\special{pa 470 614}%
\special{pa 458 617}%
\special{pa 447 620}%
\special{pa 435 623}%
\special{pa 423 627}%
\special{pa 412 630}%
\special{pa 401 634}%
\special{pa 400 634}%
\special{fp}%
% FUNC 2 0 3 0 Black White  
% 10 400 400 1600 2200 600 1200 800 1200 600 1000 400 400 1600 2200 0 2 0 1 0 0
% 1.5
\special{pn 8}%
\special{pa 900 400}%
\special{pa 900 2200}%
\special{fp}%
% LINE 2 0 3 0 Black White  
% 2 600 1200 900 690
% 
\special{pn 8}%
\special{pa 600 1200}%
\special{pa 900 690}%
\special{fp}%
% STR 2 0 3 0 Black White  
% 4 1300 1000 1300 1100 5 0 1 0
% $r$
\put(13.0000,-11.0000){\makebox(0,0){{\colorbox[named]{White}{$r$}}}}%
% STR 2 0 3 0 Black White  
% 4 1010 1200 1010 1300 5 0 1 0
% $2/3$
\put(10.1000,-13.0000){\makebox(0,0){{\colorbox[named]{White}{$2/3$}}}}%
% STR 2 0 3 0 Black White  
% 4 780 1010 780 1110 5 0 1 0
% $t$
\put(7.8000,-11.1000){\makebox(0,0){{\colorbox[named]{White}{$t$}}}}%
\end{picture}}%
}
     \end{center}
題意の条件から,
     \begin{align}
          \begin{cases}
          2rt=2\pi \\ 
          r\cos t=\dfrac{3}{2}
          \end{cases}
     \end{align}
$r$を消去して
     \[\frac{2\pi}{3}=\frac{t}{\cos t}\]
である.この右辺は$0<t<\pi/2$で単調増加な関数の積ゆえ,単調増加.加えて$t=\pi/3$が解であることから,これが唯一の解である.この時$r=3$となり,求める影の長さは
     \[(2\pi-2t)r=4\pi\]
である.$\cdots$(答)     
  
     
\newpage
\end{multicols}
\end{document}