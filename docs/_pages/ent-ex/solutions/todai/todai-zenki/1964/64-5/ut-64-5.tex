\documentclass[a4j]{jarticle}
\usepackage{amsmath}
\usepackage{ascmac}
\usepackage{amssymb}
\usepackage{enumerate}
\usepackage{multicol}
\usepackage{framed}
\usepackage{fancyhdr}
\usepackage{latexsym}
\usepackage{indent}
\usepackage{cases}
\allowdisplaybreaks
\pagestyle{fancy}
\lhead{}
\chead{}
\rhead{東京大学前期$1964$年$5$番}
\begin{document}
%分数関係


\def\tfrac#1#2{{\textstyle\frac{#1}{#2}}} %数式中で文中表示の分数を使う時


%Σ関係

\def\dsum#1#2{{\displaystyle\sum_{#1}^{#2}}} %文中で数式表示のΣを使う時


%ベクトル関係


\def\vector#1{\overrightarrow{#1}}  %ベクトルを表現したいとき(aベクトルを表現するときは\ver
\def\norm#1{|\overrightarrow{#1}|} %ベクトルの絶対値
\def\vtwo#1#2{ \left(%
      \begin{array}{c}%
      #1 \\%
      #2 \\%
      \end{array}%
      \right) }                        %2次元ベクトル成分表示
      
      \def\vthree#1#2#3{ \left(
      \begin{array}{c}
      #1 \\
      #2 \\
      #3 \\
      \end{array}
      \right) }                        %3次元ベクトル成分表示



%数列関係


\def\an#1{\verb|{|$#1$\verb|}|}


%極限関係

\def\limit#1#2{\stackrel{#1 \to #2}{\longrightarrow}}   %等式変形からの極限
\def\dlim#1#2{{\displaystyle \lim_{#1\to#2}}} %文中で数式表示の極限を使う



%積分関係

\def\dint#1#2{{\displaystyle \int_{#1}^{#2}}} %文中で数式表示の積分を使う時

\def\ne{\nearrow}
\def\se{\searrow}
\def\nw{\nwarrow}
\def\ne{\nearrow}


%便利なやつ

\def\case#1#2{%
 \[\left\{%
 \begin{array}{l}%
 #1 \\%
 #2%
 \end{array}%
 \right.\] }                           %場合分け
 
\def\1{$\cos\theta=c$,$\sin\theta=s$とおく.}  %cs表示を与える前書きシータ
\def\2{$\cos t=c$,$\sin t=s$とおく.}     %cs表示を与える前書きt
\def\3{$\cos x=c$,$\sin x=s$とおく.}                %cs表示を与える前書きx

\def\fig#1#2#3 {%
\begin{wrapfigure}[#1]{r}{#2 zw}%
\vspace*{-1zh}%
\input{#3}%
\end{wrapfigure} }           %絵の挿入


\def\a{\alpha}   %ギリシャ文字
\def\b{\beta}
\def\g{\gamma}

%問題番号のためのマクロ

\newcounter{nombre} %必須
\renewcommand{\thenombre}{\arabic{nombre}} %任意
\setcounter{nombre}{2} %任意
\newcounter{nombresub}[nombre] %親子関係を定義
\renewcommand{\thenombresub}{\arabic{nombresub}} %任意
\setcounter{nombresub}{0} %任意
\newcommand{\prob}[1][]{\refstepcounter{nombre}#1[問題 \thenombre]}
\newcommand{\probsub}[1][]{\refstepcounter{nombresub}#1(\thenombresub)}


%1-1みたいなカウンタ(todaiとtodaia)
\newcounter{todai}
\setcounter{todai}{0}
\newcounter{todaisub}[todai] 
\setcounter{todaisub}{0} 
\newcommand{\todai}[1][]{\refstepcounter{todai}#1 \thetodai-\thetodaisub}
\newcommand{\todaib}[1][]{\refstepcounter{todai}#1\refstepcounter{todaisub}#1 {\bf [問題 \thetodai.\thetodaisub]}}
\newcommand{\todaia}[1][]{\refstepcounter{todaisub}#1 {\bf [問題 \thetodai.\thetodaisub]}}


     \begin{oframed}
     曲線$xy=1$の第$1$象限の部分に定点$P(a,b)$があり,同じ曲線の第$3$象限の部分に動点
     $Q$がある.
          \begin{enumerate}[(1)]
          \item 線分$QP$の長さの最小値を$a$で表せ.
          \item 線分$QP$の長さが最小になるとき,$QP$が$x$軸の正の方向と$30^\circ$の角をなすよ
          うな$a$の値を求めよ.
          \end{enumerate}
     \end{oframed}

\setlength{\columnseprule}{0.4pt}
\begin{multicols}{2}
{\bf[解]}まず,題意から
     \begin{align}
     a>0
     b=\frac{1}{a}
     \end{align}
である.また$c>0$として$\left(-c,\dfrac{-1}{c}\right)$とおける.
     \begin{enumerate}[(1)]
     \item $|QP|^2=(c+a)^2+\left(\dfrac{1}{c}+\dfrac{1}{a}\right)^2$であるからこれを$f(c)$として
          \begin{align*}
          \frac{1}{2}f'(c)&=(c+a)+\left(\dfrac{1}{c}+\dfrac{1}{a}\right)\left(\frac{-1}{c^2}\right)  \\
          &=(a+c)\left(1-\frac{1}{ac^3}\right)  \\
          &=\frac{a+c}{ac^3}{(pc-1)\left(1+pc+(pc)^2\right)}
          \end{align*}
     となる.ただし$p=\sqrt[3]{a}$である.故に下表を得る.
          \begin{align*}
               \begin{array}{|c|c|c|c|} \hline
               c  &   &  1/p   &      \\ \hline
               f'  & - &   0     & +     \\ \hline
               f  &\se&         &\ne  \\ \hline
               \end{array}
          \end{align*}
     従って
          \begin{align*}
          \min f(c)&=f\left(\dfrac{1}{p}\right) \\
          &=\left(\frac{1}{p}+p^3\right)^2+\left(p+\dfrac{1}{p^3}\right)^2  \\
          &=\left(p^2+\frac{1}{p^2}\right)^3
          \end{align*}
     だから,もとめる最小値は($|QP|\ge0$より)
          \begin{align*}
          \min|QP|=(a^{2/3}+a^{-2/3})^{3/2} \tag{答}
          \end{align*}
     となる.      
     
     \item このとき,前問の結果から,
          \begin{align*}
          \vector{QP}&=\vtwo{p^3+1/p}{1/p^3+p}  \\
          &\parallel\vtwo{p^2}{1}
          \end{align*}
     となる.従ってこれが$x$軸の正の方向と$30^\circ$の角を成す,つまり$(\sqrt{3},1)$と平行なとき
          \begin{align*}
          p^2&=\sqrt{3} \\
          p&=\sqrt[4]{3} \\
          a&=3^{3/4}\tag{答}
          \end{align*}  
     となる.     
     \end{enumerate}
\newpage
\end{multicols}
\end{document}