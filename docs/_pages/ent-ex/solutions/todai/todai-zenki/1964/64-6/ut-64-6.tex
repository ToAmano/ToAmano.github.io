\documentclass[a4j]{jarticle}
\usepackage{amsmath}
\usepackage{ascmac}
\usepackage{amssymb}
\usepackage{enumerate}
\usepackage{multicol}
\usepackage{framed}
\usepackage{fancyhdr}
\usepackage{latexsym}
\usepackage{indent}
\usepackage{cases}
\allowdisplaybreaks
\pagestyle{fancy}
\lhead{}
\chead{}
\rhead{東京大学前期$1964$年$6$番}
\begin{document}
%分数関係


\def\tfrac#1#2{{\textstyle\frac{#1}{#2}}} %数式中で文中表示の分数を使う時


%Σ関係

\def\dsum#1#2{{\displaystyle\sum_{#1}^{#2}}} %文中で数式表示のΣを使う時


%ベクトル関係


\def\vector#1{\overrightarrow{#1}}  %ベクトルを表現したいとき(aベクトルを表現するときは\ver
\def\norm#1{|\overrightarrow{#1}|} %ベクトルの絶対値
\def\vtwo#1#2{ \left(%
      \begin{array}{c}%
      #1 \\%
      #2 \\%
      \end{array}%
      \right) }                        %2次元ベクトル成分表示
      
      \def\vthree#1#2#3{ \left(
      \begin{array}{c}
      #1 \\
      #2 \\
      #3 \\
      \end{array}
      \right) }                        %3次元ベクトル成分表示



%数列関係


\def\an#1{\verb|{|$#1$\verb|}|}


%極限関係

\def\limit#1#2{\stackrel{#1 \to #2}{\longrightarrow}}   %等式変形からの極限
\def\dlim#1#2{{\displaystyle \lim_{#1\to#2}}} %文中で数式表示の極限を使う



%積分関係

\def\dint#1#2{{\displaystyle \int_{#1}^{#2}}} %文中で数式表示の積分を使う時

\def\ne{\nearrow}
\def\se{\searrow}
\def\nw{\nwarrow}
\def\ne{\nearrow}


%便利なやつ

\def\case#1#2{%
 \[\left\{%
 \begin{array}{l}%
 #1 \\%
 #2%
 \end{array}%
 \right.\] }                           %場合分け
 
\def\1{$\cos\theta=c$,$\sin\theta=s$とおく.}  %cs表示を与える前書きシータ
\def\2{$\cos t=c$,$\sin t=s$とおく.}     %cs表示を与える前書きt
\def\3{$\cos x=c$,$\sin x=s$とおく.}                %cs表示を与える前書きx

\def\fig#1#2#3 {%
\begin{wrapfigure}[#1]{r}{#2 zw}%
\vspace*{-1zh}%
\input{#3}%
\end{wrapfigure} }           %絵の挿入


\def\a{\alpha}   %ギリシャ文字
\def\b{\beta}
\def\g{\gamma}

%問題番号のためのマクロ

\newcounter{nombre} %必須
\renewcommand{\thenombre}{\arabic{nombre}} %任意
\setcounter{nombre}{2} %任意
\newcounter{nombresub}[nombre] %親子関係を定義
\renewcommand{\thenombresub}{\arabic{nombresub}} %任意
\setcounter{nombresub}{0} %任意
\newcommand{\prob}[1][]{\refstepcounter{nombre}#1[問題 \thenombre]}
\newcommand{\probsub}[1][]{\refstepcounter{nombresub}#1(\thenombresub)}


%1-1みたいなカウンタ(todaiとtodaia)
\newcounter{todai}
\setcounter{todai}{0}
\newcounter{todaisub}[todai] 
\setcounter{todaisub}{0} 
\newcommand{\todai}[1][]{\refstepcounter{todai}#1 \thetodai-\thetodaisub}
\newcommand{\todaib}[1][]{\refstepcounter{todai}#1\refstepcounter{todaisub}#1 {\bf [問題 \thetodai.\thetodaisub]}}
\newcommand{\todaia}[1][]{\refstepcounter{todaisub}#1 {\bf [問題 \thetodai.\thetodaisub]}}


     \begin{oframed}
     関数$f(x)=x^3+ax^2+bx$は次の条件を満たすものとする.
          \begin{enumerate}[(1)]
          \item $f(1)=4$.
          \item $x\ge0$のとき$f(x)\ge0$.
          \end{enumerate}
     このとき$\int_0^1f(x)dx$の値を最大にする$a$,$b$の値,最小にする$a$,$b$の値をそれぞれ
     求めよ.     
     \end{oframed}

\setlength{\columnseprule}{0.4pt}
\begin{multicols}{2}
{\bf[解]}条件(1)から,
     \begin{align}
     a+b=3 \label{1}
     \end{align}
である.次に条件(2)を考える.$x=0$のときはこれは満たされるから,以下$x>0$で考える.このとき,
     \begin{align}
     \forall x>0,f(x)\ge0 \nonumber\\
     \Longleftrightarrow \forall x>0,x^2+ax+b\ge0 \label{2}
     \end{align}     
だから\eqref{2}について考えればよろしい.そこでこの不等式の左辺を$g(x)$,
方程式$g(x)=0$の判別式を$D$とおけば,\eqref{2}は
     \begin{align}
     &D\le0 \lor
     \left\{
          \begin{array}{l}
          D>0  \\
          f(0)\ge0 \\
          \dfrac{-a}{2}< 0
          \end{array}
     \right.     \nonumber\\
     \Longleftrightarrow
     &a^2-4b\ge0 \lor
     \left\{
          \begin{array}{l}
          a^2-4b>0  \\
          b\ge0 \\
          a>0
          \end{array}
     \right.     \nonumber\\
     \Longleftrightarrow
     &-6\le a\le2 \lor
     \left\{
          \begin{array}{l}
          a<-6,2<a  \\
          0<a\le3
          \end{array}
     \right.  \tag{$\because\eqref{1}$} \\
          \Longleftrightarrow
     &-6\le a\le 3 \label{3}
     \end{align}
となる.このもとで$h(a)=\int_0^1f(x)dx$の最大小を考える.
     \begin{align*}
     h(a)&=\left[ \frac{1}{4}x^4+\frac{a}{3}x^3+\frac{b}{2}x^2\right]_0^1 \\
     &=\frac{-1}{6}a+\frac{7}{4}\tag{$\because\eqref{1}$}
     \end{align*}
となり,$h(a)$は$a$の単調減少関数である.故に\eqref{1},\eqref{3}より,$(a,b)=(3,0)$で最小,
$(a,b)=(-6,9)$で最大となる.$\cdots$(答)          
\newpage
\end{multicols}
\end{document}