\documentclass[a4j]{jarticle}
\usepackage{amsmath}
\usepackage{ascmac}
\usepackage{amssymb}
\usepackage{enumerate}
\usepackage{multicol}
\usepackage{framed}
\usepackage{fancyhdr}
\usepackage{latexsym}
\usepackage{indent}
\usepackage{cases}
\allowdisplaybreaks
\pagestyle{fancy}
\lhead{}
\chead{}
\rhead{東京大学前期$1964$年$2$番}
\begin{document}
%分数関係


\def\tfrac#1#2{{\textstyle\frac{#1}{#2}}} %数式中で文中表示の分数を使う時


%Σ関係

\def\dsum#1#2{{\displaystyle\sum_{#1}^{#2}}} %文中で数式表示のΣを使う時


%ベクトル関係


\def\vector#1{\overrightarrow{#1}}  %ベクトルを表現したいとき(aベクトルを表現するときは\ver
\def\norm#1{|\overrightarrow{#1}|} %ベクトルの絶対値
\def\vtwo#1#2{ \left(%
      \begin{array}{c}%
      #1 \\%
      #2 \\%
      \end{array}%
      \right) }                        %2次元ベクトル成分表示
      
      \def\vthree#1#2#3{ \left(
      \begin{array}{c}
      #1 \\
      #2 \\
      #3 \\
      \end{array}
      \right) }                        %3次元ベクトル成分表示



%数列関係


\def\an#1{\verb|{|$#1$\verb|}|}


%極限関係

\def\limit#1#2{\stackrel{#1 \to #2}{\longrightarrow}}   %等式変形からの極限
\def\dlim#1#2{{\displaystyle \lim_{#1\to#2}}} %文中で数式表示の極限を使う



%積分関係

\def\dint#1#2{{\displaystyle \int_{#1}^{#2}}} %文中で数式表示の積分を使う時

\def\ne{\nearrow}
\def\se{\searrow}
\def\nw{\nwarrow}
\def\ne{\nearrow}


%便利なやつ

\def\case#1#2{%
 \[\left\{%
 \begin{array}{l}%
 #1 \\%
 #2%
 \end{array}%
 \right.\] }                           %場合分け
 
\def\1{$\cos\theta=c$,$\sin\theta=s$とおく.}  %cs表示を与える前書きシータ
\def\2{$\cos t=c$,$\sin t=s$とおく.}     %cs表示を与える前書きt
\def\3{$\cos x=c$,$\sin x=s$とおく.}                %cs表示を与える前書きx

\def\fig#1#2#3 {%
\begin{wrapfigure}[#1]{r}{#2 zw}%
\vspace*{-1zh}%
\input{#3}%
\end{wrapfigure} }           %絵の挿入


\def\a{\alpha}   %ギリシャ文字
\def\b{\beta}
\def\g{\gamma}

%問題番号のためのマクロ

\newcounter{nombre} %必須
\renewcommand{\thenombre}{\arabic{nombre}} %任意
\setcounter{nombre}{2} %任意
\newcounter{nombresub}[nombre] %親子関係を定義
\renewcommand{\thenombresub}{\arabic{nombresub}} %任意
\setcounter{nombresub}{0} %任意
\newcommand{\prob}[1][]{\refstepcounter{nombre}#1[問題 \thenombre]}
\newcommand{\probsub}[1][]{\refstepcounter{nombresub}#1(\thenombresub)}


%1-1みたいなカウンタ(todaiとtodaia)
\newcounter{todai}
\setcounter{todai}{0}
\newcounter{todaisub}[todai] 
\setcounter{todaisub}{0} 
\newcommand{\todai}[1][]{\refstepcounter{todai}#1 \thetodai-\thetodaisub}
\newcommand{\todaib}[1][]{\refstepcounter{todai}#1\refstepcounter{todaisub}#1 {\bf [問題 \thetodai.\thetodaisub]}}
\newcommand{\todaia}[1][]{\refstepcounter{todaisub}#1 {\bf [問題 \thetodai.\thetodaisub]}}


     \begin{oframed}
     平面上に$2$つの曲線
          \begin{align}
          y=x^2 \label{1}\\
          y=3x^2+24x+50 \label{2}
          \end{align}
     がある.このとき$1$点$P$をとり,曲線\eqref{1}の上の任意の点$A$に対して,線分$AP$を一定
     の比$m:n(m>0,n>0)$に内分する点$B$が必ず曲線\eqref{2}の上にあるようにしたい.
     点$P(\alpha,\beta)$の座標と比$m:n$の値とを求めよ.     
     \end{oframed}

\setlength{\columnseprule}{0.4pt}
\begin{multicols}{2}
{\bf[解]}$t\in\mathbb{R}$に対して$A(t,t^2)$と置ける. 題意の条件から$s\in\mathbb{R}$に対して
$B(s,3s^2+24s+50)$とおいてよい.このとき内分点に関する条件から
     \begin{numcases}
     {}
     \frac{nt+m\alpha}{m+n}=s \label{3}\\
     \frac{nt^2+m\beta}{m+n}=3s^2+24s+50 \label{4}
     \end{numcases}
$\forall t \exists s,\eqref{3}\land\eqref{4}$となる$P$,$m$,$n$の条件を求めればよい.     
\eqref{3}を\eqref{4}に代入して$s$を消去する.$a=m+n$として簡単のため
     \begin{align*}
     p&=\frac{6n(m\alpha+4a)}{a^2} \\
     q&=\frac{3m^2\alpha^2}{a^2}+\frac{24m\alpha}{a}+50
     \end{align*}
とおけば,     
     \begin{align*}
     &\frac{nt^2+m\beta}{a}\\
     &=3\left(\frac{nt+m\alpha}{a}\right)^2+24 \frac{nt+m\alpha}{a}+50  \\
     \Longleftrightarrow&\frac{n}{a}t^2+\frac{m\beta}{a}=\frac{3n^2}{a^2}t^2+pt+q
      \end{align*}
が恒等式であればよい. したがって
     \begin{align*}
     &\left\{
          \begin{array}{l}
          \dfrac{n}{a}=\dfrac{3n^2}{a^2}  \\
          p=0  \\
          \dfrac{m\beta}{a}=q 
          \end{array}
     \right. \\
     \Longleftrightarrow
     &\left\{
          \begin{array}{l}
          1=\dfrac{3n}{a}  \\
          m\alpha+4a=0  \\
          \dfrac{m\beta}{a}=q 
          \end{array}
     \right. \tag{$\because a,n\not=0$}\\
     \end{align*}
第$1$式および$a=m+n$から,$n:m:a=1:2:3$である.これを第$2$式に代入して$\alpha=-6$を得る.
さらに第$3$式にこれらを代入して
     \begin{align*}
     &\frac{2}{3}\beta=\frac{4}{3}\alpha^2+16\alpha+50  \\
     \therefore &\beta=3
     \end{align*}
である.以上から$P(-6,3)$,$m:n=2:1$である.$\cdots$(答)     
\newpage
\end{multicols}
\end{document}