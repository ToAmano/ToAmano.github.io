\documentclass[a4j]{jarticle}
\usepackage{amsmath}
\usepackage{ascmac}
\usepackage{amssymb}
\usepackage{enumerate}
\usepackage{multicol}
\usepackage{framed}
\usepackage{latexsym}
\usepackage{cases}
\usepackage{indent}
\begin{document}
%分数関係


\def\tfrac#1#2{{\textstyle\frac{#1}{#2}}} %数式中で文中表示の分数を使う時


%Σ関係

\def\dsum#1#2{{\displaystyle\sum_{#1}^{#2}}} %文中で数式表示のΣを使う時


%ベクトル関係


\def\vector#1{\overrightarrow{#1}}  %ベクトルを表現したいとき(aベクトルを表現するときは\ver
\def\norm#1{|\overrightarrow{#1}|} %ベクトルの絶対値
\def\vtwo#1#2{ \left(%
      \begin{array}{c}%
      #1 \\%
      #2 \\%
      \end{array}%
      \right) }                        %2次元ベクトル成分表示
      
      \def\vthree#1#2#3{ \left(
      \begin{array}{c}
      #1 \\
      #2 \\
      #3 \\
      \end{array}
      \right) }                        %3次元ベクトル成分表示



%数列関係


\def\an#1{\verb|{|$#1$\verb|}|}


%極限関係

\def\limit#1#2{\stackrel{#1 \to #2}{\longrightarrow}}   %等式変形からの極限
\def\dlim#1#2{{\displaystyle \lim_{#1\to#2}}} %文中で数式表示の極限を使う



%積分関係

\def\dint#1#2{{\displaystyle \int_{#1}^{#2}}} %文中で数式表示の積分を使う時

\def\ne{\nearrow}
\def\se{\searrow}
\def\nw{\nwarrow}
\def\ne{\nearrow}


%便利なやつ

\def\case#1#2{%
 \[\left\{%
 \begin{array}{l}%
 #1 \\%
 #2%
 \end{array}%
 \right.\] }                           %場合分け
 
\def\1{$\cos\theta=c$,$\sin\theta=s$とおく.}  %cs表示を与える前書きシータ
\def\2{$\cos t=c$,$\sin t=s$とおく.}     %cs表示を与える前書きt
\def\3{$\cos x=c$,$\sin x=s$とおく.}                %cs表示を与える前書きx

\def\fig#1#2#3 {%
\begin{wrapfigure}[#1]{r}{#2 zw}%
\vspace*{-1zh}%
\input{#3}%
\end{wrapfigure} }           %絵の挿入


\def\a{\alpha}   %ギリシャ文字
\def\b{\beta}
\def\g{\gamma}

%問題番号のためのマクロ

\newcounter{nombre} %必須
\renewcommand{\thenombre}{\arabic{nombre}} %任意
\setcounter{nombre}{2} %任意
\newcounter{nombresub}[nombre] %親子関係を定義
\renewcommand{\thenombresub}{\arabic{nombresub}} %任意
\setcounter{nombresub}{0} %任意
\newcommand{\prob}[1][]{\refstepcounter{nombre}#1[問題 \thenombre]}
\newcommand{\probsub}[1][]{\refstepcounter{nombresub}#1(\thenombresub)}


%1-1みたいなカウンタ(todaiとtodaia)
\newcounter{todai}
\setcounter{todai}{0}
\newcounter{todaisub}[todai] 
\setcounter{todaisub}{0} 
\newcommand{\todai}[1][]{\refstepcounter{todai}#1 \thetodai-\thetodaisub}
\newcommand{\todaib}[1][]{\refstepcounter{todai}#1\refstepcounter{todaisub}#1 {\bf [問題 \thetodai.\thetodaisub]}}
\newcommand{\todaia}[1][]{\refstepcounter{todaisub}#1 {\bf [問題 \thetodai.\thetodaisub]}}


\begin{oframed}
座標の定められた空間において,直線$l$は$2$点$(1,1,0)$,$(2,1,1)$を通り,直線$m$は$2$点
$(1,1,1)$,$(1,3,2)$を通る.
     \begin{enumerate}[(1)]
     \item $l$を含み$m$に平行な平面の方程式を$ax+by+cz+d=0$の形に表せ.
     \item 点$(2,0,1)$を通り$l$,$m$の両方と交わる直線を$n$とする.$l$と$n$の交点及び$m$と
     $n$の交点を求めよ.
     \end{enumerate}
\end{oframed}

\setlength{\columnseprule}{0.4pt}
\begin{multicols}{2}
{\bf[解]}$l$,$m$の方向ベクトルは各々
     \begin{align*}
     \vec{l}=\vthree{1}{0}{1} ,
     \vec{m}=\vthree{0}{2}{1}
     \end{align*}
である.また$n$の方向ベクトル$\vec{n}$とする.簡単のため
     \begin{align*}
     \vec{p}=\vthree{1}{1}{0} ,
     \vec{q}=\vthree{1}{1}{1} ,
     \vec{r}=\vthree{2}{0}{1}
     \end{align*}
とおくと,$l$,$m$,$n$の方程式は
     \begin{align*}
     &l:\vector{OX}=\vec{p}+t\vec{l} \\
     &m:\vector{OX}=\vec{q}+t\vec{m} \\
     &n:\vector{OX}=\vec{r}+t\vec{n} 
     \end{align*}
である.
     \begin{enumerate}[(1)]
     \item 題意の平面$\pi$は$(1,1,0)$を通り$\vec{l}$,$\vec{m}$と平行である.この$2$ベクトル
     に直交するベクトルに$(-2,-1,2)$があるから,求める方程式は
          \begin{align*}
          -2(x-1)-1(y-1)+2z=0 \\
          \therefore -2x-y+2z+4=0\cdots\text{(答)}
          \end{align*}
     である.
     \item 題意から$k=1,2$に対し実数$t_k$,$s_k$があって
          \begin{align*}
          \left\{
               \begin{array}{l}
               \vec{r}+t_1\vec{n}=\vec{p}+s_1\vec{l} \\
               \vec{r}+t_2\vec{n}=\vec{q}+s_2\vec{m}
               \end{array}
          \right.
          \end{align*}
     となる.このような実数$t_k$,$s_k$の存在条件を考えればよい.そこで$\vec{n}=(x,y,z)$とす
     る.まず$t_1$,$s_1$について
          \begin{subnumcases}
          { }
                2+xt_1=1+s_1& \label{1a}\\
                yt_1=1&  \label{1b}\\
                1+zt_1=s_1 & \label{1c}
          \end{subnumcases}
     \eqref{1a},\eqref{1c}から$s_1$を消して
          \begin{align*}
          (x-z)t_1=0
          \end{align*}
     である.\eqref{1b}から$t_0\not=0$だから
          \begin{align}
          x=z \label{2}
          \end{align}
     が従う.逆にこの時実数$t_1$,$s_1$は存在.
     
     次に$t_2$,$s_2$について     
         \begin{subnumcases}
         { }
         2+xt_2=1 &\label{2a}\\
         yt_2=1+2s_2 &\label{2b}\\
         1+zt_2=1+s_2 &\label{2c}
          \end{subnumcases}
     である.\eqref{2b},\eqref{2c}から$s_2$を消去して
          \begin{align*}
          t_2(y-2z)=1 \label{4}
          \end{align*}     
     これと,\eqref{2a}において$x\not=0$に注意して$t_2$を消去すれば
          \begin{align}
          2z-y=x
          \end{align}
      逆にこの時$t_2$,$s_2$は存在する.     
      
      \eqref{2}及び\eqref{4}から,$(x,y,z)=(x,x,x)$であるから$\vec{n}=(1,1,1)$としてよい.
      この時\eqref{1b},\eqref{2a}から$(t_1,t_2)=(1,-1)$であるから,求める交点の座標は
           \begin{align*}
           \left\{
                \begin{array}{l}
                \text{$l$と$n$の交点}(3,1,2) \\
                \text{$m$と$n$の交点}(1,-1,0)
                \end{array}
           \right.
           \end{align*}
     である.$\cdots$(答)        
     \end{enumerate} 
       
{\bf[(2)別解]} 
\begin{indentation}{2zw}{0pt}
実数$s$,$t$を用いて,$n$上の$3$点は
     \begin{align*}
     A(1+s,1,s) , B(1,1+2t,1+t) , C(2,0,1)
     \end{align*}
と表せる.ここで$A$は$l$と$n$の,$B$は$m$と$n$の,それぞれ交点である.
$\vector{CA}\parallel\vector{CB}$から$(s,t)=(2,-1)$だから$A(3,1,2)$,$B(1,-1,0)$である.$\cdots$
(答)    
\end{indentation}
\newpage
\end{multicols}
\end{document}