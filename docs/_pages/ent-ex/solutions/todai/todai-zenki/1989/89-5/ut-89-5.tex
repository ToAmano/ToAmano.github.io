\documentclass[a4j]{jarticle}
\usepackage{amsmath}
\usepackage{ascmac}
\usepackage{amssymb}
\usepackage{enumerate}
\usepackage{multicol}
\usepackage{framed}
\usepackage{fancyhdr}
\usepackage{latexsym}
\usepackage{indent}
\usepackage{cases}
\allowdisplaybreaks
\pagestyle{fancy}
\lhead{}
\chead{}
\rhead{東京大学前期$1989$年$5$番}
\begin{document}
%分数関係


\def\tfrac#1#2{{\textstyle\frac{#1}{#2}}} %数式中で文中表示の分数を使う時


%Σ関係

\def\dsum#1#2{{\displaystyle\sum_{#1}^{#2}}} %文中で数式表示のΣを使う時


%ベクトル関係


\def\vector#1{\overrightarrow{#1}}  %ベクトルを表現したいとき(aベクトルを表現するときは\ver
\def\norm#1{|\overrightarrow{#1}|} %ベクトルの絶対値
\def\vtwo#1#2{ \left(%
      \begin{array}{c}%
      #1 \\%
      #2 \\%
      \end{array}%
      \right) }                        %2次元ベクトル成分表示
      
      \def\vthree#1#2#3{ \left(
      \begin{array}{c}
      #1 \\
      #2 \\
      #3 \\
      \end{array}
      \right) }                        %3次元ベクトル成分表示



%数列関係


\def\an#1{\verb|{|$#1$\verb|}|}


%極限関係

\def\limit#1#2{\stackrel{#1 \to #2}{\longrightarrow}}   %等式変形からの極限
\def\dlim#1#2{{\displaystyle \lim_{#1\to#2}}} %文中で数式表示の極限を使う



%積分関係

\def\dint#1#2{{\displaystyle \int_{#1}^{#2}}} %文中で数式表示の積分を使う時

\def\ne{\nearrow}
\def\se{\searrow}
\def\nw{\nwarrow}
\def\ne{\nearrow}


%便利なやつ

\def\case#1#2{%
 \[\left\{%
 \begin{array}{l}%
 #1 \\%
 #2%
 \end{array}%
 \right.\] }                           %場合分け
 
\def\1{$\cos\theta=c$,$\sin\theta=s$とおく.}  %cs表示を与える前書きシータ
\def\2{$\cos t=c$,$\sin t=s$とおく.}     %cs表示を与える前書きt
\def\3{$\cos x=c$,$\sin x=s$とおく.}                %cs表示を与える前書きx

\def\fig#1#2#3 {%
\begin{wrapfigure}[#1]{r}{#2 zw}%
\vspace*{-1zh}%
\input{#3}%
\end{wrapfigure} }           %絵の挿入


\def\a{\alpha}   %ギリシャ文字
\def\b{\beta}
\def\g{\gamma}

%問題番号のためのマクロ

\newcounter{nombre} %必須
\renewcommand{\thenombre}{\arabic{nombre}} %任意
\setcounter{nombre}{2} %任意
\newcounter{nombresub}[nombre] %親子関係を定義
\renewcommand{\thenombresub}{\arabic{nombresub}} %任意
\setcounter{nombresub}{0} %任意
\newcommand{\prob}[1][]{\refstepcounter{nombre}#1[問題 \thenombre]}
\newcommand{\probsub}[1][]{\refstepcounter{nombresub}#1(\thenombresub)}


%1-1みたいなカウンタ(todaiとtodaia)
\newcounter{todai}
\setcounter{todai}{0}
\newcounter{todaisub}[todai] 
\setcounter{todaisub}{0} 
\newcommand{\todai}[1][]{\refstepcounter{todai}#1 \thetodai-\thetodaisub}
\newcommand{\todaib}[1][]{\refstepcounter{todai}#1\refstepcounter{todaisub}#1 {\bf [問題 \thetodai.\thetodaisub]}}
\newcommand{\todaia}[1][]{\refstepcounter{todaisub}#1 {\bf [問題 \thetodai.\thetodaisub]}}


     \begin{oframed}
     $f(x)=\pi x^2\sin\pi x^2$とする.$y=f(x)$のグラフの$0\le x\le1$の部分と$x$軸とで囲まれた
     図形を$y$軸のまわりに回転させてできる立体の体積$V$は$V=2\pi\dint{0}{1}xf(x)dx$で与えられる
     ことを示し,この値を求めよ.
     \end{oframed}

\setlength{\columnseprule}{0.4pt}
\begin{multicols}{2}
{\bf[解]} 区間内で$f(x)\ge0$に注意する.区間内に$X$,$X+\Delta x$をとり,($\Delta x>0$)
$0\le x\le X$の範囲での回転体の体積を$V(X)$と書く.また,同区間での$f(x)$の最大小
値をそれぞれ$M(X)$,$m(X)$とおくと,
     \[0\le m(X)\le f(x)\le M(X) \]
ゆえ,これを$y$軸の周りに回転させれば,
     \begin{align*}
     P&=\left\{(X+\Delta x)^2-X^2\right\}\\
     &=2X\Delta x+(\Delta x)^2 \\
     &=\Delta x(2X+\Delta x) \\
     Q&=2X+\Delta x
     \end{align*}
として
     \begin{align}
     \pi Pm(X)\le V(X+\Delta x)-V(X)\le\pi PM(X) \nonumber\\
     \pi Qm(X)\le \frac{V(X+\Delta x)-V(X)}{\Delta x}\le \pi QM(X) \label{1}
     \end{align}
である.ここで$\Delta x\to0$とすれば
     \begin{align*}
     M(X),m(x)\to f(X) \\
     \frac{V(X+\Delta x)-V(X)}{\Delta x}\to V'(X)
     \end{align*}
だから,\eqref{1}の両辺は$2\pi Xf(X)$に,真ん中は$V'(X)$に,それぞれ収束する.
挟み撃ちの定理から
     \begin{align*}
     V'(X)=2\pi Xf(X)
     \end{align*}
となるので,両辺積分して$V(0)=0$より
     \[V=2\pi\int_0^1xf(x)dx\]
である.$\Box$       

次に値を計算する.$t=\pi x^2$とすれば$\dfrac{dt}{dx}=2\pi x$,$t:0\to\pi$だから
     \begin{align*}
     V&=\int_0^\pi t\sin tdt \\
     &=\left[-t\cos t+\sin t\right]_0^\pi \\
     &=\pi
     \end{align*}
である.$\cdots$(答)
\newpage
\end{multicols}
\end{document}