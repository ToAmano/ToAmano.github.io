\documentclass[a4j]{jarticle}
\usepackage{amsmath}
\usepackage{ascmac}
\usepackage{amssymb}
\usepackage{enumerate}
\usepackage{multicol}
\usepackage{framed}
\usepackage{fancyhdr}
\usepackage{latexsym}
\usepackage{indent}
\usepackage{cases}
\allowdisplaybreaks
\pagestyle{fancy}
\lhead{}
\chead{}
\rhead{東京大学前期$1989$年$4$番}
\begin{document}
%分数関係


\def\tfrac#1#2{{\textstyle\frac{#1}{#2}}} %数式中で文中表示の分数を使う時


%Σ関係

\def\dsum#1#2{{\displaystyle\sum_{#1}^{#2}}} %文中で数式表示のΣを使う時


%ベクトル関係


\def\vector#1{\overrightarrow{#1}}  %ベクトルを表現したいとき(aベクトルを表現するときは\ver
\def\norm#1{|\overrightarrow{#1}|} %ベクトルの絶対値
\def\vtwo#1#2{ \left(%
      \begin{array}{c}%
      #1 \\%
      #2 \\%
      \end{array}%
      \right) }                        %2次元ベクトル成分表示
      
      \def\vthree#1#2#3{ \left(
      \begin{array}{c}
      #1 \\
      #2 \\
      #3 \\
      \end{array}
      \right) }                        %3次元ベクトル成分表示



%数列関係


\def\an#1{\verb|{|$#1$\verb|}|}


%極限関係

\def\limit#1#2{\stackrel{#1 \to #2}{\longrightarrow}}   %等式変形からの極限
\def\dlim#1#2{{\displaystyle \lim_{#1\to#2}}} %文中で数式表示の極限を使う



%積分関係

\def\dint#1#2{{\displaystyle \int_{#1}^{#2}}} %文中で数式表示の積分を使う時

\def\ne{\nearrow}
\def\se{\searrow}
\def\nw{\nwarrow}
\def\ne{\nearrow}


%便利なやつ

\def\case#1#2{%
 \[\left\{%
 \begin{array}{l}%
 #1 \\%
 #2%
 \end{array}%
 \right.\] }                           %場合分け
 
\def\1{$\cos\theta=c$,$\sin\theta=s$とおく.}  %cs表示を与える前書きシータ
\def\2{$\cos t=c$,$\sin t=s$とおく.}     %cs表示を与える前書きt
\def\3{$\cos x=c$,$\sin x=s$とおく.}                %cs表示を与える前書きx

\def\fig#1#2#3 {%
\begin{wrapfigure}[#1]{r}{#2 zw}%
\vspace*{-1zh}%
\input{#3}%
\end{wrapfigure} }           %絵の挿入


\def\a{\alpha}   %ギリシャ文字
\def\b{\beta}
\def\g{\gamma}

%問題番号のためのマクロ

\newcounter{nombre} %必須
\renewcommand{\thenombre}{\arabic{nombre}} %任意
\setcounter{nombre}{2} %任意
\newcounter{nombresub}[nombre] %親子関係を定義
\renewcommand{\thenombresub}{\arabic{nombresub}} %任意
\setcounter{nombresub}{0} %任意
\newcommand{\prob}[1][]{\refstepcounter{nombre}#1[問題 \thenombre]}
\newcommand{\probsub}[1][]{\refstepcounter{nombresub}#1(\thenombresub)}


%1-1みたいなカウンタ(todaiとtodaia)
\newcounter{todai}
\setcounter{todai}{0}
\newcounter{todaisub}[todai] 
\setcounter{todaisub}{0} 
\newcommand{\todai}[1][]{\refstepcounter{todai}#1 \thetodai-\thetodaisub}
\newcommand{\todaib}[1][]{\refstepcounter{todai}#1\refstepcounter{todaisub}#1 {\bf [問題 \thetodai.\thetodaisub]}}
\newcommand{\todaia}[1][]{\refstepcounter{todaisub}#1 {\bf [問題 \thetodai.\thetodaisub]}}


     \begin{oframed}
     $\dfrac{10^{210}}{10^10+3}$の整数部分の桁数と,$1$の位の数字を求めよ.
     ただし,$3^{21}=10460353203$を用いてよい.
     \end{oframed}

\setlength{\columnseprule}{0.4pt}
\begin{multicols}{2}
{\bf[解]} $t=10^{10}+3$とおく.又,与式を$A$とする.まず$A$の桁数について
     \begin{align*}
     \frac{10^{210}}{10^{11}}<A<\frac{10^{210}}{10^{10}} \\
     \therefore \,10^{199}<A<10^{200}
     \end{align*}
だから,$A$は$200$桁である.$\cdots$(答)

次に$1$位の数を求める.合同式の法を$10$とする.
     \begin{align*}
     A&=\frac{(t-3)^{21}}{t}     \\
     &=\sum_{k=1}^{21}{}_{21}C_kt^{k-1}(-3)^{21-k}-\frac{3^21}{t} \\
     &\equiv \sum_{k=1}^{21}{}_{21}C_k3^{k-1}(-3)^{21-k}-\frac{3^21}{t} \tag{$\because t\equiv 3$}\\
     &=\sum_{k=1}^{21}(-1)^{21-k}{}_{21}C_k3^{20}-\frac{3^21}{t} \\
     &=-3^{20}\sum_{k=1}^{21}(-1)^{k}{}_{21}C_k-\frac{3^21}{t}\tag{1}\label{1}
     \end{align*}
ここで,$(1+x)^{21}=\sum_{k=0}^{21}{}_{21}C_kx^k$に$x=-1$を代入して
     \begin{align*}
     0={}_{21}C_0+\sum_{k=1}^{21}{}_{21}C_k(-1)^k
     \end{align*}
だから,\eqref{1}に代入して
     \begin{align*}
     A\equiv {}_{21}C_03^{20}-\frac{3^21}{t}\tag{2}\label{3}
     \end{align*}     
である.$3^{21}=10460353203$を代入して
     \begin{align*}
     {}_{21}C_03^{20}=73^{21}\equiv 1
     \frac{3^21}{t}=1.04\dots
     \end{align*}
だから\eqref{3}に代入して
     \[A\equiv9\]
である.$\cdots$(答)
 \\
 \\
{\bf[別解]}後半部分について考える.
     \begin{align*}
     &C=\dfrac{10^{210}+3^{21}}{t} &D=\dfrac{3^21}{t}
     \end{align*}
とすれば,
     \[A=C-D\]
である.
     \[C=10^{200}+310^{190}+\cdots+3^{19}10^{10}+3^{20}\equiv 3^{20}\]
であり,題意から$3^{20}\equiv 1$,[解]より$D=1.04\dots$だから
     \[A\equiv C-D\equiv 9\]
である.$\cdots$(答)           

     
\newpage
\end{multicols}
\end{document}