\documentclass[a4j]{jarticle}
\usepackage{amsmath}
\usepackage{ascmac}
\usepackage{amssymb}
\usepackage{enumerate}
\usepackage{multicol}
\usepackage{framed}
\usepackage{fancyhdr}
\usepackage{latexsym}
\usepackage{indent}
\usepackage{cases}
\usepackage[dvips]{graphicx}
\usepackage{color}
\allowdisplaybreaks
\pagestyle{fancy}
\lhead{}
\chead{}
\rhead{東京大学前期$1992$年$5$番}
\begin{document}
%分数関係


\def\tfrac#1#2{{\textstyle\frac{#1}{#2}}} %数式中で文中表示の分数を使う時


%Σ関係

\def\dsum#1#2{{\displaystyle\sum_{#1}^{#2}}} %文中で数式表示のΣを使う時


%ベクトル関係


\def\vector#1{\overrightarrow{#1}}  %ベクトルを表現したいとき(aベクトルを表現するときは\ver
\def\norm#1{|\overrightarrow{#1}|} %ベクトルの絶対値
\def\vtwo#1#2{ \left(%
      \begin{array}{c}%
      #1 \\%
      #2 \\%
      \end{array}%
      \right) }                        %2次元ベクトル成分表示
      
      \def\vthree#1#2#3{ \left(
      \begin{array}{c}
      #1 \\
      #2 \\
      #3 \\
      \end{array}
      \right) }                        %3次元ベクトル成分表示



%数列関係


\def\an#1{\verb|{|$#1$\verb|}|}


%極限関係

\def\limit#1#2{\stackrel{#1 \to #2}{\longrightarrow}}   %等式変形からの極限
\def\dlim#1#2{{\displaystyle \lim_{#1\to#2}}} %文中で数式表示の極限を使う



%積分関係

\def\dint#1#2{{\displaystyle \int_{#1}^{#2}}} %文中で数式表示の積分を使う時

\def\ne{\nearrow}
\def\se{\searrow}
\def\nw{\nwarrow}
\def\ne{\nearrow}


%便利なやつ

\def\case#1#2{%
 \[\left\{%
 \begin{array}{l}%
 #1 \\%
 #2%
 \end{array}%
 \right.\] }                           %場合分け
 
\def\1{$\cos\theta=c$,$\sin\theta=s$とおく.}  %cs表示を与える前書きシータ
\def\2{$\cos t=c$,$\sin t=s$とおく.}     %cs表示を与える前書きt
\def\3{$\cos x=c$,$\sin x=s$とおく.}                %cs表示を与える前書きx

\def\fig#1#2#3 {%
\begin{wrapfigure}[#1]{r}{#2 zw}%
\vspace*{-1zh}%
\input{#3}%
\end{wrapfigure} }           %絵の挿入


\def\a{\alpha}   %ギリシャ文字
\def\b{\beta}
\def\g{\gamma}

%問題番号のためのマクロ

\newcounter{nombre} %必須
\renewcommand{\thenombre}{\arabic{nombre}} %任意
\setcounter{nombre}{2} %任意
\newcounter{nombresub}[nombre] %親子関係を定義
\renewcommand{\thenombresub}{\arabic{nombresub}} %任意
\setcounter{nombresub}{0} %任意
\newcommand{\prob}[1][]{\refstepcounter{nombre}#1[問題 \thenombre]}
\newcommand{\probsub}[1][]{\refstepcounter{nombresub}#1(\thenombresub)}


%1-1みたいなカウンタ(todaiとtodaia)
\newcounter{todai}
\setcounter{todai}{0}
\newcounter{todaisub}[todai] 
\setcounter{todaisub}{0} 
\newcommand{\todai}[1][]{\refstepcounter{todai}#1 \thetodai-\thetodaisub}
\newcommand{\todaib}[1][]{\refstepcounter{todai}#1\refstepcounter{todaisub}#1 {\bf [問題 \thetodai.\thetodaisub]}}
\newcommand{\todaia}[1][]{\refstepcounter{todaisub}#1 {\bf [問題 \thetodai.\thetodaisub]}}


     \begin{oframed}
     $xy$平面において,曲線$=x^3/6+1/(2x)$上の点,$(1,2/3)$を出発し,この曲線上を進む点$P$があ
     る.出発してから$t$秒後の$P$の速度$\vec{v}$の大きさは$t/2$に等しく,$\vec{v}$の$x$成分は
     正または$0$であるとする.
          \begin{enumerate}[(1)]
          \item 出発してから$t$秒後の$P$の位置を$(x,y)$として,$x$と$t$の関係式を求めよ.
          \item $\vec{v}$がベクトル$(8,15)$と平行になるのは出発してから何秒後か.
          \end{enumerate}
     \end{oframed}

\setlength{\columnseprule}{0.4pt}
\begin{multicols}{2}
{\bf[解]}
     \begin{enumerate}[(1)]
     \item 題意から,時刻$t$までに$P$は,
          \begin{align}
          \int_0^t\frac{s}{2}ds=\frac{t^2}{4}\label{1}
          \end{align} 
     だけ進む.これが,曲線上で$(1,2/3)$から$(x,y)$までの長さ$L$に等しい.
          \begin{align}
          \frac{dy}{dx}&=\frac{x^2}{2}-\frac{1}{2x^2}=\frac{1}{2}\left(x^2-\frac{1}{x^2}\right) \label{1} \\
          \therefore 1+\frac{dy}{dx}&=1+\frac{1}{4}\left(x^4-2+\frac{1}{x^4}\right) \nonumber\\
          &=\frac{1}{4}\left(x^2+\frac{1}{x^2}\right)\label{2}
          \end{align}
     であるから,
          \begin{align}
          L&=\int_1^x\sqrt{1+(dy/ds)^2}ds \nonumber\\
          &=\frac{1}{2}\int_1^s\left(s^2+\frac{1}{s^2}\right)ds\tag{$\because\eqref{2}$}\\
          &=\frac{1}{2}\left[\frac{s^3}{3}-\frac{1}{s}\right]_1^x \nonumber\\
          &=\frac{1}{2}\left(\frac{x^3}{3}-\frac{1}{x}+\frac{2}{3}\label{3}\right)
          \end{align}
     となる. \eqref{1}と\eqref{3}が等しいので,
          \[t^2=2\left(\frac{x^3}{3}-\frac{1}{x}+\frac{2}{3}\right)\]
     である.$\cdots$(答)
     
     \item $\vec{v}$は$P$での曲線の接線方向のベクトルと平行である.
          \begin{align}
          \frac{dy}{dx}=\frac{1}{2}\left(x^2-\frac{1}{x^2}\right)\label{4}
          \end{align}
     故,$(1,dy/dx)\parallel(8,15)$の時,
          \begin{align*}
          &\frac{dy}{dx}=\frac{15}{8} \\
          &\left(x^2-\frac{1}{x^2}\right)=\frac{15}{4}\tag{$\because\eqref{4}$} \\
          &p^2-\frac{15}{4}p-1=0\tag{$\because p=x^2$} \\
          &(4p+1)(p-4)=0 \\
          &p=4,\frac{-1}{4}
          \end{align*}
     である.題意から$x<1$ゆえ$p=x^2\ge0$であるから,$p=4$つまり$x=2$が従う.
     この時,前問の結果より,
          \[t^2=2\left(\frac{2^3}{3}-\frac{1}{2}+\frac{2}{3}\right)=\frac{17}{3}\]
     $t>0$より,求める $t$は$t=\sqrt{17/3}$である.$\cdots$(答)
     \end{enumerate}
           
\newpage
\end{multicols}
\end{document}