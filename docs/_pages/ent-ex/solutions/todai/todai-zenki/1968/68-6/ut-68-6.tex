\documentclass[a4j]{jarticle}
\usepackage{amsmath}
\usepackage{ascmac}
\usepackage{amssymb}
\usepackage{enumerate}
\usepackage{multicol}
\usepackage{framed}
\usepackage{fancyhdr}
\usepackage{latexsym}
\usepackage{indent}
\usepackage{cases}
\allowdisplaybreaks
\pagestyle{fancy}
\lhead{}
\chead{}
\rhead{東京大学前期$1968$年$6$番}
\begin{document}
%分数関係


\def\tfrac#1#2{{\textstyle\frac{#1}{#2}}} %数式中で文中表示の分数を使う時


%Σ関係

\def\dsum#1#2{{\displaystyle\sum_{#1}^{#2}}} %文中で数式表示のΣを使う時


%ベクトル関係


\def\vector#1{\overrightarrow{#1}}  %ベクトルを表現したいとき(aベクトルを表現するときは\ver
\def\norm#1{|\overrightarrow{#1}|} %ベクトルの絶対値
\def\vtwo#1#2{ \left(%
      \begin{array}{c}%
      #1 \\%
      #2 \\%
      \end{array}%
      \right) }                        %2次元ベクトル成分表示
      
      \def\vthree#1#2#3{ \left(
      \begin{array}{c}
      #1 \\
      #2 \\
      #3 \\
      \end{array}
      \right) }                        %3次元ベクトル成分表示



%数列関係


\def\an#1{\verb|{|$#1$\verb|}|}


%極限関係

\def\limit#1#2{\stackrel{#1 \to #2}{\longrightarrow}}   %等式変形からの極限
\def\dlim#1#2{{\displaystyle \lim_{#1\to#2}}} %文中で数式表示の極限を使う



%積分関係

\def\dint#1#2{{\displaystyle \int_{#1}^{#2}}} %文中で数式表示の積分を使う時

\def\ne{\nearrow}
\def\se{\searrow}
\def\nw{\nwarrow}
\def\ne{\nearrow}


%便利なやつ

\def\case#1#2{%
 \[\left\{%
 \begin{array}{l}%
 #1 \\%
 #2%
 \end{array}%
 \right.\] }                           %場合分け
 
\def\1{$\cos\theta=c$,$\sin\theta=s$とおく.}  %cs表示を与える前書きシータ
\def\2{$\cos t=c$,$\sin t=s$とおく.}     %cs表示を与える前書きt
\def\3{$\cos x=c$,$\sin x=s$とおく.}                %cs表示を与える前書きx

\def\fig#1#2#3 {%
\begin{wrapfigure}[#1]{r}{#2 zw}%
\vspace*{-1zh}%
\input{#3}%
\end{wrapfigure} }           %絵の挿入


\def\a{\alpha}   %ギリシャ文字
\def\b{\beta}
\def\g{\gamma}

%問題番号のためのマクロ

\newcounter{nombre} %必須
\renewcommand{\thenombre}{\arabic{nombre}} %任意
\setcounter{nombre}{2} %任意
\newcounter{nombresub}[nombre] %親子関係を定義
\renewcommand{\thenombresub}{\arabic{nombresub}} %任意
\setcounter{nombresub}{0} %任意
\newcommand{\prob}[1][]{\refstepcounter{nombre}#1[問題 \thenombre]}
\newcommand{\probsub}[1][]{\refstepcounter{nombresub}#1(\thenombresub)}


%1-1みたいなカウンタ(todaiとtodaia)
\newcounter{todai}
\setcounter{todai}{0}
\newcounter{todaisub}[todai] 
\setcounter{todaisub}{0} 
\newcommand{\todai}[1][]{\refstepcounter{todai}#1 \thetodai-\thetodaisub}
\newcommand{\todaib}[1][]{\refstepcounter{todai}#1\refstepcounter{todaisub}#1 {\bf [問題 \thetodai.\thetodaisub]}}
\newcommand{\todaia}[1][]{\refstepcounter{todaisub}#1 {\bf [問題 \thetodai.\thetodaisub]}}


     \begin{oframed}
     次の(i),(ii),(iii)に答えよ.
          \begin{enumerate}[(i)]
          \item $\a$が$0<\a<1$を満たす有理数ならば,区間$0\le x\le1$の上で不等式$1+\dfrac{\a}{2}x\le(1+x)^\a$が成り立つことを示せ.
          \item $2^{200}$の桁数はいくつか.またその最上位の数は何か.その理由を述べよ.
               \begin{itemize}
               \item[注1].例えば$2^{10}=1024$の桁数は$4$,最上位の数は$1$である.なおこの数が$10^3$に近いことに注意せよ.
               \item[注2].$\log_{10}2=0.3010$であるが,この数値を証明に用いてはならない.
               \end{itemize}
          \item $0.300<\log_{10}2<0.302$であることを示せ.
          \end{enumerate}
     \end{oframed}

\setlength{\columnseprule}{0.4pt}
\begin{multicols}{2}
{\bf[解]} 
     \begin{enumerate}[(i)]
     \item $f(x)=(1+x)^\a-\left(1+\dfrac{\a}{2}x\right)$とおく.
          \[f'(x)=\a\left[\left(\frac{1}{1+x}\right)^{1-\a}-\frac{1}{2}\right]\]
     である.
          \begin{align*}
          &0\le x\le1 &0<\a<1
          \end{align*}
     から,$\cfrac{1}{2}\le\cfrac{1}{1+x}\le1$,$0<1-\a<1$であることより,
          \[\frac{1}{2}<\left(\frac{1}{2}\right)^{1-\a}\le\left(\frac{1}{1+x}\right)^{1-\a}\]
     従って$f'(x)>0$だから,$f(x)$は単調増加.これと$f(0)=0$から,区間内で$f(x)\ge0$である.故に示された.$\Box$
     \item $2^{200}=(2^{10})^{20}$だから,$2^{10}>10^3$ゆえ,
          \begin{align}
          2^{200}>10^{60}\label{1}
          \end{align}
      である.
              
     また,(1)で$(x,\a)=(1,1/20)$とすることにより,(これらは条件を満たす.)
          \begin{align}
          &2^{1/20}\ge 1+\frac{1}{40}>1+\frac{24}{1000} \nonumber\\
           \Longleftrightarrow &2^{10}<2^{1/20}10^3 \nonumber\\
          \therefore \ &2^{200}<2×10^{60} \label{2}
          \end{align}
     を得る.\eqref{1},\eqref{2}より,
          \begin{align}
          10^{60}<2^{200}<2×10^{60}\label{3}
          \end{align}
      であるから,$61$桁で,最高位は$1$である.$\cdots$(答)
      \item \eqref{3}の各辺正より,常用対数を取る.以下$\log_{10}2=a$とする.
           \begin{align*}
           &60<200a<60+a  \\
           \Longleftrightarrow &\frac{3}{10}<a<\frac{60}{199} \\
          \therefore \ &0.300<a<0.3015\cdots
           \end{align*}
      であるから,題意は示された.$\Box$   
     \end{enumerate}
\newpage
\end{multicols}
\end{document}