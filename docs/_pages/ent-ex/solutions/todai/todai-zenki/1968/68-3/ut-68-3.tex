\documentclass[a4j]{jarticle}
\usepackage{amsmath}
\usepackage{ascmac}
\usepackage{amssymb}
\usepackage{enumerate}
\usepackage{multicol}
\usepackage{framed}
\usepackage{fancyhdr}
\usepackage{latexsym}
\usepackage{indent}
\usepackage{cases}
\usepackage[dvips]{graphicx}
\allowdisplaybreaks
\pagestyle{fancy}
\lhead{}
\chead{}
\rhead{東京大学前期$1968$年$3$番}
\begin{document}
%分数関係


\def\tfrac#1#2{{\textstyle\frac{#1}{#2}}} %数式中で文中表示の分数を使う時


%Σ関係

\def\dsum#1#2{{\displaystyle\sum_{#1}^{#2}}} %文中で数式表示のΣを使う時


%ベクトル関係


\def\vector#1{\overrightarrow{#1}}  %ベクトルを表現したいとき(aベクトルを表現するときは\ver
\def\norm#1{|\overrightarrow{#1}|} %ベクトルの絶対値
\def\vtwo#1#2{ \left(%
      \begin{array}{c}%
      #1 \\%
      #2 \\%
      \end{array}%
      \right) }                        %2次元ベクトル成分表示
      
      \def\vthree#1#2#3{ \left(
      \begin{array}{c}
      #1 \\
      #2 \\
      #3 \\
      \end{array}
      \right) }                        %3次元ベクトル成分表示



%数列関係


\def\an#1{\verb|{|$#1$\verb|}|}


%極限関係

\def\limit#1#2{\stackrel{#1 \to #2}{\longrightarrow}}   %等式変形からの極限
\def\dlim#1#2{{\displaystyle \lim_{#1\to#2}}} %文中で数式表示の極限を使う



%積分関係

\def\dint#1#2{{\displaystyle \int_{#1}^{#2}}} %文中で数式表示の積分を使う時

\def\ne{\nearrow}
\def\se{\searrow}
\def\nw{\nwarrow}
\def\ne{\nearrow}


%便利なやつ

\def\case#1#2{%
 \[\left\{%
 \begin{array}{l}%
 #1 \\%
 #2%
 \end{array}%
 \right.\] }                           %場合分け
 
\def\1{$\cos\theta=c$,$\sin\theta=s$とおく.}  %cs表示を与える前書きシータ
\def\2{$\cos t=c$,$\sin t=s$とおく.}     %cs表示を与える前書きt
\def\3{$\cos x=c$,$\sin x=s$とおく.}                %cs表示を与える前書きx

\def\fig#1#2#3 {%
\begin{wrapfigure}[#1]{r}{#2 zw}%
\vspace*{-1zh}%
\input{#3}%
\end{wrapfigure} }           %絵の挿入


\def\a{\alpha}   %ギリシャ文字
\def\b{\beta}
\def\g{\gamma}

%問題番号のためのマクロ

\newcounter{nombre} %必須
\renewcommand{\thenombre}{\arabic{nombre}} %任意
\setcounter{nombre}{2} %任意
\newcounter{nombresub}[nombre] %親子関係を定義
\renewcommand{\thenombresub}{\arabic{nombresub}} %任意
\setcounter{nombresub}{0} %任意
\newcommand{\prob}[1][]{\refstepcounter{nombre}#1[問題 \thenombre]}
\newcommand{\probsub}[1][]{\refstepcounter{nombresub}#1(\thenombresub)}


%1-1みたいなカウンタ(todaiとtodaia)
\newcounter{todai}
\setcounter{todai}{0}
\newcounter{todaisub}[todai] 
\setcounter{todaisub}{0} 
\newcommand{\todai}[1][]{\refstepcounter{todai}#1 \thetodai-\thetodaisub}
\newcommand{\todaib}[1][]{\refstepcounter{todai}#1\refstepcounter{todaisub}#1 {\bf [問題 \thetodai.\thetodaisub]}}
\newcommand{\todaia}[1][]{\refstepcounter{todaisub}#1 {\bf [問題 \thetodai.\thetodaisub]}}


     \begin{oframed}
     $\a$,$\b$は与えられた実数とするう.$x$の二次式$f(x)=ax^2+bx+c$の係数$a$,$b$,$c$が,
     $a+b+c=0$なる関係式を満たしながら動く時,座標$(f(\a),f(\b))$を持つ点の全体は,平面上のどんな集合になるか.
     \end{oframed}

\setlength{\columnseprule}{0.4pt}
\begin{multicols}{2}
{\bf[解]} $p=f(\a)$,$q=f(\b)$とし,$A(p,q)$とおけば,$A$の動く範囲を求めればよい.
$c=-a-b$だから,
     \[f(x)=a(x^2-1)+b(x-1)=(x-1)\{a(x+1)+b\}\]
である.従って,$\a$,$\b$が$1$か否かで場合分けする.
     \begin{indentation}{2zw}{0pt}
     \noindent\underline{(1)$\a=\b=1$の時}\\
     $p=q=0$ゆえ$A(0,0)$である.
     \\ \\
     \underline{(2)$\a=1,\b\not=1$の時} \\
     $p=0$,$q$は任意の実数をとるので,$A$は直線$y=0$上を動く.
     \\ \\
     \underline{(3)$\a\not=1,\b=0$の時} \\
     (2)と同様に,$A$は直線$x=0$上を動く.
     \\ \\
     \underline{(4)$\a\not=1,\b\not=1$の時} \\
     $p=(\a-1)\{a(\a+1)+b\}$を$b$についてといて,
          \begin{align*}
          b=\frac{p}{\a-1}-a(\a+1) \tag{$\because \a\not=1$}
          \end{align*}
     である.これを$q=(\b-1)\{a(\b+1)+b\}$に代入して
          \begin{align*}
          q&=(\b-1)\left\{a(\b+1)+\frac{p}{\a-1}-a(\a+1)\right\} \\
          &=\frac{\b-1}{\a-1}p+(\b-1)(\b-\a)a
          \end{align*}
     である.故に$\a=\b$なら$q=p$となり,$\a\not=\b$なら$p\not=q$であるから,
          \begin{align*}
               \begin{cases}
               \a=\b\text{の時}&y=x \\
               \a\not=\b\text{の時}&y\not=x
               \end{cases}
          \end{align*}
     となる.
     \end{indentation} 

 以上をまとめて,
      \begin{align*}
           \begin{cases}
           \a=\b=1\text{の時}&(0,0) \\
           \a=1\land\b\not=1\text{の時}&y=0 \\
            \a\not=1\land\b=1\text{の時}&x=0 \\
            \a\not=1\land\b\not=1\land\a=\b\text{の時}&y=x \\
            \a\not=1\land\b\not=1\land\a\not=\b\text{の時}&y\not=x
            \end{cases}
      \end{align*}     
である.$\cdots$(答) 
\newpage
\end{multicols}
\end{document}