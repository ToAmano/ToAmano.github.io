\documentclass[a4j]{jarticle}
\usepackage{amsmath}
\usepackage{ascmac}
\usepackage{amssymb}
\usepackage{enumerate}
\usepackage{multicol}
\usepackage{framed}
\usepackage{fancyhdr}
\usepackage{latexsym}
\usepackage{indent}
\usepackage{cases}
\usepackage[dvips]{graphicx}
\usepackage{color}
\allowdisplaybreaks
\pagestyle{fancy}
\lhead{}
\chead{}
\rhead{東京大学前期$1981$年$3$番}
\begin{document}
%分数関係


\def\tfrac#1#2{{\textstyle\frac{#1}{#2}}} %数式中で文中表示の分数を使う時


%Σ関係

\def\dsum#1#2{{\displaystyle\sum_{#1}^{#2}}} %文中で数式表示のΣを使う時


%ベクトル関係


\def\vector#1{\overrightarrow{#1}}  %ベクトルを表現したいとき(aベクトルを表現するときは\ver
\def\norm#1{|\overrightarrow{#1}|} %ベクトルの絶対値
\def\vtwo#1#2{ \left(%
      \begin{array}{c}%
      #1 \\%
      #2 \\%
      \end{array}%
      \right) }                        %2次元ベクトル成分表示
      
      \def\vthree#1#2#3{ \left(
      \begin{array}{c}
      #1 \\
      #2 \\
      #3 \\
      \end{array}
      \right) }                        %3次元ベクトル成分表示



%数列関係


\def\an#1{\verb|{|$#1$\verb|}|}


%極限関係

\def\limit#1#2{\stackrel{#1 \to #2}{\longrightarrow}}   %等式変形からの極限
\def\dlim#1#2{{\displaystyle \lim_{#1\to#2}}} %文中で数式表示の極限を使う



%積分関係

\def\dint#1#2{{\displaystyle \int_{#1}^{#2}}} %文中で数式表示の積分を使う時

\def\ne{\nearrow}
\def\se{\searrow}
\def\nw{\nwarrow}
\def\ne{\nearrow}


%便利なやつ

\def\case#1#2{%
 \[\left\{%
 \begin{array}{l}%
 #1 \\%
 #2%
 \end{array}%
 \right.\] }                           %場合分け
 
\def\1{$\cos\theta=c$,$\sin\theta=s$とおく.}  %cs表示を与える前書きシータ
\def\2{$\cos t=c$,$\sin t=s$とおく.}     %cs表示を与える前書きt
\def\3{$\cos x=c$,$\sin x=s$とおく.}                %cs表示を与える前書きx

\def\fig#1#2#3 {%
\begin{wrapfigure}[#1]{r}{#2 zw}%
\vspace*{-1zh}%
\input{#3}%
\end{wrapfigure} }           %絵の挿入


\def\a{\alpha}   %ギリシャ文字
\def\b{\beta}
\def\g{\gamma}

%問題番号のためのマクロ

\newcounter{nombre} %必須
\renewcommand{\thenombre}{\arabic{nombre}} %任意
\setcounter{nombre}{2} %任意
\newcounter{nombresub}[nombre] %親子関係を定義
\renewcommand{\thenombresub}{\arabic{nombresub}} %任意
\setcounter{nombresub}{0} %任意
\newcommand{\prob}[1][]{\refstepcounter{nombre}#1[問題 \thenombre]}
\newcommand{\probsub}[1][]{\refstepcounter{nombresub}#1(\thenombresub)}


%1-1みたいなカウンタ(todaiとtodaia)
\newcounter{todai}
\setcounter{todai}{0}
\newcounter{todaisub}[todai] 
\setcounter{todaisub}{0} 
\newcommand{\todai}[1][]{\refstepcounter{todai}#1 \thetodai-\thetodaisub}
\newcommand{\todaib}[1][]{\refstepcounter{todai}#1\refstepcounter{todaisub}#1 {\bf [問題 \thetodai.\thetodaisub]}}
\newcommand{\todaia}[1][]{\refstepcounter{todaisub}#1 {\bf [問題 \thetodai.\thetodaisub]}}


     \begin{oframed}
     放物線$y=x^2$を$C$で表す.$C$上の点$Q$を通り,$Q$における$C$の接線に垂直な直線
     を,$Q$における$C$の接線という.$0\le t\le1$とし,つぎの$3$条件をみたす点$P$を考える.
          \begin{itemize}
          \item[(イ)] $C$上の点$Q(t,t^2)$における$C$の法線上にある.
          \item[(ロ)] 領域$y\ge x^2$に含まれる.
          \item[(ハ)] $P$と$Q$の距離は$(t-t^2)\sqrt{1+4t^2}$である.
          \end{itemize}
     $t$が$0$から$1$まで嚥下する時,$P$の描く曲線を$C'$とする.このとき,$C$と$C'$とで囲ま
     れた部分の面積を求めよ.
     \end{oframed}

\setlength{\columnseprule}{0.4pt}
\begin{multicols}{2}
{\bf[解]}$(x^2)'=2x$だから,$Q$での法線方向のベクトル$\vec{l}$は,
     \[\vec{l}=\vtwo{-2t}{1}\]
と書ける.故に$P(X,Y)$は
     \[\vtwo{X}{Y}=\vtwo{t}{t^2}+s\vtwo{-2t}{1}\]
と書ける.(ハ)の条件から,
     \begin{align*}
     &\left|s\vtwo{-2t}{1}\right|=(t-t^2)\sqrt{1+4t^2} \\
     \Longleftrightarrow &s=\pm(t-t^2)
     \end{align*}
である.条件(ロ)から複合正をとって,
      \[\vtwo{X}{Y}=\vtwo{t}{t^2}+(t-t^2)\vtwo{-2t}{1}\]
である.従って
     \begin{align*}
     C: 
          &\begin{cases}
          X=2t^3-2t^2+t \\
          Y=t
          \end{cases} \\
     \Longleftrightarrow
     &X=2Y^3-2Y^2+Y&(0\le Y\le 1)
     \end{align*}
である.図示して右上図.
     \begin{center}
     \scalebox{.7}{%WinTpicVersion4.32a
{\unitlength 0.1in%
\begin{picture}(26.2000,28.0000)(3.8000,-32.0000)%
% STR 2 0 3 0 Black White  
% 4 590 2997 590 3010 4 400 0 0
% O
\put(5.9000,-30.1000){\makebox(0,0)[rt]{O}}%
% STR 2 0 3 0 Black White  
% 4 560 387 560 400 4 400 0 0
% $y$
\put(5.6000,-4.0000){\makebox(0,0)[rt]{$y$}}%
% STR 2 0 3 0 Black White  
% 4 3000 3027 3000 3040 4 400 0 0
% $x$
\put(30.0000,-30.4000){\makebox(0,0)[rt]{$x$}}%
% VECTOR 2 0 3 0 Black White  
% 2 600 3200 600 400
% 
\special{pn 8}%
\special{pa 600 3200}%
\special{pa 600 400}%
\special{fp}%
\special{sh 1}%
\special{pa 600 400}%
\special{pa 580 467}%
\special{pa 600 453}%
\special{pa 620 467}%
\special{pa 600 400}%
\special{fp}%
% VECTOR 2 0 3 0 Black White  
% 2 400 3000 3000 3000
% 
\special{pn 8}%
\special{pa 400 3000}%
\special{pa 3000 3000}%
\special{fp}%
\special{sh 1}%
\special{pa 3000 3000}%
\special{pa 2933 2980}%
\special{pa 2947 3000}%
\special{pa 2933 3020}%
\special{pa 3000 3000}%
\special{fp}%
% FUNC 2 0 3 0 Black White  
% 9 400 400 3000 3200 600 3000 2800 3000 600 800 600 400 2800 3200 0 2 0 0
% x^2
\special{pn 8}%
\special{pn 8}%
\special{pa 400 2982}%
\special{pa 410 2984}%
\special{pa 415 2984}%
\special{pa 435 2988}%
\special{pa 440 2988}%
\special{pa 450 2990}%
\special{pa 455 2990}%
\special{pa 465 2992}%
\special{pa 470 2992}%
\special{pa 475 2993}%
\special{pa 480 2993}%
\special{pa 485 2994}%
\special{pa 490 2994}%
\special{pa 495 2995}%
\special{pa 498 2995}%
\special{fp}%
\special{pa 600 3000}%
\special{pa 600 3000}%
\special{fp}%
\special{pa 600 3000}%
\special{pa 630 3000}%
\special{pa 635 2999}%
\special{pa 655 2999}%
\special{pa 660 2998}%
\special{pa 670 2998}%
\special{pa 675 2997}%
\special{pa 685 2997}%
\special{pa 690 2996}%
\special{pa 695 2996}%
\special{pa 700 2995}%
\special{pa 705 2995}%
\special{pa 710 2994}%
\special{pa 715 2994}%
\special{pa 720 2993}%
\special{pa 725 2993}%
\special{pa 730 2992}%
\special{pa 735 2992}%
\special{pa 745 2990}%
\special{pa 750 2990}%
\special{pa 760 2988}%
\special{pa 765 2988}%
\special{pa 785 2984}%
\special{pa 790 2984}%
\special{pa 850 2972}%
\special{pa 855 2970}%
\special{pa 875 2966}%
\special{pa 880 2964}%
\special{pa 890 2962}%
\special{pa 895 2960}%
\special{pa 905 2958}%
\special{pa 910 2956}%
\special{pa 915 2955}%
\special{pa 920 2953}%
\special{pa 925 2952}%
\special{pa 930 2950}%
\special{pa 935 2949}%
\special{pa 940 2947}%
\special{pa 945 2946}%
\special{pa 950 2944}%
\special{pa 955 2943}%
\special{pa 965 2939}%
\special{pa 970 2938}%
\special{pa 980 2934}%
\special{pa 985 2933}%
\special{pa 1005 2925}%
\special{pa 1010 2924}%
\special{pa 1070 2900}%
\special{pa 1075 2897}%
\special{pa 1095 2889}%
\special{pa 1100 2886}%
\special{pa 1110 2882}%
\special{pa 1115 2879}%
\special{pa 1125 2875}%
\special{pa 1130 2872}%
\special{pa 1135 2870}%
\special{pa 1140 2867}%
\special{pa 1145 2865}%
\special{pa 1150 2862}%
\special{pa 1155 2860}%
\special{pa 1160 2857}%
\special{pa 1165 2855}%
\special{pa 1170 2852}%
\special{pa 1175 2850}%
\special{pa 1185 2844}%
\special{pa 1190 2842}%
\special{pa 1200 2836}%
\special{pa 1205 2834}%
\special{pa 1225 2822}%
\special{pa 1230 2820}%
\special{pa 1290 2784}%
\special{pa 1295 2780}%
\special{pa 1315 2768}%
\special{pa 1320 2764}%
\special{pa 1330 2758}%
\special{pa 1335 2754}%
\special{pa 1345 2748}%
\special{pa 1350 2744}%
\special{pa 1355 2741}%
\special{pa 1360 2737}%
\special{pa 1365 2734}%
\special{pa 1370 2730}%
\special{pa 1375 2727}%
\special{pa 1380 2723}%
\special{pa 1385 2720}%
\special{pa 1390 2716}%
\special{pa 1395 2713}%
\special{pa 1405 2705}%
\special{pa 1410 2702}%
\special{pa 1420 2694}%
\special{pa 1425 2691}%
\special{pa 1445 2675}%
\special{pa 1450 2672}%
\special{pa 1510 2624}%
\special{pa 1515 2619}%
\special{pa 1535 2603}%
\special{pa 1540 2598}%
\special{pa 1550 2590}%
\special{pa 1555 2585}%
\special{pa 1565 2577}%
\special{pa 1570 2572}%
\special{pa 1575 2568}%
\special{pa 1580 2563}%
\special{pa 1585 2559}%
\special{pa 1590 2554}%
\special{pa 1595 2550}%
\special{pa 1600 2545}%
\special{pa 1605 2541}%
\special{pa 1610 2536}%
\special{pa 1615 2532}%
\special{pa 1625 2522}%
\special{pa 1630 2518}%
\special{pa 1640 2508}%
\special{pa 1645 2504}%
\special{pa 1665 2484}%
\special{pa 1670 2480}%
\special{pa 1730 2420}%
\special{pa 1735 2414}%
\special{pa 1755 2394}%
\special{pa 1760 2388}%
\special{pa 1770 2378}%
\special{pa 1775 2372}%
\special{pa 1785 2362}%
\special{pa 1790 2356}%
\special{pa 1795 2351}%
\special{pa 1800 2345}%
\special{pa 1805 2340}%
\special{pa 1810 2334}%
\special{pa 1815 2329}%
\special{pa 1820 2323}%
\special{pa 1825 2318}%
\special{pa 1830 2312}%
\special{pa 1835 2307}%
\special{pa 1845 2295}%
\special{pa 1850 2290}%
\special{pa 1860 2278}%
\special{pa 1865 2273}%
\special{pa 1885 2249}%
\special{pa 1890 2244}%
\special{pa 1950 2172}%
\special{pa 1955 2165}%
\special{pa 1975 2141}%
\special{pa 1980 2134}%
\special{pa 1990 2122}%
\special{pa 1995 2115}%
\special{pa 2005 2103}%
\special{pa 2010 2096}%
\special{pa 2015 2090}%
\special{pa 2020 2083}%
\special{pa 2025 2077}%
\special{pa 2030 2070}%
\special{pa 2035 2064}%
\special{pa 2040 2057}%
\special{pa 2045 2051}%
\special{pa 2050 2044}%
\special{pa 2055 2038}%
\special{pa 2065 2024}%
\special{pa 2070 2018}%
\special{pa 2080 2004}%
\special{pa 2085 1998}%
\special{pa 2105 1970}%
\special{pa 2110 1964}%
\special{pa 2170 1880}%
\special{pa 2175 1872}%
\special{pa 2195 1844}%
\special{pa 2200 1836}%
\special{pa 2210 1822}%
\special{pa 2215 1814}%
\special{pa 2225 1800}%
\special{pa 2230 1792}%
\special{pa 2235 1785}%
\special{pa 2240 1777}%
\special{pa 2245 1770}%
\special{pa 2250 1762}%
\special{pa 2255 1755}%
\special{pa 2260 1747}%
\special{pa 2265 1740}%
\special{pa 2270 1732}%
\special{pa 2275 1725}%
\special{pa 2285 1709}%
\special{pa 2290 1702}%
\special{pa 2300 1686}%
\special{pa 2305 1679}%
\special{pa 2325 1647}%
\special{pa 2330 1640}%
\special{pa 2390 1544}%
\special{pa 2395 1535}%
\special{pa 2415 1503}%
\special{pa 2420 1494}%
\special{pa 2430 1478}%
\special{pa 2435 1469}%
\special{pa 2445 1453}%
\special{pa 2450 1444}%
\special{pa 2455 1436}%
\special{pa 2460 1427}%
\special{pa 2470 1411}%
\special{pa 2480 1393}%
\special{pa 2485 1385}%
\special{pa 2490 1376}%
\special{pa 2495 1368}%
\special{pa 2505 1350}%
\special{pa 2510 1342}%
\special{pa 2520 1324}%
\special{pa 2525 1316}%
\special{pa 2545 1280}%
\special{pa 2550 1272}%
\special{pa 2610 1164}%
\special{pa 2615 1154}%
\special{pa 2635 1118}%
\special{pa 2640 1108}%
\special{pa 2650 1090}%
\special{pa 2655 1080}%
\special{pa 2665 1062}%
\special{pa 2670 1052}%
\special{pa 2675 1043}%
\special{pa 2680 1033}%
\special{pa 2690 1015}%
\special{pa 2700 995}%
\special{pa 2705 986}%
\special{pa 2710 976}%
\special{pa 2715 967}%
\special{pa 2725 947}%
\special{pa 2730 938}%
\special{pa 2740 918}%
\special{pa 2745 909}%
\special{pa 2765 869}%
\special{pa 2770 860}%
\special{pa 2800 800}%
\special{fp}%
\special{pn 8}%
\special{pa 2833 734}%
\special{pa 2835 729}%
\special{pa 2855 689}%
\special{pa 2860 678}%
\special{pa 2865 667}%
\special{fp}%
\special{pa 2897 601}%
\special{pa 2900 595}%
\special{pa 2905 585}%
\special{pa 2910 574}%
\special{pa 2915 564}%
\special{pa 2920 553}%
\special{pa 2925 543}%
\special{pa 2929 534}%
\special{fp}%
\special{pa 2960 468}%
\special{pa 2965 458}%
\special{pa 2985 414}%
\special{pa 2992 400}%
\special{fp}%
% FUNC 2 0 3 0 Black White  
% 10 400 400 3000 3200 600 3000 2800 3000 600 800 400 800 3000 3000 0 2 0 1 0 0
% 2y^3-2y^2+y
\special{pn 8}%
\special{pn 8}%
\special{pa 3000 736}%
\special{pa 2987 740}%
\special{pa 2907 765}%
\special{pa 2901 767}%
\special{fp}%
\special{pa 2800 800}%
\special{pa 2800 800}%
\special{fp}%
\special{pa 2800 800}%
\special{pa 2755 815}%
\special{pa 2741 820}%
\special{pa 2726 825}%
\special{pa 2712 830}%
\special{pa 2697 835}%
\special{pa 2669 845}%
\special{pa 2654 850}%
\special{pa 2626 860}%
\special{pa 2613 865}%
\special{pa 2571 880}%
\special{pa 2558 885}%
\special{pa 2544 890}%
\special{pa 2505 905}%
\special{pa 2491 910}%
\special{pa 2465 920}%
\special{pa 2453 925}%
\special{pa 2427 935}%
\special{pa 2415 940}%
\special{pa 2402 945}%
\special{pa 2390 950}%
\special{pa 2377 955}%
\special{pa 2341 970}%
\special{pa 2328 975}%
\special{pa 2316 980}%
\special{pa 2305 985}%
\special{pa 2269 1000}%
\special{pa 2258 1005}%
\special{pa 2246 1010}%
\special{pa 2224 1020}%
\special{pa 2212 1025}%
\special{pa 2146 1055}%
\special{pa 2136 1060}%
\special{pa 2114 1070}%
\special{pa 2104 1075}%
\special{pa 2093 1080}%
\special{pa 2063 1095}%
\special{pa 2052 1100}%
\special{pa 2022 1115}%
\special{pa 2013 1120}%
\special{pa 1993 1130}%
\special{pa 1984 1135}%
\special{pa 1964 1145}%
\special{pa 1946 1155}%
\special{pa 1936 1160}%
\special{pa 1864 1200}%
\special{pa 1856 1205}%
\special{pa 1847 1210}%
\special{pa 1839 1215}%
\special{pa 1830 1220}%
\special{pa 1822 1225}%
\special{pa 1813 1230}%
\special{pa 1781 1250}%
\special{pa 1772 1255}%
\special{pa 1765 1260}%
\special{pa 1733 1280}%
\special{pa 1726 1285}%
\special{pa 1710 1295}%
\special{pa 1703 1300}%
\special{pa 1695 1305}%
\special{pa 1674 1320}%
\special{pa 1666 1325}%
\special{pa 1624 1355}%
\special{pa 1618 1360}%
\special{pa 1604 1370}%
\special{pa 1598 1375}%
\special{pa 1584 1385}%
\special{pa 1572 1395}%
\special{pa 1565 1400}%
\special{pa 1547 1415}%
\special{pa 1540 1420}%
\special{pa 1516 1440}%
\special{pa 1511 1445}%
\special{pa 1493 1460}%
\special{pa 1488 1465}%
\special{pa 1476 1475}%
\special{pa 1471 1480}%
\special{pa 1465 1485}%
\special{pa 1455 1495}%
\special{pa 1449 1500}%
\special{pa 1439 1510}%
\special{pa 1433 1515}%
\special{pa 1398 1550}%
\special{pa 1394 1555}%
\special{pa 1379 1570}%
\special{pa 1375 1575}%
\special{pa 1370 1580}%
\special{pa 1366 1585}%
\special{pa 1361 1590}%
\special{pa 1357 1595}%
\special{pa 1352 1600}%
\special{pa 1348 1605}%
\special{pa 1343 1610}%
\special{pa 1335 1620}%
\special{pa 1330 1625}%
\special{pa 1286 1680}%
\special{pa 1283 1685}%
\special{pa 1271 1700}%
\special{pa 1268 1705}%
\special{pa 1264 1710}%
\special{pa 1261 1715}%
\special{pa 1257 1720}%
\special{pa 1254 1725}%
\special{pa 1250 1730}%
\special{pa 1247 1735}%
\special{pa 1243 1740}%
\special{pa 1237 1750}%
\special{pa 1233 1755}%
\special{pa 1224 1770}%
\special{pa 1220 1775}%
\special{pa 1187 1830}%
\special{pa 1185 1835}%
\special{pa 1173 1855}%
\special{pa 1171 1860}%
\special{pa 1165 1870}%
\special{pa 1163 1875}%
\special{pa 1160 1880}%
\special{pa 1158 1885}%
\special{pa 1155 1890}%
\special{pa 1153 1895}%
\special{pa 1150 1900}%
\special{pa 1148 1905}%
\special{pa 1145 1910}%
\special{pa 1143 1915}%
\special{pa 1140 1920}%
\special{pa 1138 1925}%
\special{pa 1135 1930}%
\special{pa 1131 1940}%
\special{pa 1128 1945}%
\special{pa 1122 1960}%
\special{pa 1119 1965}%
\special{pa 1111 1985}%
\special{pa 1108 1990}%
\special{pa 1070 2085}%
\special{pa 1069 2090}%
\special{pa 1059 2115}%
\special{pa 1058 2120}%
\special{pa 1050 2140}%
\special{pa 1049 2145}%
\special{pa 1043 2160}%
\special{pa 1042 2165}%
\special{pa 1038 2175}%
\special{pa 1037 2180}%
\special{pa 1031 2195}%
\special{pa 1030 2200}%
\special{pa 1026 2210}%
\special{pa 1025 2215}%
\special{pa 1021 2225}%
\special{pa 1020 2230}%
\special{pa 1016 2240}%
\special{pa 1015 2245}%
\special{pa 1011 2255}%
\special{pa 1010 2260}%
\special{pa 1006 2270}%
\special{pa 1005 2275}%
\special{pa 1001 2285}%
\special{pa 1000 2290}%
\special{pa 996 2300}%
\special{pa 995 2305}%
\special{pa 991 2315}%
\special{pa 990 2320}%
\special{pa 984 2335}%
\special{pa 983 2340}%
\special{pa 979 2350}%
\special{pa 978 2355}%
\special{pa 974 2365}%
\special{pa 973 2370}%
\special{pa 967 2385}%
\special{pa 966 2390}%
\special{pa 958 2410}%
\special{pa 957 2415}%
\special{pa 947 2440}%
\special{pa 946 2445}%
\special{pa 904 2550}%
\special{pa 901 2555}%
\special{pa 893 2575}%
\special{pa 890 2580}%
\special{pa 886 2590}%
\special{pa 883 2595}%
\special{pa 879 2605}%
\special{pa 876 2610}%
\special{pa 874 2615}%
\special{pa 871 2620}%
\special{pa 869 2625}%
\special{pa 866 2630}%
\special{pa 864 2635}%
\special{pa 861 2640}%
\special{pa 859 2645}%
\special{pa 856 2650}%
\special{pa 854 2655}%
\special{pa 851 2660}%
\special{pa 849 2665}%
\special{pa 840 2680}%
\special{pa 838 2685}%
\special{pa 826 2705}%
\special{pa 824 2710}%
\special{pa 797 2755}%
\special{pa 793 2760}%
\special{pa 784 2775}%
\special{pa 780 2780}%
\special{pa 774 2790}%
\special{pa 770 2795}%
\special{pa 767 2800}%
\special{pa 763 2805}%
\special{pa 757 2815}%
\special{pa 749 2825}%
\special{pa 746 2830}%
\special{pa 738 2840}%
\special{pa 735 2845}%
\special{pa 719 2865}%
\special{pa 716 2870}%
\special{pa 700 2890}%
\special{pa 695 2895}%
\special{pa 679 2915}%
\special{pa 674 2920}%
\special{pa 666 2930}%
\special{pa 661 2935}%
\special{pa 657 2940}%
\special{pa 652 2945}%
\special{pa 648 2950}%
\special{pa 643 2955}%
\special{pa 639 2960}%
\special{pa 624 2975}%
\special{pa 620 2980}%
\special{pa 600 3000}%
\special{fp}%
\special{pn 8}%
\special{pa 553 3045}%
\special{pa 548 3050}%
\special{pa 542 3055}%
\special{pa 537 3060}%
\special{pa 525 3070}%
\special{pa 520 3075}%
\special{pa 503 3089}%
\special{fp}%
\special{pa 453 3131}%
\special{pa 447 3135}%
\special{pa 435 3145}%
\special{pa 428 3150}%
\special{pa 422 3155}%
\special{pa 408 3165}%
\special{pa 402 3170}%
\special{pa 400 3171}%
\special{fp}%
% LINE 2 2 3 0 Black White  
% 4 600 800 2800 800 2800 800 2800 3000
% 
\special{pn 8}%
\special{pa 600 800}%
\special{pa 2800 800}%
\special{dt 0.045}%
\special{pa 2800 800}%
\special{pa 2800 3000}%
\special{dt 0.045}%
% STR 2 0 3 0 Black White  
% 4 2800 2900 2800 3000 2 0 1 0
% $1$
\put(28.0000,-30.0000){\makebox(0,0)[lb]{{\colorbox[named]{White}{$1$}}}}%
% LINE 3 0 3 0 Black White  
% 84 2800 2360 2160 3000 2800 2300 2100 3000 2800 2240 2040 3000 2800 2180 1980 3000 2800 2120 1920 3000 2800 2060 1860 3000 2800 2000 1800 3000 2800 1940 1740 3000 2800 1880 1680 3000 2800 1820 1620 3000 2800 1760 1560 3000 2800 1700 1500 3000 2800 1640 1440 3000 2800 1580 1380 3000 2800 1520 1320 3000 2800 1460 1260 3000 2800 1400 1200 3000 1550 2590 1140 3000 1310 2770 1080 3000 1170 2850 1020 3000 1060 2900 960 3000 960 2940 900 3000 880 2960 840 3000 2800 1340 1850 2290 2800 1280 2090 1990 2800 1220 2230 1790 2800 1160 2340 1620 2800 1100 2440 1460 2800 1040 2520 1320 2800 980 2600 1180 2800 920 2670 1050 2800 860 2730 930 2800 2420 2220 3000 2800 2480 2280 3000 2800 2540 2340 3000 2800 2600 2400 3000 2800 2660 2460 3000 2800 2720 2520 3000 2800 2780 2580 3000 2800 2840 2640 3000 2800 2900 2700 3000 2800 2960 2760 3000
% 
\special{pn 4}%
\special{pa 2800 2360}%
\special{pa 2160 3000}%
\special{fp}%
\special{pa 2800 2300}%
\special{pa 2100 3000}%
\special{fp}%
\special{pa 2800 2240}%
\special{pa 2040 3000}%
\special{fp}%
\special{pa 2800 2180}%
\special{pa 1980 3000}%
\special{fp}%
\special{pa 2800 2120}%
\special{pa 1920 3000}%
\special{fp}%
\special{pa 2800 2060}%
\special{pa 1860 3000}%
\special{fp}%
\special{pa 2800 2000}%
\special{pa 1800 3000}%
\special{fp}%
\special{pa 2800 1940}%
\special{pa 1740 3000}%
\special{fp}%
\special{pa 2800 1880}%
\special{pa 1680 3000}%
\special{fp}%
\special{pa 2800 1820}%
\special{pa 1620 3000}%
\special{fp}%
\special{pa 2800 1760}%
\special{pa 1560 3000}%
\special{fp}%
\special{pa 2800 1700}%
\special{pa 1500 3000}%
\special{fp}%
\special{pa 2800 1640}%
\special{pa 1440 3000}%
\special{fp}%
\special{pa 2800 1580}%
\special{pa 1380 3000}%
\special{fp}%
\special{pa 2800 1520}%
\special{pa 1320 3000}%
\special{fp}%
\special{pa 2800 1460}%
\special{pa 1260 3000}%
\special{fp}%
\special{pa 2800 1400}%
\special{pa 1200 3000}%
\special{fp}%
\special{pa 1550 2590}%
\special{pa 1140 3000}%
\special{fp}%
\special{pa 1310 2770}%
\special{pa 1080 3000}%
\special{fp}%
\special{pa 1170 2850}%
\special{pa 1020 3000}%
\special{fp}%
\special{pa 1060 2900}%
\special{pa 960 3000}%
\special{fp}%
\special{pa 960 2940}%
\special{pa 900 3000}%
\special{fp}%
\special{pa 880 2960}%
\special{pa 840 3000}%
\special{fp}%
\special{pa 2800 1340}%
\special{pa 1850 2290}%
\special{fp}%
\special{pa 2800 1280}%
\special{pa 2090 1990}%
\special{fp}%
\special{pa 2800 1220}%
\special{pa 2230 1790}%
\special{fp}%
\special{pa 2800 1160}%
\special{pa 2340 1620}%
\special{fp}%
\special{pa 2800 1100}%
\special{pa 2440 1460}%
\special{fp}%
\special{pa 2800 1040}%
\special{pa 2520 1320}%
\special{fp}%
\special{pa 2800 980}%
\special{pa 2600 1180}%
\special{fp}%
\special{pa 2800 920}%
\special{pa 2670 1050}%
\special{fp}%
\special{pa 2800 860}%
\special{pa 2730 930}%
\special{fp}%
\special{pa 2800 2420}%
\special{pa 2220 3000}%
\special{fp}%
\special{pa 2800 2480}%
\special{pa 2280 3000}%
\special{fp}%
\special{pa 2800 2540}%
\special{pa 2340 3000}%
\special{fp}%
\special{pa 2800 2600}%
\special{pa 2400 3000}%
\special{fp}%
\special{pa 2800 2660}%
\special{pa 2460 3000}%
\special{fp}%
\special{pa 2800 2720}%
\special{pa 2520 3000}%
\special{fp}%
\special{pa 2800 2780}%
\special{pa 2580 3000}%
\special{fp}%
\special{pa 2800 2840}%
\special{pa 2640 3000}%
\special{fp}%
\special{pa 2800 2900}%
\special{pa 2700 3000}%
\special{fp}%
\special{pa 2800 2960}%
\special{pa 2760 3000}%
\special{fp}%
% LINE 3 0 3 0 Black White  
% 92 1840 800 600 2040 1900 800 600 2100 1960 800 600 2160 2020 800 600 2220 2080 800 600 2280 1370 1570 600 2340 1190 1810 600 2400 1130 1930 600 2460 1080 2040 600 2520 1050 2130 600 2580 1020 2220 600 2640 990 2310 600 2700 950 2410 600 2760 920 2500 600 2820 870 2610 600 2880 800 2740 600 2940 2140 800 1450 1490 2200 800 1720 1280 2260 800 1870 1190 2320 800 2000 1120 2380 800 2120 1060 2440 800 2230 1010 2500 800 2330 970 2560 800 2430 930 2620 800 2530 890 2680 800 2620 860 1780 800 600 1980 1720 800 600 1920 1660 800 600 1860 1600 800 600 1800 1540 800 600 1740 1480 800 600 1680 1420 800 600 1620 1360 800 600 1560 1300 800 600 1500 1240 800 600 1440 1180 800 600 1380 1120 800 600 1320 1060 800 600 1260 1000 800 600 1200 940 800 600 1140 880 800 600 1080 820 800 600 1020 760 800 600 960 700 800 600 900 640 800 600 840
% 
\special{pn 4}%
\special{pa 1840 800}%
\special{pa 600 2040}%
\special{fp}%
\special{pa 1900 800}%
\special{pa 600 2100}%
\special{fp}%
\special{pa 1960 800}%
\special{pa 600 2160}%
\special{fp}%
\special{pa 2020 800}%
\special{pa 600 2220}%
\special{fp}%
\special{pa 2080 800}%
\special{pa 600 2280}%
\special{fp}%
\special{pa 1370 1570}%
\special{pa 600 2340}%
\special{fp}%
\special{pa 1190 1810}%
\special{pa 600 2400}%
\special{fp}%
\special{pa 1130 1930}%
\special{pa 600 2460}%
\special{fp}%
\special{pa 1080 2040}%
\special{pa 600 2520}%
\special{fp}%
\special{pa 1050 2130}%
\special{pa 600 2580}%
\special{fp}%
\special{pa 1020 2220}%
\special{pa 600 2640}%
\special{fp}%
\special{pa 990 2310}%
\special{pa 600 2700}%
\special{fp}%
\special{pa 950 2410}%
\special{pa 600 2760}%
\special{fp}%
\special{pa 920 2500}%
\special{pa 600 2820}%
\special{fp}%
\special{pa 870 2610}%
\special{pa 600 2880}%
\special{fp}%
\special{pa 800 2740}%
\special{pa 600 2940}%
\special{fp}%
\special{pa 2140 800}%
\special{pa 1450 1490}%
\special{fp}%
\special{pa 2200 800}%
\special{pa 1720 1280}%
\special{fp}%
\special{pa 2260 800}%
\special{pa 1870 1190}%
\special{fp}%
\special{pa 2320 800}%
\special{pa 2000 1120}%
\special{fp}%
\special{pa 2380 800}%
\special{pa 2120 1060}%
\special{fp}%
\special{pa 2440 800}%
\special{pa 2230 1010}%
\special{fp}%
\special{pa 2500 800}%
\special{pa 2330 970}%
\special{fp}%
\special{pa 2560 800}%
\special{pa 2430 930}%
\special{fp}%
\special{pa 2620 800}%
\special{pa 2530 890}%
\special{fp}%
\special{pa 2680 800}%
\special{pa 2620 860}%
\special{fp}%
\special{pa 1780 800}%
\special{pa 600 1980}%
\special{fp}%
\special{pa 1720 800}%
\special{pa 600 1920}%
\special{fp}%
\special{pa 1660 800}%
\special{pa 600 1860}%
\special{fp}%
\special{pa 1600 800}%
\special{pa 600 1800}%
\special{fp}%
\special{pa 1540 800}%
\special{pa 600 1740}%
\special{fp}%
\special{pa 1480 800}%
\special{pa 600 1680}%
\special{fp}%
\special{pa 1420 800}%
\special{pa 600 1620}%
\special{fp}%
\special{pa 1360 800}%
\special{pa 600 1560}%
\special{fp}%
\special{pa 1300 800}%
\special{pa 600 1500}%
\special{fp}%
\special{pa 1240 800}%
\special{pa 600 1440}%
\special{fp}%
\special{pa 1180 800}%
\special{pa 600 1380}%
\special{fp}%
\special{pa 1120 800}%
\special{pa 600 1320}%
\special{fp}%
\special{pa 1060 800}%
\special{pa 600 1260}%
\special{fp}%
\special{pa 1000 800}%
\special{pa 600 1200}%
\special{fp}%
\special{pa 940 800}%
\special{pa 600 1140}%
\special{fp}%
\special{pa 880 800}%
\special{pa 600 1080}%
\special{fp}%
\special{pa 820 800}%
\special{pa 600 1020}%
\special{fp}%
\special{pa 760 800}%
\special{pa 600 960}%
\special{fp}%
\special{pa 700 800}%
\special{pa 600 900}%
\special{fp}%
\special{pa 640 800}%
\special{pa 600 840}%
\special{fp}%
% STR 2 0 3 0 Black White  
% 4 600 700 600 800 2 0 1 0
% $1$
\put(6.0000,-8.0000){\makebox(0,0)[lb]{{\colorbox[named]{White}{$1$}}}}%
% STR 2 0 3 0 Black White  
% 4 2400 2300 2400 2400 5 0 1 0
% $S_1$
\put(24.0000,-24.0000){\makebox(0,0){{\colorbox[named]{White}{$S_1$}}}}%
% STR 2 0 3 0 Black White  
% 4 1000 1100 1000 1200 5 0 1 0
% $S_2$
\put(10.0000,-12.0000){\makebox(0,0){{\colorbox[named]{White}{$S_2$}}}}%
% STR 2 0 3 0 Black White  
% 4 3000 700 3000 800 5 0 1 0
% $C'$
\put(30.0000,-8.0000){\makebox(0,0){{\colorbox[named]{White}{$C'$}}}}%
% STR 2 0 3 0 Black White  
% 4 2860 500 2860 600 5 0 1 0
% $C$
\put(28.6000,-6.0000){\makebox(0,0){{\colorbox[named]{White}{$C$}}}}%
\end{picture}}%
}
     \end{center}


故に求める面積$S$は,上のように$S_1$,$S_2$をおいて,
     \begin{align}
     S=1-S_1-S_2\label{1}
     \end{align}
と書ける.各項計算して
     \begin{align*}
     S_1&=\int_0^1x^2dx=\frac{1}{3} \\
     S_2&=\int_0^1(2Y^3-2Y^2+Y)dY \\
     =&\left[\frac{Y^4}{2}-\frac{2Y^3}{3}+\frac{Y^2}{2}\right]_0^1 \\
     =&\frac{1}{3}
     \end{align*}
であるから,\eqref{1}に代入して
     \[S=\frac{1}{3}\]
である.$\cdots$(答)
     
\newpage
\end{multicols}
\end{document}