\documentclass[a4j]{jarticle}
\usepackage{amsmath}
\usepackage{ascmac}
\usepackage{amssymb}
\usepackage{enumerate}
\usepackage{multicol}
\usepackage{framed}
\usepackage{fancyhdr}
\usepackage{latexsym}
\usepackage{indent}
\usepackage{cases}
\usepackage[dvips]{graphicx}
\usepackage{color}
\usepackage{emath}
\usepackage{emathPp}
\allowdisplaybreaks
\pagestyle{fancy}
\lhead{}
\chead{}
\rhead{東京大学前期$1979$年$5$番}
\begin{document}
%分数関係


\def\tfrac#1#2{{\textstyle\frac{#1}{#2}}} %数式中で文中表示の分数を使う時


%Σ関係

\def\dsum#1#2{{\displaystyle\sum_{#1}^{#2}}} %文中で数式表示のΣを使う時


%ベクトル関係


\def\vector#1{\overrightarrow{#1}}  %ベクトルを表現したいとき(aベクトルを表現するときは\ver
\def\norm#1{|\overrightarrow{#1}|} %ベクトルの絶対値
\def\vtwo#1#2{ \left(%
      \begin{array}{c}%
      #1 \\%
      #2 \\%
      \end{array}%
      \right) }                        %2次元ベクトル成分表示
      
      \def\vthree#1#2#3{ \left(
      \begin{array}{c}
      #1 \\
      #2 \\
      #3 \\
      \end{array}
      \right) }                        %3次元ベクトル成分表示



%数列関係


\def\an#1{\verb|{|$#1$\verb|}|}


%極限関係

\def\limit#1#2{\stackrel{#1 \to #2}{\longrightarrow}}   %等式変形からの極限
\def\dlim#1#2{{\displaystyle \lim_{#1\to#2}}} %文中で数式表示の極限を使う



%積分関係

\def\dint#1#2{{\displaystyle \int_{#1}^{#2}}} %文中で数式表示の積分を使う時

\def\ne{\nearrow}
\def\se{\searrow}
\def\nw{\nwarrow}
\def\ne{\nearrow}


%便利なやつ

\def\case#1#2{%
 \[\left\{%
 \begin{array}{l}%
 #1 \\%
 #2%
 \end{array}%
 \right.\] }                           %場合分け
 
\def\1{$\cos\theta=c$,$\sin\theta=s$とおく.}  %cs表示を与える前書きシータ
\def\2{$\cos t=c$,$\sin t=s$とおく.}     %cs表示を与える前書きt
\def\3{$\cos x=c$,$\sin x=s$とおく.}                %cs表示を与える前書きx

\def\fig#1#2#3 {%
\begin{wrapfigure}[#1]{r}{#2 zw}%
\vspace*{-1zh}%
\input{#3}%
\end{wrapfigure} }           %絵の挿入


\def\a{\alpha}   %ギリシャ文字
\def\b{\beta}
\def\g{\gamma}

%問題番号のためのマクロ

\newcounter{nombre} %必須
\renewcommand{\thenombre}{\arabic{nombre}} %任意
\setcounter{nombre}{2} %任意
\newcounter{nombresub}[nombre] %親子関係を定義
\renewcommand{\thenombresub}{\arabic{nombresub}} %任意
\setcounter{nombresub}{0} %任意
\newcommand{\prob}[1][]{\refstepcounter{nombre}#1[問題 \thenombre]}
\newcommand{\probsub}[1][]{\refstepcounter{nombresub}#1(\thenombresub)}


%1-1みたいなカウンタ(todaiとtodaia)
\newcounter{todai}
\setcounter{todai}{0}
\newcounter{todaisub}[todai] 
\setcounter{todaisub}{0} 
\newcommand{\todai}[1][]{\refstepcounter{todai}#1 \thetodai-\thetodaisub}
\newcommand{\todaib}[1][]{\refstepcounter{todai}#1\refstepcounter{todaisub}#1 {\bf [問題 \thetodai.\thetodaisub]}}
\newcommand{\todaia}[1][]{\refstepcounter{todaisub}#1 {\bf [問題 \thetodai.\thetodaisub]}}


     \begin{oframed}
     $t$を正の数とし,次の条件(A),(B)によって定まる$x$の$3$次式を$f(x)$とする.
          \begin{enumerate}[(A)]
          \item 曲線$y=f(x)\cdots (1)$は直線$y=x\cdots (2)$の上の$2$点P$(-t,-t)$,O$(0,0)$を通る.
          \item $f'(0)=0$,$f''(0)=2$
          \end{enumerate}
     さて,曲線(1)と曲線(2)との交点のうちで,$x$座標が最大のものをQとし,曲線(1)の点Oから点Qまでの部分と,
     線分OQとで囲まれた領域の面積を$S(t)$とする.このとき$\dlim{t}{\infty}S(t)$を求めよ. 
     \end{oframed}

\setlength{\columnseprule}{0.4pt}
\begin{multicols}{2}
{\bf[解]} 条件(A)から$a,\a\in\mathbb{R}$として,
     \begin{align*}
     f(x)-x=a(x+t)x(x-\a) \\
     f(x)=x(ax^2+a(t-\a)x-at\a+1)
     \end{align*}
と書ける.従って
     \begin{align*}
     f'(x)&=3ax^2+2a(t-\a)x+(1-at\a)\\
     f''(x)&=6ax+2a(t-\a)
     \end{align*}
となるので,条件(B)から,
     \begin{align*}
          &\begin{cases}
          f'(0)=(1-at\a)=0 \\
          f''(0)=2a(t-\a)=2
          \end{cases}\\
     \Longleftrightarrow
          &\begin{cases}
          \a=\dfrac{t}{1+t} \\
          a=\dfrac{t^2}{t+1}
          \end{cases}
     \end{align*}
である.以上から,
     \begin{align*}
     f(x)-x=\frac{t^2}{t+1}(x+t)x\left(x-\frac{t}{t+1}\right)
     \end{align*}
となる.$t>0$からグラフの概形は下図.ただし$d=t/(t+1)$とした.
     
     \begin{zahyou}[ul=10mm](-2,2)(-2,2)
     \def\Fx{2*X**3+X**2}
     \def\Gx{X}
     \YKouten\Fx\Gx{}{-0.5}\tmpxi\P
     \YKouten\Fx\Gx{0.1}{}\tempxii\Q
     \YGurafu*\Fx
     \YGurafu*\Gx
     \Put\P[syaei=x,xlabel=-t]{}
     \Put\Q[syaei=x,xlabel=d]{}
     \end{zahyou}
     
従って,
     \begin{align*}
     S(t)&=\int_0^d(f(x)-x)\,dx \\
     &=\int_0^d\left(\frac{t^2}{t+1}(x+t)x\left(x-\frac{t}{t+1}\right)\right)\,dx \\
     &=\teisekibun{-\frac{t+1}{4t^2}x^4-\frac{1}{3}x^3+\frac{1}{2}x^2}{0}{d} \\
     &=\frac{-1}{4}\left(\frac{1}{t}+\frac{1}{t^2}\right)d^4-\frac{1}{3}d^3+\frac{1}{2}d^2 \\
     &\limit{t}{\infty}\frac{1}{2}-\frac{1}{3}=\frac{1}{6}
     \end{align*}
となる.$\cdots$(答)

\newpage
\end{multicols}
\end{document}