\documentclass[a4j]{jarticle}
\usepackage{amsmath}
\usepackage{ascmac}
\usepackage{amssymb}
\usepackage{enumerate}
\usepackage{multicol}
\usepackage{framed}
\usepackage{fancyhdr}
\usepackage{latexsym}
\usepackage{indent}
\usepackage{cases}
\usepackage[dvips]{graphicx}
\usepackage{color}
\allowdisplaybreaks
\pagestyle{fancy}
\lhead{}
\chead{}
\rhead{東京大学前期$1985$年$2$番}
\begin{document}
%分数関係


\def\tfrac#1#2{{\textstyle\frac{#1}{#2}}} %数式中で文中表示の分数を使う時


%Σ関係

\def\dsum#1#2{{\displaystyle\sum_{#1}^{#2}}} %文中で数式表示のΣを使う時


%ベクトル関係


\def\vector#1{\overrightarrow{#1}}  %ベクトルを表現したいとき(aベクトルを表現するときは\ver
\def\norm#1{|\overrightarrow{#1}|} %ベクトルの絶対値
\def\vtwo#1#2{ \left(%
      \begin{array}{c}%
      #1 \\%
      #2 \\%
      \end{array}%
      \right) }                        %2次元ベクトル成分表示
      
      \def\vthree#1#2#3{ \left(
      \begin{array}{c}
      #1 \\
      #2 \\
      #3 \\
      \end{array}
      \right) }                        %3次元ベクトル成分表示



%数列関係


\def\an#1{\verb|{|$#1$\verb|}|}


%極限関係

\def\limit#1#2{\stackrel{#1 \to #2}{\longrightarrow}}   %等式変形からの極限
\def\dlim#1#2{{\displaystyle \lim_{#1\to#2}}} %文中で数式表示の極限を使う



%積分関係

\def\dint#1#2{{\displaystyle \int_{#1}^{#2}}} %文中で数式表示の積分を使う時

\def\ne{\nearrow}
\def\se{\searrow}
\def\nw{\nwarrow}
\def\ne{\nearrow}


%便利なやつ

\def\case#1#2{%
 \[\left\{%
 \begin{array}{l}%
 #1 \\%
 #2%
 \end{array}%
 \right.\] }                           %場合分け
 
\def\1{$\cos\theta=c$,$\sin\theta=s$とおく.}  %cs表示を与える前書きシータ
\def\2{$\cos t=c$,$\sin t=s$とおく.}     %cs表示を与える前書きt
\def\3{$\cos x=c$,$\sin x=s$とおく.}                %cs表示を与える前書きx

\def\fig#1#2#3 {%
\begin{wrapfigure}[#1]{r}{#2 zw}%
\vspace*{-1zh}%
\input{#3}%
\end{wrapfigure} }           %絵の挿入


\def\a{\alpha}   %ギリシャ文字
\def\b{\beta}
\def\g{\gamma}

%問題番号のためのマクロ

\newcounter{nombre} %必須
\renewcommand{\thenombre}{\arabic{nombre}} %任意
\setcounter{nombre}{2} %任意
\newcounter{nombresub}[nombre] %親子関係を定義
\renewcommand{\thenombresub}{\arabic{nombresub}} %任意
\setcounter{nombresub}{0} %任意
\newcommand{\prob}[1][]{\refstepcounter{nombre}#1[問題 \thenombre]}
\newcommand{\probsub}[1][]{\refstepcounter{nombresub}#1(\thenombresub)}


%1-1みたいなカウンタ(todaiとtodaia)
\newcounter{todai}
\setcounter{todai}{0}
\newcounter{todaisub}[todai] 
\setcounter{todaisub}{0} 
\newcommand{\todai}[1][]{\refstepcounter{todai}#1 \thetodai-\thetodaisub}
\newcommand{\todaib}[1][]{\refstepcounter{todai}#1\refstepcounter{todaisub}#1 {\bf [問題 \thetodai.\thetodaisub]}}
\newcommand{\todaia}[1][]{\refstepcounter{todaisub}#1 {\bf [問題 \thetodai.\thetodaisub]}}


     \begin{oframed}
     $xy$平面において,$O$を原点,$A$を定点$(1,0)$とする.また,$P$,$Q$は円周$x^2+y^2=1$の上を動く$2$点であって,
     線分$OA$から正の向きにまわって線分$OP$に至る角と,線分$OP$から正の向きにまわって線分$OQ$に至る角が等しいという
     関係が成り立っているものとする.
     
     点$P$を通り$x$軸に垂直な直線と$x$軸との交点を$R$,点$Q$を通り$x$軸に垂直な直線と$x$軸の交点を$S$とする.
     実数$l$を与えた時,線分$RS$の長さが$l$と等しくなるような点$P$,$Q$の位置は何通りあるか.
     \end{oframed}

\setlength{\columnseprule}{0.4pt}
\begin{multicols}{2}
{\bf[解]}\2 ただし,$0\le t<2\pi$とする.すると$P(c,s)$,$Q(\cos 2t,\sin 2t )$となる.故に$R(c,0)$,$Q(\cos 2t,0)$であるから,
     \begin{align*}
     l&=|c-\cos 2t|=|2c^2-c-1| \\
     &=|(2c+1)(c-1)|\equiv f(t)
     \end{align*}
である.グラフは下図.
     \begin{center}
     \scalebox{0.7}{%WinTpicVersion4.32a
{\unitlength 0.1in%
\begin{picture}(34.0000,34.0000)(20.0000,-44.0000)%
% STR 2 0 3 0 Black White  
% 4 3590 3797 3590 3810 4 2000 0 0
% O
\put(35.9000,-38.1000){\makebox(0,0)[rt]{O}}%
% STR 2 0 3 0 Black White  
% 4 3560 987 3560 1000 4 2000 0 0
% $y$
\put(35.6000,-10.0000){\makebox(0,0)[rt]{$y$}}%
% STR 2 0 3 0 Black White  
% 4 5400 3827 5400 3840 4 2000 0 0
% $x$
\put(54.0000,-38.4000){\makebox(0,0)[rt]{$x$}}%
% VECTOR 2 0 3 0 Black White  
% 2 3600 4400 3600 1000
% 
\special{pn 8}%
\special{pa 3600 4400}%
\special{pa 3600 1000}%
\special{fp}%
\special{sh 1}%
\special{pa 3600 1000}%
\special{pa 3580 1067}%
\special{pa 3600 1053}%
\special{pa 3620 1067}%
\special{pa 3600 1000}%
\special{fp}%
% VECTOR 2 0 3 0 Black White  
% 2 2000 3800 5400 3800
% 
\special{pn 8}%
\special{pa 2000 3800}%
\special{pa 5400 3800}%
\special{fp}%
\special{sh 1}%
\special{pa 5400 3800}%
\special{pa 5333 3780}%
\special{pa 5347 3800}%
\special{pa 5333 3820}%
\special{pa 5400 3800}%
\special{fp}%
% FUNC 2 0 3 0 Black White  
% 9 2000 1000 5400 4400 3600 3800 4600 3800 3600 2800 2600 1000 4600 4400 0 4 0 0
% abs((2x+1)(x-1))
\special{pn 8}%
\special{pn 8}%
\special{pa 2449 1000}%
\special{pa 2450 1005}%
\special{pa 2451 1008}%
\special{ip}%
\special{pa 2457 1045}%
\special{pa 2458 1052}%
\special{ip}%
\special{pa 2465 1089}%
\special{pa 2466 1097}%
\special{ip}%
\special{pa 2473 1134}%
\special{pa 2475 1141}%
\special{ip}%
\special{pa 2481 1178}%
\special{pa 2483 1186}%
\special{ip}%
\special{pa 2489 1223}%
\special{pa 2491 1230}%
\special{ip}%
\special{pa 2498 1267}%
\special{pa 2499 1275}%
\special{ip}%
\special{pa 2506 1311}%
\special{pa 2507 1319}%
\special{ip}%
\special{pa 2514 1356}%
\special{pa 2515 1361}%
\special{pa 2516 1364}%
\special{ip}%
\special{pa 2522 1400}%
\special{pa 2524 1408}%
\special{ip}%
\special{pa 2531 1445}%
\special{pa 2532 1453}%
\special{ip}%
\special{pa 2539 1489}%
\special{pa 2541 1497}%
\special{ip}%
\special{pa 2548 1534}%
\special{pa 2549 1542}%
\special{ip}%
\special{pa 2556 1578}%
\special{pa 2558 1586}%
\special{ip}%
\special{pa 2565 1623}%
\special{pa 2565 1623}%
\special{pa 2566 1630}%
\special{ip}%
\special{pa 2574 1667}%
\special{pa 2575 1674}%
\special{pa 2575 1675}%
\special{ip}%
\special{pa 2582 1711}%
\special{pa 2584 1719}%
\special{ip}%
\special{pa 2591 1756}%
\special{pa 2593 1764}%
\special{ip}%
\special{pa 2600 1800}%
\special{pa 2600 1800}%
\special{ip}%
\special{pa 2600 1800}%
\special{pa 2615 1875}%
\special{pa 2620 1899}%
\special{pa 2625 1924}%
\special{pa 2630 1948}%
\special{pa 2635 1973}%
\special{pa 2665 2117}%
\special{pa 2670 2140}%
\special{pa 2675 2164}%
\special{pa 2680 2187}%
\special{pa 2685 2211}%
\special{pa 2715 2349}%
\special{pa 2720 2371}%
\special{pa 2725 2394}%
\special{pa 2730 2416}%
\special{pa 2735 2439}%
\special{pa 2765 2571}%
\special{pa 2770 2592}%
\special{pa 2775 2614}%
\special{pa 2780 2635}%
\special{pa 2785 2657}%
\special{pa 2815 2783}%
\special{pa 2820 2803}%
\special{pa 2825 2824}%
\special{pa 2830 2844}%
\special{pa 2835 2865}%
\special{pa 2865 2985}%
\special{pa 2870 3004}%
\special{pa 2875 3024}%
\special{pa 2880 3043}%
\special{pa 2885 3063}%
\special{pa 2915 3177}%
\special{pa 2920 3195}%
\special{pa 2925 3214}%
\special{pa 2930 3232}%
\special{pa 2935 3251}%
\special{pa 2965 3359}%
\special{pa 2970 3376}%
\special{pa 2975 3394}%
\special{pa 2980 3411}%
\special{pa 2985 3429}%
\special{pa 3015 3531}%
\special{pa 3020 3547}%
\special{pa 3025 3564}%
\special{pa 3030 3580}%
\special{pa 3035 3597}%
\special{pa 3065 3693}%
\special{pa 3070 3708}%
\special{pa 3075 3724}%
\special{pa 3080 3739}%
\special{pa 3085 3755}%
\special{pa 3100 3800}%
\special{pa 3115 3755}%
\special{pa 3120 3741}%
\special{pa 3125 3726}%
\special{pa 3130 3712}%
\special{pa 3135 3697}%
\special{pa 3165 3613}%
\special{pa 3170 3600}%
\special{pa 3175 3586}%
\special{pa 3180 3573}%
\special{pa 3185 3559}%
\special{pa 3215 3481}%
\special{pa 3220 3469}%
\special{pa 3225 3456}%
\special{pa 3230 3444}%
\special{pa 3235 3431}%
\special{pa 3265 3359}%
\special{pa 3270 3348}%
\special{pa 3275 3336}%
\special{pa 3280 3325}%
\special{pa 3285 3313}%
\special{pa 3315 3247}%
\special{pa 3320 3237}%
\special{pa 3325 3226}%
\special{pa 3330 3216}%
\special{pa 3335 3205}%
\special{pa 3365 3145}%
\special{pa 3370 3136}%
\special{pa 3375 3126}%
\special{pa 3380 3117}%
\special{pa 3385 3107}%
\special{pa 3415 3053}%
\special{pa 3420 3045}%
\special{pa 3425 3036}%
\special{pa 3430 3028}%
\special{pa 3435 3019}%
\special{pa 3465 2971}%
\special{pa 3470 2964}%
\special{pa 3475 2956}%
\special{pa 3480 2949}%
\special{pa 3485 2941}%
\special{pa 3515 2899}%
\special{pa 3520 2893}%
\special{pa 3525 2886}%
\special{pa 3530 2880}%
\special{pa 3535 2873}%
\special{pa 3565 2837}%
\special{pa 3570 2832}%
\special{pa 3575 2826}%
\special{pa 3580 2821}%
\special{pa 3585 2815}%
\special{pa 3615 2785}%
\special{pa 3620 2781}%
\special{pa 3625 2776}%
\special{pa 3630 2772}%
\special{pa 3635 2767}%
\special{pa 3665 2743}%
\special{pa 3670 2740}%
\special{pa 3675 2736}%
\special{pa 3680 2733}%
\special{pa 3685 2729}%
\special{pa 3715 2711}%
\special{pa 3720 2709}%
\special{pa 3725 2706}%
\special{pa 3730 2704}%
\special{pa 3735 2701}%
\special{pa 3765 2689}%
\special{pa 3770 2688}%
\special{pa 3775 2686}%
\special{pa 3780 2685}%
\special{pa 3785 2683}%
\special{pa 3815 2677}%
\special{pa 3820 2677}%
\special{pa 3825 2676}%
\special{pa 3830 2676}%
\special{pa 3835 2675}%
\special{pa 3865 2675}%
\special{pa 3870 2676}%
\special{pa 3875 2676}%
\special{pa 3880 2677}%
\special{pa 3885 2677}%
\special{pa 3915 2683}%
\special{pa 3920 2685}%
\special{pa 3925 2686}%
\special{pa 3930 2688}%
\special{pa 3935 2689}%
\special{pa 3965 2701}%
\special{pa 3970 2704}%
\special{pa 3975 2706}%
\special{pa 3980 2709}%
\special{pa 3985 2711}%
\special{pa 4015 2729}%
\special{pa 4020 2733}%
\special{pa 4025 2736}%
\special{pa 4030 2740}%
\special{pa 4035 2743}%
\special{pa 4065 2767}%
\special{pa 4070 2772}%
\special{pa 4075 2776}%
\special{pa 4080 2781}%
\special{pa 4085 2785}%
\special{pa 4115 2815}%
\special{pa 4120 2821}%
\special{pa 4125 2826}%
\special{pa 4130 2832}%
\special{pa 4135 2837}%
\special{pa 4165 2873}%
\special{pa 4170 2880}%
\special{pa 4175 2886}%
\special{pa 4180 2893}%
\special{pa 4185 2899}%
\special{pa 4215 2941}%
\special{pa 4220 2949}%
\special{pa 4225 2956}%
\special{pa 4230 2964}%
\special{pa 4235 2971}%
\special{pa 4265 3019}%
\special{pa 4270 3028}%
\special{pa 4275 3036}%
\special{pa 4280 3045}%
\special{pa 4285 3053}%
\special{pa 4315 3107}%
\special{pa 4320 3117}%
\special{pa 4325 3126}%
\special{pa 4330 3136}%
\special{pa 4335 3145}%
\special{pa 4365 3205}%
\special{pa 4370 3216}%
\special{pa 4375 3226}%
\special{pa 4380 3237}%
\special{pa 4385 3247}%
\special{pa 4415 3313}%
\special{pa 4420 3325}%
\special{pa 4425 3336}%
\special{pa 4430 3348}%
\special{pa 4435 3359}%
\special{pa 4465 3431}%
\special{pa 4470 3444}%
\special{pa 4475 3456}%
\special{pa 4480 3469}%
\special{pa 4485 3481}%
\special{pa 4515 3559}%
\special{pa 4520 3573}%
\special{pa 4525 3586}%
\special{pa 4530 3600}%
\special{pa 4535 3613}%
\special{pa 4565 3697}%
\special{pa 4570 3712}%
\special{pa 4575 3726}%
\special{pa 4580 3741}%
\special{pa 4585 3755}%
\special{pa 4600 3800}%
\special{fp}%
\special{pn 8}%
\special{pa 4603 3792}%
\special{pa 4614 3757}%
\special{ip}%
\special{pa 4617 3749}%
\special{pa 4620 3739}%
\special{pa 4625 3724}%
\special{pa 4628 3713}%
\special{ip}%
\special{pa 4631 3705}%
\special{pa 4635 3693}%
\special{pa 4642 3670}%
\special{ip}%
\special{pa 4645 3662}%
\special{pa 4656 3626}%
\special{ip}%
\special{pa 4658 3618}%
\special{pa 4665 3597}%
\special{pa 4669 3582}%
\special{ip}%
\special{pa 4672 3575}%
\special{pa 4675 3564}%
\special{pa 4680 3547}%
\special{pa 4683 3539}%
\special{ip}%
\special{pa 4685 3531}%
\special{pa 4685 3531}%
\special{pa 4696 3495}%
\special{ip}%
\special{pa 4698 3487}%
\special{pa 4708 3451}%
\special{ip}%
\special{pa 4711 3443}%
\special{pa 4715 3429}%
\special{pa 4720 3411}%
\special{pa 4721 3407}%
\special{ip}%
\special{pa 4723 3400}%
\special{pa 4725 3394}%
\special{pa 4730 3376}%
\special{pa 4734 3363}%
\special{ip}%
\special{pa 4736 3356}%
\special{pa 4746 3319}%
\special{ip}%
\special{pa 4748 3312}%
\special{pa 4758 3275}%
\special{ip}%
\special{pa 4760 3268}%
\special{pa 4765 3251}%
\special{pa 4770 3232}%
\special{pa 4770 3231}%
\special{ip}%
\special{pa 4772 3224}%
\special{pa 4775 3214}%
\special{pa 4780 3195}%
\special{pa 4782 3187}%
\special{ip}%
\special{pa 4784 3179}%
\special{pa 4785 3177}%
\special{pa 4794 3143}%
\special{ip}%
\special{pa 4796 3135}%
\special{pa 4806 3099}%
\special{ip}%
\special{pa 4808 3091}%
\special{pa 4815 3063}%
\special{pa 4817 3055}%
\special{ip}%
\special{pa 4819 3047}%
\special{pa 4820 3043}%
\special{pa 4825 3024}%
\special{pa 4828 3011}%
\special{ip}%
\special{pa 4830 3003}%
\special{pa 4835 2985}%
\special{pa 4840 2966}%
\special{ip}%
\special{pa 4842 2958}%
\special{pa 4851 2922}%
\special{ip}%
\special{pa 4853 2914}%
\special{pa 4862 2878}%
\special{ip}%
\special{pa 4864 2870}%
\special{pa 4865 2865}%
\special{pa 4870 2844}%
\special{pa 4873 2833}%
\special{ip}%
\special{pa 4875 2825}%
\special{pa 4875 2824}%
\special{pa 4880 2803}%
\special{pa 4883 2789}%
\special{ip}%
\special{pa 4885 2781}%
\special{pa 4894 2745}%
\special{ip}%
\special{pa 4896 2737}%
\special{pa 4905 2700}%
\special{ip}%
\special{pa 4907 2692}%
\special{pa 4915 2657}%
\special{pa 4915 2656}%
\special{ip}%
\special{pa 4917 2648}%
\special{pa 4920 2635}%
\special{pa 4925 2614}%
\special{pa 4926 2611}%
\special{ip}%
\special{pa 4927 2603}%
\special{pa 4930 2592}%
\special{pa 4935 2571}%
\special{pa 4936 2567}%
\special{ip}%
\special{pa 4938 2559}%
\special{pa 4946 2522}%
\special{ip}%
\special{pa 4948 2514}%
\special{pa 4956 2478}%
\special{ip}%
\special{pa 4958 2470}%
\special{pa 4965 2439}%
\special{pa 4966 2433}%
\special{ip}%
\special{pa 4968 2425}%
\special{pa 4970 2416}%
\special{pa 4975 2394}%
\special{pa 4976 2389}%
\special{ip}%
\special{pa 4978 2381}%
\special{pa 4980 2371}%
\special{pa 4985 2349}%
\special{pa 4986 2344}%
\special{ip}%
\special{pa 4988 2336}%
\special{pa 4996 2299}%
\special{ip}%
\special{pa 4998 2291}%
\special{pa 5005 2255}%
\special{ip}%
\special{pa 5007 2247}%
\special{pa 5015 2211}%
\special{pa 5015 2210}%
\special{ip}%
\special{pa 5017 2202}%
\special{pa 5020 2187}%
\special{pa 5025 2165}%
\special{ip}%
\special{pa 5026 2158}%
\special{pa 5030 2140}%
\special{pa 5034 2121}%
\special{ip}%
\special{pa 5036 2113}%
\special{pa 5044 2076}%
\special{ip}%
\special{pa 5045 2068}%
\special{pa 5053 2031}%
\special{ip}%
\special{pa 5054 2023}%
\special{pa 5062 1987}%
\special{ip}%
\special{pa 5064 1979}%
\special{pa 5065 1973}%
\special{pa 5070 1948}%
\special{pa 5071 1942}%
\special{ip}%
\special{pa 5073 1934}%
\special{pa 5075 1924}%
\special{pa 5080 1899}%
\special{pa 5080 1897}%
\special{ip}%
\special{pa 5082 1889}%
\special{pa 5085 1875}%
\special{pa 5090 1852}%
\special{ip}%
\special{pa 5091 1845}%
\special{pa 5098 1808}%
\special{ip}%
\special{pa 5100 1800}%
\special{pa 5107 1763}%
\special{ip}%
\special{pa 5109 1755}%
\special{pa 5115 1725}%
\special{pa 5116 1718}%
\special{ip}%
\special{pa 5118 1710}%
\special{pa 5120 1699}%
\special{pa 5125 1674}%
\special{pa 5125 1673}%
\special{ip}%
\special{pa 5127 1665}%
\special{pa 5130 1648}%
\special{pa 5134 1628}%
\special{ip}%
\special{pa 5135 1621}%
\special{pa 5143 1584}%
\special{ip}%
\special{pa 5144 1576}%
\special{pa 5151 1539}%
\special{ip}%
\special{pa 5153 1531}%
\special{pa 5160 1494}%
\special{ip}%
\special{pa 5161 1486}%
\special{pa 5165 1467}%
\special{pa 5168 1449}%
\special{ip}%
\special{pa 5170 1441}%
\special{pa 5170 1440}%
\special{pa 5175 1414}%
\special{pa 5177 1404}%
\special{ip}%
\special{pa 5178 1396}%
\special{pa 5180 1387}%
\special{pa 5185 1361}%
\special{pa 5185 1359}%
\special{ip}%
\special{pa 5187 1351}%
\special{pa 5194 1314}%
\special{ip}%
\special{pa 5195 1306}%
\special{pa 5202 1270}%
\special{ip}%
\special{pa 5203 1262}%
\special{pa 5210 1225}%
\special{ip}%
\special{pa 5212 1217}%
\special{pa 5215 1199}%
\special{pa 5218 1180}%
\special{ip}%
\special{pa 5220 1172}%
\special{pa 5220 1171}%
\special{pa 5225 1144}%
\special{pa 5227 1135}%
\special{ip}%
\special{pa 5228 1127}%
\special{pa 5230 1116}%
\special{pa 5235 1090}%
\special{ip}%
\special{pa 5236 1082}%
\special{pa 5243 1045}%
\special{ip}%
\special{pa 5244 1037}%
\special{pa 5250 1005}%
\special{pa 5251 1000}%
\special{ip}%
% LINE 2 2 3 0 Black White  
% 8 2600 1800 2600 3800 2600 1800 3600 1800 3850 2660 3850 3800 3850 2670 3600 2670
% 
\special{pn 8}%
\special{pa 2600 1800}%
\special{pa 2600 3800}%
\special{dt 0.045}%
\special{pa 2600 1800}%
\special{pa 3600 1800}%
\special{dt 0.045}%
\special{pa 3850 2660}%
\special{pa 3850 3800}%
\special{dt 0.045}%
\special{pa 3850 2670}%
\special{pa 3600 2670}%
\special{dt 0.045}%
% STR 2 0 3 0 Black White  
% 4 2730 3610 2730 3710 5 0 1 0
% -1
\put(27.3000,-37.1000){\makebox(0,0){{\colorbox[named]{White}{-1}}}}%
% STR 2 0 3 0 Black White  
% 4 4670 3580 4670 3680 5 0 0 0
% $1$
\put(46.7000,-36.8000){\makebox(0,0){$1$}}%
% STR 2 0 3 0 Black White  
% 4 4000 3570 4000 3670 5 0 1 0
% $1/4$
\put(40.0000,-36.7000){\makebox(0,0){{\colorbox[named]{White}{$1/4$}}}}%
% STR 2 0 3 0 Black White  
% 4 3710 2390 3710 2490 5 0 1 0
% $9/8$
\put(37.1000,-24.9000){\makebox(0,0){{\colorbox[named]{White}{$9/8$}}}}%
% STR 2 0 3 0 Black White  
% 4 3730 1600 3730 1700 5 0 1 0
% $2$
\put(37.3000,-17.0000){\makebox(0,0){{\colorbox[named]{White}{$2$}}}}%
\end{picture}}%
}
     \end{center}

故に,$t$と$c$の関係に注意し,
$t$と$P$,$Q$の位置関係が一対一対応であることより,求める場合の数は以下の通り.$\cdots$(答)
     \begin{align*}
          \begin{array}{|c|c|c|c|} \hline
          l              & c  & t & \text{位置} \\ \hline
          0             & 2 & 3 & 3 \\ \hline
          0<l<9/8   & 3 & 6 & 6 \\ \hline
          9/8          & 2 & 4 & 4 \\ \hline
          9/8<l<2   & 1 & 2 & 2 \\ \hline
          2             & 1 & 1 & 1 \\ \hline
          2<l          & 0 & 0 & 0 \\ \hline
          \end{array}
     \end{align*}

     
             
\newpage
\end{multicols}
\end{document}