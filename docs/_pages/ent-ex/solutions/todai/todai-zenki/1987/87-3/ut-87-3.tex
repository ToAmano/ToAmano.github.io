\documentclass[a4j]{jarticle}
\usepackage{amsmath}
\usepackage{ascmac}
\usepackage{amssymb}
\usepackage{enumerate}
\usepackage{multicol}
\usepackage{framed}
\usepackage{latexsym}
\usepackage{indent}
\title{}
\begin{document}
%分数関係


\def\tfrac#1#2{{\textstyle\frac{#1}{#2}}} %数式中で文中表示の分数を使う時


%Σ関係

\def\dsum#1#2{{\displaystyle\sum_{#1}^{#2}}} %文中で数式表示のΣを使う時


%ベクトル関係


\def\vector#1{\overrightarrow{#1}}  %ベクトルを表現したいとき(aベクトルを表現するときは\ver
\def\norm#1{|\overrightarrow{#1}|} %ベクトルの絶対値
\def\vtwo#1#2{ \left(%
      \begin{array}{c}%
      #1 \\%
      #2 \\%
      \end{array}%
      \right) }                        %2次元ベクトル成分表示
      
      \def\vthree#1#2#3{ \left(
      \begin{array}{c}
      #1 \\
      #2 \\
      #3 \\
      \end{array}
      \right) }                        %3次元ベクトル成分表示



%数列関係


\def\an#1{\verb|{|$#1$\verb|}|}


%極限関係

\def\limit#1#2{\stackrel{#1 \to #2}{\longrightarrow}}   %等式変形からの極限
\def\dlim#1#2{{\displaystyle \lim_{#1\to#2}}} %文中で数式表示の極限を使う



%積分関係

\def\dint#1#2{{\displaystyle \int_{#1}^{#2}}} %文中で数式表示の積分を使う時

\def\ne{\nearrow}
\def\se{\searrow}
\def\nw{\nwarrow}
\def\ne{\nearrow}


%便利なやつ

\def\case#1#2{%
 \[\left\{%
 \begin{array}{l}%
 #1 \\%
 #2%
 \end{array}%
 \right.\] }                           %場合分け
 
\def\1{$\cos\theta=c$,$\sin\theta=s$とおく.}  %cs表示を与える前書きシータ
\def\2{$\cos t=c$,$\sin t=s$とおく.}     %cs表示を与える前書きt
\def\3{$\cos x=c$,$\sin x=s$とおく.}                %cs表示を与える前書きx

\def\fig#1#2#3 {%
\begin{wrapfigure}[#1]{r}{#2 zw}%
\vspace*{-1zh}%
\input{#3}%
\end{wrapfigure} }           %絵の挿入


\def\a{\alpha}   %ギリシャ文字
\def\b{\beta}
\def\g{\gamma}

%問題番号のためのマクロ

\newcounter{nombre} %必須
\renewcommand{\thenombre}{\arabic{nombre}} %任意
\setcounter{nombre}{2} %任意
\newcounter{nombresub}[nombre] %親子関係を定義
\renewcommand{\thenombresub}{\arabic{nombresub}} %任意
\setcounter{nombresub}{0} %任意
\newcommand{\prob}[1][]{\refstepcounter{nombre}#1[問題 \thenombre]}
\newcommand{\probsub}[1][]{\refstepcounter{nombresub}#1(\thenombresub)}


%1-1みたいなカウンタ(todaiとtodaia)
\newcounter{todai}
\setcounter{todai}{0}
\newcounter{todaisub}[todai] 
\setcounter{todaisub}{0} 
\newcommand{\todai}[1][]{\refstepcounter{todai}#1 \thetodai-\thetodaisub}
\newcommand{\todaib}[1][]{\refstepcounter{todai}#1\refstepcounter{todaisub}#1 {\bf [問題 \thetodai.\thetodaisub]}}
\newcommand{\todaia}[1][]{\refstepcounter{todaisub}#1 {\bf [問題 \thetodai.\thetodaisub]}}


\begin{oframed}
$xyz$空間内の点$P(0,0,1)$を中心とする半径$1$の球面$K$がある.

$K$上の点$Q(a,b,c)$が条件$a>0,b>0,c>1$のもとで$K$上を動く時,$Q$において$K$に接する平面を$L$とし,$L$が$x$軸,$y$軸,$z$軸と交わる点をそれぞれ$A$,$B$,$C$とする.このような三角形$ABC$の面積の最小値を求めよ. 
\end{oframed}

\setlength{\columnseprule}{0.4pt}
\begin{multicols}{2}
{\bf[解]}$Q$は$K$上の点であるから
     \begin{align}
     a^2+b^2+(c-1)^2=1 \label{1}
     \end{align}
が成り立つ.$L$は$\vector{PQ}=(a,b,c-1)$に垂直で点$Q$を通るので,その方程式は
     \begin{align*}
     &a(x-a)+b(y-b)+(c-1)(z-c)=0 \\
     \therefore \ &ax+by+(c-1)z-c=0 \tag{$\because\eqref{1}$}
     \end{align*}
となる.ゆえに$a>0,b>0,c>1$から
     \begin{align*}
     A\left(\frac{c}{a},0,0\right) , B\left(0,\frac{c}{b},0\right) , C\left(0,0,\frac{c}{c-1}\right)
     \end{align*}     
である.そこで
     \begin{align}
     p=\frac{c}{a} , q=\frac{c}{b} , r=\frac{c}{c-1} \label{2}
     \end{align}
とおけば     
     \begin{align*}
     &\vector{CA}=(p,0,-r) , \vector{CB}=(0,q,-r) \\
     &|\vector{CA}|^2=p^2+r^2 ,  
     |\vector{CB}|^2=q^2+r^2 \\
     &\vector{CA}\cdot\vector{CB}=r^2
     \end{align*}
である.よって三角形$ABC$の面積$S$は
     \begin{align*}
     S&=\frac{1}{2}\sqrt{|\vector{CA}|^2|\vector{CB}|^2-(\vector{CA}\cdot\vector{CB})^2}    \\
     &=\frac{1}{2}\sqrt{(p^2+r^2)(q^2+r^2)-r^4} \\
     &=\frac{1}{2}\sqrt{(pqr)^2(\frac{1}{p^2}+\frac{1}{q^2}+\frac{1}{r^2})}     \\
     &=\frac{1}{2}\frac{pqr}{c} \tag{$\because\eqref{1},\eqref{2}$}
     \end{align*}
である.\eqref{2}の値を代入して$S=\dfrac{c^2}{2ab(c-1)}$だから,以下この最小値を求める.
$a,b>0$から,\eqref{1}にAM-GMを用いて
     \begin{align*}
     2ab\le a^2+b^2=1-(c-1)^2=2c-c^2
     \end{align*}
である.等号成立は$a=b$の時.したがって$c-1>0$に注意して
     \begin{align}
     S&=\frac{c^2}{ab(c-1)} \nonumber\\
     &\ge \frac{1}{2c-c^2}\frac{c^2}{c-1} \nonumber\\
     &=\frac{1}{3-(c+2/c)} \nonumber\\
     &\ge \frac{1}{3-2\sqrt{2}}\tag{$\because$AM-GM} \\
     &=3+2\sqrt{2} \label{3}
     \end{align}
等号成立は$c=2/c$つまり$c=\sqrt{2}$のときである($c>0$).以上の等号成立条件を\eqref{1}
に代入すれば$(a,b,c)=(\sqrt{\sqrt{2}-1},\sqrt{\sqrt{2}-1},\sqrt{2})$となって条件$a>0,b>0,c>1$を満たす.故に求める最小値は\eqref{3}の$\min S=3+2\sqrt{2}\cdots$(答)である.
\newpage
\end{multicols}
\end{document}