\documentclass[a4j]{jarticle}
\usepackage{amsmath}
\usepackage{ascmac}
\usepackage{amssymb}
\usepackage{enumerate}
\usepackage{multicol}
\usepackage{framed}
\usepackage{fancyhdr}
\usepackage{latexsym}
\usepackage{indent}
\usepackage{cases}
\allowdisplaybreaks
\pagestyle{fancy}
\lhead{}
\chead{}
\rhead{東京大学前期$1963$年$4$番}
\begin{document}
%分数関係


\def\tfrac#1#2{{\textstyle\frac{#1}{#2}}} %数式中で文中表示の分数を使う時


%Σ関係

\def\dsum#1#2{{\displaystyle\sum_{#1}^{#2}}} %文中で数式表示のΣを使う時


%ベクトル関係


\def\vector#1{\overrightarrow{#1}}  %ベクトルを表現したいとき(aベクトルを表現するときは\ver
\def\norm#1{|\overrightarrow{#1}|} %ベクトルの絶対値
\def\vtwo#1#2{ \left(%
      \begin{array}{c}%
      #1 \\%
      #2 \\%
      \end{array}%
      \right) }                        %2次元ベクトル成分表示
      
      \def\vthree#1#2#3{ \left(
      \begin{array}{c}
      #1 \\
      #2 \\
      #3 \\
      \end{array}
      \right) }                        %3次元ベクトル成分表示



%数列関係


\def\an#1{\verb|{|$#1$\verb|}|}


%極限関係

\def\limit#1#2{\stackrel{#1 \to #2}{\longrightarrow}}   %等式変形からの極限
\def\dlim#1#2{{\displaystyle \lim_{#1\to#2}}} %文中で数式表示の極限を使う



%積分関係

\def\dint#1#2{{\displaystyle \int_{#1}^{#2}}} %文中で数式表示の積分を使う時

\def\ne{\nearrow}
\def\se{\searrow}
\def\nw{\nwarrow}
\def\ne{\nearrow}


%便利なやつ

\def\case#1#2{%
 \[\left\{%
 \begin{array}{l}%
 #1 \\%
 #2%
 \end{array}%
 \right.\] }                           %場合分け
 
\def\1{$\cos\theta=c$,$\sin\theta=s$とおく.}  %cs表示を与える前書きシータ
\def\2{$\cos t=c$,$\sin t=s$とおく.}     %cs表示を与える前書きt
\def\3{$\cos x=c$,$\sin x=s$とおく.}                %cs表示を与える前書きx

\def\fig#1#2#3 {%
\begin{wrapfigure}[#1]{r}{#2 zw}%
\vspace*{-1zh}%
\input{#3}%
\end{wrapfigure} }           %絵の挿入


\def\a{\alpha}   %ギリシャ文字
\def\b{\beta}
\def\g{\gamma}

%問題番号のためのマクロ

\newcounter{nombre} %必須
\renewcommand{\thenombre}{\arabic{nombre}} %任意
\setcounter{nombre}{2} %任意
\newcounter{nombresub}[nombre] %親子関係を定義
\renewcommand{\thenombresub}{\arabic{nombresub}} %任意
\setcounter{nombresub}{0} %任意
\newcommand{\prob}[1][]{\refstepcounter{nombre}#1[問題 \thenombre]}
\newcommand{\probsub}[1][]{\refstepcounter{nombresub}#1(\thenombresub)}


%1-1みたいなカウンタ(todaiとtodaia)
\newcounter{todai}
\setcounter{todai}{0}
\newcounter{todaisub}[todai] 
\setcounter{todaisub}{0} 
\newcommand{\todai}[1][]{\refstepcounter{todai}#1 \thetodai-\thetodaisub}
\newcommand{\todaib}[1][]{\refstepcounter{todai}#1\refstepcounter{todaisub}#1 {\bf [問題 \thetodai.\thetodaisub]}}
\newcommand{\todaia}[1][]{\refstepcounter{todaisub}#1 {\bf [問題 \thetodai.\thetodaisub]}}


     \begin{oframed}
     一辺の長さ$a$の正四面体$ABCD$の辺$AB$,$AC$,$AD$の上に$A$から等距離にそれぞれ点
     $P$,$Q$,$R$をとり,$P$,$Q$,$R$から面$BCD$に下ろした垂線の足をそれぞれ
     $P'$,$Q'$,$R'$とする.
          \begin{enumerate}[(1)]
          \item 三角柱$PQR$-$P'Q'R'$の体積が最大になる時の$AP$の長さを求めよ.
          \item この三角柱の体積の最大値$V_0$と正四面体$ABCD$の体積$V$の比$\dfrac{V_0}{V}$
          を求めよ.
          \end{enumerate}
     \end{oframed}

\setlength{\columnseprule}{0.4pt}
\begin{multicols}{2}
{\bf[解]}
\begin{enumerate}[(1)]
\item  $A$から平面$BCD$に下ろした垂足$H$とする.$ABCD$が正四面体だから
$|AH|=\dfrac{\sqrt{6}}{3}a$である.
$|AP|=x \ (0<x<a)$とおく.すると相似から
     \begin{align*}
     |PP'|&=\frac{a-x}{a}|AH|=\frac{\sqrt{6}}{3}(a-x) \\
     |PQ|&=x
     \end{align*}
である.$\triangle PQR$は正三角形でその面積
     \[S=\dfrac{\sqrt{3}}{4}|PQ|^2=\dfrac{\sqrt{3}}{4}x^2\]
だから,三角錐の
体積$V(x)$は
     \begin{align*}
     V(x)&=(\triangle PQR)|PP'| \\
     &=\frac{\sqrt{6}}{3}(a-x)S \\
     &=\frac{\sqrt{2}}{4}x^2(a-x)
     \end{align*}
となる.$x,a-x>0$からAM-GMより
     \begin{align}
     V(x)&=\sqrt{2}\left(\frac{x}{2}\right)^2(a-x) \nonumber\\
     &\le\sqrt{2}\left(\frac{a}{3}\right)^3\label{2}
     \end{align}
である.等号成立は$\dfrac{x}{2}=a-x\Longleftrightarrow x=\dfrac{2}{3}a$である.(これは$0<x<a$を満たす.)
以上から求める値は$|AP|=\dfrac{2}{3}a$である.$\cdots$(答)

\item \eqref{2}から$V_0=\dfrac{\sqrt{2}}{27}a^3$であり,また
     \[V=\dfrac{1}{3}S|AH|=\dfrac{\sqrt{2}}{12}a^3 \]
だから,
     \begin{align*}
     \dfrac{V_0}{V}=\dfrac{\dfrac{\sqrt{2}}{27}a^3}{\dfrac{\sqrt{2}}{12}a^3}=\frac{4}{9}
     \end{align*}
である.$\cdots$(答)
\end{enumerate}
\newpage
\end{multicols}
\end{document}