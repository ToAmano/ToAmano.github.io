\documentclass[a4j]{jarticle}
\usepackage{amsmath}
\usepackage{ascmac}
\usepackage{amssymb}
\usepackage{enumerate}
\usepackage{multicol}
\usepackage{framed}
\usepackage{fancyhdr}
\usepackage{latexsym}
\usepackage{indent}
\usepackage{cases}
\usepackage[dvips]{graphicx}
\usepackage{color}
\allowdisplaybreaks
\pagestyle{fancy}
\lhead{}
\chead{}
\rhead{東京大学前期$1963$年$6$番}
\begin{document}
%分数関係


\def\tfrac#1#2{{\textstyle\frac{#1}{#2}}} %数式中で文中表示の分数を使う時


%Σ関係

\def\dsum#1#2{{\displaystyle\sum_{#1}^{#2}}} %文中で数式表示のΣを使う時


%ベクトル関係


\def\vector#1{\overrightarrow{#1}}  %ベクトルを表現したいとき(aベクトルを表現するときは\ver
\def\norm#1{|\overrightarrow{#1}|} %ベクトルの絶対値
\def\vtwo#1#2{ \left(%
      \begin{array}{c}%
      #1 \\%
      #2 \\%
      \end{array}%
      \right) }                        %2次元ベクトル成分表示
      
      \def\vthree#1#2#3{ \left(
      \begin{array}{c}
      #1 \\
      #2 \\
      #3 \\
      \end{array}
      \right) }                        %3次元ベクトル成分表示



%数列関係


\def\an#1{\verb|{|$#1$\verb|}|}


%極限関係

\def\limit#1#2{\stackrel{#1 \to #2}{\longrightarrow}}   %等式変形からの極限
\def\dlim#1#2{{\displaystyle \lim_{#1\to#2}}} %文中で数式表示の極限を使う



%積分関係

\def\dint#1#2{{\displaystyle \int_{#1}^{#2}}} %文中で数式表示の積分を使う時

\def\ne{\nearrow}
\def\se{\searrow}
\def\nw{\nwarrow}
\def\ne{\nearrow}


%便利なやつ

\def\case#1#2{%
 \[\left\{%
 \begin{array}{l}%
 #1 \\%
 #2%
 \end{array}%
 \right.\] }                           %場合分け
 
\def\1{$\cos\theta=c$,$\sin\theta=s$とおく.}  %cs表示を与える前書きシータ
\def\2{$\cos t=c$,$\sin t=s$とおく.}     %cs表示を与える前書きt
\def\3{$\cos x=c$,$\sin x=s$とおく.}                %cs表示を与える前書きx

\def\fig#1#2#3 {%
\begin{wrapfigure}[#1]{r}{#2 zw}%
\vspace*{-1zh}%
\input{#3}%
\end{wrapfigure} }           %絵の挿入


\def\a{\alpha}   %ギリシャ文字
\def\b{\beta}
\def\g{\gamma}

%問題番号のためのマクロ

\newcounter{nombre} %必須
\renewcommand{\thenombre}{\arabic{nombre}} %任意
\setcounter{nombre}{2} %任意
\newcounter{nombresub}[nombre] %親子関係を定義
\renewcommand{\thenombresub}{\arabic{nombresub}} %任意
\setcounter{nombresub}{0} %任意
\newcommand{\prob}[1][]{\refstepcounter{nombre}#1[問題 \thenombre]}
\newcommand{\probsub}[1][]{\refstepcounter{nombresub}#1(\thenombresub)}


%1-1みたいなカウンタ(todaiとtodaia)
\newcounter{todai}
\setcounter{todai}{0}
\newcounter{todaisub}[todai] 
\setcounter{todaisub}{0} 
\newcommand{\todai}[1][]{\refstepcounter{todai}#1 \thetodai-\thetodaisub}
\newcommand{\todaib}[1][]{\refstepcounter{todai}#1\refstepcounter{todaisub}#1 {\bf [問題 \thetodai.\thetodaisub]}}
\newcommand{\todaia}[1][]{\refstepcounter{todaisub}#1 {\bf [問題 \thetodai.\thetodaisub]}}


     \begin{oframed}
     $n$を$2$より大きい正の整数とする.曲線
          \begin{align}
          y=x^n\tag{i}\label{1}
          \end{align}
     上で,$x$座標が$0$,$1$,$2$である点をそれぞれ$O$,$A$,$B$とし,$O$,$A$,$B$を通り$y$軸に平行な軸を持つ
     放物線
          \begin{align}
          y=f(x)\tag{ii}\label{2}
          \end{align}
     をえがく.曲線\eqref{1}および曲線\eqref{2}の,$O$,$A$の間にある部分の囲む面積を$S_1$,$A$,$B$の間にある部分の囲む
     面積を$S_2$とするとき,$S_1=S_2$となるためには,$n$はどのような数でなければならないか.
     \end{oframed}

\setlength{\columnseprule}{0.4pt}
\begin{multicols}{2}
{\bf[解]} $O(0,0)$$A(1,1)$$B(2,2^n)$である.まず$f(x)$を求める.$O$,$A$を通るから$a_{\not=0}$を用いて,
     \begin{align*}
     f(x)-x=ax(x-1)
     \end{align*}
と書ける.$B$を通る条件から
     \begin{align*}
     2^n-2=2a a=2^{n-1}-1
     \end{align*}
であるから,結局
     \begin{align*}
     f(x)=(2^{n-1}-1)x^2+(2-2^{n-1})x
     \end{align*}
である.$g_n(x)=x^n-f(x)$とおく.
     \begin{align}
     g_n(x)&=x\left[x^{n-1}-(2^{n-1}-1)x-2+2^{n-1}\right] \nonumber\\
     &=x\left[x^{n-1}+x+2^{n-1}(1-x)-2\right] \nonumber\\
     &=x(x-1)h(x) \label{3}
     \end{align}
である.ただし
      \[h(x)=(x^{n-2}+x^{n-3}+\cdots+x+2)-2^{n-1}\]
とおいた.$0\le x\le2$のとき,
     \begin{align*}
     h(x)&\le (2^{n-2}+2^{n-3}+\cdots+2+2)-2^{n-1} \\
     &=2^{n-1}-1+(1-2^{n-1})=0
     \end{align*}
であるから,\eqref{3}より,
     \begin{align*}
          \begin{cases}     
          g_n(x)\ge0&(0\le x\le1) \\
          g_n(x)\le0&(1\le x\le2)
          \end{cases}
     \end{align*}
従って,
      \begin{align*}
      S_1=\int_0^1g_n(x)dx&S_2=-\int_1^2g_n(x)dx
       \end{align*}   
であり,  
     \begin{align*}
     S_1=S_2\Longleftrightarrow \int_0^2g_n(x)dx=0 \\ 
     \Longleftrightarrow \left(\frac{1}{n+1}-\frac{1}{6}\right)2^{n+1}=\frac{4}{3}
     \end{align*}
をみたす$n$を求めればよい.$n\ge5$のとき,この左辺は$0$以下になり不適だから,$n=3,4$が必要.
実際に代入すると,適するのは$n=3$のときである.$\cdots$(答)     
\newpage
\end{multicols}
\end{document}