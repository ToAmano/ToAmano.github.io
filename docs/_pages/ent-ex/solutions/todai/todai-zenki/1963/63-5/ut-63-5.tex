\documentclass[a4j]{jarticle}
\usepackage{amsmath}
\usepackage{ascmac}
\usepackage{amssymb}
\usepackage{enumerate}
\usepackage{multicol}
\usepackage{framed}
\usepackage{fancyhdr}
\usepackage{latexsym}
\usepackage{indent}
\usepackage{cases}
\usepackage[dvips]{graphicx}
\usepackage{color}
\allowdisplaybreaks
\pagestyle{fancy}
\lhead{}
\chead{}
\rhead{東京大学前期$1963$年$5$番}
\begin{document}
%分数関係


\def\tfrac#1#2{{\textstyle\frac{#1}{#2}}} %数式中で文中表示の分数を使う時


%Σ関係

\def\dsum#1#2{{\displaystyle\sum_{#1}^{#2}}} %文中で数式表示のΣを使う時


%ベクトル関係


\def\vector#1{\overrightarrow{#1}}  %ベクトルを表現したいとき(aベクトルを表現するときは\ver
\def\norm#1{|\overrightarrow{#1}|} %ベクトルの絶対値
\def\vtwo#1#2{ \left(%
      \begin{array}{c}%
      #1 \\%
      #2 \\%
      \end{array}%
      \right) }                        %2次元ベクトル成分表示
      
      \def\vthree#1#2#3{ \left(
      \begin{array}{c}
      #1 \\
      #2 \\
      #3 \\
      \end{array}
      \right) }                        %3次元ベクトル成分表示



%数列関係


\def\an#1{\verb|{|$#1$\verb|}|}


%極限関係

\def\limit#1#2{\stackrel{#1 \to #2}{\longrightarrow}}   %等式変形からの極限
\def\dlim#1#2{{\displaystyle \lim_{#1\to#2}}} %文中で数式表示の極限を使う



%積分関係

\def\dint#1#2{{\displaystyle \int_{#1}^{#2}}} %文中で数式表示の積分を使う時

\def\ne{\nearrow}
\def\se{\searrow}
\def\nw{\nwarrow}
\def\ne{\nearrow}


%便利なやつ

\def\case#1#2{%
 \[\left\{%
 \begin{array}{l}%
 #1 \\%
 #2%
 \end{array}%
 \right.\] }                           %場合分け
 
\def\1{$\cos\theta=c$,$\sin\theta=s$とおく.}  %cs表示を与える前書きシータ
\def\2{$\cos t=c$,$\sin t=s$とおく.}     %cs表示を与える前書きt
\def\3{$\cos x=c$,$\sin x=s$とおく.}                %cs表示を与える前書きx

\def\fig#1#2#3 {%
\begin{wrapfigure}[#1]{r}{#2 zw}%
\vspace*{-1zh}%
\input{#3}%
\end{wrapfigure} }           %絵の挿入


\def\a{\alpha}   %ギリシャ文字
\def\b{\beta}
\def\g{\gamma}

%問題番号のためのマクロ

\newcounter{nombre} %必須
\renewcommand{\thenombre}{\arabic{nombre}} %任意
\setcounter{nombre}{2} %任意
\newcounter{nombresub}[nombre] %親子関係を定義
\renewcommand{\thenombresub}{\arabic{nombresub}} %任意
\setcounter{nombresub}{0} %任意
\newcommand{\prob}[1][]{\refstepcounter{nombre}#1[問題 \thenombre]}
\newcommand{\probsub}[1][]{\refstepcounter{nombresub}#1(\thenombresub)}


%1-1みたいなカウンタ(todaiとtodaia)
\newcounter{todai}
\setcounter{todai}{0}
\newcounter{todaisub}[todai] 
\setcounter{todaisub}{0} 
\newcommand{\todai}[1][]{\refstepcounter{todai}#1 \thetodai-\thetodaisub}
\newcommand{\todaib}[1][]{\refstepcounter{todai}#1\refstepcounter{todaisub}#1 {\bf [問題 \thetodai.\thetodaisub]}}
\newcommand{\todaia}[1][]{\refstepcounter{todaisub}#1 {\bf [問題 \thetodai.\thetodaisub]}}


     \begin{oframed}
     一辺の長さ$a$の正方形$ABCD$の内部の動点$P$で直交する折れ線$TPU$がある(図参照).$PT$は辺$AD$と$Q$で交わり,
     $\angle AQT$は$45^\circ$に保たれている.正方形$ABCD$の面積を二等分しつつ折れ線$TPU$が動く時,線分$PQ$の通過する
     部分の面積を求めよ.
          \begin{center}
          \scalebox{.7}{%WinTpicVersion4.32a
{\unitlength 0.1in%
\begin{picture}(20.0000,22.0000)(4.0000,-26.0000)%
% FUNC 2 0 3 0 Black White  
% 9 400 400 2400 2600 1400 1600 2000 1600 1400 1000 1100 400 2400 2600 0 4 0 0
% x+1
\special{pn 8}%
\special{pn 8}%
\special{pa 400 2000}%
\special{pa 406 1994}%
\special{ip}%
\special{pa 433 1967}%
\special{pa 439 1961}%
\special{ip}%
\special{pa 467 1933}%
\special{pa 473 1927}%
\special{ip}%
\special{pa 500 1900}%
\special{pa 506 1894}%
\special{ip}%
\special{pa 533 1867}%
\special{pa 539 1861}%
\special{ip}%
\special{pa 567 1833}%
\special{pa 573 1827}%
\special{ip}%
\special{pa 600 1800}%
\special{pa 606 1794}%
\special{ip}%
\special{pa 633 1767}%
\special{pa 639 1761}%
\special{ip}%
\special{pa 667 1733}%
\special{pa 673 1727}%
\special{ip}%
\special{pa 700 1700}%
\special{pa 706 1694}%
\special{ip}%
\special{pa 733 1667}%
\special{pa 739 1661}%
\special{ip}%
\special{pa 767 1633}%
\special{pa 773 1627}%
\special{ip}%
\special{pa 800 1600}%
\special{pa 806 1594}%
\special{ip}%
\special{pa 833 1567}%
\special{pa 839 1561}%
\special{ip}%
\special{pa 867 1533}%
\special{pa 873 1527}%
\special{ip}%
\special{pa 900 1500}%
\special{pa 906 1494}%
\special{ip}%
\special{pa 933 1467}%
\special{pa 939 1461}%
\special{ip}%
\special{pa 967 1433}%
\special{pa 973 1427}%
\special{ip}%
\special{pa 1000 1400}%
\special{pa 1006 1394}%
\special{ip}%
\special{pa 1033 1367}%
\special{pa 1039 1361}%
\special{ip}%
\special{pa 1067 1333}%
\special{pa 1073 1327}%
\special{ip}%
\special{pa 1100 1300}%
\special{pa 2000 400}%
\special{fp}%
% FUNC 2 0 3 0 Black White  
% 10 400 400 2400 2600 1400 1600 2000 1600 1400 1000 1100 400 2400 2600 0 4 0 0 0 0
% -x
\special{pn 8}%
\special{pn 8}%
\special{pa 400 600}%
\special{pa 406 606}%
\special{ip}%
\special{pa 433 633}%
\special{pa 439 639}%
\special{ip}%
\special{pa 467 667}%
\special{pa 473 673}%
\special{ip}%
\special{pa 500 700}%
\special{pa 506 706}%
\special{ip}%
\special{pa 533 733}%
\special{pa 539 739}%
\special{ip}%
\special{pa 567 767}%
\special{pa 573 773}%
\special{ip}%
\special{pa 600 800}%
\special{pa 606 806}%
\special{ip}%
\special{pa 633 833}%
\special{pa 639 839}%
\special{ip}%
\special{pa 667 867}%
\special{pa 673 873}%
\special{ip}%
\special{pa 700 900}%
\special{pa 706 906}%
\special{ip}%
\special{pa 733 933}%
\special{pa 739 939}%
\special{ip}%
\special{pa 767 967}%
\special{pa 773 973}%
\special{ip}%
\special{pa 800 1000}%
\special{pa 806 1006}%
\special{ip}%
\special{pa 833 1033}%
\special{pa 839 1039}%
\special{ip}%
\special{pa 867 1067}%
\special{pa 873 1073}%
\special{ip}%
\special{pa 900 1100}%
\special{pa 906 1106}%
\special{ip}%
\special{pa 933 1133}%
\special{pa 939 1139}%
\special{ip}%
\special{pa 967 1167}%
\special{pa 973 1173}%
\special{ip}%
\special{pa 1000 1200}%
\special{pa 1006 1206}%
\special{ip}%
\special{pa 1033 1233}%
\special{pa 1039 1239}%
\special{ip}%
\special{pa 1067 1267}%
\special{pa 1073 1273}%
\special{ip}%
\special{pa 1100 1300}%
\special{pa 2400 2600}%
\special{fp}%
% FUNC 2 0 3 0 Black White  
% 10 400 400 2400 2600 1400 1600 2000 1600 1400 1000 400 1000 2400 2200 0 4 0 1 0 0
% 1
\special{pn 8}%
\special{pn 8}%
\special{pa 2000 400}%
\special{pa 2000 408}%
\special{ip}%
\special{pa 2000 446}%
\special{pa 2000 454}%
\special{ip}%
\special{pa 2000 492}%
\special{pa 2000 501}%
\special{ip}%
\special{pa 2000 538}%
\special{pa 2000 547}%
\special{ip}%
\special{pa 2000 585}%
\special{pa 2000 593}%
\special{ip}%
\special{pa 2000 631}%
\special{pa 2000 639}%
\special{ip}%
\special{pa 2000 677}%
\special{pa 2000 685}%
\special{ip}%
\special{pa 2000 723}%
\special{pa 2000 731}%
\special{ip}%
\special{pa 2000 769}%
\special{pa 2000 777}%
\special{ip}%
\special{pa 2000 815}%
\special{pa 2000 824}%
\special{ip}%
\special{pa 2000 862}%
\special{pa 2000 870}%
\special{ip}%
\special{pa 2000 908}%
\special{pa 2000 916}%
\special{ip}%
\special{pa 2000 954}%
\special{pa 2000 962}%
\special{ip}%
\special{pa 2000 1000}%
\special{pa 2000 1000}%
\special{ip}%
\special{pa 2000 1000}%
\special{pa 2000 2200}%
\special{fp}%
\special{pn 8}%
\special{pa 2000 2209}%
\special{pa 2000 2250}%
\special{ip}%
\special{pa 2000 2259}%
\special{pa 2000 2300}%
\special{ip}%
\special{pa 2000 2309}%
\special{pa 2000 2350}%
\special{ip}%
\special{pa 2000 2359}%
\special{pa 2000 2400}%
\special{ip}%
\special{pa 2000 2409}%
\special{pa 2000 2450}%
\special{ip}%
\special{pa 2000 2459}%
\special{pa 2000 2500}%
\special{ip}%
\special{pa 2000 2509}%
\special{pa 2000 2550}%
\special{ip}%
\special{pa 2000 2559}%
\special{pa 2000 2600}%
\special{ip}%
% FUNC 2 0 3 0 Black White  
% 10 400 400 2400 2600 1400 1600 2000 1600 1400 1000 400 1000 2400 2200 0 4 0 1 0 0
% -1
\special{pn 8}%
\special{pn 8}%
\special{pa 800 400}%
\special{pa 800 408}%
\special{ip}%
\special{pa 800 446}%
\special{pa 800 454}%
\special{ip}%
\special{pa 800 492}%
\special{pa 800 501}%
\special{ip}%
\special{pa 800 538}%
\special{pa 800 547}%
\special{ip}%
\special{pa 800 585}%
\special{pa 800 593}%
\special{ip}%
\special{pa 800 631}%
\special{pa 800 639}%
\special{ip}%
\special{pa 800 677}%
\special{pa 800 685}%
\special{ip}%
\special{pa 800 723}%
\special{pa 800 731}%
\special{ip}%
\special{pa 800 769}%
\special{pa 800 777}%
\special{ip}%
\special{pa 800 815}%
\special{pa 800 824}%
\special{ip}%
\special{pa 800 862}%
\special{pa 800 870}%
\special{ip}%
\special{pa 800 908}%
\special{pa 800 916}%
\special{ip}%
\special{pa 800 954}%
\special{pa 800 962}%
\special{ip}%
\special{pa 800 1000}%
\special{pa 800 1000}%
\special{ip}%
\special{pa 800 1000}%
\special{pa 800 2200}%
\special{fp}%
\special{pn 8}%
\special{pa 800 2209}%
\special{pa 800 2250}%
\special{ip}%
\special{pa 800 2259}%
\special{pa 800 2300}%
\special{ip}%
\special{pa 800 2309}%
\special{pa 800 2350}%
\special{ip}%
\special{pa 800 2359}%
\special{pa 800 2400}%
\special{ip}%
\special{pa 800 2409}%
\special{pa 800 2450}%
\special{ip}%
\special{pa 800 2459}%
\special{pa 800 2500}%
\special{ip}%
\special{pa 800 2509}%
\special{pa 800 2550}%
\special{ip}%
\special{pa 800 2559}%
\special{pa 800 2600}%
\special{ip}%
% FUNC 2 0 3 0 Black White  
% 10 400 400 2400 2600 1400 1600 2000 1600 1400 1000 800 400 2000 2600 0 4 0 0 0 0
% 1
\special{pn 8}%
\special{pn 8}%
\special{pa 400 1000}%
\special{pa 409 1000}%
\special{ip}%
\special{pa 450 1000}%
\special{pa 459 1000}%
\special{ip}%
\special{pa 500 1000}%
\special{pa 509 1000}%
\special{ip}%
\special{pa 550 1000}%
\special{pa 559 1000}%
\special{ip}%
\special{pa 600 1000}%
\special{pa 609 1000}%
\special{ip}%
\special{pa 650 1000}%
\special{pa 659 1000}%
\special{ip}%
\special{pa 700 1000}%
\special{pa 709 1000}%
\special{ip}%
\special{pa 750 1000}%
\special{pa 759 1000}%
\special{ip}%
\special{pa 800 1000}%
\special{pa 800 1000}%
\special{ip}%
\special{pa 800 1000}%
\special{pa 2000 1000}%
\special{fp}%
\special{pn 8}%
\special{pa 2009 1000}%
\special{pa 2050 1000}%
\special{ip}%
\special{pa 2059 1000}%
\special{pa 2100 1000}%
\special{ip}%
\special{pa 2109 1000}%
\special{pa 2150 1000}%
\special{ip}%
\special{pa 2159 1000}%
\special{pa 2200 1000}%
\special{ip}%
\special{pa 2209 1000}%
\special{pa 2250 1000}%
\special{ip}%
\special{pa 2259 1000}%
\special{pa 2300 1000}%
\special{ip}%
\special{pa 2309 1000}%
\special{pa 2350 1000}%
\special{ip}%
\special{pa 2359 1000}%
\special{pa 2400 1000}%
\special{ip}%
% FUNC 2 0 3 0 Black White  
% 10 400 400 2400 2600 1400 1600 2000 1600 1400 1000 800 400 2000 2600 0 4 0 0 0 0
% -1
\special{pn 8}%
\special{pn 8}%
\special{pa 400 2200}%
\special{pa 409 2200}%
\special{ip}%
\special{pa 450 2200}%
\special{pa 459 2200}%
\special{ip}%
\special{pa 500 2200}%
\special{pa 509 2200}%
\special{ip}%
\special{pa 550 2200}%
\special{pa 559 2200}%
\special{ip}%
\special{pa 600 2200}%
\special{pa 609 2200}%
\special{ip}%
\special{pa 650 2200}%
\special{pa 659 2200}%
\special{ip}%
\special{pa 700 2200}%
\special{pa 709 2200}%
\special{ip}%
\special{pa 750 2200}%
\special{pa 759 2200}%
\special{ip}%
\special{pa 800 2200}%
\special{pa 800 2200}%
\special{ip}%
\special{pa 800 2200}%
\special{pa 2000 2200}%
\special{fp}%
\special{pn 8}%
\special{pa 2009 2200}%
\special{pa 2050 2200}%
\special{ip}%
\special{pa 2059 2200}%
\special{pa 2100 2200}%
\special{ip}%
\special{pa 2109 2200}%
\special{pa 2150 2200}%
\special{ip}%
\special{pa 2159 2200}%
\special{pa 2200 2200}%
\special{ip}%
\special{pa 2209 2200}%
\special{pa 2250 2200}%
\special{ip}%
\special{pa 2259 2200}%
\special{pa 2300 2200}%
\special{ip}%
\special{pa 2309 2200}%
\special{pa 2350 2200}%
\special{ip}%
\special{pa 2359 2200}%
\special{pa 2400 2200}%
\special{ip}%
% STR 2 0 3 0 Black White  
% 4 800 900 800 1000 2 0 1 0
% $D$
\put(8.0000,-10.0000){\makebox(0,0)[lb]{{\colorbox[named]{White}{$D$}}}}%
% STR 2 0 3 0 Black White  
% 4 2000 900 2000 1000 2 0 1 0
% $A$
\put(20.0000,-10.0000){\makebox(0,0)[lb]{{\colorbox[named]{White}{$A$}}}}%
% STR 2 0 3 0 Black White  
% 4 2000 2100 2000 2200 2 0 1 0
% $B$
\put(20.0000,-22.0000){\makebox(0,0)[lb]{{\colorbox[named]{White}{$B$}}}}%
% STR 2 0 3 0 Black White  
% 4 800 2100 800 2200 2 0 1 0
% $C$
\put(8.0000,-22.0000){\makebox(0,0)[lb]{{\colorbox[named]{White}{$C$}}}}%
% STR 2 0 3 0 Black White  
% 4 1400 900 1400 1000 2 0 1 0
% $Q$
\put(14.0000,-10.0000){\makebox(0,0)[lb]{{\colorbox[named]{White}{$Q$}}}}%
% STR 2 0 3 0 Black White  
% 4 1800 500 1800 600 2 0 1 0
% $T$
\put(18.0000,-6.0000){\makebox(0,0)[lb]{{\colorbox[named]{White}{$T$}}}}%
% STR 2 0 3 0 Black White  
% 4 2200 2300 2200 2400 2 0 1 0
% $U$
\put(22.0000,-24.0000){\makebox(0,0)[lb]{{\colorbox[named]{White}{$U$}}}}%
\end{picture}}%
}
          \end{center}
     \end{oframed}
\setlength{\columnseprule}{0.4pt}
\begin{multicols}{2}
{\bf[解]} まず,一辺の長さ$\sqrt{2}$として考える.\label{1}

このとき,$A(1,0)$,$B(0,-1)$,$C(-1,0)$となるような座標を定める.すると,$AD:x+y=1$であるから,$Q(1-t,t)$と置ける.
($0\le le1$)すると,$P(x,t)$である.
     \begin{center}
     \scalebox{1}{%WinTpicVersion4.32a
{\unitlength 0.1in%
\begin{picture}(20.0000,20.0000)(4.0000,-24.0000)%
% STR 2 0 3 0 Black White  
% 4 1390 1397 1390 1410 4 400 0 0
% O
\put(13.9000,-14.1000){\makebox(0,0)[rt]{O}}%
% STR 2 0 3 0 Black White  
% 4 1360 387 1360 400 4 400 0 0
% $y$
\put(13.6000,-4.0000){\makebox(0,0)[rt]{$y$}}%
% STR 2 0 3 0 Black White  
% 4 2400 1427 2400 1440 4 400 0 0
% $x$
\put(24.0000,-14.4000){\makebox(0,0)[rt]{$x$}}%
% VECTOR 2 0 3 0 Black White  
% 2 1400 2400 1400 400
% 
\special{pn 8}%
\special{pa 1400 2400}%
\special{pa 1400 400}%
\special{fp}%
\special{sh 1}%
\special{pa 1400 400}%
\special{pa 1380 467}%
\special{pa 1400 453}%
\special{pa 1420 467}%
\special{pa 1400 400}%
\special{fp}%
% VECTOR 2 0 3 0 Black White  
% 2 400 1400 2400 1400
% 
\special{pn 8}%
\special{pa 400 1400}%
\special{pa 2400 1400}%
\special{fp}%
\special{sh 1}%
\special{pa 2400 1400}%
\special{pa 2333 1380}%
\special{pa 2347 1400}%
\special{pa 2333 1420}%
\special{pa 2400 1400}%
\special{fp}%
% FUNC 2 0 3 0 Black White  
% 9 400 400 2400 2400 1400 1400 2200 1400 1400 600 1400 400 2200 2400 0 4 0 0
% x-1
\special{pn 8}%
\special{pn 8}%
\special{pa 1200 2400}%
\special{pa 1206 2394}%
\special{ip}%
\special{pa 1233 2367}%
\special{pa 1239 2361}%
\special{ip}%
\special{pa 1267 2333}%
\special{pa 1273 2327}%
\special{ip}%
\special{pa 1300 2300}%
\special{pa 1306 2294}%
\special{ip}%
\special{pa 1333 2267}%
\special{pa 1339 2261}%
\special{ip}%
\special{pa 1367 2233}%
\special{pa 1373 2227}%
\special{ip}%
\special{ip}%
\special{pa 1400 2200}%
\special{pa 2200 1400}%
\special{fp}%
\special{pn 8}%
\special{pa 2206 1394}%
\special{pa 2233 1367}%
\special{ip}%
\special{pa 2239 1361}%
\special{pa 2267 1333}%
\special{ip}%
\special{pa 2273 1327}%
\special{pa 2300 1300}%
\special{ip}%
\special{pa 2306 1294}%
\special{pa 2333 1267}%
\special{ip}%
\special{pa 2339 1261}%
\special{pa 2367 1233}%
\special{ip}%
\special{pa 2373 1227}%
\special{pa 2400 1200}%
\special{ip}%
% FUNC 2 0 3 0 Black White  
% 10 400 400 2400 2400 1400 1400 2200 1400 1400 600 1400 400 2200 2400 0 4 0 0 0 0
% -x+1
\special{pn 8}%
\special{pn 8}%
\special{pa 1200 400}%
\special{pa 1206 406}%
\special{ip}%
\special{pa 1233 433}%
\special{pa 1239 439}%
\special{ip}%
\special{pa 1267 467}%
\special{pa 1273 473}%
\special{ip}%
\special{pa 1300 500}%
\special{pa 1306 506}%
\special{ip}%
\special{pa 1333 533}%
\special{pa 1339 539}%
\special{ip}%
\special{pa 1367 567}%
\special{pa 1373 573}%
\special{ip}%
\special{ip}%
\special{pa 1400 600}%
\special{pa 2200 1400}%
\special{fp}%
\special{pn 8}%
\special{pa 2206 1406}%
\special{pa 2233 1433}%
\special{ip}%
\special{pa 2239 1439}%
\special{pa 2267 1467}%
\special{ip}%
\special{pa 2273 1473}%
\special{pa 2300 1500}%
\special{ip}%
\special{pa 2306 1506}%
\special{pa 2333 1533}%
\special{ip}%
\special{pa 2339 1539}%
\special{pa 2367 1567}%
\special{ip}%
\special{pa 2373 1573}%
\special{pa 2400 1600}%
\special{ip}%
% FUNC 2 0 3 0 Black White  
% 10 400 400 2400 2400 1400 1400 2200 1400 1400 600 600 400 1400 2400 0 4 0 0 0 0
% x+1
\special{pn 8}%
\special{pn 8}%
\special{pa 400 1600}%
\special{pa 406 1594}%
\special{ip}%
\special{pa 433 1567}%
\special{pa 439 1561}%
\special{ip}%
\special{pa 467 1533}%
\special{pa 473 1527}%
\special{ip}%
\special{pa 500 1500}%
\special{pa 506 1494}%
\special{ip}%
\special{pa 533 1467}%
\special{pa 539 1461}%
\special{ip}%
\special{pa 567 1433}%
\special{pa 573 1427}%
\special{ip}%
\special{ip}%
\special{pa 600 1400}%
\special{pa 1400 600}%
\special{fp}%
\special{pn 8}%
\special{pa 1406 594}%
\special{pa 1433 567}%
\special{ip}%
\special{pa 1439 561}%
\special{pa 1467 533}%
\special{ip}%
\special{pa 1473 527}%
\special{pa 1500 500}%
\special{ip}%
\special{pa 1506 494}%
\special{pa 1533 467}%
\special{ip}%
\special{pa 1539 461}%
\special{pa 1567 433}%
\special{ip}%
\special{pa 1573 427}%
\special{pa 1600 400}%
\special{ip}%
% FUNC 2 0 3 0 Black White  
% 10 400 400 2400 2400 1400 1400 2200 1400 1400 600 600 400 1400 2400 0 4 0 0 0 0
% -x-1
\special{pn 8}%
\special{pn 8}%
\special{pa 400 1200}%
\special{pa 406 1206}%
\special{ip}%
\special{pa 433 1233}%
\special{pa 439 1239}%
\special{ip}%
\special{pa 467 1267}%
\special{pa 473 1273}%
\special{ip}%
\special{pa 500 1300}%
\special{pa 506 1306}%
\special{ip}%
\special{pa 533 1333}%
\special{pa 539 1339}%
\special{ip}%
\special{pa 567 1367}%
\special{pa 573 1373}%
\special{ip}%
\special{ip}%
\special{pa 600 1400}%
\special{pa 1400 2200}%
\special{fp}%
\special{pn 8}%
\special{pa 1406 2206}%
\special{pa 1433 2233}%
\special{ip}%
\special{pa 1439 2239}%
\special{pa 1467 2267}%
\special{ip}%
\special{pa 1473 2273}%
\special{pa 1500 2300}%
\special{ip}%
\special{pa 1506 2306}%
\special{pa 1533 2333}%
\special{ip}%
\special{pa 1539 2339}%
\special{pa 1567 2367}%
\special{ip}%
\special{pa 1573 2373}%
\special{pa 1600 2400}%
\special{ip}%
% LINE 2 2 3 0 Black White  
% 4 2370 1080 1170 1080 1170 1080 1170 2400
% 
\special{pn 8}%
\special{pa 2370 1080}%
\special{pa 1170 1080}%
\special{dt 0.045}%
\special{pa 1170 1080}%
\special{pa 1170 2400}%
\special{dt 0.045}%
% STR 2 0 3 0 Black White  
% 4 2200 1300 2200 1400 2 0 1 0
% $A$
\put(22.0000,-14.0000){\makebox(0,0)[lb]{{\colorbox[named]{White}{$A$}}}}%
% STR 2 0 3 0 Black White  
% 4 1400 500 1400 600 2 0 1 0
% $D$
\put(14.0000,-6.0000){\makebox(0,0)[lb]{{\colorbox[named]{White}{$D$}}}}%
% STR 2 0 3 0 Black White  
% 4 600 1300 600 1400 2 0 1 0
% $C$
\put(6.0000,-14.0000){\makebox(0,0)[lb]{{\colorbox[named]{White}{$C$}}}}%
% STR 2 0 3 0 Black White  
% 4 1170 980 1170 1080 2 0 1 0
% $P$
\put(11.7000,-10.8000){\makebox(0,0)[lb]{{\colorbox[named]{White}{$P$}}}}%
% STR 2 0 3 0 Black White  
% 4 1880 980 1880 1080 2 0 1 0
% $Q$
\put(18.8000,-10.8000){\makebox(0,0)[lb]{{\colorbox[named]{White}{$Q$}}}}%
% LINE 3 0 3 0 Black White  
% 24 1840 1400 1400 1840 1900 1400 1400 1900 1960 1400 1400 1960 2020 1400 1400 2020 2080 1400 1400 2080 2140 1400 1400 2140 1780 1400 1400 1780 1720 1400 1400 1720 1660 1400 1400 1660 1600 1400 1400 1600 1540 1400 1400 1540 1480 1400 1400 1480
% 
\special{pn 4}%
\special{pa 1840 1400}%
\special{pa 1400 1840}%
\special{fp}%
\special{pa 1900 1400}%
\special{pa 1400 1900}%
\special{fp}%
\special{pa 1960 1400}%
\special{pa 1400 1960}%
\special{fp}%
\special{pa 2020 1400}%
\special{pa 1400 2020}%
\special{fp}%
\special{pa 2080 1400}%
\special{pa 1400 2080}%
\special{fp}%
\special{pa 2140 1400}%
\special{pa 1400 2140}%
\special{fp}%
\special{pa 1780 1400}%
\special{pa 1400 1780}%
\special{fp}%
\special{pa 1720 1400}%
\special{pa 1400 1720}%
\special{fp}%
\special{pa 1660 1400}%
\special{pa 1400 1660}%
\special{fp}%
\special{pa 1600 1400}%
\special{pa 1400 1600}%
\special{fp}%
\special{pa 1540 1400}%
\special{pa 1400 1540}%
\special{fp}%
\special{pa 1480 1400}%
\special{pa 1400 1480}%
\special{fp}%
% LINE 3 0 3 0 Black White  
% 36 1800 1080 1480 1400 1740 1080 1420 1400 1680 1080 1400 1360 1620 1080 1400 1300 1560 1080 1400 1240 1500 1080 1400 1180 1440 1080 1400 1120 1860 1080 1540 1400 1900 1100 1600 1400 1930 1130 1660 1400 1960 1160 1720 1400 1990 1190 1780 1400 2020 1220 1840 1400 2050 1250 1900 1400 2080 1280 1960 1400 2110 1310 2020 1400 2140 1340 2080 1400 2170 1370 2140 1400
% 
\special{pn 4}%
\special{pa 1800 1080}%
\special{pa 1480 1400}%
\special{fp}%
\special{pa 1740 1080}%
\special{pa 1420 1400}%
\special{fp}%
\special{pa 1680 1080}%
\special{pa 1400 1360}%
\special{fp}%
\special{pa 1620 1080}%
\special{pa 1400 1300}%
\special{fp}%
\special{pa 1560 1080}%
\special{pa 1400 1240}%
\special{fp}%
\special{pa 1500 1080}%
\special{pa 1400 1180}%
\special{fp}%
\special{pa 1440 1080}%
\special{pa 1400 1120}%
\special{fp}%
\special{pa 1860 1080}%
\special{pa 1540 1400}%
\special{fp}%
\special{pa 1900 1100}%
\special{pa 1600 1400}%
\special{fp}%
\special{pa 1930 1130}%
\special{pa 1660 1400}%
\special{fp}%
\special{pa 1960 1160}%
\special{pa 1720 1400}%
\special{fp}%
\special{pa 1990 1190}%
\special{pa 1780 1400}%
\special{fp}%
\special{pa 2020 1220}%
\special{pa 1840 1400}%
\special{fp}%
\special{pa 2050 1250}%
\special{pa 1900 1400}%
\special{fp}%
\special{pa 2080 1280}%
\special{pa 1960 1400}%
\special{fp}%
\special{pa 2110 1310}%
\special{pa 2020 1400}%
\special{fp}%
\special{pa 2140 1340}%
\special{pa 2080 1400}%
\special{fp}%
\special{pa 2170 1370}%
\special{pa 2140 1400}%
\special{fp}%
% LINE 3 0 3 0 Black White  
% 32 1400 1600 1170 1830 1400 1660 1170 1890 1400 1720 1170 1950 1400 1780 1190 1990 1400 1840 1220 2020 1400 1900 1250 2050 1400 1960 1280 2080 1400 2020 1310 2110 1400 2080 1340 2140 1400 2140 1370 2170 1400 1540 1170 1770 1400 1480 1170 1710 1400 1420 1170 1650 1360 1400 1170 1590 1300 1400 1170 1530 1240 1400 1170 1470
% 
\special{pn 4}%
\special{pa 1400 1600}%
\special{pa 1170 1830}%
\special{fp}%
\special{pa 1400 1660}%
\special{pa 1170 1890}%
\special{fp}%
\special{pa 1400 1720}%
\special{pa 1170 1950}%
\special{fp}%
\special{pa 1400 1780}%
\special{pa 1190 1990}%
\special{fp}%
\special{pa 1400 1840}%
\special{pa 1220 2020}%
\special{fp}%
\special{pa 1400 1900}%
\special{pa 1250 2050}%
\special{fp}%
\special{pa 1400 1960}%
\special{pa 1280 2080}%
\special{fp}%
\special{pa 1400 2020}%
\special{pa 1310 2110}%
\special{fp}%
\special{pa 1400 2080}%
\special{pa 1340 2140}%
\special{fp}%
\special{pa 1400 2140}%
\special{pa 1370 2170}%
\special{fp}%
\special{pa 1400 1540}%
\special{pa 1170 1770}%
\special{fp}%
\special{pa 1400 1480}%
\special{pa 1170 1710}%
\special{fp}%
\special{pa 1400 1420}%
\special{pa 1170 1650}%
\special{fp}%
\special{pa 1360 1400}%
\special{pa 1170 1590}%
\special{fp}%
\special{pa 1300 1400}%
\special{pa 1170 1530}%
\special{fp}%
\special{pa 1240 1400}%
\special{pa 1170 1470}%
\special{fp}%
% LINE 3 0 3 0 Black White  
% 18 1400 1180 1180 1400 1400 1120 1170 1350 1380 1080 1170 1290 1320 1080 1170 1230 1260 1080 1170 1170 1200 1080 1170 1110 1400 1240 1240 1400 1400 1300 1300 1400 1400 1360 1360 1400
% 
\special{pn 4}%
\special{pa 1400 1180}%
\special{pa 1180 1400}%
\special{fp}%
\special{pa 1400 1120}%
\special{pa 1170 1350}%
\special{fp}%
\special{pa 1380 1080}%
\special{pa 1170 1290}%
\special{fp}%
\special{pa 1320 1080}%
\special{pa 1170 1230}%
\special{fp}%
\special{pa 1260 1080}%
\special{pa 1170 1170}%
\special{fp}%
\special{pa 1200 1080}%
\special{pa 1170 1110}%
\special{fp}%
\special{pa 1400 1240}%
\special{pa 1240 1400}%
\special{fp}%
\special{pa 1400 1300}%
\special{pa 1300 1400}%
\special{fp}%
\special{pa 1400 1360}%
\special{pa 1360 1400}%
\special{fp}%
% STR 2 0 3 0 Black White  
% 4 1170 1880 1170 1980 2 0 1 0
% $R$
\put(11.7000,-19.8000){\makebox(0,0)[lb]{{\colorbox[named]{White}{$R$}}}}%
% STR 2 0 3 0 Black White  
% 4 1400 2100 1400 2200 2 0 1 0
% $B$
\put(14.0000,-22.0000){\makebox(0,0)[lb]{{\colorbox[named]{White}{$B$}}}}%
\end{picture}}%
}
     \end{center}
$0\le x$の時,明らかに折れ線は正方形の面積を二等分しないから
     \begin{align}
     t-1\le x<0\label{1}
     \end{align}
である.この時,折れ線と$BC$の交点は$R(x,-x-1)$である.$PR$と$x$軸の交点$T$とすれば,題意の面積についての条件から,
     \begin{align*}
     &\triangle CRT=\square AQPT \\
     &\frac{1}{2}(1+s)^2=\frac{1}{2}\left((1-t-x)+(1-x)\right) \\
     &=\frac{1}{2}t(2-2x-t) \\
     &x^2+2x+1=2t-2xt-t^2 \\
     &x^2+t^2+2xt+2x-2t+1=0 \\
     &x^2+2(t+1)x+(t-1)^2=0 \\
     &x=-(t+1)\pm\sqrt{(t+1)^2-(t-1)^2} \\
     &x=-(t+1)\pm2\sqrt{t} \\
     \end{align*}
$P$は$(0,1)$を通るので,複合正をとって
     \begin{align*}
     x=-t-1+2\sqrt{t}=-(\sqrt{t}-1)^2
     \end{align*}
である.よってグラフの概形は下図.
    \begin{center}
    \scalebox{1}{%WinTpicVersion4.32a
{\unitlength 0.1in%
\begin{picture}(20.0000,10.4000)(4.0000,-14.4000)%
% STR 2 0 3 0 Black White  
% 4 1390 1397 1390 1410 4 400 0 0
% O
\put(13.9000,-14.1000){\makebox(0,0)[rt]{O}}%
% STR 2 0 3 0 Black White  
% 4 1360 387 1360 400 4 400 0 0
% $y$
\put(13.6000,-4.0000){\makebox(0,0)[rt]{$y$}}%
% STR 2 0 3 0 Black White  
% 4 2400 1427 2400 1440 4 400 0 0
% $x$
\put(24.0000,-14.4000){\makebox(0,0)[rt]{$x$}}%
% VECTOR 2 0 3 0 Black White  
% 2 1400 1400 1400 400
% 
\special{pn 8}%
\special{pa 1400 1400}%
\special{pa 1400 400}%
\special{fp}%
\special{sh 1}%
\special{pa 1400 400}%
\special{pa 1380 467}%
\special{pa 1400 453}%
\special{pa 1420 467}%
\special{pa 1400 400}%
\special{fp}%
% VECTOR 2 0 3 0 Black White  
% 2 400 1400 2400 1400
% 
\special{pn 8}%
\special{pa 400 1400}%
\special{pa 2400 1400}%
\special{fp}%
\special{sh 1}%
\special{pa 2400 1400}%
\special{pa 2333 1380}%
\special{pa 2347 1400}%
\special{pa 2333 1420}%
\special{pa 2400 1400}%
\special{fp}%
% FUNC 2 0 3 0 Black White  
% 9 400 400 2400 1400 1400 1400 2200 1400 1400 600 1400 400 2200 1400 0 4 0 0
% -x+1
\special{pn 8}%
\special{pn 8}%
\special{pa 1200 400}%
\special{pa 1206 406}%
\special{ip}%
\special{pa 1233 433}%
\special{pa 1239 439}%
\special{ip}%
\special{pa 1267 467}%
\special{pa 1273 473}%
\special{ip}%
\special{pa 1300 500}%
\special{pa 1306 506}%
\special{ip}%
\special{pa 1333 533}%
\special{pa 1339 539}%
\special{ip}%
\special{pa 1367 567}%
\special{pa 1373 573}%
\special{ip}%
\special{ip}%
\special{pa 1400 600}%
\special{pa 2200 1400}%
\special{fp}%
\special{pn 8}%
% FUNC 2 0 3 0 Black White  
% 10 400 400 2400 1400 1400 1400 2200 1400 1400 600 600 400 1400 1400 0 4 0 0 0 0
% x+1
\special{pn 8}%
\special{pn 8}%
\special{pa 600 1400}%
\special{pa 600 1400}%
\special{ip}%
\special{pa 600 1400}%
\special{pa 1400 600}%
\special{fp}%
\special{pn 8}%
\special{pa 1406 594}%
\special{pa 1433 567}%
\special{ip}%
\special{pa 1439 561}%
\special{pa 1467 533}%
\special{ip}%
\special{pa 1473 527}%
\special{pa 1500 500}%
\special{ip}%
\special{pa 1506 494}%
\special{pa 1533 467}%
\special{ip}%
\special{pa 1539 461}%
\special{pa 1567 433}%
\special{ip}%
\special{pa 1573 427}%
\special{pa 1600 400}%
\special{ip}%
% FUNC 2 0 3 0 Black White  
% 10 400 400 2400 1400 1400 1400 2200 1400 1400 600 400 600 2400 1400 0 4 0 1 0 0
% -y-1+2sqrt(y)
\special{pn 8}%
\special{pn 8}%
\special{pa 1389 400}%
\special{pa 1389 405}%
\special{pa 1390 409}%
\special{ip}%
\special{pa 1394 450}%
\special{pa 1394 450}%
\special{pa 1394 459}%
\special{ip}%
\special{pa 1397 500}%
\special{pa 1397 505}%
\special{pa 1398 509}%
\special{ip}%
\special{pa 1399 550}%
\special{pa 1399 555}%
\special{pa 1400 559}%
\special{ip}%
\special{pa 1400 600}%
\special{pa 1400 635}%
\special{pa 1399 640}%
\special{pa 1399 665}%
\special{pa 1398 670}%
\special{pa 1398 685}%
\special{pa 1397 690}%
\special{pa 1397 700}%
\special{pa 1396 705}%
\special{pa 1396 715}%
\special{pa 1395 720}%
\special{pa 1395 725}%
\special{pa 1394 730}%
\special{pa 1394 735}%
\special{pa 1393 740}%
\special{pa 1393 745}%
\special{pa 1392 750}%
\special{pa 1392 755}%
\special{pa 1390 765}%
\special{pa 1390 770}%
\special{pa 1389 775}%
\special{pa 1389 780}%
\special{pa 1386 795}%
\special{pa 1386 800}%
\special{pa 1381 825}%
\special{pa 1381 830}%
\special{pa 1375 860}%
\special{pa 1373 865}%
\special{pa 1369 885}%
\special{pa 1367 890}%
\special{pa 1364 905}%
\special{pa 1362 910}%
\special{pa 1361 915}%
\special{pa 1359 920}%
\special{pa 1358 925}%
\special{pa 1356 930}%
\special{pa 1355 935}%
\special{pa 1353 940}%
\special{pa 1352 945}%
\special{pa 1348 955}%
\special{pa 1347 960}%
\special{pa 1323 1020}%
\special{pa 1320 1025}%
\special{pa 1316 1035}%
\special{pa 1313 1040}%
\special{pa 1311 1045}%
\special{pa 1308 1050}%
\special{pa 1306 1055}%
\special{pa 1300 1065}%
\special{pa 1298 1070}%
\special{pa 1277 1105}%
\special{pa 1273 1110}%
\special{pa 1267 1120}%
\special{pa 1263 1125}%
\special{pa 1260 1130}%
\special{pa 1224 1175}%
\special{pa 1214 1185}%
\special{pa 1210 1190}%
\special{pa 1190 1210}%
\special{pa 1184 1215}%
\special{pa 1179 1220}%
\special{pa 1173 1225}%
\special{pa 1168 1230}%
\special{pa 1156 1240}%
\special{pa 1149 1245}%
\special{pa 1143 1250}%
\special{pa 1115 1270}%
\special{pa 1107 1275}%
\special{pa 1100 1280}%
\special{pa 1092 1285}%
\special{pa 1083 1290}%
\special{pa 1075 1295}%
\special{pa 1066 1300}%
\special{pa 1056 1305}%
\special{pa 1047 1310}%
\special{pa 1037 1315}%
\special{pa 1015 1325}%
\special{pa 991 1335}%
\special{pa 965 1345}%
\special{pa 950 1350}%
\special{pa 918 1360}%
\special{pa 900 1365}%
\special{pa 880 1370}%
\special{pa 858 1375}%
\special{pa 833 1380}%
\special{pa 804 1385}%
\special{pa 769 1390}%
\special{pa 721 1395}%
\special{pa 600 1400}%
\special{fp}%
% LINE 3 0 3 0 Black White  
% 52 1900 1100 1600 1400 1870 1070 1540 1400 1840 1040 1480 1400 1810 1010 1420 1400 1780 980 1400 1360 1750 950 1400 1300 1720 920 1400 1240 1690 890 1400 1180 1660 860 1400 1120 1630 830 1400 1060 1600 800 1400 1000 1570 770 1400 940 1540 740 1400 880 1510 710 1400 820 1480 680 1400 760 1450 650 1400 700 1420 620 1400 640 1930 1130 1660 1400 1960 1160 1720 1400 1990 1190 1780 1400 2020 1220 1840 1400 2050 1250 1900 1400 2080 1280 1960 1400 2110 1310 2020 1400 2140 1340 2080 1400 2170 1370 2140 1400
% 
\special{pn 4}%
\special{pa 1900 1100}%
\special{pa 1600 1400}%
\special{fp}%
\special{pa 1870 1070}%
\special{pa 1540 1400}%
\special{fp}%
\special{pa 1840 1040}%
\special{pa 1480 1400}%
\special{fp}%
\special{pa 1810 1010}%
\special{pa 1420 1400}%
\special{fp}%
\special{pa 1780 980}%
\special{pa 1400 1360}%
\special{fp}%
\special{pa 1750 950}%
\special{pa 1400 1300}%
\special{fp}%
\special{pa 1720 920}%
\special{pa 1400 1240}%
\special{fp}%
\special{pa 1690 890}%
\special{pa 1400 1180}%
\special{fp}%
\special{pa 1660 860}%
\special{pa 1400 1120}%
\special{fp}%
\special{pa 1630 830}%
\special{pa 1400 1060}%
\special{fp}%
\special{pa 1600 800}%
\special{pa 1400 1000}%
\special{fp}%
\special{pa 1570 770}%
\special{pa 1400 940}%
\special{fp}%
\special{pa 1540 740}%
\special{pa 1400 880}%
\special{fp}%
\special{pa 1510 710}%
\special{pa 1400 820}%
\special{fp}%
\special{pa 1480 680}%
\special{pa 1400 760}%
\special{fp}%
\special{pa 1450 650}%
\special{pa 1400 700}%
\special{fp}%
\special{pa 1420 620}%
\special{pa 1400 640}%
\special{fp}%
\special{pa 1930 1130}%
\special{pa 1660 1400}%
\special{fp}%
\special{pa 1960 1160}%
\special{pa 1720 1400}%
\special{fp}%
\special{pa 1990 1190}%
\special{pa 1780 1400}%
\special{fp}%
\special{pa 2020 1220}%
\special{pa 1840 1400}%
\special{fp}%
\special{pa 2050 1250}%
\special{pa 1900 1400}%
\special{fp}%
\special{pa 2080 1280}%
\special{pa 1960 1400}%
\special{fp}%
\special{pa 2110 1310}%
\special{pa 2020 1400}%
\special{fp}%
\special{pa 2140 1340}%
\special{pa 2080 1400}%
\special{fp}%
\special{pa 2170 1370}%
\special{pa 2140 1400}%
\special{fp}%
% LINE 3 0 3 0 Black White  
% 26 1400 1180 1180 1400 1400 1120 1120 1400 1400 1060 1060 1400 1190 1210 1000 1400 1020 1320 940 1400 920 1360 880 1400 1400 1000 1210 1190 1400 940 1320 1020 1400 880 1350 930 1400 820 1370 850 1400 1240 1240 1400 1400 1300 1300 1400 1400 1360 1360 1400
% 
\special{pn 4}%
\special{pa 1400 1180}%
\special{pa 1180 1400}%
\special{fp}%
\special{pa 1400 1120}%
\special{pa 1120 1400}%
\special{fp}%
\special{pa 1400 1060}%
\special{pa 1060 1400}%
\special{fp}%
\special{pa 1190 1210}%
\special{pa 1000 1400}%
\special{fp}%
\special{pa 1020 1320}%
\special{pa 940 1400}%
\special{fp}%
\special{pa 920 1360}%
\special{pa 880 1400}%
\special{fp}%
\special{pa 1400 1000}%
\special{pa 1210 1190}%
\special{fp}%
\special{pa 1400 940}%
\special{pa 1320 1020}%
\special{fp}%
\special{pa 1400 880}%
\special{pa 1350 930}%
\special{fp}%
\special{pa 1400 820}%
\special{pa 1370 850}%
\special{fp}%
\special{pa 1400 1240}%
\special{pa 1240 1400}%
\special{fp}%
\special{pa 1400 1300}%
\special{pa 1300 1400}%
\special{fp}%
\special{pa 1400 1360}%
\special{pa 1360 1400}%
\special{fp}%
\end{picture}}%
}
    \end{center}
求める面積$S$は
     \begin{align*}
     S&=\int_0^1(1-t+(\sqrt{t}-1)^2)dt \\
     &=\int_0^12(1-\sqrt{t})dt \\
     &=2\left[t-\frac{2}{3}t^3/2\right]_0^1 \\
     &=\frac{2}{3}
     \end{align*}
である.一辺$a$とすると,これの$a^2/2$倍ゆえ
     \begin{align*}
     S=\frac{1}{3}a^2
     \end{align*}
である.$\cdots$(答)
\newpage
\end{multicols}
\end{document}