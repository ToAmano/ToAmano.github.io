\documentclass[a4j]{jarticle}
\usepackage{ascmac}
\usepackage{amssymb}
\usepackage{enumerate}
\usepackage{multicol}
\usepackage{framed}
\usepackage{fancyhdr}
\usepackage{latexsym}
\usepackage{indent}
\usepackage[dvips]{graphicx}
\usepackage{color}
\usepackage{emath}
\usepackage{emathPp}
\usepackage{cases}
\usepackage{amsmath}
\allowdisplaybreaks
\pagestyle{fancy}
\lhead{}
\chead{}
\rhead{東京大学前期$1982$年$4$番}
\begin{document}
%分数関係


\def\tfrac#1#2{{\textstyle\frac{#1}{#2}}} %数式中で文中表示の分数を使う時


%Σ関係

\def\dsum#1#2{{\displaystyle\sum_{#1}^{#2}}} %文中で数式表示のΣを使う時


%ベクトル関係


\def\vector#1{\overrightarrow{#1}}  %ベクトルを表現したいとき(aベクトルを表現するときは\ver
\def\norm#1{|\overrightarrow{#1}|} %ベクトルの絶対値
\def\vtwo#1#2{ \left(%
      \begin{array}{c}%
      #1 \\%
      #2 \\%
      \end{array}%
      \right) }                        %2次元ベクトル成分表示
      
      \def\vthree#1#2#3{ \left(
      \begin{array}{c}
      #1 \\
      #2 \\
      #3 \\
      \end{array}
      \right) }                        %3次元ベクトル成分表示



%数列関係


\def\an#1{\verb|{|$#1$\verb|}|}


%極限関係

\def\limit#1#2{\stackrel{#1 \to #2}{\longrightarrow}}   %等式変形からの極限
\def\dlim#1#2{{\displaystyle \lim_{#1\to#2}}} %文中で数式表示の極限を使う



%積分関係

\def\dint#1#2{{\displaystyle \int_{#1}^{#2}}} %文中で数式表示の積分を使う時

\def\ne{\nearrow}
\def\se{\searrow}
\def\nw{\nwarrow}
\def\ne{\nearrow}


%便利なやつ

\def\case#1#2{%
 \[\left\{%
 \begin{array}{l}%
 #1 \\%
 #2%
 \end{array}%
 \right.\] }                           %場合分け
 
\def\1{$\cos\theta=c$,$\sin\theta=s$とおく.}  %cs表示を与える前書きシータ
\def\2{$\cos t=c$,$\sin t=s$とおく.}     %cs表示を与える前書きt
\def\3{$\cos x=c$,$\sin x=s$とおく.}                %cs表示を与える前書きx

\def\fig#1#2#3 {%
\begin{wrapfigure}[#1]{r}{#2 zw}%
\vspace*{-1zh}%
\input{#3}%
\end{wrapfigure} }           %絵の挿入


\def\a{\alpha}   %ギリシャ文字
\def\b{\beta}
\def\g{\gamma}

%問題番号のためのマクロ

\newcounter{nombre} %必須
\renewcommand{\thenombre}{\arabic{nombre}} %任意
\setcounter{nombre}{2} %任意
\newcounter{nombresub}[nombre] %親子関係を定義
\renewcommand{\thenombresub}{\arabic{nombresub}} %任意
\setcounter{nombresub}{0} %任意
\newcommand{\prob}[1][]{\refstepcounter{nombre}#1[問題 \thenombre]}
\newcommand{\probsub}[1][]{\refstepcounter{nombresub}#1(\thenombresub)}


%1-1みたいなカウンタ(todaiとtodaia)
\newcounter{todai}
\setcounter{todai}{0}
\newcounter{todaisub}[todai] 
\setcounter{todaisub}{0} 
\newcommand{\todai}[1][]{\refstepcounter{todai}#1 \thetodai-\thetodaisub}
\newcommand{\todaib}[1][]{\refstepcounter{todai}#1\refstepcounter{todaisub}#1 {\bf [問題 \thetodai.\thetodaisub]}}
\newcommand{\todaia}[1][]{\refstepcounter{todaisub}#1 {\bf [問題 \thetodai.\thetodaisub]}}


\preEqlabel{$\cdots$}
     \begin{oframed}
     $xy$平面上の曲線$y=\sin x$にそって,図のように左から右へ進む動点Pがある.
     Pの速さが一定$V\,\,(V>0)$であるとき,Pの加速度ベクトル$\beku{\a}$の大きさの
     最大値を求めよ.ただし,Pの速さとはPの速度ベクトル$\beku{v}=(v_1,v_2)$の大きさ
     であり,また$t$を時間として$\beku{\a}=\left(\dfrac{dv_1}{dt},\dfrac{dv_2}{dt}\right)$である.

          \begin{zahyou}[ul=10mm](-0.5,8)(-1.2,1.2)
          \YGurafu*{sin(X)}
          \tenretu*<perl>{P($pi/2,1)}
          \Put\P[n]{P}
          \kuromaru{\P}
          \end{zahyou}
          
     \end{oframed}

\setlength{\columnseprule}{0.4pt}
\begin{multicols}{2}
{\bf[解]} 時刻$t$でのPの座標を$(x,\sin x)$で表す.そこで以下
     \begin{align*}
     c=\cos x&s=\sin x
     \end{align*}
とする.速度は
     \[\beku{v}=(x',x'c )\]
であるから,題意より
     \begin{align}
     V^2=x'^2(1+c^2)\label{1}
     \end{align}
となる.\eqref{1}の両辺$t$で微分して
     \begin{align*}
     0=2x'x''(1+c^2)-2x'^3cs
     \end{align*}
$V>0$から$x'\not=0$だから
     \begin{align*}
     &x''(1+c^2)-x'^2cs=0 \\
     &x''=\frac{x'^2cs}{1+c^2}\atag\label{2}
     \end{align*}
さらに,
     \begin{align*}
     \beku{\a}=\left(x'',-x'^2s+x''c\right)
     \end{align*}
だから,加速度ベクトルの大きさ$L$として
     \begin{align*}
     L^2&=x''^2+(-x'^2s+x''c)^2 \\
     &=x''^2+x'^4s^2-2x'^2x''cs+x''^2c^2 \\
     &=x''^2(1+c^2)-2x'^2csx''+x'^4s^2 \\
     &=\left(\frac{x'^2cs}{1+c^2}\right)^2(1+c^2)-2x'^2cs\frac{x'^2cs}{1+c^2}+x'^4s^2 \\
     &\quad\quad\quad\quad\quad\quad\quad\quad\quad\quad\quad%
     \quad\quad\quad\quad\quad(\because\eqref{2}) \\
     &=\frac{-x'^4c^2s^2}{1+c^2}+x'^4s^2 \\
     &=\left(\frac{x'^4s^2}{1+c^2}\right) \\
     &=\frac{s^2}{1+c^2}\frac{V^4}{(1+c^2)^2} \quad\quad(\because\eqref{1})\\
     &=\frac{1-c^2}{(1+c^2)^3}V^4
     \end{align*}
である.ここで$0\le c^2\le1$で,同区間内で$L^2$は$c^2$について単調減少.
故に$L^2$は$c^2=0$で最大値$V^4$をとるので
     \[\max L=V^2(>0)\]
である.$\cdots$(答)
\newpage
\end{multicols}
\end{document}