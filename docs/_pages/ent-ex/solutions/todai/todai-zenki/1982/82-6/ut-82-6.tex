\documentclass[a4j]{jarticle}
\usepackage{amsmath}
\usepackage{ascmac}
\usepackage{amssymb}
\usepackage{enumerate}
\usepackage{multicol}
\usepackage{framed}
\usepackage{fancyhdr}
\usepackage{latexsym}
\usepackage{indent}
\usepackage[dvips]{graphicx}
\usepackage{color}
\usepackage{emath}
\usepackage{emathPp}
\usepackage{cases}
\allowdisplaybreaks
\pagestyle{fancy}
\lhead{}
\chead{}
\rhead{東京大学前期$1982$年$6$番}
\begin{document}
%分数関係


\def\tfrac#1#2{{\textstyle\frac{#1}{#2}}} %数式中で文中表示の分数を使う時


%Σ関係

\def\dsum#1#2{{\displaystyle\sum_{#1}^{#2}}} %文中で数式表示のΣを使う時


%ベクトル関係


\def\vector#1{\overrightarrow{#1}}  %ベクトルを表現したいとき(aベクトルを表現するときは\ver
\def\norm#1{|\overrightarrow{#1}|} %ベクトルの絶対値
\def\vtwo#1#2{ \left(%
      \begin{array}{c}%
      #1 \\%
      #2 \\%
      \end{array}%
      \right) }                        %2次元ベクトル成分表示
      
      \def\vthree#1#2#3{ \left(
      \begin{array}{c}
      #1 \\
      #2 \\
      #3 \\
      \end{array}
      \right) }                        %3次元ベクトル成分表示



%数列関係


\def\an#1{\verb|{|$#1$\verb|}|}


%極限関係

\def\limit#1#2{\stackrel{#1 \to #2}{\longrightarrow}}   %等式変形からの極限
\def\dlim#1#2{{\displaystyle \lim_{#1\to#2}}} %文中で数式表示の極限を使う



%積分関係

\def\dint#1#2{{\displaystyle \int_{#1}^{#2}}} %文中で数式表示の積分を使う時

\def\ne{\nearrow}
\def\se{\searrow}
\def\nw{\nwarrow}
\def\ne{\nearrow}


%便利なやつ

\def\case#1#2{%
 \[\left\{%
 \begin{array}{l}%
 #1 \\%
 #2%
 \end{array}%
 \right.\] }                           %場合分け
 
\def\1{$\cos\theta=c$,$\sin\theta=s$とおく.}  %cs表示を与える前書きシータ
\def\2{$\cos t=c$,$\sin t=s$とおく.}     %cs表示を与える前書きt
\def\3{$\cos x=c$,$\sin x=s$とおく.}                %cs表示を与える前書きx

\def\fig#1#2#3 {%
\begin{wrapfigure}[#1]{r}{#2 zw}%
\vspace*{-1zh}%
\input{#3}%
\end{wrapfigure} }           %絵の挿入


\def\a{\alpha}   %ギリシャ文字
\def\b{\beta}
\def\g{\gamma}

%問題番号のためのマクロ

\newcounter{nombre} %必須
\renewcommand{\thenombre}{\arabic{nombre}} %任意
\setcounter{nombre}{2} %任意
\newcounter{nombresub}[nombre] %親子関係を定義
\renewcommand{\thenombresub}{\arabic{nombresub}} %任意
\setcounter{nombresub}{0} %任意
\newcommand{\prob}[1][]{\refstepcounter{nombre}#1[問題 \thenombre]}
\newcommand{\probsub}[1][]{\refstepcounter{nombresub}#1(\thenombresub)}


%1-1みたいなカウンタ(todaiとtodaia)
\newcounter{todai}
\setcounter{todai}{0}
\newcounter{todaisub}[todai] 
\setcounter{todaisub}{0} 
\newcommand{\todai}[1][]{\refstepcounter{todai}#1 \thetodai-\thetodaisub}
\newcommand{\todaib}[1][]{\refstepcounter{todai}#1\refstepcounter{todaisub}#1 {\bf [問題 \thetodai.\thetodaisub]}}
\newcommand{\todaia}[1][]{\refstepcounter{todaisub}#1 {\bf [問題 \thetodai.\thetodaisub]}}


\preEqlabel{$\cdots$}
     \begin{oframed}
     サイコロが$1$の目を上面にしておいてある.向かい合った一組の面の中心を通る直線のまわりに
     $90^\circ$回転する操作を繰り返すことにより,サイコロの置き方を変えていく.ただし,各回ごとに,
     回転軸及び回転する向きの選び方は,それぞれ同様に確からしいとする.
      
     第$n$回目の操作のあとに$1$の目が上面にある確率を$p_n$,側面のどこかにある確率を$q_n$,底面
     にある確率を$r_n$とする.
          \begin{enumerate}[(1)]
          \item $p_1$,$q_1$,$r_1$を求めよ.
          \item $p_n$,$q_n$,$r_n$を$p_{n-1}$,$q_{n-1}$,$r_{n-1}$で表せ.
          \item $p=\dlim{n}{\infty}p_n$,$q=\dlim{n}{\infty}q_n$,$r=\dlim{n}{\infty}r_n$を求めよ.
          \end{enumerate}
     \end{oframed}

\setlength{\columnseprule}{0.4pt}
\begin{multicols}{2}
{\bf[解]} 軸の選び方が$3$通り,各々について$2$通り回転の選び方があるので,あわせて$6$通りの回転が同様に確からしい.まず,
     \begin{align*}
          \begin{array}{lll}
          p_1=\dfrac{1}{3}&q_1=\dfrac{2}{3}&r_1=0
          \end{array}
     \end{align*}
である.$\cdots$(答)

次いで,漸化式は
     \begin{align*}
          &\begin{cases}
          p_n=\dfrac{1}{3}p_{n-1}+\dfrac{1}{6}q_{n-1} \\
          q_n=2(p_{n-1}+q_{n-1}+r_{n-1})/3=2/3 \\
          r_n=\dfrac{1}{6}q_{n-1}+\dfrac{1}{3}r_{n-1}
          \end{cases} \\
          \Longleftrightarrow
          &\begin{cases}
          p_n=\dfrac{1}{3}p_{n-1}+\dfrac{1}{9} \\
          q_n=2/3 \\
          r_n=\dfrac{1}{6}q_{n-1}+\dfrac{1}{9}
          \end{cases}
     \end{align*}
である.$\cdots$(答)

変形して
     \begin{align*}
          \begin{cases}
          p_{n+1}-\dfrac{1}{6}=\dfrac{1}{3}\left(p_n-\dfrac{1}{6}\right) \\
          r_{n+1}-\dfrac{1}{6}=\dfrac{1}{3}\left(r_n-\dfrac{1}{6}\right) 
          \end{cases}
     \end{align*}
だから,繰り返し用いて,(1)と合わせて
     \begin{align*}
          \begin{cases}
          p_n=\dfrac{1}{6}\left\{1+\left(\dfrac{1}{3}\right)^{n-1}\right\}\limit{n}{\infty}\dfrac{1}{6} \\
          r_{n+1}=\dfrac{1}{6}\left\{1-\left(\dfrac{1}{3}\right)^{n-1}\right\} \limit{n}{\infty}\dfrac{1}{6}
          \end{cases}
     \end{align*}     
である.また,$r=2/3$は明白である.$\cdots$(答)
\newpage
\end{multicols}
\end{document}