\documentclass[a4j]{jarticle}
\usepackage{amsmath}
\usepackage{ascmac}
\usepackage{amssymb}
\usepackage{enumerate}
\usepackage{multicol}
\usepackage{framed}
\usepackage{fancyhdr}
\usepackage{latexsym}
\usepackage{indent}
\usepackage{cases}
\usepackage[dvips]{graphicx}
\usepackage{color}
\usepackage{emath}
\usepackage{emathPp}
\allowdisplaybreaks
\pagestyle{fancy}
\lhead{}
\chead{}
\rhead{東京大学前期$1978$年$5$番}
\begin{document}
%分数関係


\def\tfrac#1#2{{\textstyle\frac{#1}{#2}}} %数式中で文中表示の分数を使う時


%Σ関係

\def\dsum#1#2{{\displaystyle\sum_{#1}^{#2}}} %文中で数式表示のΣを使う時


%ベクトル関係


\def\vector#1{\overrightarrow{#1}}  %ベクトルを表現したいとき(aベクトルを表現するときは\ver
\def\norm#1{|\overrightarrow{#1}|} %ベクトルの絶対値
\def\vtwo#1#2{ \left(%
      \begin{array}{c}%
      #1 \\%
      #2 \\%
      \end{array}%
      \right) }                        %2次元ベクトル成分表示
      
      \def\vthree#1#2#3{ \left(
      \begin{array}{c}
      #1 \\
      #2 \\
      #3 \\
      \end{array}
      \right) }                        %3次元ベクトル成分表示



%数列関係


\def\an#1{\verb|{|$#1$\verb|}|}


%極限関係

\def\limit#1#2{\stackrel{#1 \to #2}{\longrightarrow}}   %等式変形からの極限
\def\dlim#1#2{{\displaystyle \lim_{#1\to#2}}} %文中で数式表示の極限を使う



%積分関係

\def\dint#1#2{{\displaystyle \int_{#1}^{#2}}} %文中で数式表示の積分を使う時

\def\ne{\nearrow}
\def\se{\searrow}
\def\nw{\nwarrow}
\def\ne{\nearrow}


%便利なやつ

\def\case#1#2{%
 \[\left\{%
 \begin{array}{l}%
 #1 \\%
 #2%
 \end{array}%
 \right.\] }                           %場合分け
 
\def\1{$\cos\theta=c$,$\sin\theta=s$とおく.}  %cs表示を与える前書きシータ
\def\2{$\cos t=c$,$\sin t=s$とおく.}     %cs表示を与える前書きt
\def\3{$\cos x=c$,$\sin x=s$とおく.}                %cs表示を与える前書きx

\def\fig#1#2#3 {%
\begin{wrapfigure}[#1]{r}{#2 zw}%
\vspace*{-1zh}%
\input{#3}%
\end{wrapfigure} }           %絵の挿入


\def\a{\alpha}   %ギリシャ文字
\def\b{\beta}
\def\g{\gamma}

%問題番号のためのマクロ

\newcounter{nombre} %必須
\renewcommand{\thenombre}{\arabic{nombre}} %任意
\setcounter{nombre}{2} %任意
\newcounter{nombresub}[nombre] %親子関係を定義
\renewcommand{\thenombresub}{\arabic{nombresub}} %任意
\setcounter{nombresub}{0} %任意
\newcommand{\prob}[1][]{\refstepcounter{nombre}#1[問題 \thenombre]}
\newcommand{\probsub}[1][]{\refstepcounter{nombresub}#1(\thenombresub)}


%1-1みたいなカウンタ(todaiとtodaia)
\newcounter{todai}
\setcounter{todai}{0}
\newcounter{todaisub}[todai] 
\setcounter{todaisub}{0} 
\newcommand{\todai}[1][]{\refstepcounter{todai}#1 \thetodai-\thetodaisub}
\newcommand{\todaib}[1][]{\refstepcounter{todai}#1\refstepcounter{todaisub}#1 {\bf [問題 \thetodai.\thetodaisub]}}
\newcommand{\todaia}[1][]{\refstepcounter{todaisub}#1 {\bf [問題 \thetodai.\thetodaisub]}}


\preEqlabel{$\cdots$}
     \begin{oframed}
     三角形ABCにおいて,各辺の長さを$\mathrm{BC}=a$,$\mathrm{CA}=b$,$\mathrm{BA}=c$と記す.
     いま辺BCを$n$等分する点を$\mathrm{P_1}$,$\mathrm{P_2}$,$\cdots$,$\mathrm{P_{n-1}}$とし,
     $\mathrm{P_n}=\mathrm{C}$とする.
     このとき極限
          \[\lim_{n\to\infty}\frac{1}{n}(\mathrm{AP_1}^2+\mathrm{AP_2}^2+\cdots+\mathrm{AP_n})\]
     を求め,これを$a$,$b$,$c$で表せ.
     \end{oframed}

\setlength{\columnseprule}{0.4pt}
\begin{multicols}{2}
{\bf[解]} 題意から$k=1,2,\cdots,n-1$に対して
      \begin{align*}
      \mathrm{P_kP_{k+1}}=\frac{a}{n}
      \end{align*}
である.そこで三角形$\mathrm{ABP_k}$に余弦定理を用いて
     \begin{align*}
     \mathrm{AP_k}^2&=\mathrm{AB}^2+\mathrm{BP_k}^2-2\mathrm{AB}\cdot\mathrm{BP_k}\cos \angle\mathrm{B} \\
     &=c^2+\left(\frac{ak}{n}\right)^2-2c\frac{ak}{n}\cos\angle\mathrm{B}\atag\label{1}
     \end{align*}
また,三角形ABCに余弦定理を用いて
     \begin{align*}
     \mathrm{AC}^2&=\mathrm{AB}^2+\mathrm{BC}^2-2\mathrm{AB}\cdot\mathrm{BC}\cos \angle\mathrm{B} \\
     b^2&=c^2+a^2-2ac\cos\angle\mathrm{B}\atag\label{2}
     \end{align*}
\eqref{1},\eqref{2}から     
     \begin{align*}
      \mathrm{AP_k}^2=c^2+\left(\frac{ak}{n}\right)^2-(a^2+c^2-b^2)\frac{k}{n}
     \end{align*}
$A=a^2+c^2-b^2$とおいて,これを$k$について足して
     \begin{align*}
     \frac{1}{n}\sum_{k=1}^n\mathrm{AP_k}^2=c^2+\frac{1}{n}\sum_{k=1}^n\left(%
     \left(\frac{ak}{n}\right)^2-A\frac{k}{n}\right)
     \end{align*}
であるから,
     \begin{align*}
     \lim_{n\to\infty}\frac{1}{n}\sum_{k=1}^n\mathrm{AP_k}^2&= %
     c^2+\int_0^1(a^2x^2-Ax)\, dx \\
     &=c^2+\teisekibun{\frac{a^2}{3}x^3-\frac{1}{2}Ax^2}{0}{1} \\
     &=c^2+\frac{a^2}{3}-\frac{1}{2}(a^2+c^2-b^2) \\
     &=\frac{1}{6}(-a^2+3b^2+3c^2)
     \end{align*}
となる.$\cdots$(答)
\newpage
\end{multicols}
\end{document}