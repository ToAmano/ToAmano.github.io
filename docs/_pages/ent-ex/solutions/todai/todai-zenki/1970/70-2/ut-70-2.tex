\documentclass[a4j]{jarticle}
\usepackage{amsmath}
\usepackage{ascmac}
\usepackage{amssymb}
\usepackage{enumerate}
\usepackage{multicol}
\usepackage{framed}
\usepackage{fancyhdr}
\usepackage{latexsym}
\usepackage{indent}
\usepackage{cases}
\allowdisplaybreaks
\pagestyle{fancy}
\lhead{}
\chead{}
\rhead{東京大学前期$1970$年$2$番}
\begin{document}
%分数関係


\def\tfrac#1#2{{\textstyle\frac{#1}{#2}}} %数式中で文中表示の分数を使う時


%Σ関係

\def\dsum#1#2{{\displaystyle\sum_{#1}^{#2}}} %文中で数式表示のΣを使う時


%ベクトル関係


\def\vector#1{\overrightarrow{#1}}  %ベクトルを表現したいとき(aベクトルを表現するときは\ver
\def\norm#1{|\overrightarrow{#1}|} %ベクトルの絶対値
\def\vtwo#1#2{ \left(%
      \begin{array}{c}%
      #1 \\%
      #2 \\%
      \end{array}%
      \right) }                        %2次元ベクトル成分表示
      
      \def\vthree#1#2#3{ \left(
      \begin{array}{c}
      #1 \\
      #2 \\
      #3 \\
      \end{array}
      \right) }                        %3次元ベクトル成分表示



%数列関係


\def\an#1{\verb|{|$#1$\verb|}|}


%極限関係

\def\limit#1#2{\stackrel{#1 \to #2}{\longrightarrow}}   %等式変形からの極限
\def\dlim#1#2{{\displaystyle \lim_{#1\to#2}}} %文中で数式表示の極限を使う



%積分関係

\def\dint#1#2{{\displaystyle \int_{#1}^{#2}}} %文中で数式表示の積分を使う時

\def\ne{\nearrow}
\def\se{\searrow}
\def\nw{\nwarrow}
\def\ne{\nearrow}


%便利なやつ

\def\case#1#2{%
 \[\left\{%
 \begin{array}{l}%
 #1 \\%
 #2%
 \end{array}%
 \right.\] }                           %場合分け
 
\def\1{$\cos\theta=c$,$\sin\theta=s$とおく.}  %cs表示を与える前書きシータ
\def\2{$\cos t=c$,$\sin t=s$とおく.}     %cs表示を与える前書きt
\def\3{$\cos x=c$,$\sin x=s$とおく.}                %cs表示を与える前書きx

\def\fig#1#2#3 {%
\begin{wrapfigure}[#1]{r}{#2 zw}%
\vspace*{-1zh}%
\input{#3}%
\end{wrapfigure} }           %絵の挿入


\def\a{\alpha}   %ギリシャ文字
\def\b{\beta}
\def\g{\gamma}

%問題番号のためのマクロ

\newcounter{nombre} %必須
\renewcommand{\thenombre}{\arabic{nombre}} %任意
\setcounter{nombre}{2} %任意
\newcounter{nombresub}[nombre] %親子関係を定義
\renewcommand{\thenombresub}{\arabic{nombresub}} %任意
\setcounter{nombresub}{0} %任意
\newcommand{\prob}[1][]{\refstepcounter{nombre}#1[問題 \thenombre]}
\newcommand{\probsub}[1][]{\refstepcounter{nombresub}#1(\thenombresub)}


%1-1みたいなカウンタ(todaiとtodaia)
\newcounter{todai}
\setcounter{todai}{0}
\newcounter{todaisub}[todai] 
\setcounter{todaisub}{0} 
\newcommand{\todai}[1][]{\refstepcounter{todai}#1 \thetodai-\thetodaisub}
\newcommand{\todaib}[1][]{\refstepcounter{todai}#1\refstepcounter{todaisub}#1 {\bf [問題 \thetodai.\thetodaisub]}}
\newcommand{\todaia}[1][]{\refstepcounter{todaisub}#1 {\bf [問題 \thetodai.\thetodaisub]}}


     \begin{oframed}
     $x$軸上原点から出発し,効果を投げて表がでたら右へ$1$だけ進み,裏がでたら左へ$1$だけすす
     むことにする.
          \begin{enumerate}[(1)]
          \item これを$4$回くりかえしたとき$x=0$,$x=\pm1$,$x=\pm2$,$x=\pm3$,$x=\pm4$
          の各点にいる確率をもとめよ.
          \item 一般にこれを$n$回くりかえしたとき$x=n-2$にいる確率と$x=n-4$にいる確率とを求め
          よ.
          \end{enumerate}
     \end{oframed}

\setlength{\columnseprule}{0.4pt}
\begin{multicols}{2}
{\bf[解]}
     \begin{enumerate}[(1)]
     \item $4$回操作した時,$x=\pm1,\pm3$にある確率は明らかに$0$であることに注意する.
     又,$x=\pm2$にいる確率は対称性から等しく,例えば表が$3$回出る時で
          \begin{align}
          \frac{{}_4C_1}{2^4}=\frac{1}{4}\label{1}
          \end{align}
     $x=\pm4$にいる確率は対称性から等しく,例えば表が$4$回出る時で
          \begin{align}
          \frac{{}_4C_0}{2^4}=\frac{1}{16}\label{2}
          \end{align}
     \eqref{1},\eqref{2}から,余事象より$x=0$にいる確率は
          \begin{align*}
          1-2\left(\frac{1}{4}+\frac{1}{16}\right)=\frac{3}{8}
          \end{align*}
     である.以上をまとめて
          \begin{align*}
          \left\{
               \begin{array}{ll}
               x=\pm1,\pm3 & 0     \\
               x=0                & 3/8   \\
               x=\pm2          & 1/4   \\
               x=\pm4          & 1/16
               \end{array}
          \right.
          \end{align*}
     である.$\cdots$(答)
     
     \item $x=n-2$にいるのは表が$n-1$回出た時で$\dfrac{{}_nC_{n-1}}{2^n}=\dfrac{n}{2^n}$.
     $\cdots$(答)
     また$x=n-4$にいるのは表が$n-2$回出た時で$\dfrac{{}_nC_{n-2}}{2^n}=\dfrac{n(n-1)}{2^{n+1}}$.
     $\cdots$(答)
     \end{enumerate}
\newpage
\end{multicols}
\end{document}