\documentclass[a4j]{jarticle}
\usepackage{amsmath}
\usepackage{ascmac}
\usepackage{amssymb}
\usepackage{framed}
\usepackage{enumerate}
\usepackage{multicol}
\title{東大前期86-3}
\begin{document}
%分数関係


\def\tfrac#1#2{{\textstyle\frac{#1}{#2}}} %数式中で文中表示の分数を使う時


%Σ関係

\def\dsum#1#2{{\displaystyle\sum_{#1}^{#2}}} %文中で数式表示のΣを使う時


%ベクトル関係


\def\vector#1{\overrightarrow{#1}}  %ベクトルを表現したいとき(aベクトルを表現するときは\ver
\def\norm#1{|\overrightarrow{#1}|} %ベクトルの絶対値
\def\vtwo#1#2{ \left(%
      \begin{array}{c}%
      #1 \\%
      #2 \\%
      \end{array}%
      \right) }                        %2次元ベクトル成分表示
      
      \def\vthree#1#2#3{ \left(
      \begin{array}{c}
      #1 \\
      #2 \\
      #3 \\
      \end{array}
      \right) }                        %3次元ベクトル成分表示



%数列関係


\def\an#1{\verb|{|$#1$\verb|}|}


%極限関係

\def\limit#1#2{\stackrel{#1 \to #2}{\longrightarrow}}   %等式変形からの極限
\def\dlim#1#2{{\displaystyle \lim_{#1\to#2}}} %文中で数式表示の極限を使う



%積分関係

\def\dint#1#2{{\displaystyle \int_{#1}^{#2}}} %文中で数式表示の積分を使う時

\def\ne{\nearrow}
\def\se{\searrow}
\def\nw{\nwarrow}
\def\ne{\nearrow}


%便利なやつ

\def\case#1#2{%
 \[\left\{%
 \begin{array}{l}%
 #1 \\%
 #2%
 \end{array}%
 \right.\] }                           %場合分け
 
\def\1{$\cos\theta=c$,$\sin\theta=s$とおく.}  %cs表示を与える前書きシータ
\def\2{$\cos t=c$,$\sin t=s$とおく.}     %cs表示を与える前書きt
\def\3{$\cos x=c$,$\sin x=s$とおく.}                %cs表示を与える前書きx

\def\fig#1#2#3 {%
\begin{wrapfigure}[#1]{r}{#2 zw}%
\vspace*{-1zh}%
\input{#3}%
\end{wrapfigure} }           %絵の挿入


\def\a{\alpha}   %ギリシャ文字
\def\b{\beta}
\def\g{\gamma}

%問題番号のためのマクロ

\newcounter{nombre} %必須
\renewcommand{\thenombre}{\arabic{nombre}} %任意
\setcounter{nombre}{2} %任意
\newcounter{nombresub}[nombre] %親子関係を定義
\renewcommand{\thenombresub}{\arabic{nombresub}} %任意
\setcounter{nombresub}{0} %任意
\newcommand{\prob}[1][]{\refstepcounter{nombre}#1[問題 \thenombre]}
\newcommand{\probsub}[1][]{\refstepcounter{nombresub}#1(\thenombresub)}


%1-1みたいなカウンタ(todaiとtodaia)
\newcounter{todai}
\setcounter{todai}{0}
\newcounter{todaisub}[todai] 
\setcounter{todaisub}{0} 
\newcommand{\todai}[1][]{\refstepcounter{todai}#1 \thetodai-\thetodaisub}
\newcommand{\todaib}[1][]{\refstepcounter{todai}#1\refstepcounter{todaisub}#1 {\bf [問題 \thetodai.\thetodaisub]}}
\newcommand{\todaia}[1][]{\refstepcounter{todaisub}#1 {\bf [問題 \thetodai.\thetodaisub]}}


\begin{oframed}
\begin{itemize}
\item[(1)]$xyz$空間において,$3$点$A(0,0,1/2)$,$B(0,1/2,1)$,$C(1,0,1)$を通る平面$S_0$に垂直で,長さが$1$のベクトル$\vec{n_0}$を全て求めよ.
\item[(2)]$2$点$D(1,0,0)$,$E(0,1,0)$を通る直線$l$を軸として,平面$S_0$を回転して得られる全ての平面$S$を考える.このような平面$S$に垂直で長さ$1$のベクトル$\vec{n}=(x,y,z)$の$y$成分の絶対値$|y|$は$S$と共に変化するが,その最大値及び最小値を求めよ.
\end{itemize}
\end{oframed}
\setlength{\columnseprule}{0.4pt}
\begin{multicols}{2}

{\bf[解]}
\begin{enumerate}[(1)]
\item$\vec{n_0}$は互いに反対を向いた$2$つのベクトルである.$S_0$の法線ベクトルの一つに
$\vec{n}=(1,2,-2)$があるから,求めるベクトルは$k\in\mathbb{R}$として$k\vec{n}$の形で書ける.この絶対値が$1$であるから,$k=\pm\dfrac{1}{3}$.故に
$\vec{n_0}=\pm\dfrac{1}{3}(1,2,-2)\cdots\text{(答)}$が求める答えである.

\item$l$の方向ベクトル$\vec{l}=(-1,1,0)$である.又,$\vec{n_0}=$を$\vec{l}$の周りに回転させたベクトルを考えればよい.$l$上に点$A$をとる.以下位置ベクトルの基準点を$A$として,点
\end{enumerate}
\newpage
\end{multicols}
\end{document}
