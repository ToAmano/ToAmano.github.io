\documentclass[a4j]{jarticle}
\usepackage{amsmath}
\usepackage{ascmac}
\usepackage{amssymb}
\usepackage{enumerate}
\usepackage{multicol}
\usepackage{framed}
\usepackage{fancyhdr}
\usepackage{latexsym}
\usepackage{indent}
\usepackage{cases}
\usepackage[dvips]{graphicx}
\usepackage{color}
\allowdisplaybreaks
\pagestyle{fancy}
\lhead{}
\chead{}
\rhead{東京大学前期$2005$年$6$番}
\begin{document}
%分数関係


\def\tfrac#1#2{{\textstyle\frac{#1}{#2}}} %数式中で文中表示の分数を使う時


%Σ関係

\def\dsum#1#2{{\displaystyle\sum_{#1}^{#2}}} %文中で数式表示のΣを使う時


%ベクトル関係


\def\vector#1{\overrightarrow{#1}}  %ベクトルを表現したいとき(aベクトルを表現するときは\ver
\def\norm#1{|\overrightarrow{#1}|} %ベクトルの絶対値
\def\vtwo#1#2{ \left(%
      \begin{array}{c}%
      #1 \\%
      #2 \\%
      \end{array}%
      \right) }                        %2次元ベクトル成分表示
      
      \def\vthree#1#2#3{ \left(
      \begin{array}{c}
      #1 \\
      #2 \\
      #3 \\
      \end{array}
      \right) }                        %3次元ベクトル成分表示



%数列関係


\def\an#1{\verb|{|$#1$\verb|}|}


%極限関係

\def\limit#1#2{\stackrel{#1 \to #2}{\longrightarrow}}   %等式変形からの極限
\def\dlim#1#2{{\displaystyle \lim_{#1\to#2}}} %文中で数式表示の極限を使う



%積分関係

\def\dint#1#2{{\displaystyle \int_{#1}^{#2}}} %文中で数式表示の積分を使う時

\def\ne{\nearrow}
\def\se{\searrow}
\def\nw{\nwarrow}
\def\ne{\nearrow}


%便利なやつ

\def\case#1#2{%
 \[\left\{%
 \begin{array}{l}%
 #1 \\%
 #2%
 \end{array}%
 \right.\] }                           %場合分け
 
\def\1{$\cos\theta=c$,$\sin\theta=s$とおく.}  %cs表示を与える前書きシータ
\def\2{$\cos t=c$,$\sin t=s$とおく.}     %cs表示を与える前書きt
\def\3{$\cos x=c$,$\sin x=s$とおく.}                %cs表示を与える前書きx

\def\fig#1#2#3 {%
\begin{wrapfigure}[#1]{r}{#2 zw}%
\vspace*{-1zh}%
\input{#3}%
\end{wrapfigure} }           %絵の挿入


\def\a{\alpha}   %ギリシャ文字
\def\b{\beta}
\def\g{\gamma}

%問題番号のためのマクロ

\newcounter{nombre} %必須
\renewcommand{\thenombre}{\arabic{nombre}} %任意
\setcounter{nombre}{2} %任意
\newcounter{nombresub}[nombre] %親子関係を定義
\renewcommand{\thenombresub}{\arabic{nombresub}} %任意
\setcounter{nombresub}{0} %任意
\newcommand{\prob}[1][]{\refstepcounter{nombre}#1[問題 \thenombre]}
\newcommand{\probsub}[1][]{\refstepcounter{nombresub}#1(\thenombresub)}


%1-1みたいなカウンタ(todaiとtodaia)
\newcounter{todai}
\setcounter{todai}{0}
\newcounter{todaisub}[todai] 
\setcounter{todaisub}{0} 
\newcommand{\todai}[1][]{\refstepcounter{todai}#1 \thetodai-\thetodaisub}
\newcommand{\todaib}[1][]{\refstepcounter{todai}#1\refstepcounter{todaisub}#1 {\bf [問題 \thetodai.\thetodaisub]}}
\newcommand{\todaia}[1][]{\refstepcounter{todaisub}#1 {\bf [問題 \thetodai.\thetodaisub]}}


\begin{oframed}
空間内に$3$点$P\left(1,\dfrac{1}{2},0\right)$,$Q\left(1,-\dfrac{1}{2},0\right)$,
$R\left(\dfrac{1}{4},0,\dfrac{\sqrt{3}}{4}\right)$
を頂点とする正$3$角形の板$S$がある.$S$を$z$軸のまわりに$1$回転させた時,$S$が通過する
点全体のつくる立体の体積を求めよ.
\end{oframed}

\setlength{\columnseprule}{0.4pt}
\begin{multicols}{2}
{\bf[解]}辺$PR$,$QR$の方程式はそれぞれ以下のようになる.
     \begin{align*}
     PR&:\vthree{x}{y}{z}=\vthree{1}{1/2}{0}+t\vthree{-3/4}{-1/2}{\sqrt{3}/4} \\	
     QR&:\vthree{x}{y}{z}=\vthree{1}{-1/2}{0}+t\vthree{-3/4}{1/2}{\sqrt{3}/4}
     \end{align*}
これらと平面$z=k$の交点は順に,
     \begin{align*}
     p&=\frac{1}{2}\left(1-\frac{4\sqrt{3}}{3}k\right)  \\
     q&=1-\sqrt{3}k
     \end{align*}
として
     \begin{align*}
     (q,p,k), (q,-p,k)
     \end{align*}
である.     
故に,図形の$z=k(0\le k\le \dfrac{\sqrt{3}}{4})$での切断面は
     \begin{align*}
     &x=q \\
     &|y|\le p 
     \end{align*}
となる.したがって,切断面上の点と原点との距離の最大小値は
     \begin{align*}
     r_{max}^2&=p^2+q^2 \\
     r_{min}^2&=q^2
     \end{align*}
である.故にこの平面での切断面の面積$S(k)$は
     \begin{align*}
     S(k)&=\pi(r_{max}^2- r_{min}^2) \\
     &=\pi p^2 
     \end{align*}
となる.求める体積$V$は
     \begin{align*}
     V&=\int_0^{\sqrt{3}/4}S(k)dk \\
     &=\frac{\pi}{4}\int_0^{\sqrt{3}/4}\left(1-\frac{4\sqrt{3}}{3}k\right)^2dk \\
     &=\frac{\pi}{4}\left[\frac{\sqrt{3}}{12}\left(\frac{4\sqrt{3}}{3}k-1\right)^3\right]_0^{\sqrt{3}/4} \\
     &=\frac{\sqrt{3}\pi}{48}\cdots\text{(答)}
     \end{align*}
となる.     
\newpage
\end{multicols}
\end{document}