\documentclass[a4j]{jarticle}
\usepackage{amsmath}
\usepackage{ascmac}
\usepackage{amssymb}
\usepackage{enumerate}
\usepackage{multicol}
\usepackage{framed}
\begin{document}
%分数関係


\def\tfrac#1#2{{\textstyle\frac{#1}{#2}}} %数式中で文中表示の分数を使う時


%Σ関係

\def\dsum#1#2{{\displaystyle\sum_{#1}^{#2}}} %文中で数式表示のΣを使う時


%ベクトル関係


\def\vector#1{\overrightarrow{#1}}  %ベクトルを表現したいとき(aベクトルを表現するときは\ver
\def\norm#1{|\overrightarrow{#1}|} %ベクトルの絶対値
\def\vtwo#1#2{ \left(%
      \begin{array}{c}%
      #1 \\%
      #2 \\%
      \end{array}%
      \right) }                        %2次元ベクトル成分表示
      
      \def\vthree#1#2#3{ \left(
      \begin{array}{c}
      #1 \\
      #2 \\
      #3 \\
      \end{array}
      \right) }                        %3次元ベクトル成分表示



%数列関係


\def\an#1{\verb|{|$#1$\verb|}|}


%極限関係

\def\limit#1#2{\stackrel{#1 \to #2}{\longrightarrow}}   %等式変形からの極限
\def\dlim#1#2{{\displaystyle \lim_{#1\to#2}}} %文中で数式表示の極限を使う



%積分関係

\def\dint#1#2{{\displaystyle \int_{#1}^{#2}}} %文中で数式表示の積分を使う時

\def\ne{\nearrow}
\def\se{\searrow}
\def\nw{\nwarrow}
\def\ne{\nearrow}


%便利なやつ

\def\case#1#2{%
 \[\left\{%
 \begin{array}{l}%
 #1 \\%
 #2%
 \end{array}%
 \right.\] }                           %場合分け
 
\def\1{$\cos\theta=c$,$\sin\theta=s$とおく.}  %cs表示を与える前書きシータ
\def\2{$\cos t=c$,$\sin t=s$とおく.}     %cs表示を与える前書きt
\def\3{$\cos x=c$,$\sin x=s$とおく.}                %cs表示を与える前書きx

\def\fig#1#2#3 {%
\begin{wrapfigure}[#1]{r}{#2 zw}%
\vspace*{-1zh}%
\input{#3}%
\end{wrapfigure} }           %絵の挿入


\def\a{\alpha}   %ギリシャ文字
\def\b{\beta}
\def\g{\gamma}

%問題番号のためのマクロ

\newcounter{nombre} %必須
\renewcommand{\thenombre}{\arabic{nombre}} %任意
\setcounter{nombre}{2} %任意
\newcounter{nombresub}[nombre] %親子関係を定義
\renewcommand{\thenombresub}{\arabic{nombresub}} %任意
\setcounter{nombresub}{0} %任意
\newcommand{\prob}[1][]{\refstepcounter{nombre}#1[問題 \thenombre]}
\newcommand{\probsub}[1][]{\refstepcounter{nombresub}#1(\thenombresub)}


%1-1みたいなカウンタ(todaiとtodaia)
\newcounter{todai}
\setcounter{todai}{0}
\newcounter{todaisub}[todai] 
\setcounter{todaisub}{0} 
\newcommand{\todai}[1][]{\refstepcounter{todai}#1 \thetodai-\thetodaisub}
\newcommand{\todaib}[1][]{\refstepcounter{todai}#1\refstepcounter{todaisub}#1 {\bf [問題 \thetodai.\thetodaisub]}}
\newcommand{\todaia}[1][]{\refstepcounter{todaisub}#1 {\bf [問題 \thetodai.\thetodaisub]}}


\begin{oframed}
空間内の店点の集合$\{(x,y,z)|0\le y,0\le z\}$に含まれ,原点$O$において$x$軸に接し,$xy$平面と
$45^\circ$の傾きをなす,半径$1$の円板$C$がある.座標が$(0,0,2\sqrt{2})$の位置にある点光源
$P$により,$xy$平面上に投ぜられた円板$C$の影を$S$とする.
     \begin{enumerate}[(1)]
     \item $S$の輪郭を表す$xy$平面上の曲線の方程式を求めよ.
     \item 円板$C$と影$S$の間に挟まれ,光の届かない部分のつくる立体の体積を求めよ.
     \end{enumerate}
\end{oframed}

\setlength{\columnseprule}{0.4pt}
\begin{multicols}{2}
{\bf[解]}
     \begin{enumerate}[(1)]
     \item 点光源のある点を$P$とする.題意の曲線を$C'$とし,この上の点を$Q(X,Y,0)$とする.
     題意から,$C$の中心は
     $R\left(0,\dfrac{\sqrt{2}}{2},\dfrac{\sqrt{2}}{2}\right)$であり,また$C$は平面$y=z$にあるから,
     $C$の外周$D$上の点$E$の満たす式は
          \begin{align}
          D:\left\{
               \begin{array}{l}
               y=z \\
               |\vector{ER}|=1 
               \end{array}
          \right. \label{1}
          \end{align}
     となる.$Q$の条件は,直線$PQ$が$D$と交わることである.直線$PQ$はパラメータを用いて
          \begin{align*}
          \vthree{x}{y}{z}=\vthree{0}{0}{2\sqrt{2}}+t\vthree{X}{Y}{-2\sqrt{2}}
          \end{align*}
    と表せるから,これは\eqref{1}を満たす$t$が存在することである.代入して
         \begin{align}
         tY&=2\sqrt{2}(1-t) \label{2} \\
         x^2&+(y-\dfrac{\sqrt{2}}{2})^2+(z-\dfrac{\sqrt{2}}{2})^2=1 \label{3}
         \end{align}
     $Y\ge0$に注意して\eqref{2}から$t$を消去して
     $t=\dfrac{2\sqrt{2}}{Y+2\sqrt{2}}$である.
     これを\eqref{3}に代入して,
          \begin{align*}
          &\frac{8X^2}{(Y+2\sqrt{2})^2}+2(\dfrac{2\sqrt{2}Y}{Y+2\sqrt{2}}-\frac{\sqrt{2}}{2})^2=1 \\
         \therefore&8X^2+2(\frac{3\sqrt{2}}{2}Y-2)^2=(Y+2\sqrt{2})^2 \\
         \therefore&X^2+(Y-\sqrt{2})^2=2\cdots\text{(答)}
          \end{align*}
     これがもとめる$C'$の方程式である.
     
     \item 求める体積$V$とすれば,
          \begin{align}
          V=(\text{円錐P-C'})-(\text{円錐P-C}) \label{4}
          \end{align}
     である.円錐P-C'は底面積$\sqrt{2}\sqrt{2}\pi=2\pi$,高さ$2\sqrt{2}$の円錐で,この体積$V_1$
     として
          \begin{align}
          V_1=\frac{1}{3}(2\sqrt{2})(2\pi)=\frac{4\sqrt{2}}{3}\pi  \label{5}
          \end{align}
     となる.次に円錐P-Cについて,この高さは平面$y=z$と点$P$の距離に等しく$2$,底面積は
     半径$1$の円のそれ$\pi$である.故に体積$V_2$として
          \begin{align}
          V_2=\frac{2}{3}\pi \label{6}
          \end{align}
     \eqref{5},\eqref{6}を\eqref{4}に代入して
          \begin{align*}
          V=\frac{2}{3}(2\sqrt{2}-1)\pi\cdots\text{(答)} 
          \end{align*}
     である.
     \end{enumerate}
\newpage
\end{multicols}
\end{document}