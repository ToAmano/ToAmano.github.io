\documentclass[a4j]{jarticle}
\usepackage{amsmath}
\usepackage{ascmac}
\usepackage{amssymb}
\usepackage{enumerate}
\usepackage{multicol}
\usepackage{framed}
\usepackage{fancyhdr}
\usepackage{latexsym}
\usepackage{indent}
\usepackage{cases}
\usepackage[dvips]{graphicx}
\usepackage{color}
\usepackage{emath}
\usepackage{emathPp}
\allowdisplaybreaks
\pagestyle{fancy}
\lhead{}
\chead{}
\rhead{東京大学前期$1984$年$6$番}
\begin{document}
%分数関係


\def\tfrac#1#2{{\textstyle\frac{#1}{#2}}} %数式中で文中表示の分数を使う時


%Σ関係

\def\dsum#1#2{{\displaystyle\sum_{#1}^{#2}}} %文中で数式表示のΣを使う時


%ベクトル関係


\def\vector#1{\overrightarrow{#1}}  %ベクトルを表現したいとき(aベクトルを表現するときは\ver
\def\norm#1{|\overrightarrow{#1}|} %ベクトルの絶対値
\def\vtwo#1#2{ \left(%
      \begin{array}{c}%
      #1 \\%
      #2 \\%
      \end{array}%
      \right) }                        %2次元ベクトル成分表示
      
      \def\vthree#1#2#3{ \left(
      \begin{array}{c}
      #1 \\
      #2 \\
      #3 \\
      \end{array}
      \right) }                        %3次元ベクトル成分表示



%数列関係


\def\an#1{\verb|{|$#1$\verb|}|}


%極限関係

\def\limit#1#2{\stackrel{#1 \to #2}{\longrightarrow}}   %等式変形からの極限
\def\dlim#1#2{{\displaystyle \lim_{#1\to#2}}} %文中で数式表示の極限を使う



%積分関係

\def\dint#1#2{{\displaystyle \int_{#1}^{#2}}} %文中で数式表示の積分を使う時

\def\ne{\nearrow}
\def\se{\searrow}
\def\nw{\nwarrow}
\def\ne{\nearrow}


%便利なやつ

\def\case#1#2{%
 \[\left\{%
 \begin{array}{l}%
 #1 \\%
 #2%
 \end{array}%
 \right.\] }                           %場合分け
 
\def\1{$\cos\theta=c$,$\sin\theta=s$とおく.}  %cs表示を与える前書きシータ
\def\2{$\cos t=c$,$\sin t=s$とおく.}     %cs表示を与える前書きt
\def\3{$\cos x=c$,$\sin x=s$とおく.}                %cs表示を与える前書きx

\def\fig#1#2#3 {%
\begin{wrapfigure}[#1]{r}{#2 zw}%
\vspace*{-1zh}%
\input{#3}%
\end{wrapfigure} }           %絵の挿入


\def\a{\alpha}   %ギリシャ文字
\def\b{\beta}
\def\g{\gamma}

%問題番号のためのマクロ

\newcounter{nombre} %必須
\renewcommand{\thenombre}{\arabic{nombre}} %任意
\setcounter{nombre}{2} %任意
\newcounter{nombresub}[nombre] %親子関係を定義
\renewcommand{\thenombresub}{\arabic{nombresub}} %任意
\setcounter{nombresub}{0} %任意
\newcommand{\prob}[1][]{\refstepcounter{nombre}#1[問題 \thenombre]}
\newcommand{\probsub}[1][]{\refstepcounter{nombresub}#1(\thenombresub)}


%1-1みたいなカウンタ(todaiとtodaia)
\newcounter{todai}
\setcounter{todai}{0}
\newcounter{todaisub}[todai] 
\setcounter{todaisub}{0} 
\newcommand{\todai}[1][]{\refstepcounter{todai}#1 \thetodai-\thetodaisub}
\newcommand{\todaib}[1][]{\refstepcounter{todai}#1\refstepcounter{todaisub}#1 {\bf [問題 \thetodai.\thetodaisub]}}
\newcommand{\todaia}[1][]{\refstepcounter{todaisub}#1 {\bf [問題 \thetodai.\thetodaisub]}}


\preEqlabel{$\cdots$}
     \begin{oframed}
     $xy$平面において,不等式$x^2\le y$ の表す領域を$D$とし,不等式$(x-4)^2\le y$の表す領域を$E$とする.
     
     このとき,次の条件(*)を満たす点P$(a,b)$全体の集合を求め,これを図示せよ.
          \begin{enumerate}[(*)]
          \item P$(a,b)$に関して$D$と対称な領域を$U$とするとき,
               \[D\cap U\not=\emptyset,E\cap U\not=\emptyset,D\cap E\cap U=\emptyset\]
          が同時に成り立つ.ただし$\emptyset$は空集合を表すものとする.
          \end{enumerate}
     \end{oframed}

\setlength{\columnseprule}{0.4pt}
\begin{multicols}{2}
{\bf[解]} $g(x)=(x-4)^2$とおく.
$y=x^2$をPに関して対称移動すると,
     \[(2b-y)=(2a-x)^2\]
であるから,
     \[U:y\le -(x-2a)^2+2b(\equiv f(x))\]
である.$D$と$E$のグラフは下図のようになる.\\

     \begin{zahyou}[ul=5mm](-0.5,6)(-1,6)
     \def\Fx{X**2}
     \def\Gx{(X-4)**2}
     \tenretu*{A(0,3);B(4,3);C(2,4);D(4,0)}
     \YGurafu\Fx{\xmin}{\xmax}
     \YGurafu*\Gx
     \Put\A [ne]{$D$}
     \Put\B [n]{$E$}
     \Put\C[syaei=xy]{}
     \Put\D[s]{$4$}
     \end{zahyou}

     

さて,簡単のため題意の条件を
     \begin{align*}
     A&:D\cap U\not=\emptyset \\
     B&:E\cap U\not=\emptyset \\
     C&:D\cap E\cap U=\emptyset
     \end{align*}
とおく.
     
不等号の向きから
     \begin{align}
           &A\land C \Longleftrightarrow \nonumber\\
           &f(x)=x^2\text{が$x<2$のみに実解を持つ.} \label{1}\\
           &B\land C \Longleftrightarrow \nonumber\\
           &f(x)=g(x)\text{が$2<x$のみに実解を持つ.}\label{2}
     \end{align}
である.故に
     \begin{align}
     A\land B\land C \Longleftrightarrow \eqref{1}\cap\eqref{2}\label{3}
     \end{align}
である.以下順番に考える.\\

     \begin{indentation}{2zw}{0pt}
     \noindent\underline{(i)\,\eqref{1}について}\\
     \eqref{1}の方程式
          \[(x-a)^2+a^2-b=0\]
     の左辺$h(x)$とおく.判別式を$D_1$とおく.$h(x)=0$が$x<2$にのみ実解をもつ条件は,
          \begin{align*}
               &\begin{cases}
               D_1/4\ge0 \\
               h(2)>0 \\
               a<2
               \end{cases}\\
          \Longleftrightarrow
               &\begin{cases}
               a^2-b\le0 \\
               2(a-1)^2+2>b \\
               a<2
               \end{cases}
          \end{align*}     
     である.\\
     
     \noindent\underline{(ii)\,\eqref{2}について}\\
     \eqref{2}の方程式
          \[\left\{x-(2+a)\right\}^2+(a-2)^2-b=0\]
     の左辺$t(x)$とおく.判別式を$D_2$とおく.$t(x)=0$が$2<x$にのみ実解をもつ条件は,
     \begin{align*}
               &\begin{cases}
               D_2/4\ge0 \\
               t(2)>0 \\
               2<2+a
               \end{cases}\\
          \Longleftrightarrow
               &\begin{cases}
               (a-2)^2-b\le0 \\
               2(a-1)^2+2>b \\
               0<a
               \end{cases}
          \end{align*}     
     である.\\     
     \end{indentation}
以上をまとめて
     \[\eqref{3}\Longleftrightarrow 
          \begin{cases}
          0<a<2\\
          2(a-1)^2+2>b \\
          a^2\le b \\
          (a-2)^2\le b
          \end{cases}
     \]
が求める条件である.$\cdots$(答)\\

図示すると下図斜線部.(境界は実線のみ含む.)\\

     \begin{zahyou}[ul=10mm](-0.2,3)(-1,6)
     \def\fx{X**2}
     \def\gx{(X-2)**2}
     \def\hx{2*(X-1)**2+2}
     \def\aval{1}
     \YTen\hx\aval\A
     \tenretu*{B(2,4)}
     \YKouten\fx\gx{}{}\xii\C
     \YKouten\gx\hx{}{}\xiii\D
     \YGurafu*\fx
     \YGurafu*\gx
     \YGurafu(.1)(.02)\hx{\xmin}{\xmax}
     \Put\A[syaei=xy,xlabel=1,ylabel=2]{}
     \Put\B[syaei=xy,xlabel=2,ylabel=4]{}
     \Put\C[syaei=xy,xlabel=1,ylabel=1]{}
     \YNurii*\gx\hx{0}{1}
     \YNurii*\fx\hx{1}{2}
     \tenretu*{E(0,4);F(2,4)}
     \siromaru{\E;\F}
     \end{zahyou}


\newpage
\end{multicols}
\end{document}