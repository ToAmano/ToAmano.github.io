\documentclass[a4j]{jarticle}
\usepackage{amsmath}
\usepackage{ascmac}
\usepackage{amssymb}
\usepackage{enumerate}
\usepackage{multicol}
\usepackage{framed}
\usepackage{fancyhdr}
\usepackage{latexsym}
\usepackage{indent}
\usepackage{cases}
\usepackage[dvips]{graphicx}
\usepackage{color}
\usepackage{emath}
\usepackage{emathPp}
\allowdisplaybreaks
\pagestyle{fancy}
\lhead{}
\chead{}
\rhead{東京大学前期$1976$年$1$番}
\begin{document}
%分数関係


\def\tfrac#1#2{{\textstyle\frac{#1}{#2}}} %数式中で文中表示の分数を使う時


%Σ関係

\def\dsum#1#2{{\displaystyle\sum_{#1}^{#2}}} %文中で数式表示のΣを使う時


%ベクトル関係


\def\vector#1{\overrightarrow{#1}}  %ベクトルを表現したいとき(aベクトルを表現するときは\ver
\def\norm#1{|\overrightarrow{#1}|} %ベクトルの絶対値
\def\vtwo#1#2{ \left(%
      \begin{array}{c}%
      #1 \\%
      #2 \\%
      \end{array}%
      \right) }                        %2次元ベクトル成分表示
      
      \def\vthree#1#2#3{ \left(
      \begin{array}{c}
      #1 \\
      #2 \\
      #3 \\
      \end{array}
      \right) }                        %3次元ベクトル成分表示



%数列関係


\def\an#1{\verb|{|$#1$\verb|}|}


%極限関係

\def\limit#1#2{\stackrel{#1 \to #2}{\longrightarrow}}   %等式変形からの極限
\def\dlim#1#2{{\displaystyle \lim_{#1\to#2}}} %文中で数式表示の極限を使う



%積分関係

\def\dint#1#2{{\displaystyle \int_{#1}^{#2}}} %文中で数式表示の積分を使う時

\def\ne{\nearrow}
\def\se{\searrow}
\def\nw{\nwarrow}
\def\ne{\nearrow}


%便利なやつ

\def\case#1#2{%
 \[\left\{%
 \begin{array}{l}%
 #1 \\%
 #2%
 \end{array}%
 \right.\] }                           %場合分け
 
\def\1{$\cos\theta=c$,$\sin\theta=s$とおく.}  %cs表示を与える前書きシータ
\def\2{$\cos t=c$,$\sin t=s$とおく.}     %cs表示を与える前書きt
\def\3{$\cos x=c$,$\sin x=s$とおく.}                %cs表示を与える前書きx

\def\fig#1#2#3 {%
\begin{wrapfigure}[#1]{r}{#2 zw}%
\vspace*{-1zh}%
\input{#3}%
\end{wrapfigure} }           %絵の挿入


\def\a{\alpha}   %ギリシャ文字
\def\b{\beta}
\def\g{\gamma}

%問題番号のためのマクロ

\newcounter{nombre} %必須
\renewcommand{\thenombre}{\arabic{nombre}} %任意
\setcounter{nombre}{2} %任意
\newcounter{nombresub}[nombre] %親子関係を定義
\renewcommand{\thenombresub}{\arabic{nombresub}} %任意
\setcounter{nombresub}{0} %任意
\newcommand{\prob}[1][]{\refstepcounter{nombre}#1[問題 \thenombre]}
\newcommand{\probsub}[1][]{\refstepcounter{nombresub}#1(\thenombresub)}


%1-1みたいなカウンタ(todaiとtodaia)
\newcounter{todai}
\setcounter{todai}{0}
\newcounter{todaisub}[todai] 
\setcounter{todaisub}{0} 
\newcommand{\todai}[1][]{\refstepcounter{todai}#1 \thetodai-\thetodaisub}
\newcommand{\todaib}[1][]{\refstepcounter{todai}#1\refstepcounter{todaisub}#1 {\bf [問題 \thetodai.\thetodaisub]}}
\newcommand{\todaia}[1][]{\refstepcounter{todaisub}#1 {\bf [問題 \thetodai.\thetodaisub]}}


     \begin{oframed}
     負でない実数$r$,$l$に対して,$xy$平面上の曲線
          \begin{align*}
          y=
               \begin{cases}
               x^2&(0\le x\le r) \\
               r^2 &(r\le x\le l+r) \\
               (x-l-2r)^2&(l+r\le x\le l+2r)
               \end{cases}
          \end{align*}
     を考え,これを$x$軸の周りに回転してできる回転体の体積を$V$とする.
     $r^2$と$l$の和が正の定数$c$になるように$r$と$l$を変化させるとき,$V$
     の最大値を与えるような$r$と$l$の値を求めよ.
     \end{oframed}

\setlength{\columnseprule}{0.4pt}
\begin{multicols}{2}
{\bf[解]} グラフの概形は下図.

     \begin{zahyou}[ul=15mm](-0.5,3.5)(-0.5,1.5)
     \def\Fx{X*X}
     \def\Gx{(X-3)**2}
     \YGurafu\Fx{0}{1}
     \YGurafu{1}{1}{2}
     \YGurafu\Gx{2}{3}
     \Put{(1,1)}[syaei=xy,xlabel=r,ylabel=r^2]{}
     \Put{(2,1)}[syaei=x,xlabel=r+l]{}
     \Put{(3,0)}[s]{$l+2r$}
     \end{zahyou}

また,題意の条件から
     \begin{align}
     r^2+l=c\label{1}
     \end{align}
である.求める体積は,対称性から
     \begin{align*}
     \frac{V}{\pi}=2\int_0^r(x^2)^2\,dx+(r^2)^2l
     \end{align*}
であるから\eqref{1}を代入して
     \begin{align*}
     \frac{V}{\pi}&=2\int_0^rx^4\,dx+r^4(c-r^2)\equiv f(r) 
     \end{align*}
として
     \begin{align*}
     f'(r)&=2r^4+4cr^3-6r^5 \\
     &=2r^3(-3r^2+r+2c)
     \end{align*}
である.ただし,$l,r\ge0$から,
     \begin{align}
     0\le r^2 \le c \nonumber\\
     0\le r\le \sqrt{c} \label{0}
     \end{align}
である.ここで
     \begin{align}
     -3r^2+r+2c=0 \label{2}
     \end{align}
の$2$実解を$\a$,$\b$とする.($\a<\b$)

$f(0)=2c>0$だから,$\a<0<\b$であり,
     \begin{align*}
     &\b=\frac{1+\sqrt{1+24c}}{6}\le\sqrt{c} \\
     &1+\sqrt{1+24c}\le 6\sqrt{c} \\
     &2+24c+2\sqrt{1+24c}\le 36c \\
     &0\le c(c-1)
     \end{align*}
に注意して,$c$の値によって場合分けする.\\

     \begin{indentation}{2zw}{0pt}
     \noindent\underline{(i)$1\le c$の時} \\
     下表を得る.
          \[
          \begin{array}{|c|c|c|c|c|c|}\hline
          r   &0 &     &\b  &      &\sqrt{c} \\ \hline
          f'  &   & +   & 0 & -    &            \\ \hline
          f   &  &\ne &     &\se &            \\ \hline 
          \end{array}
          \]
     従って,$r=\b$のとき$V$は最大.\\
     
     \noindent\underline{(ii)$0<c\le 1$の時} \\
     下表を得る.
          \[
          \begin{array}{|c|c|c|c|}\hline
          r   &0 &     &\sqrt{c} \\ \hline
          f'  &   & +   &            \\ \hline
          f   &   &\ne &            \\ \hline 
          \end{array}
          \]    
     故に$r=\sqrt{c}$のとき$V$は最大.
     \end{indentation}

\vspace{1zh}     
以上および\eqref{1}にも注意して,
     \begin{align*}
     c\ge1&\,\,\,\,
          \begin{cases}
          r=\dfrac{1+\sqrt{1+24c}}{6} \\
          l=\dfrac{6c-1-\sqrt{1+24c}}{18} 
          \end{cases}\\
     0<c\le1&\,\,\,\,
          \begin{cases}
          r=\sqrt{c} \\
          l=0
          \end{cases}
     \end{align*}
のとき$V$は最大である.$\cdots$(答)
\newpage
\end{multicols}
\end{document}