\documentclass[a4j]{jarticle}
\usepackage{amsmath}
\usepackage{ascmac}
\usepackage{amssymb}
\usepackage{enumerate}
\usepackage{multicol}
\usepackage{framed}
\usepackage{fancyhdr}
\usepackage{latexsym}
\usepackage{indent}
\usepackage{cases}
\usepackage[dvips]{graphicx}
\usepackage{color}
\usepackage{emath}
\usepackage{emathPp}
\allowdisplaybreaks
\pagestyle{fancy}
\lhead{}
\chead{}
\rhead{東京大学前期$1976$年$2$番}
\begin{document}
%分数関係


\def\tfrac#1#2{{\textstyle\frac{#1}{#2}}} %数式中で文中表示の分数を使う時


%Σ関係

\def\dsum#1#2{{\displaystyle\sum_{#1}^{#2}}} %文中で数式表示のΣを使う時


%ベクトル関係


\def\vector#1{\overrightarrow{#1}}  %ベクトルを表現したいとき(aベクトルを表現するときは\ver
\def\norm#1{|\overrightarrow{#1}|} %ベクトルの絶対値
\def\vtwo#1#2{ \left(%
      \begin{array}{c}%
      #1 \\%
      #2 \\%
      \end{array}%
      \right) }                        %2次元ベクトル成分表示
      
      \def\vthree#1#2#3{ \left(
      \begin{array}{c}
      #1 \\
      #2 \\
      #3 \\
      \end{array}
      \right) }                        %3次元ベクトル成分表示



%数列関係


\def\an#1{\verb|{|$#1$\verb|}|}


%極限関係

\def\limit#1#2{\stackrel{#1 \to #2}{\longrightarrow}}   %等式変形からの極限
\def\dlim#1#2{{\displaystyle \lim_{#1\to#2}}} %文中で数式表示の極限を使う



%積分関係

\def\dint#1#2{{\displaystyle \int_{#1}^{#2}}} %文中で数式表示の積分を使う時

\def\ne{\nearrow}
\def\se{\searrow}
\def\nw{\nwarrow}
\def\ne{\nearrow}


%便利なやつ

\def\case#1#2{%
 \[\left\{%
 \begin{array}{l}%
 #1 \\%
 #2%
 \end{array}%
 \right.\] }                           %場合分け
 
\def\1{$\cos\theta=c$,$\sin\theta=s$とおく.}  %cs表示を与える前書きシータ
\def\2{$\cos t=c$,$\sin t=s$とおく.}     %cs表示を与える前書きt
\def\3{$\cos x=c$,$\sin x=s$とおく.}                %cs表示を与える前書きx

\def\fig#1#2#3 {%
\begin{wrapfigure}[#1]{r}{#2 zw}%
\vspace*{-1zh}%
\input{#3}%
\end{wrapfigure} }           %絵の挿入


\def\a{\alpha}   %ギリシャ文字
\def\b{\beta}
\def\g{\gamma}

%問題番号のためのマクロ

\newcounter{nombre} %必須
\renewcommand{\thenombre}{\arabic{nombre}} %任意
\setcounter{nombre}{2} %任意
\newcounter{nombresub}[nombre] %親子関係を定義
\renewcommand{\thenombresub}{\arabic{nombresub}} %任意
\setcounter{nombresub}{0} %任意
\newcommand{\prob}[1][]{\refstepcounter{nombre}#1[問題 \thenombre]}
\newcommand{\probsub}[1][]{\refstepcounter{nombresub}#1(\thenombresub)}


%1-1みたいなカウンタ(todaiとtodaia)
\newcounter{todai}
\setcounter{todai}{0}
\newcounter{todaisub}[todai] 
\setcounter{todaisub}{0} 
\newcommand{\todai}[1][]{\refstepcounter{todai}#1 \thetodai-\thetodaisub}
\newcommand{\todaib}[1][]{\refstepcounter{todai}#1\refstepcounter{todaisub}#1 {\bf [問題 \thetodai.\thetodaisub]}}
\newcommand{\todaia}[1][]{\refstepcounter{todaisub}#1 {\bf [問題 \thetodai.\thetodaisub]}}


     \begin{oframed}
     時刻$t=0$に原点を出発し,$xy$平面上で次の条件(i),(ii)に従っていろいろに運動する動点Pがある.
          \begin{enumerate}[(i)]
          \item $t=0$におけるPの速度を表すベクトルの成分は$(1,\sqrt{3})$である.
          \item $0<t<1$において,Pは何回か($1$回以上有限回)直角に左折するが,そのときを除けばPは
          一定の速さ$2$で直進する.(ただし,左折するのに要する時間は$0$とする)
          \end{enumerate}
     このとき,時刻$t=1$においてPが到達する点をQとして,Qの存在しうる範囲を図示せよ.
     \end{oframed}

\setlength{\columnseprule}{0.4pt}
\begin{multicols}{2}
{\bf[解]} $\beku{a}=(1,\sqrt{3})$とおく.まず,$t=0$におけるPの速度を表すベクトルが$\beku{b}=(1,0)$だとして考える.$x$軸正方向の移動量を$x$,$y$軸正方向の移動量を$y$とすると,題意の条件から
     \[|x|+|y|\le 2\]
である.これを図示して下図斜線部(境界含む).

     \begin{zahyou}[ul=10mm](-2.5,2.5)(-2.5,2.5)
     \def\Fx{-1*X+2}
     \def\Gx{X+2}
     \def\Hx{-1*X-2}
     \def\Mx{X-2}
     \YGurafu\Fx{0}{2}
     \YGurafu\Gx{-2}{0}
     \YGurafu\Hx{-2}{0}
     \YGurafu\Mx{0}{2}
     \YNurii*\Fx\Mx{0}{2}
     \YNurii*\Gx\Hx{-2}{0}
     \Put{(2,0)}[ne]{$2$}
     \Put{(0,2)}[ne]{$2$}
     \Put{(-2,0)}[nw]{$-2$}
     \Put{(0,-2)}[se]{$-2$}
     \end{zahyou}
     
これを$\beku{b}$が$\beku{a}$の方向にくるまで回転した領域が求める領域で,図示して下図斜線部(境界含む).
$\cdots$(答)

     \begin{zahyou}[ul=10mm](-2.5,2.5)(-2.5,2.5)
     \tenretu*<perl>{%
     A(1,sqrt(3));B(-sqrt(3),1);C(-1,-sqrt(3));D(sqrt(3),-1)}
     \Drawline{\A\B\C\D\A}
     \Put\A[syaei=xy,xlabel=1,ylabel=\sqrt{3}]{}
     \Nuritubusi*{\A\B\C\D}
     \end{zahyou}

\newpage
\end{multicols}
\end{document}