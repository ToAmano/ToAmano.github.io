\documentclass[a4j]{jarticle}
\usepackage{amsmath}
\usepackage{ascmac}
\usepackage{amssymb}
\usepackage{enumerate}
\usepackage{multicol}
\usepackage{framed}
\usepackage{fancyhdr}
\usepackage{latexsym}
\usepackage{indent}
\usepackage{cases}
\usepackage[dvips]{graphicx}
\usepackage{color}
\usepackage{emath}
\usepackage{emathPp}
\allowdisplaybreaks
\pagestyle{fancy}
\lhead{}
\chead{}
\rhead{東京大学前期$1974$年$5$番}
\begin{document}
%分数関係


\def\tfrac#1#2{{\textstyle\frac{#1}{#2}}} %数式中で文中表示の分数を使う時


%Σ関係

\def\dsum#1#2{{\displaystyle\sum_{#1}^{#2}}} %文中で数式表示のΣを使う時


%ベクトル関係


\def\vector#1{\overrightarrow{#1}}  %ベクトルを表現したいとき(aベクトルを表現するときは\ver
\def\norm#1{|\overrightarrow{#1}|} %ベクトルの絶対値
\def\vtwo#1#2{ \left(%
      \begin{array}{c}%
      #1 \\%
      #2 \\%
      \end{array}%
      \right) }                        %2次元ベクトル成分表示
      
      \def\vthree#1#2#3{ \left(
      \begin{array}{c}
      #1 \\
      #2 \\
      #3 \\
      \end{array}
      \right) }                        %3次元ベクトル成分表示



%数列関係


\def\an#1{\verb|{|$#1$\verb|}|}


%極限関係

\def\limit#1#2{\stackrel{#1 \to #2}{\longrightarrow}}   %等式変形からの極限
\def\dlim#1#2{{\displaystyle \lim_{#1\to#2}}} %文中で数式表示の極限を使う



%積分関係

\def\dint#1#2{{\displaystyle \int_{#1}^{#2}}} %文中で数式表示の積分を使う時

\def\ne{\nearrow}
\def\se{\searrow}
\def\nw{\nwarrow}
\def\ne{\nearrow}


%便利なやつ

\def\case#1#2{%
 \[\left\{%
 \begin{array}{l}%
 #1 \\%
 #2%
 \end{array}%
 \right.\] }                           %場合分け
 
\def\1{$\cos\theta=c$,$\sin\theta=s$とおく.}  %cs表示を与える前書きシータ
\def\2{$\cos t=c$,$\sin t=s$とおく.}     %cs表示を与える前書きt
\def\3{$\cos x=c$,$\sin x=s$とおく.}                %cs表示を与える前書きx

\def\fig#1#2#3 {%
\begin{wrapfigure}[#1]{r}{#2 zw}%
\vspace*{-1zh}%
\input{#3}%
\end{wrapfigure} }           %絵の挿入


\def\a{\alpha}   %ギリシャ文字
\def\b{\beta}
\def\g{\gamma}

%問題番号のためのマクロ

\newcounter{nombre} %必須
\renewcommand{\thenombre}{\arabic{nombre}} %任意
\setcounter{nombre}{2} %任意
\newcounter{nombresub}[nombre] %親子関係を定義
\renewcommand{\thenombresub}{\arabic{nombresub}} %任意
\setcounter{nombresub}{0} %任意
\newcommand{\prob}[1][]{\refstepcounter{nombre}#1[問題 \thenombre]}
\newcommand{\probsub}[1][]{\refstepcounter{nombresub}#1(\thenombresub)}


%1-1みたいなカウンタ(todaiとtodaia)
\newcounter{todai}
\setcounter{todai}{0}
\newcounter{todaisub}[todai] 
\setcounter{todaisub}{0} 
\newcommand{\todai}[1][]{\refstepcounter{todai}#1 \thetodai-\thetodaisub}
\newcommand{\todaib}[1][]{\refstepcounter{todai}#1\refstepcounter{todaisub}#1 {\bf [問題 \thetodai.\thetodaisub]}}
\newcommand{\todaia}[1][]{\refstepcounter{todaisub}#1 {\bf [問題 \thetodai.\thetodaisub]}}


     \begin{oframed}
     原点Oに中心を持つ半径$2$の固定された円板を$A$とする.半径$1$の円板$B$を,その中心Cが点
     $(3,0)$に重なるように置くとき,点$(4,0)$に重なる$B$の周上の点をMとする.$B$を,$A$の周囲
     にそって滑らないように転がして,OCが$x$軸の正の方向と成す角が$\theta$になったときの,Mの
     位置の座標を$(X,Y)$とする.
          
     $\theta$が$0$から$\pi/2$まで動くとして,次の問いに答えよ.
          \begin{enumerate}[(1)]
          \item $X$と$Y$とを$\theta$の関数として表せ.
          \item $Y$の最大値を求めよ.
          \item Mの描く曲線の弧の長さを求めよ.
          \end{enumerate}     
     \end{oframed}

\setlength{\columnseprule}{0.4pt}
\begin{multicols}{2}
{\bf[解]} \1 グラフの概形は下図.

従って,
     \begin{align*}
          \begin{cases}
          X=3c+\cos 3\theta=3c^3 \\
          Y=3s+\sin 3\theta=6s-4s^3
          \end{cases}
     \end{align*}     
である.$\cdots$(答)

従って,  
     \begin{align*}
          \begin{cases}
          X'=-9c^2s \\
          Y'=6c(1-2s^2)
          \end{cases}
     \end{align*}
であり,下表を得る.
      \begin{align*}
           \begin{array}{|c|c|c|c|c|c|}\hline
           \theta & 0     &      & \pi/4                     &      &\pi/2 \\  \hline
           X'       &0     & -     & -                          & -    & 0     \\ \hline
           Y'       & +    & +    & 0                         &  -   & 0    \\ \hline
           (X,Y) &(4,0) &\nw &(\sqrt{2},2\sqrt{2})&\sw &(0,2) \\ \hline
           \end{array}
      \end{align*}
      
 従って,$\max Y=2\sqrt{2}$である.$\cdots$(答)\\
 
 さて,求める孤長$L$とすると,
      \begin{align*}
      L&=\int_0^{\pi/2}\sqrt{X'^2+Y'^2}\,d\theta \\
      &=\int_0^{\pi/2}\sqrt{81c^4s^2+36c^2(1-2s^2)^2}\,d\theta \\
      &=\int_0^{\pi/2}\sqrt{36c^2}\,d\theta \\
      &=\int_0^{\pi/2}6c\,d\theta \,\,\,\,\,\,\, (\because c\ge0)\\
      &=6
      \end{align*}
 である.$\cdots$(答)     
\newpage
\end{multicols}
\end{document}