\documentclass[a4j]{jarticle}
\usepackage{amsmath}
\usepackage{ascmac}
\usepackage{amssymb}
\usepackage{enumerate}
\usepackage{multicol}
\usepackage{framed}
\usepackage{fancyhdr}
\usepackage{latexsym}
\usepackage{indent}
\usepackage{cases}
\usepackage[dvips]{graphicx}
\usepackage{color}
\allowdisplaybreaks
\pagestyle{fancy}
\lhead{}
\chead{}
\rhead{東京大学前期$1965$年$4$番}
\begin{document}
%分数関係


\def\tfrac#1#2{{\textstyle\frac{#1}{#2}}} %数式中で文中表示の分数を使う時


%Σ関係

\def\dsum#1#2{{\displaystyle\sum_{#1}^{#2}}} %文中で数式表示のΣを使う時


%ベクトル関係


\def\vector#1{\overrightarrow{#1}}  %ベクトルを表現したいとき(aベクトルを表現するときは\ver
\def\norm#1{|\overrightarrow{#1}|} %ベクトルの絶対値
\def\vtwo#1#2{ \left(%
      \begin{array}{c}%
      #1 \\%
      #2 \\%
      \end{array}%
      \right) }                        %2次元ベクトル成分表示
      
      \def\vthree#1#2#3{ \left(
      \begin{array}{c}
      #1 \\
      #2 \\
      #3 \\
      \end{array}
      \right) }                        %3次元ベクトル成分表示



%数列関係


\def\an#1{\verb|{|$#1$\verb|}|}


%極限関係

\def\limit#1#2{\stackrel{#1 \to #2}{\longrightarrow}}   %等式変形からの極限
\def\dlim#1#2{{\displaystyle \lim_{#1\to#2}}} %文中で数式表示の極限を使う



%積分関係

\def\dint#1#2{{\displaystyle \int_{#1}^{#2}}} %文中で数式表示の積分を使う時

\def\ne{\nearrow}
\def\se{\searrow}
\def\nw{\nwarrow}
\def\ne{\nearrow}


%便利なやつ

\def\case#1#2{%
 \[\left\{%
 \begin{array}{l}%
 #1 \\%
 #2%
 \end{array}%
 \right.\] }                           %場合分け
 
\def\1{$\cos\theta=c$,$\sin\theta=s$とおく.}  %cs表示を与える前書きシータ
\def\2{$\cos t=c$,$\sin t=s$とおく.}     %cs表示を与える前書きt
\def\3{$\cos x=c$,$\sin x=s$とおく.}                %cs表示を与える前書きx

\def\fig#1#2#3 {%
\begin{wrapfigure}[#1]{r}{#2 zw}%
\vspace*{-1zh}%
\input{#3}%
\end{wrapfigure} }           %絵の挿入


\def\a{\alpha}   %ギリシャ文字
\def\b{\beta}
\def\g{\gamma}

%問題番号のためのマクロ

\newcounter{nombre} %必須
\renewcommand{\thenombre}{\arabic{nombre}} %任意
\setcounter{nombre}{2} %任意
\newcounter{nombresub}[nombre] %親子関係を定義
\renewcommand{\thenombresub}{\arabic{nombresub}} %任意
\setcounter{nombresub}{0} %任意
\newcommand{\prob}[1][]{\refstepcounter{nombre}#1[問題 \thenombre]}
\newcommand{\probsub}[1][]{\refstepcounter{nombresub}#1(\thenombresub)}


%1-1みたいなカウンタ(todaiとtodaia)
\newcounter{todai}
\setcounter{todai}{0}
\newcounter{todaisub}[todai] 
\setcounter{todaisub}{0} 
\newcommand{\todai}[1][]{\refstepcounter{todai}#1 \thetodai-\thetodaisub}
\newcommand{\todaib}[1][]{\refstepcounter{todai}#1\refstepcounter{todaisub}#1 {\bf [問題 \thetodai.\thetodaisub]}}
\newcommand{\todaia}[1][]{\refstepcounter{todaisub}#1 {\bf [問題 \thetodai.\thetodaisub]}}


     \begin{oframed}
     下の図において$ABC$は長さ$2$の線分$AB$を直径とし,$O$を中心とする半円周,$P$は$AB$に垂直な半径$OC$上の
     動点とする.
     
     $k$を正の定数とし,線分$PO$を$k:1$に内分する点$Q$を通って$AB$に平行な弦を$RS$とすれば,$P$をどこにとったとき
     四辺形$ROSP$の面積が最大になるか.
          \begin{center}
          \scalebox{.7}{%WinTpicVersion4.32a
{\unitlength 0.1in%
\begin{picture}(40.0000,20.0000)(6.0000,-24.0000)%
% FUNC 2 0 3 0 Black White  
% 9 600 400 4600 2400 2600 2200 4200 2200 2600 600 600 400 4600 2400 50 4 0 2
% cos(t)///sin(t)///0///pi
\special{pn 8}%
\special{pa 4200 2200}%
\special{pa 4200 2170}%
\special{pa 4199 2160}%
\special{pa 4199 2140}%
\special{pa 4198 2130}%
\special{pa 4198 2120}%
\special{pa 4197 2109}%
\special{pa 4197 2099}%
\special{pa 4195 2079}%
\special{pa 4195 2069}%
\special{pa 4189 2009}%
\special{pa 4187 1999}%
\special{pa 4185 1979}%
\special{pa 4183 1969}%
\special{pa 4182 1960}%
\special{pa 4180 1950}%
\special{pa 4179 1940}%
\special{pa 4173 1910}%
\special{pa 4172 1900}%
\special{pa 4168 1880}%
\special{pa 4166 1871}%
\special{pa 4164 1861}%
\special{pa 4161 1851}%
\special{pa 4157 1831}%
\special{pa 4155 1822}%
\special{pa 4152 1812}%
\special{pa 4150 1802}%
\special{pa 4147 1792}%
\special{pa 4145 1783}%
\special{pa 4139 1763}%
\special{pa 4136 1754}%
\special{pa 4134 1744}%
\special{pa 4131 1734}%
\special{pa 4128 1725}%
\special{pa 4122 1705}%
\special{pa 4119 1696}%
\special{pa 4115 1686}%
\special{pa 4112 1677}%
\special{pa 4109 1667}%
\special{pa 4105 1658}%
\special{pa 4102 1648}%
\special{pa 4098 1639}%
\special{pa 4095 1630}%
\special{pa 4091 1620}%
\special{pa 4088 1611}%
\special{pa 4084 1602}%
\special{pa 4080 1592}%
\special{pa 4072 1574}%
\special{pa 4068 1564}%
\special{pa 4048 1519}%
\special{pa 4043 1510}%
\special{pa 4039 1501}%
\special{pa 4035 1491}%
\special{pa 4030 1482}%
\special{pa 4026 1474}%
\special{pa 4016 1456}%
\special{pa 4012 1447}%
\special{pa 3997 1420}%
\special{pa 3992 1412}%
\special{pa 3977 1385}%
\special{pa 3972 1377}%
\special{pa 3967 1368}%
\special{pa 3962 1360}%
\special{pa 3956 1351}%
\special{pa 3951 1343}%
\special{pa 3945 1334}%
\special{pa 3940 1326}%
\special{pa 3934 1317}%
\special{pa 3929 1309}%
\special{pa 3923 1301}%
\special{pa 3918 1292}%
\special{pa 3900 1268}%
\special{pa 3894 1259}%
\special{pa 3852 1203}%
\special{pa 3845 1196}%
\special{pa 3833 1180}%
\special{pa 3826 1172}%
\special{pa 3820 1165}%
\special{pa 3813 1157}%
\special{pa 3807 1149}%
\special{pa 3800 1142}%
\special{pa 3793 1134}%
\special{pa 3787 1127}%
\special{pa 3780 1119}%
\special{pa 3766 1105}%
\special{pa 3759 1097}%
\special{pa 3731 1069}%
\special{pa 3724 1061}%
\special{pa 3703 1040}%
\special{pa 3695 1034}%
\special{pa 3688 1027}%
\special{pa 3680 1020}%
\special{pa 3666 1006}%
\special{pa 3658 1000}%
\special{pa 3650 993}%
\special{pa 3643 987}%
\special{pa 3635 980}%
\special{pa 3628 974}%
\special{pa 3620 967}%
\special{pa 3612 961}%
\special{pa 3604 954}%
\special{pa 3588 942}%
\special{pa 3581 936}%
\special{pa 3573 930}%
\special{pa 3565 923}%
\special{pa 3557 917}%
\special{pa 3548 911}%
\special{pa 3540 906}%
\special{pa 3508 882}%
\special{pa 3499 877}%
\special{pa 3483 865}%
\special{pa 3474 860}%
\special{pa 3466 854}%
\special{pa 3457 849}%
\special{pa 3449 844}%
\special{pa 3440 838}%
\special{pa 3432 833}%
\special{pa 3414 823}%
\special{pa 3406 818}%
\special{pa 3388 808}%
\special{pa 3380 803}%
\special{pa 3353 788}%
\special{pa 3344 784}%
\special{pa 3326 774}%
\special{pa 3317 770}%
\special{pa 3308 765}%
\special{pa 3290 757}%
\special{pa 3281 752}%
\special{pa 3245 736}%
\special{pa 3235 732}%
\special{pa 3208 720}%
\special{pa 3198 716}%
\special{pa 3189 712}%
\special{pa 3180 709}%
\special{pa 3170 705}%
\special{pa 3161 701}%
\special{pa 3151 698}%
\special{pa 3142 695}%
\special{pa 3132 691}%
\special{pa 3123 688}%
\special{pa 3113 685}%
\special{pa 3104 681}%
\special{pa 3094 678}%
\special{pa 3085 675}%
\special{pa 3075 672}%
\special{pa 3066 669}%
\special{pa 3056 666}%
\special{pa 3046 664}%
\special{pa 3037 661}%
\special{pa 3017 655}%
\special{pa 3008 653}%
\special{pa 2998 650}%
\special{pa 2988 648}%
\special{pa 2978 645}%
\special{pa 2969 643}%
\special{pa 2949 639}%
\special{pa 2939 636}%
\special{pa 2919 632}%
\special{pa 2910 630}%
\special{pa 2890 626}%
\special{pa 2880 625}%
\special{pa 2860 621}%
\special{pa 2850 620}%
\special{pa 2840 618}%
\special{pa 2830 617}%
\special{pa 2820 615}%
\special{pa 2800 613}%
\special{pa 2790 611}%
\special{pa 2730 605}%
\special{pa 2720 605}%
\special{pa 2700 603}%
\special{pa 2690 603}%
\special{pa 2680 602}%
\special{pa 2670 602}%
\special{pa 2660 601}%
\special{pa 2640 601}%
\special{pa 2630 600}%
\special{pa 2570 600}%
\special{pa 2560 601}%
\special{pa 2540 601}%
\special{pa 2530 602}%
\special{pa 2520 602}%
\special{pa 2509 603}%
\special{pa 2499 603}%
\special{pa 2479 605}%
\special{pa 2469 605}%
\special{pa 2409 611}%
\special{pa 2399 613}%
\special{pa 2379 615}%
\special{pa 2369 617}%
\special{pa 2360 618}%
\special{pa 2350 620}%
\special{pa 2340 621}%
\special{pa 2310 627}%
\special{pa 2300 628}%
\special{pa 2280 632}%
\special{pa 2271 634}%
\special{pa 2261 636}%
\special{pa 2251 639}%
\special{pa 2231 643}%
\special{pa 2222 645}%
\special{pa 2212 648}%
\special{pa 2202 650}%
\special{pa 2192 653}%
\special{pa 2183 655}%
\special{pa 2163 661}%
\special{pa 2154 664}%
\special{pa 2144 666}%
\special{pa 2134 669}%
\special{pa 2125 672}%
\special{pa 2105 678}%
\special{pa 2096 681}%
\special{pa 2086 685}%
\special{pa 2077 688}%
\special{pa 2067 691}%
\special{pa 2058 695}%
\special{pa 2048 698}%
\special{pa 2039 702}%
\special{pa 2030 705}%
\special{pa 2020 709}%
\special{pa 2011 712}%
\special{pa 2002 716}%
\special{pa 1992 720}%
\special{pa 1974 728}%
\special{pa 1964 732}%
\special{pa 1919 752}%
\special{pa 1910 757}%
\special{pa 1901 761}%
\special{pa 1891 765}%
\special{pa 1882 770}%
\special{pa 1874 774}%
\special{pa 1856 784}%
\special{pa 1847 788}%
\special{pa 1820 803}%
\special{pa 1812 808}%
\special{pa 1785 823}%
\special{pa 1777 828}%
\special{pa 1768 833}%
\special{pa 1760 838}%
\special{pa 1751 844}%
\special{pa 1743 849}%
\special{pa 1734 855}%
\special{pa 1726 860}%
\special{pa 1717 866}%
\special{pa 1709 871}%
\special{pa 1701 877}%
\special{pa 1692 882}%
\special{pa 1668 900}%
\special{pa 1659 906}%
\special{pa 1603 948}%
\special{pa 1596 955}%
\special{pa 1580 967}%
\special{pa 1572 974}%
\special{pa 1565 980}%
\special{pa 1557 987}%
\special{pa 1549 993}%
\special{pa 1542 1000}%
\special{pa 1534 1007}%
\special{pa 1527 1013}%
\special{pa 1519 1020}%
\special{pa 1505 1034}%
\special{pa 1497 1041}%
\special{pa 1469 1069}%
\special{pa 1461 1076}%
\special{pa 1440 1097}%
\special{pa 1434 1105}%
\special{pa 1427 1112}%
\special{pa 1420 1120}%
\special{pa 1406 1134}%
\special{pa 1400 1142}%
\special{pa 1393 1150}%
\special{pa 1387 1157}%
\special{pa 1380 1165}%
\special{pa 1374 1172}%
\special{pa 1367 1180}%
\special{pa 1361 1188}%
\special{pa 1354 1196}%
\special{pa 1342 1212}%
\special{pa 1336 1219}%
\special{pa 1330 1227}%
\special{pa 1323 1235}%
\special{pa 1317 1243}%
\special{pa 1311 1252}%
\special{pa 1306 1260}%
\special{pa 1282 1292}%
\special{pa 1277 1301}%
\special{pa 1265 1317}%
\special{pa 1260 1326}%
\special{pa 1254 1334}%
\special{pa 1249 1343}%
\special{pa 1244 1351}%
\special{pa 1238 1360}%
\special{pa 1233 1368}%
\special{pa 1223 1386}%
\special{pa 1218 1394}%
\special{pa 1208 1412}%
\special{pa 1203 1420}%
\special{pa 1188 1447}%
\special{pa 1184 1456}%
\special{pa 1174 1474}%
\special{pa 1170 1483}%
\special{pa 1165 1492}%
\special{pa 1157 1510}%
\special{pa 1152 1519}%
\special{pa 1136 1555}%
\special{pa 1132 1565}%
\special{pa 1120 1592}%
\special{pa 1116 1602}%
\special{pa 1112 1611}%
\special{pa 1109 1620}%
\special{pa 1105 1630}%
\special{pa 1101 1639}%
\special{pa 1098 1649}%
\special{pa 1095 1658}%
\special{pa 1091 1668}%
\special{pa 1088 1677}%
\special{pa 1085 1687}%
\special{pa 1081 1696}%
\special{pa 1078 1706}%
\special{pa 1075 1715}%
\special{pa 1072 1725}%
\special{pa 1069 1734}%
\special{pa 1066 1744}%
\special{pa 1064 1754}%
\special{pa 1061 1763}%
\special{pa 1055 1783}%
\special{pa 1053 1792}%
\special{pa 1050 1802}%
\special{pa 1048 1812}%
\special{pa 1045 1822}%
\special{pa 1043 1831}%
\special{pa 1039 1851}%
\special{pa 1036 1861}%
\special{pa 1032 1881}%
\special{pa 1030 1890}%
\special{pa 1026 1910}%
\special{pa 1025 1920}%
\special{pa 1021 1940}%
\special{pa 1020 1950}%
\special{pa 1018 1960}%
\special{pa 1017 1970}%
\special{pa 1015 1980}%
\special{pa 1013 2000}%
\special{pa 1011 2010}%
\special{pa 1005 2070}%
\special{pa 1005 2080}%
\special{pa 1003 2100}%
\special{pa 1003 2110}%
\special{pa 1002 2120}%
\special{pa 1002 2130}%
\special{pa 1001 2140}%
\special{pa 1001 2160}%
\special{pa 1000 2170}%
\special{pa 1000 2200}%
\special{fp}%
% FUNC 2 0 3 0 Black White  
% 10 600 400 4600 2400 2600 2200 4200 2200 2600 600 1000 400 4200 2400 0 4 0 0 0 0
% 0
\special{pn 8}%
\special{pn 8}%
\special{pa 600 2200}%
\special{pa 609 2200}%
\special{ip}%
\special{pa 650 2200}%
\special{pa 659 2200}%
\special{ip}%
\special{pa 700 2200}%
\special{pa 709 2200}%
\special{ip}%
\special{pa 750 2200}%
\special{pa 759 2200}%
\special{ip}%
\special{pa 800 2200}%
\special{pa 809 2200}%
\special{ip}%
\special{pa 850 2200}%
\special{pa 859 2200}%
\special{ip}%
\special{pa 900 2200}%
\special{pa 909 2200}%
\special{ip}%
\special{pa 950 2200}%
\special{pa 959 2200}%
\special{ip}%
\special{pa 1000 2200}%
\special{pa 1000 2200}%
\special{ip}%
\special{pa 1000 2200}%
\special{pa 4200 2200}%
\special{fp}%
\special{pn 8}%
\special{pa 4209 2200}%
\special{pa 4250 2200}%
\special{ip}%
\special{pa 4259 2200}%
\special{pa 4300 2200}%
\special{ip}%
\special{pa 4309 2200}%
\special{pa 4350 2200}%
\special{ip}%
\special{pa 4359 2200}%
\special{pa 4400 2200}%
\special{ip}%
\special{pa 4409 2200}%
\special{pa 4450 2200}%
\special{ip}%
\special{pa 4459 2200}%
\special{pa 4500 2200}%
\special{ip}%
\special{pa 4509 2200}%
\special{pa 4550 2200}%
\special{ip}%
\special{pa 4559 2200}%
\special{pa 4600 2200}%
\special{ip}%
% FUNC 2 0 3 0 Black White  
% 10 600 400 4600 2400 2600 2200 4200 2200 2600 600 1214 400 3986 2400 0 4 0 0 0 0
% 0.5
\special{pn 8}%
\special{pn 8}%
\special{pa 600 1400}%
\special{pa 608 1400}%
\special{ip}%
\special{pa 647 1400}%
\special{pa 655 1400}%
\special{ip}%
\special{pa 694 1400}%
\special{pa 702 1400}%
\special{ip}%
\special{pa 741 1400}%
\special{pa 749 1400}%
\special{ip}%
\special{pa 788 1400}%
\special{pa 796 1400}%
\special{ip}%
\special{pa 835 1400}%
\special{pa 843 1400}%
\special{ip}%
\special{pa 882 1400}%
\special{pa 890 1400}%
\special{ip}%
\special{pa 928 1400}%
\special{pa 937 1400}%
\special{ip}%
\special{pa 975 1400}%
\special{pa 984 1400}%
\special{ip}%
\special{pa 1022 1400}%
\special{pa 1031 1400}%
\special{ip}%
\special{pa 1069 1400}%
\special{pa 1078 1400}%
\special{ip}%
\special{pa 1116 1400}%
\special{pa 1124 1400}%
\special{ip}%
\special{pa 1163 1400}%
\special{pa 1171 1400}%
\special{ip}%
\special{ip}%
\special{pa 1210 1400}%
\special{pa 3985 1400}%
\special{fp}%
\special{pn 8}%
\special{pa 3993 1400}%
\special{pa 4032 1400}%
\special{ip}%
\special{pa 4041 1400}%
\special{pa 4080 1400}%
\special{ip}%
\special{pa 4088 1400}%
\special{pa 4127 1400}%
\special{ip}%
\special{pa 4135 1400}%
\special{pa 4174 1400}%
\special{ip}%
\special{pa 4183 1400}%
\special{pa 4222 1400}%
\special{ip}%
\special{pa 4230 1400}%
\special{pa 4269 1400}%
\special{ip}%
\special{pa 4277 1400}%
\special{pa 4316 1400}%
\special{ip}%
\special{pa 4325 1400}%
\special{pa 4363 1400}%
\special{ip}%
\special{pa 4372 1400}%
\special{pa 4411 1400}%
\special{ip}%
\special{pa 4419 1400}%
\special{pa 4458 1400}%
\special{ip}%
\special{pa 4466 1400}%
\special{pa 4505 1400}%
\special{ip}%
\special{pa 4514 1400}%
\special{pa 4553 1400}%
\special{ip}%
\special{pa 4561 1400}%
\special{pa 4600 1400}%
\special{ip}%
% FUNC 2 0 3 0 Black White  
% 10 600 400 4600 2400 2600 2200 4200 2200 2600 600 2600 400 3986 2400 0 4 0 0 0 0
% x/sqrt(3)
\special{pn 8}%
\special{pn 8}%
\special{pa 2254 2400}%
\special{pa 2255 2399}%
\special{pa 2261 2395}%
\special{ip}%
\special{pa 2297 2375}%
\special{pa 2305 2370}%
\special{ip}%
\special{pa 2340 2350}%
\special{pa 2348 2345}%
\special{ip}%
\special{pa 2384 2325}%
\special{pa 2391 2320}%
\special{ip}%
\special{pa 2427 2300}%
\special{pa 2435 2295}%
\special{ip}%
\special{pa 2470 2275}%
\special{pa 2478 2270}%
\special{ip}%
\special{pa 2513 2250}%
\special{pa 2521 2245}%
\special{ip}%
\special{pa 2557 2225}%
\special{pa 2564 2220}%
\special{ip}%
\special{pa 2600 2200}%
\special{pa 2600 2200}%
\special{ip}%
\special{pa 2600 2200}%
\special{pa 2620 2188}%
\special{pa 2625 2186}%
\special{pa 2665 2162}%
\special{pa 2670 2160}%
\special{pa 2710 2136}%
\special{pa 2715 2134}%
\special{pa 2750 2113}%
\special{pa 2755 2111}%
\special{pa 2795 2087}%
\special{pa 2800 2085}%
\special{pa 2840 2061}%
\special{pa 2845 2059}%
\special{pa 2885 2035}%
\special{pa 2890 2033}%
\special{pa 2930 2009}%
\special{pa 2935 2007}%
\special{pa 2975 1983}%
\special{pa 2980 1981}%
\special{pa 3015 1960}%
\special{pa 3020 1958}%
\special{pa 3060 1934}%
\special{pa 3065 1932}%
\special{pa 3105 1908}%
\special{pa 3110 1906}%
\special{pa 3150 1882}%
\special{pa 3155 1880}%
\special{pa 3195 1856}%
\special{pa 3200 1854}%
\special{pa 3240 1830}%
\special{pa 3245 1828}%
\special{pa 3280 1807}%
\special{pa 3285 1805}%
\special{pa 3325 1781}%
\special{pa 3330 1779}%
\special{pa 3370 1755}%
\special{pa 3375 1753}%
\special{pa 3415 1729}%
\special{pa 3420 1727}%
\special{pa 3460 1703}%
\special{pa 3465 1701}%
\special{pa 3505 1677}%
\special{pa 3510 1675}%
\special{pa 3545 1654}%
\special{pa 3550 1652}%
\special{pa 3590 1628}%
\special{pa 3595 1626}%
\special{pa 3635 1602}%
\special{pa 3640 1600}%
\special{pa 3680 1576}%
\special{pa 3685 1574}%
\special{pa 3725 1550}%
\special{pa 3730 1548}%
\special{pa 3765 1527}%
\special{pa 3770 1525}%
\special{pa 3810 1501}%
\special{pa 3815 1499}%
\special{pa 3855 1475}%
\special{pa 3860 1473}%
\special{pa 3900 1449}%
\special{pa 3905 1447}%
\special{pa 3945 1423}%
\special{pa 3950 1421}%
\special{pa 3985 1400}%
\special{fp}%
\special{pn 8}%
\special{pa 3992 1396}%
\special{pa 3995 1395}%
\special{pa 4026 1376}%
\special{ip}%
\special{pa 4034 1373}%
\special{pa 4035 1372}%
\special{pa 4067 1353}%
\special{ip}%
\special{pa 4074 1348}%
\special{pa 4075 1348}%
\special{pa 4080 1346}%
\special{pa 4108 1329}%
\special{ip}%
\special{pa 4115 1325}%
\special{pa 4120 1322}%
\special{pa 4125 1320}%
\special{pa 4149 1306}%
\special{ip}%
\special{pa 4156 1301}%
\special{pa 4165 1296}%
\special{pa 4170 1294}%
\special{pa 4190 1282}%
\special{ip}%
\special{pa 4197 1278}%
\special{pa 4210 1270}%
\special{pa 4215 1268}%
\special{pa 4231 1258}%
\special{ip}%
\special{pa 4238 1254}%
\special{pa 4255 1244}%
\special{pa 4260 1242}%
\special{pa 4272 1235}%
\special{ip}%
\special{pa 4279 1230}%
\special{pa 4295 1221}%
\special{pa 4300 1219}%
\special{pa 4313 1211}%
\special{ip}%
\special{pa 4320 1207}%
\special{pa 4340 1195}%
\special{pa 4345 1193}%
\special{pa 4354 1187}%
\special{ip}%
\special{pa 4361 1183}%
\special{pa 4385 1169}%
\special{pa 4390 1167}%
\special{pa 4395 1164}%
\special{ip}%
\special{pa 4402 1160}%
\special{pa 4430 1143}%
\special{pa 4435 1141}%
\special{pa 4436 1140}%
\special{ip}%
\special{pa 4444 1136}%
\special{pa 4475 1117}%
\special{pa 4477 1116}%
\special{ip}%
\special{pa 4485 1112}%
\special{pa 4518 1092}%
\special{ip}%
\special{pa 4526 1089}%
\special{pa 4559 1069}%
\special{ip}%
\special{pa 4567 1065}%
\special{pa 4600 1045}%
\special{ip}%
% FUNC 2 0 3 0 Black White  
% 10 600 400 4600 2400 2600 2200 4200 2200 2600 600 1214 400 2600 2400 0 4 0 0 0 0
% -x/sqrt(3)
\special{pn 8}%
\special{pn 8}%
\special{pa 600 1045}%
\special{pa 607 1049}%
\special{ip}%
\special{pa 641 1068}%
\special{pa 648 1073}%
\special{ip}%
\special{pa 681 1092}%
\special{pa 689 1096}%
\special{ip}%
\special{pa 722 1116}%
\special{pa 725 1117}%
\special{pa 729 1120}%
\special{ip}%
\special{pa 762 1139}%
\special{pa 765 1141}%
\special{pa 770 1143}%
\special{ip}%
\special{pa 803 1163}%
\special{pa 810 1167}%
\special{pa 810 1167}%
\special{ip}%
\special{pa 844 1186}%
\special{pa 851 1191}%
\special{ip}%
\special{pa 884 1210}%
\special{pa 892 1214}%
\special{ip}%
\special{pa 925 1233}%
\special{pa 932 1237}%
\special{ip}%
\special{pa 966 1257}%
\special{pa 973 1261}%
\special{ip}%
\special{pa 1007 1280}%
\special{pa 1014 1284}%
\special{ip}%
\special{pa 1047 1303}%
\special{pa 1054 1308}%
\special{ip}%
\special{pa 1088 1327}%
\special{pa 1095 1331}%
\special{ip}%
\special{pa 1129 1350}%
\special{pa 1136 1354}%
\special{ip}%
\special{pa 1169 1374}%
\special{pa 1170 1374}%
\special{pa 1176 1378}%
\special{ip}%
\special{pa 1210 1397}%
\special{pa 1210 1397}%
\special{ip}%
\special{pa 1210 1397}%
\special{pa 1250 1421}%
\special{pa 1255 1423}%
\special{pa 1295 1447}%
\special{pa 1300 1449}%
\special{pa 1340 1473}%
\special{pa 1345 1475}%
\special{pa 1385 1499}%
\special{pa 1390 1501}%
\special{pa 1430 1525}%
\special{pa 1435 1527}%
\special{pa 1470 1548}%
\special{pa 1475 1550}%
\special{pa 1515 1574}%
\special{pa 1520 1576}%
\special{pa 1560 1600}%
\special{pa 1565 1602}%
\special{pa 1605 1626}%
\special{pa 1610 1628}%
\special{pa 1650 1652}%
\special{pa 1655 1654}%
\special{pa 1690 1675}%
\special{pa 1695 1677}%
\special{pa 1735 1701}%
\special{pa 1740 1703}%
\special{pa 1780 1727}%
\special{pa 1785 1729}%
\special{pa 1825 1753}%
\special{pa 1830 1755}%
\special{pa 1870 1779}%
\special{pa 1875 1781}%
\special{pa 1915 1805}%
\special{pa 1920 1807}%
\special{pa 1955 1828}%
\special{pa 1960 1830}%
\special{pa 2000 1854}%
\special{pa 2005 1856}%
\special{pa 2045 1880}%
\special{pa 2050 1882}%
\special{pa 2090 1906}%
\special{pa 2095 1908}%
\special{pa 2135 1932}%
\special{pa 2140 1934}%
\special{pa 2180 1958}%
\special{pa 2185 1960}%
\special{pa 2220 1981}%
\special{pa 2225 1983}%
\special{pa 2265 2007}%
\special{pa 2270 2009}%
\special{pa 2310 2033}%
\special{pa 2315 2035}%
\special{pa 2355 2059}%
\special{pa 2360 2061}%
\special{pa 2400 2085}%
\special{pa 2405 2087}%
\special{pa 2445 2111}%
\special{pa 2450 2113}%
\special{pa 2485 2134}%
\special{pa 2490 2136}%
\special{pa 2530 2160}%
\special{pa 2535 2162}%
\special{pa 2575 2186}%
\special{pa 2580 2188}%
\special{pa 2600 2200}%
\special{fp}%
\special{pn 8}%
\special{pa 2608 2205}%
\special{pa 2620 2212}%
\special{pa 2625 2214}%
\special{pa 2643 2225}%
\special{ip}%
\special{pa 2651 2230}%
\special{pa 2665 2238}%
\special{pa 2670 2240}%
\special{pa 2687 2250}%
\special{ip}%
\special{pa 2694 2255}%
\special{pa 2710 2264}%
\special{pa 2715 2266}%
\special{pa 2730 2275}%
\special{ip}%
\special{pa 2737 2279}%
\special{pa 2750 2287}%
\special{pa 2755 2289}%
\special{pa 2773 2300}%
\special{ip}%
\special{pa 2781 2304}%
\special{pa 2795 2313}%
\special{pa 2800 2315}%
\special{pa 2816 2325}%
\special{ip}%
\special{pa 2824 2329}%
\special{pa 2840 2339}%
\special{pa 2845 2341}%
\special{pa 2860 2350}%
\special{ip}%
\special{pa 2867 2354}%
\special{pa 2885 2365}%
\special{pa 2890 2367}%
\special{pa 2903 2375}%
\special{ip}%
\special{pa 2911 2379}%
\special{pa 2930 2391}%
\special{pa 2935 2393}%
\special{pa 2945 2399}%
\special{pa 2946 2400}%
\special{ip}%
% FUNC 2 0 3 0 Black White  
% 10 600 400 4600 2400 2600 2200 4200 2200 2600 600 600 600 4600 2200 0 4 0 1 0 0
% 0
\special{pn 8}%
\special{pn 8}%
\special{pa 2600 400}%
\special{pa 2600 409}%
\special{ip}%
\special{pa 2600 450}%
\special{pa 2600 459}%
\special{ip}%
\special{pa 2600 500}%
\special{pa 2600 509}%
\special{ip}%
\special{pa 2600 550}%
\special{pa 2600 559}%
\special{ip}%
\special{ip}%
\special{pa 2600 600}%
\special{pa 2600 2200}%
\special{fp}%
\special{pn 8}%
\special{pa 2600 2209}%
\special{pa 2600 2250}%
\special{ip}%
\special{pa 2600 2259}%
\special{pa 2600 2300}%
\special{ip}%
\special{pa 2600 2309}%
\special{pa 2600 2350}%
\special{ip}%
\special{pa 2600 2359}%
\special{pa 2600 2400}%
\special{ip}%
% LINE 2 0 3 0 Black White  
% 4 1220 1400 2600 1000 2600 1000 3970 1400
% 
\special{pn 8}%
\special{pa 1220 1400}%
\special{pa 2600 1000}%
\special{fp}%
\special{pa 2600 1000}%
\special{pa 3970 1400}%
\special{fp}%
% STR 2 0 3 0 Black White  
% 4 4200 2100 4200 2200 1 0 1 0
% $B$
\put(42.0000,-22.0000){\makebox(0,0)[lt]{{\colorbox[named]{White}{$B$}}}}%
% STR 2 0 3 0 Black White  
% 4 2600 2100 2600 2200 1 0 1 0
% $O$
\put(26.0000,-22.0000){\makebox(0,0)[lt]{{\colorbox[named]{White}{$O$}}}}%
% STR 2 0 3 0 Black White  
% 4 1000 2100 1000 2200 1 0 1 0
% $A$
\put(10.0000,-22.0000){\makebox(0,0)[lt]{{\colorbox[named]{White}{$A$}}}}%
% STR 2 0 3 0 Black White  
% 4 2600 1300 2600 1400 2 0 1 0
% $Q$
\put(26.0000,-14.0000){\makebox(0,0)[lb]{{\colorbox[named]{White}{$Q$}}}}%
% STR 2 0 3 0 Black White  
% 4 2600 900 2600 1000 2 0 1 0
% $P$
\put(26.0000,-10.0000){\makebox(0,0)[lb]{{\colorbox[named]{White}{$P$}}}}%
% STR 2 0 3 0 Black White  
% 4 2600 500 2600 600 2 0 1 0
% $C$
\put(26.0000,-6.0000){\makebox(0,0)[lb]{{\colorbox[named]{White}{$C$}}}}%
% STR 2 0 3 0 Black White  
% 4 3960 1300 3960 1400 2 0 1 0
% $S$
\put(39.6000,-14.0000){\makebox(0,0)[lb]{{\colorbox[named]{White}{$S$}}}}%
% STR 2 0 3 0 Black White  
% 4 1220 1300 1220 1400 3 0 1 0
% $R$
\put(12.2000,-14.0000){\makebox(0,0)[rb]{{\colorbox[named]{White}{$R$}}}}%
\end{picture}}%
}
          \end{center}
     \end{oframed}

\setlength{\columnseprule}{0.4pt}
\begin{multicols}{2}
{\bf[解]} $O$を原点とし,$B$を$x$軸正方向,$C$を$y$軸正方向とする座標系を考える.$P(0,(1+k)t)$とおく.
すると,まず$0\le(1+k)t\le1$より
     \begin{align}
     0\le t\le\frac{1}{1+k}\label{1}
     \end{align}
である.また,内分点の条件から$Q(0,t)$であるから,$S$の$x$座標は
      \begin{align*}
      x^2+t^2=1 \\
      x=\sqrt{1-t^2}
      \end{align*}
である.故に四辺形の面積$f(t)$として
     \begin{align*}
     f(t)&=\frac{1}{2}|OP||RS| \\
     &=|OP||QS| \\
     &=(1+k)t\sqrt{1-t^2} \\
     &=(1+k)\sqrt{-\left(t^2-\frac{1}{2}\right)^2+\frac{1}{4}}
     \end{align*} 
グラフの概形は下図のようである.
     \begin{center}
     \scalebox{.5}{%WinTpicVersion4.32a
{\unitlength 0.1in%
\begin{picture}(20.0000,24.0000)(4.0000,-28.0000)%
% FUNC 2 0 3 0 Black White  
% 9 400 400 2400 2800 800 2400 2000 2400 800 1200 400 400 2400 2800 0 2 0 0
% -5x^2+5x
\special{pn 8}%
\special{pa 725 2800}%
\special{pa 725 2798}%
\special{pa 730 2770}%
\special{pa 735 2743}%
\special{pa 740 2715}%
\special{pa 745 2688}%
\special{pa 750 2660}%
\special{pa 755 2633}%
\special{pa 760 2607}%
\special{pa 765 2580}%
\special{pa 790 2450}%
\special{pa 810 2350}%
\special{pa 835 2230}%
\special{pa 840 2207}%
\special{pa 845 2183}%
\special{pa 850 2160}%
\special{pa 855 2138}%
\special{pa 860 2115}%
\special{pa 865 2093}%
\special{pa 870 2070}%
\special{pa 875 2048}%
\special{pa 880 2027}%
\special{pa 885 2005}%
\special{pa 910 1900}%
\special{pa 930 1820}%
\special{pa 955 1725}%
\special{pa 960 1707}%
\special{pa 965 1688}%
\special{pa 970 1670}%
\special{pa 975 1653}%
\special{pa 980 1635}%
\special{pa 985 1618}%
\special{pa 990 1600}%
\special{pa 995 1583}%
\special{pa 1000 1567}%
\special{pa 1005 1550}%
\special{pa 1030 1470}%
\special{pa 1050 1410}%
\special{pa 1075 1340}%
\special{pa 1080 1327}%
\special{pa 1085 1313}%
\special{pa 1090 1300}%
\special{pa 1095 1288}%
\special{pa 1100 1275}%
\special{pa 1105 1263}%
\special{pa 1110 1250}%
\special{pa 1115 1238}%
\special{pa 1120 1227}%
\special{pa 1125 1215}%
\special{pa 1150 1160}%
\special{pa 1170 1120}%
\special{pa 1195 1075}%
\special{pa 1200 1067}%
\special{pa 1205 1058}%
\special{pa 1210 1050}%
\special{pa 1215 1043}%
\special{pa 1220 1035}%
\special{pa 1225 1028}%
\special{pa 1230 1020}%
\special{pa 1235 1013}%
\special{pa 1240 1007}%
\special{pa 1245 1000}%
\special{pa 1270 970}%
\special{pa 1290 950}%
\special{pa 1315 930}%
\special{pa 1320 927}%
\special{pa 1325 923}%
\special{pa 1330 920}%
\special{pa 1335 918}%
\special{pa 1340 915}%
\special{pa 1345 913}%
\special{pa 1350 910}%
\special{pa 1355 908}%
\special{pa 1360 907}%
\special{pa 1365 905}%
\special{pa 1390 900}%
\special{pa 1410 900}%
\special{pa 1435 905}%
\special{pa 1440 907}%
\special{pa 1445 908}%
\special{pa 1450 910}%
\special{pa 1455 913}%
\special{pa 1460 915}%
\special{pa 1465 918}%
\special{pa 1470 920}%
\special{pa 1475 923}%
\special{pa 1480 927}%
\special{pa 1485 930}%
\special{pa 1510 950}%
\special{pa 1530 970}%
\special{pa 1555 1000}%
\special{pa 1560 1007}%
\special{pa 1565 1013}%
\special{pa 1570 1020}%
\special{pa 1575 1028}%
\special{pa 1580 1035}%
\special{pa 1585 1043}%
\special{pa 1590 1050}%
\special{pa 1595 1058}%
\special{pa 1600 1067}%
\special{pa 1605 1075}%
\special{pa 1630 1120}%
\special{pa 1650 1160}%
\special{pa 1675 1215}%
\special{pa 1680 1227}%
\special{pa 1685 1238}%
\special{pa 1690 1250}%
\special{pa 1695 1263}%
\special{pa 1700 1275}%
\special{pa 1705 1288}%
\special{pa 1710 1300}%
\special{pa 1715 1313}%
\special{pa 1720 1327}%
\special{pa 1725 1340}%
\special{pa 1750 1410}%
\special{pa 1770 1470}%
\special{pa 1795 1550}%
\special{pa 1800 1567}%
\special{pa 1805 1583}%
\special{pa 1810 1600}%
\special{pa 1815 1618}%
\special{pa 1820 1635}%
\special{pa 1825 1653}%
\special{pa 1830 1670}%
\special{pa 1835 1688}%
\special{pa 1840 1707}%
\special{pa 1845 1725}%
\special{pa 1870 1820}%
\special{pa 1890 1900}%
\special{pa 1915 2005}%
\special{pa 1920 2027}%
\special{pa 1925 2048}%
\special{pa 1930 2070}%
\special{pa 1935 2093}%
\special{pa 1940 2115}%
\special{pa 1945 2138}%
\special{pa 1950 2160}%
\special{pa 1955 2183}%
\special{pa 1960 2207}%
\special{pa 1965 2230}%
\special{pa 1990 2350}%
\special{pa 2010 2450}%
\special{pa 2035 2580}%
\special{pa 2040 2607}%
\special{pa 2045 2633}%
\special{pa 2050 2660}%
\special{pa 2055 2688}%
\special{pa 2060 2715}%
\special{pa 2065 2743}%
\special{pa 2070 2770}%
\special{pa 2075 2798}%
\special{pa 2075 2800}%
\special{fp}%
% FUNC 2 0 3 0 Black White  
% 10 400 400 2400 2800 800 2400 2000 2400 800 1200 400 401 2400 2800 0 2 0 1 0 0
% 0.5
\special{pn 8}%
\special{pa 1400 400}%
\special{pa 1400 2800}%
\special{fp}%
% FUNC 2 0 3 0 Black White  
% 10 400 400 2400 2800 800 2400 2000 2400 800 1200 400 2400 2400 2400 0 0 0 1 0 0
% 0.25
\special{pn 8}%
\special{pn 8}%
\special{pa 1100 400}%
\special{pa 1100 408}%
\special{fp}%
\special{pa 1100 445}%
\special{pa 1100 454}%
\special{fp}%
\special{pa 1100 491}%
\special{pa 1100 499}%
\special{fp}%
\special{pa 1100 536}%
\special{pa 1100 544}%
\special{fp}%
\special{pa 1100 582}%
\special{pa 1100 590}%
\special{fp}%
\special{pa 1100 627}%
\special{pa 1100 635}%
\special{fp}%
\special{pa 1100 673}%
\special{pa 1100 681}%
\special{fp}%
\special{pa 1100 718}%
\special{pa 1100 726}%
\special{fp}%
\special{pa 1100 764}%
\special{pa 1100 772}%
\special{fp}%
\special{pa 1100 809}%
\special{pa 1100 817}%
\special{fp}%
\special{pa 1100 855}%
\special{pa 1100 863}%
\special{fp}%
\special{pa 1100 900}%
\special{pa 1100 908}%
\special{fp}%
\special{pa 1100 945}%
\special{pa 1100 954}%
\special{fp}%
\special{pa 1100 991}%
\special{pa 1100 999}%
\special{fp}%
\special{pa 1100 1036}%
\special{pa 1100 1044}%
\special{fp}%
\special{pa 1100 1082}%
\special{pa 1100 1090}%
\special{fp}%
\special{pa 1100 1127}%
\special{pa 1100 1135}%
\special{fp}%
\special{pa 1100 1173}%
\special{pa 1100 1181}%
\special{fp}%
\special{pa 1100 1218}%
\special{pa 1100 1226}%
\special{fp}%
\special{pa 1100 1264}%
\special{pa 1100 1272}%
\special{fp}%
\special{pa 1100 1309}%
\special{pa 1100 1317}%
\special{fp}%
\special{pa 1100 1355}%
\special{pa 1100 1363}%
\special{fp}%
\special{pa 1100 1400}%
\special{pa 1100 1408}%
\special{fp}%
\special{pa 1100 1445}%
\special{pa 1100 1454}%
\special{fp}%
\special{pa 1100 1491}%
\special{pa 1100 1499}%
\special{fp}%
\special{pa 1100 1536}%
\special{pa 1100 1544}%
\special{fp}%
\special{pa 1100 1582}%
\special{pa 1100 1590}%
\special{fp}%
\special{pa 1100 1627}%
\special{pa 1100 1635}%
\special{fp}%
\special{pa 1100 1673}%
\special{pa 1100 1681}%
\special{fp}%
\special{pa 1100 1718}%
\special{pa 1100 1726}%
\special{fp}%
\special{pa 1100 1764}%
\special{pa 1100 1772}%
\special{fp}%
\special{pa 1100 1809}%
\special{pa 1100 1817}%
\special{fp}%
\special{pa 1100 1855}%
\special{pa 1100 1863}%
\special{fp}%
\special{pa 1100 1900}%
\special{pa 1100 1908}%
\special{fp}%
\special{pa 1100 1945}%
\special{pa 1100 1954}%
\special{fp}%
\special{pa 1100 1991}%
\special{pa 1100 1999}%
\special{fp}%
\special{pa 1100 2036}%
\special{pa 1100 2044}%
\special{fp}%
\special{pa 1100 2082}%
\special{pa 1100 2090}%
\special{fp}%
\special{pa 1100 2127}%
\special{pa 1100 2135}%
\special{fp}%
\special{pa 1100 2173}%
\special{pa 1100 2181}%
\special{fp}%
\special{pa 1100 2218}%
\special{pa 1100 2226}%
\special{fp}%
\special{pa 1100 2264}%
\special{pa 1100 2272}%
\special{fp}%
\special{pa 1100 2309}%
\special{pa 1100 2317}%
\special{fp}%
\special{pa 1100 2355}%
\special{pa 1100 2363}%
\special{fp}%
\special{pa 1100 2400}%
\special{pa 1100 2400}%
\special{fp}%
\special{pn 8}%
\special{pa 1100 2409}%
\special{pa 1100 2450}%
\special{fp}%
\special{pa 1100 2459}%
\special{pa 1100 2500}%
\special{fp}%
\special{pa 1100 2509}%
\special{pa 1100 2550}%
\special{fp}%
\special{pa 1100 2559}%
\special{pa 1100 2600}%
\special{fp}%
\special{pa 1100 2609}%
\special{pa 1100 2650}%
\special{fp}%
\special{pa 1100 2659}%
\special{pa 1100 2700}%
\special{fp}%
\special{pa 1100 2709}%
\special{pa 1100 2750}%
\special{fp}%
\special{pa 1100 2759}%
\special{pa 1100 2800}%
\special{fp}%
% STR 2 0 3 0 Black White  
% 4 1540 620 1540 720 5 0 1 0
% $1/2$
\put(15.4000,-7.2000){\makebox(0,0){{\colorbox[named]{White}{$1/2$}}}}%
% STR 2 0 3 0 Black White  
% 4 760 870 760 970 5 0 1 0
% $1/(1+k)^2$
\put(7.6000,-9.7000){\makebox(0,0){{\colorbox[named]{White}{$1/(1+k)^2$}}}}%
\end{picture}}%
}
     \end{center}
故に,$k$の値によってこの最大値は以下のようになる.
     \begin{indentation}{2zw}{0pt}
     \noindent\underline{(i)$1/(1+k)^2\le1/2\therefore-1+\sqrt{2}\le k$の時}\\
     $t^2=1/(1+k)^2$の時,つまり$P=C$の時$f(t)$は最大である.
     \\ \\
     \underline{(ii)$1/2\le1/(1+k)^2\therefore-1+\sqrt{2}\ge k$の時}\\
     $t^2=1/2$つまり$t=1/\sqrt{2}$,$P$の$y$座標が$(k+1)/\sqrt{2}$の時,$f(t)$は最大である.    
     \end{indentation}
以上から,求める$P$の位置は,
     \begin{align*}
          \begin{cases}
          |PO|=\dfrac{k+1}{\sqrt{2}}&(k\le-1+\sqrt{2}) \\
          P=C&(-1+\sqrt{2}\le k)
          \end{cases}
     \end{align*} 
である.
\newpage
\end{multicols}
\end{document}