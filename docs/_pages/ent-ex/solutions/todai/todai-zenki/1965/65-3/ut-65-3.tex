\documentclass[a4j]{jarticle}
\usepackage{amsmath}
\usepackage{ascmac}
\usepackage{amssymb}
\usepackage{enumerate}
\usepackage{multicol}
\usepackage{framed}
\usepackage{fancyhdr}
\usepackage{latexsym}
\usepackage{indent}
\usepackage{cases}
\allowdisplaybreaks
\pagestyle{fancy}
\lhead{}
\chead{}
\rhead{東京大学前期$1965$年$3$番}
\begin{document}
%分数関係


\def\tfrac#1#2{{\textstyle\frac{#1}{#2}}} %数式中で文中表示の分数を使う時


%Σ関係

\def\dsum#1#2{{\displaystyle\sum_{#1}^{#2}}} %文中で数式表示のΣを使う時


%ベクトル関係


\def\vector#1{\overrightarrow{#1}}  %ベクトルを表現したいとき(aベクトルを表現するときは\ver
\def\norm#1{|\overrightarrow{#1}|} %ベクトルの絶対値
\def\vtwo#1#2{ \left(%
      \begin{array}{c}%
      #1 \\%
      #2 \\%
      \end{array}%
      \right) }                        %2次元ベクトル成分表示
      
      \def\vthree#1#2#3{ \left(
      \begin{array}{c}
      #1 \\
      #2 \\
      #3 \\
      \end{array}
      \right) }                        %3次元ベクトル成分表示



%数列関係


\def\an#1{\verb|{|$#1$\verb|}|}


%極限関係

\def\limit#1#2{\stackrel{#1 \to #2}{\longrightarrow}}   %等式変形からの極限
\def\dlim#1#2{{\displaystyle \lim_{#1\to#2}}} %文中で数式表示の極限を使う



%積分関係

\def\dint#1#2{{\displaystyle \int_{#1}^{#2}}} %文中で数式表示の積分を使う時

\def\ne{\nearrow}
\def\se{\searrow}
\def\nw{\nwarrow}
\def\ne{\nearrow}


%便利なやつ

\def\case#1#2{%
 \[\left\{%
 \begin{array}{l}%
 #1 \\%
 #2%
 \end{array}%
 \right.\] }                           %場合分け
 
\def\1{$\cos\theta=c$,$\sin\theta=s$とおく.}  %cs表示を与える前書きシータ
\def\2{$\cos t=c$,$\sin t=s$とおく.}     %cs表示を与える前書きt
\def\3{$\cos x=c$,$\sin x=s$とおく.}                %cs表示を与える前書きx

\def\fig#1#2#3 {%
\begin{wrapfigure}[#1]{r}{#2 zw}%
\vspace*{-1zh}%
\input{#3}%
\end{wrapfigure} }           %絵の挿入


\def\a{\alpha}   %ギリシャ文字
\def\b{\beta}
\def\g{\gamma}

%問題番号のためのマクロ

\newcounter{nombre} %必須
\renewcommand{\thenombre}{\arabic{nombre}} %任意
\setcounter{nombre}{2} %任意
\newcounter{nombresub}[nombre] %親子関係を定義
\renewcommand{\thenombresub}{\arabic{nombresub}} %任意
\setcounter{nombresub}{0} %任意
\newcommand{\prob}[1][]{\refstepcounter{nombre}#1[問題 \thenombre]}
\newcommand{\probsub}[1][]{\refstepcounter{nombresub}#1(\thenombresub)}


%1-1みたいなカウンタ(todaiとtodaia)
\newcounter{todai}
\setcounter{todai}{0}
\newcounter{todaisub}[todai] 
\setcounter{todaisub}{0} 
\newcommand{\todai}[1][]{\refstepcounter{todai}#1 \thetodai-\thetodaisub}
\newcommand{\todaib}[1][]{\refstepcounter{todai}#1\refstepcounter{todaisub}#1 {\bf [問題 \thetodai.\thetodaisub]}}
\newcommand{\todaia}[1][]{\refstepcounter{todaisub}#1 {\bf [問題 \thetodai.\thetodaisub]}}


     \begin{oframed}
     直線$l$は双曲線$xy=1$の第一象限にある部分に接し,$l$と$x$軸との交点の$x$座標は$2$より小
     さくないとする.
     
     この条件のもとで$l$が変動するとき,四直線$l$,$y=0$,$x=1$および$x=2$で囲まれる
     部分の面積の最大値を求めよ.
     \end{oframed}

\setlength{\columnseprule}{0.4pt}
\begin{multicols}{2}
{\bf[解]}題意の接点を$P(t,1/t)$とする.ただし$t>0$とする.このとき$(1/x)'=-1/x^2$だから
     \begin{align*}
     l:y=\frac{-1}{t^2}(x-t)+\frac{1}{t} \\
     \therefore l:y=\frac{-1}{t^2}x+\frac{2}{t}\equiv f(x)
     \end{align*}
となるので,これと$x$軸の交点は$Q(2t,0)$となる.題意の条件より
     \begin{align}
     1\le2t\label{1}
     \end{align}
となる.題意の面積を$S(t)$とすると,$t$の値で場合分けして以下のようになる.
     \begin{indentation}{2zw}{0pt}
     \noindent \underline{(i)$1\le2t\le2$つまり$1/2\le t\le 1$の時}  \\
     グラフが右図のようになるので
          \begin{align*}
          S(t)&=\int_1^{2t}f(x)dx  \\
          &=\left[\frac{-1}{2t^2}x^2+\frac{2}{t}x\right]_1^{2t} \\
          &=\frac{-1}{2t^2}(4t^2-1)+\frac{2}{t}(2t-1)  \\
          &=2-\frac{2}{t}+\frac{1}{2t^2}  \\
          &=\frac{1}{2}(s-2)^2
          \end{align*}
     となる.ただし$s=1/t$として,$1\le s\le2$である.この時の最大値は,
     $s=1$のときの$\frac{1}{2}$である.     
     
     \noindent \underline{(ii)$2\le2t$つまり$1\le t$の時}  \\
          \begin{align*}
          S(t)&=\int_1^2f(x)dx \\
          &=\left[\frac{-1}{2t^2}x^2+\frac{2}{t}x\right]_1^2  \\
          &=\frac{-3}{2t^2}+\frac{2}{t}  \\
          &=\frac{-3}{2}\left(s-\frac{2}{3}\right)^2+\frac{2}{3}
          \end{align*}
      となる.この時は$0<s\le1$だから$s=2/3$で最大値$2/3$をとる.
     \end{indentation}          
以上から$\max S(t)=S(3/2)=2/3\cdots$(答)である.
\newpage
\end{multicols}
\end{document}