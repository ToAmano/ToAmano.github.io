\documentclass[a4j]{jarticle}
\usepackage{amsmath}
\usepackage{ascmac}
\usepackage{amssymb}
\usepackage{enumerate}
\usepackage{multicol}
\usepackage{framed}
\usepackage{fancyhdr}
\usepackage{latexsym}
\usepackage{indent}
\usepackage{cases}
\usepackage[dvips]{graphicx}
\usepackage{color}
\allowdisplaybreaks
\pagestyle{fancy}
\lhead{}
\chead{}
\rhead{東京大学前期$1965$年$6$番}
\begin{document}
%分数関係


\def\tfrac#1#2{{\textstyle\frac{#1}{#2}}} %数式中で文中表示の分数を使う時


%Σ関係

\def\dsum#1#2{{\displaystyle\sum_{#1}^{#2}}} %文中で数式表示のΣを使う時


%ベクトル関係


\def\vector#1{\overrightarrow{#1}}  %ベクトルを表現したいとき(aベクトルを表現するときは\ver
\def\norm#1{|\overrightarrow{#1}|} %ベクトルの絶対値
\def\vtwo#1#2{ \left(%
      \begin{array}{c}%
      #1 \\%
      #2 \\%
      \end{array}%
      \right) }                        %2次元ベクトル成分表示
      
      \def\vthree#1#2#3{ \left(
      \begin{array}{c}
      #1 \\
      #2 \\
      #3 \\
      \end{array}
      \right) }                        %3次元ベクトル成分表示



%数列関係


\def\an#1{\verb|{|$#1$\verb|}|}


%極限関係

\def\limit#1#2{\stackrel{#1 \to #2}{\longrightarrow}}   %等式変形からの極限
\def\dlim#1#2{{\displaystyle \lim_{#1\to#2}}} %文中で数式表示の極限を使う



%積分関係

\def\dint#1#2{{\displaystyle \int_{#1}^{#2}}} %文中で数式表示の積分を使う時

\def\ne{\nearrow}
\def\se{\searrow}
\def\nw{\nwarrow}
\def\ne{\nearrow}


%便利なやつ

\def\case#1#2{%
 \[\left\{%
 \begin{array}{l}%
 #1 \\%
 #2%
 \end{array}%
 \right.\] }                           %場合分け
 
\def\1{$\cos\theta=c$,$\sin\theta=s$とおく.}  %cs表示を与える前書きシータ
\def\2{$\cos t=c$,$\sin t=s$とおく.}     %cs表示を与える前書きt
\def\3{$\cos x=c$,$\sin x=s$とおく.}                %cs表示を与える前書きx

\def\fig#1#2#3 {%
\begin{wrapfigure}[#1]{r}{#2 zw}%
\vspace*{-1zh}%
\input{#3}%
\end{wrapfigure} }           %絵の挿入


\def\a{\alpha}   %ギリシャ文字
\def\b{\beta}
\def\g{\gamma}

%問題番号のためのマクロ

\newcounter{nombre} %必須
\renewcommand{\thenombre}{\arabic{nombre}} %任意
\setcounter{nombre}{2} %任意
\newcounter{nombresub}[nombre] %親子関係を定義
\renewcommand{\thenombresub}{\arabic{nombresub}} %任意
\setcounter{nombresub}{0} %任意
\newcommand{\prob}[1][]{\refstepcounter{nombre}#1[問題 \thenombre]}
\newcommand{\probsub}[1][]{\refstepcounter{nombresub}#1(\thenombresub)}


%1-1みたいなカウンタ(todaiとtodaia)
\newcounter{todai}
\setcounter{todai}{0}
\newcounter{todaisub}[todai] 
\setcounter{todaisub}{0} 
\newcommand{\todai}[1][]{\refstepcounter{todai}#1 \thetodai-\thetodaisub}
\newcommand{\todaib}[1][]{\refstepcounter{todai}#1\refstepcounter{todaisub}#1 {\bf [問題 \thetodai.\thetodaisub]}}
\newcommand{\todaia}[1][]{\refstepcounter{todaisub}#1 {\bf [問題 \thetodai.\thetodaisub]}}


     \begin{oframed}
     下の図は半径の長さ$1$の半円で,弦$AP$,$AQ$と直径$AB$のつくる角はそれぞれ$30^\circ$,$60^\circ$である.
     
     このとき,弦$AP$,$AQ$と円弧$PQ$とで囲まれる部分を直径$AB$のまわりに一回転して得られる立体の体積を求めよ. 
          \begin{center}
          \scalebox{1}{%WinTpicVersion4.32a
{\unitlength 0.1in%
\begin{picture}(20.0000,12.0000)(4.0000,-14.0000)%
% FUNC 2 0 3 0 Black White  
% 10 400 200 2400 1400 1400 1200 2200 1200 1400 400 600 200 1000 1400 0 4 0 0 0 0
% sqrt(3)(x+1)
\special{pn 8}%
\special{pn 8}%
\special{pa 485 1400}%
\special{pa 485 1399}%
\special{pa 489 1393}%
\special{ip}%
\special{pa 508 1360}%
\special{pa 512 1353}%
\special{ip}%
\special{pa 531 1320}%
\special{pa 535 1313}%
\special{pa 535 1313}%
\special{ip}%
\special{pa 554 1280}%
\special{pa 558 1273}%
\special{ip}%
\special{pa 577 1240}%
\special{pa 580 1235}%
\special{pa 581 1233}%
\special{ip}%
\special{pa 600 1200}%
\special{pa 605 1191}%
\special{pa 610 1183}%
\special{pa 620 1165}%
\special{pa 625 1157}%
\special{pa 635 1139}%
\special{pa 640 1131}%
\special{pa 650 1113}%
\special{pa 655 1105}%
\special{pa 665 1087}%
\special{pa 670 1079}%
\special{pa 680 1061}%
\special{pa 685 1053}%
\special{pa 695 1035}%
\special{pa 700 1027}%
\special{pa 710 1009}%
\special{pa 715 1001}%
\special{pa 725 983}%
\special{pa 730 975}%
\special{pa 735 966}%
\special{pa 740 958}%
\special{pa 750 940}%
\special{pa 755 932}%
\special{pa 765 914}%
\special{pa 770 906}%
\special{pa 780 888}%
\special{pa 785 880}%
\special{pa 795 862}%
\special{pa 800 854}%
\special{pa 810 836}%
\special{pa 815 828}%
\special{pa 825 810}%
\special{pa 830 802}%
\special{pa 840 784}%
\special{pa 845 776}%
\special{pa 855 758}%
\special{pa 860 750}%
\special{pa 870 732}%
\special{pa 875 724}%
\special{pa 885 706}%
\special{pa 890 698}%
\special{pa 900 680}%
\special{pa 905 672}%
\special{pa 915 654}%
\special{pa 920 646}%
\special{pa 930 628}%
\special{pa 935 620}%
\special{pa 945 602}%
\special{pa 950 594}%
\special{pa 960 576}%
\special{pa 965 568}%
\special{pa 975 550}%
\special{pa 980 542}%
\special{pa 985 533}%
\special{pa 990 525}%
\special{pa 1000 507}%
\special{fp}%
\special{pn 8}%
\special{pa 1005 499}%
\special{pa 1005 499}%
\special{pa 1015 481}%
\special{pa 1020 473}%
\special{pa 1025 463}%
\special{ip}%
\special{pa 1030 455}%
\special{pa 1030 455}%
\special{pa 1035 447}%
\special{pa 1045 429}%
\special{pa 1050 421}%
\special{pa 1051 419}%
\special{ip}%
\special{pa 1055 412}%
\special{pa 1060 403}%
\special{pa 1065 395}%
\special{pa 1075 377}%
\special{pa 1076 375}%
\special{ip}%
\special{pa 1081 368}%
\special{pa 1090 351}%
\special{pa 1095 343}%
\special{pa 1101 332}%
\special{ip}%
\special{pa 1106 324}%
\special{pa 1110 317}%
\special{pa 1120 299}%
\special{pa 1125 291}%
\special{pa 1127 288}%
\special{ip}%
\special{pa 1131 280}%
\special{pa 1135 273}%
\special{pa 1140 265}%
\special{pa 1150 247}%
\special{pa 1152 244}%
\special{ip}%
\special{pa 1157 236}%
\special{pa 1165 221}%
\special{pa 1170 213}%
\special{pa 1175 204}%
\special{pa 1177 200}%
\special{ip}%
% FUNC 2 0 3 0 Black White  
% 10 400 200 2400 1400 1400 1200 2200 1200 1400 400 600 200 1800 1400 0 4 0 0 0 0
% (x+1)/sqrt(3)
\special{pn 8}%
\special{pn 8}%
\special{pa 400 1315}%
\special{pa 405 1313}%
\special{pa 407 1312}%
\special{ip}%
\special{pa 440 1292}%
\special{pa 445 1289}%
\special{pa 447 1288}%
\special{ip}%
\special{pa 480 1269}%
\special{pa 485 1266}%
\special{pa 487 1265}%
\special{ip}%
\special{pa 520 1246}%
\special{pa 527 1242}%
\special{ip}%
\special{pa 560 1223}%
\special{pa 567 1219}%
\special{ip}%
\special{ip}%
\special{pa 600 1200}%
\special{pa 620 1188}%
\special{pa 625 1186}%
\special{pa 665 1162}%
\special{pa 670 1160}%
\special{pa 710 1136}%
\special{pa 715 1134}%
\special{pa 750 1113}%
\special{pa 755 1111}%
\special{pa 795 1087}%
\special{pa 800 1085}%
\special{pa 840 1061}%
\special{pa 845 1059}%
\special{pa 885 1035}%
\special{pa 890 1033}%
\special{pa 930 1009}%
\special{pa 935 1007}%
\special{pa 975 983}%
\special{pa 980 981}%
\special{pa 1015 960}%
\special{pa 1020 958}%
\special{pa 1060 934}%
\special{pa 1065 932}%
\special{pa 1105 908}%
\special{pa 1110 906}%
\special{pa 1150 882}%
\special{pa 1155 880}%
\special{pa 1195 856}%
\special{pa 1200 854}%
\special{pa 1240 830}%
\special{pa 1245 828}%
\special{pa 1280 807}%
\special{pa 1285 805}%
\special{pa 1325 781}%
\special{pa 1330 779}%
\special{pa 1370 755}%
\special{pa 1375 753}%
\special{pa 1415 729}%
\special{pa 1420 727}%
\special{pa 1460 703}%
\special{pa 1465 701}%
\special{pa 1505 677}%
\special{pa 1510 675}%
\special{pa 1545 654}%
\special{pa 1550 652}%
\special{pa 1590 628}%
\special{pa 1595 626}%
\special{pa 1635 602}%
\special{pa 1640 600}%
\special{pa 1680 576}%
\special{pa 1685 574}%
\special{pa 1725 550}%
\special{pa 1730 548}%
\special{pa 1765 527}%
\special{pa 1770 525}%
\special{pa 1800 507}%
\special{fp}%
\special{pn 8}%
\special{pa 1807 503}%
\special{pa 1810 501}%
\special{pa 1815 499}%
\special{pa 1841 483}%
\special{ip}%
\special{pa 1848 479}%
\special{pa 1855 475}%
\special{pa 1860 473}%
\special{pa 1882 460}%
\special{ip}%
\special{pa 1889 456}%
\special{pa 1900 449}%
\special{pa 1905 447}%
\special{pa 1923 436}%
\special{ip}%
\special{pa 1930 432}%
\special{pa 1945 423}%
\special{pa 1950 421}%
\special{pa 1964 413}%
\special{ip}%
\special{pa 1971 408}%
\special{pa 1990 397}%
\special{pa 1995 395}%
\special{pa 2005 389}%
\special{ip}%
\special{pa 2012 385}%
\special{pa 2030 374}%
\special{pa 2035 372}%
\special{pa 2046 366}%
\special{ip}%
\special{pa 2053 361}%
\special{pa 2075 348}%
\special{pa 2080 346}%
\special{pa 2087 342}%
\special{ip}%
\special{pa 2094 338}%
\special{pa 2120 322}%
\special{pa 2125 320}%
\special{pa 2128 318}%
\special{ip}%
\special{pa 2135 314}%
\special{pa 2165 296}%
\special{pa 2168 295}%
\special{ip}%
\special{pa 2176 291}%
\special{pa 2209 271}%
\special{ip}%
\special{pa 2217 267}%
\special{pa 2250 247}%
\special{ip}%
\special{pa 2257 243}%
\special{pa 2260 242}%
\special{pa 2291 223}%
\special{ip}%
\special{pa 2298 220}%
\special{pa 2300 219}%
\special{pa 2330 201}%
\special{pa 2332 200}%
\special{ip}%
% FUNC 2 0 3 0 Black White  
% 10 400 200 2400 1400 1400 1200 2200 1200 1400 400 400 200 2400 1400 100 4 0 2 0 0
% cos(t)///sin(t)///0///pi
\special{pn 8}%
\special{pa 2200 1200}%
\special{pa 2200 1175}%
\special{pa 2199 1170}%
\special{pa 2199 1155}%
\special{pa 2198 1150}%
\special{pa 2198 1140}%
\special{pa 2197 1135}%
\special{pa 2197 1130}%
\special{pa 2196 1125}%
\special{pa 2196 1120}%
\special{pa 2195 1115}%
\special{pa 2195 1110}%
\special{pa 2194 1105}%
\special{pa 2194 1100}%
\special{pa 2192 1090}%
\special{pa 2192 1085}%
\special{pa 2189 1070}%
\special{pa 2189 1065}%
\special{pa 2181 1025}%
\special{pa 2180 1021}%
\special{pa 2178 1016}%
\special{pa 2174 996}%
\special{pa 2172 991}%
\special{pa 2171 986}%
\special{pa 2170 982}%
\special{pa 2168 977}%
\special{pa 2167 972}%
\special{pa 2165 967}%
\special{pa 2164 962}%
\special{pa 2162 958}%
\special{pa 2161 953}%
\special{pa 2159 948}%
\special{pa 2158 943}%
\special{pa 2156 938}%
\special{pa 2154 934}%
\special{pa 2153 929}%
\special{pa 2151 924}%
\special{pa 2149 920}%
\special{pa 2147 915}%
\special{pa 2146 910}%
\special{pa 2144 905}%
\special{pa 2142 901}%
\special{pa 2138 891}%
\special{pa 2136 887}%
\special{pa 2134 882}%
\special{pa 2132 878}%
\special{pa 2128 868}%
\special{pa 2126 864}%
\special{pa 2124 859}%
\special{pa 2122 855}%
\special{pa 2120 850}%
\special{pa 2117 846}%
\special{pa 2115 841}%
\special{pa 2113 837}%
\special{pa 2110 832}%
\special{pa 2108 828}%
\special{pa 2106 823}%
\special{pa 2103 819}%
\special{pa 2101 815}%
\special{pa 2099 810}%
\special{pa 2096 806}%
\special{pa 2094 801}%
\special{pa 2091 797}%
\special{pa 2089 793}%
\special{pa 2086 788}%
\special{pa 2083 784}%
\special{pa 2081 780}%
\special{pa 2078 776}%
\special{pa 2075 771}%
\special{pa 2073 767}%
\special{pa 2067 759}%
\special{pa 2064 754}%
\special{pa 2062 750}%
\special{pa 2020 694}%
\special{pa 2016 690}%
\special{pa 2007 678}%
\special{pa 2003 675}%
\special{pa 1997 667}%
\special{pa 1990 660}%
\special{pa 1987 656}%
\special{pa 1976 645}%
\special{pa 1973 641}%
\special{pa 1969 638}%
\special{pa 1966 634}%
\special{pa 1962 631}%
\special{pa 1944 613}%
\special{pa 1940 610}%
\special{pa 1933 603}%
\special{pa 1925 597}%
\special{pa 1918 590}%
\special{pa 1910 584}%
\special{pa 1906 580}%
\special{pa 1850 538}%
\special{pa 1845 535}%
\special{pa 1841 533}%
\special{pa 1833 527}%
\special{pa 1829 525}%
\special{pa 1824 522}%
\special{pa 1820 519}%
\special{pa 1816 517}%
\special{pa 1812 514}%
\special{pa 1807 511}%
\special{pa 1803 509}%
\special{pa 1799 506}%
\special{pa 1794 504}%
\special{pa 1790 501}%
\special{pa 1785 499}%
\special{pa 1781 497}%
\special{pa 1777 494}%
\special{pa 1772 492}%
\special{pa 1768 489}%
\special{pa 1763 487}%
\special{pa 1759 485}%
\special{pa 1754 483}%
\special{pa 1750 480}%
\special{pa 1745 478}%
\special{pa 1741 476}%
\special{pa 1731 472}%
\special{pa 1727 470}%
\special{pa 1722 468}%
\special{pa 1718 466}%
\special{pa 1708 462}%
\special{pa 1704 460}%
\special{pa 1694 456}%
\special{pa 1690 454}%
\special{pa 1685 453}%
\special{pa 1680 451}%
\special{pa 1676 449}%
\special{pa 1671 447}%
\special{pa 1666 446}%
\special{pa 1661 444}%
\special{pa 1657 442}%
\special{pa 1652 441}%
\special{pa 1647 439}%
\special{pa 1642 438}%
\special{pa 1638 436}%
\special{pa 1633 435}%
\special{pa 1628 433}%
\special{pa 1623 432}%
\special{pa 1618 430}%
\special{pa 1613 429}%
\special{pa 1609 428}%
\special{pa 1604 426}%
\special{pa 1584 422}%
\special{pa 1579 420}%
\special{pa 1574 419}%
\special{pa 1570 418}%
\special{pa 1535 411}%
\special{pa 1530 411}%
\special{pa 1515 408}%
\special{pa 1510 408}%
\special{pa 1500 406}%
\special{pa 1495 406}%
\special{pa 1490 405}%
\special{pa 1485 405}%
\special{pa 1480 404}%
\special{pa 1475 404}%
\special{pa 1470 403}%
\special{pa 1465 403}%
\special{pa 1460 402}%
\special{pa 1450 402}%
\special{pa 1445 401}%
\special{pa 1430 401}%
\special{pa 1425 400}%
\special{pa 1375 400}%
\special{pa 1370 401}%
\special{pa 1355 401}%
\special{pa 1350 402}%
\special{pa 1340 402}%
\special{pa 1335 403}%
\special{pa 1330 403}%
\special{pa 1325 404}%
\special{pa 1320 404}%
\special{pa 1315 405}%
\special{pa 1310 405}%
\special{pa 1305 406}%
\special{pa 1300 406}%
\special{pa 1290 408}%
\special{pa 1285 408}%
\special{pa 1270 411}%
\special{pa 1265 411}%
\special{pa 1225 419}%
\special{pa 1221 420}%
\special{pa 1216 422}%
\special{pa 1196 426}%
\special{pa 1191 428}%
\special{pa 1186 429}%
\special{pa 1182 430}%
\special{pa 1177 432}%
\special{pa 1172 433}%
\special{pa 1167 435}%
\special{pa 1162 436}%
\special{pa 1158 438}%
\special{pa 1153 439}%
\special{pa 1148 441}%
\special{pa 1143 442}%
\special{pa 1138 444}%
\special{pa 1134 446}%
\special{pa 1129 447}%
\special{pa 1124 449}%
\special{pa 1120 451}%
\special{pa 1115 453}%
\special{pa 1110 454}%
\special{pa 1105 456}%
\special{pa 1101 458}%
\special{pa 1091 462}%
\special{pa 1087 464}%
\special{pa 1082 466}%
\special{pa 1078 468}%
\special{pa 1068 472}%
\special{pa 1064 474}%
\special{pa 1059 476}%
\special{pa 1055 478}%
\special{pa 1050 480}%
\special{pa 1046 483}%
\special{pa 1041 485}%
\special{pa 1037 487}%
\special{pa 1032 490}%
\special{pa 1028 492}%
\special{pa 1023 494}%
\special{pa 1019 497}%
\special{pa 1015 499}%
\special{pa 1010 501}%
\special{pa 1006 504}%
\special{pa 1001 506}%
\special{pa 997 509}%
\special{pa 993 511}%
\special{pa 988 514}%
\special{pa 984 517}%
\special{pa 980 519}%
\special{pa 976 522}%
\special{pa 971 525}%
\special{pa 967 527}%
\special{pa 959 533}%
\special{pa 954 536}%
\special{pa 950 538}%
\special{pa 894 580}%
\special{pa 890 584}%
\special{pa 878 593}%
\special{pa 875 597}%
\special{pa 867 603}%
\special{pa 860 610}%
\special{pa 856 613}%
\special{pa 845 624}%
\special{pa 841 627}%
\special{pa 838 631}%
\special{pa 834 634}%
\special{pa 831 638}%
\special{pa 813 656}%
\special{pa 810 660}%
\special{pa 803 667}%
\special{pa 797 675}%
\special{pa 790 682}%
\special{pa 784 690}%
\special{pa 780 694}%
\special{pa 738 750}%
\special{pa 735 755}%
\special{pa 733 759}%
\special{pa 727 767}%
\special{pa 725 771}%
\special{pa 722 776}%
\special{pa 719 780}%
\special{pa 717 784}%
\special{pa 714 788}%
\special{pa 711 793}%
\special{pa 709 797}%
\special{pa 706 801}%
\special{pa 704 806}%
\special{pa 701 810}%
\special{pa 699 815}%
\special{pa 697 819}%
\special{pa 694 823}%
\special{pa 692 828}%
\special{pa 689 832}%
\special{pa 687 837}%
\special{pa 685 841}%
\special{pa 683 846}%
\special{pa 680 850}%
\special{pa 678 855}%
\special{pa 676 859}%
\special{pa 672 869}%
\special{pa 670 873}%
\special{pa 668 878}%
\special{pa 666 882}%
\special{pa 662 892}%
\special{pa 660 896}%
\special{pa 656 906}%
\special{pa 654 910}%
\special{pa 653 915}%
\special{pa 651 920}%
\special{pa 649 924}%
\special{pa 647 929}%
\special{pa 646 934}%
\special{pa 644 939}%
\special{pa 642 943}%
\special{pa 641 948}%
\special{pa 639 953}%
\special{pa 638 958}%
\special{pa 636 962}%
\special{pa 635 967}%
\special{pa 633 972}%
\special{pa 632 977}%
\special{pa 630 982}%
\special{pa 629 987}%
\special{pa 628 991}%
\special{pa 626 996}%
\special{pa 622 1016}%
\special{pa 620 1021}%
\special{pa 619 1026}%
\special{pa 618 1030}%
\special{pa 611 1065}%
\special{pa 611 1070}%
\special{pa 608 1085}%
\special{pa 608 1090}%
\special{pa 606 1100}%
\special{pa 606 1105}%
\special{pa 605 1110}%
\special{pa 605 1115}%
\special{pa 604 1120}%
\special{pa 604 1125}%
\special{pa 603 1130}%
\special{pa 603 1135}%
\special{pa 602 1140}%
\special{pa 602 1150}%
\special{pa 601 1155}%
\special{pa 601 1170}%
\special{pa 600 1175}%
\special{pa 600 1200}%
\special{fp}%
% FUNC 2 0 3 0 Black White  
% 10 400 200 2400 1400 1400 1200 2200 1200 1400 400 600 200 2200 1400 0 4 0 0 0 0
% 0
\special{pn 8}%
\special{pn 8}%
\special{pa 400 1200}%
\special{pa 409 1200}%
\special{ip}%
\special{pa 450 1200}%
\special{pa 459 1200}%
\special{ip}%
\special{pa 500 1200}%
\special{pa 509 1200}%
\special{ip}%
\special{pa 550 1200}%
\special{pa 559 1200}%
\special{ip}%
\special{ip}%
\special{pa 600 1200}%
\special{pa 2200 1200}%
\special{fp}%
\special{pn 8}%
\special{pa 2209 1200}%
\special{pa 2250 1200}%
\special{ip}%
\special{pa 2259 1200}%
\special{pa 2300 1200}%
\special{ip}%
\special{pa 2309 1200}%
\special{pa 2350 1200}%
\special{ip}%
\special{pa 2359 1200}%
\special{pa 2400 1200}%
\special{ip}%
% STR 2 0 3 0 Black White  
% 4 600 1100 600 1200 2 0 1 0
% $A$
\put(6.0000,-12.0000){\makebox(0,0)[lb]{{\colorbox[named]{White}{$A$}}}}%
% STR 2 0 3 0 Black White  
% 4 2200 1100 2200 1200 2 0 1 0
% $B$
\put(22.0000,-12.0000){\makebox(0,0)[lb]{{\colorbox[named]{White}{$B$}}}}%
% STR 2 0 3 0 Black White  
% 4 1800 410 1800 510 2 0 1 0
% $P$
\put(18.0000,-5.1000){\makebox(0,0)[lb]{{\colorbox[named]{White}{$P$}}}}%
% STR 2 0 3 0 Black White  
% 4 1000 410 1000 510 2 0 1 0
% $Q$
\put(10.0000,-5.1000){\makebox(0,0)[lb]{{\colorbox[named]{White}{$Q$}}}}%
\end{picture}}%
}
          \end{center}
     \end{oframed}

\setlength{\columnseprule}{0.4pt}
\begin{multicols}{2}
{\bf[解]} $A(-1,0)$,$B(1,0)$となるように$xy$座標をおく.すると原点$O$がこの半円の中心である.故に$\angle AOQ=\pi/3$,
$\angle AOP=2\pi/3$である.$P$,$Q$から$x$軸に下ろした垂足を,$L$,$H$とする.
     \begin{center}
     \scalebox{1}{%WinTpicVersion4.32a
{\unitlength 0.1in%
\begin{picture}(20.0000,12.0000)(4.0000,-16.0000)%
% STR 2 0 3 0 Black White  
% 4 1390 1397 1390 1410 4 400 0 0
% O
\put(13.9000,-14.1000){\makebox(0,0)[rt]{O}}%
% STR 2 0 3 0 Black White  
% 4 1360 387 1360 400 4 400 0 0
% $y$
\put(13.6000,-4.0000){\makebox(0,0)[rt]{$y$}}%
% STR 2 0 3 0 Black White  
% 4 2400 1427 2400 1440 4 400 0 0
% $x$
\put(24.0000,-14.4000){\makebox(0,0)[rt]{$x$}}%
% VECTOR 2 0 3 0 Black White  
% 2 1400 1600 1400 400
% 
\special{pn 8}%
\special{pa 1400 1600}%
\special{pa 1400 400}%
\special{fp}%
\special{sh 1}%
\special{pa 1400 400}%
\special{pa 1380 467}%
\special{pa 1400 453}%
\special{pa 1420 467}%
\special{pa 1400 400}%
\special{fp}%
% VECTOR 2 0 3 0 Black White  
% 2 400 1400 2400 1400
% 
\special{pn 8}%
\special{pa 400 1400}%
\special{pa 2400 1400}%
\special{fp}%
\special{sh 1}%
\special{pa 2400 1400}%
\special{pa 2333 1380}%
\special{pa 2347 1400}%
\special{pa 2333 1420}%
\special{pa 2400 1400}%
\special{fp}%
% FUNC 2 0 3 0 Black White  
% 9 400 400 2400 1600 1400 1400 2200 1400 1400 600 400 400 2400 1600 50 4 0 2
% cos(t)///sin(t)///0///pi
\special{pn 8}%
\special{pa 2200 1400}%
\special{pa 2200 1375}%
\special{pa 2199 1370}%
\special{pa 2199 1355}%
\special{pa 2198 1350}%
\special{pa 2198 1340}%
\special{pa 2197 1335}%
\special{pa 2197 1330}%
\special{pa 2196 1325}%
\special{pa 2196 1320}%
\special{pa 2195 1315}%
\special{pa 2195 1310}%
\special{pa 2194 1305}%
\special{pa 2194 1300}%
\special{pa 2192 1290}%
\special{pa 2192 1285}%
\special{pa 2189 1270}%
\special{pa 2189 1265}%
\special{pa 2181 1225}%
\special{pa 2180 1221}%
\special{pa 2178 1216}%
\special{pa 2174 1196}%
\special{pa 2172 1191}%
\special{pa 2171 1186}%
\special{pa 2170 1182}%
\special{pa 2168 1177}%
\special{pa 2167 1172}%
\special{pa 2165 1167}%
\special{pa 2164 1162}%
\special{pa 2162 1158}%
\special{pa 2161 1153}%
\special{pa 2159 1148}%
\special{pa 2158 1143}%
\special{pa 2156 1138}%
\special{pa 2154 1134}%
\special{pa 2153 1129}%
\special{pa 2151 1124}%
\special{pa 2149 1120}%
\special{pa 2147 1115}%
\special{pa 2146 1110}%
\special{pa 2144 1105}%
\special{pa 2142 1101}%
\special{pa 2138 1091}%
\special{pa 2136 1087}%
\special{pa 2134 1082}%
\special{pa 2132 1078}%
\special{pa 2128 1068}%
\special{pa 2126 1064}%
\special{pa 2124 1059}%
\special{pa 2122 1055}%
\special{pa 2120 1050}%
\special{pa 2117 1046}%
\special{pa 2115 1041}%
\special{pa 2113 1037}%
\special{pa 2110 1032}%
\special{pa 2108 1028}%
\special{pa 2106 1023}%
\special{pa 2103 1019}%
\special{pa 2101 1015}%
\special{pa 2099 1010}%
\special{pa 2096 1006}%
\special{pa 2094 1001}%
\special{pa 2091 997}%
\special{pa 2089 993}%
\special{pa 2086 988}%
\special{pa 2083 984}%
\special{pa 2081 980}%
\special{pa 2078 976}%
\special{pa 2075 971}%
\special{pa 2073 967}%
\special{pa 2067 959}%
\special{pa 2064 954}%
\special{pa 2062 950}%
\special{pa 2020 894}%
\special{pa 2016 890}%
\special{pa 2007 878}%
\special{pa 2003 875}%
\special{pa 1997 867}%
\special{pa 1990 860}%
\special{pa 1987 856}%
\special{pa 1976 845}%
\special{pa 1973 841}%
\special{pa 1969 838}%
\special{pa 1966 834}%
\special{pa 1962 831}%
\special{pa 1944 813}%
\special{pa 1940 810}%
\special{pa 1933 803}%
\special{pa 1925 797}%
\special{pa 1918 790}%
\special{pa 1910 784}%
\special{pa 1906 780}%
\special{pa 1850 738}%
\special{pa 1845 735}%
\special{pa 1841 733}%
\special{pa 1833 727}%
\special{pa 1829 725}%
\special{pa 1824 722}%
\special{pa 1820 719}%
\special{pa 1816 717}%
\special{pa 1812 714}%
\special{pa 1807 711}%
\special{pa 1803 709}%
\special{pa 1799 706}%
\special{pa 1794 704}%
\special{pa 1790 701}%
\special{pa 1785 699}%
\special{pa 1781 697}%
\special{pa 1777 694}%
\special{pa 1772 692}%
\special{pa 1768 689}%
\special{pa 1763 687}%
\special{pa 1759 685}%
\special{pa 1754 683}%
\special{pa 1750 680}%
\special{pa 1745 678}%
\special{pa 1741 676}%
\special{pa 1731 672}%
\special{pa 1727 670}%
\special{pa 1722 668}%
\special{pa 1718 666}%
\special{pa 1708 662}%
\special{pa 1704 660}%
\special{pa 1694 656}%
\special{pa 1690 654}%
\special{pa 1685 653}%
\special{pa 1680 651}%
\special{pa 1676 649}%
\special{pa 1671 647}%
\special{pa 1666 646}%
\special{pa 1661 644}%
\special{pa 1657 642}%
\special{pa 1652 641}%
\special{pa 1647 639}%
\special{pa 1642 638}%
\special{pa 1638 636}%
\special{pa 1633 635}%
\special{pa 1628 633}%
\special{pa 1623 632}%
\special{pa 1618 630}%
\special{pa 1613 629}%
\special{pa 1609 628}%
\special{pa 1604 626}%
\special{pa 1584 622}%
\special{pa 1579 620}%
\special{pa 1574 619}%
\special{pa 1570 618}%
\special{pa 1535 611}%
\special{pa 1530 611}%
\special{pa 1515 608}%
\special{pa 1510 608}%
\special{pa 1500 606}%
\special{pa 1495 606}%
\special{pa 1490 605}%
\special{pa 1485 605}%
\special{pa 1480 604}%
\special{pa 1475 604}%
\special{pa 1470 603}%
\special{pa 1465 603}%
\special{pa 1460 602}%
\special{pa 1450 602}%
\special{pa 1445 601}%
\special{pa 1430 601}%
\special{pa 1425 600}%
\special{pa 1375 600}%
\special{pa 1370 601}%
\special{pa 1355 601}%
\special{pa 1350 602}%
\special{pa 1340 602}%
\special{pa 1335 603}%
\special{pa 1330 603}%
\special{pa 1325 604}%
\special{pa 1320 604}%
\special{pa 1315 605}%
\special{pa 1310 605}%
\special{pa 1305 606}%
\special{pa 1300 606}%
\special{pa 1290 608}%
\special{pa 1285 608}%
\special{pa 1270 611}%
\special{pa 1265 611}%
\special{pa 1225 619}%
\special{pa 1221 620}%
\special{pa 1216 622}%
\special{pa 1196 626}%
\special{pa 1191 628}%
\special{pa 1186 629}%
\special{pa 1182 630}%
\special{pa 1177 632}%
\special{pa 1172 633}%
\special{pa 1167 635}%
\special{pa 1162 636}%
\special{pa 1158 638}%
\special{pa 1153 639}%
\special{pa 1148 641}%
\special{pa 1143 642}%
\special{pa 1138 644}%
\special{pa 1134 646}%
\special{pa 1129 647}%
\special{pa 1124 649}%
\special{pa 1120 651}%
\special{pa 1115 653}%
\special{pa 1110 654}%
\special{pa 1105 656}%
\special{pa 1101 658}%
\special{pa 1091 662}%
\special{pa 1087 664}%
\special{pa 1082 666}%
\special{pa 1078 668}%
\special{pa 1068 672}%
\special{pa 1064 674}%
\special{pa 1059 676}%
\special{pa 1055 678}%
\special{pa 1050 680}%
\special{pa 1046 683}%
\special{pa 1041 685}%
\special{pa 1037 687}%
\special{pa 1032 690}%
\special{pa 1028 692}%
\special{pa 1023 694}%
\special{pa 1019 697}%
\special{pa 1015 699}%
\special{pa 1010 701}%
\special{pa 1006 704}%
\special{pa 1001 706}%
\special{pa 997 709}%
\special{pa 993 711}%
\special{pa 988 714}%
\special{pa 984 717}%
\special{pa 980 719}%
\special{pa 976 722}%
\special{pa 971 725}%
\special{pa 967 727}%
\special{pa 959 733}%
\special{pa 954 736}%
\special{pa 950 738}%
\special{pa 894 780}%
\special{pa 890 784}%
\special{pa 878 793}%
\special{pa 875 797}%
\special{pa 867 803}%
\special{pa 860 810}%
\special{pa 856 813}%
\special{pa 845 824}%
\special{pa 841 827}%
\special{pa 838 831}%
\special{pa 834 834}%
\special{pa 831 838}%
\special{pa 813 856}%
\special{pa 810 860}%
\special{pa 803 867}%
\special{pa 797 875}%
\special{pa 790 882}%
\special{pa 784 890}%
\special{pa 780 894}%
\special{pa 738 950}%
\special{pa 735 955}%
\special{pa 733 959}%
\special{pa 727 967}%
\special{pa 725 971}%
\special{pa 722 976}%
\special{pa 719 980}%
\special{pa 717 984}%
\special{pa 714 988}%
\special{pa 711 993}%
\special{pa 709 997}%
\special{pa 706 1001}%
\special{pa 704 1006}%
\special{pa 701 1010}%
\special{pa 699 1015}%
\special{pa 697 1019}%
\special{pa 694 1023}%
\special{pa 692 1028}%
\special{pa 689 1032}%
\special{pa 687 1037}%
\special{pa 685 1041}%
\special{pa 683 1046}%
\special{pa 680 1050}%
\special{pa 678 1055}%
\special{pa 676 1059}%
\special{pa 672 1069}%
\special{pa 670 1073}%
\special{pa 668 1078}%
\special{pa 666 1082}%
\special{pa 662 1092}%
\special{pa 660 1096}%
\special{pa 656 1106}%
\special{pa 654 1110}%
\special{pa 653 1115}%
\special{pa 651 1120}%
\special{pa 649 1124}%
\special{pa 647 1129}%
\special{pa 646 1134}%
\special{pa 644 1139}%
\special{pa 642 1143}%
\special{pa 641 1148}%
\special{pa 639 1153}%
\special{pa 638 1158}%
\special{pa 636 1162}%
\special{pa 635 1167}%
\special{pa 633 1172}%
\special{pa 632 1177}%
\special{pa 630 1182}%
\special{pa 629 1187}%
\special{pa 628 1191}%
\special{pa 626 1196}%
\special{pa 622 1216}%
\special{pa 620 1221}%
\special{pa 619 1226}%
\special{pa 618 1230}%
\special{pa 611 1265}%
\special{pa 611 1270}%
\special{pa 608 1285}%
\special{pa 608 1290}%
\special{pa 606 1300}%
\special{pa 606 1305}%
\special{pa 605 1310}%
\special{pa 605 1315}%
\special{pa 604 1320}%
\special{pa 604 1325}%
\special{pa 603 1330}%
\special{pa 603 1335}%
\special{pa 602 1340}%
\special{pa 602 1350}%
\special{pa 601 1355}%
\special{pa 601 1370}%
\special{pa 600 1375}%
\special{pa 600 1400}%
\special{fp}%
% FUNC 2 0 3 0 Black White  
% 10 400 400 2400 1600 1400 1400 2200 1400 1400 600 600 400 1000 1600 0 4 0 0 0 0
% sqrt(3)(x+1)
\special{pn 8}%
\special{pn 8}%
\special{pa 485 1600}%
\special{pa 485 1599}%
\special{pa 489 1593}%
\special{ip}%
\special{pa 508 1560}%
\special{pa 512 1553}%
\special{ip}%
\special{pa 531 1520}%
\special{pa 535 1513}%
\special{pa 535 1513}%
\special{ip}%
\special{pa 554 1480}%
\special{pa 558 1473}%
\special{ip}%
\special{pa 577 1440}%
\special{pa 580 1435}%
\special{pa 581 1433}%
\special{ip}%
\special{pa 600 1400}%
\special{pa 605 1391}%
\special{pa 610 1383}%
\special{pa 620 1365}%
\special{pa 625 1357}%
\special{pa 635 1339}%
\special{pa 640 1331}%
\special{pa 650 1313}%
\special{pa 655 1305}%
\special{pa 665 1287}%
\special{pa 670 1279}%
\special{pa 680 1261}%
\special{pa 685 1253}%
\special{pa 695 1235}%
\special{pa 700 1227}%
\special{pa 710 1209}%
\special{pa 715 1201}%
\special{pa 725 1183}%
\special{pa 730 1175}%
\special{pa 735 1166}%
\special{pa 740 1158}%
\special{pa 750 1140}%
\special{pa 755 1132}%
\special{pa 765 1114}%
\special{pa 770 1106}%
\special{pa 780 1088}%
\special{pa 785 1080}%
\special{pa 795 1062}%
\special{pa 800 1054}%
\special{pa 810 1036}%
\special{pa 815 1028}%
\special{pa 825 1010}%
\special{pa 830 1002}%
\special{pa 840 984}%
\special{pa 845 976}%
\special{pa 855 958}%
\special{pa 860 950}%
\special{pa 870 932}%
\special{pa 875 924}%
\special{pa 885 906}%
\special{pa 890 898}%
\special{pa 900 880}%
\special{pa 905 872}%
\special{pa 915 854}%
\special{pa 920 846}%
\special{pa 930 828}%
\special{pa 935 820}%
\special{pa 945 802}%
\special{pa 950 794}%
\special{pa 960 776}%
\special{pa 965 768}%
\special{pa 975 750}%
\special{pa 980 742}%
\special{pa 985 733}%
\special{pa 990 725}%
\special{pa 1000 707}%
\special{fp}%
\special{pn 8}%
\special{pa 1005 699}%
\special{pa 1005 699}%
\special{pa 1015 681}%
\special{pa 1020 673}%
\special{pa 1025 663}%
\special{ip}%
\special{pa 1030 655}%
\special{pa 1030 655}%
\special{pa 1035 647}%
\special{pa 1045 629}%
\special{pa 1050 621}%
\special{pa 1051 619}%
\special{ip}%
\special{pa 1055 612}%
\special{pa 1060 603}%
\special{pa 1065 595}%
\special{pa 1075 577}%
\special{pa 1076 575}%
\special{ip}%
\special{pa 1081 568}%
\special{pa 1090 551}%
\special{pa 1095 543}%
\special{pa 1101 532}%
\special{ip}%
\special{pa 1106 524}%
\special{pa 1110 517}%
\special{pa 1120 499}%
\special{pa 1125 491}%
\special{pa 1127 488}%
\special{ip}%
\special{pa 1131 480}%
\special{pa 1135 473}%
\special{pa 1140 465}%
\special{pa 1150 447}%
\special{pa 1152 444}%
\special{ip}%
\special{pa 1157 436}%
\special{pa 1165 421}%
\special{pa 1170 413}%
\special{pa 1175 404}%
\special{pa 1177 400}%
\special{ip}%
% FUNC 2 0 3 0 Black White  
% 10 400 400 2400 1600 1400 1400 2200 1400 1400 600 600 400 1800 1600 0 4 0 0 0 0
% (x+1)/sqrt(3)
\special{pn 8}%
\special{pn 8}%
\special{pa 400 1515}%
\special{pa 405 1513}%
\special{pa 407 1512}%
\special{ip}%
\special{pa 440 1492}%
\special{pa 445 1489}%
\special{pa 447 1488}%
\special{ip}%
\special{pa 480 1469}%
\special{pa 485 1466}%
\special{pa 487 1465}%
\special{ip}%
\special{pa 520 1446}%
\special{pa 527 1442}%
\special{ip}%
\special{pa 560 1423}%
\special{pa 567 1419}%
\special{ip}%
\special{ip}%
\special{pa 600 1400}%
\special{pa 620 1388}%
\special{pa 625 1386}%
\special{pa 665 1362}%
\special{pa 670 1360}%
\special{pa 710 1336}%
\special{pa 715 1334}%
\special{pa 750 1313}%
\special{pa 755 1311}%
\special{pa 795 1287}%
\special{pa 800 1285}%
\special{pa 840 1261}%
\special{pa 845 1259}%
\special{pa 885 1235}%
\special{pa 890 1233}%
\special{pa 930 1209}%
\special{pa 935 1207}%
\special{pa 975 1183}%
\special{pa 980 1181}%
\special{pa 1015 1160}%
\special{pa 1020 1158}%
\special{pa 1060 1134}%
\special{pa 1065 1132}%
\special{pa 1105 1108}%
\special{pa 1110 1106}%
\special{pa 1150 1082}%
\special{pa 1155 1080}%
\special{pa 1195 1056}%
\special{pa 1200 1054}%
\special{pa 1240 1030}%
\special{pa 1245 1028}%
\special{pa 1280 1007}%
\special{pa 1285 1005}%
\special{pa 1325 981}%
\special{pa 1330 979}%
\special{pa 1370 955}%
\special{pa 1375 953}%
\special{pa 1415 929}%
\special{pa 1420 927}%
\special{pa 1460 903}%
\special{pa 1465 901}%
\special{pa 1505 877}%
\special{pa 1510 875}%
\special{pa 1545 854}%
\special{pa 1550 852}%
\special{pa 1590 828}%
\special{pa 1595 826}%
\special{pa 1635 802}%
\special{pa 1640 800}%
\special{pa 1680 776}%
\special{pa 1685 774}%
\special{pa 1725 750}%
\special{pa 1730 748}%
\special{pa 1765 727}%
\special{pa 1770 725}%
\special{pa 1800 707}%
\special{fp}%
\special{pn 8}%
\special{pa 1807 703}%
\special{pa 1810 701}%
\special{pa 1815 699}%
\special{pa 1841 683}%
\special{ip}%
\special{pa 1848 679}%
\special{pa 1855 675}%
\special{pa 1860 673}%
\special{pa 1882 660}%
\special{ip}%
\special{pa 1889 656}%
\special{pa 1900 649}%
\special{pa 1905 647}%
\special{pa 1923 636}%
\special{ip}%
\special{pa 1930 632}%
\special{pa 1945 623}%
\special{pa 1950 621}%
\special{pa 1964 613}%
\special{ip}%
\special{pa 1971 608}%
\special{pa 1990 597}%
\special{pa 1995 595}%
\special{pa 2005 589}%
\special{ip}%
\special{pa 2012 585}%
\special{pa 2030 574}%
\special{pa 2035 572}%
\special{pa 2046 566}%
\special{ip}%
\special{pa 2053 561}%
\special{pa 2075 548}%
\special{pa 2080 546}%
\special{pa 2087 542}%
\special{ip}%
\special{pa 2094 538}%
\special{pa 2120 522}%
\special{pa 2125 520}%
\special{pa 2128 518}%
\special{ip}%
\special{pa 2135 514}%
\special{pa 2165 496}%
\special{pa 2168 495}%
\special{ip}%
\special{pa 2176 491}%
\special{pa 2209 471}%
\special{ip}%
\special{pa 2217 467}%
\special{pa 2250 447}%
\special{ip}%
\special{pa 2257 443}%
\special{pa 2260 442}%
\special{pa 2291 423}%
\special{ip}%
\special{pa 2298 420}%
\special{pa 2300 419}%
\special{pa 2330 401}%
\special{pa 2332 400}%
\special{ip}%
% LINE 2 2 3 0 Black White  
% 4 1800 710 1800 1400 1000 1400 1000 710
% 
\special{pn 8}%
\special{pa 1800 710}%
\special{pa 1800 1400}%
\special{dt 0.045}%
\special{pa 1000 1400}%
\special{pa 1000 710}%
\special{dt 0.045}%
% STR 2 0 3 0 Black White  
% 4 600 1300 600 1400 2 0 1 0
% $-1$
\put(6.0000,-14.0000){\makebox(0,0)[lb]{{\colorbox[named]{White}{$-1$}}}}%
% STR 2 0 3 0 Black White  
% 4 1000 1300 1000 1400 2 0 1 0
% $-1/2$
\put(10.0000,-14.0000){\makebox(0,0)[lb]{{\colorbox[named]{White}{$-1/2$}}}}%
% STR 2 0 3 0 Black White  
% 4 1800 1300 1800 1400 2 0 1 0
% $1/2$
\put(18.0000,-14.0000){\makebox(0,0)[lb]{{\colorbox[named]{White}{$1/2$}}}}%
% STR 2 0 3 0 Black White  
% 4 2200 1300 2200 1400 2 0 1 0
% $1$
\put(22.0000,-14.0000){\makebox(0,0)[lb]{{\colorbox[named]{White}{$1$}}}}%
\end{picture}}%
}
     \end{center}
このとき,求める体積$V$として
     \begin{align*}
     V_1&=(\triangle AQH\text{の回転体}) \\
     V_2&=(PQHL\text{の回転体}) \\
     V_3&=(\triangle APL\text{の回転体})
     \end{align*}
とおけば,
     \begin{align}
     V=V_1+V_2-V_3\label{1}
     \end{align}
である.各項計算して,
     \begin{align*}
     V_1&=\frac{1}{3}\frac{1}{2}\frac{3}{4}\pi=\frac{1}{8}\pi \\
     V_2&=\int_{-1/2}^{1/2}\pi(1-x^2)dx=\frac{11}{12}\pi \\
     V_3&=\frac{1}{3}\frac{2}{3}\frac{3}{4}\pi=\frac{3}{8}\pi
     \end{align*}
これらを\eqref{1}に代入して
     \[V=\left(\frac{1}{8}+\frac{11}{12}-\frac{3}{8}\right)\pi=\frac{2}{3}\pi\]
である.$\cdots$(答)
     
\newpage
\end{multicols}
\end{document}