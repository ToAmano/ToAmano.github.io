\documentclass[a4j]{jarticle}
\usepackage{amsmath}
\usepackage{ascmac}
\usepackage{amssymb}
\usepackage{enumerate}
\usepackage{multicol}
\usepackage{framed}
\usepackage{fancyhdr}
\usepackage{latexsym}
\usepackage{indent}
\usepackage{cases}
\usepackage[dvips]{graphicx}
\usepackage{color}
\usepackage{emath}
\usepackage{emathPp}
\allowdisplaybreaks
\pagestyle{fancy}
\lhead{}
\chead{}
\rhead{東京大学前期$1983$年$6$番}

\begin{document}
%分数関係


\def\tfrac#1#2{{\textstyle\frac{#1}{#2}}} %数式中で文中表示の分数を使う時


%Σ関係

\def\dsum#1#2{{\displaystyle\sum_{#1}^{#2}}} %文中で数式表示のΣを使う時


%ベクトル関係


\def\vector#1{\overrightarrow{#1}}  %ベクトルを表現したいとき(aベクトルを表現するときは\ver
\def\norm#1{|\overrightarrow{#1}|} %ベクトルの絶対値
\def\vtwo#1#2{ \left(%
      \begin{array}{c}%
      #1 \\%
      #2 \\%
      \end{array}%
      \right) }                        %2次元ベクトル成分表示
      
      \def\vthree#1#2#3{ \left(
      \begin{array}{c}
      #1 \\
      #2 \\
      #3 \\
      \end{array}
      \right) }                        %3次元ベクトル成分表示



%数列関係


\def\an#1{\verb|{|$#1$\verb|}|}


%極限関係

\def\limit#1#2{\stackrel{#1 \to #2}{\longrightarrow}}   %等式変形からの極限
\def\dlim#1#2{{\displaystyle \lim_{#1\to#2}}} %文中で数式表示の極限を使う



%積分関係

\def\dint#1#2{{\displaystyle \int_{#1}^{#2}}} %文中で数式表示の積分を使う時

\def\ne{\nearrow}
\def\se{\searrow}
\def\nw{\nwarrow}
\def\ne{\nearrow}


%便利なやつ

\def\case#1#2{%
 \[\left\{%
 \begin{array}{l}%
 #1 \\%
 #2%
 \end{array}%
 \right.\] }                           %場合分け
 
\def\1{$\cos\theta=c$,$\sin\theta=s$とおく.}  %cs表示を与える前書きシータ
\def\2{$\cos t=c$,$\sin t=s$とおく.}     %cs表示を与える前書きt
\def\3{$\cos x=c$,$\sin x=s$とおく.}                %cs表示を与える前書きx

\def\fig#1#2#3 {%
\begin{wrapfigure}[#1]{r}{#2 zw}%
\vspace*{-1zh}%
\input{#3}%
\end{wrapfigure} }           %絵の挿入


\def\a{\alpha}   %ギリシャ文字
\def\b{\beta}
\def\g{\gamma}

%問題番号のためのマクロ

\newcounter{nombre} %必須
\renewcommand{\thenombre}{\arabic{nombre}} %任意
\setcounter{nombre}{2} %任意
\newcounter{nombresub}[nombre] %親子関係を定義
\renewcommand{\thenombresub}{\arabic{nombresub}} %任意
\setcounter{nombresub}{0} %任意
\newcommand{\prob}[1][]{\refstepcounter{nombre}#1[問題 \thenombre]}
\newcommand{\probsub}[1][]{\refstepcounter{nombresub}#1(\thenombresub)}


%1-1みたいなカウンタ(todaiとtodaia)
\newcounter{todai}
\setcounter{todai}{0}
\newcounter{todaisub}[todai] 
\setcounter{todaisub}{0} 
\newcommand{\todai}[1][]{\refstepcounter{todai}#1 \thetodai-\thetodaisub}
\newcommand{\todaib}[1][]{\refstepcounter{todai}#1\refstepcounter{todaisub}#1 {\bf [問題 \thetodai.\thetodaisub]}}
\newcommand{\todaia}[1][]{\refstepcounter{todaisub}#1 {\bf [問題 \thetodai.\thetodaisub]}}


\begin{oframed}
放物線$y=3/4-x^2$を$y$軸のまわりに回転して得られる曲面$K$を,原点を通り回転軸と
$45^\circ$の角をなす平面$H$で切る.曲面$K$と平面$H$で囲まれた立体の体積を求めよ.
\end{oframed}

\setlength{\columnseprule}{0.4pt}
\begin{multicols}{2}
{\bf[解]}題意の放物線が偶関数であることから,曲面$K$の方程式は,$x^2$を$x^2+z^2$で置き換えて
     \[y=\frac{3}{4}-(x^2+z^2) \]
である.また,対称性から$H$の方程式は
     \[y=x \]
としてよい.

これらを平面$x=k$で切断すると
     \begin{align*}
          \begin{array}{l}
          K:y=\left\{ \dfrac{3}{4}-k^2\right\}-z^2\equiv f(z)\\
          H:y=k
          \end{array}
     \end{align*}
となる.

     \begin{zahyou}[ul=10mm](-1.5,1.5)(-0.5,1.5)
     \def\Fx{1-X**2}
     \def\Gx{0.5}
     \YGurafu*\Fx
     \YGurafu*\Gx
     \YKouten\Fx\Gx{}{0}\tempi\A
     \YKouten\Fx\Gx{0}{}\tempii\B
     \YNurii*\Fx\Gx{\tempi}{\tempii}
     \Put\A[syaei=x,xlabel=-\a]{}
     \Put\B[syaei=x,xlabel=\a]{}
     \Put{(0,0.5)}[se]{k}
     \end{zahyou}

故に曲面$K$と平面$H$で囲まれた立体$L$のこの平面での断面は
     \[ k\le y\le \left\{ \dfrac{3}{4}-k^2\right\}-z^2 \]
となる.このような$y$の存在条件は
     \begin{align}
     \alpha=\sqrt{\dfrac{3}{4}-k-k^2} \label{a}
     \end{align}
とおけば
     \begin{align}
     k&\le \left\{ \dfrac{3}{4}-k^2\right\}-z^2 \nonumber\\
     \therefore -\alpha&\le z\le \alpha \label{b}
     \end{align}
となり,この平面内での$L$の断面積$S(k)$は
     \begin{align*}
     S(k)&=\int_{-\alpha}^\alpha \left\{ f(z)-k\right\}dz \\
     &=\frac{1}{6}(\alpha+\alpha)^3 \\
     &=\frac{4}{3}\left(\frac{3}{4}-k-k^2\right)^{3/2} \\
     &=\frac{4}{3}\left\{1-\left(k+\frac{1}{2}\right)^2\right\}^{3/2}
     \end{align*}
である.\eqref{b}から$z$の存在条件は実数$\alpha$の存在条件に等しく,\eqref{a}から
     \begin{align*}
     &\frac{3}{4}-k-k^2\ge0 \\
     \therefore&\frac{-3}{2}\le k \le\frac{1}{2}
     \end{align*}
となる.以上から,もとめる体積$V$は
     \begin{align*}
     V&=\int_{-3/2}^{1/2}S(k)dk \\
     &=\frac{4}{3}\int_{-3/2}^{1/2}\left\{1-\left(k+\frac{1}{2}\right)^2\right\}^{3/2}dk \\
     &=\frac{4}{3}\int_{-1}^{1}\left(1-t^2\right)^{3/2}dt \ \ \ \ \ \ \ \ \ \ \ \ \ (t=k+\frac{1}{2}) \\
     &=\frac{8}{3}\int_0^{\pi/2}\left(1-s^2\right)^{3/2}\frac{ds}{d\theta}d\theta \ \ \ \ \ \ \ \ \ (t=s=\sin\theta%
     ,c=\cos\theta) \\
     &=\frac{8}{3}\int_0^{\pi/2}c^2(1-s^2)d\theta \\
     &=\frac{\pi}{2}\cdots\text{(答)}
     \end{align*}
となる.
\vspace{2zh}

{\bf[解2]}[解]と同様に,
     \[x\le y\le \frac{3}{4}-(x^2+z^2)\] 
の体積を求める.     
 $y=x+k$で立体を切断した断面の面積を$S(k)$とする.まず,$k$の範囲を求める.
      \begin{align*}
      \exists x\exists y\exists z\, ,\, &x\le y\le \frac{3}{4}-(x^2+z^2),y=x+k \\
      \exists x \exists z \, ,\, &x\le x+k\le \frac{3}{4}-(x^2+z^2) \\
      \exists x \exists z \, ,\, &0\le k\land \left(x+\frac{1}{2}\right)^2+z^2\le 1-k \\
      &0\le k\le 1 
      \end{align*}
である.また,切断面の$xz$平面への正射影は,$y$を消去して,
     \[ \left(x+\frac{1}{2}\right)^2+z^2\le 1-k \]
であり,これは半径$\sqrt{1-k}$の円である.故に正射影の面積は
     \[(1-k)\pi\]
$y=x+k$と$xz$平面の成す角は$\pi/4$であるから,
     \begin{align*}
     S(k)\cos \left(\frac{\pi}{4}\right)=(1-k)\pi \\
     S(k)=\sqrt{2}(1-k)\pi
     \end{align*}
である.$k$が$\Delta k$だけ変化すると,$y=x+k$は自身と垂直な方向へ$\Delta k/\sqrt{2}$だけ移動するから,
     \begin{align*}
     \int_0^1S(k)\frac{\sqrt{2}}{2}\,dk=\frac{\pi}{2}
     \end{align*}
が求める体積である.$\cdots$(答)
\newpage
\end{multicols}
\end{document}