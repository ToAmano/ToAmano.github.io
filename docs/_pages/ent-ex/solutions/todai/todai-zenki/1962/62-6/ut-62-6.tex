\documentclass[a4j]{jarticle}
\usepackage{amsmath}
\usepackage{ascmac}
\usepackage{amssymb}
\usepackage{enumerate}
\usepackage{multicol}
\usepackage{framed}
\usepackage{fancyhdr}
\usepackage{latexsym}
\usepackage{indent}
\usepackage{cases}
\usepackage[dvips]{graphicx}
\usepackage{color}
\allowdisplaybreaks
\pagestyle{fancy}
\lhead{}
\chead{}
\rhead{東京大学前期$1962$年$6$番}
\begin{document}
%分数関係


\def\tfrac#1#2{{\textstyle\frac{#1}{#2}}} %数式中で文中表示の分数を使う時


%Σ関係

\def\dsum#1#2{{\displaystyle\sum_{#1}^{#2}}} %文中で数式表示のΣを使う時


%ベクトル関係


\def\vector#1{\overrightarrow{#1}}  %ベクトルを表現したいとき(aベクトルを表現するときは\ver
\def\norm#1{|\overrightarrow{#1}|} %ベクトルの絶対値
\def\vtwo#1#2{ \left(%
      \begin{array}{c}%
      #1 \\%
      #2 \\%
      \end{array}%
      \right) }                        %2次元ベクトル成分表示
      
      \def\vthree#1#2#3{ \left(
      \begin{array}{c}
      #1 \\
      #2 \\
      #3 \\
      \end{array}
      \right) }                        %3次元ベクトル成分表示



%数列関係


\def\an#1{\verb|{|$#1$\verb|}|}


%極限関係

\def\limit#1#2{\stackrel{#1 \to #2}{\longrightarrow}}   %等式変形からの極限
\def\dlim#1#2{{\displaystyle \lim_{#1\to#2}}} %文中で数式表示の極限を使う



%積分関係

\def\dint#1#2{{\displaystyle \int_{#1}^{#2}}} %文中で数式表示の積分を使う時

\def\ne{\nearrow}
\def\se{\searrow}
\def\nw{\nwarrow}
\def\ne{\nearrow}


%便利なやつ

\def\case#1#2{%
 \[\left\{%
 \begin{array}{l}%
 #1 \\%
 #2%
 \end{array}%
 \right.\] }                           %場合分け
 
\def\1{$\cos\theta=c$,$\sin\theta=s$とおく.}  %cs表示を与える前書きシータ
\def\2{$\cos t=c$,$\sin t=s$とおく.}     %cs表示を与える前書きt
\def\3{$\cos x=c$,$\sin x=s$とおく.}                %cs表示を与える前書きx

\def\fig#1#2#3 {%
\begin{wrapfigure}[#1]{r}{#2 zw}%
\vspace*{-1zh}%
\input{#3}%
\end{wrapfigure} }           %絵の挿入


\def\a{\alpha}   %ギリシャ文字
\def\b{\beta}
\def\g{\gamma}

%問題番号のためのマクロ

\newcounter{nombre} %必須
\renewcommand{\thenombre}{\arabic{nombre}} %任意
\setcounter{nombre}{2} %任意
\newcounter{nombresub}[nombre] %親子関係を定義
\renewcommand{\thenombresub}{\arabic{nombresub}} %任意
\setcounter{nombresub}{0} %任意
\newcommand{\prob}[1][]{\refstepcounter{nombre}#1[問題 \thenombre]}
\newcommand{\probsub}[1][]{\refstepcounter{nombresub}#1(\thenombresub)}


%1-1みたいなカウンタ(todaiとtodaia)
\newcounter{todai}
\setcounter{todai}{0}
\newcounter{todaisub}[todai] 
\setcounter{todaisub}{0} 
\newcommand{\todai}[1][]{\refstepcounter{todai}#1 \thetodai-\thetodaisub}
\newcommand{\todaib}[1][]{\refstepcounter{todai}#1\refstepcounter{todaisub}#1 {\bf [問題 \thetodai.\thetodaisub]}}
\newcommand{\todaia}[1][]{\refstepcounter{todaisub}#1 {\bf [問題 \thetodai.\thetodaisub]}}


     \begin{oframed}
     曲線$y=6\sin(x/6)$の上で$x=2\pi$,$x=6\pi$なる点をそれぞれ$P$,$Q$とし,点$P$,$Q$における曲線の接線の交点を$R$とする.
     このとき
          \begin{enumerate}[(1)]
          \item $R$の座標を求めよ.
          \item 線分$PR$,$QR$と上の曲線とで囲まれる図形の面積を求めよ.
          \end{enumerate}
     \end{oframed}

\setlength{\columnseprule}{0.4pt}
\begin{multicols}{2}
{\bf[解]} 
     \begin{enumerate}[(1)]
     \item $h(x)=6\sin(x/6)$とおく.
          \[h'(x)=\cos(x/6)\]
     だから,$P$ ,$Q$における接線$l_p$,$l_q$は,
          \begin{align*}
               \begin{cases}
               l_p&:y=\dfrac{1}{2}(x-2\pi)+3\sqrt{3}\equiv f(x) \\
               l_q&:y=-(x-6\pi)\equiv g(x)
               \end{cases}
          \end{align*}
     だから,$R$はこれらの交点で,
          \[R(14\pi/3-2\sqrt{3},4\pi/3+2\sqrt{3})\]
     である.$\cdots$(答)
     
     \item グラフの概形は下図である.ただし,
          \[t=14\pi/3-2\sqrt{3}\]
     である.
          \begin{center}
          \scalebox{1}{%WinTpicVersion4.32a
{\unitlength 0.1in%
\begin{picture}(16.2000,20.0000)(3.8000,-26.0000)%
% STR 2 0 3 0 Black White  
% 4 590 2397 590 2410 4 400 0 0
% O
\put(5.9000,-24.1000){\makebox(0,0)[rt]{O}}%
% STR 2 0 3 0 Black White  
% 4 560 587 560 600 4 400 0 0
% $y$
\put(5.6000,-6.0000){\makebox(0,0)[rt]{$y$}}%
% STR 2 0 3 0 Black White  
% 4 2000 2427 2000 2440 4 400 0 0
% $x$
\put(20.0000,-24.4000){\makebox(0,0)[rt]{$x$}}%
% VECTOR 2 0 3 0 Black White  
% 2 600 2600 600 600
% 
\special{pn 8}%
\special{pa 600 2600}%
\special{pa 600 600}%
\special{fp}%
\special{sh 1}%
\special{pa 600 600}%
\special{pa 580 667}%
\special{pa 600 653}%
\special{pa 620 667}%
\special{pa 600 600}%
\special{fp}%
% VECTOR 2 0 3 0 Black White  
% 2 400 2400 2000 2400
% 
\special{pn 8}%
\special{pa 400 2400}%
\special{pa 2000 2400}%
\special{fp}%
\special{sh 1}%
\special{pa 2000 2400}%
\special{pa 1933 2380}%
\special{pa 1947 2400}%
\special{pa 1933 2420}%
\special{pa 2000 2400}%
\special{fp}%
% FUNC 2 0 3 0 Black White  
% 9 400 600 2000 2600 600 2400 800 2400 600 2200 600 600 1800 2600 0 4 1 0
% 6sin(x/6)
\special{pn 8}%
\special{pn 8}%
\special{pa 536 2600}%
\special{pa 539 2591}%
\special{ip}%
\special{pa 552 2550}%
\special{pa 555 2541}%
\special{ip}%
\special{pa 568 2500}%
\special{pa 571 2491}%
\special{ip}%
\special{pa 584 2450}%
\special{pa 587 2441}%
\special{ip}%
\special{ip}%
\special{pa 600 2400}%
\special{pa 605 2384}%
\special{pa 610 2369}%
\special{pa 620 2337}%
\special{pa 625 2322}%
\special{pa 635 2290}%
\special{pa 640 2275}%
\special{pa 650 2243}%
\special{pa 655 2228}%
\special{pa 660 2212}%
\special{pa 665 2197}%
\special{pa 670 2181}%
\special{pa 680 2151}%
\special{pa 685 2135}%
\special{pa 695 2105}%
\special{pa 700 2089}%
\special{pa 730 1999}%
\special{pa 735 1985}%
\special{pa 745 1955}%
\special{pa 750 1941}%
\special{pa 755 1926}%
\special{pa 765 1898}%
\special{pa 770 1883}%
\special{pa 790 1827}%
\special{pa 795 1814}%
\special{pa 805 1786}%
\special{pa 815 1760}%
\special{pa 820 1746}%
\special{pa 835 1707}%
\special{pa 840 1695}%
\special{pa 850 1669}%
\special{pa 885 1585}%
\special{pa 895 1563}%
\special{pa 900 1551}%
\special{pa 905 1540}%
\special{pa 910 1530}%
\special{pa 920 1508}%
\special{pa 930 1488}%
\special{pa 935 1477}%
\special{pa 940 1467}%
\special{pa 945 1458}%
\special{pa 955 1438}%
\special{pa 975 1402}%
\special{pa 980 1394}%
\special{pa 985 1385}%
\special{pa 1010 1345}%
\special{pa 1035 1310}%
\special{pa 1040 1304}%
\special{pa 1045 1297}%
\special{pa 1055 1285}%
\special{pa 1060 1280}%
\special{pa 1065 1274}%
\special{pa 1090 1249}%
\special{pa 1110 1233}%
\special{pa 1115 1230}%
\special{pa 1120 1226}%
\special{pa 1135 1217}%
\special{pa 1140 1215}%
\special{pa 1145 1212}%
\special{pa 1155 1208}%
\special{pa 1160 1207}%
\special{pa 1165 1205}%
\special{pa 1190 1200}%
\special{pa 1210 1200}%
\special{pa 1235 1205}%
\special{pa 1240 1207}%
\special{pa 1245 1208}%
\special{pa 1255 1212}%
\special{pa 1260 1215}%
\special{pa 1265 1217}%
\special{pa 1280 1226}%
\special{pa 1285 1230}%
\special{pa 1290 1233}%
\special{pa 1310 1249}%
\special{pa 1335 1274}%
\special{pa 1340 1280}%
\special{pa 1345 1285}%
\special{pa 1355 1297}%
\special{pa 1360 1304}%
\special{pa 1365 1310}%
\special{pa 1390 1345}%
\special{pa 1415 1385}%
\special{pa 1420 1394}%
\special{pa 1425 1402}%
\special{pa 1445 1438}%
\special{pa 1455 1458}%
\special{pa 1460 1467}%
\special{pa 1465 1477}%
\special{pa 1470 1488}%
\special{pa 1480 1508}%
\special{pa 1490 1530}%
\special{pa 1495 1540}%
\special{pa 1500 1551}%
\special{pa 1505 1563}%
\special{pa 1515 1585}%
\special{pa 1550 1669}%
\special{pa 1560 1695}%
\special{pa 1565 1707}%
\special{pa 1580 1746}%
\special{pa 1585 1760}%
\special{pa 1595 1786}%
\special{pa 1605 1814}%
\special{pa 1610 1827}%
\special{pa 1630 1883}%
\special{pa 1635 1898}%
\special{pa 1645 1926}%
\special{pa 1650 1941}%
\special{pa 1655 1955}%
\special{pa 1665 1985}%
\special{pa 1670 1999}%
\special{pa 1700 2089}%
\special{pa 1705 2105}%
\special{pa 1715 2135}%
\special{pa 1720 2151}%
\special{pa 1730 2181}%
\special{pa 1735 2197}%
\special{pa 1740 2212}%
\special{pa 1745 2228}%
\special{pa 1750 2243}%
\special{pa 1760 2275}%
\special{pa 1765 2290}%
\special{pa 1775 2322}%
\special{pa 1780 2337}%
\special{pa 1790 2369}%
\special{pa 1795 2384}%
\special{pa 1800 2400}%
\special{fp}%
\special{pn 8}%
\special{pa 1803 2409}%
\special{pa 1805 2416}%
\special{pa 1810 2431}%
\special{pa 1816 2450}%
\special{ip}%
\special{pa 1819 2459}%
\special{pa 1820 2463}%
\special{pa 1825 2478}%
\special{pa 1832 2500}%
\special{ip}%
\special{pa 1835 2509}%
\special{pa 1835 2510}%
\special{pa 1840 2525}%
\special{pa 1848 2550}%
\special{ip}%
\special{pa 1851 2559}%
\special{pa 1855 2572}%
\special{pa 1860 2588}%
\special{pa 1864 2600}%
\special{ip}%
% FUNC 2 0 3 0 Black White  
% 10 400 600 2000 2600 600 2400 800 2400 600 2200 400 600 2000 2600 0 4 1 0 0 0
% -x+6pi
\special{pn 8}%
\special{pa 1227 600}%
\special{pa 1230 609}%
\special{pa 1240 641}%
\special{pa 1245 656}%
\special{pa 1260 704}%
\special{pa 1265 719}%
\special{pa 1275 751}%
\special{pa 1280 766}%
\special{pa 1290 798}%
\special{pa 1295 813}%
\special{pa 1310 861}%
\special{pa 1315 876}%
\special{pa 1325 908}%
\special{pa 1330 923}%
\special{pa 1345 971}%
\special{pa 1350 986}%
\special{pa 1360 1018}%
\special{pa 1365 1033}%
\special{pa 1380 1081}%
\special{pa 1385 1096}%
\special{pa 1395 1128}%
\special{pa 1400 1143}%
\special{pa 1410 1175}%
\special{pa 1415 1190}%
\special{pa 1430 1238}%
\special{pa 1435 1253}%
\special{pa 1445 1285}%
\special{pa 1450 1300}%
\special{pa 1465 1348}%
\special{pa 1470 1363}%
\special{pa 1480 1395}%
\special{pa 1485 1410}%
\special{pa 1500 1458}%
\special{pa 1505 1473}%
\special{pa 1515 1505}%
\special{pa 1520 1520}%
\special{pa 1530 1552}%
\special{pa 1535 1567}%
\special{pa 1550 1615}%
\special{pa 1555 1630}%
\special{pa 1565 1662}%
\special{pa 1570 1677}%
\special{pa 1585 1725}%
\special{pa 1590 1740}%
\special{pa 1600 1772}%
\special{pa 1605 1787}%
\special{pa 1620 1835}%
\special{pa 1625 1850}%
\special{pa 1635 1882}%
\special{pa 1640 1897}%
\special{pa 1650 1929}%
\special{pa 1655 1944}%
\special{pa 1670 1992}%
\special{pa 1675 2007}%
\special{pa 1685 2039}%
\special{pa 1690 2054}%
\special{pa 1705 2102}%
\special{pa 1710 2117}%
\special{pa 1720 2149}%
\special{pa 1725 2164}%
\special{pa 1740 2212}%
\special{pa 1745 2227}%
\special{pa 1755 2259}%
\special{pa 1760 2274}%
\special{pa 1770 2306}%
\special{pa 1775 2321}%
\special{pa 1790 2369}%
\special{pa 1795 2384}%
\special{pa 1805 2416}%
\special{pa 1810 2431}%
\special{pa 1825 2479}%
\special{pa 1830 2494}%
\special{pa 1840 2526}%
\special{pa 1845 2541}%
\special{pa 1855 2573}%
\special{pa 1864 2600}%
\special{fp}%
% STR 2 0 3 0 Black White  
% 4 590 2397 590 2410 4 400 0 0
% O
\put(5.9000,-24.1000){\makebox(0,0)[rt]{O}}%
% STR 2 0 3 0 Black White  
% 4 560 587 560 600 4 400 0 0
% $y$
\put(5.6000,-6.0000){\makebox(0,0)[rt]{$y$}}%
% STR 2 0 3 0 Black White  
% 4 2000 2427 2000 2440 4 400 0 0
% $x$
\put(20.0000,-24.4000){\makebox(0,0)[rt]{$x$}}%
% VECTOR 2 0 3 0 Black White  
% 2 600 2600 600 600
% 
\special{pn 8}%
\special{pa 600 2600}%
\special{pa 600 600}%
\special{fp}%
\special{sh 1}%
\special{pa 600 600}%
\special{pa 580 667}%
\special{pa 600 653}%
\special{pa 620 667}%
\special{pa 600 600}%
\special{fp}%
% VECTOR 2 0 3 0 Black White  
% 2 400 2400 2000 2400
% 
\special{pn 8}%
\special{pa 400 2400}%
\special{pa 2000 2400}%
\special{fp}%
\special{sh 1}%
\special{pa 2000 2400}%
\special{pa 1933 2380}%
\special{pa 1947 2400}%
\special{pa 1933 2420}%
\special{pa 2000 2400}%
\special{fp}%
% FUNC 2 0 3 0 Black White  
% 10 400 600 2000 2600 600 2400 800 2400 600 2200 400 600 2000 2600 0 4 1 0 1 0
% x/2-pi+3sqrt(3)
\special{pn 8}%
\special{pa 400 2303}%
\special{pa 405 2295}%
\special{pa 410 2288}%
\special{pa 440 2240}%
\special{pa 445 2233}%
\special{pa 475 2185}%
\special{pa 480 2178}%
\special{pa 510 2130}%
\special{pa 515 2123}%
\special{pa 545 2075}%
\special{pa 550 2068}%
\special{pa 575 2028}%
\special{pa 580 2021}%
\special{pa 610 1973}%
\special{pa 615 1966}%
\special{pa 645 1918}%
\special{pa 650 1911}%
\special{pa 680 1863}%
\special{pa 685 1856}%
\special{pa 715 1808}%
\special{pa 720 1801}%
\special{pa 750 1753}%
\special{pa 755 1746}%
\special{pa 785 1698}%
\special{pa 790 1691}%
\special{pa 815 1651}%
\special{pa 820 1644}%
\special{pa 850 1596}%
\special{pa 855 1589}%
\special{pa 885 1541}%
\special{pa 890 1534}%
\special{pa 920 1486}%
\special{pa 925 1479}%
\special{pa 955 1431}%
\special{pa 960 1424}%
\special{pa 990 1376}%
\special{pa 995 1369}%
\special{pa 1025 1321}%
\special{pa 1030 1314}%
\special{pa 1055 1274}%
\special{pa 1060 1267}%
\special{pa 1090 1219}%
\special{pa 1095 1212}%
\special{pa 1125 1164}%
\special{pa 1130 1157}%
\special{pa 1160 1109}%
\special{pa 1165 1102}%
\special{pa 1195 1054}%
\special{pa 1200 1047}%
\special{pa 1230 999}%
\special{pa 1235 992}%
\special{pa 1260 952}%
\special{pa 1265 945}%
\special{pa 1295 897}%
\special{pa 1300 890}%
\special{pa 1330 842}%
\special{pa 1335 835}%
\special{pa 1365 787}%
\special{pa 1370 780}%
\special{pa 1400 732}%
\special{pa 1405 725}%
\special{pa 1435 677}%
\special{pa 1440 670}%
\special{pa 1470 622}%
\special{pa 1475 615}%
\special{pa 1480 607}%
\special{pa 1484 600}%
\special{fp}%
% LINE 2 2 3 0 Black White  
% 4 1000 1350 1000 2400 1310 870 1310 2400
% 
\special{pn 8}%
\special{pa 1000 1350}%
\special{pa 1000 2400}%
\special{dt 0.045}%
\special{pa 1310 870}%
\special{pa 1310 2400}%
\special{dt 0.045}%
% STR 2 0 3 0 Black White  
% 4 1310 2300 1310 2400 2 0 1 0
% $t$
\put(13.1000,-24.0000){\makebox(0,0)[lb]{{\colorbox[named]{White}{$t$}}}}%
% STR 2 0 3 0 Black White  
% 4 1000 2300 1000 2400 2 0 1 0
% $2pi$
\put(10.0000,-24.0000){\makebox(0,0)[lb]{{\colorbox[named]{White}{$2pi$}}}}%
% STR 2 0 3 0 Black White  
% 4 1800 2300 1800 2400 2 0 1 0
% $6pi$
\put(18.0000,-24.0000){\makebox(0,0)[lb]{{\colorbox[named]{White}{$6pi$}}}}%
% STR 2 0 3 0 Black White  
% 4 1800 2300 1800 2400 4 0 1 0
% $Q$
\put(18.0000,-24.0000){\makebox(0,0)[rt]{{\colorbox[named]{White}{$Q$}}}}%
% STR 2 0 3 0 Black White  
% 4 1000 1240 1000 1340 2 0 1 0
% $P$
\put(10.0000,-13.4000){\makebox(0,0)[lb]{{\colorbox[named]{White}{$P$}}}}%
% STR 2 0 3 0 Black White  
% 4 1310 780 1310 880 2 0 1 0
% $R$
\put(13.1000,-8.8000){\makebox(0,0)[lb]{{\colorbox[named]{White}{$R$}}}}%
% LINE 3 0 3 0 Black White  
% 24 1410 1170 1330 1250 1420 1220 1350 1290 1440 1260 1380 1320 1450 1310 1400 1360 1470 1350 1420 1400 1480 1400 1450 1430 1500 1440 1470 1470 1390 1130 1310 1210 1380 1080 1310 1150 1370 1030 1310 1090 1350 990 1310 1030 1340 940 1310 970
% 
\special{pn 4}%
\special{pa 1410 1170}%
\special{pa 1330 1250}%
\special{fp}%
\special{pa 1420 1220}%
\special{pa 1350 1290}%
\special{fp}%
\special{pa 1440 1260}%
\special{pa 1380 1320}%
\special{fp}%
\special{pa 1450 1310}%
\special{pa 1400 1360}%
\special{fp}%
\special{pa 1470 1350}%
\special{pa 1420 1400}%
\special{fp}%
\special{pa 1480 1400}%
\special{pa 1450 1430}%
\special{fp}%
\special{pa 1500 1440}%
\special{pa 1470 1470}%
\special{fp}%
\special{pa 1390 1130}%
\special{pa 1310 1210}%
\special{fp}%
\special{pa 1380 1080}%
\special{pa 1310 1150}%
\special{fp}%
\special{pa 1370 1030}%
\special{pa 1310 1090}%
\special{fp}%
\special{pa 1350 990}%
\special{pa 1310 1030}%
\special{fp}%
\special{pa 1340 940}%
\special{pa 1310 970}%
\special{fp}%
% LINE 3 0 3 0 Black White  
% 12 1310 1090 1200 1200 1310 1030 1120 1220 1310 970 1140 1140 1310 910 1240 980 1310 1150 1250 1210 1310 1210 1290 1230
% 
\special{pn 4}%
\special{pa 1310 1090}%
\special{pa 1200 1200}%
\special{fp}%
\special{pa 1310 1030}%
\special{pa 1120 1220}%
\special{fp}%
\special{pa 1310 970}%
\special{pa 1140 1140}%
\special{fp}%
\special{pa 1310 910}%
\special{pa 1240 980}%
\special{fp}%
\special{pa 1310 1150}%
\special{pa 1250 1210}%
\special{fp}%
\special{pa 1310 1210}%
\special{pa 1290 1230}%
\special{fp}%
\end{picture}}%
}
          \end{center}
     従って,求める面積$S$として,
          \begin{align}
          S&=\int_{2\pi}^tfdx+\int_t^{6\pi}gdx-\int_{2\pi}^{6\pi}hdx \nonumber\\
          &=\frac{8}{3}\pi^2+8\sqrt{3}\pi-63\nonumber
          \end{align}
     である.$\cdots$(答)
     \end{enumerate}
\newpage
\end{multicols}
\end{document}