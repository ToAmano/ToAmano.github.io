\documentclass[a4j]{jarticle}
\usepackage{amsmath}
\usepackage{ascmac}
\usepackage{amssymb}
\usepackage{enumerate}
\usepackage{multicol}
\usepackage{framed}
\usepackage{fancyhdr}
\usepackage{latexsym}
\usepackage{indent}
\usepackage{cases}
\usepackage[dvips]{graphicx}
\usepackage{color}
\allowdisplaybreaks
\pagestyle{fancy}
\lhead{}
\chead{}
\rhead{東京大学後期$2006$年$2$番}
\begin{document}
%分数関係


\def\tfrac#1#2{{\textstyle\frac{#1}{#2}}} %数式中で文中表示の分数を使う時


%Σ関係

\def\dsum#1#2{{\displaystyle\sum_{#1}^{#2}}} %文中で数式表示のΣを使う時


%ベクトル関係


\def\vector#1{\overrightarrow{#1}}  %ベクトルを表現したいとき(aベクトルを表現するときは\ver
\def\norm#1{|\overrightarrow{#1}|} %ベクトルの絶対値
\def\vtwo#1#2{ \left(%
      \begin{array}{c}%
      #1 \\%
      #2 \\%
      \end{array}%
      \right) }                        %2次元ベクトル成分表示
      
      \def\vthree#1#2#3{ \left(
      \begin{array}{c}
      #1 \\
      #2 \\
      #3 \\
      \end{array}
      \right) }                        %3次元ベクトル成分表示



%数列関係


\def\an#1{\verb|{|$#1$\verb|}|}


%極限関係

\def\limit#1#2{\stackrel{#1 \to #2}{\longrightarrow}}   %等式変形からの極限
\def\dlim#1#2{{\displaystyle \lim_{#1\to#2}}} %文中で数式表示の極限を使う



%積分関係

\def\dint#1#2{{\displaystyle \int_{#1}^{#2}}} %文中で数式表示の積分を使う時

\def\ne{\nearrow}
\def\se{\searrow}
\def\nw{\nwarrow}
\def\ne{\nearrow}


%便利なやつ

\def\case#1#2{%
 \[\left\{%
 \begin{array}{l}%
 #1 \\%
 #2%
 \end{array}%
 \right.\] }                           %場合分け
 
\def\1{$\cos\theta=c$,$\sin\theta=s$とおく.}  %cs表示を与える前書きシータ
\def\2{$\cos t=c$,$\sin t=s$とおく.}     %cs表示を与える前書きt
\def\3{$\cos x=c$,$\sin x=s$とおく.}                %cs表示を与える前書きx

\def\fig#1#2#3 {%
\begin{wrapfigure}[#1]{r}{#2 zw}%
\vspace*{-1zh}%
\input{#3}%
\end{wrapfigure} }           %絵の挿入


\def\a{\alpha}   %ギリシャ文字
\def\b{\beta}
\def\g{\gamma}

%問題番号のためのマクロ

\newcounter{nombre} %必須
\renewcommand{\thenombre}{\arabic{nombre}} %任意
\setcounter{nombre}{2} %任意
\newcounter{nombresub}[nombre] %親子関係を定義
\renewcommand{\thenombresub}{\arabic{nombresub}} %任意
\setcounter{nombresub}{0} %任意
\newcommand{\prob}[1][]{\refstepcounter{nombre}#1[問題 \thenombre]}
\newcommand{\probsub}[1][]{\refstepcounter{nombresub}#1(\thenombresub)}


%1-1みたいなカウンタ(todaiとtodaia)
\newcounter{todai}
\setcounter{todai}{0}
\newcounter{todaisub}[todai] 
\setcounter{todaisub}{0} 
\newcommand{\todai}[1][]{\refstepcounter{todai}#1 \thetodai-\thetodaisub}
\newcommand{\todaib}[1][]{\refstepcounter{todai}#1\refstepcounter{todaisub}#1 {\bf [問題 \thetodai.\thetodaisub]}}
\newcommand{\todaia}[1][]{\refstepcounter{todaisub}#1 {\bf [問題 \thetodai.\thetodaisub]}}


     \begin{oframed}
     $a$を正の実数,$\theta$を$0\le\theta\le\pi/2$をみたす実数とする.$xyz$空間において,点$(a,0,0)$と点$(a+\cos\theta,0,\sin\theta)$
     を結ぶ線分を,$x$軸のまわりに一回転させてできる曲面を$S$とする,さらに$S$を$y$軸のまわりに一回転させてできる立体の
     体積を$V$とする.
          \begin{enumerate}[(1)]
          \item $V$を$a$と$\theta$で表せ.
          \item $a=4$とする.$V$を$\theta$の関数と考えて,$V$の最大値を求めよ.
          \end{enumerate}
     \end{oframed}

\setlength{\columnseprule}{0.4pt}
\begin{multicols}{2}
{\bf[解]} \1 $S$は$A(a,0,0)$を頂点とし,$x$軸を軸とする円錐側面の一部である.$S$の方程式は,
     \begin{align*}
          &\begin{cases}
          \sqrt{(x-a)^2+y^2+z^2}c=(x-a) \\
          a\le x\le a+c  
          \end{cases} \\
          &\begin{cases}
          ((x-a)^2+y^2+z^2)c^2=(x-a)^2 \\
          a\le x\le a+c 
          \end{cases} \\
          &\begin{cases}
          y^2+z^2=(x-a)^2\tan^2\theta \\
          a\le x\le a+c
          \end{cases} 
     \end{align*}
これは$y$軸に関して対称だから,$V$のうち$y\ge0$の部分の体積$v$として
     \begin{align}
     V=2v\label{1}
     \end{align}
である.$S$を$y=t(0\le y\le s)$で切ると
     \begin{align}
          \begin{cases}
          \tan^2\theta(x-a)^2-z^2=t^2 \\
          a\le x\le a+c
          \end{cases}
     \end{align}
これは双曲線であり,概形は下図.
     \begin{center}
     \scalebox{1}{a}
     \end{center}     
従って,上図の記号を用いて,この平面での立体の面積$S(t)$は,
     \begin{align*}
     \frac{S(t)}{\pi}&=|OA|^2-|OB|^2 \\
     &=\left\{(a+c)^2+(s^2-t^2)\right\}-\left(a+\frac{c}{s}t\right)^2 \\
     &=-\left(1+\frac{c^2}{s^2}\right)t^2-2a\frac{c}{s}t+c^2+s^2+2ac \\
     &=-\frac{1}{s^2}t^2-\frac{2ac}{s}t+1+2ac
     \end{align*}
だから,求める体積は
     \begin{align*}
     v&=\int_0^sS(t)dt \\
     &=\pi\left[\frac{-1}{3s^2}t^3-\frac{ac}{s}t^2+(2ac+1)t\right]_0^s \\
     &=\pi\left(\frac{2}{3}+ac\right)s
    \end{align*}          
だから\eqref{1}に代入して,
     \[V=2\pi\left(\frac{2}{3}+ac\right)s\]     
である.$\cdots$((1)の答)

$a=4$として代入すると,
      \begin{align*}
      V&=\frac{4\pi}{3}(6cs+s) \\
      V'&=\frac{4\pi}{3}(6\cos2\theta+c) \\
      &=\frac{4\pi}{3}(12c^2+c-6) \\
      &=\frac{4\pi}{3}(3c-2)(4c+3)
      \end{align*}
だから,$0\le\theta\le\pi/2$に注意して,同区間内で$\cos\theta_0=2/3$を満たす数を用いて下表を得る.
     \begin{align*}
          \begin{array}{|c|c|c|c|c|c|} \hline
          \theta&0  &    &\theta_0&   &\pi/2 \\\hline
          c       & 1 &    &  2/3      &     &    0  \\\hline
          V'     &    & +  &    0       &-   &         \\\hline
          V      &    &\ne&            &\se&        \\\hline
          \end{array}
     \end{align*}      
故に$c=2/3$のとき(このとき$s=\sqrt{5}/3$)
     \[\max V=\frac{20\sqrt{5}}{9}\pi\]
をとる.$\cdots$((2)の答)
\newpage
\end{multicols}
\end{document}