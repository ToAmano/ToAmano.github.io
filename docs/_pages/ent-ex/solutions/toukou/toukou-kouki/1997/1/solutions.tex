\documentclass[a4paper,10pt]{ltjsarticle}
\usepackage{luatexja}
\usepackage[hiragino-pron]{luatexja-preset}

\usepackage[truedimen,top=25truemm,bottom=20truemm,left=15truemm,right=15truemm]{geometry}
\setlength{\textwidth}{54\zw}
\setlength{\textheight}{73\zw}

\usepackage{amsmath,amssymb,ascmac}
\usepackage{enumerate}
\usepackage{multicol}
\usepackage{physics}
\usepackage{cleveref}
\usepackage{framed}
\usepackage{fancyhdr}
\usepackage{latexsym}
\usepackage{mathtools}
\usepackage{tikz}
\usepackage{tikz-3dplot}
\usepackage{pgfplots}
\pgfplotsset{compat=1.18}
\usetikzlibrary{intersections}
\usepgfplotslibrary{fillbetween}
 \usetikzlibrary{math}
 \usetikzlibrary{calc}
 \usetikzlibrary{angles}
 \usetikzlibrary{quotes}
% \usepackage{indent}
\usepackage{cases}
\usepackage{float}
\allowdisplaybreaks
\pagestyle{fancy}
\lhead{}
\chead{}
\rhead{東工大後期$1997$年$1$番}
\begin{document}
\begin{oframed}
放物線 $y=x^2$ を $C_1$ とし, $C_1$ 上に両端をもつ長さ $1$ の線分の中点の軌跡を $C_2$ とする. $C_1$, $C_2$ および $2$ 直線 $x=\pm a$ ($a>0$) で囲まれる部分の面積を $S_a$ とするとき, $\displaystyle \lim_{a \to \infty} S_a$ を求めよ.
\end{oframed}
\setlength{\columnseprule}{0.4pt}
\begin{multicols}{2}
{\bf[解]}

(1)
\subsection{$C_2$の軌跡}
まず,$C_2$ の軌跡を求める. 題意の直線の 2 端点 P$(\alpha,\alpha^2)$, Q$(\beta,\beta^2)$ ($\alpha < \beta$) とおく.この概形は\cref{fig:2}である.

\begin{figure}[H]
\begin{tikzpicture}{scale=2.0}
\begin{axis}[
    axis lines=middle,
    xlabel=$x$,
    ylabel=$y$,
    xmin=-2.5, xmax=2.5,
    ymin=0, ymax=2.0,
    xtick={\empty},
    ytick={\empty},
    tick style={draw=none},
    no markers,
    clip=false % Allows drawing outside the axis limits if needed
]

% Define the functions
\addplot[name path=C1, thick, domain=-1.2:1.2] {x^2};
\draw (axis cs:0.5,0.25) -- (axis cs:-1, 1);
\node[right] at (axis cs:0.5,0.25) {$Q$};
\node[left] at (axis cs:-1,1) {$P$};
\fill (axis cs:0.5,0.25) circle [radius=1pt];
\fill (axis cs:-1,1) circle [radius=1pt];

% Labels for C1 and C2
\node at (axis cs:1.2, 0.5) {$C_1$};

\end{axis}
\end{tikzpicture}
\caption{$C_1$の概形}
\label{fig:2}
\end{figure}

$|PQ| > 0$, $|PQ|=1 \Longleftrightarrow |PQ|^2=1$ だから,
\begin{align}
&(\beta-\alpha)^2+(\beta^2-\alpha^2)^2 = 1 \nonumber \\
\therefore &(\beta-\alpha)^2\left[1 + (\alpha+\beta)^2\right] = 1 \label{eq:1}
\end{align}
ここで,$p=\alpha+\beta$, $q=\beta-\alpha$ とおく.$\alpha,\beta \in\mathbb{R}$ および $\alpha < \beta$から
\begin{align}
 q >0    \label{eq:2}
\end{align}
となる.題意の中点を$M(X,Y)$ とすると
\begin{align}
X &= \frac{1}{2}(\alpha+\beta)=\frac{p}{2} \label{eq:3}\\
Y &= \frac{1}{2}\left(\alpha^2+\beta^2\right) = \frac{1}{4}\left[(\alpha+\beta)^2+(\beta-\alpha)^2\right] = \frac{1}{4}\left(p^2+q^2\right)\label{eq:4}
\end{align}
となることに注意する.さらに,\cref{eq:1}を$p$ と $q$ で書くと
\begin{align*}
 & q^2(1+p^2) &= 1 \\
 \therefore 
 &q^2 = \frac{1}{1+p^2} \quad (\because 1+p^2\neq 0)
\end{align*}
で,これは\cref{eq:2}を満たす.\cref{eq:3,eq:4}に代入して$p$および$q$を消去して$X$, $Y$の関係を求めれば,それが求める軌跡である.
\cref{eq:3}から$p=2X$だから,\cref{eq:4}から,
\begin{align*}
Y 
&= \frac{1}{4}\left(p^2+1/(1+p^2)\right) \\
&= \frac{1}{4}\left((2X)^2+1/(1+(2X)^2)\right) \\
&= \frac{1}{4}\left((4X^2+1/(1+4X^2)\right) 
\end{align*}
である,これが$C_2$の軌跡である.$X$はすべての実数を取る.

\subsection{$S_{a}$の極限値}

以上より,求める面積$S_{a}$は\cref{fig:1}の斜線部である.

\begin{figure}[H]
\begin{tikzpicture}{scale=2.0}
\begin{axis}[
    axis lines=middle,
    xlabel=$x$,
    ylabel=$y$,
    xmin=-2.5, xmax=2.5,
    ymin=0, ymax=2.0,
    xtick={\empty},
    ytick={\empty},
    tick style={draw=none},
    no markers,
    clip=false % Allows drawing outside the axis limits if needed
]

% Define the functions
\addplot[name path=C1, blue, thick, domain=-1.2:1.2] {x^2};
\addplot[name path=C2, red, thick, domain=-1.2:1.2] {x^2 + 1/(4*(1+4*x^2))};

% Draw the vertical lines at x=-2 and x=2
\draw[dashed] (axis cs:-1,0) -- (axis cs:-1, {(-1)^2 + 1/(4*(1+4*(-1)^2))});
\draw[dashed] (axis cs:1,0) -- (axis cs:1, {1^2 + 1/(4*(1+4*1^2))});

% Fill the region between C1 and C2 from x=-2 to x=2
\addplot[gray!50, fill opacity=0.5] fill between[of=C2 and C1];

% Labels for C1 and C2
\node[blue] at (axis cs:1.2, 0.5) {$C_1$};
\node[red] at  (axis cs:1, 1.5) {$C_2$};

% Labels for -a and a, which are now -2 and 2
\node[below] at (axis cs:-1, 0) {$-a$};
\node[below] at (axis cs:1, 0) {$a$};

\end{axis}
\end{tikzpicture}
\caption{$C_1$と$C_2$の概形}
\label{fig:1}
\end{figure}

$C_1$と$C_2$が偶関数だから,$0\le x$の部分の面積の$2$倍が求める面積であり,
\begin{align}
\frac{1}{2}S_a 
&= \int_0^a \left(x^2+\frac{1}{4(1+4x^2)} - x^2 \right)dx \nonumber \\
&= \frac{1}{4}\int_0^a \frac{1}{1+4x^2} dx \label{eq:5}
\end{align}
ここで,$x=\frac{1}{2}\tan\theta$ ($0 \le \theta < \pi/2$) とすると,$\frac{dx}{d\theta} = \frac{1}{2\cos^2\theta}$,又 $a = \frac{1}{2}\tan \alpha$ となる $\alpha$ があるので,
\cref{eq:5}に代入して
\begin{align}
\frac{1}{2}S_a 
&= \frac{1}{4}\int_0^\alpha \frac{1}{1+\tan^2\theta} \cdot \frac{1}{a2\cos^2\theta} d\theta 
 \nonumber \\
&= \frac{1}{8}\int_0^\alpha d\theta \nonumber \\
&= \frac{\alpha}{8} \nonumber \\
\therefore 
S_a &= \frac{\alpha}{4} \label{eq:6}
\end{align}
と面積が求まる.$a = \frac{1}{2}\tan \alpha$より,$a\to\infty$で$\alpha\to\pi/2$だから,\cref{eq:6}の極限は
\begin{align*}
 \lim_{a\to\infty} S_a = \frac{1}{4} \frac{\pi}{2} = \frac{\pi}{8}     
\end{align*}
となる.$\cdots$(答)

{\bf[解説]}

     \newpage
\end{multicols}
\end{document}