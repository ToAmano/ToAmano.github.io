% TODO :: 解説の方の式変形を拡充する
\documentclass[a4paper,10pt]{ltjsarticle}
\usepackage{luatexja}
\usepackage[hiragino-pron]{luatexja-preset}

\usepackage[truedimen,top=25truemm,bottom=20truemm,left=15truemm,right=15truemm]{geometry}
\setlength{\textwidth}{54\zw}
\setlength{\textheight}{73\zw}

\usepackage{amsmath,amssymb,ascmac}
\usepackage{enumerate}
\usepackage{multicol}
\usepackage{cleveref}
\usepackage{framed}
\usepackage{fancyhdr}
\usepackage{latexsym}
\usepackage{mathtools}
\usepackage{tikz}
\usepackage{tikz-3dplot}
\usepackage{pgfplots}
 \usetikzlibrary{math}
% \usepackage{indent}
\usepackage{cases}
\usepackage{float}
\usepackage{cases}
  \usepackage{caption}
  \usepackage[subrefformat=parens]{subcaption}
\allowdisplaybreaks
\tdplotsetmaincoords{70}{110}
\pagestyle{fancy}
\lhead{}
\chead{}
\rhead{東工大後期$1995$年$2$番}
\begin{document}
\begin{oframed}
四角形 $ABCD$ と頂点 $O$ からなる四角錐を考える. $5$ 点 $A, B, C, D, O$ の中の $2$ 点は, ある辺の両端にあるとき, 互いに隣接点であるという.

今, $O$ から出発し, その隣接点の中から $1$ 点を等確率で選んでその点を $X_1$ とする. 次に $X_1$ の隣接点の中から $1$ 点を等確率で選びその点を $X_2$ とする. この様にして順次 $X_1, X_2, X_3, \dots, X_n$ を定めるとき, $X_n$ が $O$ に一致する確率を求めよ.
\end{oframed}
\setlength{\columnseprule}{0.4pt}
\begin{multicols}{2}
{\bf[解]}

\begin{figure}[H]
  \centering
\begin{tikzpicture}[scale=1,tdplot_main_coords]
        % Define coordinates
        \coordinate (O) at (0,0,3);
        \coordinate (A) at (-2,-2,0);
        \coordinate (B) at (2,-2,0);
        \coordinate (C) at (2,2,0);
        \coordinate (D) at (-2,2,0);

        % Draw base (dashed for back edges)
        \draw[dashed] (A) -- (D);
        \draw[dashed] (A) -- (B);
        \draw (B) -- (C) -- (D);

        % Draw edges to apex
        \draw[dashed] (O) -- (A);
        \draw (O) -- (B);
        \draw (O) -- (C);
        \draw (O) -- (D);

        % Label vertices
        \node at (O) [above] {O};
        \node at (A) [below] {A};
        \node at (B) [below] {B};
        \node at (C) [below] {C};
        \node at (D) [below] {D};
\end{tikzpicture}
\caption{四角錐OABCD}
\end{figure}

 対称性から、$X_n$がAに一致する確率を$q_n$とおくと、$X_n$がB,C,Dに一致する確率も$q_n$にひとしい.
$X_n$がOに一致する確率を$p_n$とおく.$X_n$はO,A,B,C,Dのうちいずれかに一致するから,全体の確率が$1$となることより
\begin{align}
 p_n + 4q_n = 1 \label{eq:1}
\end{align}
を満たす.

次に$p_n$の漸化式を求める.$n+1$で$X_n$が$O$に等しい時,$n$では$X_n$はA,B,C,Dのいずれかに存在し,そこから確率$1/3$で$O$に移動する.したがって漸化式は
\begin{align*}
 p_{n+1} = \frac{4}{3}q_n
\end{align*}
である.\cref{eq:1}を代入して$q_n$を消去して
\begin{align}
  p_{n+1} &= \frac{1}{3}(1-p_n)
\end{align}
これと$p_1=0$から
\begin{align}
p_n 
&= \left(-\frac{1}{3}\right)^{n-1}\left(0-\frac{1}{4}\right)+\frac{1}{4} \\
&= \frac{1}{4}\left\{1-\left(-\frac{1}{3}\right)^{n-1}\right\}
\end{align}
となる.$\cdots$(答)

\vspace{10pt}
{\bf[解説]}


\newpage
\end{multicols}
\end{document}