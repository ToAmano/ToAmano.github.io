\documentclass[a4paper,10pt]{ltjsarticle}
\usepackage{luatexja}
\usepackage[hiragino-pron]{luatexja-preset}

\usepackage[truedimen,top=25truemm,bottom=20truemm,left=15truemm,right=15truemm]{geometry}
\setlength{\textwidth}{54\zw}
\setlength{\textheight}{73\zw}

\usepackage{amsmath,amssymb,ascmac}
\usepackage{enumerate}
\usepackage{multicol}
\usepackage{cleveref}
\usepackage{framed}
\usepackage{fancyhdr}
\usepackage{latexsym}
\usepackage{mathtools}
\usepackage{tikz}
\usepackage{tikz-3dplot}
\usepackage{pgfplots}
 \usetikzlibrary{math}
 \usetikzlibrary{calc}
 \usetikzlibrary{angles}
 \usetikzlibrary{quotes}
 \pgfplotsset{compat=1.18}
% \usepackage{indent}
\usepackage{caption}
\usepackage{subcaption}
\usepackage{cases}
\usepackage{float}
\usepackage{cases}
  \usepackage{caption}
  \usepackage[subrefformat=parens]{subcaption}
\allowdisplaybreaks
\tdplotsetmaincoords{70}{110}
\pagestyle{fancy}
\lhead{}
\chead{}
\rhead{東工大後期$2008$年$2$番}
\begin{document}
\begin{oframed}
  自然数 $n$ に対して
  \begin{align*}
    I_n = \int_0^1 x^2 |\sin n\pi x| dx
  \end{align*}
  とおく.極限値 $\displaystyle \lim_{n \to \infty} I_n$ を求めよ.
\end{oframed}


\setlength{\columnseprule}{0.4pt}
\begin{multicols}{2}
  {\bf[解]}
  以下,
  \begin{align*}
    a_k = \int_{\frac{k-1}{n}}^{\frac{k}{n}} x^2 \left|\sin n\pi x\right| \,dx
  \end{align*}
  とおく.
  $f(x)=x^2$ とおき,$\left[ \frac{k-1}{n}, \frac{k}{n} \right]$ で $f(x)$ の最大,最小を与える$x$ をそれぞれ $M_k, m_k$ とすると,$|\sin n\pi x| \ge 0$ から,
  \begin{align*}
    f(m_k) |\sin n\pi x| \le f(x) |\sin n\pi x| \le f(M_k) |\sin n\pi x|
  \end{align*}
  なる不等式を満たす.
  両辺を$\left[ \frac{k-1}{n}, \frac{k}{n} \right]$で積分して
  \begin{align}
    \int_{\frac{k-1}{n}}^{\frac{k}{n}} f(m_k) |\sin n\pi x| \,dx \le a_k \le \int_{\frac{k-1}{n}}^{\frac{k}{n}} f(M_k) |\sin n\pi x| \,dx \label{eq:1}
  \end{align}
  である.ここで両辺の積分は
  \begin{align*}
    \int_{\frac{k-1}{n}}^{\frac{k}{n}} |\sin n\pi x| \,dx = \frac{2}{n\pi}
  \end{align*}
  と実行できるから,\cref{eq:1}に代入して
  \begin{align*}
    \frac{2}{n\pi} f(m_k) \le a_k \le \frac{2}{n\pi} f(M_k)
  \end{align*}
  を得る.$k=1,2,\cdots,n$ について和をとって
  \begin{align}
    \frac{2}{n\pi} \sum_{k=1}^{n} f(m_k) \le \sum_{k=1}^{n} a_k \le \frac{2}{n\pi} \sum_{k=1}^{n} f(M_k) \nonumber \\
    \therefore
    \frac{2}{n\pi} \sum_{k=1}^{n} f(m_k) \le I_n \le \frac{2}{n\pi} \sum_{k=1}^{n} f(M_k) \label{eq:2}
  \end{align}
  である.両辺の和は区分求積法によって評価でき,$n\to\infty$のとき
  \begin{align*}
    \begin{dcases}
      \lim_{n\to\infty}\frac{1}{n} \sum_{k=1}^{n} f(m_k) = \int_{0}^{1} f(x) \,dx = \frac{1}{3} \\
      \lim_{n\to\infty}\frac{1}{n} \sum_{k=1}^{n} f(M_k) = \int_{0}^{1} f(x) \,dx = \frac{1}{3}
    \end{dcases}
  \end{align*}
  だから,\cref{eq:2}に代入して,挟み撃ちの原理から求める極限値は
  \begin{align*}
    \lim_{n\to\infty}I_n = \frac{2}{3\pi}
  \end{align*}
  である.$\cdots$(答)


  \vspace{10pt}
  {\bf[解説]}
  典型的な積分と極限の問題.
  類題として1999年の第1問が挙げられるが,解法はほぼ同じなので
  過去問演習を行なっていた人にとってはボーナス問題と思われる.



  \newpage
\end{multicols}
\end{document}