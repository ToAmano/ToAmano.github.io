\documentclass[a4paper,10pt]{ltjsarticle}
\usepackage{luatexja}
\usepackage[hiragino-pron]{luatexja-preset}

\usepackage[truedimen,top=25truemm,bottom=20truemm,left=15truemm,right=15truemm]{geometry}
\setlength{\textwidth}{54\zw}
\setlength{\textheight}{73\zw}

\usepackage{amsmath,amssymb,ascmac}
\usepackage{enumerate}
\usepackage{multicol}
\usepackage{physics}
\usepackage{cleveref}
\usepackage{framed}
 \usepackage{caption}
 \usepackage{subcaption}
\usepackage{fancyhdr}
\usepackage{latexsym}
\usepackage{mathtools}
\usepackage{tikz}
\usepackage{tikz-3dplot}
\usepackage{pgfplots}
\pgfplotsset{compat=1.18}
\usetikzlibrary{intersections}
\usepgfplotslibrary{fillbetween}
 \usetikzlibrary{math}
 \usetikzlibrary{calc}
 \usetikzlibrary{angles}
 \usetikzlibrary{quotes}
 \usetikzlibrary{patterns}

% \usepackage{indent}

\usepackage{cases}
\usepackage{float}
\allowdisplaybreaks
\pagestyle{fancy}
\lhead{}
\chead{}
\rhead{東工大後期$2008$年$1$番}
\begin{document}

\begin{oframed}
  次の問に答えよ.
  \begin{enumerate}
    \item 実数 $a_1, a_2, x_1, x_2, y_1, y_2$ が
          \begin{align*} 0 &< a_1 \le a_2 \\ a_1 x_1 &\le a_1 y_1 \\ a_1 x_1 + a_2 x_2 &\le a_1 y_1 + a_2 y_2 \end{align*}
          をみたすとしている.このとき $x_1 + x_2 \le y_1 + y_2$ であることを証明せよ.
    \item $n$ を $2$ 以上の整数とし,$3n$ 個の実数 $a_1, a_2, \dots, a_n, x_1, x_2, \dots, x_n, y_1, y_2, \dots, y_n$ が
          \[ 0 < a_1 \le a_2 \le \dots \le a_n \]
          および $n$ 個の不等式
          \[ \sum_{i=1}^j a_i x_i \le \sum_{i=1}^j a_i y_i \quad (j=1, 2, \dots, n) \]
          をみたしているならば,
          \[ \sum_{i=1}^n x_i \le \sum_{i=1}^n y_i \]
          であることを証明せよ.
  \end{enumerate}
\end{oframed}
\setlength{\columnseprule}{0.4pt}
\begin{multicols}{2}
  {\bf[解]}

  問題の構成として,(1)は(2)の$n=2$の時を証明する問題になっている.
  そこで,(2)を数学的帰納法により証明すれば(1)が含まれていることになるため,いきなり(2)を考える.

  各不等式に式番号を与える.
  \begin{align}
     & 0 < a_1 \le a_2 \le \dots \le a_n                                          \label{eq:1}  \\
     & A_j: \sum_{i=1}^j a_i x_i \le \sum_{i=1}^j a_i y_i \quad (j=1, 2, \dots, n) \label{eq:2} \\
     & \sum_{i=1}^n x_i \le \sum_{i=1}^n y_i \label{eq:3}
  \end{align}

  \subsection{$n=2$のとき}
  まず,$A_1$は
  \begin{align*}
    a_1x_1 \le a_1y_1
  \end{align*}
  であり,\cref{eq:1}より$a_1>0$だから両辺$a_1$で割って
  \begin{align}
    x_1 \le y_1 \label{eq:4}
  \end{align}
  である.

  $A_2$は
  \begin{align*}
    a_1x_1+a_2x_2 \le a_1y_1+a_2y_2
  \end{align*}
  であり,まず$a_1$と$a_2$についてこの不等式を整理すると
  \begin{align}
    a_2\left(x_2-y_2\right) \le a_1\left(y_1-x_1\right) \label{eq:5}
  \end{align}
  である.さらに,\cref{eq:1}より$0<a_1\le a_2$だから,右側の不等式はさらに上から評価すると
  \begin{align}
    a_1\left(y_1-x_1\right) \le a_2 \left(y_1-x_1\right) \label{eq:6}
  \end{align}
  ただし,\cref{eq:4}より$y_1-x_1\ge 0$であることを用いた.以上\cref{eq:5,eq:6}から
  \begin{align*}
    a_2\left(x_2-y_2\right) \le a_2 \left(y_1-x_1\right)
  \end{align*}
  である.両辺$a_2>0$で割って
  \begin{align*}
     & x_2-y_2 \le y_1-x_1 \\
    \therefore
     & x_1+x_2 \le y_1+y_2
  \end{align*}
  であり,これは\cref{eq:3}であるから$n=2$のとき\cref{eq:3}は成り立つ.
  ((1)の証明)  $\cdots$(答)

  \subsection{$n\le k (k \in \mathbb{N} \ge 2)$での成立を仮定}

  \cref{eq:1,eq:2}を仮定する.
  $A_j (j=3,4,\cdots,k+1)$について.$n=2$の時と同じく
  \begin{align*}
     & a_2 (x_2 - y_2) + \sum_{i=3}^{j} a_i x_i \le a_1 (y_1 - x_1) + \sum_{i=3}^{j} a_i y_i \\
     & a_1 (y_1 - x_1) + \sum_{i=3}^{j} a_i y_i \le a_2 (y_1 - x_1) + \sum_{i=3}^{j} a_i y_i
  \end{align*}
  が成り立つ.従って
  \begin{align}
    a_2 (x_2 - y_2) + \sum_{i=3}^{j} a_i x_i \le a_2 (y_1 - x_1) + \sum_{i=3}^{j} a_i y_i \label{eq:7}
  \end{align}
  である.

  ここで新しい数列$\{a'_i\}, \{X_i\}, \{Y_i\}$を
  \begin{align}
    \begin{cases}
      a'_i = a_{i+1}                 & (i=1,2,\cdots,k) \\
      X_1 = x_2 - y_2, X_i = x_{i+1} & (i=2,3,\cdots,k) \\
      Y_1 = y_1 - x_1, Y_i = y_{i+1} & (i=2,3,\cdots,k)
    \end{cases} \label{eq:10}
  \end{align}
  と定めると,$a_i$が\cref{eq:1}を満たすから$a'_{i}$は
  \begin{align}
    0 < a'_1 \le a'_2 \le \cdots \le a'_{k} \label{eq:8}
  \end{align}
  を満たす.さらに,\cref{eq:7}より
  \begin{align}
     & a'_1 X_1 + \sum_{i=2}^{j-1} a'_i X_i \le  a'_1 Y_1 + \sum_{i=2}^{j-1} a'_i Y_i \nonumber \\
     & \sum_{i=1}^{j-1} a'_{i}X_{i} \le \sum_{i=1}^{j-1} a'_{i}Y_{i} \label{eq:9}
  \end{align}
  である.以上\cref{eq:8,eq:9}から,数列$\{a'_i\}, \{X_i\}, \{Y_i\}$は\cref{eq:1,eq:2}を満たしている.
  これと帰納法の仮定より,$\{X_i\}, \{Y_i\}$に対して\cref{eq:3}が成り立つ.
  \begin{align*}
    \sum_{i=1}^{k} X_i \le \sum_{i=1}^{k} Y_i
  \end{align*}
  ここに\cref{eq:10}を代入して元の$x_i,y_i$の式に戻すと
  \begin{align*}
         & \left(x_2 - y_2\right) + \sum_{i=2}^{k+1} x_i
    \le \left(y_1 - x_1\right) + \sum_{i=2}^{k+1} y_i    \\
    \iff &
    \sum_{i=1}^{k+1} x_i \le \sum_{i=1}^{k+1} y_i
  \end{align*}
  となり,数列$\{x_i\}, \{x_i\}$は$n=k+1$に対して\cref{eq:3}を満たす.
  よって,$n=k+1$でも題意は成立する.

  \vspace{10pt}
  以上から,数学的帰納法により任意の$n \in \mathbb{N} \ge 2$に対し$\diamond$は成立する.
  よって題意は示された.$\cdots$(答)

  \vspace{10pt}
  {\bf[解説]}





  \newpage
\end{multicols}
\end{document}