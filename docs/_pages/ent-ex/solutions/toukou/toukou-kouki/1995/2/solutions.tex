% TODO :: 解説の方の式変形を拡充する
\documentclass[a4paper,10pt]{ltjsarticle}
\usepackage{luatexja}
\usepackage[hiragino-pron]{luatexja-preset}

\usepackage[truedimen,top=25truemm,bottom=20truemm,left=15truemm,right=15truemm]{geometry}
\setlength{\textwidth}{54\zw}
\setlength{\textheight}{73\zw}

\usepackage{amsmath,amssymb,ascmac}
\usepackage{enumerate}
\usepackage{multicol}
\usepackage{cleveref}
\usepackage{framed}
\usepackage{fancyhdr}
\usepackage{latexsym}
\usepackage{mathtools}
\usepackage{tikz}
\usepackage{pgfplots}
 \usetikzlibrary{math}
% \usepackage{indent}
\usepackage{cases}
\usepackage{float}
\usepackage{cases}
  \usepackage{caption}
  \usepackage[subrefformat=parens]{subcaption}
\allowdisplaybreaks
\pagestyle{fancy}
\lhead{}
\chead{}
\rhead{東工大後期$1995$年$2$番}
\begin{document}
\begin{oframed}
楕円 $C: \frac{x^2}{a^2} + y^2 = 1$ ($a \ge 1$) が与えられている.
    \begin{enumerate}
        \item $C$ の外部の点 $P(X,Y)$ から $C$ への $2$ 接線が直交するように $P$ を動かす.$P$ の軌跡を求めよ.
        \item $S$ を (1) で求めた $P$ の軌跡とする.$S$ と $C$ で囲まれた部分を直線 $x=2a$ を軸として,回転してできる回転体の体積を求めよ.
    \end{enumerate}
\end{oframed}
\setlength{\columnseprule}{0.4pt}
\begin{multicols}{2}
{\bf[解]}
楕円$C$と点$P$は\cref{fig:1}のように与えられる.

\begin{figure}[H]
  \centering
\begin{tikzpicture}[scale=1.3]
    % 座標軸
    \draw[->] (-2.5,0) -- (2.5,0) node[right] {$x$};
    \draw[->] (0,-1.5) -- (0,1.5) node[above] {$y$};
    
    % 楕円 x^2/3 + y^2 = 1
    \draw[thick, blue] (0,0) ellipse ({sqrt(3)} and 1);
    
    % 点P(2cos(π/4), 2sin(π/4)) = (√2, √2)
    \coordinate (P) at ({sqrt(2)}, {sqrt(2)});
    \fill[red] (P) circle (1pt);
    \node[above right] at (P) {$P$};
    
    % 接点の計算
    % 楕円 x^2/3 + y^2 = 1 上の点(√3cosφ, sinφ)における接線の方程式:
    % (√3cosφ)x/3 + (sinφ)y = 1
    % この接線が点P(√2, √2)を通る条件:
    % (√3cosφ)(√2)/3 + (sinφ)(√2) = 1
    % √6cosφ/3 + √2sinφ = 1
    % √2cosφ/√3 + √2sinφ = 1
    % cosφ/√3 + sinφ = 1/√2
    
    % 接点1: φ₁ ≈ 0.3398 rad
    % \coordinate (T1) at ({sqrt(3)*cos(19.46)}, {sin(19.46)});
    \coordinate (T1) at ({sqrt(3)*cos(7.76)}, {sin(7.76)});
    % 接点2: φ₂ ≈ 2.0344 rad  
    \coordinate (T2) at ({sqrt(3)*cos(116.57)}, {sin(116.57)});
    % 接点をマーク
    \fill[green] (T1) circle (0.8pt);
    \fill[green] (T2) circle (0.8pt);
    % \node[below right] at (T1) {$T_1$};
    % \node[above left] at (T2) {$T_2$};
    
    % 接線1
    \draw[thick, red] (P) -- (T1);
    % \draw[thick, red] (T1) -- (2.2, {(1 - sqrt(3)*cos(19.46)*(2.2)/3)/sin(19.46)});
    \draw[thick, red] (T1) -- (2.2, {(1 - sqrt(3)*cos(7.76)*(2.2)/3)/sin(7.76)});
    
    % 接線2  
    \draw[thick, red] (P) -- (T2);
    \draw[thick, red] (T2) -- (-1.5, {(1 - sqrt(3)*cos(116.57)*(-1.5)/3)/sin(116.57)});
    % 楕円の方程式を表示
    \node[below] at (0, -1.3) {$\frac{x^2}{3} + y^2 = 1$};
    
\end{tikzpicture}
\caption{楕円とPの概形}
\label{fig:1}
\end{figure}

\begin{figure}[H]
  \centering
\begin{tikzpicture}[scale=1]
    % 座標軸
    \draw[->] (-2.5,0) -- (2.5,0) node[right] {$x$};
    \draw[->] (0,-1.5) -- (0,1.5) node[above] {$y$};
    
    % 楕円 x^2/3 + y^2 = 1
    \draw[thick, blue] (0,0) ellipse ({sqrt(3)} and 1);
    
    % 点P(2cos(π/4), 2sin(π/4)) = (√2, √2)
    \coordinate (P) at ({sqrt(3)}, 1);
    \coordinate (Q) at ({sqrt(3)}, -1);
    \coordinate (R) at ({-sqrt(3)}, 1);
    \coordinate (S) at ({-sqrt(3)},-1);
    \fill[red] (P) circle (1pt);
    \fill[red] (Q) circle (1pt);
    \fill[red] (R) circle (1pt);
    \fill[red] (S) circle (1pt);
    
    % 接点1: 
    \coordinate (T1) at ({sqrt(3)}, 0);
    \coordinate (T2) at (-{sqrt(3)}, 0);
    \coordinate (T3) at (0, 1);
    \coordinate (T4) at (0,-1);

    % 接点をマーク
    \fill[green] (T1) circle (0.8pt);
    \fill[green] (T2) circle (0.8pt);
    \fill[green] (T3) circle (0.8pt);
    \fill[green] (T4) circle (0.8pt);

    % 接線
    \draw[thick, red] (P) -- (Q) -- (S) -- (R) -- cycle;    

\end{tikzpicture}
\caption{$X=\pm a$の時}
\label{fig:2}
\end{figure}

\subsection{$X\pm a$の時}

接線のうち一本が$y$軸並行となるため,もう一本の接線は$x$軸並行である必要がある.
このような条件を満たすのは$Y=\pm 1$(複号同順)である.この時\cref{fig:2}のようになる.

\subsection{$X \neq a$の時}
    
接線が$Y$軸平行でないので接線$l_1, l_2$を実数$m, m_z$を用いて
\begin{align*}
  &l_k: y = m_k(x-X)+Y & (k=1.2)
\end{align*}
と表すことが出来る.題意よりこれら二つの接線が直交する,すなわち
\begin{align}
  m_1m_2=-1 \label{eq:1}
\end{align}
となる条件をもとめる.ここで新しい座標系$(x', y')$を
\begin{align*}
  X'=\frac{x}{a}, \quad Y'=y
\end{align*}
によって定め、新しい座標系での図形に「$'$」をつけて表す.この時
\begin{align*}
  C':   & x'^2+y'^2=1 \\
  l_k': & y'=m_k(ax'-X)+Y
\end{align*}
とかけ,$C'$は円となる.$C', l_k'$が接するので,$l_k'$と$C'$の中心$(0,0)$の距離が1である.
点と直線の距離の公式から
\begin{align*}
  \frac{|-m_k X+Y|}{\sqrt{(am_k)^2+1}}=1  
\end{align*}
両辺正だから2乗して
\begin{align}
  (-m_k X+Y)^2 = (am_k)^2+1 \nonumber \\
  m_k^2 X^2 - 2m_k XY + Y^2 = a^2 m_k^2 + 1 \nonumber \\ 
  (a^2-X^2)m_k^2 + 2XY m_k + 1- Y^2 = 0 \label{eq:2}
\end{align}
\cref{eq:2}が$m_1, m_2$で成立するので、$m_1, m_2$は$m$の2次方程式
\begin{align}
  (a^2-X^2)m^2+2XY m + 1-Y^2 = 0   \label{eq:3}
\end{align}
の2つの解 ($X_k \neq a$から2次係数は$0$でない)である.判別式$D$として
\begin{align*}
  \frac{D}{4}
  &=(XY)^2-(a^2-X^2)(1-Y^2) \\
  &=X^2 + aY^2-a^2
\end{align*}
とかけるが,点$P=(X,Y)$は常に楕円$C$の外側にあるので,これは正である.
従って二次方程式\cref{eq:3}はたしかに異なる2つの実解を持つ.
この時,接線の傾きの積が$-1$になる条件(\cref{eq:1})は
\begin{align}
  &m_1m_2 = \frac{1-Y^2}{a^2-x^2}=-1 \nonumber \\
  \therefore 
  & X^2+Y^2=a^2+1  \label{eq:4}
\end{align}
と書ける.

以上二つの場合分けで全ての場合が尽くされた.
$X\pm a$の時の条件$Y=\pm 1$も\cref{eq:4}で表されることがわかる.
また,\cref{eq:4}の軌跡は楕円$C$の外側にある.
以上のことから,求める軌跡は
\begin{align*}
  X^2+Y^2=a^2+1  
\end{align*}
という半径$\sqrt{a^2+1}$の円になる.$\cdots$(答)

\vspace{10pt}
(2)

\begin{figure}[H]
\begin{tikzpicture}[scale=1]
    % 座標軸
    \draw[->] (-2.5,0) -- (4,0) node[right] {$x$};
    \draw[->] (0,-3) -- (0,3) node[above] {$y$};
    
    % 楕円 x^2/3 + y^2 = 1
    \draw[thick, blue] (0,0) ellipse ({sqrt(3)} and 1);
    \draw[thick, blue] (0,0) circle (2);
    
    \draw ({2*sqrt(3)},-3) -- ({2*sqrt(3)},3);
    
    \node[above left] at ({sqrt(3)},0) {$a$};
    \node[above right] at ({2*sqrt(3)},0) {$2a$};
    \node[above right] at (2,0) {$\sqrt{a^2+1}$};
  
    \node[above right] at (0,1) {$C$};
    \node[above right] at (0,2) {$S$};

\end{tikzpicture}
\caption{回転させる部分の概形}
\label{fig:3}
\end{figure}

$S$,$C$,および直線$l:x=2a$の関係は\cref{eq:3}のようになる.
従って,求めるべき体積$V$は,$S$内部を$l$周りに回転させた体積$V_S$から,$C$内部を$l$周りに回転させた体積$V_C$を減じたものである.
\begin{align}
  V = V_S - V_C\label{eq:9}
\end{align}
そこで以下右辺を別々に求める.$C$, $S$のうち$x\le 0$の部分を$x_{+}$, $x\ge 0$の部分を$x_{-}$と書くことにすると,
\begin{align}
  x^{S}_{\pm} &= \pm \sqrt{a^2+1-y^2} \label{eq:5} \\
  x^{C}_{\pm} &= \pm a\sqrt{1-y^2} \label{eq:6}
\end{align}
だから,対称性より$0\le y$のみ考えて
\begin{align}
  V_S 
&= 2\pi \int_{0}^{a^2+1} \{(x^{S}_{-} - 2a)^2 - (x^{S}_{+} + 2a)^2\} dy \nonumber \\
&= 2\pi \int_{0}^{a^2+1} 4a(x^{S}_{+} - x^{S}_{-}) dy \nonumber \\
&= 16a\pi \int_{0}^{a^2+1} \sqrt{(a^2+1-y)} dy \label{eq:7}
\end{align}
となる.ただし最後の行に\cref{eq:5}を用いた.同様に
\begin{align}
 V_C 
 &= 2\pi \int_{0}^{1} \{(x^{C}_{-} - 2a)^2 - (x^{C}_{+} - 2a)^2\} dy \nonumber \\ 
 &= 2\pi \int_{0}^{1} 4a(x^{C}_{+} - x^{C}_{-}) dy \nonumber \\ 
 &= 16a^2\pi \int_{0}^{1} \sqrt{1-y^2} dy \label{eq:8}
\end{align}
となる.ただし最後の行に\cref{eq:6}を用いた.
\cref{eq:7,eq:8}の積分部分は,各々半径$\sqrt{a^2+1}$,$1$の四分円の面積に等しいから
\begin{align}
  &\int_{0}^{a^2+1} \sqrt{(a^2+1-y)} dy = \frac{\pi}{4}(a^2+1) \\
  &\int_{0}^{1} \sqrt{1-y^2} dy = \frac{\pi}{4}
\end{align}
となり,\cref{eq:7,eq:8}に代入すると
\begin{align}
  V_S &= 4a\pi^2 (a^2+1) \\
  V_C &= 4a^2\pi^2 
\end{align}
となり,求めるべき体積は\cref{eq:9}に代入して
\begin{align}
  V 
  &= V_S - V_C \\
  &= 4\pi^2a (a^2-a+1)
\end{align}
である.$\cdots$(答)


\vspace{10pt}
{\bf[解説]}

楕円の性質について問う図形の問題.(1)がもとまれば(2)はほぼボーナス問題である.
(1)の性質自体は楕円の性質としてある程度知られたものであり,求めた軌跡$S$は楕円の準円と呼ばれる.
この問題はいくつかの方針で答えを求めることができる.

まず細かいところだが,本解答中で二次方程式\cref{eq:2}を出すところでは,座標変換と点と直線の距離公式を用いたが,そのまま接線が楕円と接する条件として
楕円の式に接線の式を代入した
\begin{align*}
  \frac{x^2}{a^2}+\left(m_k(x-X)+Y\right)^2 = 1
\end{align*}
が重解を持つとしても良い.式を整理すると
\begin{align*}
  \left(\frac{1}{a^2}+m_k^2\right)x^2 + 2m_k\left(-m_kX+Y\right)x +\left(-m_kX+Y\right)^2 -1 = 0
\end{align*}
となり,この判別式$D=0$が満たされれば良いから
\begin{align*}
 &m_k^2\left(-m_kX+Y\right)^2 - \left(\frac{1}{a^2}+m_k^2\right)\left[\left(-m_kX+Y\right)^2 -1\right] = 0 \\
 &\frac{1}{a^2}\left(-m_kX+Y\right)^2 -\left(\frac{1}{a^2}+m_k^2\right) = 0 \\
 &\left(\frac{X^2}{a^2}-1\right)m_k^2 - \frac{2XY}{a^2}m_k + Y^2 -\frac{1}{a^2} = 0 \\
 &\left(X^2-a^2\right)m_k^2 - 2XYm_k + a^2Y^2 - 1= 0 
\end{align*}
を得る.これは\cref{eq:2}と同じだから,以下同様に解くことができる.
計算量的には点と直線の距離を用いた方が楽ではあるがやり方としてはこちらの方が標準的と思われる.

さて,もっと別の手段として接線の方程式を異なる方法で置くこともできる.
上の解答では点$(X,Y)$を通るとして立式してから楕円と接する条件を求めに行ったが,
別の方法としては先に楕円の接線をおいて,それが点$(X,Y)$をとおるように条件を定めることもできる.
以下ではこの方法も紹介しよう.

まず前提として,楕円
\begin{align*}
  \frac{x^2}{a^2}+\frac{y^2}{b^2} = 1
\end{align*}
上の点$(x_0,y_0)$における接線は,
\begin{align*}
  \frac{x_0 x}{a^2}+\frac{y_0 y}{b^2} = 1
\end{align*}
で与えられる.これは円の場合の自然な拡張になっている.

楕円$C$上の点$Q=(\alpha,\beta)=(a\cos\theta,\sin\theta)$における接線の方程式は
\begin{align*}
  &\frac{\alpha x}{a^2} + \beta y = 1 \\
  &\frac{\cos\theta x}{a} + \sin\theta y = 1 
\end{align*}
と表せる.点$P$から引いた2本の接線が$\theta=\theta_1, \theta_2$で与えられるとする.
\begin{align*}
  & \frac{\cos\theta_{i} x}{a} + \sin\theta_{i} y = 1  & (i = 1,2)
\end{align*}
これら2本の直線が直交するので,
\begin{align}
  \sin\theta_1\sin\theta_2 + \frac{\cos\theta_1\cos\theta_2}{a^2} = 0 
\end{align}
また,これら2本の直線が点$P$を通るので
\begin{align}
  & \frac{\cos\theta_{i} X}{a} + \sin\theta_{i} Y = 1  & (i = 1,2)
\end{align}
このような条件を満たす$\theta_i$が存在する点$(X,Y)$の条件を考えれば良い.
二つ目の式から
\begin{align}
  & \sin\theta_{i} Y = 1 - \frac{\cos\theta_{i} X}{a}  & (i = 1,2)
\end{align}
だから,一つ目の式に代入すると
\begin{align}
  &\left(1 - \frac{\cos\theta_{1} X}{a} \right)\left(1 - \frac{\cos\theta_{2} X}{a} \right) +Y^2\frac{\cos\theta_1\cos\theta_2}{a^2} = 0 \\
  &\left(X^2+Y^2\right)\cos\theta_1\cos\theta_2 -aX(\cos\theta_1+\cos\theta_2) + a^2 = 0
\end{align}
同様に,$\cos\theta_{i}^2+\sin\theta_{i}^2=1$にも代入して
\begin{align}
  &\left(1 - \frac{\cos\theta_{i} X}{a} \right)^2+\left(Y\sin\theta_{i}\right) = Y^2 \\
  &\left(X^2+a^2Y^2\right)\cos^2\theta_{i} - 2aX\cos\theta + a^2\left(1-Y^2\right) = 0
\end{align}
これが$-1\le\cos\theta\le 1$に実数解を持つことが必要.これは少し計算は面倒なのだが,以下のように示される.
まず,判別式$D=4a^2Y^2(X^2+a^2Y^2-a^2)$は点$P$が楕円の外側にあるから常に$0$以上であり,実数解を持つことがわかる.
この解が$-1\le\cos\theta\le 1$にあることを示すには,解の公式から
\begin{align*}
  \cos\theta = \frac{2aX\pm 2a\sqrt{Y^2(X^2+a^2Y^2-a^2)}}{2(X^2+a^2Y^2)}
\end{align*}
と書けることを利用して
\begin{align*}
  &-1 \le  \frac{2aX\pm 2a\sqrt{Y^2(X^2+a^2Y^2-a^2)}}{2(X^2+a^2Y^2)} \le 1 \\
  &-(X^2+a^2Y^2) \le aX\pm a\sqrt{Y^2(X^2+a^2Y^2-a^2)} \le (X^2+a^2Y^2) \\
  &-(X^2+aX+a^2Y^2) \le \pm a\sqrt{Y^2(X^2+a^2Y^2-a^2)} \le (X^2-aX + a^2Y^2) \\
\end{align*}
を得る.

その上で,解と係数の関係から
\begin{align}
  &\cos\theta_1 + \cos\theta_2 = \frac{2aX}{X^2+a^2Y^2} \\
  &\cos\theta_1\cos\theta_2    = \frac{a^2\left(1-Y^2\right)}{X^2+a^2Y^2} 
\end{align}
だから,これを代入して
\begin{align*}
  &\left(X^2+Y^2\right)\frac{a^2\left(1-Y^2\right)}{X^2+a^2Y^2} -aX\frac{2aX}{X^2+a^2Y^2} + a^2 = 0 \\
  &\left(X^2+Y^2\right)\left(1-Y^2\right) -2X^2 + X^2+a^2Y^2 = 0 \\
  &Y^2\left((1+a^2) - X^2-Y^2\right) = 0 
\end{align*}
となる.$Y=0$の方は排除できるから,$X^2+Y^2=a^2+1$が求める軌跡である.


\vspace{10pt}
(2)の体積を求める問題については,解答のように$y$軸方向に積分しても,いわゆるバームクーヘン公式
\begin{align*}
  V = \int_{\alpha}^{\beta} 2\pi x |f(x)| dx
\end{align*}
を利用して$x$軸方向に積分しても,どちらでも計算の手間はそこまで変わらない.

さて,この問題はパップスギュルタンの定理で検算が効くので確認しよう.
パップスギュルタンの定理とは面積が$S$である平面図形$A$を直線$l$ の回りに回転させてできる回転体の体積$V$は,
\begin{align*}
  V=2\pi g_x S = (\text{重心の移動距離}) S
\end{align*}
と書けるというもの.ただし$g_x$は$l$と$S$の重心の距離であり,
また,制約として$A$を回転させる過程で$A$自身とは重ならないとする.

定理の証明自体は簡単で,いわゆるバームクーヘン積分の公式の簡単な言い換えに過ぎないため,
答案中で証明して利用しても問題ないだろう.

今回の場合,回転させる図形の重心は原点$(0,0)$であり,これと直線$l$の距離は$g_x =2a$である.
回転させる図形の面積は円の面積から楕円の面積を引いたもので
\begin{align*}
  \pi(a^2+1) - \pi a
\end{align*}
である.従って,求める体積は
\begin{align*}
    V 
    &= 2\pi g_x S \\
    &= 2\pi (2a) \cdot \pi \{(a^2+1)-a\} \\
    &= 4\pi^2 a(a^2-a+1) 
\end{align*}
となり,同じ回答を得た.

\newpage
\end{multicols}
\end{document}