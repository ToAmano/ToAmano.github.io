\documentclass[a4paper,10pt]{ltjsarticle}
\usepackage{luatexja}
\usepackage[hiragino-pron]{luatexja-preset}

\usepackage[truedimen,top=25truemm,bottom=20truemm,left=15truemm,right=15truemm]{geometry}
\setlength{\textwidth}{54\zw}
\setlength{\textheight}{73\zw}

\usepackage{amsmath,amssymb,ascmac}
\usepackage{enumerate}
\usepackage{multicol}
\usepackage{physics}
\usepackage{cleveref}
\usepackage{framed}
\usepackage{fancyhdr}
\usepackage{latexsym}
\usepackage{mathtools}
\usepackage{tikz}
\usepackage{tikz-3dplot}
\usepackage{pgfplots}
\usetikzlibrary{math}
\usetikzlibrary{calc}
\usetikzlibrary{angles}
\usetikzlibrary{quotes}
% \usepackage{indent}
\usepackage{cases}
\usepackage{float}
\allowdisplaybreaks
\pagestyle{fancy}
\lhead{}
\chead{}
\rhead{東工大後期$1995$年$1$番}
\begin{document}
\begin{oframed}
    一辺の長さが $2$ の立方体 $C$ がある.
    $S_0$ を $C$ の $6$ つの面に内接する球とする.
    次に $S_0$ に外接し,$C$ の $3$ つの面と内接する球 $S_1$ を取る.
    $S_1$ に外接し,$C$ の $3$ つの面に内接する球 $S_2$ を $S_1$ の外側($S_0$ と反対側)に取る.
    以下帰納的に,$S_0, \ldots, S_n$ まで取れたとして,$S_n$ に外接し,$C$ の $3$ つの面に内接する球 $S_{n+1}$ を $S_n$ の外側に取る.
    \begin{enumerate}
        \item $S_n$ の半径を $n$ の式で表せ.
        \item 立方体 $C$ の中でどの $S_n$ ($n=0,1,2,\ldots$) にも含まれない部分の体積を求めよ.
    \end{enumerate}
\end{oframed}
\setlength{\columnseprule}{0.4pt}
\begin{multicols}{2}
    {\bf[解]}

    (1)
    \cref{fig:1}のように,立方体の頂点A, B, C, D, E, F, G, Hに対し題意の3面を
    ABCD, AEFB, AEHDとする. 各球の中心は
    立方体及び球の対称性から対角線AG上にある.
    $S_n$の中心を$O_n$,半径を$r_n$とおく.
    断面AEGCを\cref{fig:2}に示す.$AC=2\sqrt{2}$,$AE=2$より,$\angle\mathrm{GAC}=|theta$と置くと
    \begin{align}
        \sin\theta & = \frac{\mathrm{AE}}{\mathrm{AG}} = \frac{1}{\sqrt{3}} \label{eq:1}        \\
        \cos\theta & = \frac{\mathrm{AC}}{\mathrm{AG}} = \frac{\sqrt{2}}{\sqrt{3}} \label{eq:2}
    \end{align}
    が成り立つことに注意する.

    \begin{figure}[H]
        \centering
        \tdplotsetmaincoords{70}{110}
        \begin{tikzpicture}[scale=2, tdplot_main_coords]
            \coordinate (D) at (0,0,0);
            \coordinate (A) at (1,0,0);
            \coordinate (C) at (0,1,0);
            \coordinate (B) at (1,1,0);
            \coordinate (H) at (0,0,1);
            \coordinate (E) at (1,0,1);
            \coordinate (G) at (0,1,1);
            \coordinate (F) at (1,1,1);

            \draw[dashed] (D) -- (A);
            \draw[dashed] (D) -- (C);
            \draw[dashed] (D) -- (H);
            \draw (A) -- (B) -- (C) -- (G) -- (F) -- (E) -- (A);
            \draw (C) -- (B);
            \draw (H) -- (E);
            \draw (H) -- (G);
            \draw (B) -- (F);


            \node[below] at (D) {D};
            \node[below] at (A) {A};
            \node[below] at (B) {B};
            \node[below] at (C) {C};
            \node[above] at (H) {H};
            \node[above] at (E) {E};
            \node[above] at (G) {G};
            \node[above] at (F) {F};
        \end{tikzpicture}
        \caption{立方体と頂点の定義}
        \label{fig:1}
    \end{figure}

    \begin{figure}[H]
        \centering
        \begin{tikzpicture}[scale=2]
            % First circle (radius rn)
            \coordinate (E) at (0, 0); % Center of the first circle
            \coordinate (G) at ({2*sqrt(2)}, 0); % Center of the first circle
            \coordinate (C) at ({2*sqrt(2)}, 2); % Center of the first circle
            \coordinate (A) at (0, 2); % Center of the first circle
            \coordinate (On) at ({sqrt(2)}, 1); % Center of the first circle
            \coordinate (O1) at ({2*sqrt(2)-sqrt(6)}, {sqrt(3)});
            \draw (A) -- (C) -- (G) -- (E) -- cycle;
            \draw (A) -- (G);
            \fill (On) circle [radius=0.5pt];
            \fill (O1) circle [radius=0.5pt];


            \draw (On) circle (1); % Radius 1.5cm for visualization
            \draw (O1) circle ({2-sqrt(3)}); % Radius 1.5cm for visualization
            \node[above] at ({1.5*sqrt(2)-0.5*sqrt(6)},{0.5+sqrt(3)*0.5}) {$r_0$}; % Label for radius
            \node[below] at (On) {$O_0$};
            \node[below] at (O1) {$O_1$};
            \node[left]  at (E) {E};
            \node[left]  at (A) {A};
            \node[right]  at (C) {C};
            \node[right]  at (G) {G};


            \draw[dashed] (On) -- ({sqrt(2)},2);
            \draw[dashed] (O1) -- ({2*sqrt(2)-sqrt(6)},2);
            \draw[dashed] (O1) -- ({sqrt(2)},{sqrt(3)});

            \node[left] at (0,1) {$2$};
            \node[above] at ({sqrt(2)},2) {$2\sqrt{2}$};

            \path pic["$\theta$",draw,angle radius=7mm,angle eccentricity=1.3] {angle = G--A--C};

        \end{tikzpicture}
        \caption{断面AEGC}
        \label{fig:2}
    \end{figure}

    半径$r_n$に関する漸化式を導出することで$r_n$の一般項を求める.
    円$S_n$と$S_{n+1}$に着目して\cref{fig:3}を考える.

    \begin{figure}[H]
        \centering
        \begin{tikzpicture}[scale=2]
            % First circle (radius rn)
            \coordinate (G) at ({2*sqrt(2)}, 0); % Center of the first circle
            \coordinate (C) at ({2*sqrt(2)}, 2); % Center of the first circle
            \coordinate (A) at (0, 2); % Center of the first circle
            \coordinate (On) at ({sqrt(2)}, 1); % Center of the first circle
            \coordinate (O1) at ({2*sqrt(2)-sqrt(6)}, {sqrt(3)});
            \draw (A) -- (C);
            \draw (A) -- (G);
            \fill (On) circle [radius=0.5pt];
            \fill (O1) circle [radius=0.5pt];

            \draw (On) circle (1); % Radius 1.5cm for visualization
            \draw (O1) circle ({2-sqrt(3)}); % Radius 1.5cm for visualization
            \node[above] at ({1.5*sqrt(2)-0.5*sqrt(6)},{0.5+sqrt(3)*0.5}) {$r_0$}; % Label for radius
            \node[below] at (On) {$O_n$};
            \node[below] at (O1) {$O_{n+1}$};
            \node[left]  at (A) {A};
            \node[right]  at (C) {C};

            \draw[dashed] (On) -- ({sqrt(2)},2);
            \draw[dashed] (O1) -- ({2*sqrt(2)-sqrt(6)},2);
            \draw[dashed] (O1) -- ({sqrt(2)},{sqrt(3)});

            \node[above] at ({sqrt(2)},2) {$T_n$};
            \node[above] at ({2*sqrt(2)-sqrt(6)},2) {$T_{n+1}$};
            \node[right] at ({sqrt(2)},{sqrt(3)}) {$R_{n}$};

            \path pic["$\theta$",draw,angle radius=7mm,angle eccentricity=1.3] {angle = G--A--C};

        \end{tikzpicture}
        \caption{$S_n$と$S_{n+1}$の関係}
        \label{fig:3}
    \end{figure}

    $O_n$から$AC$に引いた垂線と$AC$との交点を$T_n$と置くと,その定義より$O_nT_n$の長さは$r_n$に等しい.
    一方で,$O_{n+1}$から$O_nT_n$に引いた垂線と$O_nT_n$の交点を$R_n$と置くと,
    \begin{align*}
        O_nT_n
         & =O_{n+1}T_{n+1}+O_nR_n                        \\
         & =O_{n+1}+O_nO_{n+1}\sin\theta                 \\
         & =r_{n+1}+\left(r_{n}+r_{n+1}\sin\theta\right)
    \end{align*}
    と表されるので,$r_n$と$r_{n+1}$の関係は
    \begin{align}
        r_{n}
                & =r_{n+1}+(r_{n}+r_{n+1}\sin\theta)        \\
        \therefore
        r_{n+1} & = \frac{1-\sin\theta}{1+\sin\theta} r_{n}
    \end{align}
    となる.$r_0=1$と合わせると,この等比級数の解は
    \begin{align}
        r_{n}
         & = \left(\frac{1-\sin\theta}{1+\sin\theta}\right)^n \\
         & = (2-\sqrt{3})^n
    \end{align}
    となる.ただし,\cref{eq:1}を用いた.$\cdots$(答)

    \vspace{10pt}
    (2)
    立方体 $C$ の中でどの $S_k$ ($k=0,1,\ldots, n$) にも含まれない部分の体積を$V_n$とする.
    求めるべき値は$\displaystyle V = \lim_{n\to\infty}V_n$である.
    $S_k\, (k=0,1,\cdots,n)$同士は互いに体積を共有することはないから,
    体積$V_n$は立方体$C$の体積から,$S_k\, (k=0,1,\cdots,n)$の体積を減じたものに等しく,
    \begin{align*}
        V_n & = 8 - \frac{4}{3}\pi \sum_{k=0}^{n} r_k^3                             \\
            & = 8 - \frac{4}{3}\pi \sum_{k=0}^{n} (2-\sqrt{3})^{3k}                 \\
            & = 8 - \frac{4}{3}\pi \frac{1-(2-\sqrt{3})^{3(n+1)}}{1-(2-\sqrt{3})^3}
    \end{align*}
    となる.

    $(2-\sqrt{3})^3 < 1$ だから求める体積$V$は
    \begin{align*}
        V & = \lim_{n \to \infty} V_n                       \\
          & = 8 - \frac{4}{3}\pi \frac{1}{1-(2-\sqrt{3})^3} \\
          & = 8 - \frac{6\sqrt{3}+10}{15}\pi
    \end{align*}
    である.$\cdots$(答)

    {\bf[解説]}

    \newpage
\end{multicols}
\end{document}