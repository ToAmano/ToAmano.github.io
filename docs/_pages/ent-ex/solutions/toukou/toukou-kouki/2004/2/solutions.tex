% TODO :: 図1の更新
\documentclass[a4paper,10pt]{ltjsarticle}
\usepackage{luatexja}
\usepackage[hiragino-pron]{luatexja-preset}

\usepackage[truedimen,top=25truemm,bottom=20truemm,left=15truemm,right=15truemm]{geometry}
\setlength{\textwidth}{54\zw}
\setlength{\textheight}{73\zw}

\usepackage{amsmath,amssymb,ascmac}
\usepackage{enumerate}
\usepackage{multicol}
\usepackage{cleveref}
\usepackage{framed}
\usepackage{fancyhdr}
\usepackage{latexsym}
\usepackage{mathtools}
\usepackage{tikz}
\usepackage{tikz-3dplot}
\usepackage{pgfplots}
 \usetikzlibrary{math}
 \usetikzlibrary{calc}
 \usetikzlibrary{angles}
 \usetikzlibrary{quotes}
 \pgfplotsset{compat=1.18}
% \usepackage{indent}
\usepackage{caption}
\usepackage{subcaption}
\usepackage{cases}
\usepackage{float}
\usepackage{cases}
  \usepackage{caption}
  \usepackage[subrefformat=parens]{subcaption}
\allowdisplaybreaks
\tdplotsetmaincoords{70}{110}
\pagestyle{fancy}
\lhead{}
\chead{}
\rhead{東工大後期$2004$年$2$番}
\begin{document}
\begin{oframed}
  $n$を2以上の偶数とする.2つの曲線$C_1: y=x^n$と$C_2: y=n^x$について,次の問いに答えよ.

  \begin{enumerate}
    \item $C_1$と$C_2$は$x<0$において,ただ1つの点$P_n$で交わることを示せ.
    \item $C_1$と$C_2$の交点の個数を求めよ.
    \item $P_n$の$n \to \infty$のときの極限の位置を求めよ.
  \end{enumerate}
\end{oframed}


\setlength{\columnseprule}{0.4pt}
\begin{multicols}{2}
  {\bf[解]}
  (1)
  $n$ は偶数とし, $f_n(x) = x^n$, $g_n(x) = n^x$ とおく.

  (1) $x < 0$ の時. $t=-x$とおくと,$t>0$であり,
  $f_n(x)$ と $g_n(x)$ が一致するとすると
  \begin{align*}
    f_n(x) & = g_n(x)                                  \\
    \iff
    t^n    & = \left(\frac{1}{n}\right)^t \quad (\ast)
  \end{align*}
  両辺正だから, 自然対数をとって
  \begin{align}
    n \log t         & = t\log n  \nonumber                                                       \\
    \iff
    \frac{\log t}{t} & = \frac{\log n}{n} \quad (\text{ただし } x > 0) \quad (\ast\ast) \label{eq:1}
  \end{align}
  である.
  ここで $h(t) = \frac{\log t}{t}$ とおくと, 一階微分は
  \begin{align*}
    h'(t) = \frac{1 - \log t}{t^2}
  \end{align*}
  であり,また$h(t)$の極限値は
  \begin{align*}
    h(t) & \to -\infty \quad (t \to +0) \\
    h(t) & \to 0 \quad (t \to +\infty)
  \end{align*}
  で与えられるから,$h(t)$の増減表は\cref{table:1}となる.

  \begin{table}[H]
    \centering
    \caption{$h(x)$の増減表}
    \label{table:1}
    \begin{tabular}{|c|c|c|c|c|c|}
      \hline
      $t$  & $(0)$       & $\cdots$   & $e$           & $\cdots$   & $(\infty)$ \\
      \hline
      $h'$ &             & $+$        & $0$           & $-$        &            \\
      \hline
      $h$  & $(-\infty)$ & $\nearrow$ & $\frac{1}{e}$ & $\searrow$ & $(0)$      \\
      \hline
    \end{tabular}
  \end{table}

  よって$h(t)$のグラフは\cref{fig:1}となる.

  \begin{figure}[H]
    %\centering
    \begin{tikzpicture}
      \begin{axis}[
          axis lines=middle,
          xmin=0, xmax=6,
          ymin=-2, ymax=1,
          xlabel=$t$,
          ylabel=$y$,
          xtick={1,e},
          xticklabels={$1$,$e$},
          ytick={0.367879, -0.5}, % 1/e の近似値と -log n / n の例
          yticklabels={$\frac{1}{e}$, $-\frac{\log n}{n}$},
          grid=none,
          %no markers,
          clip=false, % ラベルが軸からはみ出しても表示されるように
        ]

        % h(x) = (ln x) / x のプロット
        \addplot[blue, thick, domain=0.5:5, samples=100,smooth] {ln(x)/x};

        % h(t)ラベル
        \node[above right] at (axis cs:e, {ln(e)/e}) {$h(t)$};

        % 極大点とy軸のラベルを結ぶ破線
        \draw[dashed] (axis cs:e, {ln(e)/e}) -- (axis cs:0, {ln(e)/e});
        \draw[dashed] (axis cs:e, {ln(e)/e}) -- (axis cs:e, 0);

        % -log n / n の水平線(例として)
        \draw[red, dashed] (axis cs:0.01, -0.5) -- (axis cs:6, -0.5);
      \end{axis}
    \end{tikzpicture}
    \caption{$h(t)$の概形.$t=e$で極大値をとる.}
    \label{fig:1}
  \end{figure}

  ここで $n$ が$2$以上の偶数であるから$h(x)<0$であり,
  グラフの形から\cref{eq:1}が成立する$t$が$0<t<1$にただひとつ存在する.
  したがって $C_1, C_2$ は$x<0$にただ 1 つ交点を持つ.$\cdots$(答)

  \vspace{10pt}
  (2)
  $x>0$ の時. $f_n(x), g_n(x)$ 共に正だから, (1) と同様に自然対数をとって考えると
  \begin{align*}
    f_n(x)           & = g_n(x)           \\
    \iff
    n \log x         & = x \log n         \\
    \iff
    \frac{\log x}{x} & = \frac{\log n}{n}
  \end{align*}
  である.
  $n$が$2$以上の偶数だから$\frac{\log n}{n} > 0$ であり,
  また\cref{fig:1}より$\frac{\log n}{n}< 1/e$である.
  従って\cref{fig:1}からこれみたす $x$ は 2 つある.

  最後に$x=0$の時は $f_n(0) = 0$, $g_n(0) = 1$ だから, $C_1, C_2$ は交わらない.

  以上で$x$について全ての場合が考えられた.$x<0$で一つ,$x=0$で0個,$x>0$で2つの解が存在する.
  $C_1, C_2$ の交点の数は $f_n(x) = g_n(x)$ の実解の数に等しいことから
  あわせて $3$ つの交点がある.$\cdots$(答)

  \vspace{10pt}
  (3) $P_n(-x_n, y_n)$ とおくと,(1)の結果から $0 < x_n < 1$ である.
  $P_n$ の条件から,
  \begin{align}
    y_n = (-x_n)^n = n^{-x_n} \label{eq:2}
  \end{align}
  である.

  $n\to\infty$のとき,\cref{fig:1}から $\frac{\log n}{n} \to 0$ だから,
  \cref{eq:1}より
  \begin{align*}
    \lim_{n\to\infty} \frac{\log x_n}{x_n} = 0
  \end{align*}
  である. これを満たすには\cref{fig:1}および$0<x_n<1$から
  \begin{align}
    \lim_{n\to\infty} x_n = 1 \label{eq:3}
  \end{align}
  である.これを\cref{eq:2}に代入して
  \begin{align}
    \lim_{n\to\infty} y_n = \lim_{n\to\infty} \left(\frac{1}{n}\right)^{x_n} = 0 \label{eq:4}
  \end{align}
  である.\cref{eq:3,eq:4}より,求める$P_n$の極限値は
  \begin{align*}
    \lim_{n\to\infty}P_n = (-1,0)
  \end{align*}
  である.

  \newpage
\end{multicols}
\end{document}