\documentclass[a4paper,10pt]{ltjsarticle}
\usepackage{luatexja}
\usepackage[hiragino-pron]{luatexja-preset}

\usepackage[truedimen,top=25truemm,bottom=20truemm,left=15truemm,right=15truemm]{geometry}
\setlength{\textwidth}{54\zw}
\setlength{\textheight}{73\zw}

\usepackage{amsmath,amssymb,ascmac}
\usepackage{enumerate}
\usepackage{multicol}
\usepackage{cleveref}
\usepackage{framed}
\usepackage{fancyhdr}
\usepackage{latexsym}
\usepackage{mathtools}
\usepackage{tikz}
\usepackage{pgfplots}
 \usetikzlibrary{math}
% \usepackage{indent}
\usepackage{cases}
\usepackage{float}
\usepackage{cases}
  \usepackage{caption}
  \usepackage[subrefformat=parens]{subcaption}
\allowdisplaybreaks
\pagestyle{fancy}
\lhead{}
\chead{}
\rhead{東工大後期$1994$年$2$番}
\begin{document}
\begin{oframed}
自然数 $n=1,2,3,\cdots\cdots$ に対して,$(2-\sqrt{3})^n$ という形の数を考える.これらの数
はいずれも,それぞれ適当な自然数 $m$ が存在して $\sqrt{m}-\sqrt{m-1}$ という表示をもつこと
を示せ.
\end{oframed}
\setlength{\columnseprule}{0.4pt}
\begin{multicols}{2}
{\bf[解]}

$a_n = (2-\sqrt{3})^n$ とおく。二項係数が整数であるから,
$A_n, B_n$を自然数として
\begin{align*} 
  a_n 
  =& (2-\sqrt{3})^n \\
  =& (2^n + {}_n\mathrm{C}_2 \cdot 2^{n-2} \cdot 3 + \dots) \\
   & -\sqrt{3}({}_n\mathrm{C}_1 \cdot 2^{n-1} + {}_n\mathrm{C}_3 \cdot 2^{n-3} \cdot 3 + \dots) \\ 
  =& A_n - \sqrt{3}B_n 
\end{align*}
と表せる.$A_n$および$B_n$のみたす漸化式を求めるため,$a_{n+1}=(2-\sqrt{3})a_{n}$を$A$および$B$で表すと
\begin{align*} 
  A_{n+1} - \sqrt{3}B_{n+1} 
  &= (2-\sqrt{3})(A_n - \sqrt{3}B_n) \\ 
  &= 2A_n + 3B_n - \sqrt{3}(A_n + 2B_n) \\
\therefore &
\begin{cases} 
  A_{n+1} = 2A_n+ 3B_n \\ 
  B_{n+1} = A_n + 2B_n 
\end{cases} \quad \text{①}
\end{align*}
となる.初期条件は$n=1$のとき
\begin{align}\label{eq:2}
  &A_{1} = 2 & B_{1} = 1
\end{align}
となる.

以下, 条件式
\begin{align}\label{eq:3}
A_n^2 = 3B_n^2 + 1
\end{align} 
が任意の$n \in \mathbb{N}$で成り立つことを数学的帰納法で示す.この命題が示されれば$m=A_{n}^2$とおくことで$a_{n}=\sqrt{m}+\sqrt{m-1}$と表すことができる.

$n=1$の時の成立は\cref{eq:2}より明らかだから, $n=k \in \mathbb{N}$での\cref{eq:3}の成立を仮定し, $n=k+1$での成立を示す。漸化式\cref{eq:1}より
\begin{align*} 
  A_{k+1}^2 
  &= (2A_k + 3B_k)^2 \\ 
  &= 4A_k^2 + 9B_k^2 + 12A_kB_k \\ 
  &= 4(3B_k^2 + 1) + 9B_k^2 + 12A_kB_k \quad (\text{∵仮定}) \\ 
  &= 21B_k^2 + 4 + 12A_kB_k 
\end{align*}
および
\begin{align*} 
  3B_{k+1}^2 + 1 
  &= 3(A_k + 2B_k)^2 + 1 \\ 
  &= 3(A_k^2 + 4A_kB_k + 4B_k^2) + 1 \\ 
  &= 3A_k^2 + 12A_kB_k + 12B_k^2 + 1 \\ 
  &= 3(3B_k^2 + 1) + 12A_kB_k + 12B_k^2 + 1 \quad (\text{∵仮定}) \\ 
  &= 21B_k^2 + 4 + 12A_kB_k 
\end{align*}
だから,$A_{k+1}^2 = 3B_{k+1}^2 + 1$ が成り立つ.
よって$n=k+1$でも\cref{eq:3}は成立.

以上から\cref{eq:3}は示された.したがって
$m=A_n^2$ とおくと $3B_n^2 = m-1$となり,
\begin{align*}
  a_n 
  &= A_n - \sqrt{3} B_n \\
  &= \sqrt{m} - \sqrt{m-1}
\end{align*}
を得る.以上で題意は示された.$\cdots$(答)

\vspace{10pt}
{\bf[解説]}

いわゆる二次体の問題である.二次体とは,平方数でない整数$d$を用いて
\begin{align*}
  q = A + B\sqrt{d}
\end{align*}
と表されるような数の集合である.ここで$A,B \in \mathbb{Q}$である.

$a_n$とペアとなる$(2-\sqrt{3})^n$を同時に考えることでもっと綺麗に解けるので別解として示す.
\begin{align*}
  b_n = (2+\sqrt{3})^n
\end{align*}
と定義すると,$a_n$と同様に
\begin{align*}
  b_n = A_n + \sqrt{3} B_n
\end{align*}
とかける.$a_n$と$b_n$の積を計算すると
\begin{align*}
  a_nb_n 
  &= (2+\sqrt{3})^n(2-\sqrt{3})^n \\
  &= \left(4-3\right)^n \\
  &= 1
\end{align*}
である.一方この積は$A_n$と$B_n$を使うと
\begin{align*}
  a_nb_n 
  &= (A_n+\sqrt{3}B_n)(A_n-\sqrt{3}B_n) \\
  &= A_n^2 -3B_n2 
\end{align*}
と書ける.従って
\begin{align*}
  A_n^2 = 3B_n^2 +1
\end{align*}
をえる.これは\cref{eq:3}であり,以下容易に題意が示される.

\newpage
\end{multicols}
\end{document}