\documentclass[a4paper,10pt]{ltjsarticle}
\usepackage{luatexja}
\usepackage[hiragino-pron]{luatexja-preset}

\usepackage[truedimen,top=25truemm,bottom=20truemm,left=15truemm,right=15truemm]{geometry}
\setlength{\textwidth}{54\zw}
\setlength{\textheight}{73\zw}

\usepackage{amsmath,amssymb,ascmac}
\usepackage{enumerate}
\usepackage{multicol}
\usepackage{physics}
\usepackage{cleveref}
\usepackage{framed}
\usepackage{fancyhdr}
\usepackage{latexsym}
\usepackage{mathtools}
\usepackage{tikz}
\usepackage{pgfplots}
 \usetikzlibrary{math}
% \usepackage{indent}
\usepackage{cases}
\usepackage{float}
\allowdisplaybreaks
\pagestyle{fancy}
\lhead{}
\chead{}
\rhead{東工大後期$1992$年$1$番}
\begin{document}
\begin{oframed}
関数 $f(x)$ に対し
\[ F(x) = \int_0^x f(t)dt \]
とおく.ある定数 $a, b, c$ が存在して
\[ F(x) = x^2 + ax|x-b| + cx \]
が常に成立し,さらに $3$ つの条件
\begin{enumerate}
    \item[(i)] $f(x)$ は連続
    \item[(ii)] $F(1) = 0$
    \item[(iii)] $f(0) = 1$
\end{enumerate}
が満たされているとする.このとき $f(x)$ を求めよ.
\end{oframed}
\setlength{\columnseprule}{0.4pt}
\begin{multicols}{2}
{\bf[解]}
題意の2つ目の条件により,$F(x)$ は以下のように定義される:
\begin{align}\label{eq:1}
F(x) = 
\begin{dcases}
(1+a)x^2 + (c-ab)x & (x \ge b)   \\
(1-a)x^2 + (c+ab)x & (x \le b)  
\end{dcases}
\end{align}
\cref{eq:1}から $F(x)$ は $x\neq b$ で微分可能である.そこで,題意の一つ目の条件式を両辺$x$で微分することで
\begin{align}
F'(x) = f(x) \quad (x \ne b)
\end{align}
がなりたつ.\cref{eq:1}の両辺を$x$で微分して,
\begin{align}\label{eq:2}
f(x) = 
\begin{dcases}
2(1+a)x + (c-ab) & (x > b) \\
2(1-a)x + (c+ab) & (x < b)
\end{dcases}
\end{align}
を得る.したがって$f(x)$は$x\neq b$で連続である.

さて,題意の条件(i)より,$f(x)$は$x=b$でも連続でなければならない.そのためには$\displaystyle \lim_{x\to b} f(x)$の両側極限が同じ値に収束すればよい.
\begin{align*}
\lim_{x \to b+0} f(x) &= 2(a+1)b + (c-ab) \\
\lim_{x \to b-0} f(x) &= 2(1-a)b + (c+ab)
\end{align*}
これらの左右極限がひとしいためには
\begin{align*}
2(1+a)b + (c-ab) &= 2(1-a)b + (c+ab) \\
\therefore 2ab &= 0 \\
\therefore ab &= 0 
\end{align*}
でありこの時$a=0$ または $b=0$である.また,
\begin{align}
f(b) = 2b+c
\end{align}
となる.以下場合分けして残りの条件(ii),(iii)から定数$a$,$b$,$c$を決定することで$f(x)$を求める.

\vspace{1em} % Add some vertical space
(a) $a=0$ の時

\cref{eq:1}から$F(x)=x^2 + cx$となる.(ii)から$F(1)=0$すなわち$c=-1$である.
したがって\cref{eq:2}から$f(x)=2x-1$だが,これは(iii)の$f(0)=1$に反して矛盾.

\vspace{1em} % Add some vertical space
(b) $b=0$ の時

\cref{eq:1,eq:2}から,$F(x)$および$f(x)$は
\begin{align*}
F(x) = \begin{cases}
    (1+a)x^2 + cx & (x \ge 0) \\
    (1-a)x^2 + cx & (x \le 0)
    \end{cases} \\
f(x) = \begin{cases}
    2(1+a)x + c & (x \ge 0) \\
    2(1-a)x + c & (x \le 0)
    \end{cases}
\end{align*}
と表せる.(ii)(iii)を代入して,
\begin{align*}
    \begin{cases}
    a+1+c = 0 \\
    c=1
    \end{cases} \\
    \therefore a = -2, \ c=1
\end{align*}
を得る.以上からもとめる$f(x)$は
\begin{align*}
f(x) = \begin{cases}
-2x+1 & (x \ge 0) \\
6x+1 & (x \le 0)
\end{cases}
\end{align*}
である.この関数$f(x)$は題意のすべての条件を満たす.$\cdots$(答)

{\bf[解説]}
簡単な多変数関数の問題.5つの条件を条件をひとつずつ処理していけば容易に答えにたどり着ける.
$f$の原始関数$F$が二次関数なので$f$は一次関数となり,検算で答えがもっともらしいことを確認できる.

     \newpage
\end{multicols}
\end{document}