\documentclass[a4paper,10pt]{ltjsarticle}
\usepackage{luatexja}
\usepackage[hiragino-pron]{luatexja-preset}

\usepackage[truedimen,top=25truemm,bottom=20truemm,left=15truemm,right=15truemm]{geometry}
\setlength{\textwidth}{54\zw}
\setlength{\textheight}{73\zw}

\usepackage{amsmath,amssymb,ascmac}
\usepackage{enumerate}
\usepackage{multicol}
\usepackage{cleveref}
\usepackage{framed}
\usepackage{fancyhdr}
\usepackage{latexsym}
\usepackage{mathtools}
\usepackage{tikz}
\usepackage{tikz-3dplot}
\usepackage{pgfplots}
 \usetikzlibrary{math}
 \usetikzlibrary{calc}
 \usetikzlibrary{angles}
 \usetikzlibrary{quotes}
 \usetikzlibrary{patterns}
 \usetikzlibrary{intersections}
 \usepgfplotslibrary{fillbetween}
 \pgfplotsset{compat=1.18}
% \usepackage{indent}
\usepackage{caption}
\usepackage{subcaption}
\usepackage{cases}
\usepackage{float}
\usepackage{cases}
  \usepackage{caption}
  \usepackage[subrefformat=parens]{subcaption}
\allowdisplaybreaks
\tdplotsetmaincoords{70}{110}
\pagestyle{fancy}
\lhead{}
\chead{}
\rhead{東工大後期$2011$年$2$番}
\begin{document}
\begin{oframed}
  次の式 $x = \tan\theta$,$y = \frac{1}{\cos\theta}$ ($0 \le \theta < \frac{\pi}{2}$) で表される $xy$ 平面上の曲線 $C$ を考え
  る.定数 $t>0$ に対し点 P$(t,0)$ を通り $x$ 軸に垂直な直線 $l$ と曲線 $C$ の交点を Q とする.
  曲線 $C$,$x$ 軸,$y$ 軸,および直線 $l$ で囲まれた図形の面積を $S_1$ とし,$\triangle$OPQ の面積を $S_2$ とする.
  \begin{enumerate}
    \item $S_1$, $S_2$ を $t$ を用いて表せ.
    \item 極限 $\displaystyle\lim_{t \to \infty} \frac{S_1 - S_2}{\log t}$ を求めよ.
  \end{enumerate}
\end{oframed}

\setlength{\columnseprule}{0.4pt}
\begin{multicols}{2}
  {\bf[解]}
  先に曲線$C$の概形を求める.
  $0 < \theta < \pi/2$ で $x=\tan\theta$ は非負だから両辺二乗しても同値であって,
  \begin{align*}
    x^2
     & = \frac{1}{\tan^2 \theta}             \\
     & = \frac{1-\cos\theta^2}{\cos\theta^2} \\
     & = \frac{1}{\cos\theta^2}-1            \\
     & = y^2-1
  \end{align*}
  となり,$C$は双曲線である.
  $0 < \theta < \pi/2$ から, $0 < x, 1 \leq y$ で,
  求める$C$の概形は\cref{fig:1}である.

  \begin{figure}[H]
    \begin{tikzpicture}
      \begin{axis}[
          axis lines=middle,
          xlabel=$x$,
          ylabel=$y$,
          xmin=0, xmax=3, % Adjust xmax as needed to show the curve well
          ymin=0, ymax=3, % Adjust ymax as needed
          ytick={1}, % Custom y-tick for '1'
          xtick={2},
          xticklabels={$t$},
          tick label style={font=\footnotesize},
          xlabel style={at={(axis description cs:1,0.05)},anchor=north west},
          ylabel style={at={(axis description cs:0.05,1)},anchor=south east},
          grid=none,
          clip=false, % Allow drawing outside the axis bounding box for labels like C
        ]

        % Define a value for t for demonstration. In a real problem, t would be variable.
        \pgfmathsetmacro{\tval}{2} % Example value for t
        \node[above] at (axis cs:\tval, 2.236) {Q}; % sqrt(5)
        \node[above right ] at (axis cs:\tval, 0) {P};

        % Plot the function y=sqrt(x^2+1)
        \addplot[blue, thick, samples=100, domain=0:3] {sqrt(x^2+1)}; %  coordinate[pos=1] (Q);

        % Fill the area S1
        \addplot[fill=gray!20, draw=none, domain=0:\tval] {sqrt(x^2+1)} \closedcycle;

        % Draw the line x=t
        \draw[dashed] (\tval,0) -- (\tval,2.236);
        \draw[dashed] (0,0) -- (\tval,2.236);

        % % Asymptote y=x
        \draw[thick,dotted] (axis cs:0,0) -- (axis cs:3,3) node[above right] {$y=x$};

        % % Label C near the curve
        \node at (axis cs:2.5, 3.0) {C};

        % % Label S1
        \node[fill=white] at (axis cs:1,0.5) {$S_1$}; % Adjust position as needed

        % % Add O for origin
        \node at (axis cs:0,0) [below left] {O};

      \end{axis}
    \end{tikzpicture}
    \caption{$C$の概形と点$P$, $Q$の関係}
    \label{fig:1}
  \end{figure}

  \vspace{10pt}
  (1)
  $S_1$は$y=\sqrt{x^2+1}$を$0\le x\le t$で積分したものであり,
  \begin{align*}
    S_1
     & = \int_{0}^{t} \sqrt{x^2+1} \, dx                                                \\
     & = \left[\frac{x}{2}\sqrt{x^2+1} + \frac{1}{2}\log(x+\sqrt{x^2+1})\right]_{0}^{t} \\
     & = \frac{1}{2}\left[t\sqrt{t^2+1} + \log(t+\sqrt{t^2+1})\right]                   \\
  \end{align*}
  である. $\cdots$(答)

  次に$S_2$は底辺$t$,高さ$\sqrt{t^2+1}$の三角形の面積であり,
  \begin{align*}
    S_2 = \frac{1}{2}t\sqrt{t^2+1}
  \end{align*}
  である. $\cdots$(答)

  \vspace{10pt}
  (2)
  (1)の結果から,与式は
  \begin{align*}
    \frac{S_1 - S_2}{\log t}
     & = \frac{1}{2}\frac{\log(t+\sqrt{t^2+1})}{\log t}                   \\
     & = \frac{1}{2}\frac{\log t + \log(1+\sqrt{1+1/t^2})}{\log t}        \\
     & = \frac{1}{2}\left(1+\frac{\log (1+\sqrt{1+1/t^2})}{\log t}\right) \\
     & =\to \frac{1}{2} \quad (t \to \infty)
  \end{align*}
  となり,求めるべき極限値は$1/2$である. $\cdots$(答)


  \vspace{10pt}
  {\bf[解説]}
  平面図形の問題.素直に条件を置いていけば良く,計算負荷も軽いので
  かなり簡単な問題だろう.

  (2)は要するに,$C$の漸近線である$y=x$と曲線$C$に囲まれた部分の面積が
  どのように振る舞うかというのを示す問題である.この様子を\cref{fig:2}に示す.
  この問題から,ここの部分の面積が対数で発散していくということがわかる.

  \begin{figure}[H]
    \begin{tikzpicture}
      \begin{axis}[
          axis lines=middle,
          xlabel=$x$,
          ylabel=$y$,
          xmin=0, xmax=3, % Adjust xmax as needed to show the curve well
          ymin=0, ymax=3, % Adjust ymax as needed
          ytick={1}, % Custom y-tick for '1'
          xtick={2},
          xticklabels={$t$},
          tick label style={font=\footnotesize},
          xlabel style={at={(axis description cs:1,0.05)},anchor=north west},
          ylabel style={at={(axis description cs:0.05,1)},anchor=south east},
          grid=none,
          clip=false, % Allow drawing outside the axis bounding box for labels like C
        ]

        % Define a value for t for demonstration. In a real problem, t would be variable.
        \pgfmathsetmacro{\tval}{2} % Example value for t
        \node[above] at (axis cs:\tval, 2.236) {Q}; % sqrt(5)
        \node[above right ] at (axis cs:\tval, 0) {P};

        % Plot the function y=sqrt(x^2+1)
        \addplot[blue, thick, samples=100, domain=0:3,name path=A] {sqrt(x^2+1)}; %  coordinate[pos=1] (Q);

        % Fill the area S1
        %\addplot[fill=gray!20, draw=none, domain=0:\tval] {sqrt(x^2+1)} \closedcycle;

        % Draw the line x=t
        \draw[dashed] (\tval,0) -- (\tval,2.236);
        \draw[blue, name path=B] (0,0) -- (\tval,2.236);
        \addplot[pattern=north west lines] fill between [of=A and B,soft clip={domain=0:2}];

        % % Asymptote y=x
        \draw[thick,dotted] (axis cs:0,0) -- (axis cs:3,3) node[above right] {$y=x$};

        % % Label C near the curve
        \node at (axis cs:2.5, 3.0) {C};

        % % Add O for origin
        \node at (axis cs:0,0) [below left] {O};

      \end{axis}
    \end{tikzpicture}
    \caption{$S_1$と$S_2$の差は塗りつぶされた部分である.}
    \label{fig:2}
  \end{figure}

  \newpage
\end{multicols}
\end{document}