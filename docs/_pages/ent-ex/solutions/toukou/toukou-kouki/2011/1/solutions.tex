entclass[a4paper,10pt]{ltjsarticle}
\usepackage{luatexja}
\usepackage[hiragino-pron]{luatexja-preset}

\usepackage[truedimen,top=25truemm,bottom=20truemm,left=15truemm,right=15truemm]{geometry}
\setlength{\textwidth}{54\zw}
\setlength{\textheight}{73\zw}

\usepackage{amsmath,amssymb,ascmac}
\usepackage{enumerate}
\usepackage{multicol}
\usepackage{physics}
\usepackage{cleveref}
\usepackage{framed}
 \usepackage{caption}
 \usepackage{subcaption}
\usepackage{fancyhdr}
\usepackage{latexsym}
\usepackage{mathtools}
\usepackage{tikz}
\usepackage{tikz-3dplot}
\usepackage{pgfplots}
\pgfplotsset{compat=1.18}
\usetikzlibrary{intersections}
\usepgfplotslibrary{fillbetween}
 \usetikzlibrary{math}
 \usetikzlibrary{calc}
 \usetikzlibrary{angles}
 \usetikzlibrary{quotes}
 \usetikzlibrary{patterns}
 \usetikzlibrary{arrows.meta}

% \usepackage{indent}

\usepackage{cases}
\usepackage{float}
\allowdisplaybreaks
\pagestyle{fancy}
\lhead{}
\chead{}
\rhead{東工大後期$2011$年$1$番}
\begin{document}

\begin{oframed}
  正の実数 $t$ に対して,座標空間における 4 点 O$(0,0,0)$,A$(t,0,0)$,B$(0,1,0)$,C$(0,0,1)$ を考える.このとき,次の問に答えよ.
  \begin{enumerate}
    \item 四面体 OABC のすべての面に内接する球 $P$ の半径 $r$ を $t$ を用いて表せ.
    \item $t$ が動くとき,球 $P$ の体積を四面体 OABC の体積で割った値の最大値を求めよ.
  \end{enumerate}
\end{oframed}
\setlength{\columnseprule}{0.4pt}
\begin{multicols}{2}
  {\bf[解]}

  (1)

  \begin{figure}[H]
    \centering
    \tdplotsetmaincoords{70}{120}

    \begin{tikzpicture}[
        scale=2.0,
        tdplot_main_coords,
        point/.style={circle,fill=gray,inner sep=0pt,minimum size=2pt},
        label/.style={anchor=north west}
      ]

      % Origin
      \coordinate (O) at (0,0,0);
      \node [point] at (O) {};
      \node [anchor=south west] at (O) {O};

      % Points A, B, C (assuming t=1)
      \coordinate (A) at (2,0,0);
      \coordinate (B) at (0,1,0);
      \coordinate (C) at (0,0,1);

      % Axes
      \draw [-Stealth,thick] (O) -> ++(2.5,0,0) node [label] {x};
      \draw [-Stealth,thick] (O) -> ++(0,2.5,0) node [label] {y};
      \draw [-Stealth,thick] (O) -> ++(0,0,2.5) node [label] {z};

      % Draw the tetrahedron
      \draw [thick] (O) -- (A) node [midway, below] {};
      \draw [thick] (O) -- (B) node [midway, left] {};
      \draw [thick] (O) -- (C) node [midway, above] {};
      \draw [thick] (A) -- (B) node [midway, below right] {};
      \draw [thick] (B) -- (C) node [midway, above left] {};
      \draw [thick] (C) -- (A) node [midway, right] {};

      % % Label the points
      \node [label=below] at (A) {A(t,0,0)};
      \node [label=right] at (B) {B(0,1,0)};
      \node [label=above] at (C) {C(0,0,1)};
    \end{tikzpicture}
    \caption{四面体$OABC$の様子}
    \label{fig:1}
  \end{figure}

  四面体$OABC$の様子を\cref{fig:1}に示す.
  四面体$OABC$の体積$V$をふた通りで表すことで$r$と$t$の関係を導く.
  まず,底面$\triangle OAB=t/2$,高さ$CO$の四面体と見なすことで,$V$は
  \begin{align}
    V
     & = \frac{1}{3} \triangle \mathrm{OAB}\cdot \mathrm{CO} \nonumber \\
     & = \frac{1}{3} \frac{t}{2} \cdot 1 \nonumber                     \\
     & = \frac{t}{6} \label{eq:1}
  \end{align}
  と表せる.

  次に,内接円の半径$r$,中心$P$とすると,
  \begin{align}
    V
     & = \triangle \mathrm{POAB} + \triangle \mathrm{POBC} + \triangle \mathrm{POAC} +  \triangle \mathrm{PABC} \label{eq:2}
  \end{align}
  と,$4$つの四面体の体積の和として表せる.底面の三角形の面積はそれぞれ
  \begin{align*}
    \triangle \mathrm{OBC} & = \frac{1}{2}                                                        \\
    \triangle \mathrm{OAC} & = \frac{t}{2}                                                        \\
    \triangle \mathrm{OAB} & = \frac{t}{2}                                                        \\
    \triangle \mathrm{ABC} & = \frac{1}{2}\sqrt{|\vec{AB}|^2\vec{AC}^2-(\vec{AB}\cdot\vec{AC})^2} \\
                           & = \frac{1}{2}\sqrt{(t^2+1)^2-t^4}                                    \\
                           & = \frac{1}{2}\sqrt{2t^2+1}
  \end{align*}
  だから,\cref{eq:2}に代入して
  \begin{align}
    V = \frac{r}{3}\left[\frac{1}{2}+t+\frac{1}{2}\sqrt{2t^2+1}\right]\label{eq:3}
  \end{align}
  である.従って,\cref{eq:1}と\cref{eq:2}が等しいので
  \begin{align*}
    \frac{t}{6}  & = \frac{r}{3} \left( \frac{1}{2} + t + \frac{1}{2} \sqrt{2t^2+1} \right) \\
    t            & = r (1 + 2t + \sqrt{2t^2+1})                                             \\
    \therefore r & = \frac{t}{1+2t+\sqrt{2t^2+1}}
  \end{align*}
  と表せる.  $\cdots$(答)

  \vspace{10pt}
  (2)
  Pの体積$V_1$, 四面体OABCの体積$V_2$, その比$f(t) = \frac{V_1}{V_2}$ とおく.
  (1)から
  \begin{align*}
    V_1 & = \frac{4}{3} \pi r^3 \\
    V_2 & = \frac{1}{6} t
  \end{align*}
  だから,$f(t)$は
  \begin{align}
    f(t) & = 8\pi \frac{r^3}{t}                                          \nonumber \\
         & = 8\pi \frac{1}{t}\left(\frac{t}{1+2t+\sqrt{2t^2+1}}\right)^3 \nonumber \\
         & = 8\pi \frac{t^2}{(1+2t+\sqrt{2t^2+1})^3} \label{eq:4}
  \end{align}
  $t$を$t>0$で動かした時の$f(t)$の最大値を求めれば良い.
  そこで$f(t)$の増減表を得るため,$f'(t)$を求める.

  まず,
  \begin{align}
    g(t) = 1+2t+\sqrt{2t^2+1}  \label{eq:5}
  \end{align}
  とおくと $g'(t) = 2 + \frac{2t}{\sqrt{2t^2+1}}$ で,
  \begin{align*}
    f'(t) & = 8\pi \frac{2t g(t)^3 - 3t^2 g(t)^2 g'(t)}{g(t)^6}      \\
          & = \frac{8\pi t}{g(t)^4} \left[ 2 g(t) - 3t g'(t) \right]
  \end{align*}

  よって$f'(t)$の符号は $h(t)=2 g(t) - 3t g'(t)$ の符号にひとしい.
  \cref{eq:5}を代入して具体的に$h(t)$の形を求めると
  \begin{align*}
    h(t) & = 2(1+2t+\sqrt{2t^2+1}) - 3t\left(2 + \frac{2t}{\sqrt{2t^2+1}}\right)   \\
         & = 2-2t + 2\sqrt{2t^2+1} - \frac{6t^2}{\sqrt{2t^2+1}}                    \\
         & = \frac{2}{\sqrt{2t^2+1}} \left[ (1-t)\sqrt{2t^2+1} + (1-t^2)\right]    \\
         & = \frac{2}{\sqrt{2t^2+1}}\left(1-t\right)\left(1+t+\sqrt{2t^2+1}\right)
  \end{align*}
  である.符号が反転するのは$(1-t)$の部分のみである.
  従って$f(x)$の増減表は\cref{table:1}のようになる.
  \begin{table}[H]
    \centering
    \caption{$f$の増減表}
    \label{table:1}
    \begin{tabular}{|c||c|c|c|c|c|}
      \hline
      $t$  & $0$ &            & $1$ &            & ($\infty$) \\
      \hline
      $f'$ &     & +          & $0$ & -          &            \\
      \hline
      $f$  & $0$ & $\nearrow$ &     & $\searrow$ & $0$        \\
      \hline
    \end{tabular}
  \end{table}
  よって求める最大値は$t=1$の時で,この時\cref{eq:5}より
  \begin{align*}
    g(1)=3+\sqrt{3}
  \end{align*}
  だから
  \begin{align*}
    f(1)
     & = \frac{8\pi}{(3+\sqrt{3})^3} \\
     & = \frac{18-10\sqrt{3}}{9}\pi
  \end{align*}
  が求める最大値である.$\cdots$(答)


  \vspace{10pt}
  {\bf[解説]}
  求めた$f(t)$の$t=0,\infty$で$f(t)=0$というのは幾何学的に考えても自然なので,それらしい結果が出ていると安心できる.
  また,答えが$t=1$の時なのは最もらしく,この時三角形ABCが正三角形になっている.

  参考までに$f(t)$は\cref{fig:2}のようになる.

  \begin{figure}[H]
    %\centering
    \begin{tikzpicture}
      \begin{axis}[
          axis lines=middle,
          xmin=0, xmax=4,
          ymin=0, ymax=0.5,
          xlabel=$t$,
          ylabel=$y$,
          xtick={1},
          xticklabels={$1$},
          ytick={0.237}, % 1/e の近似値と -log n / n の例
          yticklabels={$f(1)$},
          grid=none,
          %no markers,
          clip=false, % ラベルが軸からはみ出しても表示されるように
        ]

        % f(x) のプロット
        \addplot[blue, thick, domain=0.:4, samples=100,smooth] {8*pi*(x*x)/(1+2*x+sqrt(2*x*x+1))^3};

        % h(t)ラベル
        \node[above right] at (axis cs:2, 0.2) {$f(t)$};


        % 極大点とy軸のラベルを結ぶ破線
        \fill [black] (axis cs:1, 0.237) circle (2pt);
        \draw[dashed] (axis cs:1, 0.237) -- (axis cs:0, 0.237);
        \draw[dashed] (axis cs:1, 0.237) -- (axis cs:1, 0);
      \end{axis}
    \end{tikzpicture}
    \caption{$f(t)$の概形.$t=1$で最大値をとる.}
    \label{fig:2}
  \end{figure}


  \newpage
\end{multicols}
\end{document}