\documentclass[a4paper,10pt]{ltjsarticle}
\usepackage{luatexja}
\usepackage[hiragino-pron]{luatexja-preset}

\usepackage[truedimen,top=25truemm,bottom=20truemm,left=15truemm,right=15truemm]{geometry}
\setlength{\textwidth}{54\zw}
\setlength{\textheight}{73\zw}

\usepackage{amsmath,amssymb,ascmac}
\usepackage{enumerate}
\usepackage{multicol}
\usepackage{cleveref}
\usepackage{framed}
\usepackage{fancyhdr}
\usepackage{latexsym}
\usepackage{tikz}
\usepackage{pgfplots}
 \usetikzlibrary{math}
% \usepackage{indent}
\usepackage{cases}
\usepackage{float}
\allowdisplaybreaks
\pagestyle{fancy}
\lhead{}
\chead{}
\rhead{東工大後期$1991$年$1$番}
\begin{document}
\begin{oframed}
10進表示の$n$桁の正の整数で,隣り合う桁の数字が互いに相異なるような数の個数 を$a_n$とするとき,次の問いに答えよ.
    \begin{enumerate}
        \item $a_n$を求めよ.
        \item 上の数のうちで,1の位の数字が0である数の個数を$b_n$とするとき,$ \displaystyle \lim_{n \to \infty} \frac{b_n}{a_n}$を求めよ.
    \end{enumerate}
\end{oframed}
\setlength{\columnseprule}{0.4pt}
\begin{multicols}{2}
{\bf[解]}

(1) 最高位の選び方は$1$から$9$のなかから一つえらぶ$9$通り,,それ以下の位は$0$から$9$のうち一つ上の位で使われた以外の一つを選ぶ$9$通りなので,結局すべての桁について$9$通りの選び方があるから,答えは $a_n=9^n$ 通り.$\cdots$(答)

\vspace{10pt}
(2)
 $a_n$通りの整数のうち,1の位が$0$のものを$b_n$個,1の位が$1$であるものを$c_n$個とする.対称性から,$1$位が$2,3,\dots,9$であるものも$c_n$個ずつあり,これらは排反だから
    \begin{align}\label{eq:1}
        b_n+9c_n &= a_n  
    \end{align}
が成り立つ.

次に,$b_{n+1}$を考える.$n+1$桁の正整数の下$2$桁に注目すると,$1$の位は$0$であるから,$10$の位は$1,2,\dots,9$のうち一つを選ぶことができる.このような$1$から$n+1$の位の選び方は$9c_n$に等しいので,
    \begin{align*}
        b_{n+1} &= 9c_n = a_n-b_n \quad (\because \text{\cref{eq:1}}) 
    \end{align*}
となる.両辺を$a_n$で割ると
    \begin{align*}
        \frac{b_{n+1}}{a_n} &= 1-\frac{b_n}{a_n} 
    \end{align*}
ここで(1)から$a_n=9^n$だから$a_{n+1}=9 a_{n}$となることを利用して
    \begin{align*}
        9\frac{b_{n+1}}{a_{n+1}} &= 1-\frac{b_n}{a_n} 
    \end{align*}
を得る.$d_n=b_{n}/a_{n}$とおくと,$d_n$に関する漸化式
    \begin{align*}
        9d_{n+1} &= 1-d_n
    \end{align*}
を得る.$d_n$の初期条件は$b_1=0$から$d_1=0$である.したがってこの漸化式は
    \begin{align*}
        d_{n+1}-\frac{1}{10} &= -\frac{1}{9}\left(d_n-\frac{1}{10}\right) \\
        d_n &= \frac{1}{10} \left\{-\left(\frac{-1}{9}\right)^{n-1}+1\right\}
    \end{align*}
と解ける.したがって求めるべき極限値は
    \begin{align*}
        \lim_{n \to \infty} \frac{b_n}{a_n} &= \lim_{n \to \infty} d_n \\
        &= \lim_{n \to \infty} \frac{1}{10}\left\{-\left(\frac{-1}{9}\right)^{n-1}+1\right\} \\
        &= \frac{1}{10}
    \end{align*}
である.$\cdots$(答)

{\bf[解説]}
典型的な場合の数と数列の問題であり,解法も漸化式がたてば簡単である.
(2)の解答も$1/10$であり,これは桁数が増えてくれば$1$の位の値がほぼランダムに分布することを意味しており,非常にそれらしい値になっている.

     \newpage
\end{multicols}
\end{document}