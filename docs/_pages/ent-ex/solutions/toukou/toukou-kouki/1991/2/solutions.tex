\documentclass[a4paper,10pt]{ltjsarticle}
\usepackage{luatexja}
\usepackage[hiragino-pron]{luatexja-preset}

\usepackage[truedimen,top=25truemm,bottom=20truemm,left=15truemm,right=15truemm]{geometry}
\setlength{\textwidth}{54\zw}
\setlength{\textheight}{73\zw}

\usepackage{amsmath,amssymb,ascmac}
\usepackage{enumerate}
\usepackage{multicol}
\usepackage{cleveref}
\usepackage{framed}
\usepackage{fancyhdr}
\usepackage{latexsym}
\usepackage{mathtools}
\usepackage{tikz}
\usepackage{pgfplots}
 \usetikzlibrary{math}
% \usepackage{indent}
\usepackage{cases}
\usepackage{float}
\allowdisplaybreaks
\pagestyle{fancy}
\lhead{}
\chead{}
\rhead{東工大後期$1990$年$2$番}
\begin{document}
\begin{oframed}
$n$ を $2$ 以上の整数とする.
\begin{enumerate}
  \item $n-1$ 次多項式 $P_n(x)$ と $n$ 次多項式 $Q_n(x)$ で実数 $\theta$ に対して
\[ \sin(2n\theta) = n \sin(2\theta) P_n(\sin^2\theta), \quad \cos(2n\theta) = Q_n(\sin^2\theta) \]
を満たすものが存在することを帰納法を用いて示せ.
  \item $k=1, 2, \dots, n-1$ に対して $\alpha_k = \left(\sin\frac{k\pi}{2n}\right)^{-2}$ とおくと
\[ P_n(x) = (1-\alpha_1 x)(1-\alpha_2 x)\cdots(1-\alpha_{n-1} x) \]
となることを示せ.
  \item $\displaystyle \sum_{k=1}^{n-1} \alpha_k = \frac{2n^2-2}{3}$ を示せ.
\end{enumerate}
\end{oframed}
\setlength{\columnseprule}{0.4pt}
\begin{multicols}{2}
{\bf[解]}

$n \in \mathbb{N}$ に拡張して考えて良い.

(1) 数学的帰納法で題意を示す.

\vspace{10pt}

(i)
$n=1, 2$ のとき
\begin{align*}
    P_1(x) &= 1, P_2(x) = 1-2x \\
    Q_1(x) &= 1-2x, Q_2(x) = 8x^2 - 8x + 1 \quad \cdots (*)
\end{align*}
とすれば良く成立する.

\vspace{10pt}

(ii)
以下 $n=k, k+1 \in \mathbb{N}$ での成立を仮定し,$n=k+2$ での成立を示す.
和積公式より,
\begin{align*}
    \sin(2k+4)\theta 
    &= 2\sin(2k+2)\theta\cos2\theta - \sin2k\theta \\
    \cos(2k+4)\theta 
    &= 2\cos(2k+2)\theta\cos2\theta - \cos2k\theta 
\end{align*}
であり,ここに$n=k, k+1$ のときの $P_n(x), Q_n(x)$ を代入して
\begin{align*}
    \sin(2k+4)\theta 
    =& 2 (k+1) \sin(2\theta) P_{k+1}(\sin^2\theta) \cos 2\theta \\
     &- 2 k \sin(2\theta) P_k(\sin^2\theta) \\
    =& \left\{2(1-2\sin^2\theta)(k+1)P_{k+1}(\sin^2\theta) \right. \\
     & \left. - kP_k(\sin^2\theta)\right\}\sin(2\theta) \\
    \cos(2k+4)\theta
    =& 2Q_{k+1}(\sin^2\theta)\cos2\theta - Q_{k}(\sin^2\theta) \\
    =& 2(1-2\sin^2\theta)Q_{k+1}(\sin^2\theta) - Q_k(\sin^2\theta)
\end{align*}
を得る.ただし途中で半角公式$\cos2\theta = 1-2\sin^2\theta$を利用した.
従って$x=\sin^2\theta$ とおいて
\begin{align}\label{eq:condition}
    \begin{dcases}
        P_{k+2}(x) &= \frac{1}{k+2}\{2(1-2x)(k+1)P_{k+1}(x) - kP_k(x)\} \\
        Q_{k+2}(x) &= 2(1-2x)Q_{k+1}(x) - Q_k(x)
    \end{dcases}
\end{align}
とすれば,$P_{k+2}(x), Q_{k+2}(x)$ は $k+1, k+2$ 次多公式であり条件を満たす.以上から $n=k+2$ でも成立.

(i), (ii) より,数学的帰納法により題意は示された.

\vspace{10pt}
(2) 題意を示すには,因数定理より$P_n(x)$ の零点が$x=1/\alpha_{k}$であること,および$0$次の係数が$1$であることを示せば良い.
まずは前者から示す.$x=\sin^2\theta$ とすると、$\sin^2\theta$ の周期性から
\begin{align}\label{eq:1}
 0 \le \theta < \frac{\pi}{2}
\end{align}
の範囲のみ考えれば良い.以下 $\theta$ をこの範囲で考える.
\begin{align}\label{eq:2}
 \sin(2\theta) = 0 \Leftrightarrow \theta = 0 
\end{align}
だから、これ以外の時、(1)から
\begin{align*}
  P_n(\sin^2\theta) = \frac{\sin(2n\theta)}{2n\sin(2\theta)} \qquad \cdots\cdots ②
\end{align*}
とかける.したがって$P_n(\sin^2\theta)=0$ となるのは $\sin(2n\theta)=0$ のときである.この条件は\cref{eq:1,eq:2}に注意して
\begin{align*}
  \theta = \frac{k\pi}{2n} \quad (k=1, 2, \dots, n-1)
\end{align*}
となる. これら $n-1$ 個の$\theta$に対して$x$は異なる値を取り,さらに $P_n(x)$ は $n-1$ 次式だから、これが $P_n(x)=0$ の全ての解である.これで1つ目の条件は示された.したがって因数定理から,$A\ne 0$ として
\begin{align}\label{eq:3}
  P_n(x) = A \prod_{k=1}^{n-1} \left(x - \sin^2\frac{k\pi}{2n}\right) 
\end{align}
とおける。以下 $A$ を求める。$P_n(x)$ の定数項を $a_n$ とすると、\cref{eq:condition}から
\begin{align}
 \begin{dcases} 
  & a_1=1, a_2=1 \\ 
  & a_{n+2} = \frac{1}{n+2}\{2(n+1)a_{n+1} - na_n\} 
\end{dcases}
\end{align}
となり、帰納的に $a_n=1$ である.これで2つ目の条件が示された.\cref{eq:3}で定数項の係数を比較して
\begin{align*}
  A &\prod_{k=1}^{n-1} \left(-\sin^2\frac{k\pi}{2n}\right) = 1 \\
  \therefore A &= \frac{1}{\prod_{k=1}^{n-1} \left(-\sin^2\frac{k\pi}{2n}\right)}
\end{align*}
だから、これを\cref{eq:3}に代入して
\begin{align}
  P_n(x) = \prod_{k=1}^{n-1} \left(1 - \frac{x}{\sin^2\frac{k\pi}{2n}}\right) = \prod_{k=1}^{n-1} (1-\alpha_k x) 
\end{align}
を得る.以上より題意は示された.

\vspace{10pt}
(3)
$\displaystyle \sum_{k=1}^{n-1} \alpha_k$ は、(2)の結果より$P_n(x)$ の $x$ の1次の項の係数を $b_n$ として、
\begin{align}\label{eq:5}
\sum_{k=1}^{n-1} \alpha_k = -b_n 
\end{align}
と表せる。したがって,題意を示すには
\begin{align}\label{eq:4}
b_n = -\frac{2}{3}(n^2 - 1)
\end{align}
を示せばよい.ここで,$b_n$ については\cref{eq:condition}から
\begin{align}\label{eq:6}
\begin{dcases}
b_1 = 0, \ b_2 = -2 \\
b_{n+2} = \frac{1}{n+2} [2(n+1)(b_{n+1} - 2) - n b_n] \quad (\because a_n=1)
\end{dcases}
\end{align}
なる漸化式が成立する.これを用いて以下 \cref{eq:4} がすべての$n$について成立することを数学的帰納法により示す.

\vspace{10pt}
(i) $n=1, 2$ のとき
\begin{align*}
b_1 &= -\frac{2}{3}(1^2-1) = 0 \\
b_2 &= -\frac{2}{3}(2^2-1) = -2  
\end{align*}
となり、\cref{eq:6} の初期条件と一致するため成立する.

\vspace{10pt}
(ii) $n=k, k+1$ での成立を仮定する。すなわち、
$b_k = -\frac{2}{3}(k^2 - 1)$, $b_{k+1} = -\frac{2}{3}((k+1)^2 - 1)$
が成り立つと仮定する.このとき、$n=k+2$ での成立を示す.
\cref{eq:6} の漸化式に $n=k$ を代入して仮定を用いると,
\begin{align*}
b_{k+2} 
&= \frac{1}{k+2} \left[ 2(k+1)(b_{k+1} - 2) - k b_k \right] \\
&= \frac{1}{k+2} \left[ 2(k+1) \left\{ -\frac{2}{3}((k+1)^2 - 1) - 2 \right\} - k \left\{ -\frac{2}{3}(k^2 - 1) \right\} \right] \\
&= \frac{1}{k+2} \left[ 2(k+1) \left\{ -\frac{2}{3}(k^2+2k) - 2 \right\} + \frac{2}{3}k(k^2 - 1) \right] \\
&= \frac{1}{k+2} \left[ -\frac{4}{3}(k+1)(k^2+2k) - 4(k+1) + \frac{2}{3}k(k^2-1) \right] \\
&= \frac{2}{3(k+2)} \left[ -2(k+1)k(k+2) - 6(k+1) + k(k-1)(k+1) \right] \\
&= \frac{2(k+1)}{3(k+2)} \left[ -2k(k+2) - 6 + k(k-1) \right] \\
&= \frac{2(k+1)}{3(k+2)} (-k^2 - 5k - 6) \\
&= -\frac{2(k+1)}{3(k+2)} (k+2)(k+3) \\
&= -\frac{2}{3}(k+1)(k+3) \\
&= -\frac{2}{3}((k+2)^2 - 1)
\end{align*}
よって、$n=k+2$ のときも成立する.

\vspace{10pt}

(i), (ii) より、すべての自然数 $n$ に対して $b_n = -\frac{2}{3}(n^2 - 1)$ が示された.
これと \cref{eq:5} から、
\begin{align*}
\sum_{k=1}^{n-1} \alpha_k = -b_n = \frac{2}{3}(n^2-1)
  \end{align*}
となり題意は示された.

\vspace{10pt}
{\bf[解説]}
この問題は帰納法を用いて多項式の性質を示し,最終的には$\sin$の逆二乗和を求める問題である.帰納法を利用するだけなので問題の難易度としては高くないが,この問題の背景としてはチェビシフ多項式と呼ばれる三角関数から派生した多項式と,バーゼル問題という無限級数の和を求める問題があり,これを知っているとより深く理解できる.

第一種および第二種チェビシフ多項式$T_n(x), U_n(x)$は以下の式で定義される.
\begin{align*}
    T_n(x) &= \cos(n\theta) \quad (\text{ただし } x = \cos\theta) \\
    U_n(x) &= \frac{\sin(n\theta)}{\sin\theta} \quad (\text{ただし } x = \cos\theta)
\end{align*}
ここで、$T_n(x)$ も $U_n(x)$ も $n$ 次の多項式であることが帰納的に示せる.このように三角関数が$\cos\theta$に関する多項式に変換できるのは本問中の証明でみも見たように加法定理の帰結である.いくつか$n$の小さい値に対して例示すると
\begin{align*}
    T_0(x) &= 1, \quad T_1(x) = x, \\
    T_2(x) &= 2x^2 - 1, T_3(x) &= 4x^3 - 3x \\
    U_0(x) &= 1, \quad U_1(x) = 2x, \\
    U_2(x) = 4x^2 - 1, \quad U_3(x) = 8x^3 - 4x
\end{align*}
のようになる.
チェビシフ多項式は以下のような三項間漸化式を満たす.
\begin{align*}
    T_{n+1}(x) &= 2xT_n(x) - T_{n-1}(x) \\
    U_{n+1}(x) &= 2xU_n(x) - U_{n-1}(x)
\end{align*}
本文では$P_n(x)$と$Q_n(x)$が$\sin^{2}\theta$による関数として定義されているため漸化式が複雑になっている.また,$T_n(x)$と$U_n(x)$は
\begin{align*}
    nU_{n-1}(x) = T'_n(x)
\end{align*}
というきれいな微分の関係で結ばれている.

チェビシフ多項式には数多くの重要な性質があるが,ここでは例として母関数および直交性を紹介する.
チェビシフ多項式の母関数は以下のように表される
\begin{align*}
    \sum_{n=0}^{\infty} T_n(x) t^n &= \frac{1 - xt}{1 - 2xt + t^2} \\
    \sum_{n=0}^{\infty} U_n(x) t^n &= \frac{1}{1 - 2xt + t^2}
\end{align*}
ここで、$|t|<1$の範囲で収束する.
また、チェビシフ多項式は直交性を満たす.
すなわち、$n, m \in \mathbb{Z}$ にたいして
\begin{align*}
    \int_{-1}^{1} T_n(x) T_m(x)\frac{1}{\sqrt{1-x^2}} \, dx &= N_{n}\delta_{nm} \\
    \int_{-1}^{1}  U_n(x) U_m(x) \sqrt{1-x^2} \, dx &= \frac{\pi}{2} \delta_{nm} 
\end{align*}
が成り立つ.ただし$N_0=\pi$、$N_n = \frac{\pi}{2}$ ($n \ge 1$)である.
この直交性は、チェビシフ多項式が$[-1, 1]$の区間で定義される重み関数$\sqrt{1-x^2}^{\pm 1}$に対して直交することを意味する.
この性質は最小二乗近似の問題において重要な役割を果たす.
また、チェビシフ多項式は数値解析や信号処理などの分野でも広く利用されている.

次に,バーゼル問題について紹介する.バーゼル問題は逆二乗和の収束値を求める問題であり,以下のように定義される.
\begin{align*}
    \sum_{n=1}^{\infty} \frac{1}{n^2} = \frac{\pi^2}{6}
\end{align*}
このようになぜか右辺に$\pi$が現れるのが面白い問題である.この問題はオイラーによって解かれたのち,いくつかの解法が知られているが,
ここでは本文と関連する証明方法を説明する.

自然数$n$にたいして,$k=1,2,\dots,n$ として $\theta_k = \frac{k\pi}{2n+1}$ とおく.$0 < \theta_k < \frac{\pi}{2}$ から 
\begin{align}
 0 < \sin\theta_k < \theta_k < \tan\theta_k
\end{align}
だから,逆数をとって2乗して整理すると,
\begin{align*}
 \frac{1}{\tan^2\theta_k} &\le \frac{1}{\theta_k^2} \le \frac{1}{\sin^2\theta_k} \\
 \frac{\pi^2}{(2n+1)^2} \left(-1 + \frac{1}{\sin^2\theta_k}\right) &\le \frac{1}{k^2} \le \frac{\pi^2}{(2n+1)^2} \frac{1}{\sin^2\theta_k}
\end{align*}
$k$ について和をとって $\displaystyle A_n = \sum_{k=1}^n \frac{1}{\sin^2\theta_k}$ とおくと,
\begin{align}
\frac{\pi^2}{(2n+1)^2} \left(-n + A_n\right) \le \sum_{k=1}^n \frac{1}{k^2} \le \frac{\pi^2}{(2n+1)^2} A_n
\end{align}
だから,$A_n$ を求めると $\displaystyle \sum_{k=1}^n \frac{1}{k^2}$ が求まる.そして,$A_n$が本問(3)で求めたそのものである.したがって,(3)の結果を代入すると
\begin{align*}
  \frac{\pi^2}{(2n+1)^2} \left(-n + \frac{2}{3}(n^2-1)\right) \le \sum_{k=1}^n \frac{1}{k^2} \le \frac{\pi^2}{(2n+1)^2} \frac{2}{3}(n^2-1)
\end{align*}
両辺とも$n\to\infty$の極限を取ると$\pi^2/6$に収束するから,挟み撃ちの定理により
\begin{align*}
 \sum_{k=1}^n \frac{1}{k^2} \longrightarrow \frac{\pi^2}{6}
\end{align*}
が得られる.


\newpage
\end{multicols}
\end{document}