\documentclass[a4paper,10pt]{ltjsarticle}
\usepackage{luatexja}
\usepackage[hiragino-pron]{luatexja-preset}

\usepackage[truedimen,top=25truemm,bottom=20truemm,left=15truemm,right=15truemm]{geometry}
\setlength{\textwidth}{54\zw}
\setlength{\textheight}{73\zw}

\usepackage{amsmath,amssymb,ascmac}
\usepackage{enumerate}
\usepackage{multicol}
\usepackage{cleveref}
\usepackage{framed}
\usepackage{fancyhdr}
\usepackage{latexsym}
\usepackage{mathtools}
\usepackage{tikz}
\usepackage{pgfplots}
 \usetikzlibrary{math}
% \usepackage{indent}
\usepackage{cases}
\usepackage{float}
\usepackage{cases}
  \usepackage{caption}
  \usepackage[subrefformat=parens]{subcaption}
\allowdisplaybreaks
\pagestyle{fancy}
\lhead{}
\chead{}
\rhead{東工大後期$1992$年$2$番}
\begin{document}
\begin{oframed}
$0 < a < 1$ とする.
座標平面上で原点 $A_0$ から出発して $x$ 軸の正の方向に $a$ だけ進んだ点を $A_1$ とする.
次に $A_1$ で進行方向を反時計回りに $120^\circ$ 回転し $a^2$ だけ進んだ点を $A_2$ とする.
以後同様に $A_{n-1}$ で反時計回りに $120^\circ$ 回転して $a^n$ だけ進んだ点を $A_n$ とする.
このとき点列 $A_0, A_1, A_2, \dots$ の極限の座標を求めよ.
\end{oframed}
\setlength{\columnseprule}{0.4pt}
\begin{multicols}{2}
{\bf[解]}


ベクトル $\vec{A_n A_{n+1}}$ を表す複素数を $d_n$ と表す.
また、$e(\theta) = \cos\theta + i\sin\theta$ とし,$p=e(2\pi/3)$とする.題意から,
\begin{align*}
\begin{dcases}
d_{n+1} = a p d_n & (n \ge 0) \\
d_0 = a &
\end{dcases}  
\end{align*}
となる.これは初項 $d_0 = a$、公比 $ap$ の等比数列だから,その一般項は,
\begin{align*}
d_n = a(ap)^n  
\end{align*}
となる.
したがって,点 $A_n$ を表す複素数を $t_n$ とおくと,$n\ge 1$に対して
\begin{align}
 t_n = \sum_{k=0}^{n-1} d_k = a \frac{1-(ap)^n}{1-ap}  
\end{align}
となる.題意より$0<a<1$であり,またその定義から$|p|=1$であるから
$\lim_{n \to \infty} (ap)^n = 0$となり,$t_n$の極限値は
\begin{align*}
    \lim_{n \to \infty} t_n &= \frac{a}{1-ap} 
\end{align*}
である.ここに$p=-1/2+i\sqrt{3}/2$を代入すると
\begin{align*}
    \lim_{n \to \infty} t_n
    &= \frac{a(1-a\bar{p})}{(1-ap)(1-a\bar{p})} \\
    &= \frac{a(1-a\bar{p})}{1-2a\Re p + a^2} \\
    &= \frac{2a-a^2(-1-i\sqrt{3})}{2(1+a + a^2)} \\
    &= \frac{a(2+a)+i\sqrt{3}a^2}{2(1+a + a^2)} \\
\end{align*}
したがって、求める座標は
\begin{align*}
\left( \frac{a(2+a)}{2(a^2+a+1)}, \frac{\sqrt{3}a^2}{2(a^2+a+1)} \right)  
\end{align*}
である.$\cdots$(答)

\vspace{10pt}
{\bf[解説]}
複素数の問題で,点$A_n$の座標に関する漸化式さえ立てられれば容易に解ける問題だ.

\newpage
\end{multicols}
\end{document}