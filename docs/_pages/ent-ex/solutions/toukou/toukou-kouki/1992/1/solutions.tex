\documentclass[a4paper,10pt]{ltjsarticle}
\usepackage{luatexja}
\usepackage[hiragino-pron]{luatexja-preset}

\usepackage[truedimen,top=25truemm,bottom=20truemm,left=15truemm,right=15truemm]{geometry}
\setlength{\textwidth}{54\zw}
\setlength{\textheight}{73\zw}

\usepackage{amsmath,amssymb,ascmac}
\usepackage{enumerate}
\usepackage{multicol}
\usepackage{physics}
\usepackage{cleveref}
\usepackage{framed}
\usepackage{fancyhdr}
\usepackage{latexsym}
\usepackage{mathtools}
\usepackage{tikz}
\usepackage{pgfplots}
 \usetikzlibrary{math}
% \usepackage{indent}
\usepackage{cases}
\usepackage{float}
\allowdisplaybreaks
\pagestyle{fancy}
\lhead{}
\chead{}
\rhead{東工大後期$1992$年$1$番}
\begin{document}
\begin{oframed}
$x$ の関数 $\displaystyle F(x) = \int_0^1 \frac{|t-x|}{t+1} dt$ の最小値を求めよ.
\end{oframed}
\setlength{\columnseprule}{0.4pt}
\begin{multicols}{2}
{\bf[解]}
$f_{x}(t) = \dfrac{t-x}{t+1}$ とおく.$0\le t\le 1$より$t+1>0$だから,
\begin{align}
 F(x) = \int_{0}^{1} \left|f_{x}(t)\right| dt   
\end{align}
である.

$f(t)=0$ となる $t=x$ を境に $f_{x}(t)$ の符号が変わるため場合分けを行うと,
\begin{align}\label{eq:1}
F(x) =
\begin{dcases}
    \int_{0}^{1} f_{x}(t) dt & (x<0) \\
    -\int_{0}^{x} f_{x}(t) dt + \int_{x}^{1} f_{x}(t) dt & (0 \le x \le 1) \\
    -\int_{0}^{1} f_{x}(t) dt & (x>1)
\end{dcases}
\end{align}
と表せる.ここで,
\begin{align*}
\int_{0}^{1} f(t) dt 
&= \int_{0}^{1} \left(1-\frac{x+1}{t+1}\right) dt \\
&= \left[t - (x+1)\log(t+1)\right]_{0}^{1} \\
&= 1 - (x+1)\log 2
\end{align*}
だから,$\displaystyle\int_{0}^{1} f(t) dt$は$x$の単調減少関数であり,\cref{eq:1}より$F(x)$は$0\le x \le 1$で最小値を取る.
以下この場合を調べる.

$G(t,x)$ を $f_{x}(t)$ の変数$t$に関する原始関数とすると,$F(x)$は$G(t,x)$を用いて
\begin{align}
 F(x) 
 &= [-G(t,x)]_{0}^{x} + [G(t,x)]_{x}^{1} \nonumber \\
 &= G(0,x) + G(1,x) - 2G(x,x) \label{eq:3}
\end{align}
と表せる.
\begin{align*}
f_{x}(t) 
&= \frac{t-x}{t+1} \\
&= 1 - \frac{x+1}{t+1}
\end{align*}
ゆえ,$G(t,x)$はこれを$t$について積分して,積分定数は一般に$0$とおいてよく,
\begin{align*}
G(t,x) = t - (x+1)\log(t+1)
\end{align*}
を得る.これを\cref{eq:3}に代入して
\begin{align}
F(x) 
&= \left[1-(x+1)\log 2\right] -2 \left[x-(x+1)\log(x+1)\right] \nonumber \\
&= \left[1-(x+1)\log 2\right] -2 \left[x-(x+1)\log(x+1)\right] \nonumber \\
&= (x+1) \log\frac{\left(x+1\right)^2}{2} -2x + 1 \label{eq:6}
\end{align}
となる.一階微分を計算すると
\begin{align*}
F'(x) 
&= \log\frac{\left(x+1\right)^2}{2} + 2 -2 \\
&= \log\frac{\left(x+1\right)^2}{2}
\end{align*}
を得る.従って$F(x)$ の増減表は以下の通り.
$$
\begin{array}{c|ccccc}
    \hline
    x & 0 & \cdots & \sqrt{2}-1 & \cdots & 1 \\
    \hline
    F'(x) & & - & 0 & + & \\
    \hline
    F(x) & & \searrow & \text{最小} & \nearrow & \\
    \hline
\end{array}
$$
したがって、$F(x)$ は $x=\sqrt{2}-1$ で最小値をとる.この時の最小値は\cref{eq:6}に代入して
\begin{align*}
    \min F(x) 
    &= F(\sqrt{2}-1) \\
    &= -2(\sqrt{2}-1) + 1 \\
    &= 3 - 2\sqrt{2}
\end{align*}
である.$\cdots$(答)

{\bf[解説]}
典型的な積分の問題である.特にこのような絶対値およびパラメータ付きの定積分の最小値を求める問題は頻出であり,以下のようなパターンはよく知られているところだ.
\begin{align}
 I(x) &= \int_{a}^{b}\left| f(t) - x\right| \dd t \\
 J(x) &= \int_{a}^{b}\left| f(t) - xt\right| \dd t 
\end{align}
ただし積分区間内で絶対値の中身は一回だけ符号が入れ替わるとする.下図のような状況をイメージしてほしい.

これらはもう答えがどうなるかというところまで簡単にわかってしまうので以下一つづつ解説しよう.
まずこれらの場合はいずれも,本問と同様のロジックで$a<x<b$で最小値を取る.従ってその区間で考える.
また,$f$の原始関数を$F$とすると,$I$,$J$は$F$を用いて
\begin{align}
 I(x) &= 2F(\alpha) - F(a) - F(b) - 2\alpha x + ax + bx \\
 J(x) &= 2F(\beta) - F(a) - F(b)  - \beta^2 x + \frac{1}{2}a^2 x + \frac{1}{2}b^2 x
\end{align}
とかける.ただし,$\alpha$,$\beta$はそれぞれ
\begin{align}
    f(\alpha) = x \\
    f(\beta)  = \beta x
\end{align}
を満たす$x$に依存する数である.$I$および$J$を$x$で一階微分して
\begin{align*}
 I'(x) 
 &= 2f(\alpha)\alpha' - (2\alpha + 2\alpha' x) + a +b \\
 &= -2\alpha + a + b \\
 J'(x) 
 &= 2f(\beta)\beta'   - (2\beta\beta' x + 2\beta^2) + \frac{1}{2}a^2 + \frac{1}{2}b^2 \\
 &= -2\beta^2 + \frac{1}{2}a^2 + \frac{1}{2}b^2
\end{align*}
を得る.従って,$I$および$J$が最小になるような$x$は$I'(x)=0$, $J'(x)=0$から
\begin{align}
 \alpha &= \frac{a+b}{2}  \\
 \beta &= \frac{\sqrt{a^2+b^2}}{2}
\end{align}
をみたさなければならない.

これを図形的に解釈すると,
$I$については,$x$がちょうど区間$[a,b]$の中点での関数$f$と等しい時に面積が最小になる,ということである.
これは直感的にも納得できる.$J$についてはも同様である.
このように関数$f$の具体的な形がわからなくても,定積分の最小値を与える$x$は求まる,というわけである.
具体的な関数の計算を避けることで,計算ミスを避けながら素早く解くことができる.

さて,翻ってこの問題では,上の$I(x)$や$J(x)$とは異なり,原始関数$G(t,x)$を直接求めることで最小値を求めた.
この問題の構造をより明確に理解するため,極力原始関数を求めずに抽象的なまま式展開を進めよう.
そもそもなぜ$G(x,t)$を求めないといけなくなったかといえば,$f_{x}(t)$の$t$に関する原始関数を求めたのち,$x$について微分するという変数の取り替えがあったからだった.
$I$や$J$では元々$f$が$t$のみの関数として分離されていたので簡単だったのだ.
これを防ぐために,最初から$f$を$x$と$t$で明示的に分離した次のような形を考える.
\begin{align}
    f_{x}(t) 
    &= \frac{t}{t+1} - \frac{x}{t+1} \\
    &= p(t) - q(t) x
\end{align}
ただし,
\begin{align}
    q(t) &= \frac{1}{t+1} \\
    p(t) &= \frac{t}{t+1} = tq(t) \label{eq:4}
\end{align}
とおいた.このように$f$を$t$に依存する部分と$x$に依存する部分に分離して書けば,
\cref{eq:3}の$F(x)$は$p$および$q$の原始関数$P$,$Q$を用いて
\begin{align}
 F(x) = 
 &\left[P(0)-Q(0)x \right] + \left[P(1)-Q(1)x \right] \\
 & -2 \left[P(x)-Q(x)x \right]\label{eq:7}
\end{align}
と表せる.これは明確に$x$で微分することができて,
\begin{align*}
 F'(x) 
 &= -Q(0) - Q(1) -2 \left[P'(x)-Q'(x)x -Q(x)\right] \\
 &= -Q(0) - Q(1) -2 \left[p(x)-q(x)x -Q(x)\right] \\
 &= -Q(0) - Q(1) -2 \left[q(x)x-q(x)x -Q(x)\right] \\
 &= -Q(0) - Q(1) + 2Q(x) 
\end{align*}
を得る.ただし,三行目で$p$と$q$の関係$p(x)=xq(x)$ (\cref{eq:4})を利用した.
ここから,$F$が極小となるには$F'(x)=0$すなわち
\begin{align}
 Q(x) = \frac{Q(0)+Q(1)}{2} \label{eq:5}
\end{align}
となることが必要であることがわかる.
これは図形的に考えると,$q(t)$の$0\le t\le 1$の面積をちょうど半分に分割するような$x$がいくつか,ということに他ならない.
かくして$I(x)$や$J(x)$と同様に,図形的な解釈にまで持ち込むことができるわけである.

残念ながら$q(t)=1/(t+1)$なので図形的に解釈できたからと言って解けるわけではないものの,この手の問題の背景を掴むことができる.
また,\cref{eq:5}までわかれば$Q(x) = \log (x+1)$から容易に$x=\sqrt{2}-1$を導くことができる.
この時の最小値$F(x)$は,\cref{eq:7}に\cref{eq:5}を代入すれば$Q$に関する項が全て消えて
\begin{align}
 \min F(x) = P(0) + P(1) - 2P(x)
\end{align}
となる.この方法であればほぼ具体的な計算をせずに求められるため,計算ミスのおそれも少ないだろう.

     \newpage
\end{multicols}
\end{document}