\documentclass[a4paper,10pt]{ltjsarticle}
\usepackage{luatexja}
\usepackage[hiragino-pron]{luatexja-preset}

\usepackage[truedimen,top=25truemm,bottom=20truemm,left=15truemm,right=15truemm]{geometry}
\setlength{\textwidth}{54\zw}
\setlength{\textheight}{73\zw}

\usepackage{amsmath,amssymb,ascmac}
\usepackage{enumerate}
\usepackage{multicol}
\usepackage{physics}
\usepackage{cleveref}
\usepackage{framed}
 \usepackage{caption}
 \usepackage{subcaption}
\usepackage{fancyhdr}
\usepackage{latexsym}
\usepackage{mathtools}
\usepackage{tikz}
\usepackage{tikz-3dplot}
\usepackage{pgfplots}
\pgfplotsset{compat=1.18}
\usetikzlibrary{intersections}
\usepgfplotslibrary{fillbetween}
 \usetikzlibrary{math}
 \usetikzlibrary{calc}
 \usetikzlibrary{angles}
 \usetikzlibrary{quotes}
 \usetikzlibrary{patterns}

% \usepackage{indent}

\usepackage{cases}
\usepackage{float}
\allowdisplaybreaks
\pagestyle{fancy}
\lhead{}
\chead{}
\rhead{東工大後期$2005$年$1$番}
\begin{document}

\begin{oframed}
  数列 $\{a_m\}$ (ただし $a_m = m$ とする) に対し $\displaystyle b_n = \sum_{m=1}^{n} a_m$ とおく.
  \begin{enumerate}
    \item $0 < r < 1$ とするとき,$\displaystyle \lim_{n \to \infty} nr^n = 0$ および $\displaystyle \lim_{n \to \infty} n^2r^n = 0$ となることを証明せよ.
    \item $S_m = a_1r + a_2r^2 + \dots + a_mr^m$, $T_n = b_1r + b_2r^2 + \dots + b_nr^n$ とおくとき,
          $\displaystyle \lim_{m \to \infty} S_m$ および $\displaystyle \lim_{n \to \infty} T_n$ を求めよ.
  \end{enumerate}
\end{oframed}
\setlength{\columnseprule}{0.4pt}
\begin{multicols}{2}
  {\bf[解]}

  (1)
  $0 < r < 1$ の時,$r = \frac{1}{t} (t>1)$ とおける.
  一般化二項定理より
  \begin{align*}
    t^n
    = & \{1+(t-1)\}^n                                  \\
    = & 1 + (t-1) + \frac{n(n-1)}{2!}(t-1)^2+\cdots    \\
      & +\frac{n(n-1)\cdots(n-k+1)}{k!}(t-1)^k +\cdots \\
    > & \frac{n(n-1)\cdots(n-k+1)}{k!}(t-1)^k
  \end{align*}
  だから,$n > 0$ の時
  \begin{align}
    0 < n^k \cdot r^n = \frac{n^k}{t^n} < \frac{n^k}{\frac{n(n-1)\cdots(n-k+1)}{k!}(t-1)^k} \nonumber \\
    = \frac{n^k}{n(n-1)\cdots(n-k)}\frac{(k+1)!}{(t-1)^{k+1}}\label{eq:1}
  \end{align}
  であり,
  \begin{align*}
    \frac{n^k }{n(n-1)\cdots(n-k+1)} = \frac{1}{\left(1-\frac{1}{n}\right)\cdots\left(1-\frac{k-1}{n}\right) } \xrightarrow{n\to\infty} 0
  \end{align*}
  となるから,\cref{eq:1}の右辺は$0$に収束する.したがってはさみうちの定理から,
  \begin{align*}
    \lim_{n\to\infty} n^k \cdot r^n = 0
  \end{align*}
  である.$k=1,2$として題意は示された.$\cdots$(答)

  \vspace{10pt}
  (2)
  まず,
  \begin{align*}
    b_n = \frac{n(n+1)}{2}
  \end{align*}
  である.題意より
  \begin{align*}
    S_m & = \sum_{k=1}^m k \cdot r^k                 \\
    T_n & = \frac{1}{2}\sum_{k=1}^n k(k+1) \cdot r^k
  \end{align*}
  である.
  先に$S_m$から考える.
  $S_m - rS_m$を計算すると
  \begin{align*}
    S_m - rS_m
     & =  \sum_{k=1}^m k \cdot r^k -  \sum_{k=1}^m k \cdot r^{k+1}               \\
     & =  \sum_{k=1}^m k \cdot r^k -  \sum_{k=2}^{m+1} (k-1) \cdot r^{k}         \\
     & =  \sum_{k=1}^m k \cdot r^k -  \sum_{k=1}^{m} (k-1) \cdot r^{k} -mr^{m+1} \\
     & =  \sum_{k=1}^m r^k -mr^{m+1}                                             \\
     & =  \frac{r(1-r^m)}{1-r} -mr^{m+1}
  \end{align*}
  と書けるから,
  \begin{align*}
    S_m  = \frac{1}{1-r}\left[\frac{r(1-r^m)}{1-r} -mr^{m+1}\right]
  \end{align*}
  を得る.(1)の結果を用いて$m\to\infty$の極限では,
  \begin{align}
    \lim_{m\to\infty}S_m = \frac{r}{(1-r)^2}\label{eq:2}
  \end{align}
  となる.$\cdots$(答)

  \vspace{10pt}
  次に,$T_n$を考える.
  $T_n-rT_n$を計算すると
  \begin{align*}
    T_n -rT_n
     & = \frac{1}{2}\sum_{k=1}^n k(k+1) \cdot r^k - \frac{1}{2}\sum_{k=1}^n k(k+1) \cdot r^{k+1}                       \\
     & = \frac{1}{2}\sum_{k=1}^n k(k+1) \cdot r^k - \frac{1}{2}\sum_{k=2}^{n+1} (k-1)k \cdot r^{k}                     \\
     & = \frac{1}{2}\sum_{k=1}^n k(k+1) \cdot r^k - \frac{1}{2}\sum_{k=1}^{n} (k-1)k \cdot r^{k} - \frac{n(n+1)}{2}r^n \\
     & = \frac{1}{2}\sum_{k=1}^n \left[k(k+1)- (k-1)k \right] \cdot r^k - \frac{n(n+1)}{2}r^n                          \\
     & = \sum_{k=1}^n k \cdot r^k - \frac{n(n+1)}{2}r^n                                                                \\
     & = S_n - \frac{n(n+1)}{2}r^n
  \end{align*}
  だから,
  \begin{align*}
    T_n = \frac{1}{1-r}\left[ S_n - \frac{n(n+1)}{2}r^n  \right]
  \end{align*}
  である.(1)から,二項目は$n\to\infty$で$0$に収束する.従って,\cref{eq:2}から
  \begin{align*}
    \lim_{n\to\infty}T_n
     & = \frac{1}{1-r}\lim_{n\to\infty}S_n \\
     & = \frac{r}{(1-r)^3}
  \end{align*}
  が求める極限値である.$\cdots$(答)

  \vspace{10pt}
  {\bf[解説]}

  \newpage
\end{multicols}
\end{document}