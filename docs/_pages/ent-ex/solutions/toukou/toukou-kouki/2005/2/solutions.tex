\documentclass[a4paper,10pt]{ltjsarticle}
\usepackage{luatexja}
\usepackage[hiragino-pron]{luatexja-preset}

\usepackage[truedimen,top=25truemm,bottom=20truemm,left=15truemm,right=15truemm]{geometry}
\setlength{\textwidth}{54\zw}
\setlength{\textheight}{73\zw}

\usepackage{amsmath,amssymb,ascmac}
\usepackage{enumerate}
\usepackage{multicol}
\usepackage{cleveref}
\usepackage{framed}
\usepackage{fancyhdr}
\usepackage{latexsym}
\usepackage{mathtools}
\usepackage{tikz}
\usepackage{tikz-3dplot}
\usepackage{pgfplots}
 \pgfplotsset{compat=1.18}
\usetikzlibrary{math}
 \usetikzlibrary{calc}
 \usetikzlibrary{angles}
 \usetikzlibrary{quotes}
 \usepgfplotslibrary{fillbetween} 
 \usetikzlibrary{patterns}
% \usepackage{indent}
\usepackage{caption}
\usepackage{subcaption}
\usepackage{cases}
\usepackage{float}
\usepackage{cases}
  \usepackage{caption}
  \usepackage[subrefformat=parens]{subcaption}

\usepackage{tikz}
\usetikzlibrary{arrows.meta}

\allowdisplaybreaks
\tdplotsetmaincoords{70}{110}
\pagestyle{fancy}
\lhead{}
\chead{}
\rhead{東工大後期$2005$年$2$番}
\begin{document}
\begin{oframed}
  $C$ を半径$1$の円とし,その周上に長さ$\theta$の円弧$PQ$をおく.$C$と$P$で接し$C$の内部にある円を$A$, $C$と$Q$で接し, $A$にも接する円を$B$とする.
  \begin{enumerate}
    \item $A$ と $B$ の面積の和の最小値 $S_\theta$ を $\theta$ で表せ.
    \item $\theta$ が $0$ から $2\pi$ まで動くとき,$S_\theta$ の最大値を求めよ.
  \end{enumerate}
\end{oframed}


\setlength{\columnseprule}{0.4pt}
\begin{multicols}{2}
  {\bf[解]}
  座標平面で考える.
  円$C$を原点中心とし,$x^2+y^2=1$とおく.
  $P(1,0)$, $Q(\cos\theta, \sin \theta)$とおく.
  対称性より
  \begin{align*}
    0 \le \theta < \pi
  \end{align*}
  で考える.

  \vspace{10pt}
  (1)
  円 $A,B$ の半径 $r_A, r_B$,中心 $O_A, O_B$ とする.
  まず,点$P$で$A$と$C$が接するから,$O_A$は線分OP上にある.
  同様に点$Q$で$B$と$C$が接するから,$O_B$は線分OQ上にある.
  従って,これらの円の概要は図のようになる.

  \begin{figure}[H]
    \centering
    \begin{tikzpicture}[scale=2.0]
      % Define the coordinates for centers and radii based on the derived values
      % Outer circle radius R = 1, center O = (0,0)
      % Circle A: radius r_A = 0.5, center O_A = (0.5,0)
      % Circle B: radius r_B = 1/3, center O_B = (0, 2/3)

      % Draw the axes
      \draw[->] (-1.5,0) -- (1.5,0) node[right] {$x$};
      \draw[->] (0,-1.5) -- (0,1.5) node[above] {$y$};

      % Draw the outer circle (x^2+y^2=1)
      \draw[thick] (0,0) circle (1);

      % Draw Circle A
      \coordinate (OA) at (0.6,0);
      \node[below right] at (OA) {$O_A$};
      \draw[thick] (OA) circle (0.4);
      \node[above right, xshift=0.2cm, yshift=0.2cm] at (0.5,0) {A}; % Label for circle A

      % Draw Circle B
      % \theta=3pi/4のとき
      % r_B = (1-r_A)/(1+tr_A)
      % ただしt = (2-sqrt(2))/(2+sqrt(2))= 0.1716
      % r_A = 0.4とすると
      % r_B = 0.5615
      \coordinate (OB) at (-0.707*0.4385, 0.707*0.4385);
      \node[below left] at (OB) {$O_B$};
      \draw[thick] (OB) circle (0.5615);
      \node[above left, xshift=-0.2cm, yshift=0.2cm] at (0, 0.5) {B}; % Label for circle B
      \node[] at (0.1, 0.1) {$\theta$};

      % Mark points P and Q
      \coordinate (P) at (1,0);
      \fill (P) circle (1pt) node[right] {P};

      % Q is the tangency point of circle B with the outer circle.
      % Since O_B is (0, 2/3) and r_B = 1/3, Q is (0, 2/3 + 1/3) = (0,1)
      % theta = 3\pi/4 
      \coordinate (Q) at (-0.707,0.707); % 1/sqrt(2)
      \fill (Q) circle (1pt) node[above left] {Q};

      % Draw the centers of the small circles
      \fill (OA) circle (1pt);
      \fill (OB) circle (1pt);

      % Draw dashed lines
      \draw[dashed] (0,0) -- (OA);
      \draw[dashed] (0,0) -- (OB);
      \draw[dashed] (OA) -- (OB);
      \draw[dashed] (Q) -- (OB); % Line from Q to O_B

      % Draw the angle theta (from x-axis to O_B)
      % Since O_B is on the y-axis in this specific diagram, theta is 90 degrees.
      % The diagram shows an arc for theta, so let's draw it from x-axis to O_B line
      \draw[dashed] (P) -- (OA); % Tangency line to P from O_A

      % \pic ["$\theta$", draw=black, angle eccentricity=1.2, angle radius=0.7cm] {angle = (1,0)-- (0,0) -- (0,1)};

    \end{tikzpicture}
    \caption{円$A,B,C$の概形}
  \end{figure}

  以下,円$A$と$B$が接することから,$r_A$および$r_B$に関する条件を導く.
  まず,
  \begin{align*}
    0 < r_A, r_B < 1
  \end{align*}
  のもとで,題意の条件より
  \begin{align}
    \begin{dcases}
      \overline{OO_A}    = 1-r_A \\
      \overline{OO_B}    = 1-r_B \\
      \overline{O_AO_B} = r_A+r_B
    \end{dcases}\label{eq:1}
  \end{align}
  である.
  $\theta\neq\pi$ の時 $\triangle O_A O O_B$ に余弦定理を用いて,\cref{eq:1}から
  \begin{align}
               & \overline{O_A O_B}^2 = \overline{OO_A}^2 + \overline{OO_B}^2 -2 \overline{OO_A}\overline{OO_B}\cos\theta \nonumber \\
    \therefore & (r_A+r_B)^2 = (1-r_A)^2 + (1-r_B)^2 - 2(1-r_A)(1-r_B)\cos\theta                                          \nonumber \\
    \therefore & 2r_A r_B = -2(r_A+r_B) + 2 - 2(1-r_A)(1-r_B)\cos\theta \label{eq:2}
  \end{align}
  を得る.
  一方$\theta=\pi$のとき,$O,O_A,O_B$は$x$軸上にあり,
  \begin{align*}
    r_A+r_B = 1
  \end{align*}
  となるが,これは\cref{eq:2}に内包される.そこで以下,$0<\theta\le\pi$で\cref{eq:2}を考える.

  新しい変数$\alpha,\beta$を
  \begin{align*}
    \alpha & =r_A+r_B \\
    \beta  & =r_A r_B
  \end{align*}
  とおくと,\cref{eq:2}は
  \begin{align}
    \beta               & = -\alpha+1 - (1+\beta-\alpha)\cos\theta \nonumber \\
    (1+\cos\theta)\beta & = (1-\cos\theta)(1-\alpha) \label{eq:3}
  \end{align}
  となる.

  また,$r_A, r_B$ は $x$ の2次方程式 $f(x)=x^2-\alpha x+\beta=0$ の $0<x<1$ をみたす2実解(重解含む)だから,
  このような$r_A,r_B$が存在するような$\alpha,\beta$の条件として
  \begin{align}
     & \begin{dcases}
         \text{端点:}  & f(0)>0, f(1)>0           \\
         \text{判別式:} & D\ge 0                   \\
         \text{軸:}   & 0 < \frac{\alpha}{2} < 1
       \end{dcases}       \nonumber          \\
    \therefore
     & \begin{dcases}
         \text{端点:}  & \beta \ge 0, 1-\alpha+\beta \ge 0 \\
         \text{判別式:} & \alpha^2-4\beta \ge 0             \\
         \text{軸:}   & 0 < \frac{\alpha}{2} < 1
       \end{dcases} \nonumber \\
    \therefore
     & \begin{dcases}
          & 0 < \beta, \alpha-1 < \beta   \\
          & \beta \le \frac{1}{4}\alpha^2 \\
          & 0 < \alpha < 2
       \end{dcases} \label{eq:4}
  \end{align}
  が課せられる.

  \cref{eq:3,eq:4}を$\alpha\beta$平面に図示すると,\cref{fig:2,fig:3}の太線部となる.

  \begin{figure}[H]
    \centering
    \begin{tikzpicture}
      \begin{axis}[
          xlabel={$\alpha$}, % x軸ラベル
          ylabel={$\beta$}, % y軸ラベル
          xmin=0, xmax=2.5, % x軸の表示範囲
          ymin=-0.2, ymax=1.5, % y軸の表示範囲 (少し0以下も表示)
          axis lines=middle, % 軸を中央に配置
          axis on top, % 軸の線をプロットの上に表示
          enlargelimits={0.1,0.1}, % 軸の表示範囲に少し余白を追加
          xtick={0,1,2}, % x軸の目盛り
          ytick={0,1}, % y軸の目盛り
          clip=false
        ]

        % 放物線 beta = 1/4 * alpha^2 を描画
        \addplot[
          domain=0:2.5, % 描画範囲
          samples=100, % サンプル数 (滑らかな曲線のため)
          thick,       % 線の太さ
          blue,        % 線の色
        ] {0.25*x^2};

        % 直線 beta = alpha - 1 を描画
        \addplot[
          domain=0.5:2.5, % 描画範囲
          samples=2,   % 直線なので2点でも十分
          thick,       % 線の太さ
          red,         % 線の色
        ] {x-1};

        % 指定された領域(太線部)を塗りつぶす
        % 境界は (0,0), (1,0), (2,1) の直線部分と、
        % (2,1) から (0,0) へ向かう放物線 beta = 1/4 * alpha^2 の曲線部分で構成される
        \addplot[
          fill=gray!30, % 塗りつぶしの色 (薄い灰色)
          draw=black,   % 塗りつぶされた領域の輪郭線
          thick,        % 輪郭線の太さ
          line cap=round, % 線の端のスタイル
          line join=round, % 線の結合部のスタイル
        ] coordinates {(0,0) (1,0) (2,1)} % (0,0) -> (1,0) -> (2,1) へ直線を描画
        -- plot[domain=2:0, samples=50] (\x, {0.25*\x^2}) % (2,1) から (0,0) へ放物線を描画(逆順にプロット)
        -- cycle; % パスを閉じて塗りつぶしを完了

        % 領域の主要な交点をマーク
        \fill (axis cs:0,0) circle (2pt); %node[below left] {(0,0)};
        \fill (axis cs:1,0) circle (2pt); %node[below] {(1,0)};
        \fill (axis cs:2,1) circle (2pt); %node[above right] {(2,1)};
        \fill (axis cs:0.828,0.172) circle (2pt); %node[above right] {(2,1)};

        % 直線 beta = tan^2(theta/2) * (1-alpha) の例を描画
        % ケース1: theta = pi/2 の場合 -> tan^2(pi/4) = 1 -> beta = 1-alpha
        \addplot[
          domain=0:1.5, % この直線は(0,1)から(1,0)を通る
          samples=2,
          dashed,
          thick,  % 太さ
        ] {1-x};

        \addplot[
          domain=0.828:1, % 2(\sqrt(2)-1):1
          samples=2,
          ultra thick,  % 太さ
        ] {1-x};
        % 直線上にラベルを配置 
        \node[] at (axis cs:1.0, 0.8) {$\beta=t^2(1-\alpha)$};

        \draw[dashed] (2,0) -- (2,1) -- (0,1);

      \end{axis}
    \end{tikzpicture}
    \caption{$\theta\neq \pi$の時に$r_A,r_B$が存在するための$\alpha,\beta$の条件}
    \label{fig:2}
  \end{figure}

  \begin{figure}[H]
    \centering
    \begin{tikzpicture}
      \begin{axis}[
          xlabel={$\alpha$}, % x軸ラベル
          ylabel={$\beta$}, % y軸ラベル
          xmin=0, xmax=2.5, % x軸の表示範囲
          ymin=-0.2, ymax=1.5, % y軸の表示範囲 (少し0以下も表示)
          axis lines=middle, % 軸を中央に配置
          axis on top, % 軸の線をプロットの上に表示
          enlargelimits={0.1,0.1}, % 軸の表示範囲に少し余白を追加
          xtick={0,1,2}, % x軸の目盛り
          ytick={0,1}, % y軸の目盛り
          clip=false
        ]

        % 放物線 beta = 1/4 * alpha^2 を描画
        \addplot[
          domain=0:2.5, % 描画範囲
          samples=100, % サンプル数 (滑らかな曲線のため)
          thick,       % 線の太さ
          blue,        % 線の色
        ] {0.25*x^2};

        % 直線 beta = alpha - 1 を描画
        \addplot[
          domain=0.5:2.5, % 描画範囲
          samples=2,   % 直線なので2点でも十分
          thick,       % 線の太さ
          red,         % 線の色
        ] {x-1};

        % 指定された領域(太線部)を塗りつぶす
        % 境界は (0,0), (1,0), (2,1) の直線部分と、
        % (2,1) から (0,0) へ向かう放物線 beta = 1/4 * alpha^2 の曲線部分で構成される
        \addplot[
          fill=gray!30, % 塗りつぶしの色 (薄い灰色)
          draw=black,   % 塗りつぶされた領域の輪郭線
          thick,        % 輪郭線の太さ
          line cap=round, % 線の端のスタイル
          line join=round, % 線の結合部のスタイル
        ] coordinates {(0,0) (1,0) (2,1)} % (0,0) -> (1,0) -> (2,1) へ直線を描画
        -- plot[domain=2:0, samples=50] (\x, {0.25*\x^2}) % (2,1) から (0,0) へ放物線を描画(逆順にプロット)
        -- cycle; % パスを閉じて塗りつぶしを完了

        % 領域の主要な交点をマーク
        \fill (axis cs:0,0) circle (2pt); %node[below left] {(0,0)};
        \fill (axis cs:1,0) circle (2pt); %node[below] {(1,0)};
        \fill (axis cs:2,1) circle (2pt); %node[above right] {(2,1)};
        \fill (axis cs:1,1/4) circle (2pt); %node[above right] {(2,1)};

        \draw[dashed] (2,0) -- (2,1) -- (0,1);
        \draw[dashed] (1,-0.2) -- (1,1);
        \draw[ultra thick] (1,0) -- (1,1/4);
        \node[] at (axis cs:1.0, 0.8) {$\alpha=1$};
      \end{axis}
    \end{tikzpicture}
    \caption{$\theta=\pi$の時に$r_A,r_B$が存在するための$\alpha,\beta$の条件}
    \label{fig:3}
  \end{figure}

  これは数式に起こすと$t = \tan\frac{\theta}{2}$,$c=\cos\frac{\theta}{2}$, $s=\sin\frac{\theta}{2}$ として, として
  \begin{align}
    \begin{dcases}
      0 < \theta < \pi \text{の時.} & \beta = t^2(1-\alpha) \quad \left(2\frac{(1-s)t}{c} \le \alpha < 1\right) \\
      \theta = \pi \text{の時.}     & \alpha=1, 0 < \beta \le \frac{1}{4}
    \end{dcases}\label{eq:6}
  \end{align}
  である.


  さて,$A$ と $B$ の面積和を $T$ として,
  \begin{align}
    T = \pi(r_A^2+r_B^2) = \pi(\alpha^2-2\beta) \label{eq:5}
  \end{align}
  だから,\cref{eq:6}の条件のもとで,\cref{eq:5}を最小化する.

  \subsection{$0 < \theta < \pi$ の時}

  \cref{eq:5}に\cref{eq:6}を代入して$\beta$を除去すると
  \begin{align*}
    \frac{T}{\pi}
     & = \alpha^2-2\beta          \\
     & = \alpha^2-2t^2(1-\alpha)  \\
     & = \alpha^2+2t^2\alpha-2t^2 \\
     & = (\alpha+t^2)^2-t^4-2t^2
  \end{align*}
  で $t>0$ と\cref{eq:6}から
  \begin{align*}
    \alpha = \frac{2(1-s)t}{c}
  \end{align*}
  で $T$ は最小で,この時の最小値は
  \begin{align}
    \frac{\min T}{\pi}
     & = \left[ 4\frac{(1-s)^2}{c^2} + 4\frac{s(1-s)}{c^2} - 2 \right] t^2 \nonumber \\
     & = \frac{2(s-1)^2 s^2}{c^4}                                          \nonumber \\
     & = \frac{2(1-s)^2 s^2}{(1-s^2)^2}                                    \nonumber \\
     & = \frac{2s^2}{(1+s)^2} \label{eq:7}
  \end{align}
  を得る.

  \subsection{$\theta=\pi$ の時}
  \cref{eq:6}から$\min T$ は
  $(\alpha,\beta)=(1,\frac{1}{4})$ の時の $\frac{1}{2}$ で,この時も\cref{eq:7}で良い.


  又,$\pi \le \theta \le 2\pi$ の時も対称性よりこれで良いので,($0 \to 2\pi-\theta$ ととりかえて同じになる)
  求める最小値は$0<\theta<2\pi$に対して
  \begin{align*}
    S_\theta
     & = 2\pi \left(\frac{\sin\frac{\theta}{2}}{1+\sin\frac{\theta}{2}}\right)^2
  \end{align*}
  である.$\cdots$(答)

  \vspace{10pt}
  (2) $x=1+s$ とすると,$0\le x < 2\pi$で$0 \le x \le 2$ であり,
  \begin{align*}
    S_\theta
     & = 2\pi \left(\frac{x-1}{x}\right)^2 \\
     & = 2\pi \left(1-\frac{1}{x}\right)^2
  \end{align*}
  で.これば $x=2$ すなわち $\theta=\pi$ の時最大値
  \begin{align*}
    S_\theta = \frac{\pi}{2}
  \end{align*}
  をとる.$\cdots$(答)


  \vspace{10pt}
  {\bf[解説]}
  (1)はもっと直接的に,$O_A$および$O_B$の座標を置いても良さそう.
  \newpage
\end{multicols}
\end{document}