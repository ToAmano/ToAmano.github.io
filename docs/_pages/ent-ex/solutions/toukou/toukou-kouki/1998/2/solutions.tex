\documentclass[a4paper,10pt]{ltjsarticle}
\usepackage{luatexja}
\usepackage[hiragino-pron]{luatexja-preset}

\usepackage[truedimen,top=25truemm,bottom=20truemm,left=15truemm,right=15truemm]{geometry}
\setlength{\textwidth}{54\zw}
\setlength{\textheight}{73\zw}

\usepackage{amsmath,amssymb,ascmac}
\usepackage{enumerate}
\usepackage{multicol}
\usepackage{cleveref}
\usepackage{framed}
\usepackage{fancyhdr}
\usepackage{latexsym}
\usepackage{mathtools}
\usepackage{tikz}
\usepackage{tikz-3dplot}
\usepackage{pgfplots}
 \usetikzlibrary{math}
 \usetikzlibrary{calc}
 \usetikzlibrary{angles}
 \usetikzlibrary{quotes}
 \usepgfplotslibrary{groupplots}
 \usetikzlibrary{patterns}
 \usetikzlibrary{intersections, arrows.meta}
 \usepgfplotslibrary{fillbetween} % https://tex.stackexchange.com/questions/164991/pgfplots-how-to-fill-bounded-area-under-a-curve-using-addplot-and-fill
\usepackage{multirow}
\usepackage{url}
% \usepackage{indent}
\usepackage{caption}
\usepackage{subcaption}
\usepackage{cases}
\usepackage{float}
\usepackage{cases}
  \usepackage{caption}
  \usepackage[subrefformat=parens]{subcaption}
\allowdisplaybreaks
\tdplotsetmaincoords{70}{110}
\pagestyle{fancy}
\lhead{}
\chead{}
\rhead{東工大後期$1998$年$2$番}
\begin{document}
\begin{oframed}
  $yz$ 平面の直線 $y=z$ を $l_1$, 直線 $y=z+\sqrt{2}$ を $l_2$ とする.$xyz$ 空間において $l_1$ を
  軸にして $l_2$ を回転してできる円柱面(内部は含まない)を $C$ とする.さらに $z$ 軸を軸と
  して $C$ を回転してできる回転体 $R$ とする.
  \begin{enumerate}
    \item $xy$ 平面で $C$ を切った切り口に現れる楕円の方程式を求めよ.
    \item $R$ の $yz$ 平面による断面を図示せよ.
    \item $R$ の $-2 \le z \le 2$ の部分の体積を求めよ.
  \end{enumerate}
\end{oframed}


\setlength{\columnseprule}{0.4pt}
\tdplotsetmaincoords{70}{100}
\begin{multicols}{2}
  {\bf[解]}

  \begin{figure}[H]
    \centering
    \begin{tikzpicture}[tdplot_main_coords,scale=1.5]

      % 座標軸の描画
      \draw[->] (0,0,0) -- (3,0,0) node[anchor=north east]{$x$};
      \draw[->] (0,0,0) -- (0,3,0) node[anchor=north west]{$y$};
      \draw[->] (0,0,0) -- (0,0,3) node[anchor=south]{$z$};

      % 直線 z = k (x = 0) - yz平面内の水平線
      \draw[dashed] (0,0,1.5) -- (0,3,1.5);
      \node at (0,-0.5,1.5) [anchor=south east] {$z = k$};

      % 直線 z = y (x = 0) - yz平面内の対角線
      \draw[thick] (0,-1,-1) -- (0,3,3);
      \node at (0,2.0,1.5) [anchor=south west] {$l_1: z = y$};

      % 直線 z = y + √2 (x = 0) - yz平面内の平行線
      \draw[thick] (0,-1,-1+1.414) -- (0,3,4.414);
      \node at (0,-1,2.0) [anchor=south west] {$l_2: z = y + \sqrt{2}$};

      % 直線 z = y - √2 (x = 0) - yz平面内の平行線
      \draw[dotted] (0,0,-1.414) -- (0,3,3-1.414);

      % yz平面の一部をうっすらと表示(オプション)
      \fill[gray!20,opacity=0.3] (0,0,0) -- (0,3,0) -- (0,3,4.5) -- (0,0,4.5) -- cycle;

      % 楕円
      \draw[dashed,red] plot[domain=0:360,samples=100]
      ({cos(\x)},{1.5+1.414*sin(\x)},{1.5});

      % 補助線(点線)
      %\draw[dashed,thin] (0,0,0) -- (0,0,1.5);
      %\draw[dashed,thin] (0,0,1.5) -- (0,3,1.5);
      %\draw[dashed,thin] (0,0,1.414) -- (0,3,1.414);

      % 座標軸の目盛り
      \foreach \x in {1,2,3}
      \draw (\x,0,0) -- (\x,0.1,0);
      \foreach \y in {1,2,3}
      \draw (0,\y,0) -- (0.1,\y,0);
      \foreach \z in {1,2,3}
      \draw (0,0,\z) -- (0.1,0,\z);

    \end{tikzpicture}
    \caption{$l_1$,$l_2$の様子}
    \label{fig:1}
  \end{figure}

  (1)
  まず、$Z = k$ での $C$ の切断面を求める.(1)の答えは最後に$k=0$とすれば求まる.
  対称性から $0 \le k$ とする.
  $l_1$と$l_2$の様子を\cref{fig:1}に示す.
  $l_1$ と $l_2$ の距離が $1$ だから, $C$ 上の点 $P(X,Y,Z)$ のみたす条件は,
  $P$ と $l_1, l_2$ の距離が $1$ であること.
  $P$ から $l_1$ に下ろした垂足 $H$ とすると,
  $\vec{OH}$ は $\vec{OP}$ の $l_1$ への射影だから,
  $l_1$の方向ベクトルを$\vec{n}=(0,1,1)$ として
  \begin{align}
    \vec{OH}
     & = \frac{\vec{OP} \cdot \vec{n}}{|\vec{n}|^2} \vec{n}, \quad |\vec{OH}| \nonumber \\
    \therefore
    \left|\vec{OH}\right|
     & = \frac{|\vec{OP}\cdot\vec{n}|}{|\vec{n}|}      \nonumber                        \\
     & = \frac{(Y+Z)^2}{2} \label{eq:1}
  \end{align}
  である.$\triangle OPH$ にピタゴラスの定理を用いて\cref{eq:1}を代入すると
  \begin{align*}
    PH^2 = OP^2 - OH^2 = (X^2+Y^2+Z^2) - \frac{(Y+Z)^2}{2}
  \end{align*}
  これが $1$ に等しいので,
  \begin{align}
    2(X^2+Y^2+Z^2) - (Y+Z)^2 = 2 \label{eq:2}
  \end{align}
  を得る.これが$C$ の方程式である.

  $Z=k$での切断面は,\cref{eq:2}に$Z=k$を代入して
  \begin{align}
    2X^2 + Y^2 - 2kY + k^2 - 2 = 0 \nonumber \\
    2X^2 + (Y-k)^2 = 2 \label{eq:3}
  \end{align}
  という楕円である. 図示すると\cref{fig:2}のようになる.

  \begin{figure}[H]
    \centering
    \begin{tikzpicture}
      \begin{groupplot}[
          group style={
              group size=2 by 1,
              horizontal sep=0.5cm,
            },
          width=6cm,
          height=6cm,
          axis lines=middle,
          xlabel={$x$},
          ylabel={$y$},
          xmin=-2.5, xmax=2.5,
          ymin=-1.5, ymax=5.0,
          grid= none,
          xticklabel=\empty,
          yticklabel=\empty,
          % grid style={line width=0.1pt, draw=gray!30},
          % tick align=outside,
          enlargelimits={abs=0.2},
        ]
        \nextgroupplot[title={$k \ge \sqrt{2} $}]

        % 左側のプロット (k = 1)
        % \nextgroupplot[title={$k $}]
        % 楕円 2x^2 + (y-k)^2 = 2
        \def\k{2}  % k の値
        \addplot[thick, blue, samples=100, domain=-1:1] {\k + sqrt(2 - 2*x^2)};
        \addplot[thick, blue, samples=100, domain=-1:1] {\k - sqrt(2 - 2*x^2)};

        % 重要な点をプロット
        \addplot[mark=*, mark size=2pt, blue] coordinates {(0,\k)};
        \addplot[mark=*, mark size=2pt, blue] coordinates {(0,{\k+sqrt(2)})};
        \addplot[mark=*, mark size=2pt, blue] coordinates {(0,{\k-sqrt(2)})};
        \addplot[mark=*, mark size=2pt, blue] coordinates {(1,\k)};
        \addplot[mark=*, mark size=2pt, blue] coordinates {(-1,\k)};
        \node[anchor=south] at (axis cs:0,{\k+sqrt(2)}) {$k+\sqrt{2}$};
        \node[anchor=north] at (axis cs:0,{\k-sqrt(2)}) {$k-\sqrt{2}$};


        % 右側のプロット (k = 0)
        \nextgroupplot[title={$k \le \sqrt{2} $}]
        \def\k{1/2}  % k の値
        % 楕円 2x^2 + y^2 = 2
        \addplot[thick, red, samples=100, domain=-1:1] {\k+sqrt(2 - 2*x^2)};
        \addplot[thick, red, samples=100, domain=-1:1] {\k-sqrt(2 - 2*x^2)};

        % 重要な点をプロット
        \addplot[mark=*, mark size=2pt, red] coordinates {(0,\k)};
        \addplot[mark=*, mark size=2pt, red] coordinates {(0,{\k+sqrt(2)})};
        \addplot[mark=*, mark size=2pt, red] coordinates {(0,{\k-sqrt(2)})};
        \addplot[mark=*, mark size=2pt, red] coordinates {(1,\k)};
        \addplot[mark=*, mark size=2pt, red] coordinates {(-1,\k)};
        \node[anchor=south] at (axis cs:0,{\k+sqrt(2)}) {$k+\sqrt{2}$};
        \node[anchor=north] at (axis cs:0,{\k-sqrt(2)}) {$k-\sqrt{2}$};

      \end{groupplot}
    \end{tikzpicture}
    \caption{楕円: $2x^2 + (y-k)^2 = 2$の様子}
    \label{fig:2}
  \end{figure}

  従って,\cref{eq:3}に$k=0$を代入して,$xy$平面での断面は
  \begin{align*}
    2x^2+y^2=2
  \end{align*}
  である.$\cdots$(答)


  \vspace{10pt}
  (2)
  次に,$C$を$z$軸周りに回転させた$R$の方程式について考える.
  (2)の答えは,最後に$x=0$を代入すれば求まる.

  $C$ の $z=k$ での切断面 $C_k$ を原点$O_{k}(0,0,k)$中心に回転させたものが$R$である.
  従って,$C_k$上の点で, 原点からの距離が最大の点を $A_k$, 最小の点 $B_k$ とおくと,
  この平面での回転体$R$の範囲は
  \begin{align*}
    O_KB_k^2 \le x^2+y^2 \le O_kA_k^2
  \end{align*}
  であり,面積 $S_k$ は
  \begin{align}
    S_k = \pi (O_kA_k^2 - O_kB_k^2)\label{eq:4}
  \end{align}
  で与えられる.$R$の様子を\cref{fig:3}に示す.


  \begin{figure}[H]
    \centering
    \begin{tikzpicture}[scale=0.8]
      % 座標軸
      \draw[->] (-5,0) -- (5,0) node[right] {$x$};
      \draw[->] (0,-5) -- (0,5) node[above] {$y$};

      % 原点
      \fill (0,0) circle (1pt);
      \node[below left] at (0,0) {$O$};

      % 楕円 2x^2 + (y-k)^2 = 2
      % 標準形: x^2/1 + (y-k)^2/2 = 1
      % a = 1, b = √2, 中心 (0, k)
      \def\k{2}  % k の値
      \draw[thick] (0,\k) ellipse (1 and 1.414);

      \fill[gray!30, even odd rule, opacity=0.3]
      (0,0) circle ({\k+sqrt(2)})
      (0,0) circle ({\k-sqrt(2)});


      % 軸上の点にラベル
      \draw[fill=black] (0,\k) circle (1pt);
      \node[left] at (0,\k) {$k$};

      % y軸上の重要な点
      \draw[fill=black] (0,\k+1.414) circle (1pt);
      \node[above] at (0,\k+1.414) {$k+\sqrt{2}$};

      \draw[fill=black] (0,\k-1.414) circle (1pt);
      \node[above] at (0,\k-1.414) {$k-\sqrt{2}$};

      % 回転体
      \draw (0,0) circle ({\k+sqrt(2)});
      \draw (0,0) circle ({\k-sqrt(2)});

      % 座標軸の目盛り
      \foreach \x in {-2,-1,1,2} {
          \draw (\x,0.05) -- (\x,-0.05);
        }
      \foreach \y in {-1,1,2} {
          \draw (0.05,\y) -- (-0.05,\y);
        }

    \end{tikzpicture}
    \caption{回転体$R$の$z=k$での断面}
    \label{fig:3}
  \end{figure}

  $k \ge \sqrt{2}$ の時は明らかに $A_k(0, k+\sqrt{2}), B_k(0, k-\sqrt{2})$ である.
  以下 $0 \le k \le \sqrt{2}$ の時を考える.
  $C_k$ 上の点 $Q$ は, $Q(\cos\theta, k+\sqrt{2}\sin\theta)$ とおける.
  ただし$-\pi \le \theta \le \pi$ とする.
  この時の$OQ$を考えると
  \begin{align*}
    \overline{\text{OQ}}^2
     & = \cos\theta^2+2\sin\theta^2+\sqrt{2}k\sin\theta+k^2 \\
     & = \sin\theta^2+2\sqrt{2}k\sin\theta+k^2+1            \\
     & = (\sin\theta+\sqrt{2}k)^2+1-k^2
  \end{align*}
  であり,$-1 \le \sin\theta \le 1$だから,
  $\overline{\text{OQ}}^2$の最小値,最大値は$k$によって以下のようになる.
  まず最小値は
  \begin{align}\label{eq:5}
    \begin{dcases}
      0 \le k \le \frac{\sqrt{2}}{2}, & \min \overline{\text{OQ}}^2 = 1-k^2, (\sin\theta=-\sqrt{2}k)          \\
      \frac{\sqrt{2}}{2} < k \le 1,   & \min \overline{\text{OQ}}^2 = (k-\sqrt{2})^2, (\sin\theta>-\sqrt{2}k)
    \end{dcases}
  \end{align}
  であり,一方最大値は$\theta=\pi/2$のときで,$A_k(0, k+\sqrt{2})$で一定である.
  この時$\max \overline{\text{OQ}}^2 = \overline{\mathrm{OA_k}}^2 = (k+\sqrt{2})^2$である.
  したがって, \cref{eq:4,eq:5}から$Z=k$での$R$の領域及び面積$S_k$は以下のようになる.
  \begin{align}
     & R =      \nonumber                                                                                                               \\
     & \begin{dcases}
         1-k^2 \le x^2+y^2 \le (k+\sqrt{2})^2             & 0 \le k \le \frac{\sqrt{2}}{2} \\
         (k-\frac{1}{2})^2 \le x^2+y^2 \le (k+\sqrt{2})^2 & \frac{\sqrt{2}}{2} \le k \le 1
       \end{dcases} \label{eq:6}                                 \\
     & S_k=               \nonumber                                                                                                     \\
     & \begin{dcases}
         \pi \left\{ \left(k+\sqrt{2}\right)^2 + k^2 - 1 \right\} = \pi (2k^2+2\sqrt{2}k+1) & 0\le k\le \frac{\sqrt{2}}{2} \\
         \pi \left\{ (k+\sqrt{2})^2-(k-\sqrt{2})^2 \right\} = 4\sqrt{2}\pi k                & \frac{\sqrt{2}}{2} \le k
       \end{dcases} \label{eq:7}
  \end{align}

  従って,求める$R$の$yz$平面での断面は,\cref{eq:6}に$k=z,x=0$を代入して
  \begin{align*}
    R & =
    \begin{dcases}
      1-z^2 \le y^2 \le (z+\sqrt{2})^2          & 0 \le z \le \frac{\sqrt{2}}{2} \\
      (z-\sqrt{2})^2 \le y^2 \le (z+\sqrt{2})^2 & \frac{\sqrt{2}}{2} \le z \le 1
    \end{dcases}
  \end{align*}
  であり,図示すると\cref{fig:4}のようになる.ただし境界を含む.$\cdots$(答)

  \begin{figure}[H]
    \centering
    \begin{tikzpicture}
      \begin{axis}[
          axis lines = middle,
          xlabel = $y$,
          ylabel = $z$,
          xmin = -4, xmax = 4,
          ymin = -4, ymax = 4,
          yticklabel=\empty,
          xticklabel=\empty,
        ]

        %\fill[gray!20,opacity=0.3](1.4,0) -- ({sqrt(2)},0) -- (2,{2+sqrt(2)}) -- (2,{2-sqrt(2)}) -- cycle;
        %\filldraw [lightgray]plot[domain=0:2,smooth](1,0)--(2.9,1.9)--(2.9,0); %この範囲で塗りつぶす
        %\path[pattern=horizontal lines,pattern color=orange] plot[domain=2:sqrt(2)] ({\x},{\x-sqrt(2)}) plot[domain=sqrt(2):0] ({\x},{-\x+sqrt(2)});
        \addplot[pattern=horizontal lines] coordinates {
            (0,{sqrt(2)}) (4,{4+sqrt(2)}) (4,{4-sqrt(2)}) ({sqrt(2)},0)
          }; % fill=gray!20, fill opacity=1.0
        \addplot[pattern=horizontal lines] coordinates {
            (0,{sqrt(2)}) (-4,{4+sqrt(2)}) (-4,{4-sqrt(2)}) ({-sqrt(2)},0)
          };
        \addplot[pattern=horizontal lines] coordinates {
            (0,{-sqrt(2)}) (4,{-4-sqrt(2)}) (4,{-4+sqrt(2)}) ({sqrt(2)},0)
          };
        \addplot[pattern=horizontal lines] coordinates {
            (0,{-sqrt(2)}) (-4,{-4-sqrt(2)}) (-4,{-4+sqrt(2)}) ({-sqrt(2)},0)
          };

        %plot[domain=0:2,smooth]
        % Define the boundary curves based on the given conditions
        \addplot[name path=p,domain=-4:4, samples=100, thick, black] {x-sqrt(2)};
        \addplot[name path=q, domain=-4:4, samples=100, thick, black] {-x+sqrt(2)};
        \addplot[name path=r,domain=-4:4, samples=100, thick, black] {x+sqrt(2)};
        \addplot[name path=s,domain=-4:4, samples=100, thick, black] {-(x+sqrt(2))};

        \addplot[name path=a,domain={sqrt(2)/2}:1, samples=100, thick, black] {sqrt(1-x*x)};
        \addplot[name path=b,domain=-1:-0.707, samples=100, thick, black] {sqrt(1-x*x)};
        \addplot[name path=c,domain=0.707:1, samples=100, thick, black] {-sqrt(1-x*x)};
        \addplot[name path=d,domain=-1:-0.707, samples=100, thick, black] {-sqrt(1-x*x)};
        \addplot[domain=-0.707:0.707, samples=100, dashed,thick, black] {sqrt(1-x*x)};
        \addplot[domain=-0.707:0.707, samples=100, dashed,thick, black] {-sqrt(1-x*x)};

        % \path [name path=c] (1,0) -- ({sqrt(2)},0) -- ({sqrt(2)/2},{sqrt(2)/2});
        % \addplot [
        %   thick,
        %   fill=gray!20,
        % ]
        % fill between[ of=c and a,soft clip={domain=0.707:1.414} ];

        \addplot [pattern=horizontal lines]fill between[ of=q and a,soft clip={domain=0.707:1.414}]; % fill=gray!20
        \addplot [pattern=horizontal lines]fill between[ of=p and c,soft clip={domain=0.707:1.414}];
        \addplot [pattern=horizontal lines]fill between[ of=r and b,soft clip={domain=-1.414:-0.707}];
        \addplot [pattern=horizontal lines]fill between[ of=s and d,soft clip={domain=-1.414:-0.707}];

        ]
        % % Add labels for the boundary lines
        \node[black,font=\small,rotate=40] at (axis cs:1,3) {$z = y + \sqrt{2}$};
        \node[black,font=\small,rotate=-40] at (axis cs:-1.2,3.0) {$z = -y + \sqrt{2}$};
        \node[black,font=\small,rotate=40] at (axis cs:3,1) {$z = y - \sqrt{2}$};
        \node[black,font=\small,rotate=-40] at (axis cs:-3,1) {$z = -y - \sqrt{2}$};
        \node[above right,black,font=\small] at (axis cs:0,{sqrt(2)}) {$\sqrt{2}$};
        \node[below right,black,font=\small] at (axis cs:0,{-sqrt(2)}) {$-\sqrt{2}$};
        \node[below right,black,font=\small] at (axis cs:{sqrt(2)},0) {$\sqrt{2}$};
        \node[below left,black,font=\small] at (axis cs:{-sqrt(2)},0) {$-\sqrt{2}$};


        % Mark special points and labels
        \addplot[mark=*, mark size=2pt, black] coordinates {({sqrt(2)/2},{sqrt(2)/2})};
        \addplot[mark=*, mark size=2pt, black] coordinates {({sqrt(2)/2},{-sqrt(2)/2})};
        \addplot[mark=*, mark size=2pt, black] coordinates {({-sqrt(2)/2},{sqrt(2)/2})};
        \addplot[mark=*, mark size=2pt, black] coordinates {({-sqrt(2)/2},{-sqrt(2)/2})};

      \end{axis}
    \end{tikzpicture}
    \caption{(2)の回答の領域}
    \label{fig:4}
  \end{figure}


  \vspace{10pt}
  (3)
  対称性から,\cref{eq:7}を$0\le k \le 2$で積分したものの2倍が求める体積である.
  \begin{align}
    \frac{1}{2}V
     & = \int_0^2 S_k dk                                                                                       \nonumber \\
     & = \int_0^{\sqrt{2/2}} \pi (2k^2+2\sqrt{2}k+1)dk + \int_{\sqrt{2}/2}^{2} 4\sqrt{2}\pi k dk \label{eq:8}
  \end{align}
  である。各項積分すると
  \begin{align*}
    \int_0^{\sqrt{2}/2} (2k^2+2\sqrt{2}k+1)dk
     & = \left[ \frac{2}{3}k^3+\sqrt{2}k^2 + k \right]_0^{1/2} \\
     & = \frac{7}{6}\sqrt{2}                                   \\
    \int_{\sqrt{2}/2}^{\sqrt{2}} 4\sqrt{2}kdk
     & = 2\sqrt{2}\biggr [k^2\biggr ]_{\sqrt{2}/2}^{2}         \\
     & = 7\sqrt{2}
  \end{align*}
  だから\cref{eq:8}に代入して
  \begin{align*}
    V
     & = 2\pi \left\{ \frac{7}{6}\sqrt{2} + \frac{7}{2}\sqrt{2} \right\} \\
     & = \frac{49}{3}\sqrt{2}\pi
  \end{align*}
  が求める体積である.$\cdots$(答)

  \vspace{10pt}
  {\bf[解説]}

  \newpage
\end{multicols}
\end{document}