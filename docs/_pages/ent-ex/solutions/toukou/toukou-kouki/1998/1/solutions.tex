\documentclass[a4paper,10pt]{ltjsarticle}
\usepackage{luatexja}
\usepackage[hiragino-pron]{luatexja-preset}

\usepackage[truedimen,top=25truemm,bottom=20truemm,left=15truemm,right=15truemm]{geometry}
\setlength{\textwidth}{54\zw}
\setlength{\textheight}{73\zw}

\usepackage{amsmath,amssymb,ascmac}
\usepackage{enumerate}
\usepackage{multicol}
\usepackage{physics}
\usepackage{cleveref}
\usepackage{framed}
\usepackage{fancyhdr}
\usepackage{latexsym}
\usepackage{mathtools}
\usepackage{tikz}
\usepackage{tikz-3dplot}
\usepackage{pgfplots}
\pgfplotsset{compat=1.18}
\usetikzlibrary{intersections}
\usepgfplotslibrary{fillbetween}
 \usetikzlibrary{math}
 \usetikzlibrary{calc}
 \usetikzlibrary{angles}
 \usetikzlibrary{quotes}
% \usepackage{indent}
\usepackage{cases}
\usepackage{float}
\allowdisplaybreaks
\pagestyle{fancy}
\lhead{}
\chead{}
\rhead{東工大後期$1998$年$1$番}
\begin{document}
\begin{oframed}
実数 $a, b$ に対し $\displaystyle x_n = \frac{1}{n^b} \left\{ \frac{1}{n^a} + \frac{1}{(n+1)^a} + \dots + \frac{1}{(2n-1)^a} \right\}$, $n=1,2,3,\dots$ とおく.$n \to \infty$ のとき $x_n$ が収束するための $a, b$ の条件およびそのときの極限値を求めよ.
\end{oframed}
\setlength{\columnseprule}{0.4pt}
\begin{multicols}{2}
{\bf[解]}

まずは$x_n$の一般項を求める.
\begin{align*}
x_n 
&= \frac{1}{n^{b}} \sum_{k=0}^{n-1} \frac{1}{(n+k)^a} \\
&= \frac{1}{n^{b}} \sum_{k=n}^{2n-1} \frac{1}{\left(1+k\right)^a} \\
&= \frac{1}{n^{a+b-1}} \frac{1}{n}\sum_{k=0}^{n-1} \frac{1}{\left(1+\frac{k}{n}\right)^a} \quad \cdots \text{①}
\end{align*}
ここで
\begin{align*}
    A_n = \frac{1}{n}\sum_{k=0}^{n-1} \frac{1}{\left(1+\frac{k}{n}\right)^a}
\end{align*}
とおくと,これは区分求積の形になっており,$n\to\infty$の時以下のように収束する.
\begin{align*}
 \lim_{n\to\infty} A_n 
 &= \int_1^2 \frac{1}{x^a} dx \\
 &= 
  \begin{dcases} 
    \log 2 & (a=1) \\ 
    \frac{1}{1-a}(2^{1-a}-1) & (a \ne 1) 
  \end{dcases} \quad (\ne 0)
\end{align*}
次に$x_n$の残りの部分を考えると,
\begin{align*}
    \frac{1}{n^{a+b-1}} \xrightarrow{n \to \infty} 
    \begin{dcases} 
        0 & (a+b-1 > 0) \\ 
        1 & (a+b-1 = 0) \\ 
        \infty & (a+b-1 < 0) 
    \end{dcases} 
\end{align*}
である.以上をまとめると,$a+b-1 \ge 0$ の時収束し,その収束値は以下のようになる.
\begin{align*}
 \lim_{n\to\infty} x_n = 
 \begin{dcases}
    0  & (n \to \infty) \\
    \log 2 & (a+b=1, a=1) \\
    \frac{2^{1-a}-1}{1-a} & (a+b=1, a \ne 1)
 \end{dcases}
\end{align*}
$\cdots$(答)

{\bf[解説]}

     \newpage
\end{multicols}
\end{document}