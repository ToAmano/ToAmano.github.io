\documentclass[a4paper,10pt]{ltjsarticle}
\usepackage{luatexja}
\usepackage[hiragino-pron]{luatexja-preset}

\usepackage[truedimen,top=25truemm,bottom=20truemm,left=15truemm,right=15truemm]{geometry}
\setlength{\textwidth}{54\zw}
\setlength{\textheight}{73\zw}

\usepackage{amsmath,amssymb,ascmac}
\usepackage{enumerate}
\usepackage{multicol}
\usepackage{cleveref}
\usepackage{framed}
\usepackage{fancyhdr}
\usepackage{latexsym}
\usepackage{mathtools}
\usepackage{tikz}
\usepackage{tikz-3dplot}
\usepackage{pgfplots}
 \usetikzlibrary{math}
 \usetikzlibrary{calc}
 \usetikzlibrary{angles}
 \usetikzlibrary{quotes}
 \usepgfplotslibrary{fillbetween} 

 \pgfplotsset{compat=1.18}
% \usepackage{indent}
\usepackage{caption}
\usepackage{subcaption}
\usepackage{cases}
\usepackage{float}
\usepackage{cases}
  \usepackage{caption}
  \usepackage[subrefformat=parens]{subcaption}

\usepackage{tikz}
\usetikzlibrary{arrows.meta}

\allowdisplaybreaks
\tdplotsetmaincoords{70}{110}
\pagestyle{fancy}
\lhead{}
\chead{}
\rhead{東工大後期$2006$年$2$番}
\begin{document}
\begin{oframed}
  自然数 $a, b, c$ が
  \begin{align*}
    3a = b^3, \quad 5a = c^2
  \end{align*}
  を満たし,$d^6$ が $a$ を割り切るような自然数 $d$ は $d=1$ に限るとする.

  \begin{enumerate}
    \item $a$ は $3$ と $5$ で割り切れることを示せ.
    \item $a$ の素因数は $3$ と $5$ 以外にないことを示せ.
    \item $a$ を求めよ.
  \end{enumerate}
\end{oframed}


\setlength{\columnseprule}{0.4pt}
\begin{multicols}{2}
  {\bf[解]}
  自然数$a,b,c$に対して
  \begin{align}
    \begin{dcases}
      3a & =b^3 \\
      5a & =c^2
    \end{dcases}\label{eq:1}
  \end{align}
  とおく.

  \vspace{10pt}
  (1)
  \cref{eq:1}および$3,5$は素数であることから,$b,c$は各々$3,5$でわり切れる.
  したがって,$b'=\frac{1}{3}b$, $c'=\frac{1}{5}c ( \in \mathbb{N} )$として
  \cref{eq:1}に代入して
  \begin{align}
    \begin{dcases}
      a & =9b'^3 \\
      a & =5c'^2
    \end{dcases} \label{eq:2}
  \end{align}
  と書ける.従って$a$は$3^25^1$でわり切れる. $\cdots$(答)

  \vspace{10pt}
  (2)
  背理法を用いて示す.
  $a$の素因数として,$3,5$以外の素数 $p \in \mathbb{N}_{\ne 1}$ があると仮定する.
  すると(1)と同様に,\cref{eq:1}から
  \begin{align*}
    a & =p a'   \\
    b & =3p b'' \\
    c & =5p c''
  \end{align*}
  なる $a',b'',c'' \in \mathbb{N}$が存在する.
  \cref{eq:1}に代入すると
  \begin{align}
    a'=9p^2b''^3 = 5pc''^2 \label{eq:3}
  \end{align}
  である.

  したがって,$a'$が$p^2$でわり切れるので,$a'=p^{2}a''$ なる$a'' \in \mathbb{N}$が存在する.
  これを\cref{eq:3}に代入して
  \begin{align*}
    a''=9b''^3 = \frac{5 c''^2}{p}
  \end{align*}
  を得る.$5c''^2/p$が整数であるためには,$p$が$5$と互いに素であることから,
  $c''$が$p$を因数に持って,
  \begin{align*}
    c'' = p c'''
  \end{align*}
  なる$c'''\in\mathbb{N}$がある.\cref{eq:3}に代入すると,
  \begin{align*}
    a''=9b''^3 = 5pc'''^2
  \end{align*}
  $p$は$9$と互いに素だから,$b''$は$p$を因数に持つ.
  従って$a''$が$p^3$を因数に持つ.
  $a=p^{3}a''$だったから,結局$a$が$p^6$を因数に持つ.

  一方,題意から$\alpha^t$ ($t \in \mathbb{N}_{\ge 6}$)の形の素因数$a$はもたない.
  これは矛盾.従って背理法より,$p=1,3,5$となり,題意は示された. $\cdots$(答)


  \vspace{10pt}
  (3)
  (1), (2)から,$a=3^k \cdot 5^l$ ($k,l \in \mathbb{N}$) とおける.
  ただし
  \begin{align}
    \begin{dcases}
      2 \le k \le 5 \\
      1 \le k \le 5
    \end{dcases} \label{eq:4}
  \end{align}
  である.
  \cref{eq:2}に代入して
  \begin{align}
    a = 3^k \cdot 5^l = 9 \cdot b'^3 = 5 \cdot c'^2\label{eq:3}
  \end{align}
  だから,$b'$および$c'$も同じく$3,5$のみ素因数にもち,
  \begin{align*}
    b'=3^n \cdot 5^m, \\
    c'=3^x \cdot 5^y, \\
    n,m,x,y \in \mathbb{Z}_{\ge 0}
  \end{align*}
  と書ける.
  変数の範囲は\cref{eq:4}より
  \begin{align}
    0 \le n,m,x,y \le 5 \label{eq:5}
  \end{align}
  である.
  \cref{eq:3}に代入して
  \begin{align*}
    3^k \cdot 5^l = 3^{3n+2} \cdot 5^{3m} = 3^{2x} \cdot 5^{2y+1} \\
    \therefore
    \begin{cases}
      k=3n+2=2x \\
      l=3m=2y+1
    \end{cases}
  \end{align*}
  である.
  これをみたす$(k,l)$は\cref{eq:4,eq:5}の条件では
  $(k,l)=(2,3)$のみであり,この時
  \begin{align*}
    a=3^2 \cdot 5^3 = 1125
  \end{align*}
  である.$\cdots$(答)
  \newpage
\end{multicols}
\end{document}