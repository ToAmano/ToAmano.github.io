\documentclass[a4paper,10pt]{ltjsarticle}
\usepackage{luatexja}
\usepackage[hiragino-pron]{luatexja-preset}

\usepackage[truedimen,top=25truemm,bottom=20truemm,left=15truemm,right=15truemm]{geometry}
\setlength{\textwidth}{54\zw}
\setlength{\textheight}{73\zw}

\usepackage{amsmath,amssymb,ascmac}
\usepackage{enumerate}
\usepackage{multicol}
\usepackage{physics}
\usepackage{cleveref}
\usepackage{framed}
 \usepackage{caption}
 \usepackage{subcaption}
\usepackage{fancyhdr}
\usepackage{latexsym}
\usepackage{mathtools}
\usepackage{tikz}
\usepackage{tikz-3dplot}
\usepackage{pgfplots}
\pgfplotsset{compat=1.18}
\usetikzlibrary{intersections}
\usepgfplotslibrary{fillbetween}
 \usetikzlibrary{math}
 \usetikzlibrary{calc}
 \usetikzlibrary{angles}
 \usetikzlibrary{quotes}
 \usetikzlibrary{patterns}

% \usepackage{indent}

\usepackage{cases}
\usepackage{float}
\allowdisplaybreaks
\pagestyle{fancy}
\lhead{}
\chead{}
\rhead{東工大後期$2006$年$1$番}
\begin{document}

\begin{oframed}
  $a, b$ を正の数とする.$xy$ 座標平面において,楕円 $ax^2 + by^2 = 1$ の第 4 象限
  $(x \ge 0, y \le 0)$ に含まれる部分を $C$,傾き $t \ge 0$ の半直線 $y = tx \ (x \ge 0)$ を $l_t$ とする.
  $l_t$ 上の点 $P$ と $C$ 上の点 $P'$ を結ぶ線分 $PP'$ が $y$ 軸に平行になるように動くとき,線分
  $PP'$ の長さを最大にする $P$ を $P_t$ で表し,$t \ge 0$ が変化するときに $P_t$ が描く曲線を $C'$ と
  する.また,楕円 $ax^2 + by^2 = 1$ と $C'$ との交点を $Q(\alpha, \beta)$ とする.

  \begin{enumerate}
    \item 曲線 $C'$ の方程式 $y = f(x)$ を求めよ.
    \item $\alpha$ と $\beta$ を求めよ.
    \item 直線 $y = \beta$,曲線 $C'$ および $y$ 軸が囲む領域を $D$ とする.$D$ を $y$ 軸の回りに $1$ 回
          回転してできる回転体の体積 $V$ を求めよ.
  \end{enumerate}
\end{oframed}
\setlength{\columnseprule}{0.4pt}
\begin{multicols}{2}
  {\bf[解]}

  題意より,
  \begin{align*}
     & C: ax^2 + by^2 = 1 \quad (x \ge 0, y \le 0) \\
     & l_t: y = tx \quad (t \ge 0, x \ge 0)
  \end{align*}
  である.$l_t$,$C$上の点$P,P'$は
  $P(X,tX), P'(X,Y)$とおける.ただし
  \begin{align*}
     & 0 \le X \le \frac{1}{\sqrt{a}} \\
     & Y = -\sqrt{\frac{1 - aX^2}{b}}
  \end{align*}
  である.この様子を\cref{fig:1}に示す.

  \begin{figure}[H]
    \centering
    \begin{tikzpicture}
      \begin{axis}[
          axis lines=middle, % Draws x and y axes intersecting at the origin
          xlabel=$x$,
          ylabel=$y$,
          xmin=-0.2, xmax=1.8,
          ymin=-1.8, ymax=1.8,
          xtick=\empty, % Suppress numerical x-axis ticks
          ytick=\empty, % Suppress numerical y-axis ticks
          clip=false, % Allow nodes/labels to extend beyond the axis limits
          %xlabel style={at={(axis description cs:1,0.5)},anchor=north west}, % Position x-axis label
          %ylabel style={at={(axis description cs:0.5,1)},anchor=south east}, % Position y-axis label
        ]

        % Define parameters a, b, t. Choose values that create a similar visual.
        \pgfmathsetmacro{\aval}{0.5} % Example value for 'a' (e.g., 1/2)
        \pgfmathsetmacro{\bval}{0.5} % Example value for 'b' (e.g., 1/2)
        \pgfmathsetmacro{\tval}{1.5} % Example value for 't' for the line y=tx

        % Plot the curve C: ax^2 + by^2 = 1 in the 4th quadrant (x >= 0, y <= 0)
        % Solve for y: by^2 = 1 - ax^2  =>  y = -sqrt((1-ax^2)/b)
        % The x-domain for this curve is from 0 to 1/sqrt(a) (where y=0)
        \addplot[domain=0:sqrt(1/\aval), samples=100, blue, thick] plot (\x, {-sqrt((1-\aval*\x*\x)/\bval)});
        \node[below right] at (axis cs:1.2,-1.2) {C}; % Label for curve C

        % Plot the line lt: y = tx
        % Determine an x-coordinate for point P to ensure the line is drawn appropriately.
        \pgfmathsetmacro{\xp}{0.8} % x-coordinate for points P and P'
        \addplot[domain=0:\xp*1.2, samples=2, red, thick] plot (\x, {\tval*\x});
        \node[above right] at (axis cs:\xp*1.1, \tval*\xp*1.2) {$l_t$}; % Label for line lt

        % Calculate coordinates for P and P'
        \pgfmathsetmacro{\ypP}{\tval*\xp} % y-coordinate for P on line lt
        \pgfmathsetmacro{\ypPprime}{-sqrt((1-\aval*\xp*\xp)/\bval)} % y-coordinate for P' on curve C

        % Draw point P
        \addplot[only marks, mark=*, mark size=2pt] coordinates {(\xp, \ypP)};
        \node[above left] at (axis cs:\xp, \ypP) {P};

        \addplot[only marks, mark=*, mark size=2pt] coordinates {(sqrt(1/\aval), 0)};
        \addplot[only marks, mark=*, mark size=2pt] coordinates {(0,-1.414)};


        % Draw point P'
        \addplot[only marks, mark=*, mark size=2pt] coordinates {(\xp, \ypPprime)};
        \node[below left] at (axis cs:\xp, \ypPprime) {P'};

        % Draw the vertical dashed line connecting P and P'
        \draw[dashed, gray] (axis cs:\xp, \ypP) -- (axis cs:\xp, \ypPprime);
        \draw[dashed, gray] (axis cs:\xp, \ypP) -- (axis cs:0, \ypP);
        \draw[dashed, gray] (axis cs:0, \ypPprime) -- (axis cs:\xp, \ypPprime);


        % Label for 1/sqrt(a) on the x-axis
        \node[above] at (axis cs:{sqrt(1/\aval)},0) {$\frac{1}{\sqrt{a}}$};

        \node[left] at (axis cs:\xp,0) {$X$};
        \node[left] at (axis cs:0,\ypP) {$tX$};
        \node[left] at (axis cs:0,\ypPprime) {$Y$};


        % Label for the origin
        \node[below left] at (axis cs:0,0) {O};

      \end{axis}
    \end{tikzpicture}
    \caption{$C$および$l_t$の概形}
    \label{fig:1}
  \end{figure}




  $PP' = g(X)$とおくと,
  \begin{align*}
    g(X)
     & = tX + |Y|                       \\
     & = tX + \sqrt{\frac{1 - aX^2}{b}}
  \end{align*}
  より,一階微分は
  \begin{align*}
    g'(x)
     & = t + \frac{1}{\sqrt{b}} \frac{-2ax}{2\sqrt{1-ax^2}}                              \\
     & = \frac{t\sqrt{b(1-ax^2)}-ax}{\sqrt{b(1-ax^2)}}                                   \\
     & = \frac{t^2b(1-ax^2)-a^2x^2}{\sqrt{b(1-ax^2)}\left(t\sqrt{b(1-ax^2)}+ax\right)}   \\
     & = \frac{bt^2 - (abt^2+a^2)x^2}{\sqrt{b(1-ax^2)}\left(t\sqrt{b(1-ax^2)}+ax\right)}
    % & = \frac{(bt)^2 - (bt)^2 a^2 X^2}{b\sqrt{1-ax^2}\{\sqrt{b}t\sqrt{1-ax^2} - a x \sqrt{t}\}}
  \end{align*}
  と書ける.
  分母は常に正であり,$g'=0$となるのは
  \begin{align*}
    \alpha = \sqrt{\frac{bt^2}{abt^2 + a^2}}
  \end{align*}
  の時である.
  増減表は\cref{table:1}となる.
  \begin{table}[H]
    \centering
    \caption{$g(x)$の増減表}
    \label{table:1}
    \begin{tabular}{|c|c|c|c|c|c|}
      \hline
      $X$  & $0$          &     & $\alpha$ &     & $\frac{1}{\sqrt{a}}$ \\
      \hline
      $g'$ &              & $+$ & $0$      & $-$ &                      \\
      \hline
      $g$  & $1/\sqrt{b}$ &     &          &     & $t/\sqrt{a}$         \\
      \hline
    \end{tabular}
  \end{table}

  よって $X = \alpha$ で $g(X)$は最大だから,$P_t$の座標は
  \begin{align*}
    X & = \alpha  \\
    Y & = t\alpha
  \end{align*}
  で与えられる.

  ここから$t$を消去すれば,題意の曲線$C'$の表示を得る.
  まず,$t=0$の時, $P_0(0,0)$である.
  次に $t \ne 0$の時$X \ne 0$から
  \begin{align*}
    t = \frac{Y}{X}
  \end{align*}
  であり,$X=\alpha$ に代入して
  \begin{align*}
         & X  = \sqrt{\frac{bt^2}{abt^2 + a^2}}                         \\
         & X  = \sqrt{\frac{b(Y/X)^2}{ab(Y/X)^2 + a^2 }}                \\
    \iff & X^2 \{ ab Y^2  + a^2 X^2 \} = b Y^2 \quad (\because X \ge 0) \\
    \iff & Y^2 (b-abX^2) = a^2 X^4                                      \\
    \iff & Y = \frac{aX^2}{\sqrt{b(1-aX^2)}}
  \end{align*}
  ただし,$X=1/\sqrt{3}$の時はこの式は成り立たないから,$0 < X < \frac{1}{\sqrt{a}}$である.
  この式は$t=0$でも成立するから,求めるべき曲線$C'$の表示は
  \begin{align*}
    f(X) = \frac{aX^2}{\sqrt{b(1-aX^2)}} \quad \left(0 \le X < \frac{1}{\sqrt{a}}\right)
  \end{align*}
  である.$\cdots$(答)

  \begin{figure}[H]
    \centering
    \begin{tikzpicture}
      \begin{axis}[
          axis lines=middle, % Draws x and y axes intersecting at the origin
          xlabel=$x$,
          ylabel=$y$,
          xmin=-0.2, xmax=1.8,
          ymin=-1.8, ymax=1.8,
          xtick=\empty, % Suppress numerical x-axis ticks
          ytick=\empty, % Suppress numerical y-axis ticks
          clip=false, % Allow nodes/labels to extend beyond the axis limits
          %xlabel style={at={(axis description cs:1,0.5)},anchor=north west}, % Position x-axis label
          %ylabel style={at={(axis description cs:0.5,1)},anchor=south east}, % Position y-axis label
        ]

        % Define parameters a, b, t. Choose values that create a similar visual.
        \pgfmathsetmacro{\aval}{0.5} % Example value for 'a' (e.g., 1/2)
        \pgfmathsetmacro{\bval}{0.5} % Example value for 'b' (e.g., 1/2)
        \pgfmathsetmacro{\tval}{1.5} % Example value for 't' for the line y=tx

        % Plot the curve C: ax^2 + by^2 = 1 in the 4th quadrant (x >= 0, y <= 0)
        % Solve for y: by^2 = 1 - ax^2  =>  y = -sqrt((1-ax^2)/b)
        % The x-domain for this curve is from 0 to 1/sqrt(a) (where y=0)
        \addplot[domain=0:sqrt(1/\aval)-0.2, samples=100, red, thick] plot (\x, {\aval*x*x/sqrt(\bval*(1-\aval*x*x))});
        \addplot[domain=0:sqrt(1/\aval), samples=100, blue, thick] plot (\x, {-sqrt((1-\aval*\x*\x)/\bval)});
        \addplot[domain=0:sqrt(1/\aval), samples=100, blue, thick] plot (\x, {sqrt((1-\aval*\x*\x)/\bval)});
        \node[below right] at (axis cs:1.2,-1.2) {C}; % Label for curve C

        % Plot the line lt: y = tx
        % Determine an x-coordinate for point P to ensure the line is drawn appropriately.
        \pgfmathsetmacro{\xp}{0.8} % x-coordinate for points P and P'
        \node[above right] at (axis cs:\xp*1.1, \tval*\xp*1.2) {$C'$}; % Label for line lt

        % Calculate coordinates for P and P'
        \pgfmathsetmacro{\ypP}{\tval*\xp} % y-coordinate for P on line lt
        \pgfmathsetmacro{\ypPprime}{-sqrt((1-\aval*\xp*\xp)/\bval)} % y-coordinate for P' on curve C
        \pgfmathsetmacro{\xmax}{sqrt(1/\aval)} % y-coordinate for P on line lt

        % Draw point P
        \addplot[only marks, mark=*, mark size=2pt] coordinates {(\xmax, 0)};
        % Draw the vertical dashed line connecting P and P'
        \draw[dashed, black, thick] (axis cs:\xmax,0) -- (axis cs:\xmax,1.8);

        % Label for 1/sqrt(a) on the x-axis
        \node[above right] at (axis cs:{sqrt(1/\aval)},0) {$\frac{1}{\sqrt{a}}$};
        % Label for the origin
        \node[below left] at (axis cs:0,0) {O};

        % Draw point P'
        \pgfmathsetmacro{\alpha}{1/sqrt(2*\aval)};
        \pgfmathsetmacro{\beta}{1/sqrt(2*\bval)};
        \addplot[only marks, mark=*, mark size=2pt] coordinates {(\alpha, \beta)};
        \node[below left] at (axis cs:\alpha, \beta) {Q};
      \end{axis}
    \end{tikzpicture}
    \caption{$C$および$C'$の概形}
    \label{fig:2}
  \end{figure}


  \vspace{10pt}
  (2) $C$と$C'$の方程式の連立解を求める.
  楕円$C$の式に$y=f(x)$を代入して$y$を除去すると,
  \begin{align*}
               & ax^2 + b\frac{(ax^2)^2}{b(1-ax^2)} = 1 \\
    \therefore & ax^2(1-ax^2) + a^2x^4 = 1-ax^2         \\
    \therefore & 2ax^2 = 1                              \\
    \therefore & x = \sqrt{\frac{1}{2a}}
  \end{align*}
  であり,この時,
  \begin{align*}
    y & = f(x)                         \\
      & =\frac{a/2a}{\sqrt{b(1-a/2a)}} \\
      & = \sqrt{\frac{1}{2a}}
  \end{align*}
  である.以上から
  \begin{align*}
     & \alpha = \sqrt{\frac{1}{2a}}
     & \beta  = \sqrt{\frac{1}{2b}}
  \end{align*}
  である.$\cdots$(答)

  \vspace{10pt}
  (3)
  $f(x)$は区間内で単調増加で, グラフ概形および領域$D$は\cref{fig:3}である.

  \begin{figure}[H]
    \centering
    \begin{tikzpicture}
      \begin{axis}[
          axis lines=middle, % Draws x and y axes intersecting at the origin
          xlabel=$x$,
          ylabel=$y$,
          xmin=-0.2, xmax=1.8,
          ymin=-0.2, ymax=1.8,
          xtick=\empty, % Suppress numerical x-axis ticks
          ytick=\empty, % Suppress numerical y-axis ticks
          clip=false, % Allow nodes/labels to extend beyond the axis limits
          %xlabel style={at={(axis description cs:1,0.5)},anchor=north west}, % Position x-axis label
          %ylabel style={at={(axis description cs:0.5,1)},anchor=south east}, % Position y-axis label
        ]

        % Define parameters a, b, t. Choose values that create a similar visual.
        \pgfmathsetmacro{\aval}{0.5} % Example value for 'a' (e.g., 1/2)
        \pgfmathsetmacro{\bval}{0.5} % Example value for 'b' (e.g., 1/2)
        \pgfmathsetmacro{\tval}{1.5} % Example value for 't' for the line y=tx

        % Plot the curve C: ax^2 + by^2 = 1 in the 4th quadrant (x >= 0, y <= 0)
        % Solve for y: by^2 = 1 - ax^2  =>  y = -sqrt((1-ax^2)/b)
        % The x-domain for this curve is from 0 to 1/sqrt(a) (where y=0)
        \addplot[name path=A, domain=0:sqrt(1/\aval)-0.2, samples=100, red, thick] plot (\x, {\aval*x*x/sqrt(\bval*(1-\aval*x*x))});
        \addplot[domain=0:sqrt(1/\aval), samples=100, blue, thick] plot (\x, {sqrt((1-\aval*\x*\x)/\bval)});
        \node[below right] at (axis cs:1.2,1.2) {C}; % Label for curve C


        % Plot the line lt: y = tx
        % Determine an x-coordinate for point P to ensure the line is drawn appropriately.
        \pgfmathsetmacro{\xp}{0.8} % x-coordinate for points P and P'
        \node[above right] at (axis cs:\xp*1.1, \tval*\xp*1.2) {$C'$}; % Label for line lt

        % Calculate coordinates for P and P'
        \pgfmathsetmacro{\ypP}{\tval*\xp} % y-coordinate for P on line lt
        \pgfmathsetmacro{\ypPprime}{-sqrt((1-\aval*\xp*\xp)/\bval)} % y-coordinate for P' on curve C
        \pgfmathsetmacro{\xmax}{sqrt(1/\aval)} % y-coordinate for P on line lt

        % Draw point P
        \addplot[only marks, mark=*, mark size=2pt] coordinates {(\xmax, 0)};
        % Draw the vertical dashed line connecting P and P'
        \draw[dashed, black, thick] (axis cs:\xmax,0) -- (axis cs:\xmax,1.8);

        % Label for 1/sqrt(a) on the x-axis
        \node[above right] at (axis cs:{sqrt(1/\aval)},0) {$\frac{1}{\sqrt{a}}$};
        % Label for the origin
        \node[below left] at (axis cs:0,0) {O};

        % Draw point P'
        \pgfmathsetmacro{\alphaa}{1/sqrt(2*\aval)};
        \pgfmathsetmacro{\betaa}{1/sqrt(2*\bval)};
        \addplot[only marks, mark=*, mark size=2pt] coordinates {(\alphaa, \betaa)};
        \node[below left] at (axis cs:\alphaa, \betaa) {Q};
        \draw[name path = B, dashed, black, thick] (axis cs:0,\betaa) -- (axis cs:\alphaa,\betaa);
        \addplot[pattern=north west lines] fill between [of=A and B, soft clip={domain=0:\alphaa}];
        \node[left] at (axis cs:0,\betaa) {$\beta$};
        \node[fill=white] at (axis cs:0.5,0.5) {$D$};
        \draw[dashed, black, thick] (axis cs:\alphaa,0) -- (axis cs:\alphaa,\betaa);
        \node[above right] at (axis cs:\alphaa,0) {$\alpha$};

      \end{axis}
    \end{tikzpicture}
    \caption{$D$の概形}
    \label{fig:3}
  \end{figure}


  以下,バームクーヘン積分法によって$D$を$y$軸周りに回転した回転体の体積$V$を求める.
  $x \sim x+dx \quad (dx \ll 1)$の部分を軸まわりに回した
  立体の体積は幅$dx$, 高さ$\frac{1}{\sqrt{2b}}-f(x)$, 長さ$2\pi x$の直方体で近似できるので,
  求める立体の体積$V$として,
  \begin{align}
    V = 2\pi \int_{0}^{\sqrt{\frac{1}{2a}}} \left(\frac{1}{\sqrt{2b}} - f(x)\right)x dx \label{eq:1}
  \end{align}
  である.一項目は
  \begin{align}
    \int_{0}^{\sqrt{\frac{1}{2a}}} \frac{1}{\sqrt{2b}} x dx
     & = \frac{1}{2} \frac{1}{\sqrt{2b}} x^2 \Big|_{0}^{\sqrt{\frac{1}{2a}}} \nonumber \\
     & = \frac{1}{4\sqrt{2}} \frac{1}{a\sqrt{b}} \label{eq:2}
  \end{align}
  であり,二項目は$s=1-ax^2$と置換すると
  \begin{align}
    \int_{0}^{\sqrt{\frac{1}{2a}}} f(x)x dx
     & = \int_{1}^{1/2} \frac{1-s}{\sqrt{bs}} \frac{1}{-2a} ds                        \nonumber       \\
     & = \frac{1}{2a\sqrt{b}} \int_{1/2}^{1} \left(\frac{1}{\sqrt{s}} - \sqrt{s}\right) ds  \nonumber \\
     & = \frac{1}{2a\sqrt{b}} \left[2\sqrt{s} - \frac{2}{3} s^{3/2}\right]_{1/2}^{1}     \nonumber    \\
     & = \frac{1}{2a\sqrt{b}} \left[2\left(1-\frac{1}{\sqrt{2}}\right) -
    \frac{2}{3}\left(1-\frac{1}{2\sqrt{2}}\right)\right]                               \nonumber      \\
     & = \frac{1}{2a\sqrt{b}} \left(\frac{4}{3} - \frac{5\sqrt{2}}{6}\right) \label{eq:3}
  \end{align}
  だから,\cref{eq:2,eq:2}を\cref{eq:1}に代入して
  \begin{align*}
    V & = 2\pi \left[ \frac{1}{4\sqrt{2}a\sqrt{b}}  - \frac{1}{2a\sqrt{b}} \left(\frac{4}{3} - \frac{5\sqrt{2}}{6}\right) \right] \\
      & = \frac{2\pi}{a\sqrt{b}} \left[ \frac{13\sqrt{2}}{24} - \frac{2}{3} \right]
  \end{align*}
  が求める体積である.  $\cdots$(答)

  \vspace{10pt}
  {\bf[解説]}

  \newpage
\end{multicols}
\end{document}