% TODO :: 図1の更新
\documentclass[a4paper,10pt]{ltjsarticle}
\usepackage{luatexja}
\usepackage[hiragino-pron]{luatexja-preset}

\usepackage[truedimen,top=25truemm,bottom=20truemm,left=15truemm,right=15truemm]{geometry}
\setlength{\textwidth}{54\zw}
\setlength{\textheight}{73\zw}

\usepackage{amsmath,amssymb,ascmac}
\usepackage{enumerate}
\usepackage{multicol}
\usepackage{cleveref}
\usepackage{framed}
\usepackage{fancyhdr}
\usepackage{latexsym}
\usepackage{mathtools}
\usepackage{tikz}
\usepackage{tikz-3dplot}
\usepackage{pgfplots}
 \usetikzlibrary{math}
 \usetikzlibrary{calc}
 \usetikzlibrary{angles}
 \usetikzlibrary{quotes}
% \usepackage{indent}
\usepackage{caption}
\usepackage{subcaption}
\usepackage{cases}
\usepackage{float}
\usepackage{cases}
  \usepackage{caption}
  \usepackage[subrefformat=parens]{subcaption}
\allowdisplaybreaks
\tdplotsetmaincoords{70}{110}
\pagestyle{fancy}
\lhead{}
\chead{}
\rhead{東工大後期$1999$年$2$番}
\begin{document}
\begin{oframed}
\begin{enumerate}
  \item 半径 $1$ の円に内接する $6$ 個の半径の等しい円を図 $1$ のように描き,さらに図 $2$ のように $6$ 個の小さな半径の等しい円を描く,この操作を無限にくり返したとき,$6$ 個ずつ次々に描かれる円の面積の総和 $S_2$ と,それらの円の円周の長さの総和 $C_2$ を求めよ.
  \item (1) で $6$ 個の円を次々に描いていった.一般に,自然数 $n \ge 2$ に対して $3n$ 個の円を用いて同様の操作を行うとき,描かれる円の面積の総和 $S_n$ と,それらの円の円周の長さの総和 $C_n$ を求めよ.
  \item 数列 $S_2, S_3, S_4, \cdots$ の極限値を求めよ.
\end{enumerate}

\begin{figure}[H]
  \centering
  \begin{subcaptionblock}{0.5\linewidth}
    \centering
\begin{tikzpicture}
    % 中心に配置する円(オプション: 問題文には明示されていないが、図1には中心に円がない)
    % \draw (0,0) circle (0.5cm);

    % 周囲の6つの円
    % 外側の円の半径をR=3cmとする。
    % 周囲の円の半径をrとする。
    % 中心から周囲の円の中心までの距離をdとする。
    % d = R - r
    % また、中心から周囲の円の中心までの距離がd、周囲の円の半径がrであるとき、
    % sin(30度) = r / d より、d = 2r
    % よって、R - r = 2r => R = 3r => r = R/3
    % d = 2R/3
    \def\R{3} % 外側の円の半径
    \tikzmath{
    \r = \R / 3; % 周囲の円の半径
    \d = \R - \r; % 中心から周囲の円の中心までの距離
    }
    % 外側の大きな円
    \draw (0,0) circle (\R); % 適当なサイズに調整
    \foreach \i in {0, 60, ..., 300} {
        \draw ({\d*cos(\i)}, {\d*sin(\i)}) circle (\r cm);
    }
\end{tikzpicture}
\subcaption{図1}
\end{subcaptionblock}\hfill
  \begin{subcaptionblock}{0.5\linewidth}
    \centering
\begin{tikzpicture}
    \def\R{3} % 外側の円の半径
    \tikzmath{
    \r = \R / 3; % 周囲の円の半径
    \d = \R - \r; % 中心から周囲の円の中心までの距離
    }
    % 最も外側の大きな円(半径1と仮定)
    \draw (0,0) circle (\R);
    % 外側の大きな円
    \foreach \i in {0, 60, ..., 300} {
      \coordinate (C\i) at ({\d*cos(\i)}, {\d*sin(\i)});
      \draw (C\i) circle (\r);
      \draw (0,0) -- (C\i); % 中心から外側の円の中心への直線
    }
    \tikzmath{
    \rsmall = \r / 3; % 周囲の円の半径
    \dsmall = \r - \rsmall; % 中心から周囲の円の中心までの距離
    }
    % 内側の小きな円
    \foreach \i in {0, 60, ..., 300} {
        \draw ({\dsmall*cos(\i)}, {\dsmall*sin(\i)}) circle (\rsmall);
    }
\end{tikzpicture}
\subcaption{図2}
\end{subcaptionblock}\hfill
\end{figure}

\end{oframed}
\setlength{\columnseprule}{0.4pt}
\begin{multicols}{2}
{\bf[解]}
はじめから(2)のような一般的な状況を考える.すなわち,自然数 $n\ge 2$ に対して $3n$ 個の円を用いて操作を行っていく.そこで最初に(2)から考えよう.
$k$回目の操作で描かれる円の半径を$r_{n}(k)$とおく.

\begin{figure}[H]
  \centering
\begin{tikzpicture}
    % 中心に配置する円(オプション: 問題文には明示されていないが、図1には中心に円がない)
    % \draw (0,0) circle (0.5cm);

    % 周囲の6つの円
    % 外側の円の半径をR=3cmとする。
    % 周囲の円の半径をrとする。
    % 中心から周囲の円の中心までの距離をdとする。
    % d = R - r
    % また、中心から周囲の円の中心までの距離がd、周囲の円の半径がrであるとき、
    % sin(30度) = r / d より、d = 2r
    % よって、R - r = 2r => R = 3r => r = R/3
    % d = 2R/3
    \def\R{3} % 外側の円の半径
    \tikzmath{
    \r = \R / 3; % 周囲の円の半径
    \d = \R - \r; % 中心から周囲の円の中心までの距離
    }
    % 外側の大きな円
    \draw (0,0) circle (\R); % 適当なサイズに調整
    \foreach \i in {0, 60, ..., 300} {
        \draw ({\d*cos(\i)}, {\d*sin(\i)}) circle (\r cm);
    }
    \node[left] at (0,0) {$O$};
    \node[below] at ({\d*cos(0)}, {\d*sin(0)}) {$A$};
    \node[above] at ({3*\d/4}, {sqrt(3)*\d/4}) {$B$};
    \draw (0,0) -- ({\d*cos(0)}, {\d*sin(0)});
    \draw (0,0) -- ({3*\d/4}, {sqrt(3)*\d/4});
    \draw ({\d*cos(0)}, {\d*sin(0)}) -- ({3*\d/4}, {sqrt(3)*\d/4});
    \node[below] at ({\d*cos(0)/2}, {\d*sin(0)}) {$1-r_n(1)$};
    \node[right] at ({7*\d/8}, {sqrt(3)*\d/8}) {$r_n(1)$};
\end{tikzpicture}
\caption{$k=1$の様子}
\end{figure}

\begin{figure}[H]
    \centering
\begin{tikzpicture}
        \coordinate (O_prime) at (0,0);
        \coordinate (B_prime) at (3,0);
        \coordinate (A_prime) at (3,2);

        \draw (O_prime) node[below left] {O} -- (B_prime) node[below left] {B};
        \draw (B_prime) -- (A_prime) node[above right] {A};
        \draw (A_prime) -- (O_prime);

        \node[below right] at (0.4,1.7) {$1-r_n(1)$};
        \node[right] at  (3, 1) {$r_n(1)$};
        \draw (0.3,0) arc (0:33.7:0.3) node[right] {$\pi/3n$}; %shift={(0.05, 0.1)}
\end{tikzpicture}
\caption{$k=1$のときの半径の導出}
\label{fig:1}
\end{figure}

まず初期条件$k=1$を考える.ひとつの円の中心を$A$,隣接する円との接点$B$とする.\cref{fig:1}で $\angle \text{AOB}= \dfrac{\pi}{3n}$ だから
\begin{align}
    \sin \dfrac{\pi}{3n} 
    &= \frac{AB}{OA} \nonumber \\
    &=\frac{r_n(1)}{1-r_n(1)} \nonumber \\
    \therefore 
    r_n(1) &= \frac{\sin \frac{\pi}{3n}}{1+\sin \frac{\pi}{3n}} \label{eq:1}
\end{align}
と求まる.

\begin{figure}[H]
    \centering
\begin{tikzpicture}[scale=2]
    % First circle (radius rn)
    \coordinate (G) at ({2*sqrt(2)}, 0); % Center of the first circle
    \coordinate (C) at ({2*sqrt(2)}, 2); % Center of the first circle
    \coordinate (A) at (0, 2); % Center of the first circle
    \coordinate (On) at ({sqrt(2)}, 1); % Center of the first circle
    \coordinate (O1) at ({2*sqrt(2)-sqrt(6)}, {sqrt(3)});
    \draw (A) -- (C);
    \draw (A) -- (G);
    \fill (On) circle [radius=0.5pt];
    \fill (O1) circle [radius=0.5pt];

    \draw (On) circle (1); % Radius 1.5cm for visualization
    \draw (O1) circle ({2-sqrt(3)}); % Radius 1.5cm for visualization
    \node[above] at ({1.5*sqrt(2)-0.5*sqrt(6)},{0.5+sqrt(3)*0.5}) {$r_n(k)$}; % Label for radius
    \node[below] at (On) {$O_k$};
    \node[below] at (O1) {$O_{k+1}$};
    \node[left]  at (A) {A};
    \node[right]  at (C) {C};

    \draw[dashed] (On) -- ({sqrt(2)},2);
    \draw[dashed] (O1) -- ({2*sqrt(2)-sqrt(6)},2);
    \draw[dashed] (O1) -- ({sqrt(2)},{sqrt(3)});

    \node[above] at ({sqrt(2)},2) {$T_k$};
    \node[above] at ({2*sqrt(2)-sqrt(6)},2) {$T_{k+1}$};
    \node[right] at ({sqrt(2)},{sqrt(3)}) {$R_{k}$};
    
    \path pic["$\theta$",draw,angle radius=7mm,angle eccentricity=1.3] {angle = G--A--C};

\end{tikzpicture}
\caption{$r_n(k+1)$と$r_{n}(k)$の関係}
\label{fig:3}
\end{figure}

円の半径の一般項$r_n(k)$と$r_n(k+1)$を考える.\cref{fig:3}で$k$回目の円の中心$O_k$とおく.半径$1$の最初の円の中心$A$とし,$A$から各円に接する直線を$AC$とする.$O_k$から$AC$におろした垂線の足を$T_k$とおく.$O_k$から$O_kT_k$におろした垂線の足を$R_k$とおく.

線分$O_kT_k$の長さは半径$r_n(k)$に等しい.別の表現で$O_kT_k$を計算すると
\begin{align*}
  O_kT_k 
  &= O_kR_k + R_kT_k \\
  &= O_kO_{k+1}\sin\frac{\pi}{3n} + r_n(k+1) \\
  &= \left(r_n(k)+r_n(k+1)\right)\sin\frac{\pi}{3n} + r_n(k+1) 
\end{align*}
となる.したがって
\begin{align*}
  r_n(k) &= \left(r_n(k)+r_n(k+1)\right)\sin\frac{\pi}{3n} + r_n(k+1)  \\
  r_n(k+1) &= \frac{1-a_n}{1+a_n}r_n(k) \label{eq:2}
\end{align*}
なる漸化式を得る.
\cref{eq:1,eq:2}から等比数列の公式より,一般項は
\begin{align}
r_n(k) 
 &= \left(\frac{1-a_n}{1+a_n}\right)^{k-1} r_n(1) \\
 &= \left(\frac{1-a_n}{1+a_n}\right)^{k-1}\frac{a_n}{1+a_n}
\end{align}
となる.$A = \left(\frac{1-a_n}{1+a_n}\right)$ として $|A|<1$ だから,円の面積の総和は
\begin{align}
S_n 
&= \lim_{k \to \infty} \sum_{p=1}^{k} r^2_n(p) \cdot \pi \cdot 3n \\
&= \pi \left(\frac{a_n}{1+a_n}\right)^2 \frac{1}{1-A^2} \cdot 3n \\
&= \frac{3\pi}{4} n a_n  \label{eq:3}
\end{align}
である.円周の総和は
\begin{align}
C_n 
&= \lim_{p=1}^{k} \sum_{p=1}^{k} 2\pi r_n(p) \cdot 3n \\
&= 6n\pi \frac{a_n}{1+a_n} \frac{1}{1-A} \\
&= 3n\pi \label{eq:4}
\end{align}
と求まる.

\vspace{10pt}
(1)
\cref{eq:3,eq:4}に$n=2$を代入して
\begin{align*}
  &S_2 = \dfrac{3}{4}\pi & C_2 = 6\pi
\end{align*}
である.$\cdots$(答)

\vspace{10pt}
(2)
\cref{eq:3,eq:4}が答えである.
\begin{align*}
  S_n &= \frac{3\pi}{4} n \sin \dfrac{\pi}{3n} \\
  C_n &= 3n\pi
\end{align*}$\cdots$(答)

\vspace{10pt}
(3)
\cref{eq:3}の$n\to\infty$の極限をとって
\begin{align*}
  \lim_{n\to\infty} S_n 
  &= \lim_{n\to\infty} \dfrac{\pi^2}{4} \dfrac{3n}{\pi} \sin \dfrac{\pi}{3n} \\
  &= \frac{\pi^2}{4} 
\end{align*}
である.ただし,$n\to\infty$で$3n/\pi\to 0$であることを用いた.$\cdots$(答)

\vspace{10pt}
{\bf[解説]}

平面図形の問題.1995年の1番とほぼ同様の円を敷き詰めていく問題である.特に漸化式の導出はほぼ全く同じなため,過去問演習をやっていた人にとってはボーナス問題だった可能性が高い.
問題としても漸化式が立てば極限値の計算は非常に容易なので,簡単な部類である.
(1)は本来誘導の意図だろうが,漸化式が立つという見通しがあれば先に(2)に手を出して,(1)はあとから値を代入すればだいぶ時間の短縮になる.

\newpage
\end{multicols}
\end{document}