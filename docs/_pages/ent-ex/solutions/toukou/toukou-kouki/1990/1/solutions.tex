\documentclass[a4paper,10pt]{ltjsarticle}
\usepackage{luatexja}
\usepackage[hiragino-pron]{luatexja-preset}

\usepackage[truedimen,top=25truemm,bottom=20truemm,left=15truemm,right=15truemm]{geometry}
\setlength{\textwidth}{54\zw}
\setlength{\textheight}{73\zw}

\usepackage{amsmath,amssymb,ascmac}
\usepackage{enumerate}
\usepackage{multicol}
\usepackage{cleveref}
\usepackage{framed}
\usepackage{fancyhdr}
\usepackage{latexsym}
\usepackage{tikz}
\usepackage{pgfplots}
 \usetikzlibrary{math}
% \usepackage{indent}
\usepackage{cases}
\usepackage{float}
\allowdisplaybreaks
\pagestyle{fancy}
\lhead{}
\chead{}
\rhead{東工大後期$1990$年$1$番}
\begin{document}
\begin{oframed}
$(x+1)(x-2)$ の小数第 $1$ 位を四捨五入したものが $1+5x$ と等しくなるような実数 $x$を求めよ.
\end{oframed}
\setlength{\columnseprule}{0.4pt}
\begin{multicols}{2}
{\bf[解]}

実数 $x$ に対して $f(x) = (x+1)(x-2)$, $g(x) = 1+5x$ とおく.題意から,
    \begin{align}\label{eq:1}
        g(x) \in \mathbb{Z}, \quad g(x) - \frac{1}{2} \le f(x) < g(x) + \frac{1}{2} 
    \end{align}
    をみたす $x \in \mathbb{R}$ をもとめればよい.
    $g(x) \in \mathbb{Z}$ から $5x \in \mathbb{Z}$.つまり,$x = \frac{t}{5}$ ($t \in \mathbb{Z}$)
    とかける.\cref{eq:1}に代入して$t$の条件式を求めると,
    \begin{align}
        t + \frac{1}{2} &\le \left(\frac{t}{5}+1\right)\left(\frac{t}{5}-2\right) < t+\frac{3}{2} \nonumber \\
        t + \frac{1}{2} &\le \frac{1}{25}t^2  - \frac{1}{5}t - 2 < t+\frac{3}{2} \nonumber \\
        \therefore 125 &\le 2t^2 - 60t < 175 \label{eq:2}
    \end{align}
    を得る.

    ここで,二次関数$y = p(t) = 2t^2 - 60t$ のグラフの概形は\cref{fig:graph}のようになっており,
    \begin{align*}
     &p(-3) = 198 & p(-1) = 62 \\
     &p(31) = 62  & p(33) = 198 
    \end{align*}
    だから,\cref{eq:2}を満たすような$t \in \mathbb{Z}$は $t=-2, 32$ である.したがって求めるべき$x=t/5$は$x = \frac{32}{5}, -\frac{2}{5}$ である. $\cdots$(答)

\begin{figure}[H]
     \centering
     % Define values for clarity
\def\yBoundaryLower{125}
\def\yBoundaryUpper{175}
\def\yAtIntPoints{128} % y-value for t=-2 and t=32

% % Calculate t-values for y=125
\pgfmathsetmacro{\tLowerSmallMinus}{15 - 0.5 * sqrt(1150)} % 15 - (5/2)*sqrt(46) approx -1.95
\pgfmathsetmacro{\tUpperSmallPlus}{15 + 0.5 * sqrt(1150)} % 15 + (5/2)*sqrt(46) approx 31.95

% % Calculate t-values for y=175
\pgfmathsetmacro{\tLowerLargeMinus}{15 - 0.5 * sqrt(1250)} % 15 - (25/2)*sqrt(2) approx -2.675
\pgfmathsetmacro{\tUpperLargePlus}{15 + 0.5 * sqrt(1250)} % 15 + (25/2)*sqrt(2) approx 32.675

\begin{tikzpicture}[
    xscale=0.1, % Adjust x-axis scale for width
    % yscale=0.005, % Original value causing error
    yscale=0.01,
    ]
% Define t-range for plotting
\def\tmin{-3}
\def\tmax{33}
\def\ymin{-10}
\def\ymax{250}

% Axes
\draw[->] (\tmin, 0) -- (\tmax, 0) node[right] {$t$};
\draw[->] (0, \ymin) -- (0, \ymax) node[above] {$y$};

% Label the origin
\node at (0,0) [below right] {$0$};

% Plot the function y = 2t^2 - 60t
\draw[blue, thick, domain=\tmin:\tmax, samples=100] plot (\x, {2*\x*\x - 60*\x});

% % Mark Vertex
\fill[red] (-2, 128) circle (5pt);
\fill[red] (32, 128) circle (5pt);
% \draw[dashed, gray] (\tVertex, 0) -- (\tVertex, \yVertex);
% \node at (\tVertex, 0) [below] {$15$};
% \node at (0, \yVertex) [left] {$-450$};


% % Vertical lines and points for t=-2 and t=32 (where y=128)
% % These are the integer solutions mentioned in the problem description for the given y range.
\draw[dashed, gray] (-2, 0) -- (-2, \yAtIntPoints);
\node at (-2, 0) [below] {$-2$};
\draw[dashed, gray] (32, 0) -- (32, \yAtIntPoints);
\node at (32, 0) [below] {$32$};
\draw[dashed, gray] (0, \yAtIntPoints) -- (32, \yAtIntPoints);
\node at (0, \yAtIntPoints) [right] {$128$};

% % Dashed lines from the approximate intersection points to the t-axis
% % For y=125
\draw[dashed, gray] (\tmin, \yBoundaryLower) -- (\tmax, \yBoundaryLower);
% % For y=175
\draw[dashed, gray] (\tmin, \yBoundaryUpper) -- (\tmax, \yBoundaryUpper);

% % Highlight the parts of the parabola where 125 <= y < 175
\draw[red, ultra thick, domain=\tLowerLargeMinus:\tLowerSmallMinus, samples=50] plot (\x, {2*\x*\x - 60*\x});
\draw[red, ultra thick, domain=\tUpperSmallPlus:\tUpperLargePlus, samples=50] plot (\x, {2*\x*\x - 60*\x});
\end{tikzpicture}
\caption{二次関数 $y = 2t^2 - 60t$ のグラフ}
\label{fig:graph}
\end{figure}


{\bf[解説]}
二次関数の問題.条件を素直に式に落としていけば解ける比較的容易な問題である.
二次関数 $(x+1)(x-2)$ と一次関数 $1+5x$ がほぼ等しくなるような条件なので,解はこれらの交点に近くなるだろうというのが予想できる.
実際にこれを解いてみると
\begin{align*}
    (x+1)(x-2) &= 1 + 5x \\
    x^2 - 6x - 3 &= 0 \\
    x = 3 \pm 2\sqrt{3} 
\end{align*}
であり,$x \approx 6.46$ と $x \approx -0.46$ が得られる.
本問題の解答である$x = \frac{32}{5}, -\frac{2}{5}$ はこれらの値にほぼ等しく,検算として利用できるだろう.

     \newpage
\end{multicols}
\end{document}