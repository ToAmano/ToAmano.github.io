\documentclass[a4paper,10pt]{ltjsarticle}
\usepackage{luatexja}
\usepackage[hiragino-pron]{luatexja-preset}

\usepackage[truedimen,top=25truemm,bottom=20truemm,left=15truemm,right=15truemm]{geometry}
\setlength{\textwidth}{54\zw}
\setlength{\textheight}{73\zw}

\usepackage{amsmath,amssymb,ascmac}
\usepackage{enumerate}
\usepackage{multicol}
\usepackage{framed}
\usepackage{fancyhdr}
\usepackage{latexsym}
% \usepackage{indent}
\usepackage{cases}
\allowdisplaybreaks
\pagestyle{fancy}
\lhead{}
\chead{}
\rhead{東京大学前期$1961$年$6$番}
\begin{document}
%分数関係


\def\tfrac#1#2{{\textstyle\frac{#1}{#2}}} %数式中で文中表示の分数を使う時


%Σ関係

\def\dsum#1#2{{\displaystyle\sum_{#1}^{#2}}} %文中で数式表示のΣを使う時


%ベクトル関係


\def\vector#1{\overrightarrow{#1}}  %ベクトルを表現したいとき(aベクトルを表現するときは\ver
\def\norm#1{|\overrightarrow{#1}|} %ベクトルの絶対値
\def\vtwo#1#2{ \left(%
      \begin{array}{c}%
      #1 \\%
      #2 \\%
      \end{array}%
      \right) }                        %2次元ベクトル成分表示
      
      \def\vthree#1#2#3{ \left(
      \begin{array}{c}
      #1 \\
      #2 \\
      #3 \\
      \end{array}
      \right) }                        %3次元ベクトル成分表示



%数列関係


\def\an#1{\verb|{|$#1$\verb|}|}


%極限関係

\def\limit#1#2{\stackrel{#1 \to #2}{\longrightarrow}}   %等式変形からの極限
\def\dlim#1#2{{\displaystyle \lim_{#1\to#2}}} %文中で数式表示の極限を使う



%積分関係

\def\dint#1#2{{\displaystyle \int_{#1}^{#2}}} %文中で数式表示の積分を使う時

\def\ne{\nearrow}
\def\se{\searrow}
\def\nw{\nwarrow}
\def\ne{\nearrow}


%便利なやつ

\def\case#1#2{%
 \[\left\{%
 \begin{array}{l}%
 #1 \\%
 #2%
 \end{array}%
 \right.\] }                           %場合分け
 
\def\1{$\cos\theta=c$,$\sin\theta=s$とおく.}  %cs表示を与える前書きシータ
\def\2{$\cos t=c$,$\sin t=s$とおく.}     %cs表示を与える前書きt
\def\3{$\cos x=c$,$\sin x=s$とおく.}                %cs表示を与える前書きx

\def\fig#1#2#3 {%
\begin{wrapfigure}[#1]{r}{#2 zw}%
\vspace*{-1zh}%
\input{#3}%
\end{wrapfigure} }           %絵の挿入


\def\a{\alpha}   %ギリシャ文字
\def\b{\beta}
\def\g{\gamma}

%問題番号のためのマクロ

\newcounter{nombre} %必須
\renewcommand{\thenombre}{\arabic{nombre}} %任意
\setcounter{nombre}{2} %任意
\newcounter{nombresub}[nombre] %親子関係を定義
\renewcommand{\thenombresub}{\arabic{nombresub}} %任意
\setcounter{nombresub}{0} %任意
\newcommand{\prob}[1][]{\refstepcounter{nombre}#1[問題 \thenombre]}
\newcommand{\probsub}[1][]{\refstepcounter{nombresub}#1(\thenombresub)}


%1-1みたいなカウンタ(todaiとtodaia)
\newcounter{todai}
\setcounter{todai}{0}
\newcounter{todaisub}[todai] 
\setcounter{todaisub}{0} 
\newcommand{\todai}[1][]{\refstepcounter{todai}#1 \thetodai-\thetodaisub}
\newcommand{\todaib}[1][]{\refstepcounter{todai}#1\refstepcounter{todaisub}#1 {\bf [問題 \thetodai.\thetodaisub]}}
\newcommand{\todaia}[1][]{\refstepcounter{todaisub}#1 {\bf [問題 \thetodai.\thetodaisub]}}


\begin{oframed}
$a$,$b$,$c$は定数であって,函数$f(x)=a\sin x+b\cos x+c\cos 2x$は$x=\dfrac{\pi}{4}$において
極大値$6\sqrt{2}$をとり,また$\int_0^{2\pi} f(x)\cos x dx=5\pi$である.このとき
\begin{enumerate}[(1)]
\item $a$,$b$,$c$を求めよ.
\item $0\le x\le 2\pi$の範囲で$f(x)$を最小にする$x$の値とその時の$f(x)$の値とを求めよ.
\end{enumerate}
\end{oframed}
\setlength{\columnseprule}{0.4pt}
\begin{multicols}{2}
{\bf[解]}
     $\cos x=t$,$\sin x=s$とおく.
     \[f'(x)=at-bs+2c\cos 2x\]
     である.まず,極大値の条件から
     $f\left(\dfrac{\pi}{4}\right)=0$が必要である.これと題意の条件から
          \begin{align*}
          &\left\{
               \begin{array}{l}
               f'\left(\dfrac{\pi}{4}\right)=0 \\
               f\left(\dfrac{\pi}{4}\right)=6\sqrt{2} \\
               \dint{0}{2\pi}f(x)\cos x dx=5\pi 
               \end{array}
          \right.  \\
          \Leftrightarrow
          &\left\{
               \begin{array}{l}
               \dfrac{\sqrt{2}}{2}(a-b)=0 \\
               \dfrac{\sqrt{2}}{2}(a+b)+c=6\sqrt{2} \\
               b\pi=5\pi 
               \end{array}
         \right.
         \end{align*}
    故に$(a,b,c)=(5,5,\sqrt{2})$となる.
      $f'(x)$に値を代入する.
          \begin{align*}
          f'(x)&=5t-5s+2\sqrt{2}(t^2-s^2) \\
          &=(t-s)\left(5+4\sin\left(x+\frac{\pi}{4}\right)\right)
          \end{align*}
     したがって,下表を得る.
      \begin{align*}
     \begin{array}{|c|c|c|c|c|c|c|c|} \hline
     x & 0 &    &  \pi/4&     &5\pi/4            &     &2\pi  \\ \hline
     f' &    &+  & 0      & -    & 0                 &+   &        \\ \hline
     f &  5 &\ne&        &\se &-4\sqrt{2}      &\ne&        \\ \hline
     \end{array}
     \end{align*}
     故に$x=\dfrac{\pi}{4}$で極大となり十分である.以上から
     \begin{enumerate}[(1)]
     \item $(a,b,c)=(5,5,\sqrt{2})$ 
     \item $\min f(x)=f\left(\dfrac{5\pi}{4}\right)=-4\sqrt{2}$ 
     \end{enumerate}
     となる.$\cdots$(答)
          \newpage
\end{multicols}
\end{document}