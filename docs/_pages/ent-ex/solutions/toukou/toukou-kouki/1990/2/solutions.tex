\documentclass[a4paper,10pt]{ltjsarticle}
\usepackage{luatexja}
\usepackage[hiragino-pron]{luatexja-preset}

\usepackage[truedimen,top=25truemm,bottom=20truemm,left=15truemm,right=15truemm]{geometry}
\setlength{\textwidth}{54\zw}
\setlength{\textheight}{73\zw}

\usepackage{amsmath,amssymb,ascmac}
\usepackage{enumerate}
\usepackage{multicol}
\usepackage{cleveref}
\usepackage{framed}
\usepackage{fancyhdr}
\usepackage{latexsym}
\usepackage{mathtools}
\usepackage{tikz}
\usepackage{pgfplots}
 \usetikzlibrary{math}
% \usepackage{indent}
\usepackage{cases}
\usepackage{float}
\allowdisplaybreaks
\pagestyle{fancy}
\lhead{}
\chead{}
\rhead{東工大後期$1990$年$2$番}
\begin{document}
\begin{oframed}
$n$ を $2$ 以上の整数とする.
\begin{enumerate}
  \item $n-1$ 次多項式 $P_n(x)$ と $n$ 次多項式 $Q_n(x)$ で実数 $\theta$ に対して
\[ \sin(2n\theta) = n \sin(2\theta) P_n(\sin^2\theta), \quad \cos(2n\theta) = Q_n(\sin^2\theta) \]
を満たすものが存在することを帰納法を用いて示せ.
  \item $k=1, 2, \dots, n-1$ に対して $\alpha_k = \left(\sin\frac{k\pi}{2n}\right)^{-2}$ とおくと
\[ P_n(x) = (1-\alpha_1 x)(1-\alpha_2 x)\cdots(1-\alpha_{n-1} x) \]
となることを示せ.
  \item $\sum_{k=1}^{n-1} \alpha_k = \frac{2n^2-2}{3}$ を示せ.
\end{enumerate}
\end{oframed}
\setlength{\columnseprule}{0.4pt}
\begin{multicols}{2}
{\bf[解]}

$n \in \mathbb{N}$ に拡張して考えて良い.

(1) 数学的帰納法で題意を示す.

\vspace{10pt}

(i)
$n=1, 2$ のとき
\begin{align*}
    P_1(x) &= 1, P_2(x) = 1-2x \\
    Q_1(x) &= 1-2x, Q_2(x) = 8x^2 - 8x + 1 \quad \cdots (*)
\end{align*}
とすれば良く成立する.

\vspace{10pt}

(ii)
以下 $n=k, k+1 \in \mathbb{N}$ での成立を仮定し,$n=k+2$ での成立を示す.
和積公式より,
\begin{align*}
    \sin(2k+4)\theta 
    &= 2\sin(2k+2)\theta\cos2\theta - \sin2k\theta \\
    \cos(2k+4)\theta 
    &= 2\cos(2k+2)\theta\cos2\theta - \cos2k\theta 
\end{align*}
であり,ここに$n=k, k+1$ のときの $P_n(x), Q_n(x)$ を代入して
\begin{align*}
    \sin(2k+4)\theta 
    &= \{2(1-2\sin^2\theta)(k+1)P_{k+1}(\sin^2\theta) - kP_k(\sin^2\theta)\}\sin(2\theta) \\
    \cos(2k+4)\theta
    &= 2(1-2\sin^2\theta)Q_{k+1}(\sin^2\theta) - Q_k(\sin^2\theta)
\end{align*}
を得る.従って,
\begin{align*}
    \begin{dcases}
        P_{k+2}(x) &= \frac{1}{k+2}\{2(1-2x)(k+1)P_{k+1}(x) - kP_k(x)\} \\
        Q_{k+2}(x) &= 2(1-2x)Q_{k+1}(x) - Q_k(x)
    \end{dcases}
\end{align*}
とすれば、$P_{k+2}(x), Q_{k+2}(x)$ は $k+1, k+2$ 次多公式であり条件を満たす.以上から $n=k+2$ でも成立.

(i), (ii) より,数学的帰納法により題意は示された.

\vspace{10pt}
(2) 題意を示すには,因数定理より$P_n(x)$ の零点が$x=1/\alpha_{k}$であること,および$0$次の係数が$1$であることを示せば良い.
まずは前者から示す.
$0 \le \theta < 2\pi$ とする.
$\sin(2n\theta) = 0 \Leftrightarrow 2n\theta = 0, \pi, 2\pi, \dots$
だから、これ以外の時、(1)から
$$P_n(\sin^2\theta) = \frac{\sin(2n\theta)}{n\sin(2\theta)} \qquad \cdots\cdots ②$$
だから、$P_n(\sin^2\theta)=0 \Leftrightarrow \sin(2n\theta)=0$
$\theta_k = \frac{k\pi}{2n} \quad (k=1, 2, \dots, 2n-1, \text{ただし} k \ne n, 2n, 3n, \dots)$ となる。$x=\sin^2\theta$ とすると、$\sin^2\theta$ の周期性から
$P_n(x)=0 \Leftrightarrow x = \sin^2\frac{k\pi}{2n} \quad (k=1, 2, \dots, n-1)$
これら $n-1$ 個の解は互いに異なり、さらに $P_n(x)$ は $n-1$ 次式だから、これが $P_n(x)=0$ の全ての解である。$A\ne 0$ として
$$P_n(x) = A \prod_{k=1}^{n-1} \left(x - \sin^2\frac{k\pi}{2n}\right) \qquad \cdots\cdots ③$$
とおける。以下 $A$ を求める。$P_n(x)$ の定数項を $a_n$ とすると、(1)及び(2)から
$$\begin{cases} a_1=1, a_2=1 \\ a_{n+2} = \frac{1}{n+2}\{2(n+1)a_{n+1} - na_n\} \end{cases}$$
となり、帰納的に $a_n=1$ である。③で係数比較して
$$A \prod_{k=1}^{n-1} \left(-\sin^2\frac{k\pi}{2n}\right) = 1$$
$$\therefore A = \frac{1}{\prod_{k=1}^{n-1} \left(-\sin^2\frac{k\pi}{2n}\right)}$$
だから、③に代入して
$$P_n(x) = \prod_{k=1}^{n-1} \left(1 - \frac{x}{\sin^2\frac{k\pi}{2n}}\right) = \prod_{k=1}^{n-1} (1-d_k x) \qquad \cdots\cdots ④$$
と表せる。

(3)
$\sum_{k=1}^{n-1} \alpha_k$ は、$P_n(x)$ の $x$ の1次の項の係数を $b_n$ として、
$$
\sum_{k=1}^{n-1} \alpha_k = -b_n \quad \cdots\cdots {5}
$$
と表せる。ここで、$b_n$ について、($*$) から
$$
\begin{cases}
b_1 = 0, \ b_2 = -2 \\
b_{n+2} = \frac{1}{n+2} [2(n+1)(b_{n+1} - 2) - n b_n] \quad (\because a_n=1)
\end{cases}
\quad \cdots\cdots {6}
$$
となる。以下、$b_n = -\frac{2}{3}(n^2 - 1)$ となることを数学的帰納法により示す。

\vspace{10pt}

(i) $n=1, 2$ のとき
$b_1 = -\frac{2}{3}(1^2-1) = 0$
$b_2 = -\frac{2}{3}(2^2-1) = -2$
となり、${6}$ の初期条件と一致するため成立する。

\vspace{10pt}

(ii) $n=k, k+1$ での成立を仮定する。すなわち、
$b_k = -\frac{2}{3}(k^2 - 1)$, $b_{k+1} = -\frac{2}{3}((k+1)^2 - 1)$
が成り立つと仮定する。このとき、$n=k+2$ での成立を示す。
${6}$ の漸化式に $n=k$ を代入し、仮定を用いると、
\begin{align*}
b_{k+2} &= \frac{1}{k+2} \left[ 2(k+1)(b_{k+1} - 2) - k b_k \right] \\
&= \frac{1}{k+2} \left[ 2(k+1) \left\{ -\frac{2}{3}((k+1)^2 - 1) - 2 \right\} - k \left\{ -\frac{2}{3}(k^2 - 1) \right\} \right] \\
&= \frac{1}{k+2} \left[ 2(k+1) \left\{ -\frac{2}{3}(k^2+2k) - 2 \right\} + \frac{2}{3}k(k^2 - 1) \right] \\
&= \frac{1}{k+2} \left[ -\frac{4}{3}(k+1)(k^2+2k) - 4(k+1) + \frac{2}{3}k(k^2-1) \right] \\
&= \frac{2}{3(k+2)} \left[ -2(k+1)k(k+2) - 6(k+1) + k(k-1)(k+1) \right] \\
&= \frac{2(k+1)}{3(k+2)} \left[ -2k(k+2) - 6 + k(k-1) \right] \\
&= \frac{2(k+1)}{3(k+2)} \left[ -2k^2 - 4k - 6 + k^2 - k \right] \\
&= \frac{2(k+1)}{3(k+2)} (-k^2 - 5k - 6) \\
&= -\frac{2(k+1)}{3(k+2)} (k+2)(k+3) \\
&= -\frac{2}{3}(k+1)(k+3) \\
&= -\frac{2}{3}((k+2)^2 - 1)
\end{align*}
よって、$n=k+2$ のときも成立する。

\vspace{10pt}

(i), (ii) より、すべての自然数 $n$ に対して $b_n = -\frac{2}{3}(n^2 - 1)$ が示された。
これと ${5}$ から、
$$
\sum_{k=1}^{n-1} \alpha_k = -b_n = \frac{2}{3}(n^2-1)
$$
となる。

     \newpage
\end{multicols}
\end{document}