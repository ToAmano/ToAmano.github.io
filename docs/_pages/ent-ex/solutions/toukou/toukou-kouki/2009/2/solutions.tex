\documentclass[a4paper,10pt]{ltjsarticle}
\usepackage{luatexja}
\usepackage[hiragino-pron]{luatexja-preset}

\usepackage[truedimen,top=25truemm,bottom=20truemm,left=15truemm,right=15truemm]{geometry}
\setlength{\textwidth}{54\zw}
\setlength{\textheight}{73\zw}

\usepackage{amsmath,amssymb,ascmac}
\usepackage{enumerate}
\usepackage{multicol}
\usepackage{cleveref}
\usepackage{framed}
\usepackage{fancyhdr}
\usepackage{latexsym}
\usepackage{mathtools}
\usepackage{tikz}
\usepackage{tikz-3dplot}
\usepackage{pgfplots}
 \usetikzlibrary{math}
 \usetikzlibrary{calc}
 \usetikzlibrary{angles}
 \usetikzlibrary{quotes}
 \pgfplotsset{compat=1.18}
% \usepackage{indent}
\usepackage{caption}
\usepackage{subcaption}
\usepackage{cases}
\usepackage{float}
\usepackage{cases}
  \usepackage{caption}
  \usepackage[subrefformat=parens]{subcaption}
\allowdisplaybreaks
\tdplotsetmaincoords{70}{110}
\pagestyle{fancy}
\lhead{}
\chead{}
\rhead{東工大後期$2009$年$2$番}
\begin{document}
\begin{oframed}
  $p>0, q>0$ であるような点 P$(p,q)$ から双曲線 $y = -\frac{1}{x}$ へ引いた $2$ 本の接線の接点を $A, B$ とする.
  $pq$ を $t$ とおいて,三角形 PAB の面積を $t$ の式で表せ.
  また,この面積の最小値を求めよ.
\end{oframed}


\setlength{\columnseprule}{0.4pt}
\begin{multicols}{2}
  {\bf[解]}
  双曲線と点$P$は\cref{fig:1}のような関係になる.

  \begin{figure}[H]
    \begin{tikzpicture}
      \begin{axis}[
          axis lines=middle,
          xlabel=$x$,
          ylabel=$y$,
          xmin=-3.5, xmax=5,
          ymin=-2.5, ymax=2.5,
          grid=both,
          xtick={\empty},
          ytick={\empty},
          grid style={line width=.1pt, draw=gray!10},
          major grid style={line width=.2pt,draw=gray!50},
          tick label style={font=\tiny},
          xlabel style={at={(ticklabel cs:1.05)},anchor=north west},
          ylabel style={at={(ticklabel cs:1.05)},anchor=south west},
          clip=false
        ]
        % Draw the hyperbola y = 1/x
        \addplot[domain=0.5:5, samples=100, blue, thick]{-1/x};
        \addplot[domain=-3.5:-0.5, samples=100, blue, thick]{-1/x};

        % Define points A, B, P
        \coordinate (A) at (-2,0.5);
        \coordinate (B) at (1,-1);
        \coordinate (P) at (4,2);
        % Draw points A, B, P
        \fill (A) circle (1.5pt) node[above left ] {$A(m,-1/m)$};
        \fill (B) circle (1.5pt) node[below right] {$B(s,-1/s)$};
        \fill (P) circle (1.5pt) node[above right] {$P(p,q)$};

        % Draw tangent line l_A: y = -x + 2
        \draw[red, dashed, thick] (A) -- (P) node[above left, pos=0.8] {$l_A$};
        \draw[red, dashed, thick] (B) -- (P) node[below right, pos=0.8] {$l_B$};
        \draw[red, dashed, thick] (A) -- (B);
      \end{axis}
    \end{tikzpicture}
    \caption{三角形$PAB$の様子}
    \label{fig:1}
  \end{figure}
  従って,接点$A,B$の$x$成分を$m,s$として,
  \begin{align*}
    m < 0 < s
  \end{align*}
  と置いて良い.

  この時,点$A,B$での接線$l_A,l_B$の方程式は
  \begin{align*}
    l_A & : y = \frac{1}{m^2}x - \frac{2}{m} \\
    l_B & : y = \frac{1}{s^2}x - \frac{2}{s}
  \end{align*}
  である.この2本の直線の交点が$P(p,q)$だから,$l_A, l_B$を連立して
  \begin{align*}
    \frac{1}{m^2}p - \frac{2}{m}                                               & = \frac{1}{s^2}p - \frac{2}{s} \\
    \left(\frac{1}{m^2}-\frac{1}{s^2}\right) p                                 & = -\frac{2}{s} +  \frac{2}{m}  \\
    \left(\frac{1}{m}-\frac{1}{s}\right)\left(\frac{1}{m}+\frac{1}{s}\right) p & = -\frac{2}{s} +  \frac{2}{m}  \\
    \left(\frac{1}{m}+\frac{1}{s}\right) p                                     & = 2
  \end{align*}
  だから
  \begin{align}
    p & = \frac{2}{\left(\frac{1}{m}+\frac{1}{s}\right)} \nonumber \\
      & = \frac{2ms}{m+s} \label{eq:1}
  \end{align}
  である.この時$q$は$l_A$の式に代入して
  \begin{align}
    q
     & = \frac{1}{m^2} p - \frac{2}{m}                     \nonumber   \\
     & = \frac{1}{m^2} \cdot \frac{2ms}{m+s} - \frac{2}{m} \nonumber   \\
     & = \frac{2s}{m(m+s)} - \frac{2}{m}                   \nonumber   \\
     & = \frac{2s - 2(m+s)}{m(m+s)}                        \nonumber   \\
     & = \frac{-2m}{m(m+s)}                                  \nonumber \\
     & = \frac{-2}{m+s} \label{eq:2}
  \end{align}
  となる.

  以下\cref{eq:1,eq:2}を用いて三角形PABの面積を求める.
  まず,
  \begin{align*}
    \vec{PA} = \begin{pmatrix} m - p \\ -\frac{1}{m} - q \end{pmatrix} \\
    \vec{PB} = \begin{pmatrix} s - p \\ -\frac{1}{s} - q \end{pmatrix}
  \end{align*}
  であることに注意する.これら二つのベクトルの作る三角形の面積公式から,
  $\triangle PAB$の面積$f$として
  \begin{align}
    f
     & = \frac{1}{2} \left| (m-p)\left(-\frac{1}{s}-q\right) - (s-p)\left(-\frac{1}{m}-q\right) \right|  \nonumber                                                                         \\
     & = \frac{1}{2} \left| \left(m-\frac{2ms}{m+s}\right)\left(-\frac{1}{s}+\frac{2}{m+s}\right) - \left(s-\frac{2ms}{m+s}\right)\left(-\frac{1}{m}+\frac{2}{m+s}\right) \right|\nonumber \\
     & = \frac{1}{2} \left| \frac{m^2-ms}{m+s} \cdot \frac{-m+s}{s(m+s)} - \frac{s^2-ms}{m+s} \cdot \frac{-s+m}{m(m+s)} \right|     \nonumber                                              \\
     & = \frac{1}{2(m+s)^2} \left| \frac{m(m-s)(-m+s)}{s} - \frac{s(s-m)(-s+m)}{m} \right|           \nonumber                                                                             \\
     & = \frac{1}{2(m+s)^2} \left| \frac{-m(s-m)^2}{s} + \frac{s(s-m)^2}{m} \right|                           \nonumber                                                                    \\
     & = \frac{1}{2(m+s)^2} \left| -(s-m)^2 \left( \frac{m}{s} - \frac{s}{m} \right) \right|                           \nonumber                                                           \\
     & = \frac{1}{2(m+s)^2} \left| -(s-m)^2 \frac{m^2-s^2}{ms} \right|                                                          \nonumber                                                  \\
     & = \frac{1}{2(m+s)^2} \left|\frac{(s-m)^3 (m+s)}{ms}        \right|                                                                \nonumber                                         \\
     & = \frac{1}{2} \left|\frac{(s-m)^3}{ms(m+s)}  \right| \label{eq:3}
  \end{align}
  となる.

  次に,これを$t=pq$を用いて書き直すため,
  \begin{align*}
    \alpha & =m+s        \\
    \beta  & =s-m\, (>0)
  \end{align*}
  とおくと,
  \begin{align*}
    sm
     & =\frac{1}{4}\left[(m+s)^2-(s-m)^2\right] \\
     & =\frac{1}{4}(\alpha^2-\beta^2)
  \end{align*}
  だから,\cref{eq:3}に代入して
  \begin{align}
    f
     & =\frac{1}{2}\left|\frac{\beta^3}{\frac{1}{4}(\alpha^2-\beta^2)\alpha}\right| \nonumber \\
     & =2\left|\frac{\beta^3}{\alpha(\alpha^2-\beta^2)}\right| \label{eq:4}
  \end{align}
  を得る.一方で,$t$を$\alpha,\beta$で表すと
  \begin{align}
    t
     & = pq                                     \nonumber            \\
     & = -4\frac{ms}{(m+s)^2}                             \nonumber  \\
     & = -4\frac{\frac{1}{4}(\alpha^2-\beta^2)}{\alpha^2}  \nonumber \\
     & =-\frac{\alpha^2-\beta^2}{\alpha^2}                 \nonumber \\
     & =-1+\left(\frac{\beta}{\alpha}\right)^2 \label{eq:5}
  \end{align}
  $a = \alpha/\beta$と置いて\cref{eq:4,eq:5}に代入して$f$と$t$を表すと
  \begin{align*}
    f & =2\left|\frac{1}{a(a^2-1)}\right| \\
    t & =-1+\frac{1}{a^2}
  \end{align*}
  である.第二式から$|a|=\sqrt{\frac{1}{t+1}}$ だから,第一式に代入して
  \begin{align}
    f
     & =2\left|\frac{1}{\sqrt{\frac{1}{t+1}}\left(\frac{1}{t+1}-1\right)}\right| \nonumber \\                                    \\
     & =2\frac{\sqrt{t+1}(t+1)}{t} \quad (\because t>0) \label{eq:6}
  \end{align}
  を得る.これが三角形PABの面積を$t=pq$で表したものである.$\cdots$(答)

  \vspace{10pt}
  次に,この三角形の面積の最小値を求める.
  点Pが第一章限にあることから$t>0$での\cref{eq:6}の最小値を求めれば良い.
  新しく
  \begin{align*}
    x = t^{1/3} \quad (x>0)
  \end{align*}
  と置いて式を整理すると
  \begin{align*}
    f
     & = 2\sqrt{\frac{(t+1)^3}{t^2}}                                 \\
     & = 2\sqrt{\frac{(x^3+1)^3}{x^6}}                               \\
     & = 2\sqrt{\left(x + \frac{1}{x^2}\right)^3}                    \\
     & = 2\left(\frac{x}{2}+\frac{x}{2} + \frac{1}{x^2}\right)^{3/2} \\
     & \ge 2 \left(3\sqrt[3]{\frac{1}{2}\frac{1}{2}}\right)^{3/2}    \\
     & = 3\sqrt{3}
  \end{align*}
  である.ただし下から2行目の不等式は相加相乗平均の不等式による.
  等号成立は
  \begin{align*}
     & \frac{x}{2} =  \frac{1}{x^2} \\
     & x =\sqrt[3]{2}
  \end{align*}
  の時,すなわち
  \begin{align*}
    t = 2
  \end{align*}
  の時である.これは$t>0$を満たしているから,求める面積の最小値は
  \begin{align*}
    \min f = 3\sqrt{3}
  \end{align*}
  である.$\cdots$(答)

  \newpage
\end{multicols}
\end{document}