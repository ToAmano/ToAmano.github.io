% TODO :: 解説の正射影ベクトルのtikz図を追加
\documentclass[a4paper,10pt]{ltjsarticle}
\usepackage{luatexja}
\usepackage[hiragino-pron]{luatexja-preset}

\usepackage[truedimen,top=25truemm,bottom=20truemm,left=15truemm,right=15truemm]{geometry}
\setlength{\textwidth}{54\zw}
\setlength{\textheight}{73\zw}

\usepackage{amsmath,amssymb,ascmac}
\usepackage{enumerate}
\usepackage{multicol}
\usepackage{physics}
\usepackage{cleveref}
\usepackage{framed}
 \usepackage{caption}
 \usepackage{subcaption}
\usepackage{fancyhdr}
\usepackage{latexsym}
\usepackage{mathtools}
\usepackage{tikz}
\usepackage{tikz-3dplot}
\usepackage{pgfplots}
\pgfplotsset{compat=1.18}
\usetikzlibrary{intersections}
\usepgfplotslibrary{fillbetween}
 \usetikzlibrary{math}
 \usetikzlibrary{calc}
 \usetikzlibrary{angles}
 \usetikzlibrary{quotes}
 \usetikzlibrary{patterns}

% \usepackage{indent}

\usepackage{cases}
\usepackage{float}
\allowdisplaybreaks
\pagestyle{fancy}
\lhead{}
\chead{}
\rhead{東工大後期$2009$年$1$番}
\begin{document}

\begin{oframed}
  $a$ が与えられた実数のとき,
  $xyz$ 空間の点 C$(a,0,3)$ から出た光が球 $x^2+y^2+(z-1)^2 \leqq 1$ でさえぎられてできる $xy$ 平面上の影を $S$ とする.
  点 $(X,Y,0)$ が $S$ に含まれる条件を求めよ.
\end{oframed}
\setlength{\columnseprule}{0.4pt}
\begin{multicols}{2}
  {\bf[解]}
  $P(X,Y,0)$, 球の中心$O' (0,0,1)$とする.
  球の様子は\cref{fig:1}のようになる.

  \begin{figure}[H]
    \centering
    \begin{tikzpicture}
      \begin{axis}[
          axis lines=center,
          xlabel=$x$,
          ylabel=$y$,
          zlabel=$z$,
          xlabel style={at={(axis cs:1.2,0,0)}, anchor=west}, % S1軸ラベルの位置を調整
          ylabel style={at={(axis cs:0,1.2,0)}, anchor=south}, % S2軸ラベルの位置を調整
          zlabel style={at={(axis cs:0,0,1.2)}, anchor=south}, % S3軸ラベルの位置を調整
          xmin=0, xmax=4,
          ymin=0, ymax=4,
          zmin=0, zmax=4,
          xtick=\empty, ytick=\empty, ztick=\empty, % 目盛りを非表示
          view={120}{30}, % 視点を調整
          axis equal image, % 軸のスケールを等しくする
          clip=false, % ラベルがクリップされないように
        ]
        % 点光源 (a,0,3) の描画
        % a の具体的な値がないため、ここでは仮に a=2 とします。
        \coordinate (LightSource) at (2,0,4); % 例として a=2
        \coordinate (ShadowCenterProjection) at (-2,0,0);

        % 影の形状を示す楕円をxy平面に描画 (これはあくまで概念的なものです)
        \draw[fill=gray!20, thick,opacity = 0.4] (ShadowCenterProjection) ellipse (1.5cm and 1cm); % 仮の楕円
        \node at (ShadowCenterProjection) [below right] {P(X,Y,0)};

        \fill (LightSource) circle (2pt) node[above right] {$C(a,0,3)$};

        % 原点からの点線のパスを描画
        %\addplot3[dashed, thick, gray] coordinates {(0,0,0) (0.577,0.577,0.577)};

        % 特定の点を描画 (S/√3, S/√3, S/√3)
        % ここではS=1として単位球上の点を描画
        %\addplot3[only marks, mark=*, mark size=2pt, blue] coordinates {(0.577, 0.577, 0.577)};
        % xy平面の描画
        % \fill[blue!10, opacity=0.5] (-3,-3,0) rectangle (3,3,0);
        \draw[thick] (-3,-3,0) -- (3,-3,0) -- (3,3,0) -- (-3,3,0) -- cycle;
        % 軸の描画 (オプション)
        \draw[->] (0,0,0) -- (4,0,0) node[anchor=north east]{$x$};
        \draw[->] (0,0,0) -- (0,4,0) node[anchor=north west]{$y$};
        \draw[->] (0,0,0) -- (0,0,4) node[anchor=south]{$z$};
        % 単位球の一部 (x, y, z > 0) を描画
        \addplot3[
          %surf,
          domain=0:360, domain y=0:360,
          opacity=0.7, % 透明度を設定
          samples=20, samples y=20,
          % z buffer=sort, % 描画順序を正しくする
          % shader=flat, % シェーディングスタイル
          % mesh/color=gray!30, % 球の色
        ] ( {cos(x)*cos(y)}, {sin(x)*cos(y)}, {sin(y)+1} );

        % 影P(X,Y,0)の描画 (これは計算結果によって変わるため、ここでは概念的な線を描きます)
        % 点光源から球を通り、xy平面に落ちる光線を描く
        % 影の形状は楕円になることが示唆されているため、その中心と一部を描画
        % ここでは、点光源から球の中心を通る線を引いて、xy平面との交点を影の中心と仮定します。
        \draw[dashed, gray] (LightSource) -- (0,0,1); % 点光源から球の中心へ
        \path (LightSource) -- (0,0,1) coordinate[pos=1.5] (ShadowCenterProjection); % 影の中心の投影 (Z=0まで延長)
        \draw[dashed, gray] (0,0,1) -- (ShadowCenterProjection);
        \fill (ShadowCenterProjection) circle (1.5pt) node[below left] {(X,Y,0)};

      \end{axis}
    \end{tikzpicture}
    \caption{球と点光源の様子}
    \label{fig:1}
  \end{figure}


  線分$CP$が球に遮られるためには,
  線分$CP$と球の中心$O'$の距離が球の半径である$1$より小さければ良い.
  直線$CP$はパラメータ表示で
  \begin{align*}
    \mqty(x   \\y \\ z)
    = \mqty(a \\ 0 \\ 3) + t \mqty(X-a \\ Y \\ -3)\quad (t \in \mathbb{R})
  \end{align*}
  だから、CP上の点$Q(t)$として、
  \begin{align*}
      & |\vec{O'Q(t)}|^2                              \\
    = & \{ a+t(X-a) \}^2 + (tY)^2 + \{ 3-3t-1 \}^2    \\
    = & \{ a+t(X-a) \}^2 + (tY)^2 + (2-3t)^2          \\
    = & \{(X-a)^2+Y^2+9\}t^2 + 2\{a(X-a)-6\}t + a^2+4
  \end{align*}
  となる.ここで簡単のため
  \begin{align*}
    A & =(X-a)^2+Y^2+9(>0) \\
    B & =aX-a^2-6
  \end{align*}
  とおいて,式変形を続けると
  \begin{align*}
    |O'Q|^2= & A\left(t+\frac{B}{A}\right)+a^2+4-\frac{B^2}{A}
  \end{align*}
  だから,$t = - B/A$ の時最小値$a^2+4-B^2/A$をとる.
  よって条件は
  \begin{align*}
                    & \min |\vec{O'Q}|^2 \le 1                                    \\
    \Leftrightarrow & a^2+4 - \frac{\{a(X-a)-6\}^2}{(X-a)^2+Y^2+9} \le 1          \\
    \Leftrightarrow & a^2+3 - \frac{\{a(X-a)-6\}^2}{(X-a)^2+Y^2+9} \le 0          \\
    \Leftrightarrow & (a^2+3) \{(X-a)^2+Y^2+9\} - \{a(X-a)-6\}^2 \le 0            \\
    \Leftrightarrow & (a^2+3)(X-a)^2+(a^2+3)Y^2+9(a^2+3)                          \\
                    & - \{a^2(X-a)^2-12a(X-a)+36\} \le 0                          \\
    \Leftrightarrow & 3(X-a)^2+(a^2+3)Y^2+12a(X-a)+9a^2-9 \le 0                   \\
    \Leftrightarrow & 3\left[(X-a)+2a\right]^2+(a^2+3)Y^2+12a(X-a)-3(a^2+3) \le 0 \\
    \Leftrightarrow & \frac{(X+a)^2}{a^2+3} + \frac{Y^2}{3} \le 1
  \end{align*}
  という楕円となり,これが求めるべき条件である.従って答えは
  \begin{align*}
    \frac{(X+a)^2}{a^2+3} + \frac{Y^2}{3} \le 1
  \end{align*}
  である.$\cdots$(答)


  \vspace{10pt}
  {\bf[解説]}
  $CP$と$O'$の距離が$1$以下となる条件を立式する際,
  解答では直線$CP$上の点と$O'$の距離の最小値が$1$以下となるように立式した.
  ほぼ同じだが別の方法として,点と直線の距離を直接求めに行く方法も紹介しよう.
  一定検算に利用できるだろう.
  $O'$から直線$CP$におろした垂線の足を$H$とする.

  $CH$はCO'のCPへの正射影ベクトルだから
  \begin{align*}
    \vec{CH} = \frac{\vec{CP}\cdot\vec{CO'}}{|\vec{CP}|^2}\vec{CP}
  \end{align*}
  であるから,
  \begin{align*}
    \vec{O'H}
     & = \vec{CH}-\vec{CO}                                              \\
     & = \frac{\vec{CP}\cdot\vec{CO'}}{|\vec{CP}|^2}\vec{CP} - \vec{CO}
  \end{align*}
  となる.各ベクトルは
  \begin{align*}
    \vec{CO'} & = \mqty(-a  \\ 0 \\ -2)\\
    \vec{CP}  & = \mqty(X-a \\ Y \\ -3)
  \end{align*}
  だから,
  \begin{align*}
    \vec{O'H}
     & = \frac{-a(X-a)+6}{(X-a)^2+Y^2+9}\mqty(X-a \\ Y \\ -3) - \mqty(-a  \\ 0 \\ -2) \\
     & = \frac{-B}{A}\mqty(X-a                    \\ Y \\ -3) - \mqty(-a  \\ 0 \\ -2)
  \end{align*}
  この長さが$1$よりも小さければ良いので,
  \begin{align*}
    \left|\vec{O'H}\right|^2 \le 1
  \end{align*}
  が条件である.
  \begin{align*}
    \left|\vec{O'H}\right|^2
     & = \left(\frac{-B}{A}(X-a)+a\right)^2+\left(\frac{-B}{A}Y\right)^2 + \left(\frac{3B}{A}+2\right)^2 \\
     & = \frac{1}{A^2}\left(-B(X-a)+aA\right)^2+\frac{B^2Y^2}{A^2} + \frac{1}{A^2}\left(3B+2A\right)^2   \\
  \end{align*}
  を得る.さらに整理すると
  \begin{align*}
    \left|\vec{O'H}\right|^2
     & = \frac{B^2}{A^2}\left((X-a)^2+Y^2+9\right)
    + \frac{-2aAB(X-a)}{A^2} + a^2 + 4 + \frac{12AB}{A^2}  \\
     & = \frac{B^2}{A} + a^2 + 4 + \frac{-2B(a(X-a)-6)}{A} \\
     & = \frac{B^2}{A} + a^2 + 4 + \frac{-2B^2}{A}         \\
     & = -\frac{B^2}{A} + a^2 + 4
  \end{align*}
  となり,解答の最小値と一致する.

  これでふた通りの方法でもとまった.
  導出を見ればわかるようにどちらも計算量には大差ないので,お好みで使い分けると良いだろう.


  \newpage
\end{multicols}
\end{document}