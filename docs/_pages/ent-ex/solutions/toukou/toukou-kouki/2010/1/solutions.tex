\documentclass[a4paper,10pt]{ltjsarticle}
\usepackage{luatexja}
\usepackage[hiragino-pron]{luatexja-preset}

\usepackage[truedimen,top=25truemm,bottom=20truemm,left=15truemm,right=15truemm]{geometry}
\setlength{\textwidth}{54\zw}
\setlength{\textheight}{73\zw}

\usepackage{amsmath,amssymb,ascmac}
\usepackage{enumerate}
\usepackage{multicol}
\usepackage{physics}
\usepackage{cleveref}
\usepackage{framed}
 \usepackage{caption}
 \usepackage{subcaption}
\usepackage{fancyhdr}
\usepackage{latexsym}
\usepackage{mathtools}
\usepackage{tikz}
\usepackage{tikz-3dplot}
\usepackage{pgfplots}
\pgfplotsset{compat=1.18}
\usetikzlibrary{intersections}
\usepgfplotslibrary{fillbetween}
 \usetikzlibrary{math}
 \usetikzlibrary{calc}
 \usetikzlibrary{angles}
 \usetikzlibrary{quotes}
 \usetikzlibrary{patterns}

% \usepackage{indent}

\usepackage{cases}
\usepackage{float}
\allowdisplaybreaks
\pagestyle{fancy}
\lhead{}
\chead{}
\rhead{東工大後期$2010$年$1$番}
\begin{document}

\begin{oframed}
  $a, b, t$は実数で, $a \ge 0 > b$とする.
  次の漸化式により, 数列$a_n, b_n$ $(n=1, 2, \dots)$を定める.
  \[ a_1=a, \quad b_1=b \]
  \[ a_{n+1} = \left(\frac{t}{2} + \frac{5}{t^2+1}\right) a_n + \left(\frac{t}{2} - \frac{5}{t^2+1}\right) b_n, \quad b_{n+1} = \left(\frac{t}{2} - \frac{5}{t^2+1}\right) a_n + \left(\frac{t}{2} + \frac{5}{t^2+1}\right) b_n \]
  \begin{enumerate}
    \item $a_n$を$a, b, t, n$を用いて表せ.
    \item $n \to \infty$とするとき, $a_n$が収束するための$a, b, t$についての必要十分条件を求めよ.
  \end{enumerate}
\end{oframed}
\setlength{\columnseprule}{0.4pt}
\begin{multicols}{2}
  {\bf[解]}

  (1)
  $p \ge 0, b \ge 0$
  $a, b, t \in \mathbb{R}$
  表記の簡潔さのため,
  \begin{align*}
    X & = \frac{1}{2}t + \frac{5}{t^2+1} \\
    Y & = \frac{1}{2}t - \frac{5}{t^2+1}
  \end{align*}
  とおくと,$X,Y$は
  \begin{align*}
    X+Y & =t                \\
    X-Y & =\frac{10}{t^2+1}
  \end{align*}
  を満たす.題意の漸化式の辺々足し引きして,
  \begin{align*}
    \begin{dcases}
      a_{n+1} + b_{n+1} = (X+Y)(a_n+b_n) = t(a_n+b_n) \\
      a_{n+1} - b_{n+1} = (X-Y)(a_n-b_n) = \frac{10}{t^2+1}(a_n-b_n)
    \end{dcases}
  \end{align*}
  だから,$a_n+b_n$および$a_n-b_n$は等比級数である.
  これと初期条件 $a_1=a$, $b_1=b$ から,
  一般項は
  \begin{align*}
    \begin{dcases}
      a_n + b_n & = t^{n-1}(a+b)                             \\
      a_n - b_n & = \left(\frac{10}{t^2+1}\right)^{n-1}(a-b)
    \end{dcases}
  \end{align*}
  $a_n$を求めるために辺々足して
  \begin{align*}
    a_n = \frac{1}{2}(a+b)t^{n-1} + \frac{1}{2}(a-b)\left(\frac{10}{t^2+1}\right)^{n-1}
  \end{align*}
  を得る.$\cdots$(答)

  \vspace{10pt}
  (2)
  表記の簡潔さのため$A = \frac{a+b}{2}$, $B = \frac{a-b}{2}$ とする.
  題意の条件$a \ge 0 > b$から,$B>0$である.
  さらに $s = \frac{10}{t^2+1}$ とすると (1) で得た一般項は
  \begin{align*}
    a_n = A \cdot t^{n-1} + B \cdot S^{n-1}
  \end{align*}
  となる.
  以下 $a_n$ が収束する条件を$t$と$s$の大小関係に注目して考える.
  $B>0$, $s>0$ であることに注意する.
  まず $A \ne 0$ のとき,
  \begin{enumerate}
    \item[$1^\circ$] $|t| < s \iff -2 < t < 2$ の時
          \begin{align*}
            a_n = s^{n-1} \left\{ B + A\left(\frac{t}{s}\right)^{n-1} \right\}
          \end{align*}
          に於いて,
          \begin{align*}
            \left\{ B+A\left(\frac{t}{S}\right)^{n-1} \right\} \xrightarrow{n \to \infty} B (\ne 0)
          \end{align*}
          だから,$a_n$ の収束条件は
          \begin{align*}
            -1 < s \le 1 \iff t \le -3, 3 \le t
          \end{align*}
          となる.$-2 < t < 2$ と同時にこの条件を満たす $t$ はないから,この領域で$a_n$は収束しない.
    \item[$2^\circ$] $|t| = s \iff t = \pm 2$ の時

          まず$t=2$の時,$a_n = a \cdot 2^{n-1}$だから,$a\ge 0$より収束条件は$a=0$である.

          次に$t=-2$の時,$a_n = A(-2)^{n-1} + B(-2)^{n-1}$だから,$B>0$より$a_n$は収束しない.
    \item[$3^\circ$] $|t| > s \iff t < -2 \text{ or } 2 < t$ の時
          \begin{align*}
            a_n = t^{n-1} \left\{ A + B\left(\frac{S}{t}\right)^{n-1} \right\}
          \end{align*}
          である.
          \begin{align*}
            \left\{ A+B\left(\frac{s}{t}\right)^{n-1} \right\} \xrightarrow{n \to \infty} A (\ne 0)
          \end{align*}
          より収束条件は$-1 < t \le 1$ だが $t < -2 \text{ or } 2 < t$ と同時にこの条件を満たす$t$は存在しない.
          よってこの領域で$a_n$は収束しない.
  \end{enumerate}

  次に $A=0 \iff a+b=0$ の時,
  \begin{align*}
    a_n = B \cdot s^{n-1}
  \end{align*}
  より,$a_n$が収束する条件は
  \begin{align*}
    -1 < s \le 1 \iff t \le -3, 3 \le t
  \end{align*}
  である.

  以上4つの場合分けから,もとめる条件は
  \begin{align*}
    (a+b=0 \land (t \le -3 \text{ or } 3 \le t)) \text{ または } \\
    (a+b \ne 0 \land a=0 \land t = 2)
  \end{align*}
  である.$\cdots$(答)
  \vspace{10pt}
  {\bf[解説]}

  \newpage
\end{multicols}
\end{document}