% TODO :: 図1の更新
\documentclass[a4paper,10pt]{ltjsarticle}
\usepackage{luatexja}
\usepackage[hiragino-pron]{luatexja-preset}

\usepackage[truedimen,top=25truemm,bottom=20truemm,left=15truemm,right=15truemm]{geometry}
\setlength{\textwidth}{54\zw}
\setlength{\textheight}{73\zw}

\usepackage{amsmath,amssymb,ascmac}
\usepackage{enumerate}
\usepackage{multicol}
\usepackage{cleveref}
\usepackage{framed}
\usepackage{fancyhdr}
\usepackage{latexsym}
\usepackage{mathtools}
\usepackage{tikz}
\usepackage{tikz-3dplot}
\usepackage{pgfplots}
 \usetikzlibrary{math}
 \usetikzlibrary{calc}
 \usetikzlibrary{angles}
 \usetikzlibrary{quotes}
% \usepackage{indent}
\usepackage{caption}
\usepackage{subcaption}
\usepackage{cases}
\usepackage{float}
\usepackage{cases}
  \usepackage{caption}
  \usepackage[subrefformat=parens]{subcaption}
\allowdisplaybreaks
\tdplotsetmaincoords{70}{110}
\pagestyle{fancy}
\lhead{}
\chead{}
\rhead{東工大後期$2003$年$2$番}
\begin{document}
\begin{oframed}
  $m$ を $0$ 以上の整数とする. 直線 $2x+3y=m$ 上の点 $(x,y)$ で, $x, y$ がともに $0$ 以上の整数であるものの個数を $N(m)$ とする.
  \begin{enumerate}
    \item $N(m+6)=N(m)+1$ を証明せよ.
    \item $N(m)=1-m+\left[\frac{m}{2}\right]+\left[\frac{2m}{3}\right]$ を証明せよ. ただし, $[a]$ は $a$ 以下の最大の整数を表すものとする.
  \end{enumerate}
\end{oframed}


\setlength{\columnseprule}{0.4pt}
\begin{multicols}{2}
  {\bf[解]}
  題意の直線$2x+3y=m$は図のようになる.

  \begin{figure}[H]
    \centering
    \begin{tikzpicture}
      % Axes
      \draw (0,0) -- (4,0) node[right] {$x$};
      \draw (0,0) -- (0,4) node[above] {$y$};

      % Line 2x+3y=m
      % Intercepts:
      % When x=0, 3y=m => y=m/3
      % When y=0, 2x=m => x=m/2
      % Assuming m is a positive integer for the first quadrant
      % Let's choose an arbitrary m for visualization, e.g., m=6 for clear intercepts
      % If m=6, then y-intercept is 2 and x-intercept is 3.
      % The line is drawn from (0, m/3) to (m/2, 0)

      % Define m for calculations and labels.
      % In a real application, 'm' might be a variable. For this plot,
      % we'll pick a value that fits well, e.g., m=6.
      % We can represent m symbolically in labels.
      \def\mval{6} % Example value for m to draw the line

      \draw[thick] (0, \mval/3) coordinate (y_intercept) -- (\mval/2, 0) coordinate (x_intercept);

      % Labels for intercepts
      \node[left] at (y_intercept) {$\frac{m}{3}$};
      \node[below] at (x_intercept) {$\frac{m}{2}$};
      \node[above] at (2,2) {$2x+3y=m$};

    \end{tikzpicture}
  \end{figure}

  さて,$x$,$y$をパラメータ表示するため,
  \begin{align*}
    \begin{dcases}
      2x+3y=m \\
      -2m+3m=m
    \end{dcases}
  \end{align*}
  を辺々引いて,
  \begin{align*}
    2(x+m)+3(y-m)=0
  \end{align*}
  $2$ と $3$ は互いに素だから, $k \in \mathbb{Z}$ として,
  \begin{align*}
     & \begin{dcases}
         x+m = 3k \\
         y-m = -2k
       \end{dcases}         \\
    \therefore
     & (x,y) = (3k-m, -2k+m)
  \end{align*}
  と表すことができる.
  $x$,$y$が非負だから,$k$に対する条件は
  \begin{align*}
    \frac{m}{3} \le k \frac{m}{2}
  \end{align*}
  であり,従って$N(m)$は
  \begin{align}
    N(m) = \left(\frac{m}{3} \le k \le \frac{m}{2} \text{をみたす } k \in \mathbb{Z} \text{の数}\right) \label{eq:1}
  \end{align}
  と言い換えることができる.

  \vspace{10pt}
  (1)
  \cref{eq:1}に注意して $m$ を $6$ で割ったあまりで場合分けする.
  $t \in \mathbb{N}$ として,
  \begin{align}
    \begin{cases}
      m=6t \text{ の時,}   & N(m) = 3t-2t+1 = t+1       \\
      m=6t-1 \text{ の時,} & N(m) = (3t-1)-(2t-1) = t   \\
      m=6t-2 \text{ の時,} & N(m) = (3t-1)-(2t-1) = t   \\
      m=6t-3 \text{ の時,} & N(m) = (3t-2)-(2t-2) = t   \\
      m=6t-4 \text{ の時,} & N(m) = (3t-2)-(2t-2) = t   \\
      m=6t-5 \text{ の時,} & N(m) = (3t-3)-(2t-2) = t+1
    \end{cases}\label{eq:2}
  \end{align}
  を得る.$N(m+6)$は$t\to t+1$とした時に該当し,このときいずれも$N(m)$は$1$だけ増えるから,
  たしかに $N(m+6) = N(m)+1$ が成立する.
  よって題意は示された.

  \vspace{10pt}
  (2)
  簡単のため
  \begin{align*}
    f(m) = 1-m+\left[\frac{m}{2}\right]+\left[\frac{2}{3}m\right]
  \end{align*}
  とおく. (1)と同様に$m$を$6$で割った余りで場合わけすると,
  \begin{align}
    \begin{dcases}
      m=6t \text{ の時,} f(m) = 1-6t+3t+4t = t+1             \\
      m=6t-1 \text{ の時,} f(m) = 1-(6t-1)+(3t-1)+(4t-1) = t \\
      m=6t-2 \text{ の時,} f(m) = 1-(6t-2)+(3t-1)+(4t-2) = t \\
      m=6t-3 \text{ の時,} f(m) = 1-(6t-3)+(3t-2)+(4t-2) = t \\
      m=6t-4\text{ の時,} f(m) = 1-(6t-4)+(3t-2)+(4t-3) = t  \\
      m=6t-5 \text{ の時,} f(m) = 1-(6t-5)+(3t-3)+(4t-4) = t-1
    \end{dcases}\label{eq:3}
  \end{align}
  である.
  \cref{eq:2,eq:3}を比較することで,
  たしかに $f(m)=N(m)$ であることがわかる.
  以上から題意は示された.


  \newpage
\end{multicols}
\end{document}