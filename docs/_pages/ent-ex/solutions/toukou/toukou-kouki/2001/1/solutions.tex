\documentclass[a4paper,10pt]{ltjsarticle}
\usepackage{luatexja}
\usepackage[hiragino-pron]{luatexja-preset}

\usepackage[truedimen,top=25truemm,bottom=20truemm,left=15truemm,right=15truemm]{geometry}
\setlength{\textwidth}{54\zw}
\setlength{\textheight}{73\zw}

\usepackage{amsmath,amssymb,ascmac}
\usepackage{enumerate}
\usepackage{multicol}
\usepackage{physics}
\usepackage{cleveref}
\usepackage{framed}
 \usepackage{caption}
 \usepackage{subcaption}
\usepackage{fancyhdr}
\usepackage{latexsym}
\usepackage{mathtools}
\usepackage{tikz}
\usepackage{tikz-3dplot}
\usepackage{pgfplots}
\pgfplotsset{compat=1.18}
\usetikzlibrary{intersections}
\usepgfplotslibrary{fillbetween}
 \usetikzlibrary{math}
 \usetikzlibrary{calc}
 \usetikzlibrary{angles}
 \usetikzlibrary{quotes}
 \usetikzlibrary{patterns}
% \usepackage{indent}
\usepackage{cases}
\usepackage{float}
\allowdisplaybreaks
\pagestyle{fancy}
\lhead{}
\chead{}
\rhead{東工大後期$2001$年$1$番}
\begin{document}
\begin{oframed}
  $n=1,2,3,\dots$ に対して $a_n = \tan(11n)$ とおく.このとき,次の (1)~(4) を示せ.
  ただし,$\pi = 3.14159265\dots$ は円周率である.

  \begin{enumerate}
    \item $\dfrac{\pi}{711} < 11 - \dfrac{7\pi}{2} < \dfrac{\pi}{709}$.
    \item $a_1 < 0 < a_2$.
    \item $a_1, a_3, a_5, a_7, \dots, a_{707}, a_{709}$ は増加数列である.
    \item 無限数列 $a_1, a_3, a_5, a_7, \dots$ は増加数列ではない.
  \end{enumerate}
\end{oframed}
\setlength{\columnseprule}{0.4pt}
\begin{multicols}{2}
  {\bf[解]}

  (1) 与式を同値変形すると
  \begin{align}
     & \dfrac{\pi}{711} < 11 - \dfrac{7\pi}{2} < \dfrac{\pi}{709} \nonumber \\
    \iff
     & \frac{15598}{4965} < \pi < \frac{15642}{4979}\label{eq:1}
  \end{align}
  だから,これを示す.両辺計算すると
  \begin{align*}
    \mathrm{LHS} = \frac{15598}{4965} = 3.141591... < \pi \\
    \mathrm{RHS} = \frac{15642}{4979} = 3.141594... > \pi
  \end{align*}
  となり,\cref{eq:1}は示された.$\cdots$(答)


  \vspace{10pt}
  (2)
  まず$a_1=\tan(11)$について考える.
  $\tan \theta$ は周期 $\pi$ の周期関数だから $\tan 11 = \tan (11-3\pi)$ である.
  (1) で示された不等式の各辺に$\pi/2$を足して
  \begin{align*}
    \dfrac{\pi}{711} + \frac{\pi}{2} < 11 - 3\pi < \dfrac{\pi}{709} + \frac{\pi}{2}
  \end{align*}
  であり,従って
  \begin{align*}
    \frac{\pi}{2} < 11 - 3\pi < \pi
  \end{align*}
  が成り立つ.この範囲では$\tan\theta$は負であるから,
  \begin{align}
    a_1 = \tan(11) = \tan\left(11-3\pi\right) < 0 \label{eq:2}
  \end{align}
  である.

  次に$a_2=\tan(22)$について考える.同様に(1)で示した不等式の各辺$2$倍して
  \begin{align*}
    \dfrac{2\pi}{711} < 22 - 7\pi < \dfrac{2\pi}{709}
  \end{align*}
  であり,従って
  \begin{align*}
    0 < 22 - 7\pi < \dfrac{\pi}{2}
  \end{align*}
  を得る.この範囲では$\tan\theta$は正であり,$\tan\theta$の周期性より$\tan 22 = \tan (22-7\pi)$だから,
  \begin{align}
    a_2 = \tan(22) = \tan\left(22-7\pi\right) > 0 \label{eq:3}
  \end{align}
  を得る.以上\cref{eq:2,eq:3}より
  \begin{align*}
    a_1 < 0 < a_2
  \end{align*}
  であり題意は示された.$\cdots$(答)

  \vspace{10pt}
  (3)
  簡単のため
  \begin{align*}
    \theta_n = 22n-11-7n\pi
  \end{align*}
  とおく.$\tan\theta$の周期性より
  \begin{align}
    a_{2n-1} = \tan\theta_n \label{eq:6}
  \end{align}
  となることに注意する.

  (1) で示した式の各辺 $2n-1$ ($>0$) 倍して,
  \begin{align}
     & \frac{2n-1}{709}\pi < 22n-11 - \frac{\pi}{2}(2n-1)\pi < \frac{2n-1}{709}\pi  \nonumber                    \\
     & \frac{2n-1}{711}\pi + \frac{7}{2}\pi < 22n-11 - 7n\pi < \frac{2n-1}{709}\pi + \frac{7}{2}\pi \label{eq:4}
  \end{align}
  である.$n=1,2,\cdots,355$ の時,
  \begin{align*}
    0 < \frac{2n-1}{711}\pi < \frac{2n-1}{709}\pi \le \pi
  \end{align*}
  である.従って,
  \begin{align*}
    \frac{7}{2}\pi<\theta_n<\frac{9}{2}\pi
  \end{align*}
  である.この区間では$\tan\theta_n$は単調増加である.

  さらに,$\theta_n$は短調増加であることが以下のように示せる.
  \cref{eq:4}より,$n=1,2,\cdots,354$に対して
  \begin{align}
     & \theta_n < \frac{2n-1}{709}\pi + \frac{1}{2}\pi < \frac{2n+1}{711}\pi + \frac{1}{2}\pi < \theta_{n+1} \nonumber \\
    \therefore
     & \theta_n < \theta_{n+1} \label{eq:5}
  \end{align}
  である.ただし
  \begin{align*}
    \frac{2n-1}{711} - \frac{2n-1}{709} = \frac{(2n-1)(-4n+1420)}{711 \cdot 709} \ge \frac{709 \cdot 0}{711 \cdot 709} = 0
  \end{align*}
  を利用した.

  以上\cref{eq:4,eq:5}より,
  \begin{align*}
    \frac{7\pi}{2} < \theta_1 < \theta_2 < \cdots < \theta_{355} < \frac{9\pi}{2}
  \end{align*}
  が成り立つ.この区間内で$\tan\theta_n$は単調増加だから,
  \begin{align*}
    \tan \theta_1 < \tan \theta_2 < \cdots < \tan \theta_{355}
  \end{align*}
  が成り立つ.従って\cref{eq:6}より$n=355$のとき$2n+1=709$に注意して
  \begin{align*}
    a_1 < a_3 < \cdots < a_{709}
  \end{align*}
  となる.よって題意は示された.$\cdots$(答)

  \vspace{10pt}
  (4)

  $a_{711}<0<a_{709}$であることを示せば,題意は示される.
  (3) の\cref{eq:4}で$n=355$として
  \begin{align*}
     & \frac{709}{711}\pi + \frac{7}{2}\pi < \theta_{355} < \frac{709}{709}\pi + \frac{7}{2}\pi \\
    \therefore
     & 4\pi < \theta_{355} < 4\pi + \frac{1}{2}\pi
  \end{align*}
  この区間で$0<\tan\theta$だから,
  \begin{align*}
    0 < \tan\theta_{355} = a_{709}
  \end{align*}
  だから$0<a_{709}$である.

  次に$a_{711}$について,\cref{eq:4}で$n=356$として
  \begin{align*}
     & \frac{711}{711}\pi + \frac{7}{2}\pi < \theta_{356} < \frac{711}{709}\pi + \frac{7}{2}\pi \\
    \therefore
     & 4\pi + \frac{1}{2}\pi < \theta_{356} < 4\pi + \pi
  \end{align*}
  である.この区間で$\tan\theta$は負だから
  \begin{align*}
    a_{711} = \tan\theta_{356} < 0
  \end{align*}
  である.以上より$a_{711}<0<a_{709}$であり,$a_{2n-1}, (n=1,2,\cdots)$が増加列ではないことが示された.$\cdots$(答)

  \vspace{10pt}
  {\bf[解説]}

  \newpage
\end{multicols}
\end{document}