\documentclass[a4paper,10pt]{ltjsarticle}
\usepackage{luatexja}
\usepackage[hiragino-pron]{luatexja-preset}

\usepackage[truedimen,top=25truemm,bottom=20truemm,left=15truemm,right=15truemm]{geometry}
\setlength{\textwidth}{54\zw}
\setlength{\textheight}{73\zw}

\usepackage{amsmath,amssymb,ascmac}
\usepackage{enumerate}
\usepackage{multicol}
\usepackage{cleveref}
\usepackage{framed}
\usepackage{fancyhdr}
\usepackage{latexsym}
\usepackage{mathtools}
\usepackage{tikz}
\usepackage{tikz-3dplot}
\usepackage{pgfplots}
 \usetikzlibrary{math}
 \usetikzlibrary{calc}
 \usetikzlibrary{angles}
 \usetikzlibrary{quotes}
% \usepackage{indent}
\usepackage{caption}
\usepackage{subcaption}
\usepackage{cases}
\usepackage{float}
\usepackage{cases}
  \usepackage{caption}
  \usepackage[subrefformat=parens]{subcaption}
\allowdisplaybreaks
\tdplotsetmaincoords{70}{110}
\pagestyle{fancy}
\lhead{}
\chead{}
\rhead{東工大後期$2002$年$2$番}
\begin{document}
\begin{oframed}
  $xy$ 平面上に原点 $O$ を中心とする半径 $1$ の円 $C$ がある.
  $C$ を底面,$(0,0,\sqrt{3})$ を頂点とする直円すい $S$ を考える.
  点 $P(1,0,0)$ および $Q(-2,0,0)$ をとる.
  さらに,動点 $M(\cos\theta,\sin\theta,0)$ ($0 \leq \theta < 2\pi$) を線分 $MQ$ が $M$ 以外に $C$ と交わらないように動かす.

  \begin{enumerate}
    \item
          $\theta$ のとりうる値の範囲を求めよ.
    \item
          点 $P$ から動点 $M$ までは直円すい $S$ の側面を通り,$M$ からは直線にそって点 $Q$ へ向かう道を考える.このような $P$ から $Q$ までの全ての道の長さの最小値を求 めよ.
  \end{enumerate}
\end{oframed}


\setlength{\columnseprule}{0.4pt}
\begin{multicols}{2}
  {\bf[解]}

  $xyz$空間の原点$O(0,0,0)$, 円錐の頂点を$R(0,0,\sqrt{3})$とする.
  円錐の概形は\cref{fig:1}となる.

  \begin{figure}[H]
    \centering
    \tdplotsetmaincoords{70}{110} % 視点の角度 (仰角, 方位角)
    \begin{tikzpicture}[tdplot_main_coords,scale=2.0]
      \pgfsetlinewidth{0.8pt}

      % 軸の描画
      \draw[-stealth] (-2,0,0) -- (2,0,0) node [pos=1.1] {x};
      \draw[-stealth] (0,-2,0) -- (0,2,0) node [pos=1.1] {y};
      \draw[-stealth] (0,0,-0.5) -- (0,0,2) node [pos=1.1] {z};

      % 原点のラベル
      \node at (0,0,0) [anchor=north west] {O};

      % 頂点
      \coordinate (apex) at (0,0,{sqrt(3)});
      \node at (apex) [anchor=south west] {$\sqrt{3}$};
      \node at (apex) [anchor=south east] {$R$};

      % 底面の円 (xy平面上)
      \tdplotsetrotatedcoords{0}{0}{0}
      \draw [tdplot_rotated_coords, thick] (0,0) circle (1);

      % 底面の補助線 (x軸上)
      \draw [dashed] (1,0,0) -- (-1,0,0);
      \node [tdplot_rotated_coords, anchor=west] at (1,0) {P};
      \node [tdplot_rotated_coords, anchor=east] at (-2,0) {Q};

      % 底面の補助線 (y軸上)
      \draw [dashed] (0,1,0) -- (0,-1,0);
      \node [tdplot_rotated_coords, anchor=south] at (-0.707,0.707) {M};

      % 円錐の側面
      \draw (apex) -- (1,0,0);
      \draw (apex) -- (0,1,0);
      \draw (apex) -- (-1,0,0);
      \draw (apex) -- (0,-1,0);

      % 円錐の側面 (背面を破線で表現 - 概略)
      \draw [dashed] (apex) -- ({cos(135)},{sin(135)},0);
      \draw [dashed] (apex) -- ({cos(225)},{sin(225)},0);

      % 母線の長さを表現 (概略)
      % \draw (0.3,0,1.2) -- (1,0,0);
      % \node [anchor=south west] at (0,-0.5,1.2) {$|PM|=2$};
    \end{tikzpicture}
    \caption{円錐の様子}
    \label{fig:1}
  \end{figure}


  (1)
  $xy$平面で考える.
  線分 $MQ$ が $M$ 以外に $C$ と交わらないということは,
  $\theta$の境界の値では,$MQ$と$C$が接するので,まずはこの場合を考える.
  この様子を\cref{fig:2}に示す.

  \begin{figure}[H]
    \centering
    \begin{tikzpicture}
      \begin{axis}[
          axis lines=middle,
          xmin=-2.5, xmax=2.5,
          ymin=-2.5, ymax=2.5,
          xtick={-2,-1,1}, % x軸の目盛り
          ytick={-1,1}, % y軸の目盛り (円の半径と関連)
          xlabel=$x$,
          ylabel=$y$,
          enlargelimits=false, % 軸の限界を拡大しない
          grid=none, % グリッドは描画しない
          every axis x label/.style={at={(ticklabel* cs:1)},anchor=north east}, % x軸ラベルの位置
          every axis y label/.style={at={(ticklabel* cs:1)},anchor=south west}, % y軸ラベルの位置
          xmajorgrids=false, % x軸のグリッド線なし
          ymajorgrids=false, % y軸のグリッド線なし
        ]

        % 円 x^2 + y^2 = 1
        \addplot[domain=0:360, samples=100, blue, thick] ({cos(x)}, {sin(x)});

        % 点 (0,0) を示す
        \node[label={-90:O}] at (axis cs:0,0) {};

        % 接線 y = 1/2(x-2)
        \addplot[domain=-2.5:2, red, dashed] {(x+2)/sqrt(3)};

        % 接線 y = -1/2(x-2)
        \addplot[domain=-2.5:2, red, dashed] {-(x+2)/sqrt(3)};

        % 描画された図に一致する接点と直角マークを追加 (近似的に配置)
        % 上の接点
        \coordinate (P1) at (axis cs:-1/2, {sqrt(3)/2});
        \fill[black] (P1) circle (1.5pt);

        % 下の接点
        \coordinate (P2) at (axis cs:-1/2, {-sqrt(3)/2});
        \fill[black] (P2) circle (1.5pt);

        % P
        \coordinate (P) at (axis cs:1, 0);
        \fill[black] (P) circle (1.5pt);

        % Q
        \coordinate (Q) at (axis cs:-2, 0);
        \fill[black] (Q) circle (1.5pt);


        % 接線と半径の直角マーク
        % 上の接点
        \draw[black] (axis cs:0,0) -- (P1);
        \draw[fill=white, draw=black] (0.2, 0) -- (0.2, 0.2) -- (0, 0.2) -- cycle; % 円に近い直角

        % 下の接点
        \draw[black] (axis cs:0,0) -- (P2);
        \draw[fill=white, draw=black] (0.2, 0) -- (0.2, -0.2) -- (0, -0.2) -- cycle; % 円に近い直角

        \node[above right] at (P) {$P$};
        \node[above] at (P1) {$M$};
        \node[above] at (Q) {$Q$};

      \end{axis}
    \end{tikzpicture}
    \caption{$xy$平面での$P,Q,M$の様子}
    \label{fig:2}
  \end{figure}

  $M$での$C$の接線の方程式は
  \begin{align*}
    \cos\theta x + \sin\theta y = 1
  \end{align*}
  であり,これが$Q(-2,0)$を通るので,
  \begin{align*}
    -2\cos\theta = 1 \\
    \therefore
    \theta = \pm \frac{4\pi}{3}
  \end{align*}
  である.
  従って$\theta$の取り得る範囲はこれらの$\theta$の時より点$M$が$Q$側になる場合で,
  \begin{align*}
    \frac{2\pi}{3} \le \theta \le \frac{4\pi}{3}
  \end{align*}
  である.$\cdots$(答)

  \vspace{10pt}
  (2)
  Sの展開図上で線分PMとなる曲線を$D$とする.
  PからMまでの最短経路は, PからMまで$D$上通り,
  MからQまで直線MQを通る経路である.$\cdots$(*)

  最短経路の長さ$f(\theta)$とすると
  \begin{align}
    f(\theta) = |PM| + |QM|\label{eq:1}
  \end{align}
  である.

  まず線分PMの長さを求める.
  Sの側面の展開図は\cref{fig:3}である.

  \begin{figure}[H]
    \centering
    \begin{tikzpicture}[scale=2.5]
      % 点R
      \coordinate (R) at (0,0);
      \node at (R) [below] {R};

      % 点P (左)
      \coordinate (P1) at (1,0);
      \node at (P1) [below] {P};

      % 点P (右)
      \coordinate (P2) at (-1,0);
      \node at (P2) [below] {P};

      % 点O
      \coordinate (O) at (0,1);

      % 弧を描画
      \draw (P1) arc [radius=1, start angle=0, end angle=180];

      % 点M (弧上の適当な点 - 中心角を考慮)
      \coordinate (M) at ({1/sqrt(2)},{1/sqrt(2)});
      \node at (M) [above] {M};

      % 線分P1P2
      \draw (P1) -- (P2);

      % 線分RM
      \draw (R) -- (M);

      % 線分MP2
      \draw [ultra thick] (M) -- (P1);

      % 直角マーク
      \draw (-0.1,0) -- (-0.1,0.1) -- (0.1,0.1) -- (0.1,0);

      % 角の記号
      \draw (0.2,0) arc (0:45:0.2);
      \node at (0.25,0.1) {$\frac{\theta}{2}$};

      % 破線 (中心からRへ)
      \draw [dashed] (O) -- (R);
    \end{tikzpicture}
    \caption{円錐側面の展開図}
    \label{fig:3}
  \end{figure}

  対称性から
  \begin{align}
    0 \le \theta \le \pi \label{eq:4}
  \end{align}
  として考える.

  まず,PMの劣弧の長さは,$\angle POQ=\theta$より
  \begin{align*}
    \text{劣弧PM} & = \theta
  \end{align*}
  だから,円錐の側面の長さが$PR=2$であることにも留意すると
  \begin{align*}
    \angle \text{MRP} & = \frac{\theta}{2}
  \end{align*}
  となるので,展開図上のPMの長さ$l(\theta)$は
  \begin{align}
    l(\theta) & = 4 \sin \frac{\theta}{2} \label{eq:2}
  \end{align}
  となる.

  次に,線分QMの長さを求める.これは各点の座標から,
  \begin{align}
    |\text{QM}|
     & = \sqrt{(\cos\theta+2)^2 + \sin^2\theta} \\
     & = \sqrt{5+4\cos\theta} \label{eq:3}
  \end{align}
  となる.

  以上\cref{eq:2,eq:3}を\cref{eq:1}に代入すると,$\theta$を与えた時の最短経路の長さ$f(\theta)$は
  \begin{align}
    f(\theta)
     & = l(\theta) + |\text{QM}|                                   \\
     & = 4\sin\frac{\theta}{2} + \sqrt{5+4\cos\theta} \label{eq:5}
  \end{align}
  となる.
  (1)および\cref{eq:4}から,この関数の
  \begin{align}
    \frac{2\pi}{3} \le \theta \le \pi \label{eq:6}
  \end{align}
  での最小値をもとめにば良い.新しく
  \begin{align*}
    t=\cos\frac{\theta}{2}
  \end{align*}
  とおくと,半角および倍角の公式から
  \begin{align*}
    \sin\frac{\theta}{4}
     & = \sqrt{\frac{1-\cos\frac{\theta}{2}}{2}} \quad (\sin\frac{\theta}{4}\ge 0) \\
     & = \sqrt{\frac{1-t}{2}}
  \end{align*}
  および
  \begin{align*}
    \cos\theta
     & = 2\cos^2\frac{\theta}{2} - 1 \\
     & = 2t^2-1
  \end{align*}
  から,$f$の変数を$\theta$から$t$に移して
  \begin{align}
    f(t)
     & = 4\sqrt{\frac{1-t}{2}} + \sqrt{5+4(2t^2-1)} \nonumber \\
     & = 2\sqrt{2(1-t)} + \sqrt{8t^2+1} \label{eq:9}
  \end{align}
  となる.
  \cref{eq:6}から
  \begin{align}
    0\le t \le \frac{1}{2} \label{eq:7}
  \end{align}
  である.$f$の一階微分は
  \begin{align*}
    f'(t)
     & = \frac{-2}{\sqrt{2(1-t)}} + \frac{16t}{2\sqrt{8t^2+1}}                                                            \\                                      \\
     & = \frac{-2\sqrt{8t^2+1} + 8t\sqrt{2(1-t)}}{\sqrt{2(1-t)}\sqrt{8t^2+1}}                                             \\
     & = \frac{-4(8t^2+1) + 2\cdot 8^2t^2(1-t)}{2\sqrt{2(1-t)}\sqrt{8t^2+1}\left(2\sqrt{8t^2+1} + 8t\sqrt{2(1-t)}\right)} \\
     & = \frac{4(-32t^3 + 24t^2-1)}{\sqrt{2(1-t)}\sqrt{8t^2+1}\left(\sqrt{8t^2+1} + 8t\sqrt{2(1-t)}\right)}
  \end{align*}
  である.この分母は常に正だから,$f'(t)$の正負は分子の正負と等しい.
  そこで分子を因数分解すると
  \begin{align*}
    -32t^3 + 24t^2-1
     & = -32\left(t^3-\frac{3}{4}t^2+\frac{1}{32}\right)                                                                           \\
     & = -32\left(t-\frac{1}{4}\right)\left(t^2-\frac{1}{2}t-\frac{1}{8}\right)                                                    \\
     & = -32\left(t-\frac{1}{4}\right)\left(t-\frac{1}{4}+\sqrt{\frac{3}{16}}\right)\left(t-\frac{1}{4}-\sqrt{\frac{3}{16}}\right)
  \end{align*}
  である.\cref{eq:7}の$t$の区間に注意して,
  \begin{align*}
    \frac{1}{4}-\sqrt{\frac{3}{16}} < 0 < \frac{1}{4} < \frac{1}{2} < \frac{1}{4}+\sqrt{\frac{3}{16}}
  \end{align*}
  だから,$f(t)$の増減表は\cref{table:1}となる.

  \begin{table}[H]
    \centering
    \caption{$f(t)$の増減表}
    \label{table:1}
    \begin{tabular}{|c||c|c|c|c|c|}
      \hline
      $t$  & $0$           &            & $1/4$ &            & $1/2$        \\
      \hline
      $f'$ &               & $-$        & $0$   & $+$        &              \\
      \hline
      $f$  & $1+2\sqrt{2}$ & $\searrow$ &       & $\nearrow$ & $2+\sqrt{3}$ \\
      \hline
    \end{tabular}
  \end{table}

  従って,求める最小値は$t=1/4$の時で
  \begin{align*}
    \min f(t)
     & = f\left(\frac{1}{4}\right)                                                  \\
     & = 2\sqrt{2\left(1-\frac{1}{4}\right)} + \sqrt{8\left(\frac{1}{4}\right)^2+1} \\
     & = \sqrt{6} + \sqrt{\frac{3}{2}}                                              \\
     & = \frac{3\sqrt{6}}{2}
  \end{align*}
  である.$\cdots$(答)


  \vspace{10pt}
  {\bf[解説]}
  最後の$f(\theta)$の最小値を求める部分はいくつかやり方がある.
  いずれもこの形のまま計算するのは厳しく,なんらか新しい変数を置いてやることになるだろう.
  今回の解答のように,$\theta/2$を利用する方が$\theta/4$と$\theta$を対称に扱っていて計算は早い.

  それを見るために,$t=\sin\frac{\theta}{4}$として求めてみよう.
  $\theta$の動く範囲が\cref{eq:4}だから,
  \begin{align}
    \frac{1}{2} \le t \le \frac{\sqrt{2}}{2} \label{eq:10}
  \end{align}
  である.\cref{eq:9}に倍角公式を利用して$t$で表すと
  \begin{align*}
    f(t) & = 4t + \sqrt{9-32t^2(1-t^2)}
  \end{align*}
  だから,一階微分は
  \begin{align*}
    f'(t)
     & = 4 + \frac{128t^3 - 64t}{2\sqrt{9-32t^2(1-t^2)}}              \\
     & = 4 \frac{\sqrt{32t^4-32t^2+9}+16t^3-8t}{\sqrt{32t^4-32t^2+9}}
  \end{align*}
  となり,分母は常に正だから$f'$の符号は分子の符号に等しい.
  \begin{align*}
                    & f'(t) \ge 0                                                                            \\
    \Leftrightarrow & \sqrt{32t^4-32t^2+9} \ge 8t(1-2t^2) \quad (\ge 0) \quad (\because \text{\cref{eq:10}}) \\
    \Leftrightarrow & 32t^4-32t^2+9 \ge 64t^2(1-2t^2)^2                                                      \\
    \Leftrightarrow & 256s^3-288s^2+96s-9 \le 0 \quad (s=t^2, \frac{1}{4} \le s \le \frac{1}{2})             \\
    \Leftrightarrow & (s-\frac{3}{8})(256s^2-192s+24) \le 0                                                  \\
    \Leftrightarrow & (s-\frac{3}{8})(s-\frac{3+\sqrt{3}}{16})(s-\frac{3-\sqrt{3}}{16}) \le 0
  \end{align*}
  だから. $f$の増減表は\cref{table:2}となる.
  \begin{table}[H]
    \centering
    \caption{$f$の増減表}
    \label{table:2}
    \begin{tabular}{|c||c|c|c|c|c|}
      \hline
      $t$     & $\frac{1}{2}$ &            & $\frac{\sqrt{6}}{4}$ &            & $\frac{\sqrt{2}}{2}$ \\
      \hline
      $s=t^2$ & $\frac{1}{4}$ &            & $\frac{3}{8}$        &            & $\frac{1}{2}$        \\
      \hline
      $f'$    &               & $-$        & $0$                  & $+$        &                      \\
      \hline
      $f$     &               & $\searrow$ &                      & $\nearrow$ &                      \\
      \hline
    \end{tabular}
  \end{table}
  したがって,$f(t)$は
  \begin{align*}
    t=\frac{\sqrt{6}}{4}
  \end{align*}
  で最小値
  \begin{align*}
    \min f(\theta)
     & = \sqrt{6}+\frac{\sqrt{6}}{2} \\
     & = \frac{3\sqrt{6}}{2}
  \end{align*}
  をとる.


  \newpage
\end{multicols}
\end{document}