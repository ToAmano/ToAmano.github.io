\documentclass[a4paper,10pt]{ltjsarticle}
\usepackage{luatexja}
\usepackage[hiragino-pron]{luatexja-preset}

\usepackage[truedimen,top=25truemm,bottom=20truemm,left=15truemm,right=15truemm]{geometry}
\setlength{\textwidth}{54\zw}
\setlength{\textheight}{73\zw}

\usepackage{amsmath,amssymb,ascmac}
\usepackage{enumerate}
\usepackage{multicol}
\usepackage{physics}
\usepackage{cleveref}
\usepackage{framed}
 \usepackage{caption}
 \usepackage{subcaption}
\usepackage{fancyhdr}
\usepackage{latexsym}
\usepackage{mathtools}
\usepackage{tikz}
\usepackage{tikz-3dplot}
\usepackage{pgfplots}
\pgfplotsset{compat=1.18}
\usetikzlibrary{intersections}
\usepgfplotslibrary{fillbetween}
 \usetikzlibrary{math}
 \usetikzlibrary{calc}
 \usetikzlibrary{angles}
 \usetikzlibrary{quotes}
 \usetikzlibrary{patterns}
% \usepackage{indent}
\usepackage{cases}
\usepackage{float}
\allowdisplaybreaks
\pagestyle{fancy}
\lhead{}
\chead{}
\rhead{東工大後期$2002$年$1$番}
\begin{document}

\begin{oframed}
  $n$ を自然数とする.

  \begin{enumerate}
    \item
          実数 $x$ に対して,$\displaystyle \sum_{k=0}^{n} (-1)^k x^{2k} - \frac{1}{1+x^2}$ を求めよ.

    \item
          不等式 $\displaystyle \left| \sum_{k=0}^{n} \frac{(-1)^k}{2k+1} - \int_0^1 \frac{1}{1+x^2} dx \right| \leq \frac{1}{2n+3}$ が成り立つことを示せ.

    \item
          極限 $\displaystyle \lim_{n \to \infty} \sum_{k=0}^{n} \frac{(-1)^k}{2k+1}$ を求めよ.
  \end{enumerate}
\end{oframed}
\setlength{\columnseprule}{0.4pt}
\begin{multicols}{2}
  {\bf[解]}

  (1)
  $x \in \mathbb{R}$だから$-x^2 \neq 1$なので,第一項は公比$-x^2$の等比数列の和である.
  \begin{align*}
    \text{(与式)}
     & = \frac{1-(-x^2)^{n+1}}{1-(-x^2)} - \frac{1}{1+x^2} \\
     & = \frac{-(-x^2)^{n+1}}{1+x^2}
  \end{align*}
  となり題意は示された.

  \vspace{10pt}
  (2)
  (1)の両辺を$[0,1]$で積分して絶対値をとると
  \begin{align}
    \left| \int_0^1 \sum_{k=0}^n (-x^2)^k dx - \int_0^1 \frac{1}{1+x^2} dx \right|
    = \left| \int_0^1 \frac{x^{2n+2}}{1+x^2} dx \right| \label{eq:1}
  \end{align}
  である.以下各項を評価する.まず左辺第一項は先に積分を処理することで
  \begin{align}
    \int_0^1 \sum_{k=0}^n (-x^2)^k dx
     & = \sum_{k=0}^n \left[ \frac{(-1)^k x^{2k+1}}{2k+1} \right]_0^1 \nonumber \\
     & = \sum_{k=0}^n \frac{(-1)^k}{2k+1} \label{eq:2}
  \end{align}
  であり,右辺は積分区間内で$0\le x\le 1$より$1+x^2\ge 1$だから
  \begin{align}
    \int_0^1 \frac{x^{2n+2}}{1+x^2} dx
     & \le \int_0^1 x^{2n+2} dx      \nonumber \\
     & = \frac{1}{2n+3} \label{eq:3}
  \end{align}
  となる.\cref{eq:2,eq:3}を\cref{eq:1}に代入して
  \begin{align}
    \left| \sum_{k=0}^n \frac{(-1)^k}{2k+1} - \int_0^1 \frac{1}{1+x^2} dx \right| \le \frac{1}{2n+3} \label{eq:4}
  \end{align}
  となり,題意は示された.

  \vspace{10pt}
  (3)
  (2)で示した\cref{eq:4}の右辺は$n\to\infty$で$0$に収束するから,挟み撃ちの定理より
  \begin{align}
    \lim_{n\to\infty} \sum_{k=0}^n \frac{(-1)^k}{2k+1} = \int_0^1 \frac{1}{1+x^2} dx \label{eq:5}
  \end{align}
  である.右辺の積分は$x=\tan\theta$と置換すると
  \begin{align*}
    \int_0^1 \frac{1}{1+x^2} dx
     & = \int_0^{\pi/4} \frac{1}{1+\tan^2\theta} \frac{dx}{d\theta} d\theta \\
     & = \int_0^{\pi/4} \cos^2\theta \frac{1}{\cos^2\theta} d\theta         \\
     & = \int_0^{\pi/4} d\theta                                             \\
     & = \frac{\pi}{4}
  \end{align*}
  だから,\cref{eq:5}に代入して
  \begin{align*}
    \lim_{n\to\infty} \sum_{k=0}^n \frac{(-1)^k}{2k+1} = \frac{\pi}{4}
  \end{align*}
  である.  $\cdots$(答)


  \vspace{10pt}
  {\bf[解説]}

  \newpage
\end{multicols}
\end{document}