\documentclass[a4paper,10pt]{ltjsarticle}
\usepackage{luatexja}
\usepackage[hiragino-pron]{luatexja-preset}

\usepackage[truedimen,top=25truemm,bottom=20truemm,left=15truemm,right=15truemm]{geometry}
\setlength{\textwidth}{54\zw}
\setlength{\textheight}{73\zw}

\usepackage{amsmath,amssymb,ascmac}
\usepackage{enumerate}
\usepackage{multicol}
\usepackage{cleveref}
\usepackage{framed}
\usepackage{fancyhdr}
\usepackage{latexsym}
\usepackage{mathtools}
\usepackage{tikz}
\usepackage{tikz-3dplot}
\usepackage{pgfplots}
 \usetikzlibrary{math}
 \usetikzlibrary{calc}
 \usetikzlibrary{angles}
 \usetikzlibrary{quotes}
 \usetikzlibrary{patterns}
 \usetikzlibrary{intersections}
 \usepgfplotslibrary{fillbetween}
 \pgfplotsset{compat=1.18}
% \usepackage{indent}
\usepackage{caption}
\usepackage{subcaption}
\usepackage{cases}
\usepackage{float}
\usepackage{cases}
  \usepackage{caption}
  \usepackage[subrefformat=parens]{subcaption}
\allowdisplaybreaks
\tdplotsetmaincoords{70}{110}
\pagestyle{fancy}
\lhead{}
\chead{}
\rhead{東工大後期$2007$年$2$番}
\begin{document}
\begin{oframed}
  $0 < x < \frac{\pi}{2}$ に対して関数 $\displaystyle f(x) = \frac{x}{\tan x}$, $\displaystyle g(x) = \frac{x}{\tan x} + \frac{\tan x}{x}$ を考える.
  \begin{enumerate}
    \item $f'(x), f''(x)$ の正負を判定し, $y=f(x)$ のグラフをかけ.
    \item $g'(x), g''(x)$ の正負を判定し, $y=g(x)$ のグラフをかけ.
    \item 正定数 $a$ に対して, 2 曲線 $y = \log \frac{a}{f(x)}$ と $y=g(x)$ のグラフが交わるための条件を求めよ.
  \end{enumerate}
\end{oframed}

\setlength{\columnseprule}{0.4pt}
\begin{multicols}{2}
  {\bf[解]}

  (1)
  $f(x)$の一階微分は
  \begin{align*}
    f'(x)
     & = \frac{\tan x - x/\cos^2 x}{\tan^2x} \\
     & = \frac{\sin x\cos x - x}{\sin^2 x}
  \end{align*}
  である.
  $0<x<\pi/2$では$\sin x < x$, $0<\cos x<1$より
  \begin{align}
    f'(x)<0 \label{eq:1}
  \end{align}
  が成立する.$\cdots$(答)

  次に$f(x)$の二階微分は
  \begin{align*}
    f''(x)
     & = \frac{d}{dx}\left( \frac{1}{\tan x} - \frac{x}{\sin^2 x} \right)            \\
     & = \left( -\frac{1}{\sin^2x} - \frac{\sin^2x-2x\sin x\cos x}{\sin^4 x} \right) \\
     & = \frac{2\cos x(x-\tan x)}{\sin^3 x}
  \end{align*}
  である.$0<x<\pi/2$では$x < \tan x$より
  \begin{align}
    f''(x) < 0 \label{eq:2}
  \end{align}
  となる.$\cdots$(答)

  \cref{eq:1,eq:2}より,$0<x<\pi/2$で$f(x)$は上に凸で単調減少する.
  また,極限値は
  \begin{align*}
    \begin{dcases}
      f(x) \xrightarrow{x \to 0} 1 \\
      f(x) \xrightarrow{x \to \pi/2} 0
    \end{dcases}
  \end{align*}
  だから,グラフの概形は\cref{fig:1}である.

  \begin{figure}[H]
    \centering
    \begin{tikzpicture}
      \begin{axis}[
          axis lines=middle,
          xmin=0, xmax=2,
          ymin=0, ymax=2,
          xlabel=$x$,
          ylabel=$y$,
          xtick={1.57},
          xticklabels={$\pi/2$},
          ytick={1},
          yticklabels={$1$},
          grid=none,
          %no markers,
          clip=false, % ラベルが軸からはみ出しても表示されるように
        ]

        % h(x) = (ln x) / x のプロット
        \addplot[blue, thick, domain=0:1.57,samples=100,smooth] {x*cos(deg(x))/sin(deg(x))};

        % f(t)ラベル
        \node[above] at (axis cs:0.8, 0.8) {$f(x)$};
        \addplot[only marks, mark=*, black] coordinates {(1.57,0)};
        \addplot[only marks, mark=*, black] coordinates {(0,1)};

      \end{axis}
    \end{tikzpicture}
    \caption{$f(x)$の概形.}
    \label{fig:1}
  \end{figure}


  \vspace{10pt}
  (2)
  $g(x) = f(x) + \frac{1}{f(x)}$ だから
  $g(x)$の一階微分は
  \begin{align*}
    g'(x)
     & = f'(x) - \frac{f'(x)}{f^2(x)}           \\
     & = f'(x)\left(1 - \frac{1}{f^2(x)}\right)
  \end{align*}
  である.\cref{eq:1}および$0<f(x)<1$より,
  \begin{align}
    g'(x)>0 \label{eq:3}
  \end{align}
  である.$\cdots$(答)

  次に$g(x)$の二階微分は
  \begin{align*}
    g''(x)
     & = f''(x)\left(1 - \frac{1}{f(x)^2}\right) + 2 \frac{f'(x)^2}{f(x)^3}
  \end{align*}
  であり,\cref{eq:1,eq:2}および$0<f(x)<1$から
  \begin{align}
    g''(x)>0 \label{eq:4}
  \end{align}
  である.$\cdots$(答)

  また,極限値は
  \begin{align*}
    \begin{dcases}
      g(x) \xrightarrow{x \to 0} 2 \\
      g(x) \xrightarrow{x \to \pi/2} \infty
    \end{dcases}
  \end{align*}
  であるから,グラフの概形は\cref{fig:2}である.

  \begin{figure}[H]
    \centering
    \begin{tikzpicture}
      \begin{axis}[
          axis lines=middle,
          xmin=0, xmax=2,
          ymin=0, ymax=10,
          xlabel=$x$,
          ylabel=$y$,
          xtick={1.57},
          xticklabels={$\pi/2$},
          ytick={2},
          yticklabels={$2$},
          grid=none,
          %no markers,
          clip=false, % ラベルが軸からはみ出しても表示されるように
        ]

        % h(x) = (ln x) / x のプロット
        \addplot[blue, thick, domain=0.01:1.5,samples=100,smooth] {x*cos(deg(x))/sin(deg(x))+sin(deg(x))/(x*cos(deg(x)))};

        % f(t)ラベル
        \node[above] at (axis cs:0.8, 0.8) {$g(x)$};
        \addplot[only marks, mark=*, black] coordinates {(0,2)};

        % 極大点とy軸のラベルを結ぶ破線
        % \draw[dashed] (axis cs:e, {ln(e)*ln(e)}) -- (axis cs:0, {ln(e)*ln(e)});
        \draw[dashed] (axis cs:pi/2, 0) -- (axis cs:pi/2, 10);
      \end{axis}
    \end{tikzpicture}
    \caption{$g(x)$の概形.}
    \label{fig:2}
  \end{figure}


  \vspace{10pt}
  (3)
  新しく
  \begin{align*}
    h(x) = \log \frac{a}{f(x)} - g(x)
  \end{align*}
  とおく.
  $0 < x < \pi/2$ で $h(x)=0$ が実解を持つ条件をもとめれば良い.
  以下 $A = \log a$ とする.
  $h(x)$ の一階微分は\cref{eq:3}より
  \begin{align*}
    h'(x)
     & = - \frac{f'(x)}{f(x)} - g'(x)                                                       \\
     & = - \frac{f'(x)}{f(x)} - f'(x)\left(1 - \frac{1}{f(x)^2}\right) \quad (\because (3)) \\
     & = \frac{f'(x)}{f(x)^2} (f(x)^2 + f(x)-1)
  \end{align*}
  である.
  \begin{align*}
    f(x)^2 + f(x) -1 = 0 \\
    \iff
    f(x) = \frac{-1 \pm \sqrt{5}}{2}
  \end{align*}
  だが,$0<f(x)<1$よりあり得るのは符号が負の場合で
  \begin{align*}
    f(x) = \frac{-1 + \sqrt{5}}{2}
  \end{align*}
  である.(1)より$f(x)$は単調減少だから,これを満たす$x$がただ一つあるから,
  それを$x=\alpha$とおくと,
  $h(x)$の増減表は\cref{table:1}となる.

  \begin{table}[H]
    \centering
    \caption{$h(x)$の増減表}
    \label{table:1}
    \begin{tabular}{|c||c|c|c|c|c|}
      \hline
      $x$  & $(0)$   & $\cdots$   & $\alpha$ & $\cdots$   & $(\pi/2)$   \\
      \hline
      $h'$ &         & $+$        & $0$      & $-$        &             \\
      \hline
      $h$  & $(A-2)$ & $\nearrow$ &          & $\searrow$ & $(-\infty)$ \\
      \hline
    \end{tabular}
  \end{table}
  ここで,
  \begin{align*}
    f(\alpha)
     & = \frac{-1+\sqrt{5}}{2}                         \\
    g(\alpha)
     & = f(\alpha) + \frac{1}{f(\alpha)}               \\
     & = \frac{-1+\sqrt{5}}{2} + \frac{2}{-1+\sqrt{5}} \\
     & = \sqrt{5}
  \end{align*}
  より,
  \begin{align*}
    h(\alpha)
     & = A - \log f(\alpha) - g(\alpha)                           \\
     & = A - \log \frac{-1+\sqrt{5}}{2} - \sqrt{5}                \\
     & = A - \log \left[\frac{-1+\sqrt{5}}{2} e^{\sqrt{5}}\right]
  \end{align*}
  である.したがって,増減表とあわせて$h(x)=0$が実数解を持つ条件は
  $h(\alpha)\ge 0$であり,求める$a$の条件は
  \begin{align*}
    h(\alpha) \ge 0                                            \\
    \iff
    A \ge \log \left[\frac{-1+\sqrt{5}}{2} e^{\sqrt{5}}\right] \\
    \iff
    a \ge \frac{1+\sqrt{5}}{2} e^{\sqrt{5}}
  \end{align*}
  となる.
  ただし$\log x$が単調増加であることを利用した.$\cdots$(答)

  \vspace{10pt}
  {\bf[解説]}

  \newpage
\end{multicols}
\end{document}