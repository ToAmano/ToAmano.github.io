\documentclass[a4paper,10pt]{ltjsarticle}
\usepackage{luatexja}
\usepackage[hiragino-pron]{luatexja-preset}

\usepackage[truedimen,top=25truemm,bottom=20truemm,left=15truemm,right=15truemm]{geometry}
\setlength{\textwidth}{54\zw}
\setlength{\textheight}{73\zw}

\usepackage{amsmath,amssymb,ascmac}
\usepackage{enumerate}
\usepackage{multicol}
\usepackage{physics}
\usepackage{cleveref}
\usepackage{framed}
 \usepackage{caption}
 \usepackage{subcaption}
\usepackage{fancyhdr}
\usepackage{latexsym}
\usepackage{mathtools}
\usepackage{tikz}
\usepackage{tikz-3dplot}
\usepackage{pgfplots}
\pgfplotsset{compat=1.18}
\usetikzlibrary{intersections}
\usepgfplotslibrary{fillbetween}
 \usetikzlibrary{math}
 \usetikzlibrary{calc}
 \usetikzlibrary{angles}
 \usetikzlibrary{quotes}
 \usetikzlibrary{patterns}
 \usetikzlibrary{arrows.meta}

% \usepackage{indent}

\usepackage{cases}
\usepackage{float}
\allowdisplaybreaks
\pagestyle{fancy}
\lhead{}
\chead{}
\rhead{東工大後期$2007$年$1$番}
\begin{document}
\begin{oframed}
  1 から 6 までの目がそれぞれ $\frac{1}{6}$ の確率で出るサイコロを 3 回振って出た目を順に
  $n_1, n_2, n_3$ とし, 次の 3 次方程式を考える.
  $$ x^3 - n_1 x + (-1)^{n_2} n_3 = 0 $$
  \begin{enumerate}
    \item この方程式が相異なる 3 個の実数解をもつ確率を求めよ.
    \item この方程式が自然数解をもつ確率を求めよ.
  \end{enumerate}
\end{oframed}
\setlength{\columnseprule}{0.4pt}
\begin{multicols}{2}
  {\bf[解]}
  表記の簡潔さのため
  \begin{align*}
    f(x) = x^3 - n_1 x + (-1)^{n_2} n_3
  \end{align*}
  とおく.方程式$f(x)=0$について考える.

  \vspace{10pt}
  (1)

  $f(x)$のグラフの概形を調べるため,一階微分を計算すると
  \begin{align*}
    f'(x) = 3x^2 - n_1
  \end{align*}
  だから,$n_1>0$より$f(x)$の増減表は\cref{table:1}となる.

  \begin{table}[H]
    \centering
    \caption{$f(x)$の増減表}
    \label{table:1}
    \begin{tabular}{|c||c|c|c|c|c|c|c|}
      \hline
      $x$  & $-\infty$   & $\cdots$   & $-\sqrt{\frac{n_1}{3}}$ & $\cdots$   & $\sqrt{\frac{n_1}{3}}$ & $\cdots$   & $\infty$   \\
      \hline
      $f'$ &             & $+$        & $0$                     & $-$        & $0$                    & $+$        &            \\
      \hline
      $f$  & ($-\infty$) & $\nearrow$ & 極大                      & $\searrow$ & 極小                     & $\nearrow$ & $(\infty)$ \\
      \hline
    \end{tabular}
  \end{table}

  したがって,$f(x)=0$ が $3$ 実数解を持つ条件は
  \begin{align}
     & f(\sqrt{n_1/3}) f(-\sqrt{n_1/3}) < 0                                                                          \nonumber                        \\
     & \left(-\frac{2}{3}n_1\sqrt{\frac{n_1}{3}} + (-1)^{n_2}n_3\right) \left(\frac{2}{3}n_1\sqrt{\frac{n_1}{3}} + (-1)^{n_2}n_3\right) < 0 \nonumber \\
     & -\frac{2}{3}n_1\sqrt{\frac{n_1}{3}} <(-1)^{n_2}n_3 < \frac{2}{3}n_1\sqrt{\frac{n_1}{3}}                          \nonumber                     \\
     & n_3 < \frac{2}{3}n_1\sqrt{\frac{n_1}{3}} \label{eq:1}
  \end{align}
  と表せる.ただし,最終行で$n_3>0$を利用した.
  以下,\cref{eq:1}を満たす整数$(n_1,n_2,n_3)$を考える.
  まず,$n_2$については任意であり,$n_1$と$n_3$のみ考えれば良い.
  表記の簡潔さのため
  \begin{align*}
    A & = \frac{2}{3}n_1\sqrt{\frac{n_1}{3}} \\
    B & = (-1)^{n_2}n_3
  \end{align*}
  とおくと,$n_1=1,2,\cdots,6$に対して値は以下のようになる.
  \begin{table}[H]
    \centering
    \caption{$A$の$n_1$による値の変化}
    \begin{tabular}{|c||c|c|c|c|c|c|}
      \hline
      $n_1$ & $1$                   & $2$                   & $3$ & $4$                    & $5$                     & $6$         \\
      \hline
      $A$   & $\frac{2}{9}\sqrt{3}$ & $\frac{4}{9}\sqrt{6}$ & $2$ & $\frac{16}{9}\sqrt{3}$ & $\frac{10}{9}\sqrt{15}$ & $4\sqrt{2}$ \\
      \hline
    \end{tabular}
  \end{table}
  これらの値は
  \begin{align*}
    \frac{2}{9}\sqrt{3} < 1 < \frac{4}{9}\sqrt{6} < 2 < \frac{16}{9}\sqrt{3} < 3 < \frac{10}{9}\sqrt{15} < 4 < 4\sqrt{2} < 6
  \end{align*}
  という大小関係にあるから,$n_1$ に対応して\cref{eq:1}を満たす$n_3$をリストアップすると\cref{table:2}のようになる.
  \begin{table}[H]
    \centering
    \caption{\cref{eq:1}を満たす$n_1,n_3$の一覧}
    \label{table:2}
    \begin{tabular}{|c||c|c|c|c|c|c|}
      \hline
      $n_1$   & $1$ & $2$ & $3$ & $4$        & $5$        & $6$        \\
      \hline
      $n_3$   & 無し  & $1$ & $1$ & $1 \sim 3$ & $1 \sim 4$ & $1 \sim 5$ \\
      \hline
      $n_3$の数 & $0$ & $1$ & $1$ & $3$        & $4$        & $5$        \\
      \hline
    \end{tabular}
  \end{table}
  したがって求める確率は
  \begin{align*}
    \frac{1}{6}\cdot\frac{1}{6}
    + \frac{1}{6}\cdot\frac{1}{6}
    + \frac{1}{6}\cdot\frac{3}{6}
    + \frac{1}{6}\cdot\frac{4}{6}
    + \frac{1}{6}\cdot\frac{5}{6} = \frac{7}{18}
  \end{align*}
  である.$\cdots$(答)

  \vspace{10pt}
  (2) 題意の自然数解を $k\in\mathbb{N}$ とおくと,$f(k)=0$ゆえ
  \begin{align*}
    k(k^2 - n_1) = (-1)^{n_2+1} n_3
  \end{align*}
  だから,$k$と$|k^2-n_1|$の積が$n_3$であることが必要である.
  まずはこの必要条件を満たす$(k,k^2-n_1)$を$n_3$に応じてリストアップする.
  $k$に対して,$n_1=1,2,\cdots,6$より
  \begin{align*}
    |k^2-n_1| \le |k^2-1| \\
    |k^2-n_1| \le |k^2-6|
  \end{align*}
  であるから,この上限によって一定の制限があることに注意すると,一覧は\cref{table:3}となる.
  \begin{table}[H]
    \centering
    \caption{$k|k^2-n_1|=n_3$を満たす$(n_3,k,k^2-n_1)$の組}
    \label{table:3}
    \begin{tabular}{|c||c|c|c|c|c|c|}
      \hline
      $n_3$                                                                     & $1$                                     & $2$                                        & $3$                                     & $4$                                        & $5$                                     & $6$                                     \\
      \hline
      \raisebox{-1.5ex}[0pt][0pt]{$\begin{pmatrix} k \\ k^2-n_1 \end{pmatrix}$} & $\begin{pmatrix} 1 \\ -1 \end{pmatrix}$ & $\begin{pmatrix} 1 \\ -2 \end{pmatrix}$    & $\begin{pmatrix} 1 \\ -3 \end{pmatrix}$ & $\begin{pmatrix} 1 \\ -4 \end{pmatrix}$    & $\begin{pmatrix} 1 \\ -5 \end{pmatrix}$ & $\begin{pmatrix} 2 \\  3 \end{pmatrix}$ \\
                                                                                &                                         & $\begin{pmatrix} 2 \\ \pm 1 \end{pmatrix}$ &                                         & $\begin{pmatrix} 2 \\ \pm 2 \end{pmatrix}$ &                                         &                                         \\
      \hline
    \end{tabular}
  \end{table}
  これを満たす$(k,n_1)$を一覧化すると
  \begin{table}[H]
    \begin{tabular}{|c||c|c|c|c|c|c|}
      \hline
      $n_3$      & $1$     & $2$     & $3$     & $4$     & $5$     & $6$     \\
      \hline
      $(k, n_1)$ & $(1,2)$ & $(1,3)$ & $(1,4)$ & $(1,5)$ & $(1,6)$ & $(2,1)$ \\
                 &         & $(2,3)$ &         & $(2,2)$ &         &         \\
                 &         & $(2,5)$ &         & $(2,6)$ &         &         \\
      \hline
    \end{tabular}
  \end{table}
  である.それぞれの組に対して$n_2$は3通り(偶数か奇数)が対応するので,$(n_1,n_2,n_3)$の場合の数は
  \begin{align*}
    10*3 = 30
  \end{align*}
  通りであり,求める確率は
  \begin{align*}
    \frac{30}{6^3} = \frac{5}{36}
  \end{align*}
  である.$\cdots$(答)

  \vspace{10pt}
  {\bf[解説]}

  \newpage
\end{multicols}
\end{document}