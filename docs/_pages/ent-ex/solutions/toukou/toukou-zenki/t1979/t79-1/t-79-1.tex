\documentclass[a4j]{jarticle}
\usepackage{amsmath}
\usepackage{ascmac}
\usepackage{amssymb}
\usepackage{enumerate}
\usepackage{multicol}
\usepackage{framed}
\usepackage{fancyhdr}
\usepackage{latexsym}
\usepackage{indent}
\usepackage{cases}
\usepackage[dvips]{graphicx}
\usepackage{color}
\allowdisplaybreaks
\pagestyle{fancy}
\lhead{}
\chead{}
\rhead{東工大前期$1979$年$1$番}
\begin{document}
%分数関係


\def\tfrac#1#2{{\textstyle\frac{#1}{#2}}} %数式中で文中表示の分数を使う時


%Σ関係

\def\dsum#1#2{{\displaystyle\sum_{#1}^{#2}}} %文中で数式表示のΣを使う時


%ベクトル関係


\def\vector#1{\overrightarrow{#1}}  %ベクトルを表現したいとき(aベクトルを表現するときは\ver
\def\norm#1{|\overrightarrow{#1}|} %ベクトルの絶対値
\def\vtwo#1#2{ \left(%
      \begin{array}{c}%
      #1 \\%
      #2 \\%
      \end{array}%
      \right) }                        %2次元ベクトル成分表示
      
      \def\vthree#1#2#3{ \left(
      \begin{array}{c}
      #1 \\
      #2 \\
      #3 \\
      \end{array}
      \right) }                        %3次元ベクトル成分表示



%数列関係


\def\an#1{\verb|{|$#1$\verb|}|}


%極限関係

\def\limit#1#2{\stackrel{#1 \to #2}{\longrightarrow}}   %等式変形からの極限
\def\dlim#1#2{{\displaystyle \lim_{#1\to#2}}} %文中で数式表示の極限を使う



%積分関係

\def\dint#1#2{{\displaystyle \int_{#1}^{#2}}} %文中で数式表示の積分を使う時

\def\ne{\nearrow}
\def\se{\searrow}
\def\nw{\nwarrow}
\def\ne{\nearrow}


%便利なやつ

\def\case#1#2{%
 \[\left\{%
 \begin{array}{l}%
 #1 \\%
 #2%
 \end{array}%
 \right.\] }                           %場合分け
 
\def\1{$\cos\theta=c$,$\sin\theta=s$とおく.}  %cs表示を与える前書きシータ
\def\2{$\cos t=c$,$\sin t=s$とおく.}     %cs表示を与える前書きt
\def\3{$\cos x=c$,$\sin x=s$とおく.}                %cs表示を与える前書きx

\def\fig#1#2#3 {%
\begin{wrapfigure}[#1]{r}{#2 zw}%
\vspace*{-1zh}%
\input{#3}%
\end{wrapfigure} }           %絵の挿入


\def\a{\alpha}   %ギリシャ文字
\def\b{\beta}
\def\g{\gamma}

%問題番号のためのマクロ

\newcounter{nombre} %必須
\renewcommand{\thenombre}{\arabic{nombre}} %任意
\setcounter{nombre}{2} %任意
\newcounter{nombresub}[nombre] %親子関係を定義
\renewcommand{\thenombresub}{\arabic{nombresub}} %任意
\setcounter{nombresub}{0} %任意
\newcommand{\prob}[1][]{\refstepcounter{nombre}#1[問題 \thenombre]}
\newcommand{\probsub}[1][]{\refstepcounter{nombresub}#1(\thenombresub)}


%1-1みたいなカウンタ(todaiとtodaia)
\newcounter{todai}
\setcounter{todai}{0}
\newcounter{todaisub}[todai] 
\setcounter{todaisub}{0} 
\newcommand{\todai}[1][]{\refstepcounter{todai}#1 \thetodai-\thetodaisub}
\newcommand{\todaib}[1][]{\refstepcounter{todai}#1\refstepcounter{todaisub}#1 {\bf [問題 \thetodai.\thetodaisub]}}
\newcommand{\todaia}[1][]{\refstepcounter{todaisub}#1 {\bf [問題 \thetodai.\thetodaisub]}}


     \begin{oframed}
     直線$l:x=y=z$と直線$m:x/2=(y-1)/3=-z$上にそれぞれ点列$P_1,P_2,\cdots,P_n,\cdots$および$Q_1,Q_2,\cdots,Q_n\cdots$
     があり,全ての$n$について線分$P_nQ_n$と$m$,線分$Q_nP_{n+1}$と$l$とはそれぞれ直交しているとする.$n$を限りなく大
     きくするとき,点$P_n$,$Q_n$はそれぞれどのような点に近づくか.それらの点の座標を求めよ.
     \end{oframed}

\setlength{\columnseprule}{0.4pt}
\begin{multicols}{2}
{\bf[解]} $\vec{a}=(1,1,1)$,$\vec{b}=(2,3,-1)$,$\vec{c}=(0,1,0)$とすると,$\vec{a}$,$\vec{b}$はそれぞれ,
$l$,$m$の方向ベクトルである.$P_n$,$Q_n$が各直線上にあるので,実数$t_n$,$s_n$を用いて
     \begin{align*}
     &\vector{OP_n}=t_n\vec{a}&\vector{OQ_n}=\vec{c}+s_n\vec{b}
     \end{align*}
と表される.題意の条件から
     \begin{align}
          &\begin{cases}
          \vec{b}\cdot\vector{P_nQ_n} \\
          \vec{a}\cdot\vector{Q_nP_{n+1}}
          \end{cases}\nonumber \\
     \Longleftrightarrow     
          &\begin{cases}
          \vec{b}\cdot(\vec{c}+s_n\vec{b}-t_n\vec{a})=0 \\
          \vec{a}\cdot(\vec{c}+s_n\vec{b}-t_{n+1}\vec{a})=0
          \end{cases}\nonumber \\
     \Longleftrightarrow
          &\begin{cases}
          \vec{b}\cdot\vec{c}+s_n|\vec{b}|^2-t_n\vec{a}\cdot\vec{b}=0 \\
          \vec{a}\cdot\vec{c}+s_n\vec{a}\cdot\vec{b}-t_{n+1}|\vec{a}|^2=0 
          \end{cases}\label{1}
     \end{align}
ここで,
     \begin{align*}
     &|\vec{a}|^2=3&|\vec{b}|^2=14 \\
     &\vec{a}\cdot\vec{c}=1&\vec{b}\cdot\vec{c}=3 \\
     &\vec{a}\cdot\vec{b}=4
     \end{align*}
を代入して
     \begin{align} 
          \begin{cases}
          3+14s_n-4t_n=0 \\
          1+4s_n-3t_{n+1}=0
          \end{cases}\label{2}
     \end{align}
\eqref{2}から$s_n$を消去して
     \begin{align*}
     t_{n+1}=\frac{1}{21}(8t_n+1) \\
     t_{n+1}-\frac{1}{13}=\frac{8}{21}\left(t_n-\frac{1}{13}\right) 
     \end{align*}
繰り返し用いて
     \begin{align}
     t_n=\frac{1}{13}+\left(\frac{8}{21}\right)^{n-1}\left(t_1-\frac{1}{13}\right)\nonumber \\
     \limit{n}{\infty}\frac{1}{13}\label{3}
     \end{align}
故に\eqref{2}から
     \begin{align}
     \lim_{n\to\infty}s_n&=\lim_{n\to\infty}\left(\frac{4t_n-3}{14}\right)\nonumber \\
     &=\frac{1}{14}\left(4\lim_{n\to\infty}t_n-3\right)\nonumber\\
     &=\frac{1}{14}\left(4\frac{1}{13}-3\right)&(\because\eqref{3})\nonumber \\
     &=\frac{-5}{26}\label{4}
     \end{align}          
である.以上の結果から,
     \begin{align*}
     \vector{OP_n}&=t_n\vec{a}\\
     &\limit{n}{\infty}\frac{1}{13}\vec{a} \\
     &=\left(\frac{1}{13},\frac{1}{13},\frac{1}{13}\right) \\ 
     \vector{OQ_n}&=\vec{c}+s_n\vec{b}\\
     &\limit{n}{\infty}\vec{c}-\frac{5}{26}\vec{b} \\
     &=\left(\frac{-5}{13},\frac{11}{26},\frac{5}{26}\right)
     \end{align*}
となって,求める点は
     \begin{align*}
     &P_n\to\left(\frac{1}{13},\frac{1}{13},\frac{1}{13}\right) \\
     &Q_n\to\left(\frac{-5}{13},\frac{11}{26},\frac{5}{26}\right)
     \end{align*}
である.$\cdots$(答)     
\newpage
\end{multicols}
\end{document}