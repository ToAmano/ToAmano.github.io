\documentclass[a4j]{jarticle}
\usepackage{amsmath}
\usepackage{ascmac}
\usepackage{amssymb}
\usepackage{enumerate}
\usepackage{multicol}
\usepackage{framed}
\usepackage{fancyhdr}
\usepackage{latexsym}
\usepackage{indent}
\usepackage{cases}
\usepackage[dvips]{graphicx}
\usepackage{color}
\allowdisplaybreaks
\pagestyle{fancy}
\lhead{}
\chead{}
\rhead{東工大前期$2009$年$4$番}
\begin{document}
%分数関係


\def\tfrac#1#2{{\textstyle\frac{#1}{#2}}} %数式中で文中表示の分数を使う時


%Σ関係

\def\dsum#1#2{{\displaystyle\sum_{#1}^{#2}}} %文中で数式表示のΣを使う時


%ベクトル関係


\def\vector#1{\overrightarrow{#1}}  %ベクトルを表現したいとき(aベクトルを表現するときは\ver
\def\norm#1{|\overrightarrow{#1}|} %ベクトルの絶対値
\def\vtwo#1#2{ \left(%
      \begin{array}{c}%
      #1 \\%
      #2 \\%
      \end{array}%
      \right) }                        %2次元ベクトル成分表示
      
      \def\vthree#1#2#3{ \left(
      \begin{array}{c}
      #1 \\
      #2 \\
      #3 \\
      \end{array}
      \right) }                        %3次元ベクトル成分表示



%数列関係


\def\an#1{\verb|{|$#1$\verb|}|}


%極限関係

\def\limit#1#2{\stackrel{#1 \to #2}{\longrightarrow}}   %等式変形からの極限
\def\dlim#1#2{{\displaystyle \lim_{#1\to#2}}} %文中で数式表示の極限を使う



%積分関係

\def\dint#1#2{{\displaystyle \int_{#1}^{#2}}} %文中で数式表示の積分を使う時

\def\ne{\nearrow}
\def\se{\searrow}
\def\nw{\nwarrow}
\def\ne{\nearrow}


%便利なやつ

\def\case#1#2{%
 \[\left\{%
 \begin{array}{l}%
 #1 \\%
 #2%
 \end{array}%
 \right.\] }                           %場合分け
 
\def\1{$\cos\theta=c$,$\sin\theta=s$とおく.}  %cs表示を与える前書きシータ
\def\2{$\cos t=c$,$\sin t=s$とおく.}     %cs表示を与える前書きt
\def\3{$\cos x=c$,$\sin x=s$とおく.}                %cs表示を与える前書きx

\def\fig#1#2#3 {%
\begin{wrapfigure}[#1]{r}{#2 zw}%
\vspace*{-1zh}%
\input{#3}%
\end{wrapfigure} }           %絵の挿入


\def\a{\alpha}   %ギリシャ文字
\def\b{\beta}
\def\g{\gamma}

%問題番号のためのマクロ

\newcounter{nombre} %必須
\renewcommand{\thenombre}{\arabic{nombre}} %任意
\setcounter{nombre}{2} %任意
\newcounter{nombresub}[nombre] %親子関係を定義
\renewcommand{\thenombresub}{\arabic{nombresub}} %任意
\setcounter{nombresub}{0} %任意
\newcommand{\prob}[1][]{\refstepcounter{nombre}#1[問題 \thenombre]}
\newcommand{\probsub}[1][]{\refstepcounter{nombresub}#1(\thenombresub)}


%1-1みたいなカウンタ(todaiとtodaia)
\newcounter{todai}
\setcounter{todai}{0}
\newcounter{todaisub}[todai] 
\setcounter{todaisub}{0} 
\newcommand{\todai}[1][]{\refstepcounter{todai}#1 \thetodai-\thetodaisub}
\newcommand{\todaib}[1][]{\refstepcounter{todai}#1\refstepcounter{todaisub}#1 {\bf [問題 \thetodai.\thetodaisub]}}
\newcommand{\todaia}[1][]{\refstepcounter{todaisub}#1 {\bf [問題 \thetodai.\thetodaisub]}}


     \begin{oframed}
     $xyz$空間の原点と点$(1,1,1)$を通る直線を$l$とする.
          \begin{enumerate}[(1)]
          \item $l$上の点$(t/3,t/3,t/3)$を通り$l$と垂直な平面が,$xy$平面と交わってできる直線の方程式を求めよ.
          \item 不等式$0\le y\le x(1-x)$の表す$xy$平面内の領域を$D$とする.$l$を軸として$D$を回転させて得られる回転体の体積を求め
          よ.
          \end{enumerate}
     \end{oframed}

\setlength{\columnseprule}{0.4pt}
\begin{multicols}{2}
{\bf[解]} 題意の平面$\Pi$として,
     \begin{align*}
     &\left(x-\frac{t}{3}\right)+\left(x-\frac{t}{3}\right)+\left(x-\frac{t}{3}\right)=0 \\
     &x+y+z=t
     \end{align*}
であるから,$z=0$との交線は,
     \[x+y=t,z=0\]
である.$\cdots((1)の答)$     

$D$と(1)で求めた交線の共有部分を考える.$y$を消去して,
     \begin{align}
     0\le t-x\le x(1-x) \\
     \Longleftrightarrow
          \begin{cases}
          x\le t \\
          1-\sqrt{1-t}\le x\le1+\sqrt{1-t}
          \end{cases}\label{1}
     \end{align}
である.ただし,第$2$の不等式での$x$の存在条件から,
     \begin{align}
     1-t\ge0\Longleftrightarrow t\le1\label{2}
     \end{align}
である.このもとで,    
     \[t\le1+\sqrt{1-t}\]
であるから,$x$の範囲は,\eqref{1}から,
     \begin{align*}
     1-\sqrt{1-t}\le x<t
     \end{align*} 
である.再び$x$の存在条件から,
     \begin{align}
     &1-\sqrt{1-t}\le t \Longleftrightarrow 0\le t&(\because\eqref{2})
     \end{align}
である.以上から,共有部分は,
     \begin{align}
          E:&\begin{cases}
          x+y=t \\
          1-\sqrt{1-t}\le x\le t
          \end{cases}
     &(0\le t\le1)\label{3}
     \end{align}
である.(右上図)
     \begin{center}
     \scalebox{.7}{%WinTpicVersion4.32a
{\unitlength 0.1in%
\begin{picture}(20.2000,20.0000)(15.8000,-38.0000)%
% STR 2 0 3 0 Black White  
% 4 1790 3597 1790 3610 4 1600 0 0
% O
\put(17.9000,-36.1000){\makebox(0,0)[rt]{O}}%
% STR 2 0 3 0 Black White  
% 4 1760 1787 1760 1800 4 1600 0 0
% $y$
\put(17.6000,-18.0000){\makebox(0,0)[rt]{$y$}}%
% STR 2 0 3 0 Black White  
% 4 3600 3627 3600 3640 4 1600 0 0
% $x$
\put(36.0000,-36.4000){\makebox(0,0)[rt]{$x$}}%
% VECTOR 2 0 3 0 Black White  
% 2 1800 3800 1800 1800
% 
\special{pn 8}%
\special{pa 1800 3800}%
\special{pa 1800 1800}%
\special{fp}%
\special{sh 1}%
\special{pa 1800 1800}%
\special{pa 1780 1867}%
\special{pa 1800 1853}%
\special{pa 1820 1867}%
\special{pa 1800 1800}%
\special{fp}%
% VECTOR 2 0 3 0 Black White  
% 2 1600 3600 3600 3600
% 
\special{pn 8}%
\special{pa 1600 3600}%
\special{pa 3600 3600}%
\special{fp}%
\special{sh 1}%
\special{pa 3600 3600}%
\special{pa 3533 3580}%
\special{pa 3547 3600}%
\special{pa 3533 3620}%
\special{pa 3600 3600}%
\special{fp}%
% FUNC 2 0 3 0 Black White  
% 9 1600 1800 3600 3800 1800 3600 4600 3600 1800 800 1800 1800 1800 3800 0 0 0 0
% 0.5-x
\special{pn 8}%
\special{pn 8}%
\special{pa 1600 2000}%
\special{pa 1606 2006}%
\special{fp}%
\special{pa 1633 2033}%
\special{pa 1639 2039}%
\special{fp}%
\special{pa 1667 2067}%
\special{pa 1673 2073}%
\special{fp}%
\special{pa 1700 2100}%
\special{pa 1706 2106}%
\special{fp}%
\special{pa 1733 2133}%
\special{pa 1739 2139}%
\special{fp}%
\special{pa 1767 2167}%
\special{pa 1773 2173}%
\special{fp}%
\special{pa 1800 2200}%
\special{pa 1800 2200}%
\special{fp}%
\special{pn 8}%
\special{pa 1806 2206}%
\special{pa 1832 2232}%
\special{fp}%
\special{pa 1838 2238}%
\special{pa 1864 2264}%
\special{fp}%
\special{pa 1870 2270}%
\special{pa 1896 2296}%
\special{fp}%
\special{pa 1902 2302}%
\special{pa 1928 2328}%
\special{fp}%
\special{pa 1934 2334}%
\special{pa 1960 2360}%
\special{fp}%
\special{pa 1966 2366}%
\special{pa 1992 2392}%
\special{fp}%
\special{pa 1998 2398}%
\special{pa 2024 2424}%
\special{fp}%
\special{pa 2030 2430}%
\special{pa 2056 2456}%
\special{fp}%
\special{pa 2062 2462}%
\special{pa 2088 2488}%
\special{fp}%
\special{pa 2094 2494}%
\special{pa 2120 2520}%
\special{fp}%
\special{pa 2126 2526}%
\special{pa 2152 2552}%
\special{fp}%
\special{pa 2158 2558}%
\special{pa 2184 2584}%
\special{fp}%
\special{pa 2190 2590}%
\special{pa 2216 2616}%
\special{fp}%
\special{pa 2222 2622}%
\special{pa 2248 2648}%
\special{fp}%
\special{pa 2254 2654}%
\special{pa 2280 2680}%
\special{fp}%
\special{pa 2286 2686}%
\special{pa 2312 2712}%
\special{fp}%
\special{pa 2318 2718}%
\special{pa 2344 2744}%
\special{fp}%
\special{pa 2350 2750}%
\special{pa 2376 2776}%
\special{fp}%
\special{pa 2382 2782}%
\special{pa 2408 2808}%
\special{fp}%
\special{pa 2414 2814}%
\special{pa 2440 2840}%
\special{fp}%
\special{pa 2446 2846}%
\special{pa 2472 2872}%
\special{fp}%
\special{pa 2478 2878}%
\special{pa 2504 2904}%
\special{fp}%
\special{pa 2510 2910}%
\special{pa 2536 2936}%
\special{fp}%
\special{pa 2542 2942}%
\special{pa 2568 2968}%
\special{fp}%
\special{pa 2574 2974}%
\special{pa 2600 3000}%
\special{fp}%
\special{pa 2606 3006}%
\special{pa 2632 3032}%
\special{fp}%
\special{pa 2638 3038}%
\special{pa 2664 3064}%
\special{fp}%
\special{pa 2670 3070}%
\special{pa 2696 3096}%
\special{fp}%
\special{pa 2702 3102}%
\special{pa 2728 3128}%
\special{fp}%
\special{pa 2734 3134}%
\special{pa 2760 3160}%
\special{fp}%
\special{pa 2766 3166}%
\special{pa 2792 3192}%
\special{fp}%
\special{pa 2798 3198}%
\special{pa 2824 3224}%
\special{fp}%
\special{pa 2830 3230}%
\special{pa 2856 3256}%
\special{fp}%
\special{pa 2862 3262}%
\special{pa 2888 3288}%
\special{fp}%
\special{pa 2894 3294}%
\special{pa 2920 3320}%
\special{fp}%
\special{pa 2926 3326}%
\special{pa 2952 3352}%
\special{fp}%
\special{pa 2958 3358}%
\special{pa 2984 3384}%
\special{fp}%
\special{pa 2990 3390}%
\special{pa 3016 3416}%
\special{fp}%
\special{pa 3022 3422}%
\special{pa 3048 3448}%
\special{fp}%
\special{pa 3054 3454}%
\special{pa 3080 3480}%
\special{fp}%
\special{pa 3086 3486}%
\special{pa 3112 3512}%
\special{fp}%
\special{pa 3118 3518}%
\special{pa 3144 3544}%
\special{fp}%
\special{pa 3150 3550}%
\special{pa 3176 3576}%
\special{fp}%
\special{pa 3182 3582}%
\special{pa 3208 3608}%
\special{fp}%
\special{pa 3214 3614}%
\special{pa 3240 3640}%
\special{fp}%
\special{pa 3246 3646}%
\special{pa 3272 3672}%
\special{fp}%
\special{pa 3278 3678}%
\special{pa 3304 3704}%
\special{fp}%
\special{pa 3310 3710}%
\special{pa 3336 3736}%
\special{fp}%
\special{pa 3342 3742}%
\special{pa 3368 3768}%
\special{fp}%
\special{pa 3374 3774}%
\special{pa 3400 3800}%
\special{fp}%
% FUNC 2 0 3 0 Black White  
% 10 1600 1800 3600 3800 1800 3600 4600 3600 1800 800 1600 3600 3600 3600 0 0 0 1 0 0
% 1-1/sqrt(2)
\special{pn 8}%
\special{pn 8}%
\special{pa 2620 1800}%
\special{pa 2620 1808}%
\special{fp}%
\special{pa 2620 1845}%
\special{pa 2620 1853}%
\special{fp}%
\special{pa 2620 1890}%
\special{pa 2620 1898}%
\special{fp}%
\special{pa 2620 1935}%
\special{pa 2620 1943}%
\special{fp}%
\special{pa 2620 1980}%
\special{pa 2620 1988}%
\special{fp}%
\special{pa 2620 2025}%
\special{pa 2620 2033}%
\special{fp}%
\special{pa 2620 2070}%
\special{pa 2620 2078}%
\special{fp}%
\special{pa 2620 2115}%
\special{pa 2620 2123}%
\special{fp}%
\special{pa 2620 2160}%
\special{pa 2620 2168}%
\special{fp}%
\special{pa 2620 2205}%
\special{pa 2620 2213}%
\special{fp}%
\special{pa 2620 2250}%
\special{pa 2620 2258}%
\special{fp}%
\special{pa 2620 2295}%
\special{pa 2620 2303}%
\special{fp}%
\special{pa 2620 2340}%
\special{pa 2620 2348}%
\special{fp}%
\special{pa 2620 2385}%
\special{pa 2620 2393}%
\special{fp}%
\special{pa 2620 2430}%
\special{pa 2620 2438}%
\special{fp}%
\special{pa 2620 2475}%
\special{pa 2620 2483}%
\special{fp}%
\special{pa 2620 2520}%
\special{pa 2620 2528}%
\special{fp}%
\special{pa 2620 2565}%
\special{pa 2620 2573}%
\special{fp}%
\special{pa 2620 2610}%
\special{pa 2620 2618}%
\special{fp}%
\special{pa 2620 2655}%
\special{pa 2620 2663}%
\special{fp}%
\special{pa 2620 2700}%
\special{pa 2620 2708}%
\special{fp}%
\special{pa 2620 2745}%
\special{pa 2620 2753}%
\special{fp}%
\special{pa 2620 2790}%
\special{pa 2620 2798}%
\special{fp}%
\special{pa 2620 2835}%
\special{pa 2620 2843}%
\special{fp}%
\special{pa 2620 2880}%
\special{pa 2620 2888}%
\special{fp}%
\special{pa 2620 2925}%
\special{pa 2620 2933}%
\special{fp}%
\special{pa 2620 2970}%
\special{pa 2620 2978}%
\special{fp}%
\special{pa 2620 3015}%
\special{pa 2620 3023}%
\special{fp}%
\special{pa 2620 3060}%
\special{pa 2620 3068}%
\special{fp}%
\special{pa 2620 3105}%
\special{pa 2620 3113}%
\special{fp}%
\special{pa 2620 3150}%
\special{pa 2620 3158}%
\special{fp}%
\special{pa 2620 3195}%
\special{pa 2620 3203}%
\special{fp}%
\special{pa 2620 3240}%
\special{pa 2620 3248}%
\special{fp}%
\special{pa 2620 3285}%
\special{pa 2620 3293}%
\special{fp}%
\special{pa 2620 3330}%
\special{pa 2620 3338}%
\special{fp}%
\special{pa 2620 3375}%
\special{pa 2620 3383}%
\special{fp}%
\special{pa 2620 3420}%
\special{pa 2620 3428}%
\special{fp}%
\special{pa 2620 3465}%
\special{pa 2620 3473}%
\special{fp}%
\special{pa 2620 3510}%
\special{pa 2620 3518}%
\special{fp}%
\special{pa 2620 3555}%
\special{pa 2620 3563}%
\special{fp}%
\special{pa 2620 3600}%
\special{pa 2620 3600}%
\special{fp}%
\special{pn 8}%
\special{pa 2620 3609}%
\special{pa 2620 3650}%
\special{fp}%
\special{pa 2620 3659}%
\special{pa 2620 3700}%
\special{fp}%
\special{pa 2620 3709}%
\special{pa 2620 3750}%
\special{fp}%
\special{pa 2620 3759}%
\special{pa 2620 3800}%
\special{fp}%
% FUNC 2 0 3 0 Black White  
% 10 1600 1800 3600 3800 1800 3600 4600 3600 1800 800 1600 3600 3600 3600 0 0 0 1 0 0
% 0.5
\special{pn 8}%
\special{pn 8}%
\special{pa 3200 1800}%
\special{pa 3200 1808}%
\special{fp}%
\special{pa 3200 1845}%
\special{pa 3200 1853}%
\special{fp}%
\special{pa 3200 1890}%
\special{pa 3200 1898}%
\special{fp}%
\special{pa 3200 1935}%
\special{pa 3200 1943}%
\special{fp}%
\special{pa 3200 1980}%
\special{pa 3200 1988}%
\special{fp}%
\special{pa 3200 2025}%
\special{pa 3200 2033}%
\special{fp}%
\special{pa 3200 2070}%
\special{pa 3200 2078}%
\special{fp}%
\special{pa 3200 2115}%
\special{pa 3200 2123}%
\special{fp}%
\special{pa 3200 2160}%
\special{pa 3200 2168}%
\special{fp}%
\special{pa 3200 2205}%
\special{pa 3200 2213}%
\special{fp}%
\special{pa 3200 2250}%
\special{pa 3200 2258}%
\special{fp}%
\special{pa 3200 2295}%
\special{pa 3200 2303}%
\special{fp}%
\special{pa 3200 2340}%
\special{pa 3200 2348}%
\special{fp}%
\special{pa 3200 2385}%
\special{pa 3200 2393}%
\special{fp}%
\special{pa 3200 2430}%
\special{pa 3200 2438}%
\special{fp}%
\special{pa 3200 2475}%
\special{pa 3200 2483}%
\special{fp}%
\special{pa 3200 2520}%
\special{pa 3200 2528}%
\special{fp}%
\special{pa 3200 2565}%
\special{pa 3200 2573}%
\special{fp}%
\special{pa 3200 2610}%
\special{pa 3200 2618}%
\special{fp}%
\special{pa 3200 2655}%
\special{pa 3200 2663}%
\special{fp}%
\special{pa 3200 2700}%
\special{pa 3200 2708}%
\special{fp}%
\special{pa 3200 2745}%
\special{pa 3200 2753}%
\special{fp}%
\special{pa 3200 2790}%
\special{pa 3200 2798}%
\special{fp}%
\special{pa 3200 2835}%
\special{pa 3200 2843}%
\special{fp}%
\special{pa 3200 2880}%
\special{pa 3200 2888}%
\special{fp}%
\special{pa 3200 2925}%
\special{pa 3200 2933}%
\special{fp}%
\special{pa 3200 2970}%
\special{pa 3200 2978}%
\special{fp}%
\special{pa 3200 3015}%
\special{pa 3200 3023}%
\special{fp}%
\special{pa 3200 3060}%
\special{pa 3200 3068}%
\special{fp}%
\special{pa 3200 3105}%
\special{pa 3200 3113}%
\special{fp}%
\special{pa 3200 3150}%
\special{pa 3200 3158}%
\special{fp}%
\special{pa 3200 3195}%
\special{pa 3200 3203}%
\special{fp}%
\special{pa 3200 3240}%
\special{pa 3200 3248}%
\special{fp}%
\special{pa 3200 3285}%
\special{pa 3200 3293}%
\special{fp}%
\special{pa 3200 3330}%
\special{pa 3200 3338}%
\special{fp}%
\special{pa 3200 3375}%
\special{pa 3200 3383}%
\special{fp}%
\special{pa 3200 3420}%
\special{pa 3200 3428}%
\special{fp}%
\special{pa 3200 3465}%
\special{pa 3200 3473}%
\special{fp}%
\special{pa 3200 3510}%
\special{pa 3200 3518}%
\special{fp}%
\special{pa 3200 3555}%
\special{pa 3200 3563}%
\special{fp}%
\special{pa 3200 3600}%
\special{pa 3200 3600}%
\special{fp}%
\special{pn 8}%
\special{pa 3200 3609}%
\special{pa 3200 3650}%
\special{fp}%
\special{pa 3200 3659}%
\special{pa 3200 3700}%
\special{fp}%
\special{pa 3200 3709}%
\special{pa 3200 3750}%
\special{fp}%
\special{pa 3200 3759}%
\special{pa 3200 3800}%
\special{fp}%
% FUNC 1 0 3 0 Black White  
% 9 1600 1800 3600 3800 1800 3600 4600 3600 1800 800 2620 1800 3200 3800 0 4 0 0
% 0.5-x
\special{pn 13}%
\special{pn 13}%
\special{pa 1600 2000}%
\special{pa 1609 2009}%
\special{ip}%
\special{pa 1632 2032}%
\special{pa 1641 2041}%
\special{ip}%
\special{pa 1664 2064}%
\special{pa 1673 2073}%
\special{ip}%
\special{pa 1696 2096}%
\special{pa 1705 2105}%
\special{ip}%
\special{pa 1728 2128}%
\special{pa 1737 2137}%
\special{ip}%
\special{pa 1759 2159}%
\special{pa 1769 2169}%
\special{ip}%
\special{pa 1791 2191}%
\special{pa 1800 2200}%
\special{ip}%
\special{pa 1823 2223}%
\special{pa 1832 2232}%
\special{ip}%
\special{pa 1855 2255}%
\special{pa 1864 2264}%
\special{ip}%
\special{pa 1887 2287}%
\special{pa 1896 2296}%
\special{ip}%
\special{pa 1919 2319}%
\special{pa 1928 2328}%
\special{ip}%
\special{pa 1951 2351}%
\special{pa 1960 2360}%
\special{ip}%
\special{pa 1982 2382}%
\special{pa 1992 2392}%
\special{ip}%
\special{pa 2014 2414}%
\special{pa 2024 2424}%
\special{ip}%
\special{pa 2046 2446}%
\special{pa 2055 2455}%
\special{ip}%
\special{pa 2078 2478}%
\special{pa 2087 2487}%
\special{ip}%
\special{pa 2110 2510}%
\special{pa 2119 2519}%
\special{ip}%
\special{pa 2142 2542}%
\special{pa 2151 2551}%
\special{ip}%
\special{pa 2174 2574}%
\special{pa 2183 2583}%
\special{ip}%
\special{pa 2206 2606}%
\special{pa 2215 2615}%
\special{ip}%
\special{pa 2238 2638}%
\special{pa 2247 2647}%
\special{ip}%
\special{pa 2269 2669}%
\special{pa 2279 2679}%
\special{ip}%
\special{pa 2301 2701}%
\special{pa 2310 2710}%
\special{ip}%
\special{pa 2333 2733}%
\special{pa 2342 2742}%
\special{ip}%
\special{pa 2365 2765}%
\special{pa 2374 2774}%
\special{ip}%
\special{pa 2397 2797}%
\special{pa 2406 2806}%
\special{ip}%
\special{pa 2429 2829}%
\special{pa 2438 2838}%
\special{ip}%
\special{pa 2461 2861}%
\special{pa 2470 2870}%
\special{ip}%
\special{pa 2492 2892}%
\special{pa 2502 2902}%
\special{ip}%
\special{pa 2524 2924}%
\special{pa 2534 2934}%
\special{ip}%
\special{pa 2556 2956}%
\special{pa 2565 2965}%
\special{ip}%
\special{pa 2588 2988}%
\special{pa 2597 2997}%
\special{ip}%
\special{pa 2620 3020}%
\special{pa 2620 3020}%
\special{ip}%
\special{pa 2620 3020}%
\special{pa 3200 3600}%
\special{fp}%
\special{pn 13}%
\special{pa 3210 3610}%
\special{pa 3233 3633}%
\special{ip}%
\special{pa 3243 3643}%
\special{pa 3267 3667}%
\special{ip}%
\special{pa 3276 3676}%
\special{pa 3300 3700}%
\special{ip}%
\special{pa 3310 3710}%
\special{pa 3333 3733}%
\special{ip}%
\special{pa 3343 3743}%
\special{pa 3367 3767}%
\special{ip}%
\special{pa 3376 3776}%
\special{pa 3400 3800}%
\special{ip}%
% STR 2 0 3 0 Black White  
% 4 2620 3500 2620 3600 4 0 1 0
% $1-\sqrt{1-t}$
\put(26.2000,-36.0000){\makebox(0,0)[rt]{{\colorbox[named]{White}{$1-\sqrt{1-t}$}}}}%
% STR 2 0 3 0 Black White  
% 4 3200 3500 3200 3600 4 0 1 0
% $t$
\put(32.0000,-36.0000){\makebox(0,0)[rt]{{\colorbox[named]{White}{$t$}}}}%
\end{picture}}%
}
     \end{center}
題意の回転体を$\Pi$で切断した断面の面積$S(t)$とする.$E$上の点$P(x,t-x,0)$に対して,$Q(t/3,t/3,t/3)$との距離の$2$乗は,
     \begin{align*}
     |PQ|^2&=\left(x-\frac{t}{3}\right)^2+\left(t-x-\frac{t}{3}\right)^2+\left(\frac{t}{3}\right)^2 \\
     &=2\left(x-\frac{t}{2}\right)^2+(\text{$x$に寄らない定数項})
     \end{align*}
である.これは,\eqref{3}の範囲内では,
     \[\frac{t}{2}\le1-\sqrt{1-t}\le t\]
ゆえ,$x=t$で最大値$M$,$x=1-\sqrt{1-t}$で最小値$m$をとる.

 \begin{minipage}{0.5\hsize}
     \begin{center}
     \scalebox{.5}{%WinTpicVersion4.32a
{\unitlength 0.1in%
\begin{picture}(22.2000,22.0000)(11.8000,-40.0000)%
% STR 2 0 3 0 Black White  
% 4 1390 3797 1390 3810 4 1200 0 0
% O
\put(13.9000,-38.1000){\makebox(0,0)[rt]{O}}%
% STR 2 0 3 0 Black White  
% 4 1360 1787 1360 1800 4 1200 0 0
% $y$
\put(13.6000,-18.0000){\makebox(0,0)[rt]{$y$}}%
% STR 2 0 3 0 Black White  
% 4 3400 3827 3400 3840 4 1200 0 0
% $x$
\put(34.0000,-38.4000){\makebox(0,0)[rt]{$x$}}%
% VECTOR 2 0 3 0 Black White  
% 2 1400 4000 1400 1800
% 
\special{pn 8}%
\special{pa 1400 4000}%
\special{pa 1400 1800}%
\special{fp}%
\special{sh 1}%
\special{pa 1400 1800}%
\special{pa 1380 1867}%
\special{pa 1400 1853}%
\special{pa 1420 1867}%
\special{pa 1400 1800}%
\special{fp}%
% VECTOR 2 0 3 0 Black White  
% 2 1200 3800 3400 3800
% 
\special{pn 8}%
\special{pa 1200 3800}%
\special{pa 3400 3800}%
\special{fp}%
\special{sh 1}%
\special{pa 3400 3800}%
\special{pa 3333 3780}%
\special{pa 3347 3800}%
\special{pa 3333 3820}%
\special{pa 3400 3800}%
\special{fp}%
% FUNC 2 0 3 0 Black White  
% 9 1200 1800 3400 4000 1400 3800 2400 3800 1400 2800 1200 1800 3400 4000 0 2 0 0
% 2(X-0.5)^2+0.5
\special{pn 8}%
\special{pa 1200 2320}%
\special{pa 1215 2362}%
\special{pa 1220 2375}%
\special{pa 1225 2389}%
\special{pa 1230 2402}%
\special{pa 1235 2416}%
\special{pa 1265 2494}%
\special{pa 1270 2506}%
\special{pa 1275 2519}%
\special{pa 1280 2531}%
\special{pa 1285 2544}%
\special{pa 1315 2616}%
\special{pa 1320 2627}%
\special{pa 1325 2639}%
\special{pa 1330 2650}%
\special{pa 1335 2662}%
\special{pa 1365 2728}%
\special{pa 1370 2738}%
\special{pa 1375 2749}%
\special{pa 1380 2759}%
\special{pa 1385 2770}%
\special{pa 1415 2830}%
\special{pa 1420 2839}%
\special{pa 1425 2849}%
\special{pa 1430 2858}%
\special{pa 1435 2868}%
\special{pa 1465 2922}%
\special{pa 1470 2930}%
\special{pa 1475 2939}%
\special{pa 1480 2947}%
\special{pa 1485 2956}%
\special{pa 1515 3004}%
\special{pa 1520 3011}%
\special{pa 1525 3019}%
\special{pa 1530 3026}%
\special{pa 1535 3034}%
\special{pa 1565 3076}%
\special{pa 1570 3082}%
\special{pa 1575 3089}%
\special{pa 1580 3095}%
\special{pa 1585 3102}%
\special{pa 1615 3138}%
\special{pa 1620 3143}%
\special{pa 1625 3149}%
\special{pa 1630 3154}%
\special{pa 1635 3160}%
\special{pa 1665 3190}%
\special{pa 1670 3194}%
\special{pa 1675 3199}%
\special{pa 1680 3203}%
\special{pa 1685 3208}%
\special{pa 1715 3232}%
\special{pa 1720 3235}%
\special{pa 1725 3239}%
\special{pa 1730 3242}%
\special{pa 1735 3246}%
\special{pa 1765 3264}%
\special{pa 1770 3266}%
\special{pa 1775 3269}%
\special{pa 1780 3271}%
\special{pa 1785 3274}%
\special{pa 1815 3286}%
\special{pa 1820 3287}%
\special{pa 1825 3289}%
\special{pa 1830 3290}%
\special{pa 1835 3292}%
\special{pa 1865 3298}%
\special{pa 1870 3298}%
\special{pa 1875 3299}%
\special{pa 1880 3299}%
\special{pa 1885 3300}%
\special{pa 1915 3300}%
\special{pa 1920 3299}%
\special{pa 1925 3299}%
\special{pa 1930 3298}%
\special{pa 1935 3298}%
\special{pa 1965 3292}%
\special{pa 1970 3290}%
\special{pa 1975 3289}%
\special{pa 1980 3287}%
\special{pa 1985 3286}%
\special{pa 2015 3274}%
\special{pa 2020 3271}%
\special{pa 2025 3269}%
\special{pa 2030 3266}%
\special{pa 2035 3264}%
\special{pa 2065 3246}%
\special{pa 2070 3242}%
\special{pa 2075 3239}%
\special{pa 2080 3235}%
\special{pa 2085 3232}%
\special{pa 2115 3208}%
\special{pa 2120 3203}%
\special{pa 2125 3199}%
\special{pa 2130 3194}%
\special{pa 2135 3190}%
\special{pa 2165 3160}%
\special{pa 2170 3154}%
\special{pa 2175 3149}%
\special{pa 2180 3143}%
\special{pa 2185 3138}%
\special{pa 2215 3102}%
\special{pa 2220 3095}%
\special{pa 2225 3089}%
\special{pa 2230 3082}%
\special{pa 2235 3076}%
\special{pa 2265 3034}%
\special{pa 2270 3026}%
\special{pa 2275 3019}%
\special{pa 2280 3011}%
\special{pa 2285 3004}%
\special{pa 2315 2956}%
\special{pa 2320 2947}%
\special{pa 2325 2939}%
\special{pa 2330 2930}%
\special{pa 2335 2922}%
\special{pa 2365 2868}%
\special{pa 2370 2858}%
\special{pa 2375 2849}%
\special{pa 2380 2839}%
\special{pa 2385 2830}%
\special{pa 2415 2770}%
\special{pa 2420 2759}%
\special{pa 2425 2749}%
\special{pa 2430 2738}%
\special{pa 2435 2728}%
\special{pa 2465 2662}%
\special{pa 2470 2650}%
\special{pa 2475 2639}%
\special{pa 2480 2627}%
\special{pa 2485 2616}%
\special{pa 2515 2544}%
\special{pa 2520 2531}%
\special{pa 2525 2519}%
\special{pa 2530 2506}%
\special{pa 2535 2494}%
\special{pa 2565 2416}%
\special{pa 2570 2402}%
\special{pa 2575 2389}%
\special{pa 2580 2375}%
\special{pa 2585 2362}%
\special{pa 2615 2278}%
\special{pa 2620 2263}%
\special{pa 2625 2249}%
\special{pa 2630 2234}%
\special{pa 2635 2220}%
\special{pa 2665 2130}%
\special{pa 2670 2114}%
\special{pa 2675 2099}%
\special{pa 2680 2083}%
\special{pa 2685 2068}%
\special{pa 2715 1972}%
\special{pa 2720 1955}%
\special{pa 2725 1939}%
\special{pa 2730 1922}%
\special{pa 2735 1906}%
\special{pa 2765 1804}%
\special{pa 2766 1800}%
\special{fp}%
% FUNC 2 0 3 0 Black White  
% 10 1200 1800 3400 4000 1400 3800 2400 3800 1400 2800 1200 3800 3400 3800 0 0 0 1 0 0
% 1
\special{pn 8}%
\special{pn 8}%
\special{pa 2400 1800}%
\special{pa 2400 1808}%
\special{fp}%
\special{pa 2400 1845}%
\special{pa 2400 1854}%
\special{fp}%
\special{pa 2400 1891}%
\special{pa 2400 1899}%
\special{fp}%
\special{pa 2400 1936}%
\special{pa 2400 1944}%
\special{fp}%
\special{pa 2400 1982}%
\special{pa 2400 1990}%
\special{fp}%
\special{pa 2400 2027}%
\special{pa 2400 2035}%
\special{fp}%
\special{pa 2400 2073}%
\special{pa 2400 2081}%
\special{fp}%
\special{pa 2400 2118}%
\special{pa 2400 2126}%
\special{fp}%
\special{pa 2400 2164}%
\special{pa 2400 2172}%
\special{fp}%
\special{pa 2400 2209}%
\special{pa 2400 2217}%
\special{fp}%
\special{pa 2400 2255}%
\special{pa 2400 2263}%
\special{fp}%
\special{pa 2400 2300}%
\special{pa 2400 2308}%
\special{fp}%
\special{pa 2400 2345}%
\special{pa 2400 2354}%
\special{fp}%
\special{pa 2400 2391}%
\special{pa 2400 2399}%
\special{fp}%
\special{pa 2400 2436}%
\special{pa 2400 2444}%
\special{fp}%
\special{pa 2400 2482}%
\special{pa 2400 2490}%
\special{fp}%
\special{pa 2400 2527}%
\special{pa 2400 2535}%
\special{fp}%
\special{pa 2400 2573}%
\special{pa 2400 2581}%
\special{fp}%
\special{pa 2400 2618}%
\special{pa 2400 2626}%
\special{fp}%
\special{pa 2400 2664}%
\special{pa 2400 2672}%
\special{fp}%
\special{pa 2400 2709}%
\special{pa 2400 2717}%
\special{fp}%
\special{pa 2400 2755}%
\special{pa 2400 2763}%
\special{fp}%
\special{pa 2400 2800}%
\special{pa 2400 2808}%
\special{fp}%
\special{pa 2400 2845}%
\special{pa 2400 2854}%
\special{fp}%
\special{pa 2400 2891}%
\special{pa 2400 2899}%
\special{fp}%
\special{pa 2400 2936}%
\special{pa 2400 2944}%
\special{fp}%
\special{pa 2400 2982}%
\special{pa 2400 2990}%
\special{fp}%
\special{pa 2400 3027}%
\special{pa 2400 3035}%
\special{fp}%
\special{pa 2400 3073}%
\special{pa 2400 3081}%
\special{fp}%
\special{pa 2400 3118}%
\special{pa 2400 3126}%
\special{fp}%
\special{pa 2400 3164}%
\special{pa 2400 3172}%
\special{fp}%
\special{pa 2400 3209}%
\special{pa 2400 3217}%
\special{fp}%
\special{pa 2400 3255}%
\special{pa 2400 3263}%
\special{fp}%
\special{pa 2400 3300}%
\special{pa 2400 3308}%
\special{fp}%
\special{pa 2400 3345}%
\special{pa 2400 3354}%
\special{fp}%
\special{pa 2400 3391}%
\special{pa 2400 3399}%
\special{fp}%
\special{pa 2400 3436}%
\special{pa 2400 3444}%
\special{fp}%
\special{pa 2400 3482}%
\special{pa 2400 3490}%
\special{fp}%
\special{pa 2400 3527}%
\special{pa 2400 3535}%
\special{fp}%
\special{pa 2400 3573}%
\special{pa 2400 3581}%
\special{fp}%
\special{pa 2400 3618}%
\special{pa 2400 3626}%
\special{fp}%
\special{pa 2400 3664}%
\special{pa 2400 3672}%
\special{fp}%
\special{pa 2400 3709}%
\special{pa 2400 3717}%
\special{fp}%
\special{pa 2400 3755}%
\special{pa 2400 3763}%
\special{fp}%
\special{pa 2400 3800}%
\special{pa 2400 3800}%
\special{fp}%
\special{pn 8}%
\special{pa 2400 3809}%
\special{pa 2400 3850}%
\special{fp}%
\special{pa 2400 3859}%
\special{pa 2400 3900}%
\special{fp}%
\special{pa 2400 3909}%
\special{pa 2400 3950}%
\special{fp}%
\special{pa 2400 3959}%
\special{pa 2400 4000}%
\special{fp}%
% FUNC 2 0 3 0 Black White  
% 9 1200 1800 3400 4000 1400 3800 2400 3800 1400 3000 1200 3800 3400 3800 0 0 0 1
% 0.5
\special{pn 8}%
\special{pn 8}%
\special{pa 1900 1800}%
\special{pa 1900 1808}%
\special{fp}%
\special{pa 1900 1845}%
\special{pa 1900 1854}%
\special{fp}%
\special{pa 1900 1891}%
\special{pa 1900 1899}%
\special{fp}%
\special{pa 1900 1936}%
\special{pa 1900 1944}%
\special{fp}%
\special{pa 1900 1982}%
\special{pa 1900 1990}%
\special{fp}%
\special{pa 1900 2027}%
\special{pa 1900 2035}%
\special{fp}%
\special{pa 1900 2073}%
\special{pa 1900 2081}%
\special{fp}%
\special{pa 1900 2118}%
\special{pa 1900 2126}%
\special{fp}%
\special{pa 1900 2164}%
\special{pa 1900 2172}%
\special{fp}%
\special{pa 1900 2209}%
\special{pa 1900 2217}%
\special{fp}%
\special{pa 1900 2255}%
\special{pa 1900 2263}%
\special{fp}%
\special{pa 1900 2300}%
\special{pa 1900 2308}%
\special{fp}%
\special{pa 1900 2345}%
\special{pa 1900 2354}%
\special{fp}%
\special{pa 1900 2391}%
\special{pa 1900 2399}%
\special{fp}%
\special{pa 1900 2436}%
\special{pa 1900 2444}%
\special{fp}%
\special{pa 1900 2482}%
\special{pa 1900 2490}%
\special{fp}%
\special{pa 1900 2527}%
\special{pa 1900 2535}%
\special{fp}%
\special{pa 1900 2573}%
\special{pa 1900 2581}%
\special{fp}%
\special{pa 1900 2618}%
\special{pa 1900 2626}%
\special{fp}%
\special{pa 1900 2664}%
\special{pa 1900 2672}%
\special{fp}%
\special{pa 1900 2709}%
\special{pa 1900 2717}%
\special{fp}%
\special{pa 1900 2755}%
\special{pa 1900 2763}%
\special{fp}%
\special{pa 1900 2800}%
\special{pa 1900 2808}%
\special{fp}%
\special{pa 1900 2845}%
\special{pa 1900 2854}%
\special{fp}%
\special{pa 1900 2891}%
\special{pa 1900 2899}%
\special{fp}%
\special{pa 1900 2936}%
\special{pa 1900 2944}%
\special{fp}%
\special{pa 1900 2982}%
\special{pa 1900 2990}%
\special{fp}%
\special{pa 1900 3027}%
\special{pa 1900 3035}%
\special{fp}%
\special{pa 1900 3073}%
\special{pa 1900 3081}%
\special{fp}%
\special{pa 1900 3118}%
\special{pa 1900 3126}%
\special{fp}%
\special{pa 1900 3164}%
\special{pa 1900 3172}%
\special{fp}%
\special{pa 1900 3209}%
\special{pa 1900 3217}%
\special{fp}%
\special{pa 1900 3255}%
\special{pa 1900 3263}%
\special{fp}%
\special{pa 1900 3300}%
\special{pa 1900 3308}%
\special{fp}%
\special{pa 1900 3345}%
\special{pa 1900 3354}%
\special{fp}%
\special{pa 1900 3391}%
\special{pa 1900 3399}%
\special{fp}%
\special{pa 1900 3436}%
\special{pa 1900 3444}%
\special{fp}%
\special{pa 1900 3482}%
\special{pa 1900 3490}%
\special{fp}%
\special{pa 1900 3527}%
\special{pa 1900 3535}%
\special{fp}%
\special{pa 1900 3573}%
\special{pa 1900 3581}%
\special{fp}%
\special{pa 1900 3618}%
\special{pa 1900 3626}%
\special{fp}%
\special{pa 1900 3664}%
\special{pa 1900 3672}%
\special{fp}%
\special{pa 1900 3709}%
\special{pa 1900 3717}%
\special{fp}%
\special{pa 1900 3755}%
\special{pa 1900 3763}%
\special{fp}%
\special{pa 1900 3800}%
\special{pa 1900 3800}%
\special{fp}%
\special{pn 8}%
\special{pa 1900 3809}%
\special{pa 1900 3850}%
\special{fp}%
\special{pa 1900 3859}%
\special{pa 1900 3900}%
\special{fp}%
\special{pa 1900 3909}%
\special{pa 1900 3950}%
\special{fp}%
\special{pa 1900 3959}%
\special{pa 1900 4000}%
\special{fp}%
% FUNC 2 0 3 0 Black White  
% 10 1200 1800 3400 4000 1400 3800 2400 3800 1400 3000 1200 3800 3400 3800 0 0 0 1 0 0
% 1.3
\special{pn 8}%
\special{pn 8}%
\special{pa 2700 1800}%
\special{pa 2700 1808}%
\special{fp}%
\special{pa 2700 1845}%
\special{pa 2700 1854}%
\special{fp}%
\special{pa 2700 1891}%
\special{pa 2700 1899}%
\special{fp}%
\special{pa 2700 1936}%
\special{pa 2700 1944}%
\special{fp}%
\special{pa 2700 1982}%
\special{pa 2700 1990}%
\special{fp}%
\special{pa 2700 2027}%
\special{pa 2700 2035}%
\special{fp}%
\special{pa 2700 2073}%
\special{pa 2700 2081}%
\special{fp}%
\special{pa 2700 2118}%
\special{pa 2700 2126}%
\special{fp}%
\special{pa 2700 2164}%
\special{pa 2700 2172}%
\special{fp}%
\special{pa 2700 2209}%
\special{pa 2700 2217}%
\special{fp}%
\special{pa 2700 2255}%
\special{pa 2700 2263}%
\special{fp}%
\special{pa 2700 2300}%
\special{pa 2700 2308}%
\special{fp}%
\special{pa 2700 2345}%
\special{pa 2700 2354}%
\special{fp}%
\special{pa 2700 2391}%
\special{pa 2700 2399}%
\special{fp}%
\special{pa 2700 2436}%
\special{pa 2700 2444}%
\special{fp}%
\special{pa 2700 2482}%
\special{pa 2700 2490}%
\special{fp}%
\special{pa 2700 2527}%
\special{pa 2700 2535}%
\special{fp}%
\special{pa 2700 2573}%
\special{pa 2700 2581}%
\special{fp}%
\special{pa 2700 2618}%
\special{pa 2700 2626}%
\special{fp}%
\special{pa 2700 2664}%
\special{pa 2700 2672}%
\special{fp}%
\special{pa 2700 2709}%
\special{pa 2700 2717}%
\special{fp}%
\special{pa 2700 2755}%
\special{pa 2700 2763}%
\special{fp}%
\special{pa 2700 2800}%
\special{pa 2700 2808}%
\special{fp}%
\special{pa 2700 2845}%
\special{pa 2700 2854}%
\special{fp}%
\special{pa 2700 2891}%
\special{pa 2700 2899}%
\special{fp}%
\special{pa 2700 2936}%
\special{pa 2700 2944}%
\special{fp}%
\special{pa 2700 2982}%
\special{pa 2700 2990}%
\special{fp}%
\special{pa 2700 3027}%
\special{pa 2700 3035}%
\special{fp}%
\special{pa 2700 3073}%
\special{pa 2700 3081}%
\special{fp}%
\special{pa 2700 3118}%
\special{pa 2700 3126}%
\special{fp}%
\special{pa 2700 3164}%
\special{pa 2700 3172}%
\special{fp}%
\special{pa 2700 3209}%
\special{pa 2700 3217}%
\special{fp}%
\special{pa 2700 3255}%
\special{pa 2700 3263}%
\special{fp}%
\special{pa 2700 3300}%
\special{pa 2700 3308}%
\special{fp}%
\special{pa 2700 3345}%
\special{pa 2700 3354}%
\special{fp}%
\special{pa 2700 3391}%
\special{pa 2700 3399}%
\special{fp}%
\special{pa 2700 3436}%
\special{pa 2700 3444}%
\special{fp}%
\special{pa 2700 3482}%
\special{pa 2700 3490}%
\special{fp}%
\special{pa 2700 3527}%
\special{pa 2700 3535}%
\special{fp}%
\special{pa 2700 3573}%
\special{pa 2700 3581}%
\special{fp}%
\special{pa 2700 3618}%
\special{pa 2700 3626}%
\special{fp}%
\special{pa 2700 3664}%
\special{pa 2700 3672}%
\special{fp}%
\special{pa 2700 3709}%
\special{pa 2700 3717}%
\special{fp}%
\special{pa 2700 3755}%
\special{pa 2700 3763}%
\special{fp}%
\special{pa 2700 3800}%
\special{pa 2700 3800}%
\special{fp}%
\special{pn 8}%
\special{pa 2700 3809}%
\special{pa 2700 3850}%
\special{fp}%
\special{pa 2700 3859}%
\special{pa 2700 3900}%
\special{fp}%
\special{pa 2700 3909}%
\special{pa 2700 3950}%
\special{fp}%
\special{pa 2700 3959}%
\special{pa 2700 4000}%
\special{fp}%
% STR 2 0 3 0 Black White  
% 4 1900 3700 1900 3800 2 0 1 0
% $t/2$
\put(19.0000,-38.0000){\makebox(0,0)[lb]{{\colorbox[named]{White}{$t/2$}}}}%
% STR 2 0 3 0 Black White  
% 4 2700 3700 2700 3800 2 0 1 0
% $t$
\put(27.0000,-38.0000){\makebox(0,0)[lb]{{\colorbox[named]{White}{$t$}}}}%
% STR 2 0 3 0 Black White  
% 4 2400 2080 2400 2180 5 0 1 0
% $1-\sqrt{1-t}$
\put(24.0000,-21.8000){\makebox(0,0){{\colorbox[named]{White}{$1-\sqrt{1-t}$}}}}%
\end{picture}}%
}
     \end{center}
 \end{minipage}
 \begin{minipage}{0.5\hsize}
  \begin{center}
       \scalebox{.5}{%WinTpicVersion4.32a
{\unitlength 0.1in%
\begin{picture}(37.6000,37.6000)(5.2000,-32.8000)%
% LINE 2 0 3 0 Black White  
% 6 800 2400 1800 2400 2400 1400 1800 2400 800 2400 2400 1400
% 
\special{pn 8}%
\special{pa 800 2400}%
\special{pa 1800 2400}%
\special{fp}%
\special{pa 2400 1400}%
\special{pa 1800 2400}%
\special{fp}%
\special{pa 800 2400}%
\special{pa 2400 1400}%
\special{fp}%
% LINE 2 2 3 0 Black White  
% 2 600 2400 2600 2400
% 
\special{pn 8}%
\special{pa 600 2400}%
\special{pa 2600 2400}%
\special{dt 0.045}%
% STR 2 0 3 0 Black White  
% 4 2200 1900 2200 2000 5 0 1 0
% $m$
\put(22.0000,-20.0000){\makebox(0,0){{\colorbox[named]{White}{$m$}}}}%
% STR 2 0 3 0 Black White  
% 4 1400 1700 1400 1800 5 0 1 0
% $M$
\put(14.0000,-18.0000){\makebox(0,0){{\colorbox[named]{White}{$M$}}}}%
% STR 2 0 3 0 Black White  
% 4 1470 2470 1470 2570 5 0 1 0
% $E$
\put(14.7000,-25.7000){\makebox(0,0){{\colorbox[named]{White}{$E$}}}}%
% CIRCLE 2 2 3 0 Black White  
% 4 2400 1400 2400 2550 2400 2550 2400 2550
% 
\special{pn 8}%
\special{pn 8}%
\special{pa 3550 1400}%
\special{pa 3550 1408}%
\special{fp}%
\special{pa 3549 1445}%
\special{pa 3549 1453}%
\special{fp}%
\special{pa 3546 1490}%
\special{pa 3546 1498}%
\special{fp}%
\special{pa 3542 1534}%
\special{pa 3541 1542}%
\special{fp}%
\special{pa 3536 1579}%
\special{pa 3535 1587}%
\special{fp}%
\special{pa 3528 1623}%
\special{pa 3526 1631}%
\special{fp}%
\special{pa 3519 1667}%
\special{pa 3517 1675}%
\special{fp}%
\special{pa 3507 1710}%
\special{pa 3505 1718}%
\special{fp}%
\special{pa 3494 1753}%
\special{pa 3492 1761}%
\special{fp}%
\special{pa 3480 1796}%
\special{pa 3477 1803}%
\special{fp}%
\special{pa 3464 1838}%
\special{pa 3460 1845}%
\special{fp}%
\special{pa 3446 1879}%
\special{pa 3442 1886}%
\special{fp}%
\special{pa 3426 1919}%
\special{pa 3423 1926}%
\special{fp}%
\special{pa 3405 1959}%
\special{pa 3401 1966}%
\special{fp}%
\special{pa 3383 1998}%
\special{pa 3378 2004}%
\special{fp}%
\special{pa 3358 2035}%
\special{pa 3354 2042}%
\special{fp}%
\special{pa 3333 2073}%
\special{pa 3328 2079}%
\special{fp}%
\special{pa 3306 2108}%
\special{pa 3301 2115}%
\special{fp}%
\special{pa 3278 2143}%
\special{pa 3272 2149}%
\special{fp}%
\special{pa 3248 2177}%
\special{pa 3242 2183}%
\special{fp}%
\special{pa 3217 2209}%
\special{pa 3211 2215}%
\special{fp}%
\special{pa 3185 2240}%
\special{pa 3179 2246}%
\special{fp}%
\special{pa 3151 2271}%
\special{pa 3145 2276}%
\special{fp}%
\special{pa 3117 2300}%
\special{pa 3111 2304}%
\special{fp}%
\special{pa 3081 2327}%
\special{pa 3075 2332}%
\special{fp}%
\special{pa 3044 2352}%
\special{pa 3038 2357}%
\special{fp}%
\special{pa 3007 2377}%
\special{pa 3000 2381}%
\special{fp}%
\special{pa 2968 2400}%
\special{pa 2961 2404}%
\special{fp}%
\special{pa 2929 2422}%
\special{pa 2922 2425}%
\special{fp}%
\special{pa 2889 2441}%
\special{pa 2881 2445}%
\special{fp}%
\special{pa 2847 2459}%
\special{pa 2840 2462}%
\special{fp}%
\special{pa 2806 2476}%
\special{pa 2798 2479}%
\special{fp}%
\special{pa 2763 2491}%
\special{pa 2756 2494}%
\special{fp}%
\special{pa 2721 2504}%
\special{pa 2713 2507}%
\special{fp}%
\special{pa 2677 2516}%
\special{pa 2670 2518}%
\special{fp}%
\special{pa 2633 2526}%
\special{pa 2626 2528}%
\special{fp}%
\special{pa 2589 2534}%
\special{pa 2581 2536}%
\special{fp}%
\special{pa 2545 2541}%
\special{pa 2537 2542}%
\special{fp}%
\special{pa 2500 2546}%
\special{pa 2492 2546}%
\special{fp}%
\special{pa 2456 2549}%
\special{pa 2448 2549}%
\special{fp}%
\special{pa 2411 2550}%
\special{pa 2403 2550}%
\special{fp}%
\special{pa 2366 2550}%
\special{pa 2358 2549}%
\special{fp}%
\special{pa 2321 2547}%
\special{pa 2313 2547}%
\special{fp}%
\special{pa 2276 2543}%
\special{pa 2268 2542}%
\special{fp}%
\special{pa 2232 2538}%
\special{pa 2224 2536}%
\special{fp}%
\special{pa 2188 2530}%
\special{pa 2180 2529}%
\special{fp}%
\special{pa 2144 2521}%
\special{pa 2136 2519}%
\special{fp}%
\special{pa 2100 2510}%
\special{pa 2092 2508}%
\special{fp}%
\special{pa 2057 2498}%
\special{pa 2049 2495}%
\special{fp}%
\special{pa 2014 2484}%
\special{pa 2007 2480}%
\special{fp}%
\special{pa 1972 2468}%
\special{pa 1965 2465}%
\special{fp}%
\special{pa 1931 2450}%
\special{pa 1924 2447}%
\special{fp}%
\special{pa 1891 2431}%
\special{pa 1883 2427}%
\special{fp}%
\special{pa 1851 2410}%
\special{pa 1843 2407}%
\special{fp}%
\special{pa 1812 2388}%
\special{pa 1805 2384}%
\special{fp}%
\special{pa 1774 2364}%
\special{pa 1767 2360}%
\special{fp}%
\special{pa 1737 2339}%
\special{pa 1730 2335}%
\special{fp}%
\special{pa 1700 2313}%
\special{pa 1694 2308}%
\special{fp}%
\special{pa 1665 2285}%
\special{pa 1659 2280}%
\special{fp}%
\special{pa 1631 2255}%
\special{pa 1625 2250}%
\special{fp}%
\special{pa 1598 2225}%
\special{pa 1593 2219}%
\special{fp}%
\special{pa 1567 2193}%
\special{pa 1562 2187}%
\special{fp}%
\special{pa 1537 2160}%
\special{pa 1532 2154}%
\special{fp}%
\special{pa 1508 2125}%
\special{pa 1502 2119}%
\special{fp}%
\special{pa 1480 2090}%
\special{pa 1475 2083}%
\special{fp}%
\special{pa 1454 2053}%
\special{pa 1449 2047}%
\special{fp}%
\special{pa 1429 2016}%
\special{pa 1425 2009}%
\special{fp}%
\special{pa 1406 1978}%
\special{pa 1401 1971}%
\special{fp}%
\special{pa 1384 1939}%
\special{pa 1380 1932}%
\special{fp}%
\special{pa 1363 1899}%
\special{pa 1360 1891}%
\special{fp}%
\special{pa 1345 1858}%
\special{pa 1342 1850}%
\special{fp}%
\special{pa 1328 1816}%
\special{pa 1325 1809}%
\special{fp}%
\special{pa 1313 1774}%
\special{pa 1310 1766}%
\special{fp}%
\special{pa 1299 1731}%
\special{pa 1297 1724}%
\special{fp}%
\special{pa 1287 1688}%
\special{pa 1285 1680}%
\special{fp}%
\special{pa 1276 1644}%
\special{pa 1275 1637}%
\special{fp}%
\special{pa 1268 1600}%
\special{pa 1266 1592}%
\special{fp}%
\special{pa 1261 1556}%
\special{pa 1260 1548}%
\special{fp}%
\special{pa 1255 1512}%
\special{pa 1255 1504}%
\special{fp}%
\special{pa 1252 1467}%
\special{pa 1251 1459}%
\special{fp}%
\special{pa 1250 1422}%
\special{pa 1250 1414}%
\special{fp}%
\special{pa 1250 1377}%
\special{pa 1250 1369}%
\special{fp}%
\special{pa 1252 1332}%
\special{pa 1253 1324}%
\special{fp}%
\special{pa 1256 1287}%
\special{pa 1256 1280}%
\special{fp}%
\special{pa 1261 1243}%
\special{pa 1262 1235}%
\special{fp}%
\special{pa 1268 1199}%
\special{pa 1269 1191}%
\special{fp}%
\special{pa 1277 1155}%
\special{pa 1278 1147}%
\special{fp}%
\special{pa 1287 1111}%
\special{pa 1289 1103}%
\special{fp}%
\special{pa 1299 1068}%
\special{pa 1301 1060}%
\special{fp}%
\special{pa 1313 1025}%
\special{pa 1316 1017}%
\special{fp}%
\special{pa 1328 983}%
\special{pa 1331 976}%
\special{fp}%
\special{pa 1345 941}%
\special{pa 1348 934}%
\special{fp}%
\special{pa 1364 901}%
\special{pa 1368 893}%
\special{fp}%
\special{pa 1384 860}%
\special{pa 1388 854}%
\special{fp}%
\special{pa 1406 821}%
\special{pa 1410 815}%
\special{fp}%
\special{pa 1430 783}%
\special{pa 1434 776}%
\special{fp}%
\special{pa 1455 746}%
\special{pa 1459 739}%
\special{fp}%
\special{pa 1481 710}%
\special{pa 1485 703}%
\special{fp}%
\special{pa 1508 674}%
\special{pa 1513 668}%
\special{fp}%
\special{pa 1537 640}%
\special{pa 1542 634}%
\special{fp}%
\special{pa 1567 607}%
\special{pa 1573 601}%
\special{fp}%
\special{pa 1599 575}%
\special{pa 1605 569}%
\special{fp}%
\special{pa 1632 544}%
\special{pa 1638 539}%
\special{fp}%
\special{pa 1666 515}%
\special{pa 1672 510}%
\special{fp}%
\special{pa 1701 486}%
\special{pa 1707 482}%
\special{fp}%
\special{pa 1737 460}%
\special{pa 1744 456}%
\special{fp}%
\special{pa 1774 435}%
\special{pa 1781 431}%
\special{fp}%
\special{pa 1812 411}%
\special{pa 1819 407}%
\special{fp}%
\special{pa 1851 389}%
\special{pa 1858 385}%
\special{fp}%
\special{pa 1891 369}%
\special{pa 1898 365}%
\special{fp}%
\special{pa 1932 349}%
\special{pa 1939 346}%
\special{fp}%
\special{pa 1973 332}%
\special{pa 1981 329}%
\special{fp}%
\special{pa 2015 316}%
\special{pa 2023 314}%
\special{fp}%
\special{pa 2058 302}%
\special{pa 2065 300}%
\special{fp}%
\special{pa 2101 290}%
\special{pa 2108 288}%
\special{fp}%
\special{pa 2144 279}%
\special{pa 2152 277}%
\special{fp}%
\special{pa 2188 269}%
\special{pa 2196 268}%
\special{fp}%
\special{pa 2233 262}%
\special{pa 2241 261}%
\special{fp}%
\special{pa 2277 257}%
\special{pa 2285 256}%
\special{fp}%
\special{pa 2322 253}%
\special{pa 2330 252}%
\special{fp}%
\special{pa 2367 250}%
\special{pa 2375 250}%
\special{fp}%
\special{pa 2412 250}%
\special{pa 2420 250}%
\special{fp}%
\special{pa 2456 251}%
\special{pa 2464 252}%
\special{fp}%
\special{pa 2501 254}%
\special{pa 2509 255}%
\special{fp}%
\special{pa 2546 259}%
\special{pa 2554 260}%
\special{fp}%
\special{pa 2590 266}%
\special{pa 2598 267}%
\special{fp}%
\special{pa 2634 274}%
\special{pa 2642 276}%
\special{fp}%
\special{pa 2678 284}%
\special{pa 2686 286}%
\special{fp}%
\special{pa 2721 296}%
\special{pa 2729 298}%
\special{fp}%
\special{pa 2764 309}%
\special{pa 2772 312}%
\special{fp}%
\special{pa 2806 324}%
\special{pa 2814 327}%
\special{fp}%
\special{pa 2848 341}%
\special{pa 2855 344}%
\special{fp}%
\special{pa 2889 359}%
\special{pa 2896 362}%
\special{fp}%
\special{pa 2929 379}%
\special{pa 2936 383}%
\special{fp}%
\special{pa 2969 400}%
\special{pa 2975 404}%
\special{fp}%
\special{pa 3007 424}%
\special{pa 3014 428}%
\special{fp}%
\special{pa 3045 448}%
\special{pa 3051 453}%
\special{fp}%
\special{pa 3082 473}%
\special{pa 3088 479}%
\special{fp}%
\special{pa 3117 501}%
\special{pa 3123 506}%
\special{fp}%
\special{pa 3152 530}%
\special{pa 3158 535}%
\special{fp}%
\special{pa 3185 560}%
\special{pa 3191 565}%
\special{fp}%
\special{pa 3217 591}%
\special{pa 3223 596}%
\special{fp}%
\special{pa 3248 623}%
\special{pa 3254 629}%
\special{fp}%
\special{pa 3278 657}%
\special{pa 3283 663}%
\special{fp}%
\special{pa 3306 692}%
\special{pa 3311 698}%
\special{fp}%
\special{pa 3333 728}%
\special{pa 3338 734}%
\special{fp}%
\special{pa 3358 765}%
\special{pa 3363 771}%
\special{fp}%
\special{pa 3383 803}%
\special{pa 3387 809}%
\special{fp}%
\special{pa 3405 841}%
\special{pa 3409 848}%
\special{fp}%
\special{pa 3426 881}%
\special{pa 3430 888}%
\special{fp}%
\special{pa 3446 921}%
\special{pa 3449 929}%
\special{fp}%
\special{pa 3464 963}%
\special{pa 3467 970}%
\special{fp}%
\special{pa 3480 1004}%
\special{pa 3483 1012}%
\special{fp}%
\special{pa 3494 1047}%
\special{pa 3497 1054}%
\special{fp}%
\special{pa 3507 1090}%
\special{pa 3509 1097}%
\special{fp}%
\special{pa 3519 1133}%
\special{pa 3520 1141}%
\special{fp}%
\special{pa 3528 1177}%
\special{pa 3530 1185}%
\special{fp}%
\special{pa 3536 1221}%
\special{pa 3537 1229}%
\special{fp}%
\special{pa 3542 1266}%
\special{pa 3543 1274}%
\special{fp}%
\special{pa 3546 1310}%
\special{pa 3547 1318}%
\special{fp}%
\special{pa 3549 1355}%
\special{pa 3549 1363}%
\special{fp}%
% CIRCLE 2 2 3 0 Black White  
% 4 2400 1400 2400 3280 2400 3280 2400 3280
% 
\special{pn 8}%
\special{pn 8}%
\special{pa 4280 1400}%
\special{pa 4280 1408}%
\special{fp}%
\special{pa 4279 1445}%
\special{pa 4279 1453}%
\special{fp}%
\special{pa 4278 1490}%
\special{pa 4277 1498}%
\special{fp}%
\special{pa 4275 1535}%
\special{pa 4275 1543}%
\special{fp}%
\special{pa 4271 1580}%
\special{pa 4271 1588}%
\special{fp}%
\special{pa 4266 1625}%
\special{pa 4265 1633}%
\special{fp}%
\special{pa 4260 1670}%
\special{pa 4259 1678}%
\special{fp}%
\special{pa 4254 1714}%
\special{pa 4252 1722}%
\special{fp}%
\special{pa 4246 1759}%
\special{pa 4244 1766}%
\special{fp}%
\special{pa 4236 1803}%
\special{pa 4234 1810}%
\special{fp}%
\special{pa 4226 1846}%
\special{pa 4224 1854}%
\special{fp}%
\special{pa 4215 1890}%
\special{pa 4213 1898}%
\special{fp}%
\special{pa 4203 1934}%
\special{pa 4201 1941}%
\special{fp}%
\special{pa 4189 1977}%
\special{pa 4187 1984}%
\special{fp}%
\special{pa 4175 2020}%
\special{pa 4172 2027}%
\special{fp}%
\special{pa 4160 2062}%
\special{pa 4157 2069}%
\special{fp}%
\special{pa 4144 2104}%
\special{pa 4141 2111}%
\special{fp}%
\special{pa 4126 2145}%
\special{pa 4123 2153}%
\special{fp}%
\special{pa 4107 2187}%
\special{pa 4104 2194}%
\special{fp}%
\special{pa 4088 2227}%
\special{pa 4085 2234}%
\special{fp}%
\special{pa 4068 2268}%
\special{pa 4065 2275}%
\special{fp}%
\special{pa 4046 2307}%
\special{pa 4042 2314}%
\special{fp}%
\special{pa 4024 2347}%
\special{pa 4020 2354}%
\special{fp}%
\special{pa 4001 2385}%
\special{pa 3997 2392}%
\special{fp}%
\special{pa 3977 2423}%
\special{pa 3972 2430}%
\special{fp}%
\special{pa 3952 2461}%
\special{pa 3947 2467}%
\special{fp}%
\special{pa 3926 2498}%
\special{pa 3921 2504}%
\special{fp}%
\special{pa 3900 2534}%
\special{pa 3895 2541}%
\special{fp}%
\special{pa 3872 2570}%
\special{pa 3867 2576}%
\special{fp}%
\special{pa 3843 2605}%
\special{pa 3838 2611}%
\special{fp}%
\special{pa 3814 2639}%
\special{pa 3809 2645}%
\special{fp}%
\special{pa 3784 2673}%
\special{pa 3778 2679}%
\special{fp}%
\special{pa 3753 2705}%
\special{pa 3747 2711}%
\special{fp}%
\special{pa 3721 2738}%
\special{pa 3715 2743}%
\special{fp}%
\special{pa 3689 2769}%
\special{pa 3683 2774}%
\special{fp}%
\special{pa 3655 2799}%
\special{pa 3650 2804}%
\special{fp}%
\special{pa 3622 2829}%
\special{pa 3616 2834}%
\special{fp}%
\special{pa 3587 2858}%
\special{pa 3581 2863}%
\special{fp}%
\special{pa 3552 2886}%
\special{pa 3545 2891}%
\special{fp}%
\special{pa 3516 2913}%
\special{pa 3509 2918}%
\special{fp}%
\special{pa 3479 2939}%
\special{pa 3473 2944}%
\special{fp}%
\special{pa 3442 2965}%
\special{pa 3435 2969}%
\special{fp}%
\special{pa 3404 2989}%
\special{pa 3397 2994}%
\special{fp}%
\special{pa 3366 3013}%
\special{pa 3359 3017}%
\special{fp}%
\special{pa 3327 3036}%
\special{pa 3320 3039}%
\special{fp}%
\special{pa 3287 3057}%
\special{pa 3280 3061}%
\special{fp}%
\special{pa 3247 3078}%
\special{pa 3240 3082}%
\special{fp}%
\special{pa 3207 3098}%
\special{pa 3200 3102}%
\special{fp}%
\special{pa 3166 3117}%
\special{pa 3158 3120}%
\special{fp}%
\special{pa 3125 3135}%
\special{pa 3117 3138}%
\special{fp}%
\special{pa 3082 3151}%
\special{pa 3075 3155}%
\special{fp}%
\special{pa 3040 3168}%
\special{pa 3033 3170}%
\special{fp}%
\special{pa 2998 3183}%
\special{pa 2990 3185}%
\special{fp}%
\special{pa 2955 3196}%
\special{pa 2947 3198}%
\special{fp}%
\special{pa 2912 3209}%
\special{pa 2904 3211}%
\special{fp}%
\special{pa 2868 3221}%
\special{pa 2860 3223}%
\special{fp}%
\special{pa 2824 3232}%
\special{pa 2817 3233}%
\special{fp}%
\special{pa 2780 3241}%
\special{pa 2773 3243}%
\special{fp}%
\special{pa 2736 3250}%
\special{pa 2728 3251}%
\special{fp}%
\special{pa 2692 3257}%
\special{pa 2684 3259}%
\special{fp}%
\special{pa 2647 3263}%
\special{pa 2639 3264}%
\special{fp}%
\special{pa 2602 3269}%
\special{pa 2594 3270}%
\special{fp}%
\special{pa 2557 3273}%
\special{pa 2549 3274}%
\special{fp}%
\special{pa 2512 3277}%
\special{pa 2504 3277}%
\special{fp}%
\special{pa 2467 3279}%
\special{pa 2459 3279}%
\special{fp}%
\special{pa 2422 3280}%
\special{pa 2414 3280}%
\special{fp}%
\special{pa 2377 3280}%
\special{pa 2369 3280}%
\special{fp}%
\special{pa 2332 3279}%
\special{pa 2324 3278}%
\special{fp}%
\special{pa 2287 3277}%
\special{pa 2279 3276}%
\special{fp}%
\special{pa 2242 3273}%
\special{pa 2234 3273}%
\special{fp}%
\special{pa 2197 3269}%
\special{pa 2189 3268}%
\special{fp}%
\special{pa 2153 3263}%
\special{pa 2145 3262}%
\special{fp}%
\special{pa 2108 3257}%
\special{pa 2100 3256}%
\special{fp}%
\special{pa 2063 3250}%
\special{pa 2056 3248}%
\special{fp}%
\special{pa 2019 3241}%
\special{pa 2011 3239}%
\special{fp}%
\special{pa 1975 3232}%
\special{pa 1967 3230}%
\special{fp}%
\special{pa 1931 3221}%
\special{pa 1924 3219}%
\special{fp}%
\special{pa 1888 3209}%
\special{pa 1880 3207}%
\special{fp}%
\special{pa 1845 3196}%
\special{pa 1837 3194}%
\special{fp}%
\special{pa 1802 3182}%
\special{pa 1794 3180}%
\special{fp}%
\special{pa 1759 3168}%
\special{pa 1752 3165}%
\special{fp}%
\special{pa 1717 3151}%
\special{pa 1710 3148}%
\special{fp}%
\special{pa 1675 3135}%
\special{pa 1668 3132}%
\special{fp}%
\special{pa 1634 3117}%
\special{pa 1626 3114}%
\special{fp}%
\special{pa 1593 3098}%
\special{pa 1586 3095}%
\special{fp}%
\special{pa 1552 3078}%
\special{pa 1545 3074}%
\special{fp}%
\special{pa 1512 3057}%
\special{pa 1505 3054}%
\special{fp}%
\special{pa 1473 3036}%
\special{pa 1466 3032}%
\special{fp}%
\special{pa 1434 3013}%
\special{pa 1427 3009}%
\special{fp}%
\special{pa 1396 2989}%
\special{pa 1389 2985}%
\special{fp}%
\special{pa 1358 2965}%
\special{pa 1351 2960}%
\special{fp}%
\special{pa 1321 2939}%
\special{pa 1314 2935}%
\special{fp}%
\special{pa 1284 2913}%
\special{pa 1277 2908}%
\special{fp}%
\special{pa 1248 2885}%
\special{pa 1242 2880}%
\special{fp}%
\special{pa 1213 2858}%
\special{pa 1206 2852}%
\special{fp}%
\special{pa 1178 2829}%
\special{pa 1172 2824}%
\special{fp}%
\special{pa 1144 2799}%
\special{pa 1138 2793}%
\special{fp}%
\special{pa 1111 2768}%
\special{pa 1105 2763}%
\special{fp}%
\special{pa 1078 2737}%
\special{pa 1073 2732}%
\special{fp}%
\special{pa 1047 2705}%
\special{pa 1041 2699}%
\special{fp}%
\special{pa 1016 2672}%
\special{pa 1010 2666}%
\special{fp}%
\special{pa 986 2639}%
\special{pa 980 2633}%
\special{fp}%
\special{pa 956 2604}%
\special{pa 951 2598}%
\special{fp}%
\special{pa 928 2570}%
\special{pa 923 2563}%
\special{fp}%
\special{pa 900 2534}%
\special{pa 896 2527}%
\special{fp}%
\special{pa 873 2498}%
\special{pa 869 2491}%
\special{fp}%
\special{pa 848 2460}%
\special{pa 843 2454}%
\special{fp}%
\special{pa 823 2423}%
\special{pa 818 2416}%
\special{fp}%
\special{pa 798 2385}%
\special{pa 794 2378}%
\special{fp}%
\special{pa 775 2346}%
\special{pa 771 2339}%
\special{fp}%
\special{pa 753 2307}%
\special{pa 749 2300}%
\special{fp}%
\special{pa 732 2267}%
\special{pa 728 2260}%
\special{fp}%
\special{pa 712 2227}%
\special{pa 708 2220}%
\special{fp}%
\special{pa 692 2186}%
\special{pa 689 2179}%
\special{fp}%
\special{pa 674 2145}%
\special{pa 670 2138}%
\special{fp}%
\special{pa 656 2104}%
\special{pa 653 2096}%
\special{fp}%
\special{pa 640 2062}%
\special{pa 638 2054}%
\special{fp}%
\special{pa 625 2019}%
\special{pa 622 2012}%
\special{fp}%
\special{pa 611 1976}%
\special{pa 608 1969}%
\special{fp}%
\special{pa 597 1933}%
\special{pa 595 1926}%
\special{fp}%
\special{pa 585 1890}%
\special{pa 583 1882}%
\special{fp}%
\special{pa 574 1846}%
\special{pa 572 1838}%
\special{fp}%
\special{pa 564 1802}%
\special{pa 562 1794}%
\special{fp}%
\special{pa 554 1758}%
\special{pa 553 1750}%
\special{fp}%
\special{pa 546 1714}%
\special{pa 545 1706}%
\special{fp}%
\special{pa 540 1669}%
\special{pa 538 1661}%
\special{fp}%
\special{pa 534 1624}%
\special{pa 533 1617}%
\special{fp}%
\special{pa 529 1580}%
\special{pa 528 1572}%
\special{fp}%
\special{pa 525 1535}%
\special{pa 524 1527}%
\special{fp}%
\special{pa 522 1490}%
\special{pa 522 1482}%
\special{fp}%
\special{pa 521 1445}%
\special{pa 520 1437}%
\special{fp}%
\special{pa 520 1400}%
\special{pa 520 1392}%
\special{fp}%
\special{pa 521 1355}%
\special{pa 521 1347}%
\special{fp}%
\special{pa 522 1309}%
\special{pa 523 1301}%
\special{fp}%
\special{pa 525 1265}%
\special{pa 525 1257}%
\special{fp}%
\special{pa 529 1220}%
\special{pa 529 1212}%
\special{fp}%
\special{pa 534 1175}%
\special{pa 535 1167}%
\special{fp}%
\special{pa 540 1130}%
\special{pa 541 1122}%
\special{fp}%
\special{pa 546 1085}%
\special{pa 548 1078}%
\special{fp}%
\special{pa 554 1041}%
\special{pa 556 1033}%
\special{fp}%
\special{pa 564 997}%
\special{pa 566 989}%
\special{fp}%
\special{pa 574 953}%
\special{pa 576 945}%
\special{fp}%
\special{pa 585 909}%
\special{pa 587 902}%
\special{fp}%
\special{pa 597 866}%
\special{pa 599 858}%
\special{fp}%
\special{pa 611 823}%
\special{pa 613 815}%
\special{fp}%
\special{pa 625 780}%
\special{pa 628 773}%
\special{fp}%
\special{pa 640 738}%
\special{pa 643 730}%
\special{fp}%
\special{pa 656 696}%
\special{pa 659 688}%
\special{fp}%
\special{pa 674 654}%
\special{pa 678 647}%
\special{fp}%
\special{pa 693 613}%
\special{pa 696 606}%
\special{fp}%
\special{pa 712 573}%
\special{pa 716 565}%
\special{fp}%
\special{pa 732 532}%
\special{pa 736 525}%
\special{fp}%
\special{pa 754 493}%
\special{pa 758 486}%
\special{fp}%
\special{pa 776 453}%
\special{pa 780 446}%
\special{fp}%
\special{pa 799 414}%
\special{pa 803 408}%
\special{fp}%
\special{pa 823 376}%
\special{pa 828 370}%
\special{fp}%
\special{pa 848 339}%
\special{pa 853 332}%
\special{fp}%
\special{pa 874 302}%
\special{pa 879 296}%
\special{fp}%
\special{pa 901 266}%
\special{pa 905 259}%
\special{fp}%
\special{pa 928 230}%
\special{pa 933 224}%
\special{fp}%
\special{pa 957 195}%
\special{pa 962 189}%
\special{fp}%
\special{pa 986 161}%
\special{pa 991 155}%
\special{fp}%
\special{pa 1016 127}%
\special{pa 1022 121}%
\special{fp}%
\special{pa 1047 95}%
\special{pa 1053 89}%
\special{fp}%
\special{pa 1079 62}%
\special{pa 1085 57}%
\special{fp}%
\special{pa 1111 31}%
\special{pa 1117 26}%
\special{fp}%
\special{pa 1145 1}%
\special{pa 1151 -5}%
\special{fp}%
\special{pa 1178 -29}%
\special{pa 1184 -34}%
\special{fp}%
\special{pa 1213 -58}%
\special{pa 1219 -63}%
\special{fp}%
\special{pa 1249 -86}%
\special{pa 1255 -91}%
\special{fp}%
\special{pa 1284 -113}%
\special{pa 1291 -118}%
\special{fp}%
\special{pa 1321 -139}%
\special{pa 1327 -144}%
\special{fp}%
\special{pa 1358 -165}%
\special{pa 1365 -169}%
\special{fp}%
\special{pa 1396 -189}%
\special{pa 1403 -194}%
\special{fp}%
\special{pa 1434 -213}%
\special{pa 1441 -217}%
\special{fp}%
\special{pa 1473 -236}%
\special{pa 1480 -240}%
\special{fp}%
\special{pa 1513 -257}%
\special{pa 1520 -261}%
\special{fp}%
\special{pa 1553 -278}%
\special{pa 1560 -282}%
\special{fp}%
\special{pa 1593 -298}%
\special{pa 1600 -301}%
\special{fp}%
\special{pa 1634 -317}%
\special{pa 1642 -320}%
\special{fp}%
\special{pa 1675 -335}%
\special{pa 1683 -338}%
\special{fp}%
\special{pa 1717 -351}%
\special{pa 1725 -355}%
\special{fp}%
\special{pa 1760 -368}%
\special{pa 1767 -370}%
\special{fp}%
\special{pa 1802 -383}%
\special{pa 1810 -385}%
\special{fp}%
\special{pa 1845 -396}%
\special{pa 1853 -398}%
\special{fp}%
\special{pa 1888 -409}%
\special{pa 1896 -412}%
\special{fp}%
\special{pa 1932 -421}%
\special{pa 1939 -423}%
\special{fp}%
\special{pa 1976 -432}%
\special{pa 1983 -433}%
\special{fp}%
\special{pa 2020 -441}%
\special{pa 2027 -443}%
\special{fp}%
\special{pa 2064 -450}%
\special{pa 2072 -451}%
\special{fp}%
\special{pa 2108 -458}%
\special{pa 2116 -459}%
\special{fp}%
\special{pa 2153 -463}%
\special{pa 2161 -464}%
\special{fp}%
\special{pa 2198 -469}%
\special{pa 2206 -470}%
\special{fp}%
\special{pa 2243 -473}%
\special{pa 2251 -474}%
\special{fp}%
\special{pa 2288 -477}%
\special{pa 2296 -477}%
\special{fp}%
\special{pa 2333 -479}%
\special{pa 2341 -479}%
\special{fp}%
\special{pa 2378 -480}%
\special{pa 2386 -480}%
\special{fp}%
\special{pa 2423 -480}%
\special{pa 2431 -480}%
\special{fp}%
\special{pa 2468 -479}%
\special{pa 2476 -478}%
\special{fp}%
\special{pa 2513 -477}%
\special{pa 2521 -476}%
\special{fp}%
\special{pa 2558 -473}%
\special{pa 2566 -473}%
\special{fp}%
\special{pa 2603 -469}%
\special{pa 2611 -468}%
\special{fp}%
\special{pa 2647 -463}%
\special{pa 2655 -462}%
\special{fp}%
\special{pa 2692 -457}%
\special{pa 2700 -456}%
\special{fp}%
\special{pa 2737 -450}%
\special{pa 2744 -448}%
\special{fp}%
\special{pa 2781 -441}%
\special{pa 2789 -439}%
\special{fp}%
\special{pa 2825 -432}%
\special{pa 2833 -430}%
\special{fp}%
\special{pa 2869 -421}%
\special{pa 2877 -419}%
\special{fp}%
\special{pa 2912 -409}%
\special{pa 2920 -407}%
\special{fp}%
\special{pa 2955 -396}%
\special{pa 2963 -394}%
\special{fp}%
\special{pa 2998 -382}%
\special{pa 3006 -380}%
\special{fp}%
\special{pa 3041 -368}%
\special{pa 3048 -365}%
\special{fp}%
\special{pa 3083 -351}%
\special{pa 3090 -348}%
\special{fp}%
\special{pa 3125 -335}%
\special{pa 3132 -332}%
\special{fp}%
\special{pa 3166 -317}%
\special{pa 3174 -314}%
\special{fp}%
\special{pa 3207 -298}%
\special{pa 3214 -295}%
\special{fp}%
\special{pa 3248 -278}%
\special{pa 3255 -274}%
\special{fp}%
\special{pa 3288 -257}%
\special{pa 3295 -253}%
\special{fp}%
\special{pa 3327 -236}%
\special{pa 3334 -232}%
\special{fp}%
\special{pa 3366 -213}%
\special{pa 3373 -209}%
\special{fp}%
\special{pa 3404 -189}%
\special{pa 3411 -185}%
\special{fp}%
\special{pa 3442 -165}%
\special{pa 3449 -160}%
\special{fp}%
\special{pa 3479 -139}%
\special{pa 3486 -135}%
\special{fp}%
\special{pa 3516 -113}%
\special{pa 3522 -108}%
\special{fp}%
\special{pa 3552 -86}%
\special{pa 3558 -81}%
\special{fp}%
\special{pa 3587 -58}%
\special{pa 3594 -53}%
\special{fp}%
\special{pa 3622 -29}%
\special{pa 3628 -24}%
\special{fp}%
\special{pa 3656 1}%
\special{pa 3662 6}%
\special{fp}%
\special{pa 3689 32}%
\special{pa 3695 37}%
\special{fp}%
\special{pa 3721 63}%
\special{pa 3727 68}%
\special{fp}%
\special{pa 3753 95}%
\special{pa 3759 101}%
\special{fp}%
\special{pa 3784 128}%
\special{pa 3789 133}%
\special{fp}%
\special{pa 3814 161}%
\special{pa 3820 167}%
\special{fp}%
\special{pa 3843 195}%
\special{pa 3849 202}%
\special{fp}%
\special{pa 3872 230}%
\special{pa 3877 237}%
\special{fp}%
\special{pa 3900 266}%
\special{pa 3904 272}%
\special{fp}%
\special{pa 3926 302}%
\special{pa 3931 309}%
\special{fp}%
\special{pa 3952 339}%
\special{pa 3957 346}%
\special{fp}%
\special{pa 3977 377}%
\special{pa 3982 383}%
\special{fp}%
\special{pa 4001 415}%
\special{pa 4005 422}%
\special{fp}%
\special{pa 4024 454}%
\special{pa 4028 460}%
\special{fp}%
\special{pa 4046 493}%
\special{pa 4050 500}%
\special{fp}%
\special{pa 4068 533}%
\special{pa 4072 540}%
\special{fp}%
\special{pa 4089 573}%
\special{pa 4092 580}%
\special{fp}%
\special{pa 4107 613}%
\special{pa 4111 621}%
\special{fp}%
\special{pa 4126 654}%
\special{pa 4129 662}%
\special{fp}%
\special{pa 4144 696}%
\special{pa 4147 703}%
\special{fp}%
\special{pa 4160 738}%
\special{pa 4162 746}%
\special{fp}%
\special{pa 4175 781}%
\special{pa 4178 788}%
\special{fp}%
\special{pa 4189 823}%
\special{pa 4191 831}%
\special{fp}%
\special{pa 4203 866}%
\special{pa 4205 874}%
\special{fp}%
\special{pa 4215 910}%
\special{pa 4217 917}%
\special{fp}%
\special{pa 4226 953}%
\special{pa 4228 961}%
\special{fp}%
\special{pa 4236 997}%
\special{pa 4238 1005}%
\special{fp}%
\special{pa 4246 1041}%
\special{pa 4247 1049}%
\special{fp}%
\special{pa 4254 1086}%
\special{pa 4255 1094}%
\special{fp}%
\special{pa 4260 1130}%
\special{pa 4261 1138}%
\special{fp}%
\special{pa 4266 1175}%
\special{pa 4267 1183}%
\special{fp}%
\special{pa 4271 1220}%
\special{pa 4272 1228}%
\special{fp}%
\special{pa 4275 1265}%
\special{pa 4276 1273}%
\special{fp}%
\special{pa 4278 1310}%
\special{pa 4278 1318}%
\special{fp}%
\special{pa 4279 1355}%
\special{pa 4280 1363}%
\special{fp}%
\special{pa 4280 1400}%
\special{pa 4280 1400}%
\special{fp}%
\end{picture}}%
}
  \end{center}
 \end{minipage}

故に,右上図から,
     \begin{align*}
     \frac{S(t)}{\pi}&=M-m \\
     &=2\left(t-\frac{t}{2}\right)^2-2\left(1-\sqrt{1-t}-\frac{t}{2}\right)^2     \\
     &=4(1-t)+(2-t)\sqrt{1-t} \\
     &=2[2(1-t)+\sqrt{1-t}+(1-t)\sqrt{1-t}]
     \end{align*}
である.

さて,$|(t/3,t/3,t/3)|=\sqrt{3}/3$に注意すれば,求める体積$V$は,
     \begin{align*}
     &\frac{V}{\pi}=\int_0^1S(t)\frac{\sqrt{3}}{3}dt \\
     \therefore \  &\frac{\sqrt{3}V}{2\pi}=\frac{1}{2}\int_0^1S(t)dt \\
     &=\int_0^1\{2(t-1)+\sqrt{1-t}+(1-t)\sqrt{1-t}\}dt
     \end{align*}
である.各項計算すれば,
     \begin{align*}
     &\int_0^12(t-1)dt=\left[(t-1)^2\right]_0^1=-1 \\
     &\int_0^1\sqrt{1-t}dt=\left[\frac{-2}{3}(1-t)^3/2\right]_0^1=\frac{2}{3} \\
     &\int_0^1(1-t)\sqrt{1-t}dt=\left[\frac{-2}{5}(1-t)^5/2\right]_0^1=\frac{2}{5}
     \end{align*}
であるから,代入して,
     \begin{align*}
     V=\frac{2\sqrt{3}\pi}{3}\left[-1+\frac{2}{3}+\frac{2}{5}\right]=\frac{2\sqrt{3}\pi}{45}
     \end{align*}
である.$\cdots$((2)の答)
\newpage
\end{multicols}
\end{document}