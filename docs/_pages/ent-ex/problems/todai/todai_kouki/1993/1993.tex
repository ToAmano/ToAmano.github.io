 % LuaLaTeX文書; 文字コーAドはUTF-8
 \documentclass[unicode,12pt, A4j]{ltjsarticle}% 'unicode'が必要
 %\usepackage{luatexja}% 日本語したい
 \usepackage{luatexja-fontspec}
 %\usepackage[hiragino-pron]{luatexja-preset}% IPAexフォントしたい(ipaex)
 \usepackage[hiragino-pron,deluxe,expert,bold]{luatexja-preset}

 \usepackage[english]{babel}%多言語文書を作成する
 \usepackage{amsmath,amssymb}%標準数式表現を拡大する
 \usepackage{physics}
 \usepackage[subpreambles=true,sort=true]{standalone}
% \renewcommand{\kanjifamilydefault}{\gtdefault}% 既定をゴシック体に
 \usepackage[backend=bibtex,style=phys,articletitle=false,biblabel=brackets,chaptertitle=false,pageranges=false]{biblatex}
 %\usepackage[style=authoryear,backend=bibtex]{biblatex}


 %\addbibresource{../references/tio2_ref.bib}
 \usepackage{mhchem}
 % あとは欧文の場合と同じ

  \usepackage{caption}
  \usepackage[subrefformat=parens]{subcaption}
\title{東大数学理科後期1993年度}
\author{}
\date{}

\begin{document}
\maketitle

\section{問題1}
$n$を$3$以上の自然数とする.$xy$平面上,原点を中心とし,点$(1,0)$をひとつの頂点に持つ正$n$角形を$P$とする.
\begin{enumerate}
 \item $P$の像が$P$に完全に重なるような一次変換を表す行列を全て求めよ.
 \item (1)で求めた行列すべての和を求めよ.
\end{enumerate}

\section{問題2}
$xy$平面において,直線$l$と点$\mathrm{A}$の距離を$d(l,\mathrm{A})$と書くことにする.さらに,相異なる$3$点$\mathrm{A}=(x_1,y_1)$,$\mathrm{B}=(x_2,y_2)$,$\mathrm{C}=(x_3,y_3)$が与えられたとき,
\begin{align*}
 f(l)=d(l,\mathrm{A})^2+d(l,\mathrm{B})^2+d(l,\mathrm{C})^2
\end{align*}
とおく.

\begin{enumerate}
 \item ある与えられた直線に平行な直線のうち,$f(l)$を最小にする直線$l_0$は三角形$\mathrm{ABC}$の重心を通ることを示せ.
 \item 異なる$3$本の直線が$f(l)$を最小にするならば,三角形$\mathrm{ABC}$は生産各駅であることを示せ.
\end{enumerate}

\section{問題3}
放物線の一部$y=x^2$,$0\le x\le 2$を$y$軸のまわりに回転してできる回転体型の容器に水を満たし,このなかに,半径$r$の鉛の玉を,それが容器に使えて止まるまでゆっくり沈めた.ただし,鉛直線を$y$軸とする.このとき,次の問いに答えよ.
\begin{enumerate}
 \item もとの水面に高さから球の中心の高さを引いた差$s$を$r$の関数としてあらわせ.
 \item あふれ出る水の体積を最大にする$r$の値を求めよ.
\end{enumerate}


\end{document}