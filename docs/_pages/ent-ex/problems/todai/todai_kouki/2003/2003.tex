 % LuaLaTeX文書; 文字コーAドはUTF-8
 \documentclass[unicode,12pt, A4j]{ltjsarticle}% 'unicode'が必要
 %\usepackage{luatexja}% 日本語したい
 \usepackage{luatexja-fontspec}
 %\usepackage[hiragino-pron]{luatexja-preset}% IPAexフォントしたい(ipaex)
 \usepackage[hiragino-pron,deluxe,expert,bold]{luatexja-preset}

 \usepackage[english]{babel}%多言語文書を作成する
 \usepackage{amsmath,amssymb}%標準数式表現を拡大する
 \usepackage{physics}
 \usepackage[subpreambles=true,sort=true]{standalone}
% \renewcommand{\kanjifamilydefault}{\gtdefault}% 既定をゴシック体に
 \usepackage[backend=bibtex,style=phys,articletitle=false,biblabel=brackets,chaptertitle=false,pageranges=false]{biblatex}
 %\usepackage[style=authoryear,backend=bibtex]{biblatex}


 %\addbibresource{../references/tio2_ref.bib}
 \usepackage{mhchem}
 % あとは欧文の場合と同じ

  \usepackage{caption}
  \usepackage[subrefformat=parens]{subcaption}

% https://mathlandscape.com/latex-style/
\everymath{\displaystyle}
\title{東大数学理科後期2003年度}
\author{}
\date{}

\begin{document}
\maketitle

\section{問題1}
 \begin{enumerate}
  \item $x\ge 0$のとき,次の不等式を示せ.
        \begin{align*}
 	x-\frac{x^3}{3!}\le \sin x \le  x-\frac{x^3}{3!}+\frac{x^5}{5!}
        \end{align*}
  \item 曲線$y=\sin x\, (0\le x\le \pi)$と$x$軸で囲まれた図形を$x$軸のまわりに一回転してできる立体を考える.この立体を$x$軸に垂直な$2n-1$個の平面によって体積が等しい$2n$個の部分に分割する.ただし$n$は2以上の自然数である.
        \begin{enumerate}
 	\item これら$2n-1$個の平面と$x$軸との交点の$x$座標のうち,$\frac{\pi}{2}$より小さくかつ$\frac{\pi}{2}$に最も近いものを$a_n$とする.このとき,$\lim_{n\to\infty} n(\frac{\pi}{2}-a_n)$を求めよ.
 	\item $2n-1$個の平面と$x$軸との交点の$x$座標のうち最も小さいものを$b_n$とする.数列$\{n^pb_n\}$が$n\to\infty$のとき$0$でない有限な値に収束するような実数$p$の値を求めよ.また,$p$をそのようにとったとき,$\lim_{n\to\infty}n^pb_n$を求めよ.
        \end{enumerate}
 \end{enumerate}


\section{問題2}
$p$,$q$,$N$,$M$を自然数とする.ただし$\sqrt{p}$は自然数ではないとする.このとき次の問に答えよ.
\begin{enumerate}
 \item 自然数$I$に対してある整数$A$,$B$があって$(\sqrt{p}-[\sqrt{p}])^l=A\sqrt{p}+B$と表せることを示せ.ただし$[\sqrt{p}] $は$\sqrt{p}$より小さい整数のうちで最大のものを表す.
 \item $xy$平面において,$x$座標および$y$座標がともに整数であるような点を格子点という.このとき,直線$y=\sqrt{p}x$との距離が$\frac{1}{N}$以下で$x$座標が$N$以上であるような格子点がそんざいすることを示せ.
 \item 双曲線$y^2-px^2=q$の上の点$mathrm{P}$と格子点$\mathrm{Q}$で,線分$\mathrm{PQ}$の長さが$\frac{1}{M}$以下であるようなものが存在することを示せ.
 \item $p=5$,$q=2$,$M=100$として(3)の条件を満たすような格子点$\mathrm{Q}$を一つ求めよ.すなわち,格子点$\mathrm{Q}$であって,双曲線$y^2-5x^2=2$の上の点$\mathrm{P}$を適当にとれば$\mathrm{PQ}$の長さを$\frac{1}{100}$いかにすることができるようなものを一つ求めよ.ただし$2.23606<\sqrt{5}<2.23607$を用いてよい.
\end{enumerate}


\section{問題3}
 \begin{enumerate}
  \item 全ての$n$について$a_n\ge 2$であるような数列$\{a_n\}$が与えられたとして数列$\{x_n\}$に関する漸化式
 \begin{align*}
  (A)\,\, x_{n+2}-a_{n+1}x_{n+1}+x_n=0 \, (n=0,1,2,\cdots)
 \end{align*}
 を考える.このとき,自然数$m$を一つ決めて固定すれば,漸化式(A)を満たし,$x_0$,$x_m=1$であるような数列$\{x_n\}$がただ一つ存在することを示せ.また,この数列について$0<x_n<1\, (n=1,2,\cdots,m-1)$が成り立つことを示せ.ただし$m$は$3$以上とする.
  \item 数列$\{a_n\}$と正の定数$b$が与えられ,すべての$n$について$a_n\ge 1+b$が成り立つと家庭して,数列$\{y_n\}$に関する漸化式
 \begin{align*}
  (B)\,\, y_{n+2}-a_{n+1}y_{n+1}+by_n=0 \, (n=0,1,2,\cdots)
 \end{align*}
 を考える.このとき,自然数$m$を一つ決めて固定すれば,漸化式(B)を満たし,$y_0=0$,$y_m=1$であるような数列$\{y_n\}$がただ一つ存在して$0<y_n<1\, (n=1,2,\cdots,m-1)$が成り立つことを示せ.ただし$m$は$3$以上とする.
  \item $c$を$2$より大きな定数として,全ての$n$について$a_n\ge e$が成り立つと仮定する.このとき,$c$から決まる$m$によらない正の定数$r$で$e<1$を満たすものが存在し,(1)で得られた数列$\{x_n\}$は$x_n<r^{m-n}\, (n=1,2,\cdots,m-1)$を満たすことを示せ.
 \end{enumerate}


\end{document}