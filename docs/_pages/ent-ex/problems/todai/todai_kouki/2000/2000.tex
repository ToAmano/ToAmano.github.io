 % LuaLaTeX文書; 文字コーAドはUTF-8
 \documentclass[unicode,12pt, A4j]{ltjsarticle}% 'unicode'が必要
 %\usepackage{luatexja}% 日本語したい
 \usepackage{luatexja-fontspec}
 %\usepackage[hiragino-pron]{luatexja-preset}% IPAexフォントしたい(ipaex)
 \usepackage[hiragino-pron,deluxe,expert,bold]{luatexja-preset}
\usepackage{tikz}
\usetikzlibrary{arrows.meta}
\usetikzlibrary{calc}
\usepackage{ifthen}
 \usepackage[english]{babel}%多言語文書を作成する
 \usepackage{amsmath,amssymb}%標準数式表現を拡大する
 \usepackage{physics}
 \usepackage[subpreambles=true,sort=true]{standalone}
% \renewcommand{\kanjifamilydefault}{\gtdefault}% 既定をゴシック体に

 \usepackage{mhchem}
 % あとは欧文の場合と同じ




  \usepackage{caption}
  \usepackage[subrefformat=parens]{subcaption}
\title{東大数学理科後期2000年度}
\author{}
\date{}

\begin{document}
\maketitle

\section{問題1}
$k$ を正整数とし, $x$ を変数とする $k$ 次多項式 $P_k(x)$ について次の条件
\[
(C) \quad
\begin{cases}
P_k(x) - P_k(x-1) = x^{k-1} \\
P_k(0) = 0
\end{cases}
\]
を考える. ただし, $x^0 = 1$ と定める. このとき, 次の問に答えよ.
\begin{enumerate}
    \item $k = 1, 2$ に対し, $P_k(x)$ を求めよ.
    \item すべての $k \ge 3$ に対し, 条件 $(C)$ を満たす $P_k(x)$ が存在し, しかもただ一つであることを示せ.
    \item 正整数 $k$ に対し, $k$ 次の多項式 $Q_k(x)$ を次の条件が成立するように定める.
    \[
    \begin{cases}
    Q_k(0) = Q_k(1) = \cdots = Q_k(k-1) = 0 \\
    Q_k(k) = 1
    \end{cases}
    \]
    このとき, $k$ 個の整数 $c_1, c_2, \dots, c_k$ がそれぞれただ一つ存在して,
    \[
    P_k(x) = \sum_{j=1}^k c_j Q_j(x)
    \]
    と表されることを示せ.
\end{enumerate}


\section{問題2}
正整数 $l$ を与える. 各正整数 $n$ に対して, 関数
\[
y = x^l \sin nx, \quad 0 \le x \le 2\pi
\]
のグラフと $x$ 軸で囲まれる図形を $C_n$ とする.
\begin{enumerate}
    \item $C_n$ を $x$ 軸のまわりに回転させてできる回転体の体積を $V_n$ とするとき, 極限値
    \[
    \lim_{n \to \infty} V_n
    \]
    を求めよ.
    \item $C_n$ を $y$ 軸のまわりに回転させてできる回転体の体積を $W_n$ とするとき, 極限値
    \[
    \lim_{n \to \infty} W_n
    \]
    を求めよ.
\end{enumerate}

\section{問題3}
背番号 1 から 5 までを順に付けた 5 人が, 何も置かれていないテーブルに向かっている. 最初 5 人は各自 3 枚のコインを持っている. それを背番号順に必ず 1 枚または 2 枚テーブルの上に置いてゆく. ただし, 手もとに 2 枚以上のコインがあるときに 1 枚だけコインを置く確率を $p$ とし, $p$ は人によらず一定とする.

背番号 5 の人が置き終わったところ (一巡目が終わったところ) で, 再び背番号 1 の人から順に手もとに残ったコインをテーブルに置いてゆく.

\begin{enumerate}
    \item 一巡目が終わったとき, テーブルの上に 7 枚のコインが置かれている確率 $Q$ を求めよ. また, その $Q$ を最大にする $p$ の値と, そのときの $Q$ の値を求めよ.
    \item 一巡目を終えるとき, 背番号 5 の人が, テーブルの上に 7 枚目のコインを置く確率 $R$ を求めよ. また, その $R$ を最大にする $p$ の値を求めよ.
    \item 二巡目が終わったときのテーブルの上のコインの数の期待値を求めよ.
\end{enumerate}

\end{document}