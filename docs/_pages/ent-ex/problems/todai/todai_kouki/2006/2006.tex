 % LuaLaTeX文書; 文字コーAドはUTF-8
 \documentclass[unicode,12pt, A4j]{ltjsarticle}% 'unicode'が必要
 %\usepackage{luatexja}% 日本語したい
 \usepackage{luatexja-fontspec}
 %\usepackage[hiragino-pron]{luatexja-preset}% IPAexフォントしたい(ipaex)
 \usepackage[hiragino-pron,deluxe,expert,bold]{luatexja-preset}

 \usepackage[english]{babel}%多言語文書を作成する
 \usepackage{amsmath,amssymb}%標準数式表現を拡大する
 \usepackage{physics}
 \usepackage[subpreambles=true,sort=true]{standalone}
% \renewcommand{\kanjifamilydefault}{\gtdefault}% 既定をゴシック体に
 \usepackage[backend=bibtex,style=phys,articletitle=false,biblabel=brackets,chaptertitle=false,pageranges=false]{biblatex}
 %\usepackage[style=authoryear,backend=bibtex]{biblatex}


 %\addbibresource{../references/tio2_ref.bib}
 \usepackage{mhchem}
 % あとは欧文の場合と同じ

  \usepackage{caption}
  \usepackage[subrefformat=parens]{subcaption}
\title{東大数学理科後期2006年度}
\author{}
\date{}

\begin{document}
\maketitle

\section{問題1}
$xy$平面上で$t$を変数とする媒介変数表示
\begin{align}
 x&=2t+t^2 \\
 y&=t+t^2  
\end{align}
で表される曲線を$C$とする.次の問に答えよ.
\begin{enumerate}
 \item $t\neq -1$のとき,$\dv{y}{x}$を$t$の式であらわせ.
 \item 曲線$C$上で$\dv{y}{x}=-\dfrac{1}{2}$を満たす点$\mathrm{A}$の座標を求めよ.
 \item 曲線$C$上の点$(x,y)$を点$(X,Y)$に移す移動が
       \begin{align}
	  X&=\frac{1}{\sqrt{5}}(2x-y) \\
	  Y&=\frac{1}{\sqrt{5}}(x+2y) 
       \end{align}
       で表されているとする.このとき$Y$を$X$を用いてあらわせ.
 \item 曲線$C$の概形を$xy$平面上に描け.
\end{enumerate}


\section{問題2}
$a$を正の実数,$\theta$を$0\le\theta\le\dfrac{\pi}{2}$を満たす実数とする.$xyz$空間において,点$(a,0,0)$と点$(a+\cos\theta,0,\sin\theta)$を結ぶ線分を,$x$軸のまわりに$1$回転させてできる曲面を$S$とする.さらに,$S$を$y$軸のまわりに$1$回転させてできる立体の体積を$V$とする.次の問に答えよ.
\begin{enumerate}
 \item $V$を$a$と$\theta$を用いてあらわせ.
 \item $a=4$とする.$V$を$\theta$の関数と考えて,$V$の最大値を求めよ.
\end{enumerate}


\section{問題3}
数列の和の公式
\begin{equation}
 \sum_{k=1}^{n}k=\frac{1}{2}n(n+1),  \sum_{k=1}^{n}k^2=\frac{1}{6}n(n+1)(2n+1), \sum_{k=1}^{n}k^3=\left\{\frac{1}{2}n(n+1)\right\}^2
\end{equation}
などについて,次のような一般的な考察をしてみよう.$p$,$n$を自然数とする.

\begin{enumerate}
 \item $p+1$次多項式$S_p(x)$があって,数列の和$\sum_{k=1}^{n}k^p$が$S_p(n)$と表されることを示せ.
 \item $q$を自然数とする.(1)の多項式$S_1(x), S_3(x),\cdots,S_{2q-1}(x)$に対して,
       \begin{equation}
	\sum_{j=1}^{q}a_{j}S_{2j-1}(x)=x^q(x+1)^q
       \end{equation}
       が恒等式となるような定数$a_1,\cdots,a_q$を$q$を用いてあらわせ.
 \item $q$を$2$以上の自然数とする.(1)の多項式$S_2(x),S_4(x),\cdots,S_{2q-2}(x)$に対して
       \begin{equation}
	\sum_{j=1}^{q-1}b_{j}S_{2j}(x)=x^{q-1}(x+1)^{q-1}(cx+q)
       \end{equation}
       が恒等式となるような定数$c$と$b_1,\cdots,b_{q-1}$を$q$を用いてあらわせ.
 \item $p$を$3$以上の奇数とする.このとき
       \begin{equation}
	\dv{x}S_{p}(x)=pS_{p-1}(x)
       \end{equation}
       を示せ.
\end{enumerate}


\end{document}