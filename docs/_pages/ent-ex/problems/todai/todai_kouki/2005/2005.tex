 % LuaLaTeX文書; 文字コーAドはUTF-8
 \documentclass[unicode,12pt, A4j]{ltjsarticle}% 'unicode'が必要
 %\usepackage{luatexja}% 日本語したい
 \usepackage{luatexja-fontspec}
 %\usepackage[hiragino-pron]{luatexja-preset}% IPAexフォントしたい(ipaex)
 \usepackage[hiragino-pron,deluxe,expert,bold]{luatexja-preset}

 \usepackage[english]{babel}%多言語文書を作成する
 \usepackage{amsmath,amssymb}%標準数式表現を拡大する
 \usepackage{physics}
 \usepackage[subpreambles=true,sort=true]{standalone}
% \renewcommand{\kanjifamilydefault}{\gtdefault}% 既定をゴシック体に
 \usepackage[backend=bibtex,style=phys,articletitle=false,biblabel=brackets,chaptertitle=false,pageranges=false]{biblatex}
 %\usepackage[style=authoryear,backend=bibtex]{biblatex}


 %\addbibresource{../references/tio2_ref.bib}
 \usepackage{mhchem}
 % あとは欧文の場合と同じ

  \usepackage{caption}
  \usepackage[subrefformat=parens]{subcaption}

% https://mathlandscape.com/latex-style/
\everymath{\displaystyle}
\title{東大数学理科後期2005年度}
\author{}
\date{}

\begin{document}
\maketitle

\section{問題1}
$xy$平面の原点を$\mathrm{O}$として,$2$点$\mathrm{P}(\cos\theta,\sin\theta)$,$\mathrm{Q}(1,0)$をとる.ただし,$0<\theta<\pi$とする.点$A$は線分$\mathrm{PQ}$上を,また点$\mathrm{B}$は線分$\mathrm{OQ}$上を動き,線分$\mathrm{AB}$は$\triangle OPQ$の面積を二等分しているとする.このような線分$\mathrm{AB}$で最も短いものの長さを$l$とおき,これを$\theta$の関数と考えて$l^2=f(\theta)$と表す.
 \begin{enumerate}
  \item 線分$\mathrm{AQ}$の長さを$a$,$\mathrm{BQ}$の長さを$b$とすると,$ab=\sin\frac{\theta}{2}$が成立することを示せ.
  \item $\mathrm{PQ}\ge \frac{1}{2}$,$\mathrm{PQ}<\frac{1}{2}$それぞれの場合について,$f(\theta)$を$\theta$を用いてあらわせ.
  \item 関数$f(\theta)$は$0<\theta<\pi$で微分可能であることを示し,そのグラフの概形を描け.また,$f(\theta)$の最大値を求めよ.
 \end{enumerate}

\section{問題2}
$10$枚のカードに$1$から$10$までの数が$1$つづつ書かれている.これらのカードを用いた次のようなゲームを考える.$r$を自然数とする.このゲームは最大$r$ラウンドからなり,第一ラウンドから始まる.各ラウンドで,プレーヤーは,$10$枚のカードから$1$枚のカードを抜き出し,その数をみてから,「停止」または「続行」のいずれかを選択する.「停止」を選択した場合は,そのラウンドでゲームは終了し,最後に抜き出したカードに書かれた数が特典となる.「続行」を選択した場合は,抜き出したカードをもとに戻して,次のラウンドを実行する.最終ラウンドでは,「停止」しか選択できず,そのラウンドで抜きただしたカードに書かれた数が得点となる.ただし,各ラウンドで,どのカードも等しい確率$\frac{1}{10}$で抜き出されるものとする.

抜き出したカードに書かれた数$x$によって「停止」または「続行」を選択する規則を,そのラウンドにおける戦略という.戦略はラウンドごとに,$0$または$1$の値をとる関数$f(x)\,(x=1,2,\cdots,10)$によって,$f(x)=0$ならば「続行」,$f(x)=1$ならば「停止」と定める.

\begin{enumerate}
 \item $k$は$1\le k < 10$を満たす自然数とする.関数$f_k(x)$を
 \item ラウンド数$r$が$2$のとき,得点の期待値が最大となるような,第一ラウンドでの戦略を与え,その時の得点の期待値を求めよ.
 \item ラウンド数$r$が$3$のとき,特典の期待値が最大となるような,第一ラウンドおよび第二ラウンドでの戦略をそれぞれ与え,その時の得点の期待値を求めよ.
\end{enumerate}

\section{問題3}
$a$は実数で,$-\frac{1}{2}\le a<2$を満たすとする.$xy$平面の領域$D$,$E$を
\begin{align*}
   D & 1\le x^2+y^2\le 4 \\
   E & a\le x \le a+1
\end{align*}
で定める.領域$D$と$E$の共通部分の面積を$a$の関数と考えて$S(a)$とおく.

\begin{enumerate}
 \item $S(a)$を定積分であらわせ.
 \item 導関数$S'(a)$を$a$の関数として求めよ.
 \item $S(a)$を最大にするような実数$a$を解にもつ$4$次方程式$3x^4+px^3+qx^2+rx+s=0$($p$,$q$,$r$,$s$は整数)を求めよ.
 \item (3)で求めた方程式で,$x=\sqrt{2}t$とおき,さらに$z=t-\frac{1}{t}$とすることにより,この方程式を$z$についての$2$次方程式としてあらわせ.
 \item $S(a)$を最大にするような$a$の値を求めよ.
\end{enumerate}



\end{document}