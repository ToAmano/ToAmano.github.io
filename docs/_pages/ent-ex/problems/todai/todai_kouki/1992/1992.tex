 % LuaLaTeX文書; 文字コーAドはUTF-8
 \documentclass[unicode,12pt, A4j]{ltjsarticle}% 'unicode'が必要
 %\usepackage{luatexja}% 日本語したい
 \usepackage{luatexja-fontspec}
 %\usepackage[hiragino-pron]{luatexja-preset}% IPAexフォントしたい(ipaex)
 \usepackage[hiragino-pron,deluxe,expert,bold]{luatexja-preset}

 \usepackage[english]{babel}%多言語文書を作成する
 \usepackage{amsmath,amssymb}%標準数式表現を拡大する
 \usepackage{physics}
 \usepackage[subpreambles=true,sort=true]{standalone}
% \renewcommand{\kanjifamilydefault}{\gtdefault}% 既定をゴシック体に
 \usepackage[backend=bibtex,style=phys,articletitle=false,biblabel=brackets,chaptertitle=false,pageranges=false]{biblatex}
 %\usepackage[style=authoryear,backend=bibtex]{biblatex}


 %\addbibresource{../references/tio2_ref.bib}
 \usepackage{mhchem}
 % あとは欧文の場合と同じ

  \usepackage{caption}
  \usepackage[subrefformat=parens]{subcaption}
\title{東大数学理科後期1992年度}
\author{}
\date{}

\begin{document}
\maketitle

\section{問題1}
定数$a$にたいして,曲線$y=\sqrt{x^2-1}+\frac{a}{x}$の$x\ge 1$の部分を$C(a)$とおく.

\begin{enumerate}
 \item $C(a)$が直線$y=x$の下部$y<x$に含まれるような実数$a$の最大値$a_0$を求めよ.
 \item $0<\theta<\frac{\pi}{2}$のとき,$C(a_0)$と$3$直線$y=x$,$x=1$,$x=\frac{1}{\cos\theta}$によって囲まれる図形を$x$軸のまわりに回転させてできる立体$V$の体積$V(\theta)$をもとめよ.
 \item $\lim_{\theta\to\frac{\pi}{2}-0}V(\theta)$をもとめよ.
\end{enumerate}

\section{問題2}
\begin{enumerate}
 \item 空間内の直線$L$を共通の境界線とし,角$\theta$で交わる$2$つの半平面$H_1$,$H_2$がある.$H_1$上に点$A$,$L$上に点$B$,$H_2$上に点$\mathrm{C}$がそれぞれ固定されている.ただし,$\mathrm{A}$,$\mathrm{C}$は$L$上にはないものとする.半平面$H_1$を,$L$を軸として,$0\le\theta\le\pi$の範囲で回転させる.このとき,$\theta$が増加すると$\angle \mathrm{ABC}$も増加することを証明せよ.
 \item 空間内の相異なる$4$点$\mathrm{A}$,$\mathrm{B}$,$\mathrm{C}$,$\mathrm{D}$について,不等式
       \begin{equation}
	  \angle \mathrm{ABC}+\angle \mathrm{BCD}+\angle \mathrm{CDA}+\angle \mathrm{DAB} \le 2\pi
       \end{equation}
       が成り立つことを証明せよ.
       ただし,角の単位はラジアンを用いる.
\end{enumerate}


\section{問題3}
多項式の列$P_0(x)=0$,$P_1(x)=1$,$P_2(x)=1+x$,$\cdots$,$P_n(x)=\sum_{k=0}^{n-1}x^k$,$\cdots$をかんがえる.
\begin{enumerate}
 \item 正の整数$n$,$m$に対して,$P_n(x)$を$P_m(x)$で割ったあまりは$P_0(x)$,$P_1(x)$,$\cdots$,$P_{m-1}(x)$のいずれかであることを証明せよ.
 \item 等式$P_1(x)P_m(x^2)P_n(x^4)=P_{100}(x)$が成立するような正の整数の組$(l,m,n)$をすべて求めよ.
\end{enumerate}


\end{document}