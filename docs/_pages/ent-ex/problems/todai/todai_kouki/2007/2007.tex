 % LuaLaTeX文書; 文字コーAドはUTF-8
 \documentclass[unicode,12pt, A4j]{ltjsarticle}% 'unicode'が必要
 %\usepackage{luatexja}% 日本語したい
 \usepackage{luatexja-fontspec}
 %\usepackage[hiragino-pron]{luatexja-preset}% IPAexフォントしたい(ipaex)
 \usepackage[hiragino-pron,deluxe,expert,bold]{luatexja-preset}

 \usepackage[english]{babel}%多言語文書を作成する
 \usepackage{amsmath,amssymb}%標準数式表現を拡大する
 \usepackage{physics}
 \usepackage[subpreambles=true,sort=true]{standalone}
% \renewcommand{\kanjifamilydefault}{\gtdefault}% 既定をゴシック体に
 \usepackage[backend=bibtex,style=phys,articletitle=false,biblabel=brackets,chaptertitle=false,pageranges=false]{biblatex}
 %\usepackage[style=authoryear,backend=bibtex]{biblatex}


 %\addbibresource{../references/tio2_ref.bib}
 \usepackage{mhchem}
 % あとは欧文の場合と同じ

  \usepackage{caption}
  \usepackage[subrefformat=parens]{subcaption}

% https://mathlandscape.com/latex-style/
\everymath{\displaystyle}
\title{東大数学理科後期2007年度}
\author{}
\date{}

\begin{document}
\maketitle

\section{問題1}
$xy$平面の曲線$C:xy^2=4$上に$1$点$\mathrm{P}_0(x_0,y_0)\, (y_0>0)$をとる.$\mathrm{P}_0$における$C$の接線と$C$との共有点のうち,$\mathrm{P}_0$と異なるものを$\mathrm{P}_1(x_1,y_1)$とする.また,$\mathrm{P}_1$における$C$の接線と$C$との共有点のうち,$P_1$と異なるものを$\mathrm{P}_2(x_2,y_2)$とする.次の問に答えよ.
\begin{enumerate}
 \item $\mathrm{P}_1$,$\mathrm{P}_2$の座標を$y_0$を用いてあらわせ.
 \item $\triangle \mathrm{P_0P_1P_2}$の面積を$T$とし,線分$\mathrm{P_0P_1}$,$\mathrm{P_1P_2}$および曲線$C$で囲まれた領域の面積を$S$とする.$\dfrac{T}{S}$の値を求めよ.
 \item $\angle \mathrm{P_0P_1P_2}$が直角となるような$y_0$の値を求めよ.
 \item 全問(3)で求めた$y_0$に対し,$\triangle \mathrm{P_0P_1P_2}$の外接円の面積を求めよ.
\end{enumerate}


\section{問題2}
次の問に答えよ.
\begin{enumerate}
 \item 実数を成分とする行列$A=\begin{pmatrix}
			       a & b \\ c & d 
			      \end{pmatrix}\, (a^2+b^2\neq 0)$に対し
       \begin{equation}
	B=\begin{pmatrix}
	   a & b \\
	   -b & a 
	  \end{pmatrix}
	\begin{pmatrix}
	   a & b \\
	   b & d 
	  \end{pmatrix}
	\begin{pmatrix}
	   a & b \\
	   -b & a 
	  \end{pmatrix}^{-1}
       \end{equation}
       とおく.行列$B$は$B=\begin{pmatrix}
                                r & s \\ s & t
                               \end{pmatrix}$の形であることを示し,$r+t$,$rt-s^2$を$a$,$b$,$c$を用いてあらわせ.
 \item 前問(1)において$r^2+s^2\ge a^2+b^2$が成り立つことを示せ.
 \item 実数$a_n$,$b_n$,$c_n\, (n=0,1,2\cdots)$を次のように定める.
       \begin{align}
	\text{$n=0$のとき} & 
	\begin{pmatrix}
            a_0 & b_0 \\
            b_0 & c_0
	\end{pmatrix} =
	\begin{pmatrix}
             1 & 1 \\
             1 & 2
	\end{pmatrix}, \\
	\text{$n\ge 1$のとき} & 
	\begin{pmatrix}
            a_n & b_n \\
            b_n & c_n
	\end{pmatrix} =
	\begin{pmatrix}
             a_{n-1} & b_{n-1} \\
             -b_{n-1} & a_{n-1}
	\end{pmatrix}
	\begin{pmatrix}
             a_{n-1} & b_{n-1} \\
             b_{n-1} & c_{n-1}
	\end{pmatrix}
	\begin{pmatrix}
             a_{n-1} & b_{n-1} \\
             -b_{n-1} & a_{n-1}
	\end{pmatrix}^{-1}
       \end{align}
       \begin{itemize}
	\item[ア] $\lim_{n\to\infty}b_n=0$を示せ.
	\item[イ] $\lim_{n\to\infty}a_n$,$\lim_{n\to\infty}c_n=0$を求めよ.
       \end{itemize}
\end{enumerate}


\section{問題3}
$N$を$2$以上の自然数とする.$x_1\le \cdots \le x_N$をみたす実数$x_1,\cdots,x_N$に対し実数$k_n$,$p_n$,$q_n\, (n=0,1,2,\cdots)$を次の手続きで定める.
\begin{itemize}
 \item[A] $k_0=1$,$p_0=x_1$,$q_0=x_N$
 \item[B] $n$が奇数のとき$k_n$は$x_i\le \frac{p_{n-1}+q_{n-1}}{2}$をみたす$x_i$の個数,$p_n=p_{n-1}$,$q_n=q_{n-1}$
 \item[C] $n$が偶数$(n\ge 2)$のとき$k_n=k_{n-1}$,$p_n=\frac{1}{k_n}\sum_{i=1}^{k_n}x_i$,$q_n=\frac{1}{N-k_n}\sum_{i=k_n+1}^{N}x_i$.
\end{itemize}
ただし$k_n=0$または$k_n=N$となったら,その時点で手続きを終了する.$x_1<x_N$であるとき,次の問に答えよ.
\begin{enumerate}
 \item すべての自然数$n$について$1\le k_n\le N-1$かつ$x_1\le p_n<q_n\le x_N$が成り立つことを示せ.
 \item 実数$J_n\,(n=0,1,2,\cdots)$を$J_n=\sum_{i=1}^{k_n}(x_i-p_n)^2+\sum_{i=k_n+1}^{N}(x_i-q_n)^2$と定めると,全ての自然数$n$に対して$J_n\le J_{n-1}$が成り立つことを示せ.
 \item $n$が十分大きいとき,$J_n=J_{n-1}$,$p_n=p_{n-1}$,$q_n=q_{n-1}$,$k_n=k_{n-1}$が成り立つことを示せ.
\end{enumerate}


\end{document}