 % LuaLaTeX文書; 文字コーAドはUTF-8
 \documentclass[unicode,12pt, A4j]{ltjsarticle}% 'unicode'が必要
 %\usepackage{luatexja}% 日本語したい
 \usepackage{luatexja-fontspec}
 %\usepackage[hiragino-pron]{luatexja-preset}% IPAexフォントしたい(ipaex)
 \usepackage[hiragino-pron,deluxe,expert,bold]{luatexja-preset}

 \usepackage[english]{babel}%多言語文書を作成する
 \usepackage{amsmath,amssymb}%標準数式表現を拡大する
 \usepackage{physics}
 \usepackage[subpreambles=true,sort=true]{standalone}
% \renewcommand{\kanjifamilydefault}{\gtdefault}% 既定をゴシック体に
 \usepackage[backend=bibtex,style=phys,articletitle=false,biblabel=brackets,chaptertitle=false,pageranges=false]{biblatex}
 %\usepackage[style=authoryear,backend=bibtex]{biblatex}


 %\addbibresource{../references/tio2_ref.bib}
 \usepackage{mhchem}
 % あとは欧文の場合と同じ

  \usepackage{caption}
  \usepackage[subrefformat=parens]{subcaption}

% https://mathlandscape.com/latex-style/
\everymath{\displaystyle}
\title{東大数学理科後期2004年度}
\author{}
\date{}

\begin{document}
\maketitle

\section{問題1}
$r$は正の実数とし,角$\theta$は$0<\theta<\frac{\pi}{2}$を満たすとする.$xy$平面の原点$\mathrm{O}$を$\mathrm{P}_0$,$(1,0)$を$\mathrm{P}_1$として,点$\mathrm{P}_2\mathrm{P}_3\cdots$を以下の条件(a),(b),(c)が$n=0,1,2,\cdots$に対して満たされるようにとる.
 \begin{enumerate}
  \item $\mathrm{P}_{n+1}\mathrm{P}_{n+2}=r\mathrm{P}_{n}\mathrm{P}_{n+1}$
  \item $\triangle \mathrm{P}_{n}\mathrm{P}_{n+1}\mathrm{P}_{n+2}=\theta$
  \item 点$\mathrm{P}_n\mathrm{P}_{n+2}\mathrm{P}_{n+3}$は同一直線上にある.
 \end{enumerate}
このとき次の問に答えよ.
 \begin{enumerate}
  \item $r$を$\theta$を用いてあらわせ.
  \item 点$\mathrm{P}_n$の座標を$(x_n,y_n)$とする.複素数$z_n=x_n+y_ni$を$\theta$を用いてあらわせ.
  \item 数列$\{x_n\}$,$\{y_n\}$がともに収束するための必要十分条件は$\frac{\pi}{3}<\theta<\frac{\pi}{2}$であることを証明せよ.
 \end{enumerate}
以下$\frac{\pi}{3}<\theta<\frac{\pi}{2}$とする.極限値$\lim_{n\to\infty}x_n$,$\lim_{n\to\infty}y_n$をそれぞれ$\theta$の関数と考えて,$\alpha(\theta)$,$\beta(\theta)$とおく.
\begin{enumerate}
 \item 極限値$\lim_{\theta\to{\frac{\pi}{3}+0}}\alpha(\theta)$,$\lim_{\theta\to{\frac{\pi}{3}+0}}\beta(\theta)$をそれぞれ求めよ.
 \item $\frac{\pi}{3}<\theta<\frac{\pi}{2}$における$\beta(\theta)$の最大値を求めよ.
\end{enumerate}



\section{問題2}
集合$A$,$B$を$A=\{0,1,2,3,4,5,6,7\}$,$B=\{0,1\}$とし,$N$を$3$以上の整数とする.また,各項が$0$または$1$からなる数列を$01$数列と呼ぶことにする.
 $01$数列$a_1,a_2,\cdots.a_N$に対し,$A$から$B$への写像$f$を用いて,新しい$01$数列$b_1,b_2,\cdots,b_N$を,
\begin{align*}
 b_1=f(a_1),b_2=f(2a_1+a_2),b_k=f(4a_{k-2}+2a_{k-1}+a_k)\, (k=3,4,\cdots, N)
\end{align*}
と定め,$b_1$,$b_2$,$\cdots$,$b_N$は$a_1$,$a_2$,$\cdots$,$a_N$から$f$によって得られるという.ただし,$A$から$B$への写像$f$とは,$A$の各要素$x$にたいして$B$の要素$f(x)$をただひとつ対応させる規則をさすものとする.次の問に答えよ.
\begin{enumerate}
 \item $A$から$B$への写像は,全部で何通りあるか.
 \item $f(0)=f(3)=f(4)=f(7)=0$,$f(1)=f(2)=f(5)=f(6)=1$,であるとき,
       \begin{align*}
	b_k=\frac{1}{2}\left\{1+(-1)^k\right\}\, (k=1,2,\cdots,N)
       \end{align*}
       となるような$01$数列$a_1$,$a_2$,$\cdots$,$a_N$を求めよ.
 \item $A$から$B$への写像$f$が,条件
       \begin{align*}	
       (P)\,\, f(2m)\neq f(2m+1)\,\, (m=0,1,2,3)
       \end{align*}
       を満たすとする.このような$f$は何通りあるか.
 \item $A$から$B$への写像$f$が条件(P)をみたすならば,どのような$N$項からなる$01$数列も,ある$01$数列$a_1$,$a_2$,$\cdots$,$a_N$から$f$によって得られることを示せ.
\end{enumerate}

\section{問題3}
$xy$平面に点$(-1,0)$を中心とする半径$1$の円$A$と,点$(0,1)$を中心とする半径$1$の円$B$をとる.円$A$の内部を$D$,円$B$の内部を$E$とする.次の問に答えよ.
\begin{enumerate}
 \item 点$(-1+\cos\theta,\sin\theta)$における円$A$の接線を$l$とする.円$B$の接線$m$が$l$と直交するとき,$l$と$m$の交点$\mathrm{P}$の座標を$\theta$を用いてあらわせ.
 \item 領域$D$にも$E$にも重ならないように$1$辺の長さが$2$の正方形を$xy$平面内で動かすとき,この正方形が通り得ない部分の面積を求めよ.
\end{enumerate}


\end{document}