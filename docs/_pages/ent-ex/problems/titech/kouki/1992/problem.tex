 % LuaLaTeX文書; 文字コーAドはUTF-8
 \documentclass[unicode,12pt, a4paper]{ltjsarticle}% 'unicode'が必要
 %\usepackage{luatexja}% 日本語したい
 \usepackage{luatexja-fontspec}
 %\usepackage[hiragino-pron]{luatexja-preset}% IPAexフォントしたい(ipaex)
 \usepackage[hiragino-pron,deluxe,expert,bold]{luatexja-preset}

 \usepackage[english]{babel}%多言語文書を作成する
 \usepackage{amsmath,amssymb}%標準数式表現を拡大する
 \usepackage{physics}
 \usepackage[subpreambles=true,sort=true]{standalone}
% \renewcommand{\kanjifamilydefault}{\gtdefault}% 既定をゴシック体に
 \usepackage[backend=bibtex,style=phys,articletitle=false,biblabel=brackets,chaptertitle=false,pageranges=false]{biblatex}
 %\usepackage[style=authoryear,backend=bibtex]{biblatex}


 %\addbibresource{../references/tio2_ref.bib}
 \usepackage{mhchem}
 % あとは欧文の場合と同じ

  \usepackage{caption}
  \usepackage[subrefformat=parens]{subcaption}
\title{東工大理系後期1992年度}

\begin{document}
\maketitle
\section{問題1}
$x$ の関数 $\displaystyle F(x) = \int_0^1 \frac{|t-x|}{t+1} dt$ の最小値を求めよ.

\section{問題2}
$0 < a < 1$ とする.
座標平面上で原点 $A_0$ から出発して $x$ 軸の正の方向に $a$ だけ進んだ点を $A_1$ とする.
次に $A_1$ で進行方向を反時計回りに $120^\circ$ 回転し $a^2$ だけ進んだ点を $A_2$ とする.
以後同様に $A_{n-1}$ で反時計回りに $120^\circ$ 回転して $a^n$ だけ進んだ点を $A_n$ とする.
このとき点列 $A_0, A_1, A_2, \dots$ の極限の座標を求めよ.


\end{document}