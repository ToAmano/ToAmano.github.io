 % LuaLaTeX文書; 文字コーAドはUTF-8
 \documentclass[unicode,12pt, a4paper]{ltjsarticle}% 'unicode'が必要
 %\usepackage{luatexja}% 日本語したい
 \usepackage{luatexja-fontspec}
 %\usepackage[hiragino-pron]{luatexja-preset}% IPAexフォントしたい(ipaex)
 \usepackage[hiragino-pron,deluxe,expert,bold]{luatexja-preset}

 \usepackage[english]{babel}%多言語文書を作成する
 \usepackage{amsmath,amssymb}%標準数式表現を拡大する
 \usepackage{physics}
 \usepackage{tikz}
 \usetikzlibrary{math}
 \usepackage{caption}
 \usepackage{subcaption}
 \usepackage[subpreambles=true,sort=true]{standalone}
% \renewcommand{\kanjifamilydefault}{\gtdefault}% 既定をゴシック体に
 \usepackage[backend=bibtex,style=phys,articletitle=false,biblabel=brackets,chaptertitle=false,pageranges=false]{biblatex}
 %\usepackage[style=authoryear,backend=bibtex]{biblatex}


 %\addbibresource{../references/tio2_ref.bib}
 \usepackage{mhchem}
 % あとは欧文の場合と同じ

  \usepackage{caption}
  \usepackage[subrefformat=parens]{subcaption}
\title{東工大理系後期2010年度}

\begin{document}
\maketitle
\section{問題1}
$a, b, t$は実数で, $a \ge 0 > b$とする. 次の漸化式により, 数列$a_n, b_n$ $(n=1, 2, \dots)$を定める.
    \[ a_1=a, \quad b_1=b \]
    \[ a_{n+1} = \left(\frac{t}{2} + \frac{5}{t^2+1}\right) a_n + \left(\frac{t}{2} - \frac{5}{t^2+1}\right) b_n, \quad b_{n+1} = \left(\frac{t}{2} - \frac{5}{t^2+1}\right) a_n + \left(\frac{t}{2} + \frac{5}{t^2+1}\right) b_n \]
    \begin{enumerate}
        \item $a_n$を$a, b, t, n$を用いて表せ.
        \item $n \to \infty$とするとき, $a_n$が収束するための$a, b, t$についての必要十分条件を求めよ.
    \end{enumerate}


\section{問題2}
座標平面上で$y = (\log x)^2 \, (x>0)$の表す曲線を$C$とし, $\alpha > 0$に対し, 点$(\alpha, (\log \alpha)^2)$における$C$の接線を$L(\alpha)$で表す.
    \begin{enumerate}
        \item $C$のグラフの概形を掛け.
        \item $C$と$L(\alpha)$との共有点の個数を$n(\alpha)$とする. $n(\alpha)$を求めよ.
        \item $0 < \alpha < 1$とし, $C$と$L(\alpha)$および$x$軸とで囲まれる領域の面積を$S(\alpha)$とする. $S(\alpha)$を求めよ.
    \end{enumerate}

\end{document}