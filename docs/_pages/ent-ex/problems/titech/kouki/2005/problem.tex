 % LuaLaTeX文書; 文字コーAドはUTF-8
 \documentclass[unicode,12pt, a4paper]{ltjsarticle}% 'unicode'が必要
 %\usepackage{luatexja}% 日本語したい
 \usepackage{luatexja-fontspec}
 %\usepackage[hiragino-pron]{luatexja-preset}% IPAexフォントしたい(ipaex)
 \usepackage[hiragino-pron,deluxe,expert,bold]{luatexja-preset}

 \usepackage[english]{babel}%多言語文書を作成する
 \usepackage{amsmath,amssymb}%標準数式表現を拡大する
 \usepackage{physics}
 \usepackage{tikz}
 \usetikzlibrary{math}
 \usepackage{caption}
 \usepackage{subcaption}
 \usepackage[subpreambles=true,sort=true]{standalone}
% \renewcommand{\kanjifamilydefault}{\gtdefault}% 既定をゴシック体に
 \usepackage[backend=bibtex,style=phys,articletitle=false,biblabel=brackets,chaptertitle=false,pageranges=false]{biblatex}
 %\usepackage[style=authoryear,backend=bibtex]{biblatex}


 %\addbibresource{../references/tio2_ref.bib}
 \usepackage{mhchem}
 % あとは欧文の場合と同じ

  \usepackage{caption}
  \usepackage[subrefformat=parens]{subcaption}
\title{東工大理系後期2005年度}

\begin{document}
\maketitle
\section{問題1}
数列 $\{a_m\}$ (ただし $a_m = m$ とする) に対し $b_n = \sum_{m=1}^{n} a_m$ とおく.
    \begin{enumerate}
        \item $0 < r < 1$ とするとき,$\lim_{n \to \infty} nr^n = 0$ および $\lim_{n \to \infty} n^2r^n = 0$ となることを証明せよ.
        \item $S_m = a_1r + a_2r^2 + \dots + a_mr^m$, $T_n = b_1r + b_2r^2 + \dots + b_nr^n$ とおくとき,
        $\lim_{m \to \infty} S_m$ および $\lim_{n \to \infty} T_n$ を求めよ.
    \end{enumerate}


\section{問題2}
$C$ を半径$1$の円とし,その周上に長さ$\theta$の円弧$PQ$をおく.$C$と$P$で接し$C$の内部にある円を$A$, $C$と$Q$で接し, $A$にも接する円を$B$とする.
    \begin{enumerate}
        \item $A$ と $B$ の面積の和の最小値 $S_\theta$ を $\theta$ で表せ.
        \item $\theta$ が $0$ から $2\pi$ まで動くとき,$S_\theta$ の最大値を求めよ.
    \end{enumerate}
\end{document}