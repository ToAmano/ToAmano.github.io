 % LuaLaTeX文書; 文字コーAドはUTF-8
 \documentclass[unicode,12pt, a4paper]{ltjsarticle}% 'unicode'が必要
 %\usepackage{luatexja}% 日本語したい
 \usepackage{luatexja-fontspec}
 %\usepackage[hiragino-pron]{luatexja-preset}% IPAexフォントしたい(ipaex)
 \usepackage[hiragino-pron,deluxe,expert,bold]{luatexja-preset}

 \usepackage[english]{babel}%多言語文書を作成する
 \usepackage{amsmath,amssymb}%標準数式表現を拡大する
 \usepackage{physics}
 \usepackage[subpreambles=true,sort=true]{standalone}
% \renewcommand{\kanjifamilydefault}{\gtdefault}% 既定をゴシック体に
 \usepackage[backend=bibtex,style=phys,articletitle=false,biblabel=brackets,chaptertitle=false,pageranges=false]{biblatex}
 %\usepackage[style=authoryear,backend=bibtex]{biblatex}


 %\addbibresource{../references/tio2_ref.bib}
 \usepackage{mhchem}
 % あとは欧文の場合と同じ

  \usepackage{caption}
  \usepackage[subrefformat=parens]{subcaption}
\title{東工大理系後期1998年度}

\begin{document}
\maketitle
\section{問題1}
実数 $a, b$ に対し $x_n = \frac{1}{n^b} \left\{ \frac{1}{n^a} + \frac{1}{(n+1)^a} + \dots + \frac{1}{(2n-1)^a} \right\}$, $n=1,2,3,\dots$ とおく.$n \to \infty$ のとき $x_n$ が収束するための $a, b$ の条件およびそのときの極限値を求めよ.


\section{問題2}
$yz$ 平面の直線 $y=z$ を $l_1$, 直線 $y=z+\sqrt{2}$ を $l_2$ とする.$xyz$ 空間において $l_1$ を
軸にして $l_2$ を回転してできる円柱面(内部は含まない)を $C$ とする.さらに $z$ 軸を軸と
して $C$ を回転してできる回転体 $R$ とする.
\begin{enumerate}
    \item $xy$ 平面で $C$ を切った切り口に現れる楕円の方程式を求めよ.
    \item $R$ の $yz$ 平面による断面を図示せよ.
    \item $R$ の $-2 \le z \le 2$ の部分の体積を求めよ.
\end{enumerate}

\end{document}