 % LuaLaTeX文書; 文字コーAドはUTF-8
 \documentclass[unicode,12pt, a4paper]{ltjsarticle}% 'unicode'が必要
 %\usepackage{luatexja}% 日本語したい
 \usepackage{luatexja-fontspec}
 %\usepackage[hiragino-pron]{luatexja-preset}% IPAexフォントしたい(ipaex)
 \usepackage[hiragino-pron,deluxe,expert,bold]{luatexja-preset}

 \usepackage[english]{babel}%多言語文書を作成する
 \usepackage{amsmath,amssymb}%標準数式表現を拡大する
 \usepackage{physics}
 \usepackage{tikz}
 \usetikzlibrary{math}
 \usepackage{caption}
 \usepackage{subcaption}
 \usepackage[subpreambles=true,sort=true]{standalone}
% \renewcommand{\kanjifamilydefault}{\gtdefault}% 既定をゴシック体に
 \usepackage[backend=bibtex,style=phys,articletitle=false,biblabel=brackets,chaptertitle=false,pageranges=false]{biblatex}
 %\usepackage[style=authoryear,backend=bibtex]{biblatex}


 %\addbibresource{../references/tio2_ref.bib}
 \usepackage{mhchem}
 % あとは欧文の場合と同じ

  \usepackage{caption}
  \usepackage[subrefformat=parens]{subcaption}
\title{東工大理系後期2006年度}

\begin{document}
\maketitle
\section{問題1}
$a, b$ を正の数とする.$xy$ 座標平面において,楕円 $ax^2 + by^2 = 1$ の第 4 象限
$(x \ge 0, y \le 0)$ に含まれる部分を $C$,傾き $t \ge 0$ の半直線 $y = tx \ (x \ge 0)$ を $l_t$ とする.
$l_t$ 上の点 $P$ と $C$ 上の点 $P'$ を結ぶ線分 $PP'$ が $y$ 軸に平行になるように動くとき,線分
$PP'$ の長さを最大にする $P$ を $P_t$ で表し,$t \ge 0$ が変化するときに $P_t$ が描く曲線を $C'$ と
する.また,楕円 $ax^2 + by^2 = 1$ と $C'$ との交点を $Q(\alpha, \beta)$ とする.

\begin{enumerate}
    \item 曲線 $C'$ の方程式 $y = f(x)$ を求めよ.
    \item $\alpha$ と $\beta$ を求めよ.
    \item 直線 $y = \beta$,曲線 $C'$ および $y$ 軸が囲む領域を $D$ とする.$D$ を $y$ 軸の回りに $1$ 回
    回転してできる回転体の体積 $V$ を求めよ.
\end{enumerate}


\section{問題2}
自然数 $a, b, c$ が
\begin{align*}
3a = b^3, \quad 5a = c^2
\end{align*}
を満たし,$d^6$ が $a$ を割り切るような自然数 $d$ は $d=1$ に限るとする.

\begin{enumerate}
    \item $a$ は $3$ と $5$ で割り切れることを示せ.
    \item $a$ の素因数は $3$ と $5$ 以外にないことを示せ.
    \item $a$ を求めよ.
\end{enumerate}

\end{document}