 % LuaLaTeX文書; 文字コーAドはUTF-8
 \documentclass[unicode,12pt, a4paper]{ltjsarticle}% 'unicode'が必要
 %\usepackage{luatexja}% 日本語したい
 \usepackage{luatexja-fontspec}
 %\usepackage[hiragino-pron]{luatexja-preset}% IPAexフォントしたい(ipaex)
 \usepackage[hiragino-pron,deluxe,expert,bold]{luatexja-preset}

 \usepackage[english]{babel}%多言語文書を作成する
 \usepackage{amsmath,amssymb}%標準数式表現を拡大する
 \usepackage{physics}
 \usepackage{tikz}
 \usetikzlibrary{math}
 \usepackage{caption}
 \usepackage{subcaption}
 \usepackage[subpreambles=true,sort=true]{standalone}
% \renewcommand{\kanjifamilydefault}{\gtdefault}% 既定をゴシック体に
 \usepackage[backend=bibtex,style=phys,articletitle=false,biblabel=brackets,chaptertitle=false,pageranges=false]{biblatex}
 %\usepackage[style=authoryear,backend=bibtex]{biblatex}


 %\addbibresource{../references/tio2_ref.bib}
 \usepackage{mhchem}
 % あとは欧文の場合と同じ

  \usepackage{caption}
  \usepackage[subrefformat=parens]{subcaption}
\title{東工大理系後期2002年度}

\begin{document}
\maketitle
\section{問題1}
$n$ を自然数とする.

\begin{enumerate}
    \item 
    実数 $x$ に対して,$\displaystyle \sum_{k=0}^{n} (-1)^k x^{2k} - \frac{1}{1+x^2}$ を求めよ.

    \item 
    不等式 $\displaystyle \left| \sum_{k=0}^{n} \frac{(-1)^k}{2k+1} - \int_0^1 \frac{1}{1+x^2} dx \right| \leq \frac{1}{2n+3}$ が成り立つことを示せ.

    \item 
    極限 $\displaystyle \lim_{n \to \infty} \sum_{k=0}^{n} \frac{(-1)^k}{2k+1}$ を求めよ.
\end{enumerate}


\section{問題2}
$xy$ 平面上に原点 $O$ を中心とする半径 $1$ の円 $C$ がある.$C$ を底面,$(0,0,\sqrt{3})$ を頂点とする直円すい $S$ を考える.点 $P(1,0,0)$ および $Q(-2,0,0)$ をとる.さらに,動点 $M(\cos\theta,\sin\theta,0)$ ($0 \leq \theta < 2\pi$) を線分 $MQ$ が $M$ 以外に $C$ と交わらないように動かす.

\begin{enumerate}
    \item 
    $\theta$ のとりうる値の範囲を求めよ.

    \item 
    点 $P$ から動点 $M$ までは直円すい $S$ の側面を通り,$M$ からは直線にそって点 $Q$ へ向かう道を考える.このような $P$ から $Q$ までの全ての道の長さの最小値を求 めよ.
\end{enumerate}
\end{document}