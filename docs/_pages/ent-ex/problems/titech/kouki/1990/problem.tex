 % LuaLaTeX文書; 文字コーAドはUTF-8
 \documentclass[unicode,12pt, a4paper]{ltjsarticle}% 'unicode'が必要
 %\usepackage{luatexja}% 日本語したい
 \usepackage{luatexja-fontspec}
 %\usepackage[hiragino-pron]{luatexja-preset}% IPAexフォントしたい(ipaex)
 \usepackage[hiragino-pron,deluxe,expert,bold]{luatexja-preset}

 \usepackage[english]{babel}%多言語文書を作成する
 \usepackage{amsmath,amssymb}%標準数式表現を拡大する
 \usepackage{physics}
 \usepackage[subpreambles=true,sort=true]{standalone}
% \renewcommand{\kanjifamilydefault}{\gtdefault}% 既定をゴシック体に
 \usepackage[backend=bibtex,style=phys,articletitle=false,biblabel=brackets,chaptertitle=false,pageranges=false]{biblatex}
 %\usepackage[style=authoryear,backend=bibtex]{biblatex}


 %\addbibresource{../references/tio2_ref.bib}
 \usepackage{mhchem}
 % あとは欧文の場合と同じ

  \usepackage{caption}
  \usepackage[subrefformat=parens]{subcaption}
\title{東工大理系後期1990年度}

\begin{document}
\maketitle
\section{問題1}
$(x+1)(x-2)$ の小数第 $1$ 位を四捨五入したものが $1+5x$ と等しくなるような実数 $x$ を求めよ.

\section{問題2}
\quad $n$ を $2$ 以上の整数とする.


\begin{enumerate}
  \item $n-1$ 次多項式 $P_n(x)$ と $n$ 次多項式 $Q_n(x)$ で実数 $\theta$ に対して
\[ \sin(2n\theta) = n \sin(2\theta) P_n(\sin^2\theta), \quad \cos(2n\theta) = Q_n(\sin^2\theta) \]
を満たすものが存在することを帰納法を用いて示せ.
  \item $k=1, 2, \dots, n-1$ に対して $\alpha_k = \left(\sin\frac{k\pi}{2n}\right)^{-2}$ とおくと
\[ P_n(x) = (1-\alpha_1 x)(1-\alpha_2 x)\cdots(1-\alpha_{n-1} x) \]
となることを示せ.
  \item $\sum_{k=1}^{n-1} \alpha_k = \frac{2n^2-2}{3}$ を示せ.
\end{enumerate}


\end{document}