 % LuaLaTeX文書; 文字コーAドはUTF-8
 \documentclass[unicode,12pt, a4paper]{ltjsarticle}% 'unicode'が必要
 %\usepackage{luatexja}% 日本語したい
 \usepackage{luatexja-fontspec}
 %\usepackage[hiragino-pron]{luatexja-preset}% IPAexフォントしたい(ipaex)
 \usepackage[hiragino-pron,deluxe,expert,bold]{luatexja-preset}

 \usepackage[english]{babel}%多言語文書を作成する
 \usepackage{amsmath,amssymb}%標準数式表現を拡大する
 \usepackage{physics}
 \usepackage[subpreambles=true,sort=true]{standalone}
% \renewcommand{\kanjifamilydefault}{\gtdefault}% 既定をゴシック体に
 \usepackage[backend=bibtex,style=phys,articletitle=false,biblabel=brackets,chaptertitle=false,pageranges=false]{biblatex}
 %\usepackage[style=authoryear,backend=bibtex]{biblatex}


 %\addbibresource{../references/tio2_ref.bib}
 \usepackage{mhchem}
 % あとは欧文の場合と同じ

  \usepackage{caption}
  \usepackage[subrefformat=parens]{subcaption}
\title{東工大理系後期1991年度}

\begin{document}
\maketitle
\section{問題1}
10進表示の$n$桁の正の整数で,隣り合う桁の数字が互いに相異なるような数の個数 を$a_n$とするとき,次の問いに答えよ.
    \begin{enumerate}
        \item $a_n$を求めよ.
        \item 上の数のうちで,1の位の数字が0である数の個数を$b_n$とするとき,$ \lim_{n \to \infty} \frac{b_n}{a_n}$を求めよ.
    \end{enumerate}

\section{問題2}
原点Oを中心とする半径2の円Kの内部に,一辺の長さが2で対角線の交点がO となるような正方形ABCDをとる.K上の点Pにおいて,線分POと角$\theta$で交わる2 本の半直線を引く.このとき,PがK上のどのような位置にあっても,これら2本の半直線が正方形ABCDを通るような$\theta$の最大値を求めよ.

\end{document}