 % LuaLaTeX文書; 文字コーAドはUTF-8
 \documentclass[unicode,12pt, a4paper]{ltjsarticle}% 'unicode'が必要
 %\usepackage{luatexja}% 日本語したい
 \usepackage{luatexja-fontspec}
 %\usepackage[hiragino-pron]{luatexja-preset}% IPAexフォントしたい(ipaex)
 \usepackage[hiragino-pron,deluxe,expert,bold]{luatexja-preset}

 \usepackage[english]{babel}%多言語文書を作成する
 \usepackage{amsmath,amssymb}%標準数式表現を拡大する
 \usepackage{physics}
 \usepackage[subpreambles=true,sort=true]{standalone}
% \renewcommand{\kanjifamilydefault}{\gtdefault}% 既定をゴシック体に
 \usepackage[backend=bibtex,style=phys,articletitle=false,biblabel=brackets,chaptertitle=false,pageranges=false]{biblatex}
 %\usepackage[style=authoryear,backend=bibtex]{biblatex}


 %\addbibresource{../references/tio2_ref.bib}
 \usepackage{mhchem}
 % あとは欧文の場合と同じ

  \usepackage{caption}
  \usepackage[subrefformat=parens]{subcaption}
\title{東工大理系後期1994年度}

\begin{document}
\maketitle
\section{問題1}
関数 $f(x)$ に対し
\[ F(x) = \int_0^x f(t)dt \]
とおく.ある定数 $a, b, c$ が存在して
\[ F(x) = x^2 + a|x-b| + cx \]
が常に成立し,さらに $3$ つの条件
\begin{enumerate}
    \item[(i)] $f(x)$ は連続
    \item[(ii)] $F(1) = 0$
    \item[(iii)] $f(0) = 1$
\end{enumerate}
が満たされているとする.このとき $f(x)$ を求めよ.


\section{問題2}
自然数 $n=1,2,3,\cdots\cdots$ に対して,$(2-\sqrt{3})^n$ という形の数を考える.これらの数
はいずれも,それぞれ適当な自然数 $m$ が存在して $\sqrt{m}-\sqrt{m-1}$ という表示をもつこと
を示せ.

\end{document}