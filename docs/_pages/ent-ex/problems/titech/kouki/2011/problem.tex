 % LuaLaTeX文書; 文字コーAドはUTF-8
 \documentclass[unicode,12pt, a4paper]{ltjsarticle}% 'unicode'が必要
 %\usepackage{luatexja}% 日本語したい
 \usepackage{luatexja-fontspec}
 %\usepackage[hiragino-pron]{luatexja-preset}% IPAexフォントしたい(ipaex)
 \usepackage[hiragino-pron,deluxe,expert,bold]{luatexja-preset}

 \usepackage[english]{babel}%多言語文書を作成する
 \usepackage{amsmath,amssymb}%標準数式表現を拡大する
 \usepackage{physics}
 \usepackage{tikz}
 \usetikzlibrary{math}
 \usepackage{caption}
 \usepackage{subcaption}
 \usepackage[subpreambles=true,sort=true]{standalone}
% \renewcommand{\kanjifamilydefault}{\gtdefault}% 既定をゴシック体に
 \usepackage[backend=bibtex,style=phys,articletitle=false,biblabel=brackets,chaptertitle=false,pageranges=false]{biblatex}
 %\usepackage[style=authoryear,backend=bibtex]{biblatex}


 %\addbibresource{../references/tio2_ref.bib}
 \usepackage{mhchem}
 % あとは欧文の場合と同じ

  \usepackage{caption}
  \usepackage[subrefformat=parens]{subcaption}
\title{東工大理系後期2011年度}

\begin{document}
\maketitle
\section{問題1}
正の実数 $t$ に対して,座標空間における 4 点 O$(0,0,0)$,A$(t,0,0)$,B$(0,1,0)$,C$(0,0,1)$ を考える.このとき,次の問に答えよ.
\begin{enumerate}
    \item 四面体 OABC のすべての面に内接する球 $P$ の半径 $r$ を $t$ を用いて表せ.
    \item $t$ が動くとき,球 $P$ の体積を四面体 OABC の体積で割った値の最大値を求めよ.
\end{enumerate}


\section{問題2}
次の式 $x = \tan\theta$,$y = \frac{1}{\cos\theta}$ ($0 \le \theta < \frac{\pi}{2}$) で表される $xy$ 平面上の曲線 $C$ を考え
る.定数 $t>0$ に対し点 P$(t,0)$ を通り $x$ 軸に垂直な直線 $l$ と曲線 $C$ の交点を Q とする.
曲線 $C$,$x$ 軸,$y$ 軸,および直線 $l$ で囲まれた図形の面積を $S_1$ とし,$\triangle$OPQ の面積を $S_2$ とする.
\begin{enumerate}
    \item $S_1$, $S_2$ を $t$ を用いて表せ.
    \item 極限 $\displaystyle\lim_{t \to \infty} \frac{S_1 - S_2}{\log t}$ を求めよ.
\end{enumerate}


\end{document}