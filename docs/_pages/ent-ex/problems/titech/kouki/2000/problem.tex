 % LuaLaTeX文書; 文字コーAドはUTF-8
 \documentclass[unicode,12pt, a4paper]{ltjsarticle}% 'unicode'が必要
 %\usepackage{luatexja}% 日本語したい
 \usepackage{luatexja-fontspec}
 %\usepackage[hiragino-pron]{luatexja-preset}% IPAexフォントしたい(ipaex)
 \usepackage[hiragino-pron,deluxe,expert,bold]{luatexja-preset}

 \usepackage[english]{babel}%多言語文書を作成する
 \usepackage{amsmath,amssymb}%標準数式表現を拡大する
 \usepackage{physics}
 \usepackage{tikz}
 \usetikzlibrary{math}
 \usepackage{caption}
 \usepackage{subcaption}
 \usepackage[subpreambles=true,sort=true]{standalone}
% \renewcommand{\kanjifamilydefault}{\gtdefault}% 既定をゴシック体に
 \usepackage[backend=bibtex,style=phys,articletitle=false,biblabel=brackets,chaptertitle=false,pageranges=false]{biblatex}
 %\usepackage[style=authoryear,backend=bibtex]{biblatex}


 %\addbibresource{../references/tio2_ref.bib}
 \usepackage{mhchem}
 % あとは欧文の場合と同じ

  \usepackage{caption}
  \usepackage[subrefformat=parens]{subcaption}
\title{東工大理系後期2000年度}

\begin{document}
\maketitle
\section{問題1}
実数 $a, b$ にたいし $f(x) = x^3 + x^2 + (a+b-a^2)x + ab$ とおく.

\begin{enumerate}
\item $f(x)$ を因数分解せよ.

\item すべての $x \ge 0$ にたいし $f(x) \ge 0$ が成り立つための $a, b$ の条件を求め,それを満たす点 $(a, b)$ の存在する範囲を図示せよ.
\end{enumerate}



\section{問題2}
\begin{enumerate}
    \item (1) $m \ge 0, n \ge 1$ である整数 $m, n$ にたいし $\displaystyle a_{m,n} = \int_0^\pi \theta^m \cos n\theta d\theta$, $\displaystyle b_{m,n} = \int_0^\pi \theta^m \sin n\theta d\theta$ とおくとき,次の式を示せ.
    \begin{align*}
 a_{m+1,n} = -\frac{m+1}{n}b_{m,n}, \quad b_{m+1,n} = (-1)^{n+1}\frac{\pi^{m+1}}{n} + \frac{m+1}{n}a_{m,n} 
\end{align*}

    \item 半径 $1$ の球球体上の定点を端点とする長さ $\pi$ のひもを考える.このひもが球の外側の空間を動くとき,ひもの通過しうる領域の体積を求めよ.
\end{enumerate}

\end{document}