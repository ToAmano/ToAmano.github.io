 % LuaLaTeX文書; 文字コーAドはUTF-8
 \documentclass[unicode,12pt, a4paper]{ltjsarticle}% 'unicode'が必要
 %\usepackage{luatexja}% 日本語したい
 \usepackage{luatexja-fontspec}
 %\usepackage[hiragino-pron]{luatexja-preset}% IPAexフォントしたい(ipaex)
 \usepackage[hiragino-pron,deluxe,expert,bold]{luatexja-preset}

 \usepackage[english]{babel}%多言語文書を作成する
 \usepackage{amsmath,amssymb}%標準数式表現を拡大する
 \usepackage{physics}
 \usepackage[subpreambles=true,sort=true]{standalone}
% \renewcommand{\kanjifamilydefault}{\gtdefault}% 既定をゴシック体に
 \usepackage[backend=bibtex,style=phys,articletitle=false,biblabel=brackets,chaptertitle=false,pageranges=false]{biblatex}
 %\usepackage[style=authoryear,backend=bibtex]{biblatex}


 %\addbibresource{../references/tio2_ref.bib}
 \usepackage{mhchem}
 % あとは欧文の場合と同じ

  \usepackage{caption}
  \usepackage[subrefformat=parens]{subcaption}
\title{東工大理系後期1997年度}

\begin{document}
\maketitle
\section{問題1}
放物線 $y=x^2$ を $C_1$ とし, $C_1$ 上に両端をもつ長さ $1$ の線分の中点の軌跡を $C_2$ とする. $C_1$, $C_2$ および $2$ 直線 $x=\pm a$ ($a>0$) で囲まれる部分の面積を $S_a$ とするとき, $\lim_{a \to \infty} S_a$ を求めよ.


\section{問題2}
四角形 $ABCD$ と頂点 $O$ からなる四角錐を考える. $5$ 点 $A, B, C, D, O$ の中の $2$ 点は, ある辺の両端にあるとき, 互いに隣接点であるという.

今, $O$ から出発し, その隣接点の中から $1$ 点を等確率で選んでその点を $X_1$ とする. 次に $X_1$ の隣接点の中から $1$ 点を等確率で選びその点を $X_2$ とする. この様にして順次 $X_1, X_2, X_3, \dots, X_n$ を定めるとき, $X_n$ が $O$ に一致する確率を求めよ.

\end{document}