 % LuaLaTeX文書; 文字コーAドはUTF-8
 \documentclass[unicode,12pt, a4paper]{ltjsarticle}% 'unicode'が必要
 %\usepackage{luatexja}% 日本語したい
 \usepackage{luatexja-fontspec}
 %\usepackage[hiragino-pron]{luatexja-preset}% IPAexフォントしたい(ipaex)
 \usepackage[hiragino-pron,deluxe,expert,bold]{luatexja-preset}

 \usepackage[english]{babel}%多言語文書を作成する
 \usepackage{amsmath,amssymb}%標準数式表現を拡大する
 \usepackage{physics}
 \usepackage{tikz}
 \usetikzlibrary{math}
 \usepackage{caption}
 \usepackage{subcaption}
 \usepackage[subpreambles=true,sort=true]{standalone}
% \renewcommand{\kanjifamilydefault}{\gtdefault}% 既定をゴシック体に
 \usepackage[backend=bibtex,style=phys,articletitle=false,biblabel=brackets,chaptertitle=false,pageranges=false]{biblatex}
 %\usepackage[style=authoryear,backend=bibtex]{biblatex}


 %\addbibresource{../references/tio2_ref.bib}
 \usepackage{mhchem}
 % あとは欧文の場合と同じ

  \usepackage{caption}
  \usepackage[subrefformat=parens]{subcaption}
\title{東工大理系後期2003年度}

\begin{document}
\maketitle
\section{問題1}
$xyz$ 空間の $2$ 点 $P, Q$ を, $\triangle OPQ (O$ は原点$)$ の面積が正の一定値 $S$ となるように動かす. $P, Q$ から $xy$ 平面に引いた垂線をそれぞれ $P', Q'$ とし, $\triangle OP'Q'$ の面積を $S_1$ とする. ただし, $O, P', Q'$ が同一線上にあるときは $S_1=0$ とする. 同様に $P, Q$ から $yz$ 平面, $zx$ 平面に垂線を引いて作っ
た三角形の面積を $S_2, S_3$ とする.
    \begin{enumerate}
        \item $S^2 = S_1^2 + S_2^2 + S_3^2$ を証明せよ.
        \item $S_1+S_2+S_3$ の最大値, 最小値を求めよ.
    \end{enumerate}


\section{問題2}
$m$ を $0$ 以上の整数とする. 直線 $2x+3y=m$ 上の点 $(x,y)$ で, $x, y$ がともに $0$ 以上の整数であるものの個数を $N(m)$ とする.
    \begin{enumerate}
        \item $N(m+6)=N(m)+1$ を証明せよ.
        \item $N(m)=1-m+\left[\frac{m}{2}\right]+\left[\frac{2m}{3}\right]$ を証明せよ. ただし, $[a]$ は $a$ 以下の最大の整数を表すものとする.
    \end{enumerate}
\end{document}