 % LuaLaTeX文書; 文字コーAドはUTF-8
 \documentclass[unicode,12pt, a4paper]{ltjsarticle}% 'unicode'が必要
 %\usepackage{luatexja}% 日本語したい
 \usepackage{luatexja-fontspec}
 %\usepackage[hiragino-pron]{luatexja-preset}% IPAexフォントしたい(ipaex)
 \usepackage[hiragino-pron,deluxe,expert,bold]{luatexja-preset}

 \usepackage[english]{babel}%多言語文書を作成する
 \usepackage{amsmath,amssymb}%標準数式表現を拡大する
 \usepackage{physics}
 \usepackage{tikz}
 \usetikzlibrary{math}
 \usepackage{caption}
 \usepackage{subcaption}
 \usepackage[subpreambles=true,sort=true]{standalone}
% \renewcommand{\kanjifamilydefault}{\gtdefault}% 既定をゴシック体に
 \usepackage[backend=bibtex,style=phys,articletitle=false,biblabel=brackets,chaptertitle=false,pageranges=false]{biblatex}
 %\usepackage[style=authoryear,backend=bibtex]{biblatex}


 %\addbibresource{../references/tio2_ref.bib}
 \usepackage{mhchem}
 % あとは欧文の場合と同じ

  \usepackage{caption}
  \usepackage[subrefformat=parens]{subcaption}
\title{東工大理系後期2008年度}

\begin{document}
\maketitle
\section{問題1}
次の問に答えよ.
\begin{enumerate}
 \item 実数 $a_1, a_2, x_1, x_2, y_1, y_2$ が
  \begin{align*} 0 &< a_1 \le a_2 \\ a_1 x_1 &\le a_1 y_1 \\ a_1 x_1 + a_2 x_2 &\le a_1 y_1 + a_2 y_2 \end{align*}
  をみたすとしている.このとき $x_1 + x_2 \le y_1 + y_2$ であることを証明せよ.
 \item $n$ を $2$ 以上の整数とし,$3n$ 個の実数 $a_1, a_2, \dots, a_n, x_1, x_2, \dots, x_n, y_1, y_2, \dots, y_n$ が
  \[ 0 < a_1 \le a_2 \le \dots \le a_n \]
  および $n$ 個の不等式
  \[ \sum_{i=1}^j a_i x_i \le \sum_{i=1}^j a_i y_i \quad (j=1, 2, \dots, n) \]
  をみたしているならば,
  \[ \sum_{i=1}^n x_i \le \sum_{i=1}^n y_i \]
  であることを証明せよ.
\end{enumerate}

\section{問題2}
自然数 $n$ に対して
\begin{align*}
   I_n = \int_0^1 x^2 |\sin n\pi x| dx   
\end{align*}
とおく.極限値 $\displaystyle \lim_{n \to \infty} I_n$ を求めよ.

\end{document}