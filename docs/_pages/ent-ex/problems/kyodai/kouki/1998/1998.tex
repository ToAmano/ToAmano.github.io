 % LuaLaTeX文書; 文字コーAドはUTF-8
 \documentclass[unicode,12pt, A4j]{ltjsarticle}% 'unicode'が必要
 %\usepackage{luatexja}% 日本語したい
 \usepackage{luatexja-fontspec}
 %\usepackage[hiragino-pron]{luatexja-preset}% IPAexフォントしたい(ipaex)
 \usepackage[hiragino-pron,deluxe,expert,bold]{luatexja-preset}

 \usepackage[english]{babel}%多言語文書を作成する
 \usepackage{amsmath,amssymb}%標準数式表現を拡大する
 \usepackage{physics}
 \usepackage[subpreambles=true,sort=true]{standalone}
% \renewcommand{\kanjifamilydefault}{\gtdefault}% 既定をゴシック体に
 \usepackage[backend=bibtex,style=phys,articletitle=false,biblabel=brackets,chaptertitle=false,pageranges=false]{biblatex}
 %\usepackage[style=authoryear,backend=bibtex]{biblatex}


 %\addbibresource{../references/tio2_ref.bib}
 \usepackage{mhchem}
 % あとは欧文の場合と同じ

  \usepackage{caption}
  \usepackage[subrefformat=parens]{subcaption}
\title{京大数学理科後期1998年度}
\author{}
\date{}

\begin{document}
\maketitle

\section{問題1}
$2$次の正方行列$X$と$Y$は$XY=YX$のとき交換可能であるという.$2$次の正方行列$A$と$B$は交換可能ではないが,$A$と$AB$は交換可能であり$A$と$BA$も交換可能であるとする.このとき
\begin{enumerate}
 \item $A=\begin{pmatrix} a&b\\ c&d \end{pmatrix}$とするとき,$ad-bc=0$を示せ.
 \item $O$を零行列とするとき,$A^2=0$であることを示せ.
\end{enumerate}


\section{問題2}
関数$f_n(x)$,$(n=1,2,3,\cdots)$は$f_1(x)=4x^2+1$,
\begin{align*}
 f_n(x)=\int_{0}^{1}\left(3x^2tf_{n-1}'(t)+3f_{n-1}(t)\right)\dd t, (n=2,3,4,\cdots)
\end{align*}
で帰納的に定義されている.この$f_n(x)$を求めよ.

\section{問題3}
$\mathrm{A}_1$,$\mathrm{A}_2$,$\mathrm{A}_3$は$xy$平面上の点で同一直線上にはないとする.$3$つの一次式
\begin{align*}
 f_1(x,y)=a_1x+b_1y+c_1, f_2(x,y)=a_2x+b_2y+c_2, f_3(x,y)=a_3x+b_3y+c_3
\end{align*}
は方程式$f_1(x,y)=0$,$f_2(x,y)=0$,$f_3(x,y)=0$によりそれぞれ直線$\mathrm{A_2A_3}$,$\mathrm{A_3A_1}$,$\mathrm{A_1A_2}$を表すとする.このとき実数$u$,$v$をうまくとると方程式
\begin{align*}
 uf_1(x,y)f_2(x,y)+vf_2(x,y)f_3(x,y)+f_3(x,y)f_1(x,y)=0
\end{align*}
が$3$点$\mathrm{A}_1$,$\mathrm{A}_2$,$\mathrm{A}_3$を通る円を表すようにできることを示せ.

\section{問題4}
$a$は$0<a<\pi$を満たす定数とする.$n=0,1,2,\cdots$に対し,$n\pi < x, (n+1)\pi$の範囲に$\sin(x+a)=x\sin x$を満たす$x$がただ一つ存在するので,この値を$x_n$とする.
\begin{enumerate}
 \item 極限値${\displaystyle \lim_{n\to\infty}(x_n-n\pi)}$を求めよ.
 \item 極限値${\displaystyle \lim_{n\to\infty}n(x_n-n\pi)}$を求めよ.
\end{enumerate}


\section{問題5}
$xy$平面上に$2n$個の点$\mathrm{A}_i(i,1)$,$\mathrm{B}_i(i,1) (i=1,2,\cdots,n)$がある.上下に隣り合う$2$点$\mathrm{A}_i$,$\mathrm{B}_i$を結ぶ線分を「縦辺」$(i=1,2,\cdots,n)$,左右に隣り合う$2$点$\mathrm{A}_i$,$\mathrm{A}_{i+1}$および$\mathrm{B}_i$,$\mathrm{B}_{i+1}$を結ぶ線分を「横辺」$(i=1,2,\cdots,n-1)$と言う.全ての横辺には,各辺独立に,確率$p$で右向きの矢印が,確率$1-p$で$\times$印が描かれている.また全ての縦辺には常に上向きの矢印が描かれている.このとき点$\mathrm{A}_1(1,1)$から出発して,矢印の描かれている辺だけを通り,矢印の方向に進んで,点$\mathrm{B}_{n}(n,2)$に到達する経路が少なくとも$1$本存在する確率を$Q_n$とする.以下の問に答えよ.
\begin{enumerate}
 \item $Q_2$,$Q_3$を求めよ.
 \item $Q_n$を求めよ.
\end{enumerate}


\section{問題6}
自然数$n$にたいし,${\displaystyle I_n=\int_{0}^{\pi/4}\cos^n 2\theta \sin^3\theta \dd\theta}$とする.
\begin{enumerate}
 \item $I_2$の値を求めよ.
 \item $xy$平面上で原点$\mathrm{O}$から点$\mathrm{P}(x,y)$への距離を$r$,$x$軸の正の方向と半直線$\mathrm{OP}$のなす(弧度法による)角を$\theta$とする.方程式$r=\sin 2\theta$,$\left(0\le\theta\le\dfrac{\pi}{2}\right)$で表される曲線を,直線$y=x$の周りに回転して得られる局面が囲む立体の体積を$V$とするとき,$V=3\pi I_3+2\pi I_2$と表されることを示せ.
\end{enumerate}
\end{document}