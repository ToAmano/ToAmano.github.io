 % LuaLaTeX文書; 文字コーAドはUTF-8
 \documentclass[unicode,12pt, A4j]{ltjsarticle}% 'unicode'が必要
 %\usepackage{luatexja}% 日本語したい
 \usepackage{luatexja-fontspec}
 %\usepackage[hiragino-pron]{luatexja-preset}% IPAexフォントしたい(ipaex)
 \usepackage[hiragino-pron,deluxe,expert,bold]{luatexja-preset}

 \usepackage[english]{babel}%多言語文書を作成する
 \usepackage{amsmath,amssymb}%標準数式表現を拡大する
 \usepackage{physics}
 \usepackage[subpreambles=true,sort=true]{standalone}
% \renewcommand{\kanjifamilydefault}{\gtdefault}% 既定をゴシック体に
 \usepackage[backend=bibtex,style=phys,articletitle=false,biblabel=brackets,chaptertitle=false,pageranges=false]{biblatex}
 %\usepackage[style=authoryear,backend=bibtex]{biblatex}


 %\addbibresource{../references/tio2_ref.bib}
 \usepackage{mhchem}
 % あとは欧文の場合と同じ

  \usepackage{caption}
  \usepackage[subrefformat=parens]{subcaption}
\title{京大数学理科後期1999年度}
\author{}
\date{}

\begin{document}
\maketitle

\section{問題1}
座標平面上で原点を通る直線と$y=x|x+2|$のグラフが相異なる$3$点で交わっている.このグラフとこの直線によって囲まれる図形で,この直線より下側にあるものの面積を$S_1$,上側にあるものの面積を$S_2$とする.$S_1:S_2=9:8$になるとき,この直線の傾きを求めよ,

\section{問題2}
$\alpha$,$\beta$,$\gamma$は$\alpha>0$,$\beta>0$,$\gamma>0$,$\alpha+\beta+\gamma=\pi$を満たすものとする.このとき,$\sin\alpha\sin\beta\sin\gamma$の最大値を求めよ.

\section{問題3}
$\alpha$を正の定数として,数列$a_n$,$b_n (n\ge 1)$を次の式で定める.
\begin{align*}
 2a_{n+1}&=\alpha\left(3a_n^2+2a_nb_n-b_n^2-a_n+b_n\right) \\
 2b_{n+1}&=\alpha\left(-a_n^2-2a_nb_n-b_n^2-a_n+b_n\right) \\
 a_1&=b_1=1
\end{align*}
\begin{enumerate}
 \item $a_2$,$b_2$,$a_3$,$b_3$,$a_4$,$b_4$を求めよ.
 \item $\dfrac{a_{2n+1}}{a_{2n}}$を求めよ.
\end{enumerate}

\section{問題4}
$\triangle \mathrm{ABC}$は鋭角三角形とする.このとき,各面全てが$\triangle\mathrm{ABC}$と合同な四面体が存在することを示せ.


\section{問題5}
$a$,$b$を整数,$u$,$v$を有理数とする.$u+v\sqrt{3}$が$x^2+ax+b=0$の解であるならば,$u$と$v$は共に整数であることを示せ.ただし$\sqrt{3}$が無理数であることは使って良い.

\section{問題6}
\begin{enumerate}
 \item $f(x)$は$a\le x \le b$で連続な関数とする.このとき
\begin{align*}
 \frac{1}{b-a}\int_{a}^{b}f(x)\dd x=f(c) & a\le c \le b
\end{align*}
となる$c$が存在することを示せ.
 \item $y=\sin x$の$0\le x\le \dfrac{\pi}{2}$の部分と$y=1$及び$y$軸が囲む図形を,$y$軸の周りに回転して得られる立体を考える.この立体を$y$軸に垂直な$n-1$個の平面によって各部分の体積が等しくなるように$n$個に分割するとき,$y=1$に最も近い平面の$y$座標を$y_n$とする.このとき${\displaystyle \lim_{n\to\infty} n\left(1-y_n\right)}$を求めよ.
\end{enumerate}

\end{document}