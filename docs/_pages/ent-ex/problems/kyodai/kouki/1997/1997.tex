 % LuaLaTeX文書; 文字コーAドはUTF-8
 \documentclass[unicode,12pt, A4j]{ltjsarticle}% 'unicode'が必要
 %\usepackage{luatexja}% 日本語したい
 \usepackage{luatexja-fontspec}
 %\usepackage[hiragino-pron]{luatexja-preset}% IPAexフォントしたい(ipaex)
 \usepackage[hiragino-pron,deluxe,expert,bold]{luatexja-preset}

 \usepackage[english]{babel}%多言語文書を作成する
 \usepackage{amsmath,amssymb}%標準数式表現を拡大する
 \usepackage{physics}
 \usepackage[subpreambles=true,sort=true]{standalone}
% \renewcommand{\kanjifamilydefault}{\gtdefault}% 既定をゴシック体に
 \usepackage[backend=bibtex,style=phys,articletitle=false,biblabel=brackets,chaptertitle=false,pageranges=false]{biblatex}
 %\usepackage[style=authoryear,backend=bibtex]{biblatex}


 %\addbibresource{../references/tio2_ref.bib}
 \usepackage{mhchem}
 % あとは欧文の場合と同じ

  \usepackage{caption}
  \usepackage[subrefformat=parens]{subcaption}
\title{京大数学理科後期1997年度}
\author{}
\date{}

\begin{document}
\maketitle

\section{問題1}
$2$つの放物線$C_1:y=x^2$,$C:y=ax^2(a\neq 1)$を考える.$C$を$x$軸方向に$p$,ついで$y$軸方向に$q$だけ平行移動した放物線を$C_{p,q}$と表す.次の条件(*)を満たすような$p$,$q$が存在するための$a$の範囲を求めよ.
\begin{itemize}
 \item[(*)] $C_{p,q}$は$C_1$と$2$点で交わる.$1$つの交点は$C_{p,q}$の頂点であり,他の交点においては両者の接線は直交する.
\end{itemize}

\section{問題2}
自然数$n$と$n$項数列$a_k (1\le k\le n)$が与えられていて,次の条件(イ),(ロ)を満たしている.
\begin{itemize}
 \item[(イ)] $a_k(1\le k\le n)$は全て正整数で,全て$1$と$2n$の間にある.$1\le a_n\le 2n$
 \item[(ロ)] $s_j=\sum_{k=1}^{j}a_k$とおくとき,$s_j(1\le j\le n)$は全て平方数である.(整数の$2$乗である数を平方数という.)
\end{itemize}
このとき
\begin{enumerate}
 \item $s_n=n^2$であることを示せ.
 \item $a_k(1\le k\le n)$を求めよ.
\end{enumerate}

\section{問題3}
点$\mathrm{O}$を中心とする半径$1$の球面上に$4$点$\mathrm{A}$,$\mathrm{B}$,$\mathrm{C}$,$\mathrm{D}$があって
\begin{align*}
 \overrightarrow{\mathrm{OA}}+ \overrightarrow{\mathrm{OB}}+ \overrightarrow{\mathrm{OC}}+ \overrightarrow{\mathrm{OD}}=\overrightarrow{0}
\end{align*}
が成立しているとする.
\begin{enumerate}
 \item $|\overrightarrow{\mathrm{AB}}|=|\overrightarrow{\mathrm{CD}}|$であることを示せ.
 \item 点$\mathrm{B'}$,$\mathrm{D'}$を$\overrightarrow{\mathrm{OB'}}=-\overrightarrow{\mathrm{OB}}$,$\overrightarrow{\mathrm{OD'}}=-\overrightarrow{\mathrm{OD}}$となるようにとる.このとき$\mathrm{A}$,$\mathrm{B}'$,$\mathrm{C}$,$\mathrm{D}'$が互いに異なるならば,これら$4$点は,この順で,ある長方形の頂点となっていることを示せ.
\end{enumerate}

\section{問題4}
次の連立方程式(*)を考える.
\begin{align*}
 (*) 
\begin{cases}
 y=2x^2-1 \\
 z=2y^2-1 \\
 x=2z^2-1 
\end{cases}
\end{align*}
\begin{enumerate}
 \item $(x,y,z)=(a,b,c)$が(*)の実数解であるとき,$|a|\le 1$,$|b|\le 1$,$|c|\le 1$であることを示せ.
 \item (*)は全部で$8$組の相異なる実数解をもつことを示せ.
\end{enumerate}

\section{問題5}
箱の中に青,赤,黄のカードがそれぞれ$3$枚,$2$枚,$1$枚,合計$6$枚入っている.$1$回の試行で,箱の中からカードを$1$枚取り出し,取り出したカードと同じ色のカードを$1$枚加えて,再び箱の中に戻す.従って,$n$回の試行を完了した時に,$(n+6)$枚のカードが箱の中にある.$n$回目の試行が完了した時箱の中にある青のカードの枚数の期待値$E(n)$を求めよ.

\section{問題6}
媒介変数表示された曲線
\begin{align*}
 C: x=e^{-t}\cos t, y=e^{-t}\sin t & \left(0\le t\le \frac{\pi}{2}\right)
\end{align*}
を考える.
\begin{enumerate}
 \item $C$の長さ$L$を求めよ.
 \item $C$と$x$軸,$y$軸で囲まれた領域の面積$S$を求めよ.
\end{enumerate}
\end{document}