 % LuaLaTeX文書; 文字コーAドはUTF-8
 \documentclass[unicode,12pt, A4j]{ltjsarticle}% 'unicode'が必要
 %\usepackage{luatexja}% 日本語したい
 \usepackage{luatexja-fontspec}
 %\usepackage[hiragino-pron]{luatexja-preset}% IPAexフォントしたい(ipaex)
 \usepackage[hiragino-pron,deluxe,expert,bold]{luatexja-preset}

 \usepackage[english]{babel}%多言語文書を作成する
 \usepackage{amsmath,amssymb}%標準数式表現を拡大する
 \usepackage{physics}
 \usepackage[subpreambles=true,sort=true]{standalone}
% \renewcommand{\kanjifamilydefault}{\gtdefault}% 既定をゴシック体に
 \usepackage[backend=bibtex,style=phys,articletitle=false,biblabel=brackets,chaptertitle=false,pageranges=false]{biblatex}
 %\usepackage[style=authoryear,backend=bibtex]{biblatex}


 %\addbibresource{../references/tio2_ref.bib}
 \usepackage{mhchem}
 % あとは欧文の場合と同じ

  \usepackage{caption}
  \usepackage[subrefformat=parens]{subcaption}
\title{京大数学理科後期1993年度}
\author{}
\date{}

\begin{document}
\maketitle

\section{問題1}
$f(x)=x^3+ax^2+bx+c$が次の3条件を満たしているとする.
\begin{enumerate}
 \item ${\displaystyle \lim_{x\to 1}=\frac{f(x)}{x^3-x}= 1}$
 \item 曲線$y=f(x)$の$x=0$における接線の傾きは負である.
 \item 2点$(0,f(0))$と$(1,f(1))$を通る直線を$l$とする.曲線$y=f(x)$と直線$l$で囲まれる図形のうち,$0\le x\le 1$の部分の面積は$\dfrac{3}{4}$である.
\end{enumerate}
このとき,$a$,$b$,$c$の値を求めよ.


\section{問題2}
実数$a$に対して,$f(x)=x^3-3ax$とおく.
\begin{enumerate}
 \item $t$を実数とする.方程式$f(x)=t$が相異なる$3$個の実数解を保つために$a$と$t$が満たすべき条件をもとめよ.
 \item $g(x)=f(f(x))$とおく.方程式$g(x)=0$が相異なる$9$個の実数解を持つような$a$の範囲を求めよ.
\end{enumerate}

\section{問題3}
原点$\mathrm{O}$を中心とする$1$つの円周上に相異なる$4$点$\mathrm{A}_0$,$\mathrm{B}_0$,$\mathrm{C}_0$,$\mathrm{D}_0$をとる.$\mathrm{A}_0$,$\mathrm{B}_0$,$\mathrm{C}_0$,$\mathrm{D}_0$の位置ベクトルをそれぞれ$\va*{a}$,$\va*{b}$,$\va*{c}$,$\va*{d}$と書く.

\begin{enumerate}
 \item $\triangle \mathrm{B_0C_0D_0}$,$\triangle \mathrm{C_0D_0A_0}$,$\triangle \mathrm{D_0A_0B_0}$,$\triangle \mathrm{A_0B_0C_0}$の重心をそれぞれ$\mathrm{A}_1$,$\mathrm{B}_1$,$\mathrm{C}_1$,$\mathrm{D}_1$とする.このとき,この4点は同一円周上にあることを示し,その円の中心$\mathrm{P}_1$の一ベクトル$\overrightarrow{\mathrm{OP_1}}$を$\va*{a}$,$\va*{b}$,$\va*{c}$,$\va*{d}$であらわせ.
 \item 4点$\mathrm{A}_1$,$\mathrm{B}_1$,$\mathrm{C}_1$,$\mathrm{D}_1$に対し上と同様に$\mathrm{A}_2$,$\mathrm{B}_2$,$\mathrm{C}_2$,$\mathrm{D}_2$を定め,$\mathrm{A}_2$,$\mathrm{B}_2$,$\mathrm{C}_2$,$\mathrm{D}_2$を通る円の中心を$\mathrm{P}_2$とする.以下,同様に$\mathrm{P}_3$,$\mathrm{P}_4$,$\cdots$を定める.$\overrightarrow{\mathrm{P}_n\mathrm{P}_{n+1}}$を$\va*{a}$,$\va*{b}$,$\va*{c}$,$\va*{d}$であらわせ.
 \item ${\displaystyle \lim_{n\to\infty} |\mathrm{P}_n\mathrm{Q}|=0}$を満たす点$\mathrm{Q}$の一ベクトルを$\va*{a}$,$\va*{b}$,$\va*{c}$,$\va*{d}$であらわせ.ただし,$|\mathrm{P}_n\mathrm{Q}|$は線分$\mathrm{P}_n\mathrm{Q}$の長さである.
\end{enumerate}


\section{問題4}
$a$は正の定数とする.不等式$a^x\ge ax$が全ての正の数$x$に対して成り立つという.このとき$a$はどのようなものか.

\section{問題5}
$n\ge 3$とする.$1,2 \cdots,n$のうちから重複を許して$6$個の数字をえらびそれを並べた順列を考える.このような順列のうちで,どの数字もそれ以外の$5$つの数字のどれかに等しくなっているようなものの個数を求めよ.


\section{問題6}
$a_1$,$a_2$,$b_1$,$b_2$,$c_1$,$c_2$を実数とする.不等式$\dfrac{a_1x+b_1}{x+c_1} > \dfrac{a_2x+b_2}{x+c_2}$が$x\neq -c_1$かつ$x\neq -c_2$となるすべての実数$x$に対して成立するための必要十分条件を求めよ.



\end{document}