 % LuaLaTeX文書; 文字コーAドはUTF-8
 \documentclass[unicode,12pt, A4j]{ltjsarticle}% 'unicode'が必要
 %\usepackage{luatexja}% 日本語したい
 \usepackage{luatexja-fontspec}
 %\usepackage[hiragino-pron]{luatexja-preset}% IPAexフォントしたい(ipaex)
 \usepackage[hiragino-pron,deluxe,expert,bold]{luatexja-preset}

 \usepackage[english]{babel}%多言語文書を作成する
 \usepackage{amsmath,amssymb}%標準数式表現を拡大する
 \usepackage{physics}
 \usepackage[subpreambles=true,sort=true]{standalone}
% \renewcommand{\kanjifamilydefault}{\gtdefault}% 既定をゴシック体に
 \usepackage[backend=bibtex,style=phys,articletitle=false,biblabel=brackets,chaptertitle=false,pageranges=false]{biblatex}
 %\usepackage[style=authoryear,backend=bibtex]{biblatex}


 %\addbibresource{../references/tio2_ref.bib}
 \usepackage{mhchem}
 % あとは欧文の場合と同じ

  \usepackage{caption}
  \usepackage[subrefformat=parens]{subcaption}
\title{京大数学理科後期2006年度}
\author{}
\date{}

\begin{document}
\maketitle

\section{問題1}
$1$次式$A(x)$,$B(x)$,$C(x)$に対して$\{A(x)\}^2+\{B(x)\}^2=\{C(x)\}^2$が成り立つとする.このとき$A(x)$と$B(x)$はともに$C(x)$の定数倍であることを示せ.



\section{問題2}
$a$を実数として,行列$A$を$A=\begin{pmatrix}a&1-a \\ 1-a&a \end{pmatrix}$と定める.$\begin{pmatrix}x_0 \\ y_0  \end{pmatrix}=\begin{pmatrix} 1 \\ 0  \end{pmatrix}$とし,数列$\{x_n\}$,$\{y_n\}$を次の式で定める.
\begin{align*}
 \begin{pmatrix}
    x_0 \\ y_0  
 \end{pmatrix}
=A
 \begin{pmatrix} 
    x_{n-1} \\ 
    y_{n-1}  
 \end{pmatrix},
n=1,2,\cdots
\end{align*}
このとき数列$\{x_n\}$が収束するための$a$の必要十分条件を求めよ.


\section{問題3}
さいころを$n$個同時に投げるとき,出た目の和が$n+3$になる確率を求めよ.


\section{問題4}
平面上の点$\mathrm{O}$を中心とし,半径$1$の円周上に相異なる$3$点$\mathrm{A}$,$\mathrm{B}$,$\mathrm{C}$がある.$\triangle{\mathrm{ABC}}$の内接円の半径$r$は$\dfrac{1}{2}$以下であることをしめせ.


\section{問題5}
$H>0$,$R>0$とする.空間内において,原点$\mathrm{O}$と点$\mathrm{P}(R,0,H)$を結ぶ線分を,$z$軸のまわりに回転させてできる容器がある.この容器に水を満たし,原点から水面までの高さが$h$のとき単位時間あたりの排水量が,$\sqrt{h}$になるように,水を排出する.すなわち,時刻$t$までに排出された水の総量を$V(t)$とおくとき,$\dfrac{\dd V}{\dd t}=\sqrt{h}$がなりたつ.このときすべての水を排出するのに要する時間をもとめよ.

\section{問題6}
$\tan 1^\circ$は有理数か.



\end{document}