 % LuaLaTeX文書; 文字コーAドはUTF-8
 \documentclass[unicode,12pt, A4j]{ltjsarticle}% 'unicode'が必要
 %\usepackage{luatexja}% 日本語したい
 \usepackage{luatexja-fontspec}
 %\usepackage[hiragino-pron]{luatexja-preset}% IPAexフォントしたい(ipaex)
 \usepackage[hiragino-pron,deluxe,expert,bold]{luatexja-preset}

 \usepackage[english]{babel}%多言語文書を作成する
 \usepackage{amsmath,amssymb}%標準数式表現を拡大する
 \usepackage{physics}
 \usepackage[subpreambles=true,sort=true]{standalone}
% \renewcommand{\kanjifamilydefault}{\gtdefault}% 既定をゴシック体に
 \usepackage[backend=bibtex,style=phys,articletitle=false,biblabel=brackets,chaptertitle=false,pageranges=false]{biblatex}
 %\usepackage[style=authoryear,backend=bibtex]{biblatex}


 %\addbibresource{../references/tio2_ref.bib}
 \usepackage{mhchem}
 % あとは欧文の場合と同じ

  \usepackage{caption}
  \usepackage[subrefformat=parens]{subcaption}
\title{京大数学理科後期1995年度}
\author{}
\date{}

\begin{document}
\maketitle

\section{問題1}
自然数$n$に対して,$x^n$を$x^2+ax+b$で割った余りを$r_nx+s_n$とする.次の$2$条件(イ),(ロ)を考える.

\begin{itemize}
 \item[(イ)] $x^2+ax+b=(x-\alpha)(x-\beta),\, \alpha>\beta>0$と表せる.
 \item[(ロ)] 全ての自然数$n$に対して$r_n<r_{n+1}$が成り立つ.
\end{itemize}

\begin{enumerate}
 \item (イ),(ロ)が満たされる時,全ての自然数$n$に対して${\displaystyle \beta-1<\left(\frac{\alpha}{\beta}\right)^n(\alpha-1)}$が成り立つことを示せ.
 \item 実数$a$,$b$がどのような範囲にあるとき (イ),(ロ)が満たされるか.必要十分条件を求め,点$(a,b)$の存在する範囲を図示せよ.
\end{enumerate}


\section{問題2}
$\mathrm{O}$を中心とする円周上に相異なる$3$点$\mathrm{A}_0$,$\mathrm{B}_0$,$\mathrm{C}_0$が時計回りの順に置かれている.自然数$n$に対し,点$\mathrm{A}_n$,$\mathrm{B}_n$,$\mathrm{C}_n$を次の規則で定めていく.
\begin{itemize}
 \item[(イ)] $\mathrm{A}_n$は弧$\mathrm{A}_{n-1}\mathrm{B}_{n-1}$を二等分する点である.(ここで弧$\mathrm{A}_{n-1}\mathrm{B}_{n-1}$は他の点$\mathrm{C}_{n-1}$を含まない方を考える.以下においても同様である.)
 \item[(ロ)] $\mathrm{B}_n$は弧$\mathrm{B}_{n-1}\mathrm{C}_{n-1}$を二等分する点である.
 \item[(ハ)] $\mathrm{C}_n$は弧$\mathrm{C}_{n-1}\mathrm{A}_{n-1}$を二等分する点である.
\end{itemize}

$\angle \mathrm{A}_n\mathrm{OB}_n$の大きさを$\alpha_n$とする.ただし,$\angle \mathrm{A}_n\mathrm{OB}_n$は点$\mathrm{C}_n$を含まない方の弧$\mathrm{A}_n\mathrm{B}_n$の中心角を表す.

\begin{enumerate}
 \item 全ての自然数$n$に対して$4\alpha_{n+1}-2\alpha_n+\alpha_{n-1}=2\pi$であることを示せ.
 \item 全ての自然数$n$に対して$\alpha_{n+2}=\dfrac{3}{4}\pi-\dfrac{1}{8}\alpha_{n-1}$であることを示せ.
 \item $\alpha_{3n}$を$\alpha_0$であらわせ.
\end{enumerate}

\section{問題3}
$a$,$b$,$c$は実数で$a\ge 0$,$b\ge 0$とする.
\begin{align*}
 p(x)&=ax^2+bx+c \\
 q(x)&=cx^2+bx+a
\end{align*}
とおく.$-1\le x\le 1$を満たす全ての$x$に対して$|p(x)|\le 1$が成り立つ時,$-1\le x\le 1$を満たす全ての$x$に対して$|q(x)|\le 2$が成り立つことを示せ.

\section{問題4}
\begin{enumerate}
 \item 平面ベクトル$\va*{x}=\begin{pmatrix}x_1\\x_2\end{pmatrix},\,\va*{y}=\begin{pmatrix}y_1\\y_2\end{pmatrix}$から2行2列の行列$P=\begin{pmatrix} x_1&y_1 \\ x_2&y_2\end{pmatrix}$を作る.$\va*{x}$,$\va*{y}$のどの一方も他方の実数倍ではない時,$P$は逆行列を持つことを示せ.
 \item $B=\begin{pmatrix} p&b \\ c&-p\end{pmatrix}$は単位行列の実数倍ではないとする.この時設問(1)のようにして作った$P$が逆行列$P^{-1}$を持ち,
\begin{align*}
 P^{-1}BP=
 \begin{pmatrix}
  0 & p^2+bc \\
  1 & 0 
 \end{pmatrix}
\end{align*}
が成り立つようなベクトル$\va*{x}$,$\va*{y}$があることを示せ.
 \item $A=\begin{pmatrix} a&b \\ c&d\end{pmatrix}$は単位行列の実数倍ではなく,$A'=\begin{pmatrix} a'&b' \\ c'&d' \end{pmatrix}$も単位行列の実数倍ではないとする.$A$,$A'$が
\begin{align*}
 a+d=a'+d',\, ad-bc=a'd'-b'c'
\end{align*}
を満たせば,$P^{-1}BP=A'$となる$P$があることを示せ.
\end{enumerate}


\section{問題5}
$A$と$B$の2人が次のようなゲームを行う.$n$を自然数とし,$A$はそれぞれ$0,1,2,\cdots,n$と書かれた$(n+1)$枚の札を持っている.$B$はそれぞれ$1,2,\cdots,n$と書かれた$n$枚の札を持っているとする.第1回目に$B$が$A$の持札から1枚の札をとり,もし番号が一致する札があればその2枚の札をその場に捨てる.番号が一致しない札はそのまま持ち続ける.次に$B$に持ち札があれば$A$が$B$の持ち札から$1$枚の札をとり,$B$と同じことをする.こうして先に札のなくなった方を勝とする.$A$が勝つ確率を$p_n$,$B$が勝つ確率を$q_n$とする.ただし相手の札を撮る時,どの札も等しい確率で撮るものとする.

\begin{enumerate}
 \item $p_1$,$p_2$,$q_1$,$q_2$を求めよ.
 \item $p_n+q_n=1$,$(n+2)p_n-np_{n-2}=1$,$(n=3,4,5,\cdots)$であることを示せ.
 \item $p_n$を求めよ.
\end{enumerate}

\section{問題6}
曲線$C:y=\dfrac{1}{x}\, (x>0)$,$3$点$\mathrm{A}=(a,0)$,$\mathrm{R}=(4,0)$,$\mathrm{Q}=(0,2)$を考える.ただし$0<a<4$とする.点$\mathrm{A}$から$\mathrm{C}$に接線$L_a$をひき,その$y$軸との交点を$\mathrm{B}$,原点を$\mathrm{O}$とする.
\begin{enumerate}
 \item 直線$\mathrm{RQ}$が接線$L_a$と第1象限の点$\mathrm{M}=(x_0,y_0)$,$x_0>0$,$y_0>0$で交わるための必要十分条件を求めよ.
\end{enumerate}
設問(1)の条件が満たされている時,四角形$\mathrm{OAMQ}$の面積を$\mathrm{T}$,$\triangle\mathrm{ARM}$の面積を$S_1$,$\triangle\mathrm{BQM}$の面積を$S_2$とする.
\begin{enumerate}
 \item[2] $r=S_1+S_2$,$m=S_1S_2$とおく時,点$(r,m)$の存在する範囲を図示せよ.
\end{enumerate}


\end{document}
