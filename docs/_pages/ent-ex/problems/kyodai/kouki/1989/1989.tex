 % LuaLaTeX文書; 文字コーAドはUTF-8
 \documentclass[unicode,12pt, A4j]{ltjsarticle}% 'unicode'が必要
 %\usepackage{luatexja}% 日本語したい
 \usepackage{luatexja-fontspec}
 %\usepackage[hiragino-pron]{luatexja-preset}% IPAexフォントしたい(ipaex)
 \usepackage[hiragino-pron,deluxe,expert,bold]{luatexja-preset}

 \usepackage[english]{babel}%多言語文書を作成する
 \usepackage{amsmath,amssymb}%標準数式表現を拡大する
 \usepackage{physics}
 \usepackage[subpreambles=true,sort=true]{standalone}
% \renewcommand{\kanjifamilydefault}{\gtdefault}% 既定をゴシック体に
 \usepackage[backend=bibtex,style=phys,articletitle=false,biblabel=brackets,chaptertitle=false,pageranges=false]{biblatex}
 %\usepackage[style=authoryear,backend=bibtex]{biblatex}


 %\addbibresource{../references/tio2_ref.bib}
 \usepackage{mhchem}
 % あとは欧文の場合と同じ

  \usepackage{caption}
  \usepackage[subrefformat=parens]{subcaption}
\title{京大理系後期1989年度}

\begin{document}
\section{問題1}
二つの奇数$a$,$b$にたいして,$m=11a+b$,$n=3a+b$とおく.つぎの(1),(2)を証明せよ.
\begin{enumerate}
 \item $m$,$n$の最大公約数は,$a$,$b$の最大公約数を$d$として,$2d$,$4d$,$8d$のいずれかである.
 \item $m$,$n$がともに平衡数であることはない.(整数の$2$乗である数を平方数という.)
\end{enumerate}

\section{問題2}
放物線$y=x^2$を$y$軸のまわりに回転してできる曲面があり,$y$軸が水平面に垂直で$y$軸の正の部分が上方にあるように置いてある.その局面の中に半径$r \left(r>\dfrac{1}{2}\right)$の球を落とし込む.このとき,この回転面と九面とで囲まれた部分の体積を求めよ.

\section{問題3}
座標平面において,つぎの条件を満たす$\triangle \mathrm{ABC}$と半平面$\mathrm{H}=\{(x,y)|x\ge 0\}$との共通部分の面積の最大値を求めよ.

$\triangle \mathrm{ABC}$は$\mathrm{AB}=\mathrm{AC}$であるような二等辺三角形であって,$\mathrm{AC}$は$y$軸に平行で,$\mathrm{A}$の座標は$(-1,0)$である.また,$\mathrm{AB}$と$y$軸との交点を$\mathrm{D}$とすると,$\mathrm{DB}=2\sqrt{3}$である.

\section{問題4}
$a$,$b$はともに$0$でない実数で,$A=\begin{pmatrix} a&b\\c&d \end{pmatrix}$とおく.$\mathrm{O}$を原点とする座標平面において,点$\mathrm{P}(x,y)$が単位円$x^2+y^2=1$の周上を動く.また,点$\mathrm{Q}(x',y')$を$A=\begin{pmatrix} x'\\y'\end{pmatrix}, A\begin{pmatrix} x\\y\end{pmatrix}$によって定める.このとき,$\triangle \mathrm{OPQ}$の面積の最大値を求めよ.

\section{問題5}
$n$は自然数で,$n\ge 3$である.数$1, 2,\cdots n$のいずれについても,それが記入されたカードが$1$枚ずつ,計$n$枚のカードがある.

A君は,それらのカードのうち$2$枚を無作為に取り出し,それらに記入されている数のうち大きい方を$A$君の得点とする.

B君は,それらのカードから$1$枚を無作為に取り出し,書かれている数を確認してから,そのカードを返すことを$2$回繰り返して,書かれている数の大きい(または小さくない)方をB君の得点とする.

A,B両君のうち特典の大きい方を勝ちとする.

A君の勝つ確率$p$とB君の勝つ確率$q$との大小を比較せよ.

\section{問題6}
$F(x)=\dfrac{ax}{x+1}$とおく.ただし,$a$は定数で$0 < a\le 1$である.関数の列$\{f_n(x)\}$を次によって定める.

\begin{itemize}
 \item[(i)] $f_1(x)=F(x)$
 \item[(ii)] $f_{n+1}(x)=F(f_n(x)) (n=1,2,3,\cdots)$
\end{itemize}
\begin{enumerate}
 \item $f_n(x)$を$a$,$x$,$n$の式で表せ.
 \item 次の条件をみたす数列$\{b_n\}$を一つ作れ.($a$の値によって,異なる数列であってもよい.)
\begin{itemize}
 \item[条件] $c>0$ならば,数列$\{b_n\cdot f_n(c)\}$は正の数に収束する.
\end{itemize}
\end{enumerate}


\end{document}