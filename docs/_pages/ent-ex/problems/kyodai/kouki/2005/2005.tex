 % LuaLaTeX文書; 文字コーAドはUTF-8
 \documentclass[unicode,12pt, A4j]{ltjsarticle}% 'unicode'が必要
 %\usepackage{luatexja}% 日本語したい
 \usepackage{luatexja-fontspec}
 %\usepackage[hiragino-pron]{luatexja-preset}% IPAexフォントしたい(ipaex)
 \usepackage[hiragino-pron,deluxe,expert,bold]{luatexja-preset}

 \usepackage[english]{babel}%多言語文書を作成する
 \usepackage{amsmath,amssymb}%標準数式表現を拡大する
 \usepackage{physics}
 \usepackage[subpreambles=true,sort=true]{standalone}
% \renewcommand{\kanjifamilydefault}{\gtdefault}% 既定をゴシック体に
 \usepackage[backend=bibtex,style=phys,articletitle=false,biblabel=brackets,chaptertitle=false,pageranges=false]{biblatex}
 %\usepackage[style=authoryear,backend=bibtex]{biblatex}


 %\addbibresource{../references/tio2_ref.bib}
 \usepackage{mhchem}
 % あとは欧文の場合と同じ

  \usepackage{caption}
  \usepackage[subrefformat=parens]{subcaption}
\title{京大数学理科後期2005年度}
\author{}
\date{}

\begin{document}
\maketitle

\section{問題1}
曲線$y=x^3$の$x>0$の部分を$C$とする.$C$上の点$\mathrm{P}$に対し,$\mathrm{P}$における$C$の接線と$x$軸との交点を$Q$とし,$\mathrm{P}$における$C$の法線と$y$軸との交点を$\mathrm{R}$とする.$\mathrm{P}$が$C$上を動く時,$\dfrac{\mathrm{OR}}{\mathrm{OQ}}$の最小値をもとめよ.ただし,$\mathrm{O}$は原点である.
	  
\section{問題2}
$\dfrac{2z+2i}{z+2i}=\bar{z}$をみたす複素数$z$をすべてもとめよ.(ただし,$i$は虚数単位,$\bar{z}$は$z$ の共役な複素数である.)


\section{問題3}
二次元列ベクトル$A_n\, (n=1,2,3,\cdots)$が
\begin{equation}
 A_1=\begin{pmatrix}\\
      2 \\1 
     \end{pmatrix},
 A_2=\begin{pmatrix}\\
      3 \\1 
     \end{pmatrix},
 A_{n+2}=\begin{pmatrix}\\
      1 & 1 \\ 1 & -1 \\ 
     \end{pmatrix}A_{n+1}+
     \begin{pmatrix}\\
      0 & -1 \\ 1 & 0 \\ 
     \end{pmatrix}A_{n},
 (n=1,2,3,\cdots)
\end{equation}
を満たす時,$A_n$を求めよ.

\section{問題4}
四面体$\mathrm{OABC}$において,三角形$\mathrm{ABC}$の重心を$\mathrm{G}$とし,線分$\mathrm{OG}$を$t:1-t\, (0<t<1)$に内分する点を$\mathrm{P}$とする.また,直線$\mathrm{AP}$と面$\mathrm{OBC}$との交点を$A'$,直線$\mathrm{BP}$と面$\mathrm{OCA}$との交点を$B'$,直線$\mathrm{CP}$と面$\mathrm{OAB}$との交点を$C'$とする.このとき,三角形$\mathrm{A'B'C'}$は三角形$\mathrm{ABC}$と相似であることをしめし,相似比を$t$であらわせ.


\section{問題5}
$n<\int_{10}^{100}\log_{10} x \dd x$を満たす最大の自然数$n$を求めよ.ただし,$0.434<\log_{10} e< 0.435$($e$は自然対数の底)である.

\section{問題6}
$n$枚の$100$円玉と$n+1$枚の$500$円玉を同時に投げたとき,表の出た$100$円玉の枚数より表の出た$500$円玉の枚数の方が多い確率を求めよ.

\end{document}