 % LuaLaTeX文書; 文字コーAドはUTF-8
 \documentclass[unicode,12pt, A4j]{ltjsarticle}% 'unicode'が必要
 %\usepackage{luatexja}% 日本語したい
 \usepackage{luatexja-fontspec}
 %\usepackage[hiragino-pron]{luatexja-preset}% IPAexフォントしたい(ipaex)
 \usepackage[hiragino-pron,deluxe,expert,bold]{luatexja-preset}

 \usepackage[english]{babel}%多言語文書を作成する
 \usepackage{amsmath,amssymb}%標準数式表現を拡大する
 \usepackage{physics}
 \usepackage[subpreambles=true,sort=true]{standalone}
% \renewcommand{\kanjifamilydefault}{\gtdefault}% 既定をゴシック体に
 \usepackage[backend=bibtex,style=phys,articletitle=false,biblabel=brackets,chaptertitle=false,pageranges=false]{biblatex}
 %\usepackage[style=authoryear,backend=bibtex]{biblatex}


 %\addbibresource{../references/tio2_ref.bib}
 \usepackage{mhchem}
 % あとは欧文の場合と同じ

  \usepackage{caption}
  \usepackage[subrefformat=parens]{subcaption}
\title{京大数学理科後期2004年度}
\author{}
\date{}

\begin{document}
\maketitle

\section{問題1}
$x\ge 0$に対して,関数$f(x)$を次のように定義する.
\begin{align*}
 f(x)=
 \begin{cases}
  x & \text{$0\le x\le 1$のとき} \\
  0 & \text{$x>1$のとき} 
 \end{cases}
\end{align*}

このとき,${\displaystyle \lim_{n\to +\infty}n\int_0^1 f(4nx(1-x))\dd x }$をもとめよ.
	  
\section{問題2}
複素数$z$の絶対値を$|z|$で表す.$|(1+i)t+1+\alpha|\le 1$を満たす実数$t$が存在するような複素数$\alpha$の範囲を,複素平面上で図示せよ.(ただし,$i$は虚数単位を表す.)(注意:複素平面のことを複素数平面ともいう.)

\section{問題3}
平面ベクトル$\va*{x}$に対して実数$f(\va*{x})$を対応させる写像$f(\va*{x})$が次の性質(*)を持っている.
\begin{itemize}
 \item[(*)] 任意の平面ベクトル$\va*{a}$,$\va*{b}$に対して,$f(\va*{a}+\va*{b})=f(\va*{a})+f(\va*{b})$が成り立つ.
\end{itemize}
この時,任意の平面ベクトル$\va*{x}$に対して,$f\left(\dfrac{1}{3}\va*{x}\right)=\dfrac{1}{3}f(\va*{x})$が成り立つことを証明せよ.


\section{問題4}
水平面$V$上の$3$点を$\mathrm{O}$,$\mathrm{A}$,$\mathrm{B}$とする.$\mathrm{A}$は線分$\mathrm{OB}$上にあり,線分$\mathrm{AB}$の長さは1メートルであるとする.$\mathrm{O}$から,$V$と垂直に棒が立っている.棒の先端$\mathrm{X}$を$\mathrm{A}$,$\mathrm{B}$から見た時の仰角がそれぞれ$45^{\circ}$,$44^{\circ}$であったという.棒の長さは何メートルか.小数点以下を四捨五入して答えよ.

ただし,$0.01745<\tan 1^{\circ} < 0.01746$である.


\section{問題5}
$n$を自然数とする.次の3つの不等式(1),(2),(3)を全て満たす自然数の組$(a,b,c,d)$はいくつあるか.$n$を用いてあらわせ.
\begin{itemize}
 \item[(1)] $1\le a < d \le n$
 \item[(2)] $a\le b < d$
 \item[(3)] $a<c\le d$
\end{itemize}

\section{問題6}
$n$を自然数とする.$xy$平面内の,原点を中心とする半径$n$の円の,内部と周を合わせたものを$C_n$で表す.次の条件(*)を満たす1辺の長さが$1$の正方形の数を$N(n)$とする.

\begin{itemize}
 \item[(*)] 正方形の4頂点は全て$C_n$に含まれ,4頂点の$x$及び$y$座標は全て整数である.
\end{itemize}

この時,${\displaystyle \lim_{n\to\infty}\frac{N(n)}{n^2}=\pi}$を証明せよ.


\end{document}