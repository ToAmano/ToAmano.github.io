 % LuaLaTeX文書; 文字コーAドはUTF-8
 \documentclass[unicode,12pt, A4j]{ltjsarticle}% 'unicode'が必要
 %\usepackage{luatexja}% 日本語したい
 \usepackage{luatexja-fontspec}
 %\usepackage[hiragino-pron]{luatexja-preset}% IPAexフォントしたい(ipaex)
 \usepackage[hiragino-pron,deluxe,expert,bold]{luatexja-preset}

 \usepackage[english]{babel}%多言語文書を作成する
 \usepackage{amsmath,amssymb}%標準数式表現を拡大する
 \usepackage{physics}
 \usepackage[subpreambles=true,sort=true]{standalone}
% \renewcommand{\kanjifamilydefault}{\gtdefault}% 既定をゴシック体に
 \usepackage[backend=bibtex,style=phys,articletitle=false,biblabel=brackets,chaptertitle=false,pageranges=false]{biblatex}
 %\usepackage[style=authoryear,backend=bibtex]{biblatex}


 %\addbibresource{../references/tio2_ref.bib}
 \usepackage{mhchem}
 % あとは欧文の場合と同じ

  \usepackage{caption}
  \usepackage[subrefformat=parens]{subcaption}
\title{京大数学理科後期1994年度}
\author{}
\date{}

\begin{document}
\maketitle

\section{問題1}
$a+b+c=0$を満たす実数$a$,$b$,$c$について,$(|a|+|b|+|c|)^2\ge 2(a^2+b^2+c^2)$が成り立つことをしめせ.また,ここで統合が成り立つのはどんな場合か.


\section{問題2}
$a$,$b$,$c$,$d$を整数とし,行列$A=\begin{pmatrix} a&b \\ c&d\end{pmatrix}$を考える.$\begin{pmatrix} a_0&b_0 \\ c_0&d_0\end{pmatrix}=\begin{pmatrix} 1&0 \\ 0&1\end{pmatrix}$とし,自然数$n$に対して$A^n=\begin{pmatrix} a_n&b_n \\ c_n&d_n\end{pmatrix}$とする.このとき,

\begin{enumerate}
 \item $n\ge 0$について,$c_{n+2}-(a+d)c_{n+1}+(ad-bc)c_n=0$を示せ.
 \item $p$を素数とし,$a+d$は$p$で割り切れないものとする.ある自然数$k$について,$c_k$と$c_{k+1}$が$p$で割り切れるならば,すべての$n$について$c_n$は$p$で割り切れることを示せ.
\end{enumerate}

\section{問題3}
$xy$平面上で,$(1,1)$を中心とする半径$1$の円を$C$とする.$\mathrm{P}$,$\mathrm{Q}$はそれぞれ$x$軸,$y$軸の正の部分にある点で,線分$\mathrm{PQ}$が円$C$に接しているとする.正三角形$\mathrm{PQR}$を第一象限内に描くとき,頂点$\mathrm{R}$の座標$(a,b)$について,$a$,$b$の間に成り立つ関係式を求めよ.


\section{問題4}
$3$人の選手$A$,$B$,$C$が次の方式で優勝を争う.

まず$A$と$B$が対戦する.そのあとは,一つの対戦が終わると,その勝者と休んでいた選手が勝負をする.このようにして対戦を繰り返し,先に$2$勝した選手を優勝者とする.(2連勝でなくてもよい.)

各回の勝負で引き分けはなく,$A$と$B$は互角の力量であるが,$C$が$A$,$B$に勝つ確率はともに$p$である.

\begin{enumerate}
 \item 2回の対戦で優勝者が決まる確率を求めよ.
 \item ちょうど4回目の対戦で優勝者が決まる確率を求めよ.
 \item $A$,$B$,$C$の優勝する確率が等しくなるような$p$の値を求めよ.
\end{enumerate}

\section{問題5}
実数$r$は$2\pi r >1$を満たすとする.半径$r$の円の周上に2点$\mathrm{P}$,$\mathrm{Q}$を,孤$\mathrm{PQ}$の長さが$1$になるようにとる.点$\mathrm{R}$が孤$\mathrm{PQ}$上を$\mathrm{P}$から$\mathrm{Q}$まで動くとき,弦$\mathrm{PR}$が動いて通過する部分の面積を$S(r)$とする.

$r$が変化するとき,面積$S(r)$の最大値を求めよ.

\section{問題6}
$n$を自然数とし,${\displaystyle I_n=\int_{1}^{e}\left(\log x\right)^n \dd x}$とおく.
\begin{enumerate}
 \item $I_{n+1}$を$I_n$を用いてあらわせ.
 \item すべての$n$に対して,$\dfrac{e-1}{n+1}\le I_n \le \dfrac{(n+1)e+1}{(n+1)(n+2)}$が成り立つことを示せ.
\end{enumerate}


\end{document}