 % LuaLaTeX文書; 文字コーAドはUTF-8
 \documentclass[unicode,12pt, A4j]{ltjsarticle}% 'unicode'が必要
 %\usepackage{luatexja}% 日本語したい
 \usepackage{luatexja-fontspec}
 %\usepackage[hiragino-pron]{luatexja-preset}% IPAexフォントしたい(ipaex)
 \usepackage[hiragino-pron,deluxe,expert,bold]{luatexja-preset}

 \usepackage[english]{babel}%多言語文書を作成する
 \usepackage{amsmath,amssymb}%標準数式表現を拡大する
 \usepackage{physics}
 \usepackage[subpreambles=true,sort=true]{standalone}
% \renewcommand{\kanjifamilydefault}{\gtdefault}% 既定をゴシック体に
 \usepackage[backend=bibtex,style=phys,articletitle=false,biblabel=brackets,chaptertitle=false,pageranges=false]{biblatex}
 %\usepackage[style=authoryear,backend=bibtex]{biblatex}


 %\addbibresource{../references/tio2_ref.bib}
 \usepackage{mhchem}
 % あとは欧文の場合と同じ

  \usepackage{caption}
  \usepackage[subrefformat=parens]{subcaption}
\title{京大数学理科後期1996年度}
\author{}
\date{}

\begin{document}
\maketitle

\section{問題1}
$n$は自然数とする.
\begin{enumerate}
 \item 全ての実数$\theta$に対し
       \begin{align*}
	\cos n\theta=f_n(\cos\theta), \sin n\theta=g_n(\sin\theta)\sin\theta
       \end{align*}
       を満たし,係数がともに全て整数である$n$次式$f_n(x)$と$n-1$次式$g_n(x)$が存在することを示せ.
 \item $f_n(x)=ng_n(x)$であることを示せ.
 \item $p$を$3$以上の素数とするとき,$f_p(x)$の$p-1$次以下の係数は全て$p$で割り切れることを示せ.
\end{enumerate}


\section{問題2}
$m$,$n$は自然数で,$m<n$を満たすものとする.$m^n+1$,$n^m+1$がともに$10$の倍数となる$m$,$n$を$1$組与えよ.

\section{問題3}
平面上に$60$度で交わる$2$直線$l$,$m$がある.この平面上に点$\mathrm{P}_1$をとり,$\mathrm{P}_1$と直線$l$について対称な点を$\mathrm{Q}_1$,$\mathrm{Q}_1$と直線$m$について対称な点を$\mathrm{P}_2$と定め,以下同様に点$mathrm{Q}_2$,$\mathrm{P}_3$,$\cdots$を定める.

\begin{enumerate}
 \item $\mathrm{P}_4=\mathrm{P}_1$となることを示せ.
 \item 点$\mathrm{P}_1$が$2$直線$l$,$m$の交点を中心とし半径$1$の円周上を動くとき,点$\mathrm{P}_1$,$\mathrm{Q}_1$,$\mathrm{P}_2$,$\mathrm{Q}_2$,$\mathrm{P}_3$,$\mathrm{Q}_3$,$\mathrm{P}_4$をこの順に結ぶ折れ線の長さの最大値を求めよ.
\end{enumerate}

\section{問題4}
$x$,$y$は$x+y>0$,$x-y>0$を満たす実数とする.ある四面体の隣り合う$2$辺の長さが$\sqrt{x+y}$,$\sqrt{x-y}$で,残り$4$辺の長さは全て$1$であるという.このような条件を満たす点$(x,y)$の存在範囲を図示せよ.

\section{問題5}
$a$は与えられた実数で,$0<a\le 1$を満たすものとする.$xyz$空間内に$1$辺の長さ$2a$の正三角形$\triangle \mathrm{PQR}$を考える.辺$\mathrm{PQ}$は$xy$平面上にあり,$\triangle\mathrm{PQR}$を含む平面は$xy$平面と垂直で,さらに点$\mathrm{R}$の$z$座標は正であるとする.
\begin{enumerate}
 \item 辺$\mathrm{PQ}$が$xy$平面の単位円の内部(周を含む)を自由に動くとき,$\triangle\mathrm{PQR}$(内部を含む)が動いてできる立体の体積$V$を求めよ.
 \item $a$が$0<a\le 1$の範囲を動くとき,体積$V$の最大値を求めよ.
\end{enumerate}

\section{問題6}
$n$を$3$以上の整数とする.円周上の$n$等分点のある点を出発点とし,$n$等分点を一定の方向に次のようにすすむ.各点でコインを投げ,表が出れば次の点に進み,裏が出れば次の点を飛び越しその次の点にすすむ.

\begin{enumerate}
 \item 最初に$1$周回ったとき,出発点を飛び越す確率$p_n$を求めよ.
 \item $k$は$2$以上の整数とする.$k-1$周目までは出発点を飛び越し,$k$周目に初めて出発点を踏む確率を$q_{n,k}$とする.このとき$\lim_{n\to\infty}q_{n,k}$を求めよ.
\end{enumerate}
\end{document}