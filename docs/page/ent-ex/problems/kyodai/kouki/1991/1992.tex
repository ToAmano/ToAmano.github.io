 % LuaLaTeX文書; 文字コーAドはUTF-8
 \documentclass[unicode,12pt, A4j]{ltjsarticle}% 'unicode'が必要
 %\usepackage{luatexja}% 日本語したい
 \usepackage{luatexja-fontspec}
 %\usepackage[hiragino-pron]{luatexja-preset}% IPAexフォントしたい(ipaex)
 \usepackage[hiragino-pron,deluxe,expert,bold]{luatexja-preset}

 \usepackage[english]{babel}%多言語文書を作成する
 \usepackage{amsmath,amssymb}%標準数式表現を拡大する
 \usepackage{physics}
 \usepackage[subpreambles=true,sort=true]{standalone}
% \renewcommand{\kanjifamilydefault}{\gtdefault}% 既定をゴシック体に
 \usepackage[backend=bibtex,style=phys,articletitle=false,biblabel=brackets,chaptertitle=false,pageranges=false]{biblatex}
 %\usepackage[style=authoryear,backend=bibtex]{biblatex}


 %\addbibresource{../references/tio2_ref.bib}
 \usepackage{mhchem}
 % あとは欧文の場合と同じ

  \usepackage{caption}
  \usepackage[subrefformat=parens]{subcaption}
\title{京大数学理科後期1992年度}
\author{}
\date{}

\begin{document}
\maketitle

\section{問題1}
$0$でない$x$の整式$f(x)$に対し,${\displaystyle F(x)=\int_0^x f(t)\dd t }$,${\displaystyle G(x)=\int_x^1 f(t)\dd t }$とおく.ある定数$p$,$q$が存在して,$F(G(x))=-\{F(x)\}^2+pG(x)+q$が成立しているとする.
\begin{enumerate}
 \item ${\displaystyle a=\int_0^1 f(t)\dd t }$とおくとき,$F(x)$を$a$を用いてあらわせ.
 \item さらに$0\le x\le 1$での$F(x)$の最大値が$\dfrac{1}{2}$であるとき,$f(x)$を求めよ.
\end{enumerate}

\section{問題2}
一辺の長さが$n$の立方体$\mathrm{ABCD-PQRS}$がある.ただし,$2$つの正方形$\mathrm{ABCD}$,$\mathrm{PQRS}$は立方体の向かい合った面で$\mathrm{AP}$,$\mathrm{BQ}$,$\mathrm{CR}$,$\mathrm{DS}$は,それぞれ,立方体の辺である.

立方体の各面は一辺の長さ$1$の正方形に碁盤目状に区切られているとする.そこで,頂点$\mathrm{A}$から頂点$\mathrm{R}$へ碁盤目上の辺を辿っていくときの最短経路を考える.

\begin{enumerate}
 \item 辺$\mathrm{BC}$上の点を通過する最短経路は全部で何通りあるか.
 \item 頂点$\mathrm{A}$から頂点$\mathrm{R}$への最短経路は全部で何通りあるか.
\end{enumerate}

\section{問題3}
放物線$y=x^2$の上の点$\mathrm{P}(t,t^2)$(ただし,$t>0$)でこの曲線に接し,かつ$y$軸にも接する円を$C_1$,$C_2$とし,それぞれの半径を$r$,$R(r<R)$とする.
\begin{enumerate}
 \item $t$が正の実数全体を動くとき,$\dfrac{R}{r}$のとり得る値の範囲を求めよ.
 \item $\dfrac{R}{r}=2$となる点$\mathrm{P}(t,t^2)$をもとめよ.
\end{enumerate}

\section{問題4}
平面ベクトル$\va*{p}$,$\va*{q}$の内積を$\va*{p}\cdot\va*{q}$と表す.$f$は平面上の一次変換とする.

\begin{enumerate}
 \item $\va*{p}$,$\va*{q}$がたがいに直交する単にベクトルとすると,$T=f(\va*{p})\cdot\va*{p}+f(\va*{q})\cdot\va*{q}$は,ベクトルの組$\va*{p}$,$\va*{q}$の取り方によらないで,$f$によって決まる値であることを示せ.
 \item 原点$\mathrm{O}$を通る$2$つの定直線$l$と$m$があって,$f$によって$l$上の任意の点$\mathrm{R}$は$\mathrm{R}$自身に移され,$m$上の任意の点$\mathrm{S}$は$\mathrm{OS}$の中点$\mathrm{S}'$に移されるとする.このとき$f$に対する$T$の値を求めよ.
\end{enumerate}

\section{問題5}
$1$から$N+2 (N\ge 2)$までの番号のついた玉$(N+2)$個を用意し,手元に$1$と$2$の番号のついた玉をおき,残り$N$個の玉を箱に入れる.さらに,
\begin{quote}
 「玉を一つ箱からとりだし,手元の玉$2$個と取り出した玉$1$個計$3$個の玉のうち最も小さい番号の玉を箱に返す.」
\end{quote}
という操作を$n$回繰り返す($n\ge 1$).最後に手元に残った$2$個の玉の番号のうち小さい方を$X$とし,大きい方を$Y$とする.
\begin{enumerate}
 \item $Y\le m$である確率$P (Y\le m)$をもとめよ($m=3,4,\cdots, N+2$).
 \item $X\le m$である確率$P (X\le m)$をもとめよ($m=2,3,\cdots, N+1$).
\end{enumerate}


\section{問題6}
$ a=\dfrac{1+\sqrt{5}}{2}$とし,空間内の原点$\mathrm{O}$と$4$つの点
\begin{align*}
 \mathrm{A}(1,1,1), \mathrm{B}(-1/a,a,0),\mathrm{C}(-a,0,1/a),\mathrm{D}(0,-1/a,a),
\end{align*}
について,次の問に答えよ.
\begin{enumerate}
 \item 四点$\mathrm{A}$,$\mathrm{B}$,$\mathrm{C}$,$\mathrm{D}$は正方形の頂点であることを示せ.
 \item 四角錐$\mathrm{O-ABCD}$を平面$x=0$によって二つの部分$W_1$,$W_2$に分けたとき,$W_1$,$W_2$の体積の比を求めよ.
\end{enumerate}


\end{document}