 % LuaLaTeX文書; 文字コーAドはUTF-8
 \documentclass[unicode,12pt, A4j]{ltjsarticle}% 'unicode'が必要
 %\usepackage{luatexja}% 日本語したい
 \usepackage{luatexja-fontspec}
 %\usepackage[hiragino-pron]{luatexja-preset}% IPAexフォントしたい(ipaex)
 \usepackage[hiragino-pron,deluxe,expert,bold]{luatexja-preset}

 \usepackage[english]{babel}%多言語文書を作成する
 \usepackage{amsmath,amssymb}%標準数式表現を拡大する
 \usepackage{physics}
 \usepackage[subpreambles=true,sort=true]{standalone}
% \renewcommand{\kanjifamilydefault}{\gtdefault}% 既定をゴシック体に
 \usepackage[backend=bibtex,style=phys,articletitle=false,biblabel=brackets,chaptertitle=false,pageranges=false]{biblatex}
 %\usepackage[style=authoryear,backend=bibtex]{biblatex}


 %\addbibresource{../references/tio2_ref.bib}
 \usepackage{mhchem}
 % あとは欧文の場合と同じ

  \usepackage{caption}
  \usepackage[subrefformat=parens]{subcaption}
\title{京大数学理科後期2001年度}
\author{}
\date{}

\begin{document}
\maketitle

\section{問題1}
方程式$x^2+2y^2+2z^2-2xy-2xz+2yz-5=0$を満たす正の整数の組$(x,y,z)$を全て求めよ.


\section{問題2}
正の整数$n$に対し,多項式$f_n(x)$を,$n=1$に対しては$f_1(x)=1$とし,$n\ge 2$の時は$f_n(x)=(1+x)f_{n-1}(x^2)$で帰納的に定める.$g_n(x)=(1-x)f_n(x)$とおくとき,$g_n(x)$を求めよ.また,$n\to\infty$のとき$f_n(x)$が収束する実数$x$の範囲を求めよ.


\section{問題3}
複素数平面上の単位円に内接する正五角形で,$1$がその頂点の$1$つとなっているものを考える.この正五角形の辺を延長してできる直線の交点のうち,元の正五角形の頂点以外のもので,実部,虚部がともに正であるものを$z$とする.
\begin{enumerate}
 \item $\alpha=\cos\dfrac{2\pi}{5}+i\sin\dfrac{2\pi}{5}$とするとき,$\alpha$を用いて$z$をあらわせ.ただし,$i$は虚数単位を表す.
 \item $3$点$1$,$\alpha^2$,$z$を通る円は,原点を通ることを示せ.
\end{enumerate}


\section{問題4}
負でない実数$a$に対し,$0\le r<1$で,$a-r$が整数となる実数$r$を$\{a\}$で表す.すなわち,$\{a\}$は,$a$の小数部分を表す.
\begin{enumerate}
 \item $\{n\log_{10} 2\}<0.02$となる正の整数$n$を一つ求めよ.
 \item $10$進法による表示で$2^n$の最高位の数字が$7$となる正の整数$n$を$1$つ求めよ.ただし,$0.3010<\log_{10}2<0.3011$,$0.8450<\log_{10}7<0.8451$である.
\end{enumerate}


\section{問題5}
行列$A=\begin{pmatrix}a&b\\ c&d\end{pmatrix}$及び実数$a$に対し,行列を用いて著された$x$,$y$に関する$2$つの連立一次方程式
\begin{itemize}
 \item[(i)] $A\begin{pmatrix} x\\ y\end{pmatrix}=\begin{pmatrix} s\\ 1-s\end{pmatrix}$
 \item[(ii)] $A\begin{pmatrix} x\\ y\end{pmatrix}=\begin{pmatrix} 4\\ 5-s\end{pmatrix}$
\end{itemize}
について,次の条件(*)を考える.
\begin{itemize}
 \item[(*)] 方程式(i)には解が存在して,方程式(ii)には解が存在しない.
\end{itemize}
このとき,次の問に答えよ.
\begin{enumerate}
 \item 条件(*)が成り立つとき,$\begin{pmatrix} a\\ c\end{pmatrix}$,$\begin{pmatrix} b\\ d\end{pmatrix}$は,いずれも$\begin{pmatrix} s\\ 1-s\end{pmatrix}$の実数倍であることを示せ.
 \item 条件(*)を満たす$2$つの連立方程式を作ることができるための$s$の条件を求めよ.
\end{enumerate}

\section{問題6}
$xy$平面上の単位円$C_1$と,条件$-1<a<-\dfrac{1}{2}$を満たす実数$a$に対し,点$\mathrm{R}(a,0)$を考える.$C_1$上の点$\mathrm{P}$における$C_1$の接線と,$\mathrm{R}$を通りこの接線と直交する直線との交点を$\mathrm{Q}$とする.点$\mathrm{P}$が$C_1$上を一周するときに,$\mathrm{Q}$が描く曲線を$C_2$とする.$C_2$上の点の$x$座標の最小値が$-1$より小さいことを示し,$C_2$で囲まれる図形の面積を求めよ.

\end{document}