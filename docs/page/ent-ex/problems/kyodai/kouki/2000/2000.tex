 % LuaLaTeX文書; 文字コーAドはUTF-8
 \documentclass[unicode,12pt, A4j]{ltjsarticle}% 'unicode'が必要
 %\usepackage{luatexja}% 日本語したい
 \usepackage{luatexja-fontspec}
 %\usepackage[hiragino-pron]{luatexja-preset}% IPAexフォントしたい(ipaex)
 \usepackage[hiragino-pron,deluxe,expert,bold]{luatexja-preset}

 \usepackage[english]{babel}%多言語文書を作成する
 \usepackage{amsmath,amssymb}%標準数式表現を拡大する
 \usepackage{physics}
 \usepackage[subpreambles=true,sort=true]{standalone}
% \renewcommand{\kanjifamilydefault}{\gtdefault}% 既定をゴシック体に
 \usepackage[backend=bibtex,style=phys,articletitle=false,biblabel=brackets,chaptertitle=false,pageranges=false]{biblatex}
 %\usepackage[style=authoryear,backend=bibtex]{biblatex}


 %\addbibresource{../references/tio2_ref.bib}
 \usepackage{mhchem}
 % あとは欧文の場合と同じ

  \usepackage{caption}
  \usepackage[subrefformat=parens]{subcaption}
\title{京大数学理科後期2000年度}
\author{}
\date{}

\begin{document}
\maketitle

\section{問題1}
$\alpha$,$\beta$,$\gamma$は互いに相異なる複素数とする.
\begin{enumerate}
 \item 複素数平面上で$\dfrac{z-\beta}{z-\alpha}$の虚数部分が正となる$z$の存在する範囲を図示せよ.
 \item 複素数$z$が,$(z-\alpha)(z-\beta)+(z-\beta)(z-\gamma)+(z-\gamma)(z-\alpha)=0$を満たしているとき,$z$は$\alpha$,$\beta$,$\gamma$を頂点とする三角形の内部に存在することを示せ.ただし,$\alpha$,$\beta$,$\gamma$は同一直線上にはないものとする.
\end{enumerate}


\section{問題2}
\begin{enumerate}
 \item $x\ge 0$のとき,不等式$e^x\ge 1+\dfrac{1}{2}x^2$が成立していることを示せ.
 \item 自然数$n$に対して関数$f_n(x)=n^2(x-1)e^{-nx}$の$x\ge 0$における最大値を$M_n$とする.このとき$\sum_{n=1}^{\infty}M_n$を求めよ,
\end{enumerate}

\section{問題3}
$xy$平面上の点で$x$座標,$y$座標がともに整数である点を格子点という.$a$,$k$は整数で$a\ge 2$とし,直線$L: ax+(a^2+1)y=k$を考える.
\begin{enumerate}
 \item 直線$L$上の格子点を一つ求めよ.
 \item $k=a(a^2+1)$のとき,$x>0$,$y>0$の領域に直線$L$上の格子点は存在しないことを示せ.
 \item $k>a(a^2+1)$ならば,$x>0$,$y>0$の領域に直線$L$上の格子点が存在することを示せ.
\end{enumerate}


\section{問題4}
直方体$\mathrm{ABCD}-\mathrm{A'B'C'D'}$において,四角形$\mathrm{ABCD}$と四角形$\mathrm{A'B'C'D'}$は向かい合った$1$組の面であり,$\mathrm{AA'}$,$\mathrm{BB'}$,$\mathrm{CC'}$,$\mathrm{DD'}$はこの直方体の辺である.ここで$\mathrm{AA'}=1$,$\mathrm{AB}=1$,$\mathrm{AD}=\sqrt{2}$とする.この直方体の内部を通る線分$\mathrm{AC'}$上に点$\mathrm{P}$をとり,$\mathrm{P}$を通り$\mathrm{AC'}$に垂直な平面による直方体の切り口を考える.
\begin{enumerate}
 \item $\mathrm{P}$が線分$\mathrm{AC'}$の中点であるとき,切り口は点$\mathrm{B'}$,$\mathrm{D}$を通ることを示せ.
 \item $\mathrm{AP}=x$であるとき,切り口の面積$S(x)$を求めよ.
\end{enumerate}

\section{問題5}
$0$と相異なる複素数$\alpha$に対して数列$\{a_n\}$を$a_n=\alpha^n+\alpha^{-n}$で定める.全ての数$n$について$|a_n|<2$が成立しているとする.
\begin{enumerate}
 \item $|\alpha|=1$が成立することを示せ.
 \item $|a_m|>1$となる自然数$m$が存在することを示せ.
\end{enumerate}

\section{問題6}
関数$f(x)$を${\displaystyle f(x)=\int_0^x\frac{1}{1+t^2}\dd t}$で定める.
\begin{enumerate}
 \item $y=f(x)$の$x=1$における法線の方程式を求めよ.
 \item (1)で求めた法線と$x$軸及び$y=f(x)$のグラフによって囲まれる図形の面積を求めよ.
\end{enumerate}
\end{document}