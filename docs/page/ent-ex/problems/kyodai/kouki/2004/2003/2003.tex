 % LuaLaTeX文書; 文字コーAドはUTF-8
 \documentclass[unicode,12pt, A4j]{ltjsarticle}% 'unicode'が必要
 %\usepackage{luatexja}% 日本語したい
 \usepackage{luatexja-fontspec}
 %\usepackage[hiragino-pron]{luatexja-preset}% IPAexフォントしたい(ipaex)
 \usepackage[hiragino-pron,deluxe,expert,bold]{luatexja-preset}

 \usepackage[english]{babel}%多言語文書を作成する
 \usepackage{amsmath,amssymb}%標準数式表現を拡大する
 \usepackage{physics}
 \usepackage[subpreambles=true,sort=true]{standalone}
% \renewcommand{\kanjifamilydefault}{\gtdefault}% 既定をゴシック体に
 \usepackage[backend=bibtex,style=phys,articletitle=false,biblabel=brackets,chaptertitle=false,pageranges=false]{biblatex}
 %\usepackage[style=authoryear,backend=bibtex]{biblatex}


 %\addbibresource{../references/tio2_ref.bib}
 \usepackage{mhchem}
 % あとは欧文の場合と同じ

  \usepackage{caption}
  \usepackage[subrefformat=parens]{subcaption}
\title{京大数学理科後期2006年度}
\author{}
\date{}

\begin{document}
\maketitle

\section{問題1}
正三角形$\mathrm{ABC}$の辺$\mathrm{AB}$上に点$\mathrm{P}_1$,$\mathrm{P}_2$が,辺$\mathrm{BC}$上に点$\mathrm{Q}_1$,$\mathrm{Q}_2$が,辺$\mathrm{CA}$上に点$\mathrm{R}_1$,$\mathrm{R}_2$があり,どの点も頂点には一致していないとする.この時三角形$\mathrm{P_1Q_1R_1}$の重心と三角形$\mathrm{P_2Q_2R_2}$の重心が一致すれば,$\mathrm{P_1P_2}=\mathrm{Q_1Q_2}=\mathrm{R_1R_2}$が成り立つことを示せ.

\section{問題2}
一辺の長さが1の正三角形$\mathrm{ABC}$の辺$\mathrm{AC}$上に点$\mathrm{D}$をとり,線分$\mathrm{BD}$に沿ってこの三角形を折り曲げ,4点$\mathrm{A}$,$\mathrm{B}$,$\mathrm{C}$,$\mathrm{D}$を頂点とする四面体を作り,その体積を最大にすることを考える.体積が最大になる時の$\mathrm{D}$の位置と,その時の四面体の体積を求めよ.

\section{問題3}
$a$,$b$を実数とする.3次方程式$x^3+ax^2+bx+1=0$は3つの複素数からなる解$\alpha_1$,$\alpha_2$,$\alpha_3$をもち,相異なる$i$,$j$に対し$|\alpha_i-\alpha_j|=\sqrt{3}$を満たしている.このような$a$,$b$の組を全て求めよ.

\section{問題4}
$\{a_n\}$を正の数からなる数列とし,$p$を正の実数とする.このとき$a_{n+1}>\dfrac{1}{2}a_n-p$を満たす番号$n$が存在することを証明せよ.

\section{問題5}
極限${\displaystyle \lim_{n\to\infty}\sum_{k=1}^{2n}(-1)^k\left(\frac{k}{2n}\right)^{100}}$を求めよ.

\section{問題6}
7つの文字を並べた列$a_1,a_2,\cdots,a_7$で,次の3つの条件を満たすものの総数を求めよ.

\begin{itemize}
 \item[i] $a_1,a_2,\cdots a_7$は$\mathrm{A}$,$\mathrm{B}$,$\mathrm{C}$,$\mathrm{D}$,$\mathrm{E}$,$\mathrm{F}$のいずれかである.
 \item[ii] $i=1,2,\cdots,6$に対し,$a_i$と$a_{i+1}$は相異なる.
 \item[iii] $i=1,2,\cdots,6$に対し,$a_i$と$a_{i+1}$は右図において線分で結ばれている.
\end{itemize}

\begin{figure}[htb]
 \centering
 \includestandalone[mode=image]{2003_06_fig}
\end{figure}

\end{document}