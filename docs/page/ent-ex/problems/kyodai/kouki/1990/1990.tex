 % LuaLaTeX文書; 文字コーAドはUTF-8
 \documentclass[unicode,12pt, A4j]{ltjsarticle}% 'unicode'が必要
 %\usepackage{luatexja}% 日本語したい
 \usepackage{luatexja-fontspec}
 %\usepackage[hiragino-pron]{luatexja-preset}% IPAexフォントしたい(ipaex)
 \usepackage[hiragino-pron,deluxe,expert,bold]{luatexja-preset}

 \usepackage[english]{babel}%多言語文書を作成する
 \usepackage{amsmath,amssymb}%標準数式表現を拡大する
 \usepackage{physics}
 \usepackage[subpreambles=true,sort=true]{standalone}
% \renewcommand{\kanjifamilydefault}{\gtdefault}% 既定をゴシック体に
 \usepackage[backend=bibtex,style=phys,articletitle=false,biblabel=brackets,chaptertitle=false,pageranges=false]{biblatex}
 %\usepackage[style=authoryear,backend=bibtex]{biblatex}


 %\addbibresource{../references/tio2_ref.bib}
 \usepackage{mhchem}
 % あとは欧文の場合と同じ

  \usepackage{caption}
  \usepackage[subrefformat=parens]{subcaption}
\title{京大数学理科後期1990年度}
\author{}
\date{}

\begin{document}
\maketitle

\section{問題1}
曲線$y=x^4-6x^2$に,点$(a,b)$を通る$4$つの接線が引けるのは,$(a,b)$がどのような範囲にあるときか,図示せよ.


\section{問題2}
$f(x)=\dfrac{a-\cos x}{x^2}$が$0 < x \le \dfrac{\pi}{2}$の範囲で増加関数となるような定数$a$のうち最大のものを求めよ.


\section{問題3}
関数$y=\log x$のグラフ上の$1$点$\mathrm{P}(xlog s) (s\ge 1)$における雪線と$y$軸の交点を$\mathrm{Q}$とする.グラフの上に$\mathrm{A}(1,0)$をとる.$\mathrm{AP}$間のグラフの長さを$\hat{\mathrm{AP}}$,線分$\mathrm{PQ}$の長さを$\overline{\mathrm{PQ}}$とし,$t=\overline{\mathrm{PQ}}-\hat{\mathrm{AP}}$とする.$t$は$s$の関数である.

\begin{enumerate}
 \item $\dfrac{\dd t}{\dd s}$を$s$であらわせ.
 \item $u=\dfrac{1}{s}$,$v=\sqrt{1+u^2}$とおくとき,$\dfrac{\dd u}{\dd t}$および$\dfrac{\dd v}{\dd t}$を$u$の関数としてあらわせ.
 \item $u$を$t$の関数としてあらわせ.
\end{enumerate}


\section{問題4}
座標空間に$3$点$\mathrm{P}$,$\mathrm{Q}$,$\mathrm{R}$があって毎秒$1$秒の速さで,それぞれ
\begin{itemize}
 \item 点$\mathrm{P}$は原点$(0,0,0)$を出発して$x$軸上を正の方向へ,
 \item 点$\mathrm{Q}$は点$(2,0,0)$を出発して$y$軸と平行に正の方向へ,
 \item 点$\mathrm{R}$は点$(2,2,0)$を出発して$z$軸と平行に正の方向へ,
\end{itemize}
進む.このとき三角形$\mathrm{PQR}$の面積$S$が最小となるのは何秒後か.


\section{問題5}
平面上に$2$つの円$\mathrm{C}$,$\mathrm{C'}$がある.一次変換$f$は逆変換をもち,かつ$\mathrm{C}$を$\mathrm{C'}$にうつしている.
\begin{enumerate}
 \item $l$を$C$上の1点$\mathrm{P}$における雪線とする.このとき$l$の$f$による像$l'$は点$f(\mathrm{P})$における$\mathrm{C'}$の接線である.この理由を述べよ.
 \item $\mathrm{A}$を$\mathrm{C}$の中心とすれば,$f(\mathrm{A})$は$\mathrm{C'}$の中心となる.この理由を述べよ.
\end{enumerate}


\section{問題6}
$n$本のくじの中に$1$本だけ当たりくじがある.このくじを無作為に$1$本ひき,引いたくじをもとに戻すという試行を$l$回繰り返す.$l$回のうち当たった回数を$X$とする.確率変数$X_i (1\le i\le l)$を次により定める.
\begin{align*}
 X_i=
 \begin{cases}
  1 & \text{$i$回目に当たりくじがでたとき,}\\
  0 &  \text{$i$回目に当たりくじがでないとき.}
 \end{cases}
\end{align*}
\begin{enumerate}
 \item 確率変数$X$を$X_i (1\le i\le l)$であらわせ.
 \item $X^2$の期待値$E(X^2)$を求めよ.
 \item $E(X^2)>2$となる最小の$l$は何か.
\end{enumerate}


\end{document}