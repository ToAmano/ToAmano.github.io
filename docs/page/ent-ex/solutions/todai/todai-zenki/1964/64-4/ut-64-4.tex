\documentclass[a4j]{jarticle}
\usepackage{amsmath}
\usepackage{ascmac}
\usepackage{amssymb}
\usepackage{enumerate}
\usepackage{multicol}
\usepackage{framed}
\usepackage{fancyhdr}
\usepackage{latexsym}
\usepackage{indent}
\usepackage{cases}
\allowdisplaybreaks
\pagestyle{fancy}
\lhead{}
\chead{}
\rhead{東京大学前期$1964$年$4$番}
\begin{document}
%分数関係


\def\tfrac#1#2{{\textstyle\frac{#1}{#2}}} %数式中で文中表示の分数を使う時


%Σ関係

\def\dsum#1#2{{\displaystyle\sum_{#1}^{#2}}} %文中で数式表示のΣを使う時


%ベクトル関係


\def\vector#1{\overrightarrow{#1}}  %ベクトルを表現したいとき(aベクトルを表現するときは\ver
\def\norm#1{|\overrightarrow{#1}|} %ベクトルの絶対値
\def\vtwo#1#2{ \left(%
      \begin{array}{c}%
      #1 \\%
      #2 \\%
      \end{array}%
      \right) }                        %2次元ベクトル成分表示
      
      \def\vthree#1#2#3{ \left(
      \begin{array}{c}
      #1 \\
      #2 \\
      #3 \\
      \end{array}
      \right) }                        %3次元ベクトル成分表示



%数列関係


\def\an#1{\verb|{|$#1$\verb|}|}


%極限関係

\def\limit#1#2{\stackrel{#1 \to #2}{\longrightarrow}}   %等式変形からの極限
\def\dlim#1#2{{\displaystyle \lim_{#1\to#2}}} %文中で数式表示の極限を使う



%積分関係

\def\dint#1#2{{\displaystyle \int_{#1}^{#2}}} %文中で数式表示の積分を使う時

\def\ne{\nearrow}
\def\se{\searrow}
\def\nw{\nwarrow}
\def\ne{\nearrow}


%便利なやつ

\def\case#1#2{%
 \[\left\{%
 \begin{array}{l}%
 #1 \\%
 #2%
 \end{array}%
 \right.\] }                           %場合分け
 
\def\1{$\cos\theta=c$,$\sin\theta=s$とおく.}  %cs表示を与える前書きシータ
\def\2{$\cos t=c$,$\sin t=s$とおく.}     %cs表示を与える前書きt
\def\3{$\cos x=c$,$\sin x=s$とおく.}                %cs表示を与える前書きx

\def\fig#1#2#3 {%
\begin{wrapfigure}[#1]{r}{#2 zw}%
\vspace*{-1zh}%
\input{#3}%
\end{wrapfigure} }           %絵の挿入


\def\a{\alpha}   %ギリシャ文字
\def\b{\beta}
\def\g{\gamma}

%問題番号のためのマクロ

\newcounter{nombre} %必須
\renewcommand{\thenombre}{\arabic{nombre}} %任意
\setcounter{nombre}{2} %任意
\newcounter{nombresub}[nombre] %親子関係を定義
\renewcommand{\thenombresub}{\arabic{nombresub}} %任意
\setcounter{nombresub}{0} %任意
\newcommand{\prob}[1][]{\refstepcounter{nombre}#1[問題 \thenombre]}
\newcommand{\probsub}[1][]{\refstepcounter{nombresub}#1(\thenombresub)}


%1-1みたいなカウンタ(todaiとtodaia)
\newcounter{todai}
\setcounter{todai}{0}
\newcounter{todaisub}[todai] 
\setcounter{todaisub}{0} 
\newcommand{\todai}[1][]{\refstepcounter{todai}#1 \thetodai-\thetodaisub}
\newcommand{\todaib}[1][]{\refstepcounter{todai}#1\refstepcounter{todaisub}#1 {\bf [問題 \thetodai.\thetodaisub]}}
\newcommand{\todaia}[1][]{\refstepcounter{todaisub}#1 {\bf [問題 \thetodai.\thetodaisub]}}


     \begin{oframed}
     $4$点$A_1(0,0)$,$A_2(1,0)$,$A_3(2,2)$,$A_4(0,2)$を頂点とする四辺形がある.この平面上
     に$4$点$P_1$,$P_2$,$P_3$,$P_4$をとって,点$P_1$は$P_4A_4$の中点,点$P_2$は
     $P_1A_1$の中点,点$P_3$は$P_2A_2$の中点,点$P_4$は$P_3A_3$の中点となるようにする.
     
     $4$点$P_1$,$P_2$,$P_3$,$P_4$の座標及び四辺形$P_1P_2P_3P_4$の面積を求めよ.
     \end{oframed}

\setlength{\columnseprule}{0.4pt}
\begin{multicols}{2}
{\bf[解]}$K=1,2,3,4$とする.$P_k(a_k,b_k)$とおく.題意から
     \begin{align*}
     \left\{
          \begin{array}{l}
          (2a_1,2b_1)=(a_4,b_4+2)  \\
          (2a_2,2b_2)=(a_1,b_1) \\
          (2a_3,2b_3)=(a_2+1,b_2)  \\
          (2a_4,2b_4)=(a_3+2,b_3+2)
          \end{array}
     \right. \\
     \Longleftrightarrow
     \left\{
          \begin{array}{l}
          2a_1=a_4  \\
          2a_2=a_1 \\
          2a_3=a_2+1  \\
          2a_4=a_3+2
          \end{array}
     \right.     
     \left\{
          \begin{array}{l}
          2b_1=b_4+2  \\
          2b_2=b_1 \\
          2b_3=b_2  \\
          2b_4=b_3+2
          \end{array}
     \right.     
     \end{align*}
である.これからまず
     \begin{align*}
     &a_1=\frac{a_4}{2}=\frac{a_3+2}{4}=\frac{a_2+5}{8}=\frac{a_1+10}{16} \\\Longleftrightarrow 
     &a_1=\frac{2}{3}     
     \end{align*}
となり,順に$(a_2,a_3,a_4)=(1/3,2/3,4/3)$である.次に同様に
     \begin{align*}
     &b_1=\frac{b_4+2}{2}=\frac{b_3+6}{4}=\frac{b_2+12}{8}=\frac{b_1+24}{16}\\ \Longleftrightarrow
     &b_1=\frac{8}{5}
     \end{align*}
であるから,$(b_2,b_3,b_4)=(4/5,2/5,6/5)$が従う.以上から
     \begin{align*}
     P_1\left(\frac{2}{3},\frac{8}{5}\right) , P_2\left(\frac{1}{3},\frac{4}{5}\right) ,
     P_3\left(\frac{2}{3},\frac{2}{5}\right) , P_4\left(\frac{4}{3},\frac{6}{5}\right)
     \end{align*}
 である.$\cdots$(答)
 
 従って四辺形$P_1P_2P_3P_4$の面積は
      \begin{align*}
      S=\frac{8-2}{5}\frac{4-1}{3}\frac{1}{2}=\dfrac{3}{5}
      \end{align*}
      である.
 $\cdots$(答)
\newpage
\end{multicols}
\end{document}