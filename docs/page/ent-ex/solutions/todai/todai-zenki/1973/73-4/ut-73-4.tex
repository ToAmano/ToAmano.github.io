\documentclass[a4j]{jarticle}
\usepackage{amsmath}
\usepackage{ascmac}
\usepackage{amssymb}
\usepackage{enumerate}
\usepackage{multicol}
\usepackage{framed}
\usepackage{fancyhdr}
\usepackage{latexsym}
\usepackage{indent}
\usepackage{cases}
\usepackage[dvips]{graphicx}
\usepackage{color}
\usepackage{emath}
\usepackage{emathPp}
\allowdisplaybreaks
\pagestyle{fancy}
\lhead{}
\chead{}
\rhead{東京大学前期$1973$年$4$番}
\begin{document}
%分数関係


\def\tfrac#1#2{{\textstyle\frac{#1}{#2}}} %数式中で文中表示の分数を使う時


%Σ関係

\def\dsum#1#2{{\displaystyle\sum_{#1}^{#2}}} %文中で数式表示のΣを使う時


%ベクトル関係


\def\vector#1{\overrightarrow{#1}}  %ベクトルを表現したいとき(aベクトルを表現するときは\ver
\def\norm#1{|\overrightarrow{#1}|} %ベクトルの絶対値
\def\vtwo#1#2{ \left(%
      \begin{array}{c}%
      #1 \\%
      #2 \\%
      \end{array}%
      \right) }                        %2次元ベクトル成分表示
      
      \def\vthree#1#2#3{ \left(
      \begin{array}{c}
      #1 \\
      #2 \\
      #3 \\
      \end{array}
      \right) }                        %3次元ベクトル成分表示



%数列関係


\def\an#1{\verb|{|$#1$\verb|}|}


%極限関係

\def\limit#1#2{\stackrel{#1 \to #2}{\longrightarrow}}   %等式変形からの極限
\def\dlim#1#2{{\displaystyle \lim_{#1\to#2}}} %文中で数式表示の極限を使う



%積分関係

\def\dint#1#2{{\displaystyle \int_{#1}^{#2}}} %文中で数式表示の積分を使う時

\def\ne{\nearrow}
\def\se{\searrow}
\def\nw{\nwarrow}
\def\ne{\nearrow}


%便利なやつ

\def\case#1#2{%
 \[\left\{%
 \begin{array}{l}%
 #1 \\%
 #2%
 \end{array}%
 \right.\] }                           %場合分け
 
\def\1{$\cos\theta=c$,$\sin\theta=s$とおく.}  %cs表示を与える前書きシータ
\def\2{$\cos t=c$,$\sin t=s$とおく.}     %cs表示を与える前書きt
\def\3{$\cos x=c$,$\sin x=s$とおく.}                %cs表示を与える前書きx

\def\fig#1#2#3 {%
\begin{wrapfigure}[#1]{r}{#2 zw}%
\vspace*{-1zh}%
\input{#3}%
\end{wrapfigure} }           %絵の挿入


\def\a{\alpha}   %ギリシャ文字
\def\b{\beta}
\def\g{\gamma}

%問題番号のためのマクロ

\newcounter{nombre} %必須
\renewcommand{\thenombre}{\arabic{nombre}} %任意
\setcounter{nombre}{2} %任意
\newcounter{nombresub}[nombre] %親子関係を定義
\renewcommand{\thenombresub}{\arabic{nombresub}} %任意
\setcounter{nombresub}{0} %任意
\newcommand{\prob}[1][]{\refstepcounter{nombre}#1[問題 \thenombre]}
\newcommand{\probsub}[1][]{\refstepcounter{nombresub}#1(\thenombresub)}


%1-1みたいなカウンタ(todaiとtodaia)
\newcounter{todai}
\setcounter{todai}{0}
\newcounter{todaisub}[todai] 
\setcounter{todaisub}{0} 
\newcommand{\todai}[1][]{\refstepcounter{todai}#1 \thetodai-\thetodaisub}
\newcommand{\todaib}[1][]{\refstepcounter{todai}#1\refstepcounter{todaisub}#1 {\bf [問題 \thetodai.\thetodaisub]}}
\newcommand{\todaia}[1][]{\refstepcounter{todaisub}#1 {\bf [問題 \thetodai.\thetodaisub]}}


     \begin{oframed}
     平面上に$1$辺の長さが$1$の正方形$S$がある.この平面上で$S$を平行移動して得られる正方形で,
     点Pを中心に持つものを$T(\mathrm{P})$とする.このとき,共通部分$S\cap P(\mathrm{P})$の
     面積が$1/2$となるような点Pの存在範囲を図示せよ.またこの範囲の面積を求めよ.
     \end{oframed}

\setlength{\columnseprule}{0.4pt}
\begin{multicols}{2}
{\bf[解]} $S$を,$xy$平面上の
     \begin{align*}
     &|x|\le\frac{1}{2}&|y|\le\frac{1}{2}
     \end{align*}
とする.P$(a,b)$とすれば,$T(\mathrm{P})$は
     \begin{align*}
     &|x-a|\le\frac{1}{2}&|y-b|\le\frac{1}{2}
     \end{align*} 
で表される.対称性から,
     \begin{align}
     a,b\ge0\label{0}
     \end{align}
で考える.

     \begin{zahyou}[ul=10mm](-1,2)(-1,2)
     \tenretu*{P(0.75,0.75)}
     \YGurafu{0.5}{-0.5}{0.5}
     \YGurafu{-0.5}{-0.5}{0.5}
     \XGurafu{0.5}{-0.5}{0.5}
     \XGurafu{-0.5}{-0.5}{0.5}
     \YGurafu{0.25}{0.25}{1.25}
     \YGurafu{1.25}{0.25}{1.25}
     \XGurafu{0.25}{0.25}{1.25}
     \XGurafu{1.25}{0.25}{1.25}
     \Put\P[s]{P}
     \kuromaru{\P}
     \end{zahyou}


これらが共通部分を持つとき,
     \begin{align}
          &\begin{cases}
          a-1/2\le1/2 \\
          b-1/2\le1/2
          \end{cases}\nonumber\\
     \Longleftrightarrow
         &\begin{cases}
          a\le 1 \\
          b\le 1
          \end{cases}\label{1}          
     \end{align}
が条件で,このもとで長方形領域
     \begin{align*}
          \begin{cases}
          a-1/2\le x\le1/2 \\
          b-1/2\le y\le1/2
          \end{cases}
     \end{align*}
が共通部分である.この面積は
     \begin{align*}
     &\left(\frac{1}{2}-\left(a-\frac{1}{2}\right)\right)\left(\frac{1}{2}-\left(b-\frac{1}{2}\right)\right) \\
     =&(1-a)(1-b)
     \end{align*}
 である.従って,求める条件は
      \begin{align}
      \frac{1}{2}\le (1-a)(1-b)\label{2}
      \end{align}
 である.以上\eqref{0},\eqref{1},\eqref{2}からPの存在範囲は下図斜線部(境界含む).

     \begin{zahyou}[ul=30mm](-0.75,0.75)(-0.75,0.75)
     \def\Fx{1-1/(2-2*X)}
     \def\Gx{-1+1/(2-2*X)}
     \def\Hx{1-1/(2+2*X)}
     \def\Mx{-1+1/(2+2*X)}
     \YGurafu\Fx{0}{0.5}
     \YGurafu\Gx{0}{0.5}
     \YGurafu\Hx{-0.5}{0}
     \YGurafu\Mx{-0.5}{0}
     \YNurii*\Fx\Gx{0}{0.5}
     \YNurii*\Hx\Mx{-0.5}{0}
     \end{zahyou}

 
 従って求める面積$U$は,対称性から
      \begin{align*}
      U&=4\int_0^{1/2}\left(1-\frac{1}{2(1-a)}\right)\,da \\
      &=4\teisekibun{a+\frac{1}{2}\log |1-a|}{0}{1/2} \\
      &=2(1-\log 2)
      \end{align*}  
 である.$\cdots$(答)  
\newpage
\end{multicols}
\end{document}