\documentclass[a4j]{jarticle}
\usepackage{amsmath}
\usepackage{ascmac}
\usepackage{amssymb}
\usepackage{enumerate}
\usepackage{multicol}
\usepackage{framed}
\usepackage{fancyhdr}
\usepackage{latexsym}
\usepackage{indent}
\usepackage{cases}
\usepackage[dvips]{graphicx}
\usepackage{color}
\usepackage{emath}
\usepackage{emathPp}
\allowdisplaybreaks
\pagestyle{fancy}
\lhead{}
\chead{}
\rhead{東京大学前期$1990$年$1$番}
\begin{document}
%分数関係


\def\tfrac#1#2{{\textstyle\frac{#1}{#2}}} %数式中で文中表示の分数を使う時


%Σ関係

\def\dsum#1#2{{\displaystyle\sum_{#1}^{#2}}} %文中で数式表示のΣを使う時


%ベクトル関係


\def\vector#1{\overrightarrow{#1}}  %ベクトルを表現したいとき(aベクトルを表現するときは\ver
\def\norm#1{|\overrightarrow{#1}|} %ベクトルの絶対値
\def\vtwo#1#2{ \left(%
      \begin{array}{c}%
      #1 \\%
      #2 \\%
      \end{array}%
      \right) }                        %2次元ベクトル成分表示
      
      \def\vthree#1#2#3{ \left(
      \begin{array}{c}
      #1 \\
      #2 \\
      #3 \\
      \end{array}
      \right) }                        %3次元ベクトル成分表示



%数列関係


\def\an#1{\verb|{|$#1$\verb|}|}


%極限関係

\def\limit#1#2{\stackrel{#1 \to #2}{\longrightarrow}}   %等式変形からの極限
\def\dlim#1#2{{\displaystyle \lim_{#1\to#2}}} %文中で数式表示の極限を使う



%積分関係

\def\dint#1#2{{\displaystyle \int_{#1}^{#2}}} %文中で数式表示の積分を使う時

\def\ne{\nearrow}
\def\se{\searrow}
\def\nw{\nwarrow}
\def\ne{\nearrow}


%便利なやつ

\def\case#1#2{%
 \[\left\{%
 \begin{array}{l}%
 #1 \\%
 #2%
 \end{array}%
 \right.\] }                           %場合分け
 
\def\1{$\cos\theta=c$,$\sin\theta=s$とおく.}  %cs表示を与える前書きシータ
\def\2{$\cos t=c$,$\sin t=s$とおく.}     %cs表示を与える前書きt
\def\3{$\cos x=c$,$\sin x=s$とおく.}                %cs表示を与える前書きx

\def\fig#1#2#3 {%
\begin{wrapfigure}[#1]{r}{#2 zw}%
\vspace*{-1zh}%
\input{#3}%
\end{wrapfigure} }           %絵の挿入


\def\a{\alpha}   %ギリシャ文字
\def\b{\beta}
\def\g{\gamma}

%問題番号のためのマクロ

\newcounter{nombre} %必須
\renewcommand{\thenombre}{\arabic{nombre}} %任意
\setcounter{nombre}{2} %任意
\newcounter{nombresub}[nombre] %親子関係を定義
\renewcommand{\thenombresub}{\arabic{nombresub}} %任意
\setcounter{nombresub}{0} %任意
\newcommand{\prob}[1][]{\refstepcounter{nombre}#1[問題 \thenombre]}
\newcommand{\probsub}[1][]{\refstepcounter{nombresub}#1(\thenombresub)}


%1-1みたいなカウンタ(todaiとtodaia)
\newcounter{todai}
\setcounter{todai}{0}
\newcounter{todaisub}[todai] 
\setcounter{todaisub}{0} 
\newcommand{\todai}[1][]{\refstepcounter{todai}#1 \thetodai-\thetodaisub}
\newcommand{\todaib}[1][]{\refstepcounter{todai}#1\refstepcounter{todaisub}#1 {\bf [問題 \thetodai.\thetodaisub]}}
\newcommand{\todaia}[1][]{\refstepcounter{todaisub}#1 {\bf [問題 \thetodai.\thetodaisub]}}


     \begin{oframed}
     $a_n=\dsum{k=1}{n}\dfrac{1}{\sqrt{k}}$,$b_n=\dsum{k=1}{n}\dfrac{1}{\sqrt{2k+1}}$とするとき,$\dlim{n}{\infty}a_n$,
     $\dlim{n}{\infty}\dfrac{b_n}{a_n}$を求めよ.
     \end{oframed}

\setlength{\columnseprule}{0.4pt}
\begin{multicols}{2}
{\bf[解]} 
グラフの概形は以下のようになる.

     \begin{zahyou}[ul=8mm](-0.2,5)(-0.2,3) 
     \def\Fx{1/sqrt(X)}
     \YGurafu*[hidarix=0]\Fx
     \kubunkyuusekizu\Fx{0.4}{4}{10}{l}
     \tenretu*{A(0.4,0)}
     \Put\A[s]{$1$}
     \tenretu*{B(4,0)}
     \Put\B[s]{$n+1$}
     \end{zahyou}\\

     \begin{zahyou}[ul=8mm](-0.2,5)(-0.2,3) 
     \def\Fx{1/sqrt(X)}
     \YGurafu*[hidarix=0]\Fx
     \kubunkyuusekizu\Fx{0.4}{4.4}{10}{r}
     \tenretu*{A(0.4,0)}
     \Put\A[s]{$1$}
     \tenretu*{C(4.4,0)}
     \Put\C[s]{$n$}
     \end{zahyou}\\

ここで簡単のため,
     \begin{align*}
     &A= \int_1^n\frac{1}{\sqrt{x}}\,dx&B= \frac{1}{\sqrt{2}}\int_1^n\frac{1}{\sqrt{x}+1/2}\,dx
     \end{align*}
とおく.グラフの面積を比較して,$n$が十分大きい時,
     \begin{align}
     &A+\frac{1}{\sqrt{n+1}}\le a_n\le A+1\nonumber \\
     &2\sqrt{n}+\frac{1}{\sqrt{n+1}}-2\le a_n\le 2\sqrt{n}-1\label{1}
     \end{align}
簡単のためこの左辺を$C$とおく.

同様にして$b_n$についても,
     \begin{align}
     &B+\frac{1}{\sqrt{2n+3}}\le b_n\le B+\frac{1}{\sqrt{3}}\nonumber \\
     &\sqrt{2n+1}+\frac{1}{\sqrt{2n+3}}-\sqrt{3}\le b_n\le \sqrt{2n+1}\label{2}
     \end{align}
となる.この左辺を$D$とおく.

\eqref{1}の両辺は$\infty$に発散するから,挟み撃ちの定理から
     \[a_n\to\infty\]
である.$\cdots$(答)

次に$n$が十分大きい時,\eqref{1},\eqref{2}の両辺は正であることから,
     \begin{align*}
     \frac{D}{2\sqrt{n}-1}\le\frac{b_n}{a_n}\le\frac{\sqrt{2n+1}}{C}
     \end{align*}     
である.
     \begin{align*}
     \text{(左辺)}&= \frac{D}{2\sqrt{n}-1}\\
     &=\frac{\sqrt{2n+1}+1/\sqrt{2n+3}-\sqrt{3}}{2\sqrt{n}-1}\\
     &=\frac{\sqrt{2+1/n}}{2-1/\sqrt{n}}+\frac{1/\sqrt{2n+3}-\sqrt{3}}{2\sqrt{n}-1}\\
     &\limit{n}{\infty}\frac{\sqrt{2}}{2}
     \end{align*}
および
     \begin{align*}
     \text{(右辺)}&=\frac{\sqrt{2n+1}}{C}\\
     &=\frac{\sqrt{2n+1}}{2\sqrt{n}+1/\sqrt{n+1}-2}\\
     &=\frac{\sqrt{2+1/n}}{2+1/\sqrt{n^2+n}-2/\sqrt{n}}\\
     &\limit{n}{\infty}\frac{\sqrt{2}}{2}
     \end{align*}
から,挟み撃ちの定理より
     \[\frac{b_n}{a_n}\to\frac{\sqrt{2}}{2}\]
である.$\cdots$(答)
\newpage
\end{multicols}
\end{document}