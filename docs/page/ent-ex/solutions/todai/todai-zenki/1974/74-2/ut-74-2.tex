\documentclass[a4j]{jarticle}
\usepackage{amsmath}
\usepackage{ascmac}
\usepackage{amssymb}
\usepackage{enumerate}
\usepackage{multicol}
\usepackage{framed}
\usepackage{fancyhdr}
\usepackage{latexsym}
\usepackage{indent}
\usepackage{cases}
\allowdisplaybreaks
\pagestyle{fancy}
\lhead{}
\chead{}
\rhead{東京大学前期$1974$年$2$番}
\begin{document}
%分数関係


\def\tfrac#1#2{{\textstyle\frac{#1}{#2}}} %数式中で文中表示の分数を使う時


%Σ関係

\def\dsum#1#2{{\displaystyle\sum_{#1}^{#2}}} %文中で数式表示のΣを使う時


%ベクトル関係


\def\vector#1{\overrightarrow{#1}}  %ベクトルを表現したいとき(aベクトルを表現するときは\ver
\def\norm#1{|\overrightarrow{#1}|} %ベクトルの絶対値
\def\vtwo#1#2{ \left(%
      \begin{array}{c}%
      #1 \\%
      #2 \\%
      \end{array}%
      \right) }                        %2次元ベクトル成分表示
      
      \def\vthree#1#2#3{ \left(
      \begin{array}{c}
      #1 \\
      #2 \\
      #3 \\
      \end{array}
      \right) }                        %3次元ベクトル成分表示



%数列関係


\def\an#1{\verb|{|$#1$\verb|}|}


%極限関係

\def\limit#1#2{\stackrel{#1 \to #2}{\longrightarrow}}   %等式変形からの極限
\def\dlim#1#2{{\displaystyle \lim_{#1\to#2}}} %文中で数式表示の極限を使う



%積分関係

\def\dint#1#2{{\displaystyle \int_{#1}^{#2}}} %文中で数式表示の積分を使う時

\def\ne{\nearrow}
\def\se{\searrow}
\def\nw{\nwarrow}
\def\ne{\nearrow}


%便利なやつ

\def\case#1#2{%
 \[\left\{%
 \begin{array}{l}%
 #1 \\%
 #2%
 \end{array}%
 \right.\] }                           %場合分け
 
\def\1{$\cos\theta=c$,$\sin\theta=s$とおく.}  %cs表示を与える前書きシータ
\def\2{$\cos t=c$,$\sin t=s$とおく.}     %cs表示を与える前書きt
\def\3{$\cos x=c$,$\sin x=s$とおく.}                %cs表示を与える前書きx

\def\fig#1#2#3 {%
\begin{wrapfigure}[#1]{r}{#2 zw}%
\vspace*{-1zh}%
\input{#3}%
\end{wrapfigure} }           %絵の挿入


\def\a{\alpha}   %ギリシャ文字
\def\b{\beta}
\def\g{\gamma}

%問題番号のためのマクロ

\newcounter{nombre} %必須
\renewcommand{\thenombre}{\arabic{nombre}} %任意
\setcounter{nombre}{2} %任意
\newcounter{nombresub}[nombre] %親子関係を定義
\renewcommand{\thenombresub}{\arabic{nombresub}} %任意
\setcounter{nombresub}{0} %任意
\newcommand{\prob}[1][]{\refstepcounter{nombre}#1[問題 \thenombre]}
\newcommand{\probsub}[1][]{\refstepcounter{nombresub}#1(\thenombresub)}


%1-1みたいなカウンタ(todaiとtodaia)
\newcounter{todai}
\setcounter{todai}{0}
\newcounter{todaisub}[todai] 
\setcounter{todaisub}{0} 
\newcommand{\todai}[1][]{\refstepcounter{todai}#1 \thetodai-\thetodaisub}
\newcommand{\todaib}[1][]{\refstepcounter{todai}#1\refstepcounter{todaisub}#1 {\bf [問題 \thetodai.\thetodaisub]}}
\newcommand{\todaia}[1][]{\refstepcounter{todaisub}#1 {\bf [問題 \thetodai.\thetodaisub]}}


     \begin{oframed}
     長さ$l$の線分が,その両端を放物線$y=x^2$の上にのせて動く.この線分の中点$M$が$x$軸に
     もっとも近い場合の$M$の座標を求めよ.ただし$l\ge1$とする.
     \end{oframed}

\setlength{\columnseprule}{0.4pt}
\begin{multicols}{2}
{\bf[解]}線分の両端を$A(a,a^2)$,$B(b,b^2)$とする.ただし
     \begin{align}
     a<b\label{1}
     \end{align}
     とする.$l$の長さが$l$であるから
     \begin{align}
     |AB|=l \Longleftrightarrow |AB|^2=l^2  \nonumber\\
     (b-a)^2+(b^2-a^2)^2=l^2  \nonumber\\
     (b-a)^2\left(1+(a+b)^2\right)=l^2 \label{2}  
     \end{align}
となる.ここで$t=a+b$,$s=b-a$とおいて,$a$,$b$の存在条件を調べる.     
まずは\eqref{1},\eqref{2}に代入して
     \begin{subnumcases}
     {}
     s>0 \label{3}\\
     s^2(1+t^2)=l^2 \label{4}
     \end{subnumcases}
次に,$ab=\dfrac{t^2-s^2}{4}$であるから,$a$,$b$は$x$の$2$次方程式$x^2-tx+\dfrac{t^2-s^2}{4}=0$の異$2$実解であるから,判別式$D$として
     \begin{align*}
     &D>0\\
     \Longleftrightarrow&t^2-4\frac{t^2-s^2}{4}>0\\
     \Longleftrightarrow&s^2>0
     \end{align*}
これは\eqref{3}から常に成立する.よって$s$,$t$の条件式は\eqref{3},\eqref{4}である.
このもとで$M(X,Y)$として$Y$が最小となる場合を考えれば良い($\because Y\ge0$).
     \begin{align*}
     \left\{
          \begin{array}{l}
          X=\dfrac{a+b}{2}=\dfrac{t}{2}  \\
          Y=\dfrac{a^2+b^2}{2}=\dfrac{t^2+s^2}{4}
          \end{array}
     \right.
     \end{align*}
であるから,\eqref{4}を用いて$t$を消去して
     \begin{align*}
     Y&=\frac{t^2+s^2}{4} \\
     &=\frac{1}{4}\left(s^2+\frac{l^2}{s^2}-1\right)  \\
     &\ge \frac{1}{4}\left(2\sqrt{l^2}-1\right) \tag{$s,l>0$故AM-GM}\\
     &=\frac{1}{4}\left(2l-1\right) \tag{$l>0$}
     \end{align*}
である.等号成立は
     \[s^2=\frac{l^2}{s^2}\Longleftrightarrow s=\sqrt{l},t=\pm\sqrt{l-1}\]
のときで,$l\ge1$からこのような$(s,t)$は必ず存在する.又この時$X=\dfrac{\pm\sqrt{l-1}}{2}$である.
よって求める座標は$\left(\dfrac{\pm\sqrt{l-1}}{2},\dfrac{1}{4}\left(2l-1\right)\right)\cdots$(答) である.   
          
\newpage
\end{multicols}
\end{document}