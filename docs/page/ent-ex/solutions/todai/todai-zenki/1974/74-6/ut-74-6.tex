\documentclass[a4j]{jarticle}
\usepackage{amsmath}
\usepackage{ascmac}
\usepackage{amssymb}
\usepackage{enumerate}
\usepackage{multicol}
\usepackage{framed}
\usepackage{fancyhdr}
\usepackage{latexsym}
\usepackage{indent}
\usepackage{cases}
\usepackage[dvips]{graphicx}
\usepackage{color}
\usepackage{emath}
\usepackage{emathPp}
\allowdisplaybreaks
\pagestyle{fancy}
\lhead{}
\chead{}
\rhead{東京大学前期$1974$年$6$番}
\begin{document}
%分数関係


\def\tfrac#1#2{{\textstyle\frac{#1}{#2}}} %数式中で文中表示の分数を使う時


%Σ関係

\def\dsum#1#2{{\displaystyle\sum_{#1}^{#2}}} %文中で数式表示のΣを使う時


%ベクトル関係


\def\vector#1{\overrightarrow{#1}}  %ベクトルを表現したいとき(aベクトルを表現するときは\ver
\def\norm#1{|\overrightarrow{#1}|} %ベクトルの絶対値
\def\vtwo#1#2{ \left(%
      \begin{array}{c}%
      #1 \\%
      #2 \\%
      \end{array}%
      \right) }                        %2次元ベクトル成分表示
      
      \def\vthree#1#2#3{ \left(
      \begin{array}{c}
      #1 \\
      #2 \\
      #3 \\
      \end{array}
      \right) }                        %3次元ベクトル成分表示



%数列関係


\def\an#1{\verb|{|$#1$\verb|}|}


%極限関係

\def\limit#1#2{\stackrel{#1 \to #2}{\longrightarrow}}   %等式変形からの極限
\def\dlim#1#2{{\displaystyle \lim_{#1\to#2}}} %文中で数式表示の極限を使う



%積分関係

\def\dint#1#2{{\displaystyle \int_{#1}^{#2}}} %文中で数式表示の積分を使う時

\def\ne{\nearrow}
\def\se{\searrow}
\def\nw{\nwarrow}
\def\ne{\nearrow}


%便利なやつ

\def\case#1#2{%
 \[\left\{%
 \begin{array}{l}%
 #1 \\%
 #2%
 \end{array}%
 \right.\] }                           %場合分け
 
\def\1{$\cos\theta=c$,$\sin\theta=s$とおく.}  %cs表示を与える前書きシータ
\def\2{$\cos t=c$,$\sin t=s$とおく.}     %cs表示を与える前書きt
\def\3{$\cos x=c$,$\sin x=s$とおく.}                %cs表示を与える前書きx

\def\fig#1#2#3 {%
\begin{wrapfigure}[#1]{r}{#2 zw}%
\vspace*{-1zh}%
\input{#3}%
\end{wrapfigure} }           %絵の挿入


\def\a{\alpha}   %ギリシャ文字
\def\b{\beta}
\def\g{\gamma}

%問題番号のためのマクロ

\newcounter{nombre} %必須
\renewcommand{\thenombre}{\arabic{nombre}} %任意
\setcounter{nombre}{2} %任意
\newcounter{nombresub}[nombre] %親子関係を定義
\renewcommand{\thenombresub}{\arabic{nombresub}} %任意
\setcounter{nombresub}{0} %任意
\newcommand{\prob}[1][]{\refstepcounter{nombre}#1[問題 \thenombre]}
\newcommand{\probsub}[1][]{\refstepcounter{nombresub}#1(\thenombresub)}


%1-1みたいなカウンタ(todaiとtodaia)
\newcounter{todai}
\setcounter{todai}{0}
\newcounter{todaisub}[todai] 
\setcounter{todaisub}{0} 
\newcommand{\todai}[1][]{\refstepcounter{todai}#1 \thetodai-\thetodaisub}
\newcommand{\todaib}[1][]{\refstepcounter{todai}#1\refstepcounter{todaisub}#1 {\bf [問題 \thetodai.\thetodaisub]}}
\newcommand{\todaia}[1][]{\refstepcounter{todaisub}#1 {\bf [問題 \thetodai.\thetodaisub]}}


     \begin{oframed}
     あるスポーツにおいて,$A$,$B$二チームが試合をして,先に三回勝った方を優勝とする.一回の試合で
     $A$が勝つ確率を$p$,$B$が勝つ確率を$q\,\, (p+q=1,p>0,q>0)$とする.このとき,$A$が優勝する確率
     を$P$,$B$が優勝する確率を$Q$とし,また,優勝チームが決まるまでの試合数を$N$として,次の
     問に答えよ.
           \begin{enumerate}[(1)]
           \item $p>q$のとき,$P-Q$と$p-q$とはどちらが大きいか.
           \item $P-p$を最大にする$p$の値を求めよ.
           \item $N$の期待値を最大にする$p$の値およびそのときの$N$の期待値を求めよ.
           \end{enumerate}
     \end{oframed}

\setlength{\columnseprule}{0.4pt}
\begin{multicols}{2}
{\bf[解]} $A$が優勝する場合は以下のいずれか.
     \begin{align}
     \left\{
          \begin{array}{ll}
          \text{$A$の$3$連勝}&p^3 \\
          \text{$A$の$3$勝$1$敗}& 3p^3q \\
          \text{$A$の$3$勝$2$敗}& 6p^3q^2
          \end{array}\right.\label{0}
     \end{align}
以上から
     \begin{align}
     P&=p^3*3p^3q+6p^3q^2 \nonumber\\
     &=p^3(1+3q+6q^2) \nonumber\\
     &=p^3(6p^2-15p+10)&(\because p+q=1)\label{1}
     \end{align}
同様に対称性から
     \begin{align}
     Q&=q^3(1+3p+6p^2) \\
     &=(1-p)^3(1+3p+6p^2)\label{2}
     \end{align}
である.
     \begin{enumerate}[(1)]
     \item $p>q\Longleftrightarrow 1/2<p<1$のとき,$P+Q=1$に注意して,
          \begin{align*}
          f(p)&=P+Q-(p-q) \\
          &=(2P-1)-(2p-1) \\
          &=2(P-p)\ \ \ \ \ \ \ \ \ \ \ \ \ \ \ \ \ \ \ \ \ \ \ \ \ \ (\because \eqref{1}) \\
          &=2p\left(p^2(6p^2-15p+10)-1\right) \\
          &=4p\left(p-\frac{1}{2}\right)(p-1)(3p^2-3p-1)
          \end{align*}     
     である.$1/2<p<1$から,
          \begin{align*}
          &p>0&p-1<0 \\
          &p-\frac{1}{2}>0&3p^2-3p-1<0
          \end{align*}
     だから,$f(p)>0$すなわち
          \[P-Q>p-q\]
     である.$\cdots$(答)
     
     \item 前問の過程から,$f(p)$を最大にする$p$を求めればよい.
          \begin{align*}
          \frac{1}{2}f'(p)&=30p^4-60p^3+30p^2-1 \\
          &=30p^2(p-1)^2-1
          \end{align*}
     だから,$0<p<1$に注意して
          \begin{align*}
          f'(p)\ge0 \Longleftrightarrow &p^2(p-1)^2\ge\frac{1}{30} \\
          \Longleftrightarrow &p(1-p)\ge\frac{\sqrt{30}}{30} \\
          \Longleftrightarrow &\a \le p\le \b
          \end{align*}
     ただし,$\a$,$\b$は
          \[p^2-p+\frac{\sqrt{30}}{30}=0\]
     の$2$解で,かつ$\a<\b$である.これは$0<p<1$の範囲にあるので,下表を得る.
          \begin{align*}
               \begin{array}{|c|c|c|c|c|c|c|c|}\hline
               p&0 &    &\a &     &\b &   &1    \\\hline
               f'&   & -  &0  & +  &0 & -  &     \\\hline
               f&   &\se&    &\ne&   &\se&    \\\hline
               \end{array}
          \end{align*}
     従って,$f(p)$を最大にするのは$p=0$または$p=\b$である.前問から,
          \[f(0)=0\]
     であり,また,
          \[\b=\frac{1}{2}\left(1+\sqrt{1-\frac{2}{15}\sqrt{30}}\right)>\frac{1}{2}\]
     だから,$f(\b)>0$.故に$f(\b)>f(0)$で,求める$p$の値は
          \[p=\b=\frac{1}{2}\left(1+\sqrt{1-\frac{2}{15}\sqrt{30}}\right)\]
     である.$\cdots$(答)
     
     \item $N$の期待値を$N(p)$とする. \eqref{0}および対称性から,
          \begin{align*}
          N(p)=&3(p^3+q^3)+4(3p^3q+3pq^3)\\
          &+5(6p^3q^2+6p^2q^3) \\
          =&3(p^3+q^3)+12(p^3q+pq^3)\\
          &+30(p^3q^2+p^2q^3)
          \end{align*}
     となる.$p+q=1$に注意して,$t=pq$とおいて変形すると
          \begin{align*}
          N(p)&=3(p^3+q^3)+12pq(p^2+q^2)+30p^2q^2 \\
          &=3(1-3t)+12t(1-2t)+30t^2 \\
          &=3+3t+6t^2 \\
          \end{align*}
     である.$t=p(1-p)$は正で,この範囲で$N(p)$は$t$について単調増加.またAM-GMから
          \begin{align*}
          &t=p(1-p)\le\frac{1}{4}&(\text{等号成立は$p=1/2$})
          \end{align*}
     であるから,$N(p)$は$p=1/2$で最大値$33/8$をとる.$\cdots$(答)
     
     
     \end{enumerate}
\newpage
\end{multicols}
\end{document}