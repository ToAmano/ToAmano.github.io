\documentclass[a4j]{jarticle}
\usepackage{amsmath}
\usepackage{ascmac}
\usepackage{amssymb}
\usepackage{enumerate}
\usepackage{multicol}
\usepackage{framed}
\usepackage{fancyhdr}
\usepackage{latexsym}
\usepackage{indent}
\usepackage{cases}
\usepackage[dvips]{graphicx}
\allowdisplaybreaks
\pagestyle{fancy}
\lhead{}
\chead{}
\rhead{東京大学前期$1992$年$3$番}
\begin{document}
%分数関係


\def\tfrac#1#2{{\textstyle\frac{#1}{#2}}} %数式中で文中表示の分数を使う時


%Σ関係

\def\dsum#1#2{{\displaystyle\sum_{#1}^{#2}}} %文中で数式表示のΣを使う時


%ベクトル関係


\def\vector#1{\overrightarrow{#1}}  %ベクトルを表現したいとき(aベクトルを表現するときは\ver
\def\norm#1{|\overrightarrow{#1}|} %ベクトルの絶対値
\def\vtwo#1#2{ \left(%
      \begin{array}{c}%
      #1 \\%
      #2 \\%
      \end{array}%
      \right) }                        %2次元ベクトル成分表示
      
      \def\vthree#1#2#3{ \left(
      \begin{array}{c}
      #1 \\
      #2 \\
      #3 \\
      \end{array}
      \right) }                        %3次元ベクトル成分表示



%数列関係


\def\an#1{\verb|{|$#1$\verb|}|}


%極限関係

\def\limit#1#2{\stackrel{#1 \to #2}{\longrightarrow}}   %等式変形からの極限
\def\dlim#1#2{{\displaystyle \lim_{#1\to#2}}} %文中で数式表示の極限を使う



%積分関係

\def\dint#1#2{{\displaystyle \int_{#1}^{#2}}} %文中で数式表示の積分を使う時

\def\ne{\nearrow}
\def\se{\searrow}
\def\nw{\nwarrow}
\def\ne{\nearrow}


%便利なやつ

\def\case#1#2{%
 \[\left\{%
 \begin{array}{l}%
 #1 \\%
 #2%
 \end{array}%
 \right.\] }                           %場合分け
 
\def\1{$\cos\theta=c$,$\sin\theta=s$とおく.}  %cs表示を与える前書きシータ
\def\2{$\cos t=c$,$\sin t=s$とおく.}     %cs表示を与える前書きt
\def\3{$\cos x=c$,$\sin x=s$とおく.}                %cs表示を与える前書きx

\def\fig#1#2#3 {%
\begin{wrapfigure}[#1]{r}{#2 zw}%
\vspace*{-1zh}%
\input{#3}%
\end{wrapfigure} }           %絵の挿入


\def\a{\alpha}   %ギリシャ文字
\def\b{\beta}
\def\g{\gamma}

%問題番号のためのマクロ

\newcounter{nombre} %必須
\renewcommand{\thenombre}{\arabic{nombre}} %任意
\setcounter{nombre}{2} %任意
\newcounter{nombresub}[nombre] %親子関係を定義
\renewcommand{\thenombresub}{\arabic{nombresub}} %任意
\setcounter{nombresub}{0} %任意
\newcommand{\prob}[1][]{\refstepcounter{nombre}#1[問題 \thenombre]}
\newcommand{\probsub}[1][]{\refstepcounter{nombresub}#1(\thenombresub)}


%1-1みたいなカウンタ(todaiとtodaia)
\newcounter{todai}
\setcounter{todai}{0}
\newcounter{todaisub}[todai] 
\setcounter{todaisub}{0} 
\newcommand{\todai}[1][]{\refstepcounter{todai}#1 \thetodai-\thetodaisub}
\newcommand{\todaib}[1][]{\refstepcounter{todai}#1\refstepcounter{todaisub}#1 {\bf [問題 \thetodai.\thetodaisub]}}
\newcommand{\todaia}[1][]{\refstepcounter{todaisub}#1 {\bf [問題 \thetodai.\thetodaisub]}}


     \begin{oframed}
     $a$,$b$を正の実数とする.座標空間の$4$点$P(0,0,0)$,$Q(a,0,0)$,$R(0,1,0)$,$S(0,1,b)$が半径$1$の同一球面上にあるとき,
     $P$,$Q$,$R$,$S$を頂点とする四面体に内接する球の半径を$r$とすれば,次の二つの不等式が成り立つことを示せ.
          \begin{align*}
          &\left(\frac{1}{r}-\frac{1}{a}-\frac{1}{b}\right)^2\ge\frac{20}{3}&\frac{1}{r}\ge2\sqrt{\frac{2}{3}}+2\sqrt{\frac{5}{3}}
          \end{align*}
     \end{oframed}

\setlength{\columnseprule}{0.4pt}
\begin{multicols}{2}
{\bf[解]} 半径$1$の円$C$の中心$T(X,Y,Z)$とする.$T$は$PQ$,$RS$,$PR$の垂直二等分面上にあるので,$(X,Y,Z)=(a/2,1/2,b/2)$
である.
    
$|PT|=1$であるから,
     \begin{align}
     \sqrt{X^2+Y^2+Z^2}=1 \nonumber\\
     \sqrt{1+a^2+b^2}=2 \nonumber\\
     a^2+b^2=3\label{1}
     \end{align}
である.

また,四面体$PQRS$の体積$V$を$2$通りで表して,
     \begin{align}
          &V=\frac{1}{3}(\triangle PQR)×b \nonumber\\ 
          &=\frac{r}{3}(\triangle PQR+\triangle PQS+\triangle PRS+\triangle QRS) \nonumber\\
     \Longleftrightarrow
     &\frac{1}{2}ab=\frac{r}{2}(a+a\sqrt{b^2+1}+b+b\sqrt{a^2+1})\nonumber \\
     \Longleftrightarrow
     &ab=r(a+b+a\sqrt{1+b^2}+b\sqrt{1+a^2})\label{2}
     \end{align}     
である.\eqref{1},\eqref{2}のもとで題意の不等式を示す.

まず,AM-GMおよび\eqref{1}より,
     \begin{align}
     ab\le\frac{a^2+b^2}{2}=\frac{3}{2}\label{3}
     \end{align}
である.次いでコーシーシュワルツの不等式から,
     \begin{align}
     A=\frac{1}{a^2}+\frac{1}{b^2}\ge\frac{(1+1)^2}{a^2+b^2}=\frac{4}{3}\label{4}
     \end{align}
である.以上に注意する.

\eqref{2}の両辺$abr_{\not=0}$で割って整理して,
     \begin{align*}
     &\frac{1}{r}=\frac{1}{b}+\frac{1}{a}+\frac{\sqrt{1+b^2}}{b}+\frac{\sqrt{1+a^2}}{a} \label{5}\tag{5}\\
     &\left(\frac{1}{r}-\frac{1}{a}-\frac{1}{b}\right)^2=\left(\frac{\sqrt{1+b^2}}{b}+\frac{\sqrt{1+a^2}}{a}\right)^2 \\
     &=\frac{1+b^2}{b^2}+\frac{1+a^2}{a^2}+2\frac{\sqrt{(1+a^2)(1+b^2)}}{ab} \\
     &=2+A+2\sqrt{1+A+\frac{1}{(ab)^2}} \\
     &\ge1+\frac{4}{3}+2\sqrt{1+\frac{4}{3}+\frac{4}{9}} \tag{$\because \eqref{1},\eqref{2}$}\\
     &=\frac{20}{3}
     \end{align*}
より一つ目の不等式が示された.$\Box$ 

次の不等式を示す.一つ目の不等式の両辺平方根をとって,\eqref{5}より左辺の中身が非負であることから,
     \begin{align}
     &\frac{1}{r}-\frac{1}{a}-\frac{1}{b}\ge 2\sqrt{\frac{5}{3}} \\
     &\frac{1}{r}\ge2\sqrt{\frac{1}{ab}}+2\sqrt{\frac{5}{3}} \tag{$\because$ AM-GM} \\
     &\frac{1}{r}\ge2\sqrt{\frac{2}{3}}+2\sqrt{\frac{5}{3}} \tag{$\because \eqref{1}$} 
     \end{align}
となる.故に示された.$\Box$     
     
\newpage
\end{multicols}
\end{document}