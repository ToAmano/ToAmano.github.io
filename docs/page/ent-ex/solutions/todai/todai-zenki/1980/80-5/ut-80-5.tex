\documentclass[a4j]{jarticle}
\usepackage{amsmath}
\usepackage{ascmac}
\usepackage{amssymb}
\usepackage{enumerate}
\usepackage{multicol}
\usepackage{framed}
\usepackage{fancyhdr}
\usepackage{latexsym}
\usepackage{indent}
\usepackage{cases}
\allowdisplaybreaks
\pagestyle{fancy}
\lhead{}
\chead{}
\rhead{東京大学前期$1980$年$5$番}
\begin{document}
%分数関係


\def\tfrac#1#2{{\textstyle\frac{#1}{#2}}} %数式中で文中表示の分数を使う時


%Σ関係

\def\dsum#1#2{{\displaystyle\sum_{#1}^{#2}}} %文中で数式表示のΣを使う時


%ベクトル関係


\def\vector#1{\overrightarrow{#1}}  %ベクトルを表現したいとき(aベクトルを表現するときは\ver
\def\norm#1{|\overrightarrow{#1}|} %ベクトルの絶対値
\def\vtwo#1#2{ \left(%
      \begin{array}{c}%
      #1 \\%
      #2 \\%
      \end{array}%
      \right) }                        %2次元ベクトル成分表示
      
      \def\vthree#1#2#3{ \left(
      \begin{array}{c}
      #1 \\
      #2 \\
      #3 \\
      \end{array}
      \right) }                        %3次元ベクトル成分表示



%数列関係


\def\an#1{\verb|{|$#1$\verb|}|}


%極限関係

\def\limit#1#2{\stackrel{#1 \to #2}{\longrightarrow}}   %等式変形からの極限
\def\dlim#1#2{{\displaystyle \lim_{#1\to#2}}} %文中で数式表示の極限を使う



%積分関係

\def\dint#1#2{{\displaystyle \int_{#1}^{#2}}} %文中で数式表示の積分を使う時

\def\ne{\nearrow}
\def\se{\searrow}
\def\nw{\nwarrow}
\def\ne{\nearrow}


%便利なやつ

\def\case#1#2{%
 \[\left\{%
 \begin{array}{l}%
 #1 \\%
 #2%
 \end{array}%
 \right.\] }                           %場合分け
 
\def\1{$\cos\theta=c$,$\sin\theta=s$とおく.}  %cs表示を与える前書きシータ
\def\2{$\cos t=c$,$\sin t=s$とおく.}     %cs表示を与える前書きt
\def\3{$\cos x=c$,$\sin x=s$とおく.}                %cs表示を与える前書きx

\def\fig#1#2#3 {%
\begin{wrapfigure}[#1]{r}{#2 zw}%
\vspace*{-1zh}%
\input{#3}%
\end{wrapfigure} }           %絵の挿入


\def\a{\alpha}   %ギリシャ文字
\def\b{\beta}
\def\g{\gamma}

%問題番号のためのマクロ

\newcounter{nombre} %必須
\renewcommand{\thenombre}{\arabic{nombre}} %任意
\setcounter{nombre}{2} %任意
\newcounter{nombresub}[nombre] %親子関係を定義
\renewcommand{\thenombresub}{\arabic{nombresub}} %任意
\setcounter{nombresub}{0} %任意
\newcommand{\prob}[1][]{\refstepcounter{nombre}#1[問題 \thenombre]}
\newcommand{\probsub}[1][]{\refstepcounter{nombresub}#1(\thenombresub)}


%1-1みたいなカウンタ(todaiとtodaia)
\newcounter{todai}
\setcounter{todai}{0}
\newcounter{todaisub}[todai] 
\setcounter{todaisub}{0} 
\newcommand{\todai}[1][]{\refstepcounter{todai}#1 \thetodai-\thetodaisub}
\newcommand{\todaib}[1][]{\refstepcounter{todai}#1\refstepcounter{todaisub}#1 {\bf [問題 \thetodai.\thetodaisub]}}
\newcommand{\todaia}[1][]{\refstepcounter{todaisub}#1 {\bf [問題 \thetodai.\thetodaisub]}}


     \begin{oframed}
     $1$辺の長さが$1$の正四面体$A_0A_1A_2A_3$がある.点$P$はこの正四面体の返上を毎秒
     $1$の速さで動き,各頂点にたっしたとき,そこから出る$3$辺のうちの$1$辺を$\dfrac{1}{3}$
     ずつの確率で選んで進む,$P$は時刻$0$において頂点$A_0$にあるとする.また$n$を$0$または
     正の整数とし,点$P$が時刻$t=n$において頂点$A_i$にある確率を$p_i(n)$で表す.$(i=0,1,2,3)$.
          \begin{enumerate}[(1)]
          \item 数学的昨日法を用いて,$p_1(n)=p_2(n)=p_3(n)$を証明せよ.
          \item $p_0(n)$と$p_1(n)$の値を求めよ.
          \end{enumerate}
     \end{oframed}

\setlength{\columnseprule}{0.4pt}
\begin{multicols}{2}
{\bf[解]}まず,題意から以下の漸化式を得る.
     \begin{align}
          p_i(n+1)=\dfrac{1}{3}(1-p_i(n)) \label{1}
     \end{align}
     \begin{enumerate}[(1)]
     \item 題意から$p_1(0)=p_2(0)=p_3(0)$であるから$n=0$では成立.次に$n=k$での成立を仮定すれ
     ば\eqref{1}から$n=k+1$でも成立.以上から示された.$\Box$
     \item \eqref{1}から,
          \begin{align*}
          p_i(n+1)-\frac{1}{4}=\frac{-1}{3}\left(p_i(n)-\frac{1}{4}\right)
          \end{align*}
     だから,繰り返し用いて
          \begin{align*}
          p_i(n)=\left(\frac{-1}{3}\right)^n\left(p_i(0)-\frac{1}{4}\right)+\frac{1}{4}
          \end{align*}
     となる.初期条件$p_1(0)=0$,$p_0(1)=1$から
          \begin{align*}
          \left\{
               \begin{array}{l}
               p_0(n)=\dfrac{3}{4}\left(\dfrac{-1}{3}\right)^n+\dfrac{1}{4} \\
               p_1(n)=\dfrac{1}{4}\left\{1-\left(\dfrac{-1}{3}\right)^n\right\}
               \end{array}
          \right.
          \end{align*}
     が求める値である.$\cdots$(答)
     \end{enumerate}   
\newpage
\end{multicols}
\end{document}