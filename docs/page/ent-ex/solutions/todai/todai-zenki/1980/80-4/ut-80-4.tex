\documentclass[a4j]{jarticle}
\usepackage{amsmath}
\usepackage{ascmac}
\usepackage{amssymb}
\usepackage{enumerate}
\usepackage{multicol}
\usepackage{framed}
\usepackage{fancyhdr}
\usepackage{latexsym}
\usepackage{indent}
\usepackage{cases}
\allowdisplaybreaks
\pagestyle{fancy}
\lhead{}
\chead{}
\rhead{東京大学前期$1980$年$4$番}
\begin{document}
%分数関係


\def\tfrac#1#2{{\textstyle\frac{#1}{#2}}} %数式中で文中表示の分数を使う時


%Σ関係

\def\dsum#1#2{{\displaystyle\sum_{#1}^{#2}}} %文中で数式表示のΣを使う時


%ベクトル関係


\def\vector#1{\overrightarrow{#1}}  %ベクトルを表現したいとき(aベクトルを表現するときは\ver
\def\norm#1{|\overrightarrow{#1}|} %ベクトルの絶対値
\def\vtwo#1#2{ \left(%
      \begin{array}{c}%
      #1 \\%
      #2 \\%
      \end{array}%
      \right) }                        %2次元ベクトル成分表示
      
      \def\vthree#1#2#3{ \left(
      \begin{array}{c}
      #1 \\
      #2 \\
      #3 \\
      \end{array}
      \right) }                        %3次元ベクトル成分表示



%数列関係


\def\an#1{\verb|{|$#1$\verb|}|}


%極限関係

\def\limit#1#2{\stackrel{#1 \to #2}{\longrightarrow}}   %等式変形からの極限
\def\dlim#1#2{{\displaystyle \lim_{#1\to#2}}} %文中で数式表示の極限を使う



%積分関係

\def\dint#1#2{{\displaystyle \int_{#1}^{#2}}} %文中で数式表示の積分を使う時

\def\ne{\nearrow}
\def\se{\searrow}
\def\nw{\nwarrow}
\def\ne{\nearrow}


%便利なやつ

\def\case#1#2{%
 \[\left\{%
 \begin{array}{l}%
 #1 \\%
 #2%
 \end{array}%
 \right.\] }                           %場合分け
 
\def\1{$\cos\theta=c$,$\sin\theta=s$とおく.}  %cs表示を与える前書きシータ
\def\2{$\cos t=c$,$\sin t=s$とおく.}     %cs表示を与える前書きt
\def\3{$\cos x=c$,$\sin x=s$とおく.}                %cs表示を与える前書きx

\def\fig#1#2#3 {%
\begin{wrapfigure}[#1]{r}{#2 zw}%
\vspace*{-1zh}%
\input{#3}%
\end{wrapfigure} }           %絵の挿入


\def\a{\alpha}   %ギリシャ文字
\def\b{\beta}
\def\g{\gamma}

%問題番号のためのマクロ

\newcounter{nombre} %必須
\renewcommand{\thenombre}{\arabic{nombre}} %任意
\setcounter{nombre}{2} %任意
\newcounter{nombresub}[nombre] %親子関係を定義
\renewcommand{\thenombresub}{\arabic{nombresub}} %任意
\setcounter{nombresub}{0} %任意
\newcommand{\prob}[1][]{\refstepcounter{nombre}#1[問題 \thenombre]}
\newcommand{\probsub}[1][]{\refstepcounter{nombresub}#1(\thenombresub)}


%1-1みたいなカウンタ(todaiとtodaia)
\newcounter{todai}
\setcounter{todai}{0}
\newcounter{todaisub}[todai] 
\setcounter{todaisub}{0} 
\newcommand{\todai}[1][]{\refstepcounter{todai}#1 \thetodai-\thetodaisub}
\newcommand{\todaib}[1][]{\refstepcounter{todai}#1\refstepcounter{todaisub}#1 {\bf [問題 \thetodai.\thetodaisub]}}
\newcommand{\todaia}[1][]{\refstepcounter{todaisub}#1 {\bf [問題 \thetodai.\thetodaisub]}}


     \begin{oframed}
     $xy$平面上の動点$P$の座標$(x,y)$は,時刻$t$を用いて
     \[\left\{
          \begin{array}{l}
          x=\sin t+\cos t  \\
          y=k\sin^2 t\cos^2 t
          \end{array}
     \right.(-\infty<t<\infty)\]
     と表されるものとする.ただし$k$は正の定数である.このとき原点と$P$との距離の二乗
     の最大値及び最小値を,$k$を用いて表せ.
     \end{oframed}

\setlength{\columnseprule}{0.4pt}
\begin{multicols}{2}
{\bf[解]}\2 ただし,周期性から$0\le t<2\pi$としてよい.$f(t)=|OP|^2$とすると
     \begin{align*}
     f(t)&=x^2+y^2 \\
     &=(s+c)^2+(ks^2c^2)^2 \\
     &=k^2p^4+2p+1\equiv g(p)
     \end{align*}
である.ただし$p=sc$とした.このとき$\left(\dfrac{-1}{2}\le p\le\dfrac{1}{2}\right)$である.
ここで$a=\sqrt[3]{\dfrac{1}{2k^2}}$とおけば
     \begin{align*}
     g'(p)&=4k^2p^3+2  \\
     &=4k^2(p+a)(p^2-ap+a^2)
     \end{align*}
から下表を得る.$(\because k>0)$ \\
     \begin{indentation}{2zw}{0pt}
     \noindent\underline{(i)$\dfrac{1}{2}\ge a$つまり$2\le k$の時} \\
          \[\begin{array}{|c|c|c|c|c|c|} \hline
          p & -1/2                  &     &-a                     &      &1/2                        \\ \hline
          g'&                         &  -  & 0                      &  +  &                             \\ \hline
          g &  \dfrac{k^2}{16}&\se&-\dfrac{3a}{2}+1&\ne &\dfrac{k^2}{16}+2  \\ \hline
          \end{array} \]    
    従って
          \[\left\{
               \begin{array}{l}
               \max g=\dfrac{k^2}{16}+2  \\
               \min g=1-\dfrac{3}{2}\sqrt[3]{\dfrac{1}{2k^2}}
               \end{array}
          \right.\]     
     である.
     \\ 
     \\
     \noindent\underline{(ii)$\dfrac{1}{2}\le a$つまり$2\ge k>0$の時} \\
     $g'(p)\ge0$より,$g(p)$は単調増加だから
          \[
               \begin{array}{ll}
               \max g=g\left(\dfrac{1}{2}\right) &  
               \min g=g\left(\dfrac{-1}{2}\right)
               \end{array}
          \]
     である.  \\   
     \end{indentation}
   
以上から求める最大小値は,
     \[\left\{
          \begin{array}{lll}
          0<k\le 2\text{の時} & \max g=\dfrac{k^2}{16}+2 &  \min g=\dfrac{k^2}{16}  \\
          2\le k\text{の時} & \max g=\dfrac{k^2}{16}+2 & \min g=1-\dfrac{3}{2}\sqrt[3]%
          {\dfrac{1}{2k^2}}
          \end{array}
     \right.\]
である.
\newpage
\end{multicols}
\end{document}