\documentclass[a4j]{jarticle}
\usepackage{amsmath}
\usepackage{ascmac}
\usepackage{amssymb}
\usepackage{enumerate}
\usepackage{multicol}
\usepackage{framed}
\usepackage{latexsym}
\usepackage{indent}
\title{}
\begin{document}
%分数関係


\def\tfrac#1#2{{\textstyle\frac{#1}{#2}}} %数式中で文中表示の分数を使う時


%Σ関係

\def\dsum#1#2{{\displaystyle\sum_{#1}^{#2}}} %文中で数式表示のΣを使う時


%ベクトル関係


\def\vector#1{\overrightarrow{#1}}  %ベクトルを表現したいとき(aベクトルを表現するときは\ver
\def\norm#1{|\overrightarrow{#1}|} %ベクトルの絶対値
\def\vtwo#1#2{ \left(%
      \begin{array}{c}%
      #1 \\%
      #2 \\%
      \end{array}%
      \right) }                        %2次元ベクトル成分表示
      
      \def\vthree#1#2#3{ \left(
      \begin{array}{c}
      #1 \\
      #2 \\
      #3 \\
      \end{array}
      \right) }                        %3次元ベクトル成分表示



%数列関係


\def\an#1{\verb|{|$#1$\verb|}|}


%極限関係

\def\limit#1#2{\stackrel{#1 \to #2}{\longrightarrow}}   %等式変形からの極限
\def\dlim#1#2{{\displaystyle \lim_{#1\to#2}}} %文中で数式表示の極限を使う



%積分関係

\def\dint#1#2{{\displaystyle \int_{#1}^{#2}}} %文中で数式表示の積分を使う時

\def\ne{\nearrow}
\def\se{\searrow}
\def\nw{\nwarrow}
\def\ne{\nearrow}


%便利なやつ

\def\case#1#2{%
 \[\left\{%
 \begin{array}{l}%
 #1 \\%
 #2%
 \end{array}%
 \right.\] }                           %場合分け
 
\def\1{$\cos\theta=c$,$\sin\theta=s$とおく.}  %cs表示を与える前書きシータ
\def\2{$\cos t=c$,$\sin t=s$とおく.}     %cs表示を与える前書きt
\def\3{$\cos x=c$,$\sin x=s$とおく.}                %cs表示を与える前書きx

\def\fig#1#2#3 {%
\begin{wrapfigure}[#1]{r}{#2 zw}%
\vspace*{-1zh}%
\input{#3}%
\end{wrapfigure} }           %絵の挿入


\def\a{\alpha}   %ギリシャ文字
\def\b{\beta}
\def\g{\gamma}

%問題番号のためのマクロ

\newcounter{nombre} %必須
\renewcommand{\thenombre}{\arabic{nombre}} %任意
\setcounter{nombre}{2} %任意
\newcounter{nombresub}[nombre] %親子関係を定義
\renewcommand{\thenombresub}{\arabic{nombresub}} %任意
\setcounter{nombresub}{0} %任意
\newcommand{\prob}[1][]{\refstepcounter{nombre}#1[問題 \thenombre]}
\newcommand{\probsub}[1][]{\refstepcounter{nombresub}#1(\thenombresub)}


%1-1みたいなカウンタ(todaiとtodaia)
\newcounter{todai}
\setcounter{todai}{0}
\newcounter{todaisub}[todai] 
\setcounter{todaisub}{0} 
\newcommand{\todai}[1][]{\refstepcounter{todai}#1 \thetodai-\thetodaisub}
\newcommand{\todaib}[1][]{\refstepcounter{todai}#1\refstepcounter{todaisub}#1 {\bf [問題 \thetodai.\thetodaisub]}}
\newcommand{\todaia}[1][]{\refstepcounter{todaisub}#1 {\bf [問題 \thetodai.\thetodaisub]}}


\begin{oframed}
$a=\sin^2\dfrac{\pi}{5}$,$b=\sin^2\dfrac{2\pi}{5}$とおく.このとき,以下のことが成り立つことを
示せ.
     \begin{enumerate}[(1)]
     \item $a+b$および$ab$は有理数である.
     \item 任意の自然数$n$に対し$(a^{-n}+b^{-n})(a+b)^n$は整数である.
     \end{enumerate}     
\end{oframed}

\setlength{\columnseprule}{0.4pt}
\begin{multicols}{2}
{\bf[解]}
$\theta=\dfrac{2\pi}{5},\dfrac{4\pi}{5}$はいずれも$5\theta=2n\pi$をみたす.
($n\in\mathbb{Z}$)ので
     \begin{align*}
     &3\theta=2n\pi-2\theta \\
     \therefore &\cos3\theta=\cos2\theta
     \end{align*}
以下$x=\cos\theta$とすれば倍角,$3$倍角の公式から
     \begin{align*}
     4t^3-3t=2t^2-1 \\
     (t-1)(4t^2+2t-1)=0  \\
     4t^2+2t-1=0 \tag{$\because t\not=0$}
     \end{align*}
ここで$\cos\left(\dfrac{2\pi}{5}\right)\not=\cos\left(\dfrac{4\pi}{5}\right)$より,$2$次方程式の$2$解はこれらである.
故に解と係数の関係から
     \begin{align}
     \left\{
          \begin{array}{l}
          p\equiv\cos\left(\dfrac{2\pi}{5}\right)+\cos\left(\dfrac{4\pi}{5}\right)=\dfrac{-1}{2}  \\
          q\equiv\cos\left(\dfrac{2\pi}{5}\right)\cos\left(\dfrac{4\pi}{5}\right)=\dfrac{-1}{4}
          \end{array}
     \right.     \label{1}
     \end{align}
である.     
     \begin{enumerate}[(1)]
     \item \eqref{1}に注意して,倍角公式から
          \begin{align}
          \left\{
               \begin{array}{l}
               a+b=\frac{1}{2}(2-p)=\frac{4}{5}\in\mathbb{Q} \\
               ab=\frac{1}{4}(1-p)(1-q)=\frac{5}{16}\in\mathbb{Q} 
               \end{array}
          \right.\label{2}     
          \end{align}
     故に示された.$\Box$     
     
     \item $a_n=(a^{-n}+b^{-n})(a+b)^n$,$b_n=a^{-n}+b^{-n}$とおくと,
          \begin{align*}
          b_{n+2}=(\frac{1}{a}+\frac{1}{b})b_{n+1}-\frac{1}{ab}b_n
          \end{align*}
     に注意して
          \begin{align*} 
          a_{n+2}&=b_{n+2}(a+b)^{n+2} \\
          &=\{(\frac{1}{a}+\frac{1}{b})b_{n+1}-\frac{1}{ab}b_n\}(a+b)^n \\
          &=\frac{(a+b)^2}{ab}(a_{n+1}-a_n) \\
          &=5(a_{n+1}-a_n) \tag{$\because\eqref{2}$}
          \end{align*}
     である.これと$a_1=5$,$a_2=15$から,帰納的に$a_n\in\mathbb{Z}$である.$\Box$     
     \end{enumerate}     
\newpage
\end{multicols}
\end{document}