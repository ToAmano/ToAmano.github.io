\documentclass[a4j]{jarticle}
\usepackage{amsmath}
\usepackage{ascmac}
\usepackage{amssymb}
\usepackage{enumerate}
\usepackage{multicol}
\usepackage{framed}
\usepackage{latexsym}
\begin{document}
%分数関係


\def\tfrac#1#2{{\textstyle\frac{#1}{#2}}} %数式中で文中表示の分数を使う時


%Σ関係

\def\dsum#1#2{{\displaystyle\sum_{#1}^{#2}}} %文中で数式表示のΣを使う時


%ベクトル関係


\def\vector#1{\overrightarrow{#1}}  %ベクトルを表現したいとき(aベクトルを表現するときは\ver
\def\norm#1{|\overrightarrow{#1}|} %ベクトルの絶対値
\def\vtwo#1#2{ \left(%
      \begin{array}{c}%
      #1 \\%
      #2 \\%
      \end{array}%
      \right) }                        %2次元ベクトル成分表示
      
      \def\vthree#1#2#3{ \left(
      \begin{array}{c}
      #1 \\
      #2 \\
      #3 \\
      \end{array}
      \right) }                        %3次元ベクトル成分表示



%数列関係


\def\an#1{\verb|{|$#1$\verb|}|}


%極限関係

\def\limit#1#2{\stackrel{#1 \to #2}{\longrightarrow}}   %等式変形からの極限
\def\dlim#1#2{{\displaystyle \lim_{#1\to#2}}} %文中で数式表示の極限を使う



%積分関係

\def\dint#1#2{{\displaystyle \int_{#1}^{#2}}} %文中で数式表示の積分を使う時

\def\ne{\nearrow}
\def\se{\searrow}
\def\nw{\nwarrow}
\def\ne{\nearrow}


%便利なやつ

\def\case#1#2{%
 \[\left\{%
 \begin{array}{l}%
 #1 \\%
 #2%
 \end{array}%
 \right.\] }                           %場合分け
 
\def\1{$\cos\theta=c$,$\sin\theta=s$とおく.}  %cs表示を与える前書きシータ
\def\2{$\cos t=c$,$\sin t=s$とおく.}     %cs表示を与える前書きt
\def\3{$\cos x=c$,$\sin x=s$とおく.}                %cs表示を与える前書きx

\def\fig#1#2#3 {%
\begin{wrapfigure}[#1]{r}{#2 zw}%
\vspace*{-1zh}%
\input{#3}%
\end{wrapfigure} }           %絵の挿入


\def\a{\alpha}   %ギリシャ文字
\def\b{\beta}
\def\g{\gamma}

%問題番号のためのマクロ

\newcounter{nombre} %必須
\renewcommand{\thenombre}{\arabic{nombre}} %任意
\setcounter{nombre}{2} %任意
\newcounter{nombresub}[nombre] %親子関係を定義
\renewcommand{\thenombresub}{\arabic{nombresub}} %任意
\setcounter{nombresub}{0} %任意
\newcommand{\prob}[1][]{\refstepcounter{nombre}#1[問題 \thenombre]}
\newcommand{\probsub}[1][]{\refstepcounter{nombresub}#1(\thenombresub)}


%1-1みたいなカウンタ(todaiとtodaia)
\newcounter{todai}
\setcounter{todai}{0}
\newcounter{todaisub}[todai] 
\setcounter{todaisub}{0} 
\newcommand{\todai}[1][]{\refstepcounter{todai}#1 \thetodai-\thetodaisub}
\newcommand{\todaib}[1][]{\refstepcounter{todai}#1\refstepcounter{todaisub}#1 {\bf [問題 \thetodai.\thetodaisub]}}
\newcommand{\todaia}[1][]{\refstepcounter{todaisub}#1 {\bf [問題 \thetodai.\thetodaisub]}}


\begin{oframed}
$1$つのサイコロを続けて投げて,それによって$a_n(n=1,2,\dots)$を以下のように定める.

出た目の数を順に$c_1,c_2,\dots$とするとき,$1\le k\le n-1$を満たすすべての整数$k$に対し
$c_k\le c_n$ならば$a_n=c_n$,それ以外の時$a_n=0$とおく.ただし$a_1=c_1$とする.
     \begin{enumerate}[(1)]
     \item $a_n$の期待値を$E(n)$とするとき,$\dlim{n}{\infty}E(n)$をもとめよ.
     \item $a_1,a_2,\dots ,a_n$のうち$2$に等しいものの個数の期待値を$N(n)$とするとき,
     $\dlim{n}{\infty}N(n)$を求めよ.
     \end{enumerate}
\end{oframed}

\setlength{\columnseprule}{0.4pt}
\begin{multicols}{2}
{\bf[解]}
     \begin{enumerate}[(1)]
     \item $k$を$1$から$6$までの整数とする.$a_n=k$となる確率は$n=1$のとき$\dfrac{1}{6}$
     で,$n\le2$のときは$a_1$から$a_{n-1}$がすべて$k$以下となる時で
          \begin{align}
          P(a_n=k)=\left(\dfrac{k}{6}\right)^{n-1}\dfrac{1}{6} \label{1}
          \end{align}
     である.これは$n=1$でも成立する.よって
          \begin{align*}
          E(n)&=\sum_{k=1}^6kP(a_n=k) \\
          &=\sum_{k=1}^6\left(\frac{k}{6}\right)^n \\
          &\limit{n}{\infty} 1\cdots\text{(答)}
          \end{align*}
     となる.
     \item 期待値の加法定理および\eqref{1}から
          \begin{align*}
          N(n)&=\sum_{l=1}^nP(a_l=2) \\
          &=\sum_{l=1}^n\left(\frac{2}{6}\right)^{n-1}\frac{1}{6} \\
          &=\frac{1}{6}\frac{1-(1/3)^n}{1-1/3} \\
          &\limit{n}{\infty}\frac{1}{4}\cdots\text{(答)}
          \end{align*}
     となる.
     \end{enumerate}     

\newpage
\end{multicols}
\end{document}