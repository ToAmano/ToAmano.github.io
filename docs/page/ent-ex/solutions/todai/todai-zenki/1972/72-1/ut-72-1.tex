\documentclass[a4j]{jarticle}
\usepackage{amsmath}
\usepackage{ascmac}
\usepackage{amssymb}
\usepackage{enumerate}
\usepackage{multicol}
\usepackage{framed}
\usepackage{fancyhdr}
\usepackage{latexsym}
\usepackage{indent}
\usepackage{cases}
\usepackage[dvips]{graphicx}
\usepackage{color}
\allowdisplaybreaks
\pagestyle{fancy}
\lhead{}
\chead{}
\rhead{東京大学前期$1972$年$1$番}
\begin{document}
%分数関係


\def\tfrac#1#2{{\textstyle\frac{#1}{#2}}} %数式中で文中表示の分数を使う時


%Σ関係

\def\dsum#1#2{{\displaystyle\sum_{#1}^{#2}}} %文中で数式表示のΣを使う時


%ベクトル関係


\def\vector#1{\overrightarrow{#1}}  %ベクトルを表現したいとき(aベクトルを表現するときは\ver
\def\norm#1{|\overrightarrow{#1}|} %ベクトルの絶対値
\def\vtwo#1#2{ \left(%
      \begin{array}{c}%
      #1 \\%
      #2 \\%
      \end{array}%
      \right) }                        %2次元ベクトル成分表示
      
      \def\vthree#1#2#3{ \left(
      \begin{array}{c}
      #1 \\
      #2 \\
      #3 \\
      \end{array}
      \right) }                        %3次元ベクトル成分表示



%数列関係


\def\an#1{\verb|{|$#1$\verb|}|}


%極限関係

\def\limit#1#2{\stackrel{#1 \to #2}{\longrightarrow}}   %等式変形からの極限
\def\dlim#1#2{{\displaystyle \lim_{#1\to#2}}} %文中で数式表示の極限を使う



%積分関係

\def\dint#1#2{{\displaystyle \int_{#1}^{#2}}} %文中で数式表示の積分を使う時

\def\ne{\nearrow}
\def\se{\searrow}
\def\nw{\nwarrow}
\def\ne{\nearrow}


%便利なやつ

\def\case#1#2{%
 \[\left\{%
 \begin{array}{l}%
 #1 \\%
 #2%
 \end{array}%
 \right.\] }                           %場合分け
 
\def\1{$\cos\theta=c$,$\sin\theta=s$とおく.}  %cs表示を与える前書きシータ
\def\2{$\cos t=c$,$\sin t=s$とおく.}     %cs表示を与える前書きt
\def\3{$\cos x=c$,$\sin x=s$とおく.}                %cs表示を与える前書きx

\def\fig#1#2#3 {%
\begin{wrapfigure}[#1]{r}{#2 zw}%
\vspace*{-1zh}%
\input{#3}%
\end{wrapfigure} }           %絵の挿入


\def\a{\alpha}   %ギリシャ文字
\def\b{\beta}
\def\g{\gamma}

%問題番号のためのマクロ

\newcounter{nombre} %必須
\renewcommand{\thenombre}{\arabic{nombre}} %任意
\setcounter{nombre}{2} %任意
\newcounter{nombresub}[nombre] %親子関係を定義
\renewcommand{\thenombresub}{\arabic{nombresub}} %任意
\setcounter{nombresub}{0} %任意
\newcommand{\prob}[1][]{\refstepcounter{nombre}#1[問題 \thenombre]}
\newcommand{\probsub}[1][]{\refstepcounter{nombresub}#1(\thenombresub)}


%1-1みたいなカウンタ(todaiとtodaia)
\newcounter{todai}
\setcounter{todai}{0}
\newcounter{todaisub}[todai] 
\setcounter{todaisub}{0} 
\newcommand{\todai}[1][]{\refstepcounter{todai}#1 \thetodai-\thetodaisub}
\newcommand{\todaib}[1][]{\refstepcounter{todai}#1\refstepcounter{todaisub}#1 {\bf [問題 \thetodai.\thetodaisub]}}
\newcommand{\todaia}[1][]{\refstepcounter{todaisub}#1 {\bf [問題 \thetodai.\thetodaisub]}}


     \begin{oframed}
     空間に座標系が定められていて,$z$軸上に$2$点$A(0,0,6)$,$B(0,0,20)$が与えられている.$xy$平面上の点$P(x,y,0)$で,
     $0\le x\le15$,$0\le y\le15$,$\angle APB\ge30^\circ$を満たすものの全体がつくる図形の面積を求めよ.
     \end{oframed}

\setlength{\columnseprule}{0.4pt}
\begin{multicols}{2}
{\bf[解]} 対称性から,まず$P$が$x$軸上にある時を考える.
     \begin{center}
     \scalebox{0.8}{%WinTpicVersion4.32a
{\unitlength 0.1in%
\begin{picture}(26.1000,26.0000)(3.9000,-30.0000)%
% STR 2 0 3 0 Black White  
% 4 590 2797 590 2810 4 400 0 0
% O
\put(5.9000,-28.1000){\makebox(0,0)[rt]{O}}%
% STR 2 0 3 0 Black White  
% 4 560 387 560 400 4 0 0 0
% $z$
\put(5.6000,-4.0000){\makebox(0,0)[rt]{$z$}}%
% STR 2 0 3 0 Black White  
% 4 3000 2827 3000 2840 4 400 0 0
% $x$
\put(30.0000,-28.4000){\makebox(0,0)[rt]{$x$}}%
% VECTOR 2 0 3 0 Black White  
% 2 600 3000 600 400
% 
\special{pn 8}%
\special{pa 600 3000}%
\special{pa 600 400}%
\special{fp}%
\special{sh 1}%
\special{pa 600 400}%
\special{pa 580 467}%
\special{pa 600 453}%
\special{pa 620 467}%
\special{pa 600 400}%
\special{fp}%
% VECTOR 2 0 3 0 Black White  
% 2 400 2800 3000 2800
% 
\special{pn 8}%
\special{pa 400 2800}%
\special{pa 3000 2800}%
\special{fp}%
\special{sh 1}%
\special{pa 3000 2800}%
\special{pa 2933 2780}%
\special{pa 2947 2800}%
\special{pa 2933 2820}%
\special{pa 3000 2800}%
\special{fp}%
% FUNC 2 0 3 0 Black White  
% 9 400 400 3000 3000 600 2800 1400 2800 600 2000 600 400 600 3000 0 2 0 0
% 0
\special{pn 8}%
\special{pn 8}%
\special{pa 400 2800}%
\special{pa 499 2800}%
\special{fp}%
\special{fp}%
\special{pn 8}%
\special{pa 666 2800}%
\special{pa 733 2800}%
\special{fp}%
\special{pa 799 2800}%
\special{pa 867 2800}%
\special{fp}%
\special{pa 933 2800}%
\special{pa 1000 2800}%
\special{fp}%
\special{pa 1066 2800}%
\special{pa 1133 2800}%
\special{fp}%
\special{pa 1199 2800}%
\special{pa 1267 2800}%
\special{fp}%
\special{pa 1333 2800}%
\special{pa 1400 2800}%
\special{fp}%
\special{pa 1466 2800}%
\special{pa 1533 2800}%
\special{fp}%
\special{pa 1599 2800}%
\special{pa 1667 2800}%
\special{fp}%
\special{pa 1733 2800}%
\special{pa 1800 2800}%
\special{fp}%
\special{pa 1866 2800}%
\special{pa 1933 2800}%
\special{fp}%
\special{pa 1999 2800}%
\special{pa 2067 2800}%
\special{fp}%
\special{pa 2133 2800}%
\special{pa 2200 2800}%
\special{fp}%
\special{pa 2266 2800}%
\special{pa 2333 2800}%
\special{fp}%
\special{pa 2399 2800}%
\special{pa 2467 2800}%
\special{fp}%
\special{pa 2533 2800}%
\special{pa 2600 2800}%
\special{fp}%
\special{pa 2666 2800}%
\special{pa 2733 2800}%
\special{fp}%
\special{pa 2799 2800}%
\special{pa 2867 2800}%
\special{fp}%
\special{pa 2933 2800}%
\special{pa 3000 2800}%
\special{fp}%
% LINE 2 0 3 0 Black White  
% 2 600 1230 2040 2800
% 
\special{pn 8}%
\special{pa 600 1230}%
\special{pa 2040 2800}%
\special{fp}%
% LINE 2 0 3 0 Black White  
% 2 2040 2800 600 2240
% 
\special{pn 8}%
\special{pa 2040 2800}%
\special{pa 600 2240}%
\special{fp}%
% STR 2 0 3 0 Black White  
% 4 600 2140 600 2240 2 0 1 0
% $A$
\put(6.0000,-22.4000){\makebox(0,0)[lb]{{\colorbox[named]{White}{$A$}}}}%
% STR 2 0 3 0 Black White  
% 4 600 1140 600 1240 2 0 1 0
% $B$
\put(6.0000,-12.4000){\makebox(0,0)[lb]{{\colorbox[named]{White}{$B$}}}}%
% STR 2 0 3 0 Black White  
% 4 2040 2700 2040 2800 2 0 1 0
% $P$
\put(20.4000,-28.0000){\makebox(0,0)[lb]{{\colorbox[named]{White}{$P$}}}}%
\end{picture}}%
}
     \end{center}

上図において,($xz$平面)
     \begin{align*}
     &\vector{PA}=\vtwo{-x}{6}&\vector{PB}=\vtwo{-x}{20}
     \end{align*}
だから,$0\le\angle APB<\pi/2$とあわせて,
     \begin{align*}
     \tan\angle APB&=\frac{|-6x+20x|}{\vector{PA}\cdot\vector{PB}} \\
     &=\frac{14x}{120+x^2} \tag{$\because x\ge0$}
     \end{align*}
$\tan\theta$が$0\le\theta<\pi/2$で単調増加であることから,$\angle APB\ge30^\circ$のとき,
     \begin{align*}
     &\tan\frac{\pi}{6}\le\frac{14x}{120+x^2} \\
     \Longleftrightarrow &x^2-14\sqrt{3}x+120\ge0 \\
     \Longleftrightarrow &(x-10\sqrt{3})(x-4\sqrt{3})\ge0  \\
     \Longleftrightarrow&4\sqrt{3}\le x\le 10\sqrt{3}
     \end{align*}
したがって,$P$を$xy$平面で動かすと,
     \begin{align}
     16×3\le x^2+y^2\le100×3\label{1}
     \end{align}
となる.従って,求める領域は
      \begin{align}
      \eqref{1}\land(0\le x\le15)\land(0\le y\le15)
      \end{align}
であり,図示して右上図斜線部.
     \begin{center}
     \scalebox{0.7}{%WinTpicVersion4.32a
{\unitlength 0.1in%
\begin{picture}(34.2000,28.0000)(3.8000,-32.0000)%
% STR 2 0 3 0 Black White  
% 4 590 2997 590 3010 4 400 0 0
% O
\put(5.9000,-30.1000){\makebox(0,0)[rt]{O}}%
% STR 2 0 3 0 Black White  
% 4 560 387 560 400 4 400 0 0
% $y$
\put(5.6000,-4.0000){\makebox(0,0)[rt]{$y$}}%
% STR 2 0 3 0 Black White  
% 4 3800 3027 3800 3040 4 400 0 0
% $x$
\put(38.0000,-30.4000){\makebox(0,0)[rt]{$x$}}%
% VECTOR 2 0 3 0 Black White  
% 2 600 3200 600 400
% 
\special{pn 8}%
\special{pa 600 3200}%
\special{pa 600 400}%
\special{fp}%
\special{sh 1}%
\special{pa 600 400}%
\special{pa 580 467}%
\special{pa 600 453}%
\special{pa 620 467}%
\special{pa 600 400}%
\special{fp}%
% VECTOR 2 0 3 0 Black White  
% 2 400 3000 3800 3000
% 
\special{pn 8}%
\special{pa 400 3000}%
\special{pa 3800 3000}%
\special{fp}%
\special{sh 1}%
\special{pa 3800 3000}%
\special{pa 3733 2980}%
\special{pa 3747 3000}%
\special{pa 3733 3020}%
\special{pa 3800 3000}%
\special{fp}%
% FUNC 2 0 3 0 Black White  
% 9 400 400 3800 3200 600 3000 2600 3000 600 1000 400 400 3800 3200 50 4 0 2
% cos(t)///sin(t)///0///pi/2
\special{pn 8}%
\special{pa 2600 3000}%
\special{pa 2600 2956}%
\special{pa 2599 2950}%
\special{pa 2599 2925}%
\special{pa 2598 2918}%
\special{pa 2598 2906}%
\special{pa 2597 2899}%
\special{pa 2597 2887}%
\special{pa 2596 2881}%
\special{pa 2596 2868}%
\special{pa 2595 2862}%
\special{pa 2595 2856}%
\special{pa 2594 2849}%
\special{pa 2594 2843}%
\special{pa 2593 2837}%
\special{pa 2593 2830}%
\special{pa 2592 2824}%
\special{pa 2592 2818}%
\special{pa 2591 2812}%
\special{pa 2591 2805}%
\special{pa 2589 2793}%
\special{pa 2589 2787}%
\special{pa 2588 2780}%
\special{pa 2586 2768}%
\special{pa 2586 2762}%
\special{pa 2585 2756}%
\special{pa 2584 2749}%
\special{pa 2583 2743}%
\special{pa 2583 2737}%
\special{pa 2582 2731}%
\special{pa 2581 2724}%
\special{pa 2578 2706}%
\special{pa 2577 2699}%
\special{pa 2573 2675}%
\special{pa 2572 2668}%
\special{pa 2567 2638}%
\special{pa 2566 2631}%
\special{pa 2565 2625}%
\special{pa 2563 2619}%
\special{pa 2560 2601}%
\special{pa 2558 2594}%
\special{pa 2555 2576}%
\special{pa 2553 2570}%
\special{pa 2552 2564}%
\special{pa 2550 2558}%
\special{pa 2549 2551}%
\special{pa 2548 2545}%
\special{pa 2546 2539}%
\special{pa 2545 2533}%
\special{pa 2543 2527}%
\special{pa 2542 2521}%
\special{pa 2540 2515}%
\special{pa 2539 2509}%
\special{pa 2537 2503}%
\special{pa 2536 2496}%
\special{pa 2532 2484}%
\special{pa 2531 2478}%
\special{pa 2527 2466}%
\special{pa 2526 2460}%
\special{pa 2522 2448}%
\special{pa 2521 2442}%
\special{pa 2513 2418}%
\special{pa 2512 2412}%
\special{pa 2482 2322}%
\special{pa 2480 2317}%
\special{pa 2477 2311}%
\special{pa 2469 2287}%
\special{pa 2466 2281}%
\special{pa 2464 2275}%
\special{pa 2462 2270}%
\special{pa 2460 2264}%
\special{pa 2457 2258}%
\special{pa 2453 2246}%
\special{pa 2450 2240}%
\special{pa 2448 2235}%
\special{pa 2445 2229}%
\special{pa 2443 2223}%
\special{pa 2440 2217}%
\special{pa 2438 2211}%
\special{pa 2435 2206}%
\special{pa 2433 2200}%
\special{pa 2430 2194}%
\special{pa 2428 2188}%
\special{pa 2425 2183}%
\special{pa 2423 2177}%
\special{pa 2420 2171}%
\special{pa 2418 2165}%
\special{pa 2415 2160}%
\special{pa 2412 2154}%
\special{pa 2410 2148}%
\special{pa 2407 2143}%
\special{pa 2404 2137}%
\special{pa 2402 2131}%
\special{pa 2399 2126}%
\special{pa 2393 2114}%
\special{pa 2390 2109}%
\special{pa 2388 2103}%
\special{pa 2385 2098}%
\special{pa 2379 2086}%
\special{pa 2376 2081}%
\special{pa 2373 2075}%
\special{pa 2370 2070}%
\special{pa 2368 2064}%
\special{pa 2365 2059}%
\special{pa 2359 2047}%
\special{pa 2356 2042}%
\special{pa 2353 2036}%
\special{pa 2350 2031}%
\special{pa 2346 2025}%
\special{pa 2343 2020}%
\special{pa 2340 2014}%
\special{pa 2334 2004}%
\special{pa 2331 1998}%
\special{pa 2328 1993}%
\special{pa 2325 1987}%
\special{pa 2321 1982}%
\special{pa 2318 1976}%
\special{pa 2312 1966}%
\special{pa 2309 1960}%
\special{pa 2305 1955}%
\special{pa 2302 1950}%
\special{pa 2299 1944}%
\special{pa 2295 1939}%
\special{pa 2292 1934}%
\special{pa 2289 1928}%
\special{pa 2285 1923}%
\special{pa 2282 1918}%
\special{pa 2278 1912}%
\special{pa 2272 1902}%
\special{pa 2268 1897}%
\special{pa 2265 1891}%
\special{pa 2261 1886}%
\special{pa 2258 1881}%
\special{pa 2254 1876}%
\special{pa 2251 1871}%
\special{pa 2247 1865}%
\special{pa 2243 1860}%
\special{pa 2240 1855}%
\special{pa 2236 1850}%
\special{pa 2233 1845}%
\special{pa 2225 1835}%
\special{pa 2222 1829}%
\special{pa 2214 1819}%
\special{pa 2211 1814}%
\special{pa 2199 1799}%
\special{pa 2196 1794}%
\special{pa 2176 1769}%
\special{pa 2173 1764}%
\special{pa 2161 1749}%
\special{pa 2157 1745}%
\special{pa 2133 1715}%
\special{pa 2129 1711}%
\special{pa 2117 1696}%
\special{pa 2108 1687}%
\special{pa 2100 1677}%
\special{pa 2096 1673}%
\special{pa 2088 1663}%
\special{pa 2083 1659}%
\special{pa 2075 1649}%
\special{pa 2066 1640}%
\special{pa 2062 1635}%
\special{pa 2058 1631}%
\special{pa 2054 1626}%
\special{pa 2049 1622}%
\special{pa 2045 1617}%
\special{pa 2023 1595}%
\special{pa 2019 1590}%
\special{pa 2014 1586}%
\special{pa 2010 1581}%
\special{pa 2005 1577}%
\special{pa 2001 1572}%
\special{pa 1996 1568}%
\special{pa 1983 1555}%
\special{pa 1978 1551}%
\special{pa 1974 1546}%
\special{pa 1964 1538}%
\special{pa 1960 1533}%
\special{pa 1955 1529}%
\special{pa 1951 1525}%
\special{pa 1946 1521}%
\special{pa 1937 1512}%
\special{pa 1927 1504}%
\special{pa 1923 1500}%
\special{pa 1918 1496}%
\special{pa 1913 1491}%
\special{pa 1908 1487}%
\special{pa 1904 1483}%
\special{pa 1884 1467}%
\special{pa 1880 1463}%
\special{pa 1850 1439}%
\special{pa 1846 1435}%
\special{pa 1836 1427}%
\special{pa 1831 1424}%
\special{pa 1806 1404}%
\special{pa 1801 1401}%
\special{pa 1786 1389}%
\special{pa 1781 1386}%
\special{pa 1776 1382}%
\special{pa 1770 1378}%
\special{pa 1765 1375}%
\special{pa 1755 1367}%
\special{pa 1750 1364}%
\special{pa 1740 1356}%
\special{pa 1734 1353}%
\special{pa 1729 1349}%
\special{pa 1724 1346}%
\special{pa 1719 1342}%
\special{pa 1714 1339}%
\special{pa 1708 1335}%
\special{pa 1703 1332}%
\special{pa 1698 1328}%
\special{pa 1693 1325}%
\special{pa 1687 1321}%
\special{pa 1677 1315}%
\special{pa 1672 1311}%
\special{pa 1666 1308}%
\special{pa 1661 1305}%
\special{pa 1656 1301}%
\special{pa 1650 1298}%
\special{pa 1645 1295}%
\special{pa 1640 1291}%
\special{pa 1634 1288}%
\special{pa 1629 1285}%
\special{pa 1623 1282}%
\special{pa 1618 1278}%
\special{pa 1613 1275}%
\special{pa 1607 1272}%
\special{pa 1602 1269}%
\special{pa 1596 1266}%
\special{pa 1591 1263}%
\special{pa 1585 1260}%
\special{pa 1580 1257}%
\special{pa 1574 1253}%
\special{pa 1569 1250}%
\special{pa 1563 1247}%
\special{pa 1558 1244}%
\special{pa 1552 1241}%
\special{pa 1547 1238}%
\special{pa 1541 1235}%
\special{pa 1536 1232}%
\special{pa 1530 1230}%
\special{pa 1525 1227}%
\special{pa 1519 1224}%
\special{pa 1514 1221}%
\special{pa 1502 1215}%
\special{pa 1497 1212}%
\special{pa 1491 1209}%
\special{pa 1485 1207}%
\special{pa 1480 1204}%
\special{pa 1474 1201}%
\special{pa 1469 1198}%
\special{pa 1463 1196}%
\special{pa 1457 1193}%
\special{pa 1452 1190}%
\special{pa 1446 1188}%
\special{pa 1434 1182}%
\special{pa 1429 1180}%
\special{pa 1423 1177}%
\special{pa 1417 1175}%
\special{pa 1412 1172}%
\special{pa 1406 1169}%
\special{pa 1400 1167}%
\special{pa 1394 1164}%
\special{pa 1388 1162}%
\special{pa 1383 1160}%
\special{pa 1377 1157}%
\special{pa 1371 1155}%
\special{pa 1365 1152}%
\special{pa 1360 1150}%
\special{pa 1354 1147}%
\special{pa 1342 1143}%
\special{pa 1336 1140}%
\special{pa 1324 1136}%
\special{pa 1319 1134}%
\special{pa 1313 1131}%
\special{pa 1289 1123}%
\special{pa 1283 1120}%
\special{pa 1277 1118}%
\special{pa 1272 1116}%
\special{pa 1188 1088}%
\special{pa 1182 1087}%
\special{pa 1158 1079}%
\special{pa 1152 1078}%
\special{pa 1140 1074}%
\special{pa 1134 1073}%
\special{pa 1122 1069}%
\special{pa 1116 1068}%
\special{pa 1109 1066}%
\special{pa 1103 1064}%
\special{pa 1097 1063}%
\special{pa 1091 1061}%
\special{pa 1085 1060}%
\special{pa 1079 1058}%
\special{pa 1073 1057}%
\special{pa 1067 1055}%
\special{pa 1061 1054}%
\special{pa 1055 1052}%
\special{pa 1048 1051}%
\special{pa 1042 1050}%
\special{pa 1036 1048}%
\special{pa 1030 1047}%
\special{pa 1024 1045}%
\special{pa 1006 1042}%
\special{pa 999 1040}%
\special{pa 981 1037}%
\special{pa 975 1035}%
\special{pa 969 1034}%
\special{pa 962 1033}%
\special{pa 938 1029}%
\special{pa 931 1028}%
\special{pa 907 1024}%
\special{pa 900 1023}%
\special{pa 876 1019}%
\special{pa 869 1018}%
\special{pa 863 1017}%
\special{pa 857 1017}%
\special{pa 851 1016}%
\special{pa 844 1015}%
\special{pa 832 1013}%
\special{pa 826 1013}%
\special{pa 819 1012}%
\special{pa 813 1011}%
\special{pa 807 1011}%
\special{pa 801 1010}%
\special{pa 794 1009}%
\special{pa 788 1009}%
\special{pa 782 1008}%
\special{pa 776 1008}%
\special{pa 769 1007}%
\special{pa 763 1007}%
\special{pa 757 1006}%
\special{pa 751 1006}%
\special{pa 744 1005}%
\special{pa 738 1005}%
\special{pa 732 1004}%
\special{pa 719 1004}%
\special{pa 713 1003}%
\special{pa 700 1003}%
\special{pa 694 1002}%
\special{pa 682 1002}%
\special{pa 675 1001}%
\special{pa 650 1001}%
\special{pa 644 1000}%
\special{pa 600 1000}%
\special{fp}%
% LINE 2 2 3 0 Black White  
% 8 600 1400 2200 1400 2200 1400 2200 3000 600 3000 2200 1800 1800 1400 600 3000
% 
\special{pn 8}%
\special{pa 600 1400}%
\special{pa 2200 1400}%
\special{dt 0.045}%
\special{pa 2200 1400}%
\special{pa 2200 3000}%
\special{dt 0.045}%
\special{pa 600 3000}%
\special{pa 2200 1800}%
\special{dt 0.045}%
\special{pa 1800 1400}%
\special{pa 600 3000}%
\special{dt 0.045}%
% STR 2 0 3 0 Black White  
% 4 2200 2900 2200 3000 2 0 1 0
% 15
\put(22.0000,-30.0000){\makebox(0,0)[lb]{{\colorbox[named]{White}{15}}}}%
% STR 2 0 3 0 Black White  
% 4 2600 2900 2600 3000 2 0 1 0
% $10\sqrt{3}$
\put(26.0000,-30.0000){\makebox(0,0)[lb]{{\colorbox[named]{White}{$10\sqrt{3}$}}}}%
% FUNC 2 0 3 0 Black White  
% 9 400 400 3800 3200 600 3000 2600 3000 600 1000 400 400 3800 3200 50 4 0 2
% cos(t)/2///sin(t)/2///0///pi/2
\special{pn 8}%
\special{pa 1600 3000}%
\special{pa 1600 2969}%
\special{pa 1599 2965}%
\special{pa 1599 2947}%
\special{pa 1598 2943}%
\special{pa 1598 2931}%
\special{pa 1597 2928}%
\special{pa 1597 2918}%
\special{pa 1596 2915}%
\special{pa 1596 2906}%
\special{pa 1595 2903}%
\special{pa 1595 2896}%
\special{pa 1594 2893}%
\special{pa 1594 2887}%
\special{pa 1593 2884}%
\special{pa 1593 2881}%
\special{pa 1592 2878}%
\special{pa 1592 2872}%
\special{pa 1591 2868}%
\special{pa 1591 2865}%
\special{pa 1590 2862}%
\special{pa 1590 2856}%
\special{pa 1589 2853}%
\special{pa 1589 2850}%
\special{pa 1588 2847}%
\special{pa 1588 2844}%
\special{pa 1587 2840}%
\special{pa 1587 2837}%
\special{pa 1586 2834}%
\special{pa 1586 2831}%
\special{pa 1585 2828}%
\special{pa 1585 2825}%
\special{pa 1583 2819}%
\special{pa 1583 2816}%
\special{pa 1582 2813}%
\special{pa 1582 2810}%
\special{pa 1581 2806}%
\special{pa 1580 2803}%
\special{pa 1580 2800}%
\special{pa 1579 2797}%
\special{pa 1579 2794}%
\special{pa 1577 2788}%
\special{pa 1577 2785}%
\special{pa 1575 2779}%
\special{pa 1575 2776}%
\special{pa 1572 2767}%
\special{pa 1572 2763}%
\special{pa 1569 2754}%
\special{pa 1569 2751}%
\special{pa 1565 2739}%
\special{pa 1565 2736}%
\special{pa 1559 2718}%
\special{pa 1559 2715}%
\special{pa 1537 2649}%
\special{pa 1535 2646}%
\special{pa 1534 2644}%
\special{pa 1529 2629}%
\special{pa 1527 2626}%
\special{pa 1523 2614}%
\special{pa 1521 2611}%
\special{pa 1520 2609}%
\special{pa 1518 2603}%
\special{pa 1516 2600}%
\special{pa 1513 2591}%
\special{pa 1511 2588}%
\special{pa 1510 2586}%
\special{pa 1509 2583}%
\special{pa 1507 2580}%
\special{pa 1505 2574}%
\special{pa 1503 2571}%
\special{pa 1502 2569}%
\special{pa 1501 2566}%
\special{pa 1499 2563}%
\special{pa 1497 2557}%
\special{pa 1495 2554}%
\special{pa 1494 2552}%
\special{pa 1492 2549}%
\special{pa 1490 2543}%
\special{pa 1488 2540}%
\special{pa 1487 2538}%
\special{pa 1485 2535}%
\special{pa 1484 2532}%
\special{pa 1482 2529}%
\special{pa 1481 2526}%
\special{pa 1479 2524}%
\special{pa 1478 2521}%
\special{pa 1476 2518}%
\special{pa 1475 2515}%
\special{pa 1473 2513}%
\special{pa 1472 2510}%
\special{pa 1470 2507}%
\special{pa 1469 2505}%
\special{pa 1465 2499}%
\special{pa 1464 2496}%
\special{pa 1462 2494}%
\special{pa 1461 2491}%
\special{pa 1459 2488}%
\special{pa 1458 2486}%
\special{pa 1454 2480}%
\special{pa 1453 2477}%
\special{pa 1451 2475}%
\special{pa 1449 2472}%
\special{pa 1448 2469}%
\special{pa 1446 2467}%
\special{pa 1444 2464}%
\special{pa 1443 2461}%
\special{pa 1441 2459}%
\special{pa 1439 2456}%
\special{pa 1438 2454}%
\special{pa 1434 2448}%
\special{pa 1432 2446}%
\special{pa 1431 2443}%
\special{pa 1429 2440}%
\special{pa 1427 2438}%
\special{pa 1425 2435}%
\special{pa 1424 2433}%
\special{pa 1422 2430}%
\special{pa 1420 2428}%
\special{pa 1416 2422}%
\special{pa 1415 2420}%
\special{pa 1413 2417}%
\special{pa 1411 2415}%
\special{pa 1409 2412}%
\special{pa 1407 2410}%
\special{pa 1405 2407}%
\special{pa 1403 2405}%
\special{pa 1402 2402}%
\special{pa 1400 2400}%
\special{pa 1398 2397}%
\special{pa 1396 2395}%
\special{pa 1394 2392}%
\special{pa 1392 2390}%
\special{pa 1390 2387}%
\special{pa 1388 2385}%
\special{pa 1386 2382}%
\special{pa 1384 2380}%
\special{pa 1382 2377}%
\special{pa 1380 2375}%
\special{pa 1378 2372}%
\special{pa 1376 2370}%
\special{pa 1374 2367}%
\special{pa 1370 2363}%
\special{pa 1368 2360}%
\special{pa 1366 2358}%
\special{pa 1364 2355}%
\special{pa 1360 2351}%
\special{pa 1358 2348}%
\special{pa 1356 2346}%
\special{pa 1354 2343}%
\special{pa 1350 2339}%
\special{pa 1348 2336}%
\special{pa 1344 2332}%
\special{pa 1342 2329}%
\special{pa 1340 2327}%
\special{pa 1337 2325}%
\special{pa 1335 2322}%
\special{pa 1331 2318}%
\special{pa 1329 2315}%
\special{pa 1325 2311}%
\special{pa 1322 2309}%
\special{pa 1320 2306}%
\special{pa 1314 2300}%
\special{pa 1312 2297}%
\special{pa 1309 2295}%
\special{pa 1305 2291}%
\special{pa 1303 2288}%
\special{pa 1300 2286}%
\special{pa 1285 2271}%
\special{pa 1282 2269}%
\special{pa 1271 2258}%
\special{pa 1268 2256}%
\special{pa 1264 2252}%
\special{pa 1261 2250}%
\special{pa 1257 2246}%
\special{pa 1254 2244}%
\special{pa 1252 2242}%
\special{pa 1249 2240}%
\special{pa 1240 2231}%
\special{pa 1237 2229}%
\special{pa 1233 2225}%
\special{pa 1230 2223}%
\special{pa 1228 2222}%
\special{pa 1225 2220}%
\special{pa 1223 2218}%
\special{pa 1220 2216}%
\special{pa 1218 2214}%
\special{pa 1215 2212}%
\special{pa 1213 2210}%
\special{pa 1210 2208}%
\special{pa 1208 2206}%
\special{pa 1205 2204}%
\special{pa 1203 2202}%
\special{pa 1200 2200}%
\special{pa 1198 2198}%
\special{pa 1195 2197}%
\special{pa 1193 2195}%
\special{pa 1190 2193}%
\special{pa 1188 2191}%
\special{pa 1185 2189}%
\special{pa 1183 2187}%
\special{pa 1180 2185}%
\special{pa 1178 2184}%
\special{pa 1172 2180}%
\special{pa 1170 2178}%
\special{pa 1167 2176}%
\special{pa 1165 2175}%
\special{pa 1159 2171}%
\special{pa 1157 2169}%
\special{pa 1154 2168}%
\special{pa 1152 2166}%
\special{pa 1146 2162}%
\special{pa 1144 2161}%
\special{pa 1138 2157}%
\special{pa 1136 2156}%
\special{pa 1130 2152}%
\special{pa 1128 2151}%
\special{pa 1122 2147}%
\special{pa 1120 2146}%
\special{pa 1114 2142}%
\special{pa 1112 2141}%
\special{pa 1109 2139}%
\special{pa 1106 2138}%
\special{pa 1104 2136}%
\special{pa 1101 2134}%
\special{pa 1098 2133}%
\special{pa 1095 2131}%
\special{pa 1093 2130}%
\special{pa 1090 2128}%
\special{pa 1087 2127}%
\special{pa 1084 2125}%
\special{pa 1082 2124}%
\special{pa 1079 2122}%
\special{pa 1076 2121}%
\special{pa 1073 2119}%
\special{pa 1071 2118}%
\special{pa 1068 2116}%
\special{pa 1065 2115}%
\special{pa 1062 2113}%
\special{pa 1060 2112}%
\special{pa 1057 2110}%
\special{pa 1051 2108}%
\special{pa 1048 2106}%
\special{pa 1046 2105}%
\special{pa 1043 2103}%
\special{pa 1037 2101}%
\special{pa 1034 2099}%
\special{pa 1031 2098}%
\special{pa 1029 2097}%
\special{pa 1026 2095}%
\special{pa 1020 2093}%
\special{pa 1017 2091}%
\special{pa 1011 2089}%
\special{pa 1009 2087}%
\special{pa 1003 2085}%
\special{pa 1000 2083}%
\special{pa 988 2079}%
\special{pa 986 2077}%
\special{pa 974 2073}%
\special{pa 971 2071}%
\special{pa 953 2065}%
\special{pa 951 2063}%
\special{pa 885 2041}%
\special{pa 882 2041}%
\special{pa 864 2035}%
\special{pa 861 2035}%
\special{pa 849 2031}%
\special{pa 846 2031}%
\special{pa 840 2029}%
\special{pa 836 2028}%
\special{pa 833 2028}%
\special{pa 824 2025}%
\special{pa 821 2025}%
\special{pa 815 2023}%
\special{pa 812 2023}%
\special{pa 806 2021}%
\special{pa 803 2021}%
\special{pa 800 2020}%
\special{pa 797 2020}%
\special{pa 794 2019}%
\special{pa 790 2018}%
\special{pa 787 2018}%
\special{pa 784 2017}%
\special{pa 781 2017}%
\special{pa 775 2015}%
\special{pa 772 2015}%
\special{pa 769 2014}%
\special{pa 766 2014}%
\special{pa 763 2013}%
\special{pa 760 2013}%
\special{pa 756 2012}%
\special{pa 753 2012}%
\special{pa 750 2011}%
\special{pa 747 2011}%
\special{pa 744 2010}%
\special{pa 738 2010}%
\special{pa 735 2009}%
\special{pa 732 2009}%
\special{pa 728 2008}%
\special{pa 725 2008}%
\special{pa 722 2007}%
\special{pa 716 2007}%
\special{pa 713 2006}%
\special{pa 707 2006}%
\special{pa 703 2005}%
\special{pa 697 2005}%
\special{pa 694 2004}%
\special{pa 685 2004}%
\special{pa 682 2003}%
\special{pa 672 2003}%
\special{pa 669 2002}%
\special{pa 656 2002}%
\special{pa 653 2001}%
\special{pa 635 2001}%
\special{pa 631 2000}%
\special{pa 600 2000}%
\special{fp}%
% STR 2 0 3 0 Black White  
% 4 800 2900 800 3000 2 0 1 0
% $pi/6$
\put(8.0000,-30.0000){\makebox(0,0)[lb]{{\colorbox[named]{White}{$pi/6$}}}}%
% LINE 3 0 3 0 Black White  
% 46 2200 2060 1560 2700 2200 2120 1570 2750 2200 2180 1580 2800 2200 2240 1590 2850 2200 2300 1600 2900 2200 2360 1600 2960 2200 2420 1620 3000 2200 2480 1680 3000 2200 2540 1740 3000 2200 2600 1800 3000 2200 2660 1860 3000 2200 2720 1920 3000 2200 2780 1980 3000 2200 2840 2040 3000 2200 2900 2100 3000 2200 2960 2160 3000 2200 2000 1540 2660 2200 1940 1530 2610 2200 1880 1510 2570 2200 1820 1490 2530 2030 1930 1470 2490 1790 2110 1440 2460 1550 2290 1420 2420
% 
\special{pn 4}%
\special{pa 2200 2060}%
\special{pa 1560 2700}%
\special{fp}%
\special{pa 2200 2120}%
\special{pa 1570 2750}%
\special{fp}%
\special{pa 2200 2180}%
\special{pa 1580 2800}%
\special{fp}%
\special{pa 2200 2240}%
\special{pa 1590 2850}%
\special{fp}%
\special{pa 2200 2300}%
\special{pa 1600 2900}%
\special{fp}%
\special{pa 2200 2360}%
\special{pa 1600 2960}%
\special{fp}%
\special{pa 2200 2420}%
\special{pa 1620 3000}%
\special{fp}%
\special{pa 2200 2480}%
\special{pa 1680 3000}%
\special{fp}%
\special{pa 2200 2540}%
\special{pa 1740 3000}%
\special{fp}%
\special{pa 2200 2600}%
\special{pa 1800 3000}%
\special{fp}%
\special{pa 2200 2660}%
\special{pa 1860 3000}%
\special{fp}%
\special{pa 2200 2720}%
\special{pa 1920 3000}%
\special{fp}%
\special{pa 2200 2780}%
\special{pa 1980 3000}%
\special{fp}%
\special{pa 2200 2840}%
\special{pa 2040 3000}%
\special{fp}%
\special{pa 2200 2900}%
\special{pa 2100 3000}%
\special{fp}%
\special{pa 2200 2960}%
\special{pa 2160 3000}%
\special{fp}%
\special{pa 2200 2000}%
\special{pa 1540 2660}%
\special{fp}%
\special{pa 2200 1940}%
\special{pa 1530 2610}%
\special{fp}%
\special{pa 2200 1880}%
\special{pa 1510 2570}%
\special{fp}%
\special{pa 2200 1820}%
\special{pa 1490 2530}%
\special{fp}%
\special{pa 2030 1930}%
\special{pa 1470 2490}%
\special{fp}%
\special{pa 1790 2110}%
\special{pa 1440 2460}%
\special{fp}%
\special{pa 1550 2290}%
\special{pa 1420 2420}%
\special{fp}%
% LINE 3 0 3 0 Black White  
% 26 2010 1590 1310 2290 2040 1620 1340 2320 2070 1650 1370 2350 2100 1680 1390 2390 2130 1710 1580 2260 2160 1740 1820 2080 2180 1780 2060 1900 1980 1560 1280 2260 1950 1530 1250 2230 1920 1500 1210 2210 1890 1470 1330 2030 1860 1440 1510 1790 1820 1420 1690 1550
% 
\special{pn 4}%
\special{pa 2010 1590}%
\special{pa 1310 2290}%
\special{fp}%
\special{pa 2040 1620}%
\special{pa 1340 2320}%
\special{fp}%
\special{pa 2070 1650}%
\special{pa 1370 2350}%
\special{fp}%
\special{pa 2100 1680}%
\special{pa 1390 2390}%
\special{fp}%
\special{pa 2130 1710}%
\special{pa 1580 2260}%
\special{fp}%
\special{pa 2160 1740}%
\special{pa 1820 2080}%
\special{fp}%
\special{pa 2180 1780}%
\special{pa 2060 1900}%
\special{fp}%
\special{pa 1980 1560}%
\special{pa 1280 2260}%
\special{fp}%
\special{pa 1950 1530}%
\special{pa 1250 2230}%
\special{fp}%
\special{pa 1920 1500}%
\special{pa 1210 2210}%
\special{fp}%
\special{pa 1890 1470}%
\special{pa 1330 2030}%
\special{fp}%
\special{pa 1860 1440}%
\special{pa 1510 1790}%
\special{fp}%
\special{pa 1820 1420}%
\special{pa 1690 1550}%
\special{fp}%
% LINE 3 0 3 0 Black White  
% 46 1420 1400 800 2020 1480 1400 850 2030 1540 1400 900 2040 1600 1400 940 2060 1660 1400 990 2070 1720 1400 1030 2090 1780 1400 1070 2110 1660 1580 1110 2130 1480 1820 1140 2160 1300 2060 1180 2180 1360 1400 750 2010 1300 1400 700 2000 1240 1400 640 2000 1180 1400 600 1980 1120 1400 600 1920 1060 1400 600 1860 1000 1400 600 1800 940 1400 600 1740 880 1400 600 1680 820 1400 600 1620 760 1400 600 1560 700 1400 600 1500 640 1400 600 1440
% 
\special{pn 4}%
\special{pa 1420 1400}%
\special{pa 800 2020}%
\special{fp}%
\special{pa 1480 1400}%
\special{pa 850 2030}%
\special{fp}%
\special{pa 1540 1400}%
\special{pa 900 2040}%
\special{fp}%
\special{pa 1600 1400}%
\special{pa 940 2060}%
\special{fp}%
\special{pa 1660 1400}%
\special{pa 990 2070}%
\special{fp}%
\special{pa 1720 1400}%
\special{pa 1030 2090}%
\special{fp}%
\special{pa 1780 1400}%
\special{pa 1070 2110}%
\special{fp}%
\special{pa 1660 1580}%
\special{pa 1110 2130}%
\special{fp}%
\special{pa 1480 1820}%
\special{pa 1140 2160}%
\special{fp}%
\special{pa 1300 2060}%
\special{pa 1180 2180}%
\special{fp}%
\special{pa 1360 1400}%
\special{pa 750 2010}%
\special{fp}%
\special{pa 1300 1400}%
\special{pa 700 2000}%
\special{fp}%
\special{pa 1240 1400}%
\special{pa 640 2000}%
\special{fp}%
\special{pa 1180 1400}%
\special{pa 600 1980}%
\special{fp}%
\special{pa 1120 1400}%
\special{pa 600 1920}%
\special{fp}%
\special{pa 1060 1400}%
\special{pa 600 1860}%
\special{fp}%
\special{pa 1000 1400}%
\special{pa 600 1800}%
\special{fp}%
\special{pa 940 1400}%
\special{pa 600 1740}%
\special{fp}%
\special{pa 880 1400}%
\special{pa 600 1680}%
\special{fp}%
\special{pa 820 1400}%
\special{pa 600 1620}%
\special{fp}%
\special{pa 760 1400}%
\special{pa 600 1560}%
\special{fp}%
\special{pa 700 1400}%
\special{pa 600 1500}%
\special{fp}%
\special{pa 640 1400}%
\special{pa 600 1440}%
\special{fp}%
% STR 2 0 3 0 Black White  
% 4 1600 2900 1600 3000 2 0 1 0
% $4\sqrt{3}$
\put(16.0000,-30.0000){\makebox(0,0)[lb]{{\colorbox[named]{White}{$4\sqrt{3}$}}}}%
% STR 2 0 3 0 Black White  
% 4 600 1300 600 1400 2 0 1 0
% $15$
\put(6.0000,-14.0000){\makebox(0,0)[lb]{{\colorbox[named]{White}{$15$}}}}%
\end{picture}}%
}
     \end{center}

この面積$S$は 
     \begin{align*}
     S&=\frac{1}{2}\frac{\pi}{6}(10\sqrt{3})^2+15×5\sqrt{3}-\frac{1}{2}\frac{\pi}{2}(4\sqrt{3})^2 \\
     &=13\pi+75\sqrt{3}
     \end{align*}
である.$\cdots$(答)
     

     
\newpage
\end{multicols}
\end{document}