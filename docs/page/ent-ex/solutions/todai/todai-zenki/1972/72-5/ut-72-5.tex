\documentclass[a4j]{jarticle}
\usepackage{amsmath}
\usepackage{ascmac}
\usepackage{amssymb}
\usepackage{enumerate}
\usepackage{multicol}
\usepackage{framed}
\usepackage{fancyhdr}
\usepackage{latexsym}
\usepackage{indent}
\usepackage{cases}
\allowdisplaybreaks
\pagestyle{fancy}
\lhead{}
\chead{}
\rhead{東京大学前期$1972$年$5$番}
\begin{document}
%分数関係


\def\tfrac#1#2{{\textstyle\frac{#1}{#2}}} %数式中で文中表示の分数を使う時


%Σ関係

\def\dsum#1#2{{\displaystyle\sum_{#1}^{#2}}} %文中で数式表示のΣを使う時


%ベクトル関係


\def\vector#1{\overrightarrow{#1}}  %ベクトルを表現したいとき(aベクトルを表現するときは\ver
\def\norm#1{|\overrightarrow{#1}|} %ベクトルの絶対値
\def\vtwo#1#2{ \left(%
      \begin{array}{c}%
      #1 \\%
      #2 \\%
      \end{array}%
      \right) }                        %2次元ベクトル成分表示
      
      \def\vthree#1#2#3{ \left(
      \begin{array}{c}
      #1 \\
      #2 \\
      #3 \\
      \end{array}
      \right) }                        %3次元ベクトル成分表示



%数列関係


\def\an#1{\verb|{|$#1$\verb|}|}


%極限関係

\def\limit#1#2{\stackrel{#1 \to #2}{\longrightarrow}}   %等式変形からの極限
\def\dlim#1#2{{\displaystyle \lim_{#1\to#2}}} %文中で数式表示の極限を使う



%積分関係

\def\dint#1#2{{\displaystyle \int_{#1}^{#2}}} %文中で数式表示の積分を使う時

\def\ne{\nearrow}
\def\se{\searrow}
\def\nw{\nwarrow}
\def\ne{\nearrow}


%便利なやつ

\def\case#1#2{%
 \[\left\{%
 \begin{array}{l}%
 #1 \\%
 #2%
 \end{array}%
 \right.\] }                           %場合分け
 
\def\1{$\cos\theta=c$,$\sin\theta=s$とおく.}  %cs表示を与える前書きシータ
\def\2{$\cos t=c$,$\sin t=s$とおく.}     %cs表示を与える前書きt
\def\3{$\cos x=c$,$\sin x=s$とおく.}                %cs表示を与える前書きx

\def\fig#1#2#3 {%
\begin{wrapfigure}[#1]{r}{#2 zw}%
\vspace*{-1zh}%
\input{#3}%
\end{wrapfigure} }           %絵の挿入


\def\a{\alpha}   %ギリシャ文字
\def\b{\beta}
\def\g{\gamma}

%問題番号のためのマクロ

\newcounter{nombre} %必須
\renewcommand{\thenombre}{\arabic{nombre}} %任意
\setcounter{nombre}{2} %任意
\newcounter{nombresub}[nombre] %親子関係を定義
\renewcommand{\thenombresub}{\arabic{nombresub}} %任意
\setcounter{nombresub}{0} %任意
\newcommand{\prob}[1][]{\refstepcounter{nombre}#1[問題 \thenombre]}
\newcommand{\probsub}[1][]{\refstepcounter{nombresub}#1(\thenombresub)}


%1-1みたいなカウンタ(todaiとtodaia)
\newcounter{todai}
\setcounter{todai}{0}
\newcounter{todaisub}[todai] 
\setcounter{todaisub}{0} 
\newcommand{\todai}[1][]{\refstepcounter{todai}#1 \thetodai-\thetodaisub}
\newcommand{\todaib}[1][]{\refstepcounter{todai}#1\refstepcounter{todaisub}#1 {\bf [問題 \thetodai.\thetodaisub]}}
\newcommand{\todaia}[1][]{\refstepcounter{todaisub}#1 {\bf [問題 \thetodai.\thetodaisub]}}


     \begin{oframed}
     $h(x)$は$-\infty<x<\infty$で$2$回微分可能なある関数で,$f(x)$がどのような一次関数であって
     も,$u(x)=\dint{o}{x}h(t)f(t)dt+h(x)\dint{x}{1}f(t)dt$とおけば,
          \begin{enumerate}[(1)]
          \item $\dfrac{d^2u}{dx^2}=f(x)$
          \end{enumerate}
     および
          \begin{enumerate}[(2)]
          \item $u(0)=0$
          \end{enumerate}
     が成り立つ.このとき,$h(x)$を求めよ.     
     \end{oframed}

\setlength{\columnseprule}{0.4pt}
\begin{multicols}{2}
{\bf[解]}$f(x)$を$f$などと略記する.題意の条件から
     \begin{numcases}
     {}
     u=\dint{o}{x}hfdt+h\dint{x}{1}fdt \label{1}\\
     u''=f \label{2}\\
     u=0\label{3}
     \end{numcases}
まず,\eqref{1}で$x=0$として,\eqref{3}から
     \begin{align*}
     0=h(0)\dint{x}{1}fdt
     \end{align*}
ここで$f$は任意の一次関数だから$\dint{x}{1}fdt\equiv0$とはならないので(例えば$f=x$)
     \begin{align}
     h(0)=0\label{0}
     \end{align}
である.

 \eqref{1}の両辺を$x$で微分する.
     \begin{align*}
     u'&=hf+h'\int_x^1fdt-hf  \\
     &=h'\int_x^1fdt
     \end{align*}
 さらに微分して
      \begin{align*}
      u''=h''\int_x^1fdt-h'f
      \end{align*}
\eqref{2}を代入して
     \begin{align}
     \{1+h'\}f=h''\int_x^1fdt\label{4}
     \end{align}
$f(x)$は任意の一次関数で,$a_{\not=0},b\in\mathbb{R}$として,\eqref{4}に$f(x)=ax+b$を代入したとき両辺$a$で割ることにより,$f(x)=x+b$のみ考えれば良いことがわかる.実際に代入して
     \begin{align*}
     &\{1+h'\}(x+b)=\left[\frac{1}{2}t^2+bt\right]_x^1h'' \\
     =&\{\frac{1}{2}(1-x^2)+b(1-x)\}h'' \\
     \end{align*}
これが$b$についての恒等式だから係数比較して,
     \begin{subnumcases}
     {}
     1+h'-(1-x)h''=0 \label{5}\\
     (1+h')x-\frac{1}{2}h''(1-x)(1+x)=0 \label{6}
     \end{subnumcases}
となる.\eqref{5}を\eqref{6}に代入して
     \begin{align*}
      &(1+h')x-\frac{1}{2}(1+h')(1+x)=0 \\
      \Longleftrightarrow
      &(1+h')(x-1)=0
      \end{align*}
これが$x$についての恒等式だから,$h'=-1$が従う.\eqref{0}と合わせて
     \begin{align*}
     h(x)=-x
     \end{align*}
である.これが\eqref{1},\eqref{2},\eqref{3}を満たすことは容易に確かめられる.ゆえに求める関数は
     \[h(x)=-x\]
である.$\cdots$(答)
     
\newpage
\end{multicols}
\end{document}