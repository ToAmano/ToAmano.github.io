\documentclass[a4j]{jarticle}
\usepackage{amsmath}
\usepackage{ascmac}
\usepackage{amssymb}
\usepackage{enumerate}
\usepackage{multicol}
\usepackage{framed}
\usepackage{fancyhdr}
\usepackage{latexsym}
\usepackage{indent}
\usepackage{cases}
\allowdisplaybreaks
\pagestyle{fancy}
\lhead{}
\chead{}
\rhead{東京大学前期$1963$年$1$番}
\begin{document}
%分数関係


\def\tfrac#1#2{{\textstyle\frac{#1}{#2}}} %数式中で文中表示の分数を使う時


%Σ関係

\def\dsum#1#2{{\displaystyle\sum_{#1}^{#2}}} %文中で数式表示のΣを使う時


%ベクトル関係


\def\vector#1{\overrightarrow{#1}}  %ベクトルを表現したいとき(aベクトルを表現するときは\ver
\def\norm#1{|\overrightarrow{#1}|} %ベクトルの絶対値
\def\vtwo#1#2{ \left(%
      \begin{array}{c}%
      #1 \\%
      #2 \\%
      \end{array}%
      \right) }                        %2次元ベクトル成分表示
      
      \def\vthree#1#2#3{ \left(
      \begin{array}{c}
      #1 \\
      #2 \\
      #3 \\
      \end{array}
      \right) }                        %3次元ベクトル成分表示



%数列関係


\def\an#1{\verb|{|$#1$\verb|}|}


%極限関係

\def\limit#1#2{\stackrel{#1 \to #2}{\longrightarrow}}   %等式変形からの極限
\def\dlim#1#2{{\displaystyle \lim_{#1\to#2}}} %文中で数式表示の極限を使う



%積分関係

\def\dint#1#2{{\displaystyle \int_{#1}^{#2}}} %文中で数式表示の積分を使う時

\def\ne{\nearrow}
\def\se{\searrow}
\def\nw{\nwarrow}
\def\ne{\nearrow}


%便利なやつ

\def\case#1#2{%
 \[\left\{%
 \begin{array}{l}%
 #1 \\%
 #2%
 \end{array}%
 \right.\] }                           %場合分け
 
\def\1{$\cos\theta=c$,$\sin\theta=s$とおく.}  %cs表示を与える前書きシータ
\def\2{$\cos t=c$,$\sin t=s$とおく.}     %cs表示を与える前書きt
\def\3{$\cos x=c$,$\sin x=s$とおく.}                %cs表示を与える前書きx

\def\fig#1#2#3 {%
\begin{wrapfigure}[#1]{r}{#2 zw}%
\vspace*{-1zh}%
\input{#3}%
\end{wrapfigure} }           %絵の挿入


\def\a{\alpha}   %ギリシャ文字
\def\b{\beta}
\def\g{\gamma}

%問題番号のためのマクロ

\newcounter{nombre} %必須
\renewcommand{\thenombre}{\arabic{nombre}} %任意
\setcounter{nombre}{2} %任意
\newcounter{nombresub}[nombre] %親子関係を定義
\renewcommand{\thenombresub}{\arabic{nombresub}} %任意
\setcounter{nombresub}{0} %任意
\newcommand{\prob}[1][]{\refstepcounter{nombre}#1[問題 \thenombre]}
\newcommand{\probsub}[1][]{\refstepcounter{nombresub}#1(\thenombresub)}


%1-1みたいなカウンタ(todaiとtodaia)
\newcounter{todai}
\setcounter{todai}{0}
\newcounter{todaisub}[todai] 
\setcounter{todaisub}{0} 
\newcommand{\todai}[1][]{\refstepcounter{todai}#1 \thetodai-\thetodaisub}
\newcommand{\todaib}[1][]{\refstepcounter{todai}#1\refstepcounter{todaisub}#1 {\bf [問題 \thetodai.\thetodaisub]}}
\newcommand{\todaia}[1][]{\refstepcounter{todaisub}#1 {\bf [問題 \thetodai.\thetodaisub]}}


     \begin{oframed}
     直方体の一つの頂点$O$から出る三つの辺を$OA$,$OB$,$OC$とし,$O$から最も遠い頂点を
     $D$とする.$BC=a$, $CA=b$, $AB=c$とするとき,$OD$の長さを$a$,$b$,$c$で表せ.
     また,$a=5$,$b=3$のとき,$c$のとりうる値の範囲を求めよ.
     \end{oframed}

\setlength{\columnseprule}{0.4pt}
\begin{multicols}{2}
{\bf[解]}$OA=x$,$OB=y$,$OC=z$とする.$\triangle OAB$に三平方の定理を用いて
     \begin{align}
     c^2=x^2+y^2 \label{1}
     \end{align}
同様にして
     \begin{align}
     b^2=x^2+z^2  \label{2}\\
     a^2=y^2+z^2 \label{3}
     \end{align}
であるから,
     \begin{align*}
     OD&=\sqrt{x^2+y^2+z^2} \\
     &=\sqrt{\frac{a^2+b^2+c^2}{2}}\tag{答}
     \end{align*}
となる.

次に後半部分について考える.\eqref{1},\eqref{2},\eqref{3}を$x$,$y$,$z$について解いて          
$a$,$b$の値を代入して
      \begin{align*}
      x=\sqrt{\frac{-a^2+b^2+c^2}{2}}=\sqrt{\frac{c^2-16}{2}} \\
      y=\sqrt{\frac{a^2-b^2+c^2}{2}}=\sqrt{\frac{c^2+16}{2}} \\
      z=\sqrt{\frac{a^2+b^2-c^2}{2}}=\sqrt{\frac{-c^2+34}{2}} 
      \end{align*}
このような正の実数$x$,$y$,$z$が存在すれば良いので,求める条件は平方根の中身が非負であること.故に
     \begin{align*}
     &\frac{c^2-16}{2}>0\land\frac{c^2+16}{2}>0\land\frac{-c^2+34}{2}>0  \\
     \Longleftrightarrow&4<c<\sqrt{34}\cdots\text{(答)}  \tag{$\because c>0$} 
     \end{align*}
が求める条件である.   
\newpage
\end{multicols}
\end{document}