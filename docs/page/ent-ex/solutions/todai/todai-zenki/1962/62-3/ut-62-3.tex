\documentclass[a4j]{jarticle}
\usepackage{amsmath}
\usepackage{ascmac}
\usepackage{amssymb}
\usepackage{enumerate}
\usepackage{multicol}
\usepackage{framed}
\usepackage{fancyhdr}
\usepackage{latexsym}
\usepackage{indent}
\usepackage[dvips]{graphicx}
\usepackage{color}
\usepackage{emath}
\usepackage{emathPp}
\usepackage{cases}
\allowdisplaybreaks
\pagestyle{fancy}
\lhead{}
\chead{}
\rhead{東京大学前期$1962$年$3$番}
\begin{document}
%分数関係


\def\tfrac#1#2{{\textstyle\frac{#1}{#2}}} %数式中で文中表示の分数を使う時


%Σ関係

\def\dsum#1#2{{\displaystyle\sum_{#1}^{#2}}} %文中で数式表示のΣを使う時


%ベクトル関係


\def\vector#1{\overrightarrow{#1}}  %ベクトルを表現したいとき(aベクトルを表現するときは\ver
\def\norm#1{|\overrightarrow{#1}|} %ベクトルの絶対値
\def\vtwo#1#2{ \left(%
      \begin{array}{c}%
      #1 \\%
      #2 \\%
      \end{array}%
      \right) }                        %2次元ベクトル成分表示
      
      \def\vthree#1#2#3{ \left(
      \begin{array}{c}
      #1 \\
      #2 \\
      #3 \\
      \end{array}
      \right) }                        %3次元ベクトル成分表示



%数列関係


\def\an#1{\verb|{|$#1$\verb|}|}


%極限関係

\def\limit#1#2{\stackrel{#1 \to #2}{\longrightarrow}}   %等式変形からの極限
\def\dlim#1#2{{\displaystyle \lim_{#1\to#2}}} %文中で数式表示の極限を使う



%積分関係

\def\dint#1#2{{\displaystyle \int_{#1}^{#2}}} %文中で数式表示の積分を使う時

\def\ne{\nearrow}
\def\se{\searrow}
\def\nw{\nwarrow}
\def\ne{\nearrow}


%便利なやつ

\def\case#1#2{%
 \[\left\{%
 \begin{array}{l}%
 #1 \\%
 #2%
 \end{array}%
 \right.\] }                           %場合分け
 
\def\1{$\cos\theta=c$,$\sin\theta=s$とおく.}  %cs表示を与える前書きシータ
\def\2{$\cos t=c$,$\sin t=s$とおく.}     %cs表示を与える前書きt
\def\3{$\cos x=c$,$\sin x=s$とおく.}                %cs表示を与える前書きx

\def\fig#1#2#3 {%
\begin{wrapfigure}[#1]{r}{#2 zw}%
\vspace*{-1zh}%
\input{#3}%
\end{wrapfigure} }           %絵の挿入


\def\a{\alpha}   %ギリシャ文字
\def\b{\beta}
\def\g{\gamma}

%問題番号のためのマクロ

\newcounter{nombre} %必須
\renewcommand{\thenombre}{\arabic{nombre}} %任意
\setcounter{nombre}{2} %任意
\newcounter{nombresub}[nombre] %親子関係を定義
\renewcommand{\thenombresub}{\arabic{nombresub}} %任意
\setcounter{nombresub}{0} %任意
\newcommand{\prob}[1][]{\refstepcounter{nombre}#1[問題 \thenombre]}
\newcommand{\probsub}[1][]{\refstepcounter{nombresub}#1(\thenombresub)}


%1-1みたいなカウンタ(todaiとtodaia)
\newcounter{todai}
\setcounter{todai}{0}
\newcounter{todaisub}[todai] 
\setcounter{todaisub}{0} 
\newcommand{\todai}[1][]{\refstepcounter{todai}#1 \thetodai-\thetodaisub}
\newcommand{\todaib}[1][]{\refstepcounter{todai}#1\refstepcounter{todaisub}#1 {\bf [問題 \thetodai.\thetodaisub]}}
\newcommand{\todaia}[1][]{\refstepcounter{todaisub}#1 {\bf [問題 \thetodai.\thetodaisub]}}


\preEqlabel{$\cdots$}
     \begin{oframed}
     図で,$g$は水平面に対する傾き$\tan\a$が$1/2$であるような定直線とし,$OA$,$AB$は$A$で(ちょうつがいで)連結された長さの等
     しい棒で,その端$O$は$g$上の定点に固定され,$OA$は$g$を含む円直面ないで自由に回転し,他の端$B$は$g$上を動くことができ
     るようになっている.
     
     このとき,折れ線$OAB$の重心$G$($OA$,$AB$の中点を結ぶ線分の中点)が最低になるのは,$OA$の水平面となす傾き$\tan\theta$
     がいくらになるときか.
     \end{oframed}

\setlength{\columnseprule}{0.4pt}
\begin{multicols}{2}
{\bf[解]} \1 $O$を原点とし,水平線を$x$軸とする下図のような座標系で考える.

     \begin{minipage}{0.5\hsize}
          \begin{center}
          \scalebox{.4}{%WinTpicVersion4.32a
{\unitlength 0.1in%
\begin{picture}(36.0000,22.0000)(6.0000,-26.0000)%
% STR 2 0 3 0 Black White  
% 4 1790 997 1790 1010 4 600 0 0
% O
\put(17.9000,-10.1000){\makebox(0,0)[rt]{O}}%
% STR 2 0 3 0 Black White  
% 4 1760 387 1760 400 4 600 0 0
% $y$
\put(17.6000,-4.0000){\makebox(0,0)[rt]{$y$}}%
% STR 2 0 3 0 Black White  
% 4 4200 1027 4200 1040 4 600 0 0
% $x$
\put(42.0000,-10.4000){\makebox(0,0)[rt]{$x$}}%
% VECTOR 2 0 3 0 Black White  
% 2 1800 2600 1800 400
% 
\special{pn 8}%
\special{pa 1800 2600}%
\special{pa 1800 400}%
\special{fp}%
\special{sh 1}%
\special{pa 1800 400}%
\special{pa 1780 467}%
\special{pa 1800 453}%
\special{pa 1820 467}%
\special{pa 1800 400}%
\special{fp}%
% VECTOR 2 0 3 0 Black White  
% 2 600 1000 4200 1000
% 
\special{pn 8}%
\special{pa 600 1000}%
\special{pa 4200 1000}%
\special{fp}%
\special{sh 1}%
\special{pa 4200 1000}%
\special{pa 4133 980}%
\special{pa 4147 1000}%
\special{pa 4133 1020}%
\special{pa 4200 1000}%
\special{fp}%
% FUNC 2 0 3 0 Black White  
% 9 600 400 4200 2600 1800 1000 2200 1000 1800 600 600 400 4200 2600 0 2 0 0
% -x/2
\special{pn 8}%
\special{pa 600 400}%
\special{pa 605 403}%
\special{pa 610 405}%
\special{pa 615 408}%
\special{pa 625 412}%
\special{pa 635 418}%
\special{pa 645 422}%
\special{pa 655 428}%
\special{pa 665 432}%
\special{pa 675 438}%
\special{pa 685 442}%
\special{pa 690 445}%
\special{pa 695 447}%
\special{pa 705 453}%
\special{pa 710 455}%
\special{pa 715 458}%
\special{pa 725 462}%
\special{pa 735 468}%
\special{pa 745 472}%
\special{pa 755 478}%
\special{pa 765 482}%
\special{pa 775 488}%
\special{pa 785 492}%
\special{pa 790 495}%
\special{pa 795 497}%
\special{pa 805 503}%
\special{pa 810 505}%
\special{pa 815 508}%
\special{pa 825 512}%
\special{pa 835 518}%
\special{pa 845 522}%
\special{pa 855 528}%
\special{pa 865 532}%
\special{pa 875 538}%
\special{pa 885 542}%
\special{pa 895 548}%
\special{pa 905 552}%
\special{pa 915 558}%
\special{pa 925 562}%
\special{pa 935 568}%
\special{pa 945 572}%
\special{pa 955 578}%
\special{pa 965 582}%
\special{pa 975 588}%
\special{pa 985 592}%
\special{pa 995 598}%
\special{pa 1005 602}%
\special{pa 1015 608}%
\special{pa 1025 612}%
\special{pa 1035 618}%
\special{pa 1045 622}%
\special{pa 1055 628}%
\special{pa 1065 632}%
\special{pa 1075 638}%
\special{pa 1085 642}%
\special{pa 1095 648}%
\special{pa 1105 652}%
\special{pa 1115 658}%
\special{pa 1125 662}%
\special{pa 1135 668}%
\special{pa 1145 672}%
\special{pa 1155 678}%
\special{pa 1165 682}%
\special{pa 1175 688}%
\special{pa 1185 692}%
\special{pa 1195 698}%
\special{pa 1205 702}%
\special{pa 1215 708}%
\special{pa 1225 712}%
\special{pa 1235 718}%
\special{pa 1245 722}%
\special{pa 1255 728}%
\special{pa 1265 732}%
\special{pa 1275 738}%
\special{pa 1285 742}%
\special{pa 1295 748}%
\special{pa 1305 752}%
\special{pa 1315 758}%
\special{pa 1325 762}%
\special{pa 1335 768}%
\special{pa 1345 772}%
\special{pa 1355 778}%
\special{pa 1365 782}%
\special{pa 1375 788}%
\special{pa 1385 792}%
\special{pa 1395 798}%
\special{pa 1405 802}%
\special{pa 1415 808}%
\special{pa 1425 812}%
\special{pa 1435 818}%
\special{pa 1445 822}%
\special{pa 1455 828}%
\special{pa 1465 832}%
\special{pa 1475 838}%
\special{pa 1485 842}%
\special{pa 1495 848}%
\special{pa 1505 852}%
\special{pa 1515 858}%
\special{pa 1525 862}%
\special{pa 1535 868}%
\special{pa 1545 872}%
\special{pa 1555 878}%
\special{pa 1565 882}%
\special{pa 1575 888}%
\special{pa 1585 892}%
\special{pa 1595 898}%
\special{pa 1605 902}%
\special{pa 1615 908}%
\special{pa 1625 912}%
\special{pa 1635 918}%
\special{pa 1645 922}%
\special{pa 1655 928}%
\special{pa 1665 932}%
\special{pa 1675 938}%
\special{pa 1685 942}%
\special{pa 1695 948}%
\special{pa 1705 952}%
\special{pa 1715 958}%
\special{pa 1725 962}%
\special{pa 1735 968}%
\special{pa 1745 972}%
\special{pa 1755 978}%
\special{pa 1765 982}%
\special{pa 1775 988}%
\special{pa 1785 992}%
\special{pa 1795 998}%
\special{pa 1805 1002}%
\special{pa 1815 1008}%
\special{pa 1825 1012}%
\special{pa 1835 1018}%
\special{pa 1845 1022}%
\special{pa 1855 1028}%
\special{pa 1865 1032}%
\special{pa 1875 1038}%
\special{pa 1885 1042}%
\special{pa 1895 1048}%
\special{pa 1905 1052}%
\special{pa 1915 1058}%
\special{pa 1925 1062}%
\special{pa 1935 1068}%
\special{pa 1945 1072}%
\special{pa 1955 1078}%
\special{pa 1965 1082}%
\special{pa 1975 1088}%
\special{pa 1985 1092}%
\special{pa 1995 1098}%
\special{pa 2005 1102}%
\special{pa 2015 1108}%
\special{pa 2025 1112}%
\special{pa 2035 1118}%
\special{pa 2045 1122}%
\special{pa 2055 1128}%
\special{pa 2065 1132}%
\special{pa 2075 1138}%
\special{pa 2085 1142}%
\special{pa 2095 1148}%
\special{pa 2105 1152}%
\special{pa 2115 1158}%
\special{pa 2125 1162}%
\special{pa 2135 1168}%
\special{pa 2145 1172}%
\special{pa 2155 1178}%
\special{pa 2165 1182}%
\special{pa 2175 1188}%
\special{pa 2185 1192}%
\special{pa 2195 1198}%
\special{pa 2205 1202}%
\special{pa 2215 1208}%
\special{pa 2225 1212}%
\special{pa 2235 1218}%
\special{pa 2245 1222}%
\special{pa 2255 1228}%
\special{pa 2265 1232}%
\special{pa 2275 1238}%
\special{pa 2285 1242}%
\special{pa 2295 1248}%
\special{pa 2305 1252}%
\special{pa 2315 1258}%
\special{pa 2325 1262}%
\special{pa 2335 1268}%
\special{pa 2345 1272}%
\special{pa 2355 1278}%
\special{pa 2365 1282}%
\special{pa 2375 1288}%
\special{pa 2385 1292}%
\special{pa 2395 1298}%
\special{pa 2405 1302}%
\special{pa 2415 1308}%
\special{pa 2425 1312}%
\special{pa 2435 1318}%
\special{pa 2445 1322}%
\special{pa 2455 1328}%
\special{pa 2465 1332}%
\special{pa 2475 1338}%
\special{pa 2485 1342}%
\special{pa 2495 1348}%
\special{pa 2505 1352}%
\special{pa 2515 1358}%
\special{pa 2525 1362}%
\special{pa 2535 1368}%
\special{pa 2545 1372}%
\special{pa 2555 1378}%
\special{pa 2565 1382}%
\special{pa 2575 1388}%
\special{pa 2585 1392}%
\special{pa 2595 1398}%
\special{pa 2605 1402}%
\special{pa 2615 1408}%
\special{pa 2625 1412}%
\special{pa 2635 1418}%
\special{pa 2645 1422}%
\special{pa 2655 1428}%
\special{pa 2665 1432}%
\special{pa 2675 1438}%
\special{pa 2685 1442}%
\special{pa 2695 1448}%
\special{pa 2705 1452}%
\special{pa 2715 1458}%
\special{pa 2725 1462}%
\special{pa 2735 1468}%
\special{pa 2745 1472}%
\special{pa 2755 1478}%
\special{pa 2765 1482}%
\special{pa 2775 1488}%
\special{pa 2785 1492}%
\special{pa 2795 1498}%
\special{pa 2805 1502}%
\special{pa 2815 1508}%
\special{pa 2825 1512}%
\special{pa 2835 1518}%
\special{pa 2845 1522}%
\special{pa 2855 1528}%
\special{pa 2865 1532}%
\special{pa 2875 1538}%
\special{pa 2885 1542}%
\special{pa 2895 1548}%
\special{pa 2905 1552}%
\special{pa 2915 1558}%
\special{pa 2925 1562}%
\special{pa 2935 1568}%
\special{pa 2945 1572}%
\special{pa 2955 1578}%
\special{pa 2965 1582}%
\special{pa 2975 1588}%
\special{pa 2985 1592}%
\special{pa 2995 1598}%
\special{pa 3005 1602}%
\special{pa 3015 1608}%
\special{pa 3025 1612}%
\special{pa 3035 1618}%
\special{pa 3045 1622}%
\special{pa 3055 1628}%
\special{pa 3065 1632}%
\special{pa 3075 1638}%
\special{pa 3085 1642}%
\special{pa 3095 1648}%
\special{pa 3105 1652}%
\special{pa 3115 1658}%
\special{pa 3125 1662}%
\special{pa 3135 1668}%
\special{pa 3145 1672}%
\special{pa 3155 1678}%
\special{pa 3165 1682}%
\special{pa 3175 1688}%
\special{pa 3185 1692}%
\special{pa 3195 1698}%
\special{pa 3205 1702}%
\special{pa 3215 1708}%
\special{pa 3225 1712}%
\special{pa 3235 1718}%
\special{pa 3245 1722}%
\special{pa 3255 1728}%
\special{pa 3265 1732}%
\special{pa 3275 1738}%
\special{pa 3285 1742}%
\special{pa 3295 1748}%
\special{pa 3305 1752}%
\special{pa 3315 1758}%
\special{pa 3325 1762}%
\special{pa 3335 1768}%
\special{pa 3345 1772}%
\special{pa 3355 1778}%
\special{pa 3365 1782}%
\special{pa 3375 1788}%
\special{pa 3385 1792}%
\special{pa 3395 1798}%
\special{pa 3405 1802}%
\special{pa 3415 1808}%
\special{pa 3425 1812}%
\special{pa 3435 1818}%
\special{pa 3445 1822}%
\special{pa 3455 1828}%
\special{pa 3465 1832}%
\special{pa 3475 1838}%
\special{pa 3485 1842}%
\special{pa 3495 1848}%
\special{pa 3505 1852}%
\special{pa 3515 1858}%
\special{pa 3525 1862}%
\special{pa 3535 1868}%
\special{pa 3545 1872}%
\special{pa 3555 1878}%
\special{pa 3565 1882}%
\special{pa 3575 1888}%
\special{pa 3585 1892}%
\special{pa 3595 1898}%
\special{pa 3605 1902}%
\special{pa 3615 1908}%
\special{pa 3625 1912}%
\special{pa 3635 1918}%
\special{pa 3645 1922}%
\special{pa 3655 1928}%
\special{pa 3665 1932}%
\special{pa 3675 1938}%
\special{pa 3685 1942}%
\special{pa 3695 1948}%
\special{pa 3705 1952}%
\special{pa 3715 1958}%
\special{pa 3725 1962}%
\special{pa 3735 1968}%
\special{pa 3745 1972}%
\special{pa 3755 1978}%
\special{pa 3765 1982}%
\special{pa 3775 1988}%
\special{pa 3785 1992}%
\special{pa 3795 1998}%
\special{pa 3805 2002}%
\special{pa 3815 2008}%
\special{pa 3825 2012}%
\special{pa 3835 2018}%
\special{pa 3845 2022}%
\special{pa 3855 2028}%
\special{pa 3865 2032}%
\special{pa 3875 2038}%
\special{pa 3885 2042}%
\special{pa 3895 2048}%
\special{pa 3905 2052}%
\special{pa 3915 2058}%
\special{pa 3925 2062}%
\special{pa 3935 2068}%
\special{pa 3945 2072}%
\special{pa 3955 2078}%
\special{pa 3965 2082}%
\special{pa 3975 2088}%
\special{pa 3985 2092}%
\special{pa 3995 2098}%
\special{pa 4005 2102}%
\special{pa 4015 2108}%
\special{pa 4025 2112}%
\special{pa 4035 2118}%
\special{pa 4045 2122}%
\special{pa 4055 2128}%
\special{pa 4065 2132}%
\special{pa 4075 2138}%
\special{pa 4085 2142}%
\special{pa 4095 2148}%
\special{pa 4105 2152}%
\special{pa 4115 2158}%
\special{pa 4125 2162}%
\special{pa 4135 2168}%
\special{pa 4145 2172}%
\special{pa 4155 2178}%
\special{pa 4165 2182}%
\special{pa 4175 2188}%
\special{pa 4185 2192}%
\special{pa 4195 2198}%
\special{pa 4200 2200}%
\special{fp}%
% LINE 2 0 3 0 Black White  
% 4 1800 1000 2420 2010 2420 2010 3820 2010
% 
\special{pn 8}%
\special{pa 1800 1000}%
\special{pa 2420 2010}%
\special{fp}%
\special{pa 2420 2010}%
\special{pa 3820 2010}%
\special{fp}%
% LINE 2 2 3 0 Black White  
% 2 2420 2010 2740 1470
% 
\special{pn 8}%
\special{pa 2420 2010}%
\special{pa 2740 1470}%
\special{dt 0.045}%
% STR 2 0 3 0 Black White  
% 4 2380 2070 2380 2170 5 0 1 0
% A
\put(23.8000,-21.7000){\makebox(0,0){{\colorbox[named]{White}{A}}}}%
% STR 2 0 3 0 Black White  
% 4 3880 1780 3880 1880 2 0 1 0
% B
\put(38.8000,-18.8000){\makebox(0,0)[lb]{{\colorbox[named]{White}{B}}}}%
% STR 2 0 3 0 Black White  
% 4 2810 1290 2810 1390 2 0 1 0
% H
\put(28.1000,-13.9000){\makebox(0,0)[lb]{{\colorbox[named]{White}{H}}}}%
% STR 2 0 3 0 Black White  
% 4 1290 460 1290 560 5 0 1 0
% $g$
\put(12.9000,-5.6000){\makebox(0,0){{\colorbox[named]{White}{$g$}}}}%
% STR 2 0 3 0 Black White  
% 4 2110 1170 2110 1270 5 0 1 0
% $\b$
\put(21.1000,-12.7000){\makebox(0,0){{\colorbox[named]{White}{$\b$}}}}%
% STR 2 0 3 0 Black White  
% 4 2300 1020 2300 1120 5 0 1 0
% $\a$
\put(23.0000,-11.2000){\makebox(0,0){{\colorbox[named]{White}{$\a$}}}}%
\end{picture}}%
}
          \end{center}
     \end{minipage}
     \begin{minipage}{0.5\hsize}
         \begin{center}
          \scalebox{.5}{%WinTpicVersion4.32a
{\unitlength 0.1in%
\begin{picture}(21.9000,9.6000)(2.7000,-12.2500)%
% LINE 2 0 3 0 Black White  
% 8 400 400 2400 400 1400 1200 2400 400 1400 1200 400 400 1400 1200 1400 400
% 
\special{pn 8}%
\special{pa 400 400}%
\special{pa 2400 400}%
\special{fp}%
\special{pa 1400 1200}%
\special{pa 2400 400}%
\special{fp}%
\special{pa 1400 1200}%
\special{pa 400 400}%
\special{fp}%
\special{pa 1400 1200}%
\special{pa 1400 400}%
\special{fp}%
% STR 2 0 3 0 Black White  
% 4 1400 1190 1400 1290 5 0 1 0
% A
\put(14.0000,-12.9000){\makebox(0,0){{\colorbox[named]{White}{A}}}}%
% STR 2 0 3 0 Black White  
% 4 2490 320 2490 420 5 0 1 0
% B
\put(24.9000,-4.2000){\makebox(0,0){{\colorbox[named]{White}{B}}}}%
% STR 2 0 3 0 Black White  
% 4 310 290 310 390 5 0 1 0
% O
\put(3.1000,-3.9000){\makebox(0,0){{\colorbox[named]{White}{O}}}}%
% STR 2 0 3 0 Black White  
% 4 1400 230 1400 330 5 0 1 0
% H
\put(14.0000,-3.3000){\makebox(0,0){{\colorbox[named]{White}{H}}}}%
% STR 2 0 3 0 Black White  
% 4 2020 800 2020 900 5 0 1 0
% $1$
\put(20.2000,-9.0000){\makebox(0,0){{\colorbox[named]{White}{$1$}}}}%
% STR 2 0 3 0 Black White  
% 4 760 780 760 880 5 0 1 0
% $1$
\put(7.6000,-8.8000){\makebox(0,0){{\colorbox[named]{White}{$1$}}}}%
% STR 2 0 3 0 Black White  
% 4 1490 700 1490 800 5 0 1 0
% G
\put(14.9000,-8.0000){\makebox(0,0){{\colorbox[named]{White}{G}}}}%
% STR 2 0 3 0 Black White  
% 4 2140 380 2140 480 4 0 1 0
% $\b$
\put(21.4000,-4.8000){\makebox(0,0)[rt]{{\colorbox[named]{White}{$\b$}}}}%
\end{picture}}%
}
          \end{center}
     \end{minipage}
     
$OA=1$として考えて一般性を失わない.また$G$が最低のときを考えるので,$A$は$g$の下側にあるとしてよい.
簡単のため$\b=\theta-\a$とおく.$A$から$g$に下ろした垂足$H$とすると,$OA=AB$ゆえ,$G$は$AH$の中点である.
$OH=\cos\b$ゆえ,
     \begin{align*}
     &A(\cos(-\theta),\sin(-\theta)) \\
     &H(\cos\b\cos(-\a),\cos\b\sin(-\b))
     \end{align*}
となるから,
     \begin{align*}
     \bekutoru{OG}&=\bekutoru{OH}+\frac{1}{2}\bekutoru{HA} \\
     &=\frac{1}{2}\bekutoru{OH}+\frac{1}{2}\bekutoru{OA} \\
     &=\frac{1}{2}\cos\b\vtwo{\cos\a}{-\sin\a}+\frac{1}{2}\vtwo{c}{-s}
     \end{align*}
である.この$y$座標$Y$として
     \begin{align*}
     Y=\frac{-1}{2}(\sin\a\cos\b+s)
     \end{align*}
だから,これが最小になる時の$\tan\theta$を求めればよい.
     \[\a+\b=\theta\]
に注意して
     \begin{align*}
     -2Y&=\frac{1}{2}(\sin(\a+\b)+\sin(\a-\b))+s \\
     &=\frac{3}{2}s+\frac{1}{2}\sin(2\a-\theta)\atag\label{1}
     \end{align*}
題意から$\tan\a=1/2$であるから
     \begin{align*}
     \sin2\a=\frac{2\tan\a}{1+\tan^2\a}=\frac{4}{5} \\
     \cos2\a=\frac{1-\tan^2\a}{1+\tan^2\a}=\frac{3}{5}
     \end{align*}
が従う.\eqref{1}に代入して
     \begin{align*}
     Y&=\frac{-1}{4}[3s+c\sin2\a -s\cos2\a ] \\
     &=\frac{-1}{4}[3s+\frac{4}{5} c-\frac{3}{5} s] \\
     &=\frac{-1}{5}(3s+c) \\
     &\ge \frac{-1}{5}\sqrt{3^2+1}&(\because \text{コーシー})
     \end{align*}
統合成立条件は
     \begin{align*}
     &\vtwo{c}{s}\parallel\vtwo{1}{3} \\
     &s-3c=0
     \end{align*}
の時で,この時
     \[\tan\theta=3\]
である.$\cdots$(答)
     
     
\newpage
\end{multicols}
\end{document}