\documentclass[a4j]{jarticle}
\usepackage{amsmath}
\usepackage{ascmac}
\usepackage{amssymb}
\usepackage{enumerate}
\usepackage{multicol}
\usepackage{framed}
\usepackage{fancyhdr}
\usepackage{latexsym}
\usepackage{indent}
\usepackage{cases}
\usepackage[dvips]{graphicx}
\usepackage{color}
\usepackage{emath}
\usepackage{emathPp}
\allowdisplaybreaks
\pagestyle{fancy}
\lhead{}
\chead{}
\rhead{東京大学前期$1976$年$3new$番}
\begin{document}
%分数関係


\def\tfrac#1#2{{\textstyle\frac{#1}{#2}}} %数式中で文中表示の分数を使う時


%Σ関係

\def\dsum#1#2{{\displaystyle\sum_{#1}^{#2}}} %文中で数式表示のΣを使う時


%ベクトル関係


\def\vector#1{\overrightarrow{#1}}  %ベクトルを表現したいとき(aベクトルを表現するときは\ver
\def\norm#1{|\overrightarrow{#1}|} %ベクトルの絶対値
\def\vtwo#1#2{ \left(%
      \begin{array}{c}%
      #1 \\%
      #2 \\%
      \end{array}%
      \right) }                        %2次元ベクトル成分表示
      
      \def\vthree#1#2#3{ \left(
      \begin{array}{c}
      #1 \\
      #2 \\
      #3 \\
      \end{array}
      \right) }                        %3次元ベクトル成分表示



%数列関係


\def\an#1{\verb|{|$#1$\verb|}|}


%極限関係

\def\limit#1#2{\stackrel{#1 \to #2}{\longrightarrow}}   %等式変形からの極限
\def\dlim#1#2{{\displaystyle \lim_{#1\to#2}}} %文中で数式表示の極限を使う



%積分関係

\def\dint#1#2{{\displaystyle \int_{#1}^{#2}}} %文中で数式表示の積分を使う時

\def\ne{\nearrow}
\def\se{\searrow}
\def\nw{\nwarrow}
\def\ne{\nearrow}


%便利なやつ

\def\case#1#2{%
 \[\left\{%
 \begin{array}{l}%
 #1 \\%
 #2%
 \end{array}%
 \right.\] }                           %場合分け
 
\def\1{$\cos\theta=c$,$\sin\theta=s$とおく.}  %cs表示を与える前書きシータ
\def\2{$\cos t=c$,$\sin t=s$とおく.}     %cs表示を与える前書きt
\def\3{$\cos x=c$,$\sin x=s$とおく.}                %cs表示を与える前書きx

\def\fig#1#2#3 {%
\begin{wrapfigure}[#1]{r}{#2 zw}%
\vspace*{-1zh}%
\input{#3}%
\end{wrapfigure} }           %絵の挿入


\def\a{\alpha}   %ギリシャ文字
\def\b{\beta}
\def\g{\gamma}

%問題番号のためのマクロ

\newcounter{nombre} %必須
\renewcommand{\thenombre}{\arabic{nombre}} %任意
\setcounter{nombre}{2} %任意
\newcounter{nombresub}[nombre] %親子関係を定義
\renewcommand{\thenombresub}{\arabic{nombresub}} %任意
\setcounter{nombresub}{0} %任意
\newcommand{\prob}[1][]{\refstepcounter{nombre}#1[問題 \thenombre]}
\newcommand{\probsub}[1][]{\refstepcounter{nombresub}#1(\thenombresub)}


%1-1みたいなカウンタ(todaiとtodaia)
\newcounter{todai}
\setcounter{todai}{0}
\newcounter{todaisub}[todai] 
\setcounter{todaisub}{0} 
\newcommand{\todai}[1][]{\refstepcounter{todai}#1 \thetodai-\thetodaisub}
\newcommand{\todaib}[1][]{\refstepcounter{todai}#1\refstepcounter{todaisub}#1 {\bf [問題 \thetodai.\thetodaisub]}}
\newcommand{\todaia}[1][]{\refstepcounter{todaisub}#1 {\bf [問題 \thetodai.\thetodaisub]}}


     \begin{oframed}
     点P$(x,y)$は$xy$平面上の点$C:(x-5)^2+(y-5)^2=r^2\, (r>0)$の上を動く動点である.このとき点P
     の点A$(9,0)$に関する対称点をQとし,また点Pを原点Oのまわりに正の向きに$\pi/2$だけ回転した
     点をRとする.点Pが円$C$の上を動くときの線分QRの長さの最小値$f(r)$と最大値$g(r)$とを求めよ.
     また$f(r)$が$0$となるような$r$の値を求めよ.
     \end{oframed}

\setlength{\columnseprule}{0.4pt}
\begin{multicols}{2}
{\bf[解]} \1 とおく.するとP$(5+rc,5+rs)$とおけるから,
     \begin{align*}
     &\mathrm{Q}(13-rc,-5-rs)&\mathrm{R}(-5-rs,5+rc)
     \end{align*}
である.故に
     \begin{align*}
     |\mathrm{QR}|^2=&\left\{(13-rc)-(-5-rs)\right\}^2 \\
     &+\left\{(-5-rs)-(5+rc)\right\}^2 \\
     =&(18-rc+rs)^2+(-10+rs+rc)^2 \\
     =&18^2+10^2+2r^2+56rs-16rc \\
     =&424+2r^2+8\sqrt{53}r\sin (\theta-\a)
     \end{align*}     
である.ここで,$\a$は
     \[\tan\a=\frac{2}{7}\]
を満たす数である.$0\le \theta <2\pi$から,
     \[-\a\le\theta-\a <2\pi -\a\]
であるから,$-1\le\sin (\theta-\a)\le 1$である.故に$r>0$から,
     \begin{align*}
          \begin{cases}
          f(r)=\sqrt{424-8\sqrt{53}r+2r^2}=\sqrt{2}|r-2\sqrt{53}| \\
          g(r)=\sqrt{424+8\sqrt{53}r+2r^2}=\sqrt{2}(r+2\sqrt{53})
          \end{cases}
     \end{align*}
である.また,$f(r)=0$のとき,$r=2\sqrt{53}$である.$\cdots$(答)
\newpage
\end{multicols}
\end{document}