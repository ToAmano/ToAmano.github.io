\documentclass[a4j]{jarticle}
\usepackage{amsmath}
\usepackage{ascmac}
\usepackage{amssymb}
\usepackage{enumerate}
\usepackage{multicol}
\usepackage{framed}
\usepackage{latexsym}
\usepackage{indent}
\usepackage{cases}
\allowdisplaybreaks
\title{}
\begin{document}
%分数関係


\def\tfrac#1#2{{\textstyle\frac{#1}{#2}}} %数式中で文中表示の分数を使う時


%Σ関係

\def\dsum#1#2{{\displaystyle\sum_{#1}^{#2}}} %文中で数式表示のΣを使う時


%ベクトル関係


\def\vector#1{\overrightarrow{#1}}  %ベクトルを表現したいとき(aベクトルを表現するときは\ver
\def\norm#1{|\overrightarrow{#1}|} %ベクトルの絶対値
\def\vtwo#1#2{ \left(%
      \begin{array}{c}%
      #1 \\%
      #2 \\%
      \end{array}%
      \right) }                        %2次元ベクトル成分表示
      
      \def\vthree#1#2#3{ \left(
      \begin{array}{c}
      #1 \\
      #2 \\
      #3 \\
      \end{array}
      \right) }                        %3次元ベクトル成分表示



%数列関係


\def\an#1{\verb|{|$#1$\verb|}|}


%極限関係

\def\limit#1#2{\stackrel{#1 \to #2}{\longrightarrow}}   %等式変形からの極限
\def\dlim#1#2{{\displaystyle \lim_{#1\to#2}}} %文中で数式表示の極限を使う



%積分関係

\def\dint#1#2{{\displaystyle \int_{#1}^{#2}}} %文中で数式表示の積分を使う時

\def\ne{\nearrow}
\def\se{\searrow}
\def\nw{\nwarrow}
\def\ne{\nearrow}


%便利なやつ

\def\case#1#2{%
 \[\left\{%
 \begin{array}{l}%
 #1 \\%
 #2%
 \end{array}%
 \right.\] }                           %場合分け
 
\def\1{$\cos\theta=c$,$\sin\theta=s$とおく.}  %cs表示を与える前書きシータ
\def\2{$\cos t=c$,$\sin t=s$とおく.}     %cs表示を与える前書きt
\def\3{$\cos x=c$,$\sin x=s$とおく.}                %cs表示を与える前書きx

\def\fig#1#2#3 {%
\begin{wrapfigure}[#1]{r}{#2 zw}%
\vspace*{-1zh}%
\input{#3}%
\end{wrapfigure} }           %絵の挿入


\def\a{\alpha}   %ギリシャ文字
\def\b{\beta}
\def\g{\gamma}

%問題番号のためのマクロ

\newcounter{nombre} %必須
\renewcommand{\thenombre}{\arabic{nombre}} %任意
\setcounter{nombre}{2} %任意
\newcounter{nombresub}[nombre] %親子関係を定義
\renewcommand{\thenombresub}{\arabic{nombresub}} %任意
\setcounter{nombresub}{0} %任意
\newcommand{\prob}[1][]{\refstepcounter{nombre}#1[問題 \thenombre]}
\newcommand{\probsub}[1][]{\refstepcounter{nombresub}#1(\thenombresub)}


%1-1みたいなカウンタ(todaiとtodaia)
\newcounter{todai}
\setcounter{todai}{0}
\newcounter{todaisub}[todai] 
\setcounter{todaisub}{0} 
\newcommand{\todai}[1][]{\refstepcounter{todai}#1 \thetodai-\thetodaisub}
\newcommand{\todaib}[1][]{\refstepcounter{todai}#1\refstepcounter{todaisub}#1 {\bf [問題 \thetodai.\thetodaisub]}}
\newcommand{\todaia}[1][]{\refstepcounter{todaisub}#1 {\bf [問題 \thetodai.\thetodaisub]}}


\begin{oframed}
$x$の整式
     \begin{align*}
     f_n(x)=1+\frac{x}{1!}+\frac{x^2}{2!}+\dots+\frac{x^n}{n!} \ \ \ (n=1,2,\dots)
     \end{align*}
について$f'_n(x)=f_{n-1}(n=2,3,\dots)$が成り立つことを証明せよ.

方程式$f_n(x)=0$は,$n$が奇数ならばただ$1$つの実根をもち,$n$が偶数ならば実根を持たないことを数学的帰納法を用いて証明せよ.     
\end{oframed}

\setlength{\columnseprule}{0.4pt}
\begin{multicols}{2}
{\bf[解]} まず前半部について,$n\ge2$のとき,
     \begin{align}
     f'_n(x)&=1+\frac{x}{1!}+\frac{x^2}{2!}+\dots+\frac{x^{n-1}}{(n-1)!}\nonumber \\
     &=f_{n-1}(x) \label{1} 
     \end{align}
であるから示された.$\Box$

そこで後半部を考える.
\begin{center}
P:方程式$f_n(x)=0$は,$n$が奇数ならばただ$1$つの実根をもつ.
また$n$が偶数ならば$y=f_n(x)>0$である.
\end{center}
として,命題$P$を帰納的に示す.
まず$n=1,2$のとき,
     \begin{align*}
     \left\{
          \begin{array}{l}
          f_1(x)=1+x \\
          f_2(x)=1+x+\dfrac{x^2}{2}=\dfrac{1}{2}(x+1)^2+\dfrac{1}{2}
          \end{array}
     \right.     
     \end{align*}
が$P$を満たすことは明らかである.そこで$k\in\mathbb{N}$に対し$n=2k-1,2k$での$P$の成立を仮定する.
     \begin{indentation}{2zw}{0pt}
      \noindent\underline{(i)$n=2k+1$のとき} \\
      \eqref{1}および仮定から$f'_{2k+1}=f'_{2k}(x)>0$だから$y=f_{2k+1}$は単調増加.
     また連続で$\dlim{x}{\pm\infty}f_{2k+1}(x)=\pm\infty$(複合同順)であるから,中間値の定理より
     $f_{2k+1}(x)=0$なる実数$x$がただひとつある.ここで$f_{2k+1}(0)=1$から$x\not=0$である. \\
     \\
     \underline{(ii)$n=2(k+1)$のとき} \noindent \\
     \eqref{1}から$f'_{2(k+1)}(x)=f_{2k+1}(x)$であり,  $n=2(k+1)$の時の考察から,$f_{2k+1}(a)=0$
     なる実数$a$がただひとつあって下表を得る.
          \begin{align*}
               \begin{array}{|c|c|c|c|} \hline
               x                  &      & a &    \\ \hline
               f'_{2(k+1)}   &  -   & 0 & +   \\ \hline
                f_{2(k+1)}   & \se&    & \ne  \\ \hline
                \end{array}
           \end{align*}
      故に
      \begin{align*}
      f_{2(k+1)}(x)&\ge f_{2(k+1)}(a) \\ 
      &= f_{2k+1}(a)+\frac{a^{2(k+1)}}{2(k+1)!}   \\
      &=\frac{a^{2(k+1)}}{2(k+1)!}  
      >0 
      \end{align*}
     である.
     \end{indentation}
  
  以上から$n=2k+1,2(k+1)$でも$P$が成り立つことが示された.故に任意の$n$で$P$は成立する.
$P$が示されたので,題意が従う.$\Box$
 
\newpage
\end{multicols}
\end{document}