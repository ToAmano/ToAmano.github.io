\documentclass[twocolumn,showpacs,prb,amsfonts,amsmath,amssymb,floatfix,groupedaddress]{revtex4-1}
\usepackage{color}
\usepackage{hhline}
\usepackage{mathrsfs}
\usepackage{graphicx}
\usepackage{dcolumn}
\usepackage{bm}% bold math
\usepackage{multirow}
\usepackage{booktabs}
\usepackage{afterpage}
\usepackage{amsmath}
\usepackage{ulem}


\newcommand{\red}[1]{\textcolor{red}{#1}}
\newcommand{\blue}[1]{\textcolor{blue}{#1}}
\arraycolsep=0.0em
\setlength{\abovecaptionskip}{0mm}
\setlength{\belowcaptionskip}{0mm}
%\setlength{\MidlineHeight}{2pt}


\newcommand{\ph}{\phantom{0}}

\renewcommand{\topfraction}{1.0}
\renewcommand{\bottomfraction}{1.0}
\renewcommand{\dbltopfraction}{1.0}
\renewcommand{\textfraction}{0.1}
\renewcommand{\floatpagefraction}{0.9}
\renewcommand{\dblfloatpagefraction}{0.9}
\begin{document}


%\title{Electronic theory for anomalous chemistry under compression}
\title{Fate of hydrogen $1s$ orbitals in hydrides under compression}
\author{Ryosuke Akashi$^{1}$}
\thanks{akashi@cms.phys.s.u-tokyo.ac.jp}
\affiliation{$^1$Department of Physics, The University of Tokyo, Hongo, Bunkyo-ku, Tokyo 113-0033, Japan}

\date{\today}
\begin{abstract}
Recent advance in the crystal structure prediction and high pressure technology has revealed

We theoretically address the electronic states emerging from the combination of atomic and molecular orbitals spatially overlapping with each other. Due to the orbital overlap, some eigenstates formed by their superposition must become spatially fragmented, resulting in high kinetic energy. With this effect, the Hilbert space spanned by these orbitals extends to very high energy regime and, we point out that, at any threshold distances between the orbitals, its upper bound diverges to infinity. This means that the low-energy electronic states cannot be well described by the molecular orbitals for isolated systems. Anomalous chemical bonds seen under strong compression are generally thought to be the consequence of the above fact; directional bonding and clathrate formation of hydrogen, interstitial electronic occupation, etc.
\end{abstract}

\maketitle

\section{Introduction}
The high pressure studies have revealed that the pressure may stabilize compounds with anomalous chemical formulas that cannot be conceived with the regular ionic valences of the composition atoms. Various irregular compounds have been found in recent crystal structure simulations. 

Anomalous chemistry

Compressed electrides

Deformed Wannier orbitals

Failure of Slater-Koster modeling

In this paper, we conceive a 
\section{Theory}
Our consideration starts from the H\"{u}ckel model for H$_{2}$
\begin{eqnarray}
\left[
\begin{array}{cc}
E_{0} & T \\
T & E_{0}
\end{array}
\right]
\Psi
=
E
\left[
\begin{array}{cc}
1 & S \\
S & 1
\end{array}
\right]
\Psi
.
\end{eqnarray}
Here, $E_{0}$ denotes the $1s$ energy level of the isolated hydrogen atom and $T$ and $S$ represent the transfer and overlap integrals of the neighboring $1s$ orbitals, respectively. The energy eigenvalues are
\begin{eqnarray}
E
=
\end{eqnarray}
Here we focus on the effect of the overlap integral $S$ on the energy levels. We see that the excited state energy diverges as $S$ approaches to unity. This effect is well explained with Fig.~\ref{fig:}(a). With only two $1s$ orbitals considered, the eigenfunctions are their sums with phase changes of $0$ or $pi$. When $S$ is close to $1$, the interatomic distance is close as well and the latter anti-bonding wavefunction has large amplitude in tiny ``exclusive or" space spanned by the two $1s$ orbitals. This state thus has a large kinetic energy. 

Next we consider the linear periodic chain of H. The corresponding secular equation reads
\begin{eqnarray}
\left[
\begin{array}{ccccc}
E_{0} & T & 0 & 0& \dots \\
T & E_{0} & T & 0 &\dots \\
0 & T & E_{0} & T & \dots \\
\vdots & \vdots & \ddots & \ddots & \ddots \\
\end{array}
\right]
\Psi
=
E
\left[
\begin{array}{ccccc}
1 & S & 0 & 0& \dots \\
S & 1 & S & 0 &\dots \\
0 & S & 1 & S & \dots \\
\vdots & \vdots & \ddots & \ddots & \ddots \\
\end{array}
\right]
\Psi
.
\end{eqnarray}
It hosts an eigenstate formed by the sum of the $1s$ orbitals with alternate sign changes [Fig.~\ref{fig:}(b)]. Again, its wavefunction has large amplitude in a fragmented ``xor" space, with which it has large kinetic energy. Furthermore, with chain length $N$ larger than 2, an anomalous phenomenon may occur. If $S>1/2$, the rank of the overlap matrix is less than $N$. This means that the overlap between the orbitals are so large that they are no more linearly independent. In the vicinity of the anomaly $S\lesssim 1/2$, physically such a thing occurs: If we try to make eigenstates with the $1s$ orbitals, at least one of them has diverging kinetic energy due to the fragmentation of the orthogonal complement space of the other eigenstates. Beyond the threshold, the dimension of the $1s$ orbital space is less than $N$, meaning that the system can host only less than $N$ physical states.

We heuristically found that the eigenvalues of the overlap matrix with the nearest overlap $S$ are given as real parts of the points of $1/N$ section of the circle in the complex space $|z-1|=2S$ with one fixed point $z=1+2S$. With this fact, the critical overlap $S_{\rm c}(N)$ depends on whether the chain length $N$ is even or odd. If $N$ is even $S_{\rm c}(N)=1/2$, whereas if $N$ is odd $S_{\rm c}(N)=1/(2{\rm cos}(\pi/N))$, both converge to $1/2$ with $N\rightarrow \infty$.

In H$_{2}$ divergence of this kind occurs only when $S=1$; an unphysical situation where the two H atoms completely overlap. In contrast, in the polyatomic chain this can indeed occur. The overlap integral of the neighboring hydrogen $1s$ orbitals is evaluated as a function of the interatomic distance $R$ in the atomic unit as
\begin{eqnarray}
S(R)
=
\left\{
1+
R+
\frac{1}{3}
R^2
\right\}
{\rm exp}
\left(-R
\right)
.
\end{eqnarray}
It becomes larger than $1/2$ if $R<XX$. This is a realistic value for compressed hydrides. At high pressures of order of 100~GPa, hydrogen atoms occasionally prefer to be distributed as atomic hydrogen than as dimer.

Of course, such divergence is not practically observed when we solve the effective one-particle equations such as the Kohn-Sham one, as long as we incorporate multiple basis orbitals per site. The divergence is then relaxed by deforming the original $1s$ orbitals by mixing with other orbitals. Here we show our first-principles band structures of the H-chain with the plane-wave basis code {\sc quantum espresso}. With the interatomic distance $R$ smaller than the threshold, the lowest band still has finite and linearly increasing bandwidth. The point is that, the lowest band states are no longer represented solely by the original $1s$ orbitals and, the appropriate basis orbitals {\it are forced to} deform from them. This fact is clarified by calculating the Wannier orbitals for the lowest band. With $R>R_{\rm th}$, the Wannier orbital is almost spherical but with $R<R_{\rm th}$, it has a pancake form normal to the chain axis. The same thing appears differently in the partial DOS of the $1s$ states
\begin{eqnarray}
D_{1s}(E)
=
\sum_{i}\delta(E-E_{i})|\langle \phi_{1s}| \psi_{i}\rangle|^2
,
\end{eqnarray}
where $\psi_{i}$ and $E_{i}$ are the Kohn-Sham eigenpair and $\phi_{1s}$ is the atomic $1s$ orbitals {\it provided by the code}. When $R<R_{\rm th}$, a thin high-energy tail appears in the spectrum, which indicates that some states formed by the $1s$ orbitals are forced to have large energies due to the aforementioned effect. Thus, the hydrogen $1s$ orbitals are not optimal as a basis set for describing the low-energy electronic states when the interatomic distances are small.

The threshold value $R_{\rm th}$ depends on the network structure of hydrogen atoms. To exemplify this, we analyze the overlap matrices of the three dimensional hydrogen networks extracted from the cubic H$_{3}$S and LaH$_{10}$. The former corresponds to the so called three-dimensional Lieb lattice and the latter forms clathrate cages. According to the numerical calculation, the rank of the overlap matrix is reduced by one at $S=$ ($R=$) and $ $ ($R=$) for the H$_{3}$S and LaH$_{10}$ cases, respectively. Interestingly, the interatomic distance of the H$_{10}$ clathrates is below the threshold at 200~GPa, where the superconductivity is observed~\cite{}. Although we have ignored the atomic orbitals on other sites, our results demonstrates that the deformation of the basis $1s$ atomic orbitals is indeed probable.

For later interpretation let us review how the generalized eigenvalue problem is solved.
\begin{eqnarray}
H\psi = \varepsilon S\psi
\end{eqnarray}
As $S$ is by definition Hermite, it can be diagonalized by the unitary transformation
\begin{eqnarray}
\tilde{H}\tilde{\psi} = \varepsilon D_{S}\tilde{\psi}
\end{eqnarray}
with $\tilde{H}=U_{S}HU_{S}^{\dagger}$, $\tilde{\psi}=U_{S}\psi$, and $D_{S}=U_{S}SU_{S}^{\dagger}$ being a diagonal matrix. Let us introduce a diagonal matrix $D_{S}^{1/2}$ so that $\left[D_{S}^{1/2}\right]^2=D_{S}$. The above equation is then transformed to
\begin{eqnarray}
\hat{H}\hat{\psi}
=\varepsilon \hat{\psi}
\end{eqnarray}
with $\hat{H}=\left[D_{S}^{1/2}\right]^{-1}\tilde{H}\left[D_{S}^{1/2}\right]^{-1}$ and $\hat{\psi}=D_{S}^{1/2} \tilde{\psi}$. As long as the overlap matrix is positive, the final equation is physical in that $\hat{H}$ is Hermite. An anomalous situation arises when $S$ has zero or negative eigenvalues. The existence of zero eigenvalues indicate that the original bases are not linearly independent. At this point the operator $\hat{H}=\left[D_{S}^{1/2}\right]^{-1}\tilde{H}\left[D_{S}^{1/2}\right]^{-1}$ diverges. When any of the eigenvalues of $S$ are negative, $\hat{H}$ becomes non-Hermite and is no longer physical. This reflects that the space spanned by the original bases has an unphysical dimension with negative norm. These problematic situation indicates the overcompleteness of the orbital bases and is practically cured by removing the redundant basis orbitals. In this paper, we demonstrate that, the $1s$ atomic orbitals built by the isolated atom setting, may become redundant in this sense.

In this case, suppose there is any finite set of the Wannier orbitals $\{ \psi^{\rm W}_{1}, \psi^{\rm W}_{2}, \dots\}$ which completely span any subspace spanned by the Bloch states $|\psi_{n{\bf k}}$ below any energy threshold $E$
\begin{eqnarray}
H_{\rm W}\psi^{\rm W}=\varepsilon_{\rm W}\psi^{\rm W};\ \ 
\bigoplus |\psi^{\rm W}\rangle \subset \bigoplus_{\varepsilon_{n{\bf k}}\leq E} |\psi_{n{\bf k}}\rangle
.
\end{eqnarray}
For any set of Wannier orbitals, no regular transformation can transform them to the atomic orbitals. The atomic orbital picture based on the isolated atomic basis thus becomes invalid.

The low-energy states are thus no longer completely spanned by the $1s$ orbitals and, in turn, different orbital states come into play. We can find such phenomena in a variety of preceding studies. Motta and coworkers have conducted various accurate first-principles calculations for H-chain~\cite{} and reported a insulator-to-metal transition by decreasing $R$. This is due to the energy lowering of the bands with $p$ character. We now understand that this phenomenon is a promising consequence of the overlap-driven divergence and its compensation. Another examples are seen in calculated Wannier orbitals in hydrides. Quan and Pickett~\cite{} and Akashi~\cite{} derived the effective Wannier models for H$_{3}$S with cubic structure. The H-centered $s$-like Wannier orbitals therein are commonly deformed from the ideal sphere and have complicated node structures, conforming to the cubic symmetry. Wang?~\cite{} found that the interstitial Wannier orbitals instead of the H-centered $s$-like ones effectively reproduce the low-energy bands. These are all consequences of the overlap divergence. Note that we eventually prove that the Slater-Koster modeling, in which the model parameter respects only the spherical harmonics, is ineffective for compressed hydrogen networks because of the enforced orbital deformation. 

To concentrate on the simplest aspect of the orbital overlap in compressed hydrides, we only considered the hydrogen $1s$ orbitals, but the phenomenon, originating from the rank reduction of the orbital overlap matrix, is obviously applicable to any kinds of orbitals. Extended theory is left as a future subject, we can at least expect that the orbital deformation, or even involvement of significantly different orbitals such as interstitial states, generally occur, only if the system is anomalously compressed. In this situation usual atomic orbital view on the chemical bonding must be broken down. The high-pressure chemistry~,\cite{} where unusual atomic coordination and low-energy interstitial states arise, is apparently owed largely by the current orbital overlap effect. This point should be worthy of further investigation.

\begin{acknowledgments}
This work was supported by (hydrogenomics) and JSPS KAKENHI Grant Numbers 20K20895 from Japan Society for the Promotion of Science (JSPS). 
\end{acknowledgments}

\bibliography{reference}

\end{document}



