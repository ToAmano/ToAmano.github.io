\documentclass[a4j]{jarticle}
\title{量子力学}
\usepackage{amsmath}	% required for `\align*' (yatex added)
\begin{document}

\section{前期量子論}
\subsection{ボーアの量子化}
アインシュタインによって光子の運動量やエネルギーが
\begin{align*}
 p&=\hbar k &E=pc
\end{align*}
によって与えられた.ドブロイはそれを拡張して,一般の物質についても$k=p/c$によって波数を定め,対応する波長
\begin{align*}
 \lambda =\frac{2\pi}{k}=\frac{h}{p}
\end{align*}
をドブロイ波長という.

ラザフォード模型を考えよう.これは原子核の周りを電子が等速円運動しているという模型である.(本当は輻射によってエネルギーが減るので,この模型はありえないのである.その辺はまた後で)



ボーアの量子化条件は,電子の角運動量が$n\hbar$に量子化されるものである.
\begin{align*}
 mvr=n\hbar
\end{align*} 
これと,ニュートンの方程式
\begin{align*}
 m\frac{v^2}{r}=\frac{e^2}{4\pi \epsilon_0 r^2}
\end{align*}
を連立すると,$v$,$r$が
\begin{align*}
 v=\\
 r=
\end{align*}
と求まり,そこからエネルギーが
\begin{align*}
 E&=\frac{mv^2}{2}-\frac{e^2}{4\pi \epsilon_0r}=\frac{-mv^2}{2}\\
 &=
\end{align*}
ともとまる.特に基底状態$n=1$の時のエネルギーが有名な
\begin{align*}
 E_1=-13.6\,[\mathrm{eV}]
\end{align*}
である.この時の半径$r$をボーア半径と言って
\begin{align*}
 a_0=\frac{4\pi \epsilon_0 \hbar^2}{m_ee^2}=0.529 A
\end{align*}
となる.

\end{document}