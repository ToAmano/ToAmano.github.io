\documentclass[a4j]{jarticle}
\title{力学公式集}

\begin{document}
\section{単位について}
\subsection{SI接頭辞}
小さい方
\begin{itemize}
 \item $m=10-3$
 \item $\mu=10-6$
 \item $n=10-9$
 \item $p=10-12$
 \item $f=10-15$
 \item $a=10-18$
\end{itemize} 

大きい方
\begin{itemize}
 \item $k=10+3$
 \item $M=10+6$
 \item $G=10+9$
 \item $T=10+12$
\end{itemize}

\subsection{単位}
電磁気の単位は非常にわかりにくいので,ここにまとめておく.流れとしてはまずSI単位を取り上げ,その後頻出するcgsガウス単位系や自然単位系を扱うことにしよう.


\subsection{覚えるべき物理量}
\begin{itemize}
 \item  プランク定数 $6.582e-16eVs$ 角運動量の単位を持つ
 \item  $\hbar c =200MeV fm$
 \item ボルツマン定数 $8.617e-5eV/K$ $k_BT$がエネルギーの単位を持つことに注意
 \item ボーア半径$0.529A$ オームストロングは$10-10m$である
 \item 素電荷$1.602e-19C$
\end{itemize}





\end{document}