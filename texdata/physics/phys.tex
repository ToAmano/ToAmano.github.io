\documentclass[a4j]{jarticle}
\usepackage{amsmath}	% required for `\align*' (yatex added)
\usepackage{bm}	% required for `\bm' (yatex added)
\begin{document}
\part{物理数学}
\section{フーリエ変換}
 多粒子系に対するフーリエ変換の応用
 

\part{古典力学}





\part{電磁気}
\section{静電気学}
\subsection{クーロンの法則}
二つの点電荷に働く力は
\begin{align*}
 F=\frac{1}{4\pi \epsilon_0}\frac{q_1q_2}{r^2}
\end{align*}
で与えられる.これをクーロンの法則という.点電荷による電場を
\begin{align*}
 E=\frac{1}{4\pi \epsilon_0}\frac{q}{r^2}
\end{align*}
で与えると,他の電荷に働く力は
\begin{align*}
 F=q_2E
\end{align*}
で与えらる.これはローレンツ力の一部である.

次のsubsectionでやる,ガウスの法則とファラデーの法則は,クーロン電場を内包しているが,ローレンツ力はクーロンの法則によって導出される.(Maxwell方程式には含まれていない.)


  \subsection{静電場の問題}
   静電場の問題は,二つのMaxwell equation によって記述される.それはガウスの法則と,ファラデーの法則(渦なし)である.
\begin{align*}
 \nabla \cdot {\bm E}&=\frac{\rho}{\epsilon_0}\\
 \nabla \times {\bm E}&={\bm 0}
\end{align*}
渦なし流にはスカラーポテンシャル$\phi$が存在して
\begin{align*}
 E=-\nabla\phi
\end{align*}
とかける.従って方程式を$\phi$で書き直せば,Laplace eq
\begin{align*}
 \Delta \phi =\frac{-\rho}{\epsilon_0}
\end{align*}
を得る.

電場を求める問題には,上にあげたLaplace eq を解く方法(これは境界値問題,あるいはGreen function の問題として一般に解くことができる.)の他に,対称性が高ければGauss の法則を用いる方法,そして最後にcoulomb積分
\begin{align*}
 \phi({\bm r})=\frac{-1}{4\pi\epsilon_0}\int dV' \frac{\rho}{|{\bm r}-{\bm r'}|}
\end{align*}
を実行する方法がある.(厳密に言えば,これはGreen functionによる解法であるが,便宜上分けておくのが良いだろう.)

特にGauss の法則で求められる場はいくつか覚えておいた方が良いだろう.

\begin{itemize}
 \item 球対称の場合$E=Q/4\pi\epsilon_0 r^2$
 \item 無限に長い線$E=\lambda /2\pi\epsilon_0 r$
 \item 無限に広がった面$E=\sigma /2\epsilon_0$
\end{itemize}

  \subsection{静磁場}



  
\subsection{マクスウェル方程式}
真空中のマクスウェル方程式は
\begin{align*}
 \nabla\cdot\bm{E}=\rho /\epsilon_0
\end{align*}
\subsection{荷電粒子の運動}
荷電粒子のラグランジアンは
\begin{align*}
 L=\frac{1}{2}mv^2-q\phi +\bm{v}\cdot \bm{A}
\end{align*}
対応してハミルトニアンは
\begin{align*}
 H=\frac{1}{2m}\left(\bm{p}-q\bm{A}\right)^2+q\phi
\end{align*}
である.




\section{量子力学}
\subsection{基礎}
確率流密度は
\begin{align*}
 J=\frac{\hbar}{m}Re\phi^{\ast}\nabla\phi
\end{align*}
で与えられ,確率密度$P=|\phi^2|$と共に連続の式
\begin{align*}
 \frac{\partial P}{\partial t}+\nabla\cdot J=0
\end{align*}
を満たす.確率流密度の覚え方であるが,単位を考えるとわかりやすい.確率流密度の単位は$J=Pv$であり,これが$\hbar/m=Lv$からナブラの分の$L$を割ったものに等しいからだ.


\subsection{ポテンシャル障壁}
ポテンシャル障壁(E<V)では,透過率$T$,反射率$R$に対して$T+R=1$なるトンネル効果が現れる.これはシュレーディンガー方程式による直接の効果である.
また,$V<E$では,障壁中で定常派が出来る時すなわち障壁の長さが波長の半整数倍の時に$T=1$となる共鳴効果が見られる.

より一般の障壁においては,(障壁が十分に大きい時)透過確率は以下のがモフ因子で与えられる.これはWKB近似で得られる結果と一致する.
\begin{align*}
 T=\exp\left[\frac{-2}{\hbar}\int_{x_1}^{x_2}\sqrt{2m(V(x)-E)}dx\right]
\end{align*}



 \section{固体物理学}
 
  \subsection{フェルミ気体の物理}
  \subsection{結晶}
 \subsection{フォノン}
    弾性波の量子化



\end{document}