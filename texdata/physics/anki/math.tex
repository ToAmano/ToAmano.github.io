\documentclass[a4j]{jarticle}
\title{数学暗記事項}
\usepackage{amsmath}	% required for `\align*' (yatex added)
\begin{document}
\maketitle

\section{微積分}
オイラーマクローリンの総和公式

\subsection{ガンマ関数}
ガンマ関数は
\begin{align*}
 \Gamma (n+1)=\int_{0}^{\infty}e^{-t}t^ndt
\end{align*}
で定義される.いくつかの著しい値$\Gamma (1)=1$および$\Gamma (1/2)=\sqrt{\pi}$および
\begin{align*}
 \Gamma (n+1)=n\Gamma (n)
\end{align*}
から整数および半整数に対するガンマ関数の値は得られる.またガンマ関数の相反公式
\begin{align*}
 \Gamma (x)\Gamma (1-x)=\frac{\pi}{\sin \pi x}
\end{align*}
は$x=1/2$で成立することを覚えて係数を調整すれば思い出せる.

ガンマ関数

\section{線形代数}
ブロック行列
\begin{align*}
 H=\left(
 \begin{array}{cc}
  A&B \\
  C&D \\
 \end{array}\right)
\end{align*}
の行列式は$A$が正則な時
\begin{align*}
 |H|=|A||D-CA^{-1}B|
\end{align*}
で与えられる.これは二次元行列の行列式$ad-bc$および右回りと覚えておけば良い.またこの結果から特にブロック三角行列$C=0$に対して$|H|=|A||D|$という著しい結果を与える.

行列$A$,$B$,$C$,$D$に対して
\begin{align*}
 \left(A+BCD\right)^{-1}=A^{-1}-A^{-1}B\left(D^{-1}+CA^{-1}B\right)^{-1}CA^{-1}
\end{align*}
が成立する.これを逆行列の補助定理(Woodburyの恒等式)という.

\end{document}