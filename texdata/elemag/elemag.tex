\documentclass[a4j]{jarticle}
\title{電磁気公式集}

\usepackage{amsmath}	% required for `\align*' (yatex added)
\begin{document}
\section{荷電粒子の運動}
非相対論的な場合は運動方程式
\begin{align*}
 m\dot{v}=q(E+v\times B)
\end{align*}
に,相対論的な場合は運動方程式
\begin{align*}
\frac{dp}{dt}=q(E+v\times B)
\end{align*}
に従う.これに加えて時と場合によっては粒子の放射が無視できない場合がある.

\subsection{非相対論的な運動~静電場政治馬の場合~}
まず非相対論的な場合から行こう.定常磁場$(0,0,B)$がある時,粒子は$xy$平面内でサイクロトロン角振動数
\begin{align*}
 \omega_c=\frac{qB}{m}
\end{align*}
の円運動をする.ここに定常電場が加わると,円運動に電場ドリフトと呼ばれる等速度運動
\begin{align*}
 v=\frac{E_{suitixyoku}\times B}{B^2}
\end{align*}
が加わる.




\end{document}