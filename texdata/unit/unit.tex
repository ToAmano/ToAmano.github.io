\documentclass[a4j]{jarticle}
\title{単位に関するレビュー}
\author{}
\date{}

\usepackage{amsmath}	% required for `\align*' (yatex added)
\begin{document}
\maketitle
\section{プランク単位(自然単位)}
プランク単位系では,以下の物理定数をdimensionlessな定数1として考える.
 \begin{table}[htb]
  \centering
 \begin{tabular}{|c|c|c|} \hline
  定数&記号 &単位( SI) \\ \hline
  光速度&c&LT-1 \\ \hline
  万有引力定数&G&M-1L3T-2\\ \hline
  ディラック定数&$\hbar$&ML2T-1\\ \hline
  クーロン定数&$1/4\pi\epsilon_0$&Q-2ML3T-2\\ \hline
  ボルツマン定数&$k_B$&ML2T-2K-1 \\ \hline
 \end{tabular}
 \end{table}
 
 本来はより深い議論があるのだが,実用上は,これによって全ての単位をエネルギーeVで測るのが便利である.例えば質量は
 \begin{align*}
  E=mc^2\to m
 \end{align*}
 によってeVの単位を持つし,時間は
 \begin{align*}
  \hbar=6.582\times 10^{-6}\,\mathrm{eVs}
 \end{align*}
によって1/eVの単位をもち,これによって$c=1$から,長さも1/eVの単位を持つことがわかる.

一方電気が関係する単位は,

 
\section{自然定数}



\end{document}