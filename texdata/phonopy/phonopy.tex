\documentclass[a4j]{jarticle}
\title{Phonopyの使い方with QE}

\usepackage{listings}	% required for `\lstlisting' (yatex added)
\usepackage[dvipdfmx]{graphicx}	% required for `\includegraphics' (yatex added)
\begin{document}
\section{PHONOPYの概要}
phonopyは直接法によってフォノンのDOSや分散を計算することができる.
この計算のためにはまずはSCF計算が必要で,ここではその計算をQEでやる場合の説明をする.



\section{スーパーセルを作る}
まず,phonopyでは基本単位を与えさえすれば,指定したサイズのsupercellを作ってくれる.
ここでは例としてSiの例で考えてみよう.まず,普通のSiのinputを用意する.ただしいくつか注意がある.
\begin{enumerate}
 \item ibrav=0のみが許される.
 \item ATOMICPOSITIONSはcrystal座標($a_1$から$a_3$の何倍か)のみが許される.
 \item ATOMICPOSITIONSなどのあとのoptionは,カッコをつけてはいけない.
\end{enumerate}

\begin{lstlisting}
&control
    calculation = 'scf'
    verbosity = 'high'
    restart_mode='from_scratch',
    pseudo_dir = '/home/amano/pseudo/'
 /
 &system
    ibrav = 0
    nat = 2
    ntyp = 1
    occupations = 'smearing'
    smearing = 'gaussian'
    degauss = 0.01
    ecutwfc = 100.0
 /
 &electrons
    diagonalization = 'david'
    conv_thr = 1.0d-8
    mixing_beta = 0.7
 /
ATOMIC_SPECIES
 Si  28.086  Si.pbe-n-kjpaw_psl.1.0.0.UPF

ATOMIC_POSITIONS crystal
 Si   0.000 0.00 0.000
 Si   -0.25 0.75 -0.25 


CELL_PARAMETERS angstrom
 -2.73308  0.00000  2.73308
  0.00000  2.73308  2.73308
 -2.73308  2.73308  0.00000

K_POINTS automatic
 8 8 8 1 1 1

\end{lstlisting}
Siはダイヤモンド構造であり,unit cellとして何を使うかはBCC,FCC,SCがあり得るが,ここではFCCを使ってみよう.
そしてQEで用いられているFCCの格子ベクトルを採用した.その様子を示したのが下図
\begin{figure}[htb]
 \centering
\end{figure}




そして,単純なsuper cellを作るのには次のようにする.
\begin{lstlisting}
 phonopy --qe -d --dim="2 2 2" -c Si.in
\end{lstlisting}
--qeでQuantum espressoを用いることを明示する.代わりに--pwscfでも良い.

-cのoptionはinputfileの名前を指定するのに必要で,これがないとデフォルトでunitcell.inをinputfileとしてしまう.


また,--dimでsuper cellのサイズを指定している.ここでは$2\times 2\times 2$のsupercellを作る.(ただし,ここで使っているのはFCCのprimitive cellなので,conventional cellの2倍でしかないことに注意しよう.
また,最後にオプション -vを入れると全部の構造をアウトプットしてくれるのでおすすめ.この時のアウトプットとしてprimitive cellとunit cellというのが出てくるが,どっちも同じ結果が出てくるように見える.

こうして出てくるアウトプットはsupercell.inとsupercell-数字.inである.supercell.inには単純に元の構造のsupercellが入っている.
supercell-数字.inには実際に原子をdisplaceした構造が出てくる.これらのdisplacementが施されたsupercellに対してscf計算を行う,つまり,はじめのdisplaceを作る部分はここで完了してしまっているのである!! 実際のdisplaceの数は愚直に$2\times 2\times 2=8$個も作るのではなく,対称性を上手く使って減らしてしまっている.実際,今のSiの例だとたった1つである.
\begin{lstlisting}
!    ibrav = 0, nat = 16, ntyp = 1
CELL_PARAMETERS bohr
  -10.3295454212955349    0.0000000000000000   10.3295454212955349
    0.0000000000000000   10.3295454212955349   10.3295454212955349
  -10.3295454212955349   10.3295454212955349    0.0000000000000000
ATOMIC_SPECIES
 Si   28.08550   Si.pbe-n-kjpaw_psl.1.0.0.UPF
ATOMIC_POSITIONS crystal
 Si   0.0000000000000000  0.0000000000000000  0.0000000000000000
 Si   0.5000000000000000  0.0000000000000000  0.0000000000000000
 Si   0.0000000000000000  0.5000000000000000  0.0000000000000000
 Si   0.5000000000000000  0.5000000000000000  0.0000000000000000
 Si   0.0000000000000000  0.0000000000000000  0.5000000000000000
 Si   0.5000000000000000  0.0000000000000000  0.5000000000000000
 Si   0.0000000000000000  0.5000000000000000  0.5000000000000000
 Si   0.5000000000000000  0.5000000000000000  0.5000000000000000
 Si   0.8750000000000000  0.3750000000000000  0.8750000000000000
 Si   0.3750000000000000  0.3750000000000000  0.8750000000000000
 Si   0.8750000000000000  0.8750000000000000  0.8750000000000000
 Si   0.3750000000000000  0.8750000000000000  0.8750000000000000
 Si   0.8750000000000000  0.3750000000000000  0.3750000000000000
 Si   0.3750000000000000  0.3750000000000000  0.3750000000000000
 Si   0.8750000000000000  0.8750000000000000  0.3750000000000000
 Si   0.3750000000000000  0.8750000000000000  0.3750000000000000 
\end{lstlisting}


\begin{lstlisting}
 !    ibrav = 0, nat = 16, ntyp = 1
CELL_PARAMETERS bohr
  -10.3295454212955349    0.0000000000000000   10.3295454212955349
    0.0000000000000000   10.3295454212955349   10.3295454212955349
  -10.3295454212955349   10.3295454212955349    0.0000000000000000
ATOMIC_SPECIES
 Si   28.08550   Si.pbe-n-kjpaw_psl.1.0.0.UPF
ATOMIC_POSITIONS crystal
 Si   0.0013690956423480  0.0000000000000000  0.0000000000000000
 Si   0.4999999999999999  0.0000000000000000  0.0000000000000000
 Si   0.0000000000000000  0.4999999999999999  0.0000000000000000
 Si   0.4999999999999999  0.4999999999999999  0.0000000000000000
 Si   0.0000000000000000  0.0000000000000000  0.4999999999999999
 Si   0.4999999999999999  0.0000000000000000  0.4999999999999999
 Si   0.0000000000000000  0.4999999999999999  0.4999999999999999
 Si   0.4999999999999999  0.4999999999999999  0.4999999999999999
 Si   0.8750000000000000  0.3750000000000000  0.8750000000000000
 Si   0.3750000000000000  0.3750000000000000  0.8750000000000000
 Si   0.8750000000000000  0.8750000000000000  0.8750000000000000
 Si   0.3750000000000000  0.8750000000000000  0.8750000000000000
 Si   0.8750000000000000  0.3750000000000000  0.3750000000000000
 Si   0.3750000000000000  0.3750000000000000  0.3750000000000000
 Si   0.8750000000000000  0.8750000000000000  0.3750000000000000
 Si   0.3750000000000000  0.8750000000000000  0.3750000000000000
\end{lstlisting}

また,もう1つdisp.yamlというファイルもできる.ここにはスーパーセルとdisplacementの情報が入っている.この点については後でもう一度帰ってこよう.今はこのファイルも後の計算で使うので,current dirに絶対にないといけないことだけ覚えておこう.



さて,これらのアウトプットファイルにはcell parameter とatomic speciesおよびatomic positionsしかないので,scf計算を行うにはこれらに色々設定を加える必要がある.これを簡単にやるには,header.inにこの部分を書いてしまって,コマンドで一括でこれらを結合してしまうことである.

注意:ということは,Si.inにもこれだけの情報があれば十分なのでは?

tstressは,系にかかる圧力を計算してくれる.もしあまりに変な格子定数を考えていると,scf計算中のこの圧力に値が異常値を示すので,おかしさに気づけるという点で保険として入れておくと良いだろう.(当然計算時間は長くなるだろうが...)また,natとntypにはしっかりスーパーセルの値を用いるようにしよう.
\begin{lstlisting}
  &control
    calculation = 'scf'
    verbosity = 'high'
    tstress = .true.
    tprnfor = .true.
    restart_mode='from_scratch',
    pseudo_dir = '/home/amano/pseudo/'
 /
 &system
    ibrav = 0
    nat = 16
    ntyp = 1
    occupations = 'smearing'
    smearing = 'gaussian'
    degauss = 0.01
    ecutwfc = 100.0
/
 &electrons
    diagonalization = 'david'
    conv_thr = 1.0d-8
    mixing_beta = 0.7
 /
K_POINTS automatic
 8 8 8 1 1 1
\end{lstlisting}

ファイルの結合には普通にshellスクリプトを用いれば良いだろう.

\begin{lstlisting}
for i in {001,002};do cat header.in supercell-$i.in >| NaCl-$i.in; done 
\end{lstlisting}


ここまでしておいて,scf計算を行う.これは普段通り

\begin{lstlisting}
 mpirun -np 8 pw.x  < Si-001.in > Si-001.out
\end{lstlisting}

とすれば良い.

注意:ここでシリコンなのになぜかFermi Energyが観測されている...

\begin{lstlisting}
 !    total energy              =    -747.62948061 Ry
 the Fermi energy is     6.3082 ev
\end{lstlisting}

次に,FORCE_SETSの計算は,phonopyを用いて行うことができる.この計算にはdisp.yamlとscf.outを用いるらしい.
\begin{lstlisting}
 phonopy --qe -f NaCl-001.out NaCl-002.out
\end{lstlisting}

実行すると以下のように出る.
\begin{lstlisting}
 Calculator interface: qe
Displacements were read from "phonopy_disp.yaml".
1. Drift force of "Si-1.scf.out" to be subtracted
  0.00000000  -0.00000000  -0.00000000
FORCE_SETS has been created.
\end{lstlisting}
こうしてFORCE SETSというファイルができる.このファイル形式はこんな感じ
\begin{lstlisting}
 16
1

1
 -0.0141421356237310   0.0000000000000000   0.0141421356237310
   0.0038762494    0.0000684213   -0.0038762494
  -0.0002728806    0.0000010013    0.0002728806
   0.0002485794    0.0001439513    0.0001288606
  -0.0001303906   -0.0001429987   -0.0002519094
  -0.0001288606    0.0001439513   -0.0002485794
   0.0002519094   -0.0001429987    0.0001303906
   0.0000137494   -0.0000013287   -0.0000137494
  -0.0000374506   -0.0000004987    0.0000374506
  -0.0003569506   -0.0000058887    0.0003482306
   0.0000696794    0.0000009513   -0.0000718994
  -0.0016064306    0.0012467712    0.0016064306
  -0.0016492806   -0.0013037688    0.0016492806
  -0.0000017006   -0.0000729387    0.0000017006
   0.0000001094    0.0000703113   -0.0000001094
  -0.0003482306   -0.0000058887    0.0003569506
   0.0000718994    0.0000009513   -0.0000696794
\end{lstlisting}

こうしてようやく,メインとなるコマンドを実行することができる.そのためには,unit cellの入力Si.inと,band confという新しいファイルが必要.これは計算に関するオプションを指定するもので,スーパーセルのサイズ,primitive axis (primitive cell の3つの基本ベクトルの方向),BANDのK-pathの3つの情報を含んでいる.


 phonopy --pwscf -c Si.in -p --dim="2 2 2" --pa="0 1/2 1/2 1/2 0 1/2 1/2 1/2 0" --band="1/2 1/2 1/2 0 0 0 1/2 0 1/2"

\end{document}