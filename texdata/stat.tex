\documentclass[a4j]{jarticle}
\title{統計力学}


\usepackage{amsmath}	% required for `\align*' (yatex added)
\begin{document}
\maketitle

\section{ギブス集団}
統計力学の基礎となるのは,ギブス集団の考え方である.
\subsection{ミクロカノニカルアンサンブル}


\section{自由電子}

ドハースファンアルフェン効果 磁化率が磁場の逆数で振動する.自由電子の場合,その振動周期は
\begin{align*}
 \Delta \left(\frac{1}{B}\right)=\frac{2\mu_B}{\mu}
\end{align*}
で与えられる.ただし$\mu$は化学ポテンシャル,$\mu_B$はボーア磁子である.ちなみにボーア磁子はサイクロトロン振動数と
\begin{align*}
 \hbar \omega_c=2\mu_BB
\end{align*}
の関係にある.これは有用な関係で,ここから覚えていなくてもボーア磁子の表式を導出できる.


シュグニコフドハース振動 電気抵抗が磁場の逆数で振動する.周期は全く同じ.

\section{超電導}
\subsection{超電導概要}
第1種超電導と第二種超電導の違いは,磁場の侵入に対して非常に特徴的である.理論的にはこれらの違いを特徴づけるのは上電動状態での伝導電子の平均自由工程である.平均自由工程が短いほどコヒーレンス長さは短く,侵入長は長い.従ってその比$\lambda /\xi$が$1$より小さいなら第1種,$1$より大きいなら第二種である.



超電導で本質的なのは超伝導体内部で磁場が$0$になることである(マイスナー効果).これから電磁気の知識を用いていくつかの結果を導くことができる.

超電導は磁化Mに跳びがあるので第2種相転移

エネルギーギャップの存在

\subsection{ロンドン方程式}
ロンドン方程式は超電導を扱うための現象論方程式であり,より高度なBCS理論から導出することができる.しかしこれを認めればいくつかの現象を説明できる.ロンドン方程式は電流がベクトルポテンシャルに比例するという式である.
\begin{align*}
 {\bf j}=-\frac{1}{\mu_0\lambda_L^2}A
\end{align*}
まず,この式と定常状態のアンペールの法則を連立することで超伝導体の内部で$\lambda_L$の長さで磁場が減衰することを示せる.これがマイスナー効果である/$\lambda_L$はロンドンの侵入長.

リング状の



\subsection{ジョセフソン効果}
超伝導体を薄い絶縁体で挟む.するとそこに電流が流れることを直流ジョセフソン効果という.両側の超伝導体に電圧$V$をかける.これが直流だと電流は(正味で)全く流れない.しかし交流の場合,その振動数が
\begin{align*}
 \omega =\frac{qV}{\hbar}
\end{align*}
の時のみ電流が流れる.これを交流ジョセフソン効果という.これらの関係式は超伝導体の波動関数
\begin{align*}
 \phi (r)=\sqrt{\rho}e^{i\theta}
\end{align*}
およびとびうつりの振幅があると考えて二準位系のシュレーディンガー方程式をとくことにより得られる.


\section{線形応答理論}

  \subsection{久保公式}
  外場$F$が摂動として扱える時,そのポテンシャルを$V$として,とある物理量$B$の平均値は
  \begin{align*}
   <B>_{all}=<B>_0+\frac{i}{\hbar}\int_{0}^{\infty}<[V(t'),B_I(t-t')]>_0dt'
  \end{align*}
  で表される.ただし$<>_0$は平衡状態(外場なし)の時の統計平均を意味する.したがって$V(t)=-AF(t)$と表される時,
  \begin{align*}
   <B>_{all}=<B>_0+\frac{i}{\hbar}\int_{0}^{\infty}<[B_I(t-t'),A]>_0F(t')dt'
  \end{align*}
  となり,物理量$B$の応答関数は
  \begin{align*}
   \Phi =
  \end{align*}
  で与えられる.これは物理量$A$と物理量$B$の相関関数を表しており,久保公式は摂動に対する物理量の変化は相関関数で与えられると解釈できる.特に$A=B$の場合,
  \begin{align*}
   \Phi =
  \end{align*}
  



\end{document}