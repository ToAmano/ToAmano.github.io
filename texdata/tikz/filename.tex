%直接pdf化するためのコード
%これで,コマンドとしては
%uplatex --jobname=fig filename.tex



% TikZのパッケージ
\usepackage{tikz}
\usepackage{circuitikz}
\pgfrealjobname{filename}%このソースコードのファイル名(.texはいらない) また,これはusepackage{tilz}の後へ!
\usetikzlibrary{calc,intersections,positioning,shapes,arrows}

% 適切な位置に図を入れる
\begin{figure}\centering
      \beginpgfgraphicnamed{fig}% ここから\endpgfgraphicnamedまでの図を別で描き出す
      \begin{circuitikz}[american voltages]
         %-----------------------------
         %各点の位置を定義
         %-----------------------------
         \coordinate (O) at (0,0);%原点
         \coordinate (X) at (5,0);%幅
         \coordinate (X2) at ($(O)!0.5!(X)$);%幅の二等分線
         \coordinate (Y) at (0,5);%高さ
         \coordinate (Y2) at ($(O)!0.5!(Y)$);%高さの二等分線
         \coordinate (XY) at (X|-Y);%図の右上端
         %-----------------------------
         %ここから描いていく
         %-----------------------------
         \draw (O) node[ground] {} -- (X) ;% アース
         \draw (Y) node[left] {High$[V]$} -- (XY) ;% High
         \draw (Y-|X2)[short,*-] to [european resistor] (Y2-|X2); % Highと抵抗を接続
         \draw (Y2-|X2) ++ (0, -1) node[nmos] (mos) {}
         (mos.gate) node[anchor=south] {G} to[short, -o] (mos.gate -| O) node[below right]{入力電圧}% 抵抗と接続
         (mos.drain) node[anchor=east] {D} to (Y2-|X2) to [short,*-o] (Y2-|X) node[above left]{出力電圧}
         (mos.source) node[anchor=east] {S}[short,-*] to (X2); % アースと接続
      \end{circuitikz}
      \endpgfgraphicnamed% \beginpgfgraphicnamedから,ここまでの図を別で描き出す
   \caption{電界効果トランジスタを用いるNOTゲートの例}
\end{figure}
% 図ここまで
