\documentclass[a4j]{jarticle}


\usepackage{feynmp}	% required for `\fmffile' (yatex added)
\begin{document}
\unitlength=1mm



%fmffileは外部latexファイルから参照できるので,ここに色々まとめておくこと.


\section{単純相互作用}

%クーロン相互作用
\begin{fmffile}{ee}
 \begin{fmfgraph}(40,25)
 \fmfleft{i1,i2}
 \fmfright{o1,o2}
 \fmf{fermion}{i1,v1,o1}
 \fmf{fermion}{i2,v2,o2}
 \fmf{photon}{v1,v2}
 \fmfdot{v1,v2}
 \end{fmfgraph}
\end{fmffile}


%電子格子相互作用
\begin{fmffile}{eph}
 \begin{fmfgraph}(40,25)
 \fmfleft{i1}
 \fmfright{o1,o2}
 \fmf{fermion}{i1,v1,o1}
 \fmf{photon}{v1,o2}
 \fmfdot{v1}
 \end{fmfgraph}
\end{fmffile}

 

\section{一粒子グリーン関数}

%bareフォック項
%fock項の様に曲がった線を書くためには,left,rightオプション,及びtensionで線を長くすることが必要である.
%fock項を綺麗に書くためには,fock部分の電子線を若干長くすること.
 \begin{fmffile}{fock}
 \begin{fmfgraph}(40,25)
 \fmfleft{i}
 \fmfright{o}
 \fmf{fermion}{i,v1,v2,o}
 \fmf{photon,left,tension=1/4}{v1,v2}
 \fmfdot{v1,v2}
 \end{fmfgraph}
 \end{fmffile}


%bareハートリー項
\begin{fmffile}{hartree}
 \begin{fmfgraph}(40,40)
 \fmfleft{i}
 \fmfright{o}  
 \fmf{fermion}{i,v1,o}
 \fmf{photon,right}{v1,v2}
 \fmf{fermion}{v2,o}
 %\fmf{fermion,right=90}{v2,v2}
 \fmfdot{v1,v2}
 \end{fmfgraph}
\end{fmffile}


%bareボルン項1
%フェルミオン線の長さが上手く行っていない
\begin{fmffile}{born1}
 \begin{fmfgraph}(40,25)
 \fmftop{i1,o1}
 \fmfbottom{i,o}
 \fmf{fermion}{i,v1,v2,o}
 \fmf{phantom}{i1,p1}
 \fmf{phantom}{o1,p2}
 \fmf{fermion,left,tension=0.8}{p1,p2}
 \fmf{fermion,left,tension=0.8}{p2,p1}
 \fmf{photon}{v1,p1}
 \fmf{photon}{p2,v2}
 \fmfdot{v1,v2,p1,p2}
 \end{fmfgraph}
\end{fmffile}


%bareボルン項2
%本当はフェルミオン線を横直線にしたい.
\begin{fmffile}{born2}
 \begin{fmfgraph}(40,25)
 \fmfleft{i}
 \fmfright{o}
 \fmf{fermion}{i,v1}
 \fmf{fermion,tension=1/2}{v1,v2}
 \fmf{fermion,tension=1/2}{v2,v3}
 \fmf{fermion,tension=1/2}{v3,v4}
 \fmf{fermion}{v4,o}
 %\fmf{fermion,tension=1/3}{v1,v2}
 %\fmf{fermion}{v2,v3}
 %\fmf{fermion}{v3,v4}
 %\fmf{fermion}{v4,o}
 \fmf{photon,left,tension=1/4}{v1,v3}
 \fmf{photon,right,tension=1/4}{v2,v4}
 \fmfdot{v1,v2,v3,v4}
 \end{fmfgraph}
\end{fmffile}

%e-ph vertex correction
%vertexの形があまりうまくいっていない.
\begin{fmffile}{fock-vertex}
 \begin{fmfgraph}(40,25)
 \fmfleft{i}
 \fmfright{o}
 \fmf{fermion}{i,v1,v2}
 \fmf{fermion}{v3,o}
 \fmf{photon,left,tension=1/3}{v1,v4}
 \fmfdot{v1} 
%ここにvertexを入れる
 \fmfpoly{default,filled=full}{v4,v2,v3}
 \end{fmfgraph}
\end{fmffile}



\section{二粒子green関数}

%二粒子green関数,Hartree(失敗版)
%この様な平行線の入るダイヤグラムの場合,普通にやると矢印が途切れてしまう.
\begin{fmffile}{twoparticlefalse}
 \begin{fmfgraph}(40,20)
 \fmfleft{i1,i2}
 \fmfright{o1,o2}  
 \fmf{fermion}{i1,v1,o1}
 \fmf{fermion}{i2,v2,o2}
 \end{fmfgraph}
\end{fmffile}

%二粒子green関数,Hartree(成功版)
%矢印が途切れてしまうのを回避するため,ダミーのvertexを導入すると良い.
\begin{fmffile}{twophartree}
 \begin{fmfgraph}(40,20)
 \fmftop{i1,o1}
 \fmfbottom{i2,o2}  
 \fmf{fermion}{i1,v1,o1}
 \fmf{fermion}{i2,v2,o2}
 \end{fmfgraph}
\end{fmffile}


%二粒子green関数,Fock
%これはvertexを交差させる必要がある.これにはtop,bottomの順番に注意しないといけない
\begin{fmffile}{twopfock}
 \begin{fmfgraph}(40,25)
 \fmftop{i1,o1}
 \fmfbottom{i2,o2}  
 \fmf{fermion}{i1,v1,o2}
 \fmf{fermion}{i2,v1,o1}
 \end{fmfgraph}
\end{fmffile}

%二粒子green,Born1
\begin{fmffile}{tpborn1}
 \begin{fmfgraph}(40,20)
 \fmfbottom{i2,o2}  
 \fmftop{i1,o1}
 \fmf{fermion}{i1,v1,o1}
 \fmf{fermion}{i2,v2,o2}
 \fmf{photon,tension=0}{v1,v2}
 \fmfdot{v1,v2}
 \end{fmfgraph}
\end{fmffile}

%二粒子green,Born2
%こっちもvertexが交差する必要がある.しかし,普通にやるわけには行かない.
%新しく見えないphantomの線を追加しないと変な感じになってしまう.
\begin{fmffile}{tpborn2}
    \begin{fmfgraph*}(40,25)
        \fmftop{i1,o1}
        \fmfbottom{i2,o2}
        \fmf{fermion}{i1,v1}
        \fmf{phantom}{v1,o1} % Invisible rubber band
        \fmf{fermion}{i2,v2}
        \fmf{phantom}{v2,o2} % also invisible rubber band
        \fmf{photon}{v1,v2}
        % These are visible, but have no tension.
        \fmf{fermion,tension=0}{v1,o2}
        \fmf{fermion,tension=0}{v2,o1}
        \fmfdot{v1,v2}
    \end{fmfgraph*}
\end{fmffile}


%二粒子green,ladder
\begin{fmffile}{ladder}
 \begin{fmfgraph}(40,20)
 \fmfbottom{i2,o2}  
 \fmftop{i1,o1}
 \fmf{fermion}{i1,v1,u1,o1}
 \fmf{fermion}{i2,v2,u2,o2}
 \fmf{photon,tension=0}{v1,v2}
 \fmf{photon,tension=0}{u1,u2}
 \fmfdot{v1,v2}
 \fmfdot{u1,u2}
 \end{fmfgraph}
\end{fmffile}


%二粒子green,vertex example
\begin{fmffile}{vertex}
\begin{fmfgraph}(40,40)
\fmfpen{thick}
\fmfleftn{l}{2}
\fmfrightn{r}{2}
\fmfrpolyn{shaded,label=$\Gamma$}{G}{4}
\fmfpolyn{empty,label=$K$}{K}{4}
\fmf{fermion}{l1,G1}\fmf{fermion}{l2,G2}
\fmf{fermion}{K1,r1}\fmf{fermion}{K2,r2}
\fmf{fermion,left=.5,tension=.5}{G3,K3}
\fmf{fermion,right=.5,tension=.5}{G4,K4}
\end{fmfgraph}
\end{fmffile}


\section{行列グリーン関数}

%全体的にdbl_plain_arrowのレイアウトがよくない.

%fock
 \begin{fmffile}{nanbu-fock}
 \begin{fmfgraph}(40,25)
 \fmfleft{i}
 \fmfright{o}
 \fmf{dbl_plain_arrow}{i,v1}
 \fmf{dbl_plain_arrow,tension=1/2}{v1,v2}
 \fmf{dbl_plain_arrow}{v2,o}
 \fmf{photon,left,tension=1/4}{v1,v2}
 \fmfdot{v1,v2}
 \end{fmfgraph}
 \end{fmffile}

%vertex
\begin{fmffile}{4vertex}
\begin{fmfgraph}(40,40)
\fmfpen{thick}
\fmfleftn{l}{2}
\fmfrightn{r}{2} 
\fmfpolyn{shaded}{z}{4} 
\fmf{dbl_plain_arrow}{l2,z3} 
\fmf{dbl_plain_arrow}{l1,z4} 
\fmf{dbl_plain_arrow}{z1,r1} 
\fmf{dbl_plain_arrow}{z2,r2} 
\fmffreeze
\fmfdraw 
\fmfvn{d.siz=2thick,d.sh=circle}{z}{4}
\end{fmfgraph}
\end{fmffile}



\end{document}