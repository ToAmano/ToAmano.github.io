\documentclass[a4j]{jarticle}

\usepackage{amsmath}	% required for `\align*' (yatex added)
\usepackage{braket}	% required for `\ket' (yatex added)
\begin{document}

\section{時間に依存しない摂動論}
ハミルトニアンが,主要な部分$H_0$と小さな部分$V$に別れていたとしよう.つまり全ハミルトニアンが
\begin{align*}
 H=H_0+V
\end{align*}
とかけているとする.この時,真面目に全ハミルトニアンを対角化する必要があるだろうか?

答えは$V$が小さければNoである.



例として,二順位系を考えよう.非摂動ハミルトニアン
\begin{align*}
 H_0=
 \begin{pmatrix}
  E & 0 \\
  0 & -E \\
 \end{pmatrix}
\end{align*}
及び,摂動ハミルトニアン
\begin{align*}
 V=
 \begin{pmatrix}
  0 & \lambda \\
  \lambda & 0\\
 \end{pmatrix}
\end{align*}
を考えよう.ただし$\lambda < E$とする.元のハミルトニアン$H_0$のエネルギー$\pm E$から,摂動によってエネルギーが$\pm \sqrt{E^2+\lambda^2}$と変化する

実際問題としては,$H_0$は解けるけれども$H$は解けない,という様な場合に摂動論を用いることになる.


\section{摂動の問題設定}
わかりやすさのため,全ハミルトニアンを
\begin{align*}
 H=H_0+\lambda V
\end{align*}
と書く.ただし$H_0$が非摂動ハミルトニアン,$V$が摂動ハミルトニアン,$\lambda$は微笑パラメータである.非摂動ハミルトニアン$H_0$の固有値$E_n$及び固有ベクトル$\ket{\psi_n}$が求まっているとする.つまり
\begin{align*}
 H_0\ket{\psi_n}=E_n\ket{\psi_n}
\end{align*}
と書けるとするときに,全ハミルトニアンの固有値,固有状態
\begin{align*}
 H\ket{\phi_n}=\epsilon_n\ket{\phi_n}
\end{align*}
を$\lambda$について展開する問題を考えよう.
\begin{align*}
 \epsilon_n&=E_n+\lambda E_n^1+\lambda^2E_n^2+\cdots \\
 \ket{\phi_n}&=\ket{\psi_n}+\lambda\ket{\psi_n^1}+ \lambda^2\ket{\psi_n^2}+\cdots \\
\end{align*}

ここでは二つのやり方を紹介することにする.


\section{解き方1}
一つ目のやり方は愚直に展開式を代入することである.例えば最低次が知りたければ,$\lambda$の$1$次まで展開すれば良いから,
\begin{align*}
 \left(H_0+\lambda V\right)\left(\ket{\psi}+\lambda \ket{\psi^1}\right)\simeq\left(E_n+\lambda E_n^1\right)\left(\ket{\psi}+\lambda \ket{\psi^1}\right) \\
 H_0\ket{\psi}+\lambda \left(H\ket{\psi^1}+V\ket{\psi}\right)=E_n\ket{\psi}+\lambda\left(E_n\ket{\psi^1}+E_n^1\ket{\psi}\right)
\end{align*}
これからまず$\lambda$の$0$次については
\begin{align*}
 H_0\ket{\psi_n}=E_n\ket{\psi_n}
\end{align*}
という$H_0$についてのシュレーディンガー方程式になる.

次に$\lambda$の$1$次については
\begin{align*}
 H_0\ket{\psi^1}+V\ket{\psi}=E_n\ket{\psi^1}+E_n^1\ket{\psi}
\end{align*}
左から$\bra{\psi}$をかけると
\begin{align*}
  \braket{\psi|H_0|\psi^1}+\braket{\psi|V|\psi}=E_n\braket{\psi|\psi^1}+E_n^1\braket{\psi|\psi}
\end{align*}
こうして,$\lambda$の$1$次から一次のエネルギーを計算できた.

さらに$1$次の波動関数を求めるためには非摂動固有ベクトルによる展開
\begin{align*}
 \ket{\psi_n^1}=\sum_{i}\braket{\psi_i|\psi_n^1}\ket{\psi_i}
\end{align*}
が分かれば良いのでこの係数を求める.      





次に$\lambda$の$2$次の項を考えよう.その方程式は
\begin{align*}
 V\ket{\psi^1}+H_0\ket{\psi^2}=E_n^2\ket{\psi}+E_n^1\ket{\psi^1}+E_n\ket{\psi^2}
\end{align*}

結果をまとめよう.摂動のエネルギーは$2$次までは
\begin{align*}
 \epsilon_n\simeq E_n+\lambda\braket{\psi|H_0|\psi}+\lambda^2\sum_{i\neq n}\frac{\left|\braket{\psi_n|V|\psi_i}\right|^2}{E_n-E_i}
\end{align*}
状態ベクトルは$1$次までで
\begin{align*}
 \ket{\phi_n}= \ket{\psi_n}+
\end{align*}
である.

\section{解き方2}
解き方1では,なかなかどういう仕組みになっているかわかりにくいので,次にもう少し統計的な方法を紹介しよう.





\end{document}