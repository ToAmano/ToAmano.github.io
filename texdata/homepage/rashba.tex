\documentclass[a4j]{jarticle}

\title{ラシュバ効果}
\usepackage{amsmath}	% required for `\align*' (yatex added)
\usepackage{braket}	% required for `\braket' (yatex added)
\begin{document}
\maketitle
 \section{ラシュバ効果とは}
ラシュバ効果(rashba effect)とは,二次元電子系に垂直な電場を加えると,スピン軌道相互作用により電子のバンドに分裂が見られる現象のことである.

このような系は半導体の界面などで実現されており,

\section{Diracハミルトニアン}
ラシュバ効果を記述するハミルトニアンのためには,ディラック方程式の非相対論的極限を考えないといけない.細かい導出はまた別の機会に譲るとして,ここでは基本的なことについてさらうことにしよう.相対論的なスピン$1/2$の粒子の運動を記述するDirac方程式は
\begin{align*}
 H\psi &=E\psi \\
 H&=c{\bf\alpha}\cdot\left({\bf p}+{\bf eA}\right)+\beta mc^2-e\Phi
\end{align*}
と記述される.ただし$\psi$は$4$つの成分を持つスピノールと呼ばれるものであり,対応してハミルトニアン$H$も$4\times 4$の行列である.具体的には,まず$4\times 4$行列$\alpha_i$および$\beta$が,パウリ行列$\sigma_i$を用いて
\begin{align*}
 &\alpha_i=
\begin{pmatrix}
 0 & \sigma_x \\
 \sigma_x& 0 
\end{pmatrix}
 &\beta=
\begin{pmatrix}
 I & 0 \\
 0 & I
\end{pmatrix}
\end{align*}
と定義されている.さらにこれらをまとめて${\bf\alpha} =\left(\alpha_1,\alpha_2,\alpha_3\right)$とベクトル表記している.また,$A$がベクトルポテンシャル,$c$が光速度,$\Phi$がスカラーポテンシャルである.

$4$成分スピノール$\phi$については,ただ$\psi$を二成分ずつに分けて
\begin{align*}
 \psi = 
 \begin{pmatrix}
  \psi_1 \\
  \psi_2 \\
  \psi_3 \\
  \psi_4 
 \end{pmatrix}
 =
\end{align*}
と書いた時,非相対論極限では上の二成分$\phi$が電子の運動を記述することさえわかっていれば問題ない.

Dirac方程式の非相対論極限を考えるには,速度について展開していけば良い.その結果,上二成分に関する方程式
\begin{align*}
 H'&=\epsilon \\
 H'&=\frac{1}{2m}\left(p+eA\right)^2-e\Phi+\frac{e\hbar}{2m}\sigma B-\frac{ie\hbar}{4m^2c^2}E(p+eA)+\frac{e\hbar}{4m^2c^2}\sigma \left(E\times(p+eA)\right)
\end{align*}
となることが知られている.この二成分はスピン$1/2$とスピン$-1/2$に対応しており,そのまま電子に対するシュレーディンガー方程式と考えることができることが知られている.式のはじめの二項は非相対論的なシュレーディンガー方程式であり,三項目がスピン磁場相互作用(ゼーマン項),そして四項目がトーマス項と呼ばれるもの,最後の五項目がスピン軌道相互作用を表す.

このうちラシュバ効果を記述するのはスピン軌道相互作用の項である.

\section{rashba 効果}
それではいよいよラシュバ効果を考えよう.前節で得られたハミルトニアンにおいてスピン軌道相互作用および運動エネルギーのみを考える.ベクトルポテンシャルはないとするとハミルトニアンは
\begin{align*}
 H=\frac{p^2}{2m}+\frac{e\hbar}{4m^2c^2}\sigma \left(E\times p\right)
\end{align*}
となる.ラシュバ効果のためには,$xy$平面に閉じ込められた二次元電子に$z$軸方向の一様電場
  \begin{align*}
   {\bf E}=\begin{pmatrix}
	    0 \\
	    0 \\
	    E 
	   \end{pmatrix}
  \end{align*}
をかければ良い.するとハミルトニアンは
\begin{align*}
 H=\frac{p^2}{2m}+\frac{e\hbar E}{4m^2c^2}\left(-\sigma_xp_y+\sigma_yp_x\right)
\end{align*}
となる.簡単のため係数を
\begin{align*}
 \alpha =\frac{e\hbar E}{4m^2c^2}
\end{align*}
と置くことにする.
ここに平面波解
\begin{align*}
 \phi =e^{i\left(k_xx+k_yy\right)}\chi
\end{align*}
を代入する.ただし$\chi$はスピン部分であり
\begin{align*}
 \chi=
\begin{pmatrix}
 a \\
 b 
\end{pmatrix}
\end{align*}
と置く.するとシュレーディンガー方程式は
\begin{align*}
 \frac{\hbar^2k^2}{2m}\chi +\frac{e\hbar^2 E}{4m^2c^2}\left(-\sigma_xk_y+\sigma_yk_x\right)\chi=\epsilon \chi \\
 \end{align*}

%   \frac{e\hbar^2 E}{4m^2c^2}
%   \begin{pmatrix}
%    (-k_x+ik_y)b \\
%    (k_x+ik_y)a
%   \end{pmatrix}
%  =-\left(\frac{\hbar^2k^2}{2m}-\epsilon\right)
%   \begin{pmatrix}
%    a \\
%    b
%   \end{pmatrix}
% \end{align*}

\begin{align*}
 \begin{pmatrix}
  \frac{\hbar^2k^2}{2m}-\epsilon & \alpha\hbar (-ik_x-k_y) \\
  \alpha\hbar (ik_x-k_y)&\frac{\hbar^2k^2}{2m}-\epsilon    \\
 \end{pmatrix}
 \chi =0 \\
 \begin{pmatrix}
  \frac{\hbar^2k^2}{2m}-\epsilon & -i\alpha\hbar ke^{-i\theta} \\
  i\alpha\hbar ke^{i\theta}      &\frac{\hbar^2k^2}{2m}-\epsilon    \\
 \end{pmatrix}
 \chi =0 \\
\end{align*}
ただし$k_x+ik_y=ke^{i\theta_k}$と置いた.この解は容易に
\begin{align*}
 &\chi_+=\frac{\sqrt{2}}{2}
\begin{pmatrix}
 -i \\
 e^{i\theta_k}
\end{pmatrix}
&\chi_-=\frac{\sqrt{2}}{2}
\begin{pmatrix}
 i \\
 e^{i\theta_k}
\end{pmatrix}
\end{align*}
と求まる.同時にエネルギー固有値は
\begin{align*}
 \epsilon_{\pm}&=\frac{\hbar^2k^2}{2m}\pm \alpha\hbar k \\
  &=\frac{\hbar^2}{2m}\left(k\pm \frac{m\alpha}{\hbar}\right)^2-\frac{m\alpha^2}{2}
\end{align*}
となる.

これらの結果をまとめよう.スピン軌道相互作用がなければハミルトニアン$H=p^2/2m$の固有状態の自由電子は$\epsilon=\hbar^2k^2/2m$のエネルギーを持ち,スピンに関して縮退している.

しかし一度ここにスピン軌道相互作用が入ると,自由電子のエネルギーは式の形に分裂を起こす.これがラシュバ効果である.

スピン部分について考えよう.それぞれの固有状態のスピンの向きの期待値は
\begin{align*}
 \braket{\chi_+| \sigma |\chi_+} =
 \begin{pmatrix}
  -\sin\theta_k \\
  \cos\theta_k \\
  0
 \end{pmatrix} \\
 \braket{\chi_-|\sigma |\chi_-} =
  \begin{pmatrix}
   \sin\theta_k \\
   -\cos\theta_k \\
  0
  \end{pmatrix} \\
\end{align*}
となりこれらは$xy$平面内にあって,与えられた波数$k$に垂直であり,お互いに逆向きである.この関係を図に示した.

エネルギー分散図にはこのことを模式的に表すために


また,等エネルギー面で見て見ると,図のようになっている.
\end{document}