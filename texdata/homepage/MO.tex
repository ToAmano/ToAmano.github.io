\documentclass[a4j]{jarticle}
\title{HL近似と分子軌道法}
\usepackage{braket}	% required for `\ket' (yatex added)
\usepackage{amsmath}	% required for `\align*' (yatex added)
\begin{document}

\section{二原子系}
量子力学において,一原子+一電子のいわゆる水素様原子は,厳密解がしられているのは有名である.その結果によると水素様原子の状態は$3$つの量子数$l$,$m$,$n$によって記述される.以下これを$\ket{nlm}$と書いて,原子軌道と言うことにする.この時のエネルギーは
\begin{align*}
 E_n=
\end{align*}
と書ける.

これによって,もし一原子の中に多数の電子があるような場合,最も簡単な取り扱いでは電子間の相互作用がないとして,水素様原子の基底$\ket{nlm}$にパウリの排他律に従ってエネルギーの低い順に電子を埋めていく.

今回考えるのは原子も多数ある場合である.これは実際の物質では分子に対応する様な状態であって,当然厳密な解は知られていない.そこで最も簡単な2陽子A,B及び2つの電子から成る系を考える.原子位置はBO近似で動かないとすると,ハミルトニアンは
\begin{align*}
 H=\frac{P_1^2}{2m_e}+\frac{P_2^2}{2m_e}+
\end{align*}
と書ける.
以下,二つの原子間の距離$R$と書くことにする.

\section{HL近似}
まず,$R$が無限大の極限を考えよう.この時,各電子は各陽子に束縛されて二つの水素原子ができると考えるのが自然である.つまり二つの電子は
$\ket{A,1s}$及び$\ket{B,1s}$として記述できるとして良いだろう.
%図が欲しい
HL近似では,軌道がこのまま不変で,$\ket{A,1s}\ket{B,1s}$と書けると近似する.すると電子の入れ替えも考慮して,singlet,tripletを作ることができる.








\section{MO}
さて,次に分子軌道法MOの紹介をしよう.HL近似では電子の軌道は変化しないとしたが,本当はそんなことはない.$R$を小さくしてくると,二つの電子軌道の重なり$\braket{A,1s|B,1s}$が値を持ってきて,これによって軌道が変化することになる.この軌道を,元の原子軌道の(線型)結合で近似的に表すことをMO法という.今考えている例だと軌道を
\begin{align*}
 \ket{MO}=a\ket{A,1s}+b\ket{B,1s}
\end{align*}
と書くことに対応する.これは,系のハミルトニアンを$\ket{A,1s}$及び$\ket{B,1s}$に関して行列表示
\begin{align*}
 H=
 \begin{pmatrix}
  \braket{A|H|A} & \braket{A|H|B} \\
 \braket{B|H|A} & \braket{B|H|B} \\
 \end{pmatrix}
\end{align*}
することになる.

\section{ヒュッケル法}
MO法の係数を決定する方法の一つとしてヒュッケル法がある.この方法では,ハミルトニアンのうち,実際に隣り合う原子間の遷移要素のみを非ゼロとする.例として1-3ブタジエンを考えよう.


\end{document}

