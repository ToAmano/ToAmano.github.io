\documentclass[a4j]{jarticle}
\usepackage{amsmath}	% required for `\align*' (yatex added)
\begin{document}

\section{baker-hausdorff公式2}
今度は演算子$A$および$B$に対するもう一つの公式
\begin{align*}
 e^{A}e^{B}
\end{align*}
を示すことにする.この公式は,c数に対する指数の公式
\begin{align*}
 e^{a}e^{b}=e^{a+b}
\end{align*}
の一般化になっており,特に$[A,B]=c$がc数のときには
\begin{align*}
 e^{A}e^{B}=e^{c/2}e^{A+B}
\end{align*}
という比較的簡単な関係式になる.

\section{証明}


\end{document}