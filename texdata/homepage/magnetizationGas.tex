\documentclass[a4j]{jarticle}
\title{気体の磁性}
\usepackage{amsmath}	% required for `\align*' (yatex added)
\usepackage{braket}	% required for `\braket' (yatex added)
\begin{document}
\section{気体の磁性}
今回は気体の常磁性,反磁性について考えよう.磁性について考えるためには磁化$M$を求め,そこから磁化率
\begin{align*}
 \kappa =\lim_{B\to 0}\frac{\partial M}{\partial B}
\end{align*}
を調べればよい.第一近似では磁化$M$を,

\section{条件設定}
簡単のためにいくつかの仮定を置く.まず気体は十分希薄であり,各気体分子にかかる有効磁場が外部磁場と等しいとする.これの意味するところは,あまりに気体が密だと,各気体分子にかかる有効磁場が周りの気体分子によって変化してしまって扱いが大変だからである.この辺の事情は誘電体を考えるときにも出てくる話である.加えて,気体分子間には相互作用がないとする.すなわちハミルトニアンは
\begin{align}
 H=\sum_{i}\frac{1}{2M}\left(p-ZeA\right)^2\label{150142_27Jun19}
\end{align}
とかけると仮定する.こうすることで分配関数を求めるには一体ハミルトニアン$H_i$のみ考えれば十分である.


磁化率を求めるには気体に対して$z$軸方向に一定の弱い磁場
\begin{align*}
 B=(0,0,B)
\end{align*}
を加えた時を考えればよい.ベクトルポテンシャルを単純ゲージで
\begin{align*}
 A=r\times B
\end{align*}
と置く.この時$A$は横波条件
\begin{align*}
 \nabla A =0
\end{align*}
を満たしており,結果として$Ap=pA$である.

\section{摂動論}
そこでハミルトニアン\ref{150142_27Jun19}で磁場$B$を摂動とみなしてエネルギー変化を計算しよう.まずハミルトニアンを
\begin{align*}
 H_i=\sum_{i}\frac{P^2_i-pA-Ap+A^2}{2M} \\
=\sum_{i}\frac{P^2_i-2pA+A^2}{2M} \\
\end{align*}
と分解する.初項が磁場のない時のハミルトニアン,2項目は軌道ー磁場相互作用,3項目は磁場によるエネルギー補正である.$H_1$および$H_2$を摂動とみなすと,磁場の$2$次まででのエネルギーは
\begin{align*}
 E&=E_0+\braket{\psi_n|H_1+H_2|\psi_n}+\sum_{j\neq n}\frac{|\braket{\psi_n | H_1 |\psi_j}|^2}{E_n^0-E_j^0} \\
&=
\end{align*}
となる.ここから統計和を計算すると,


\end{document}