\documentclass[a4j]{jarticle}

\title{baker-hausdolff formula}

\usepackage{amsmath}	% required for `\align*' (yatex added)
\begin{document}
\section{baker hausdolff公式}
baker hausdolff公式とは,演算子$A$,$B$に対する展開公式
\begin{align*}
 e^{B}Ae^{-B}=A+[B,A]+\frac{1}{2}\left[B,[B,A]\right]+\frac{1}{3!}\left[B,\left[B,[B,A]\right]\right]+\cdots +\frac{1}{n!}\left[B,\cdots,[B,A]\right]+\cdots
\end{align*}
のことである.ただし,$n$番目の項にはBが$n-1$個含まれている.

物理学では左辺で$B=iHt/\hbar$とするとハイゼンベルグ表示になることから,ハイゼンベルグ形式の演算子を変形するときに用いられることが多い.

例えば簡単な例として調和振動子を考えてみよう.ハミルトニアンは
\begin{align*}
 H=\sum_{i=0}^{\infty}\hbar\omega_i\left(a^{\dagger}_ia_i+\frac{1}{2}\right)
\end{align*}
であり,交換関係
\begin{align*}
 \left[a_i,H\right]=\hbar\omega_i\left[a_i,a^{\dagger}_ia_i\right] =\hbar\omega_ia_i
\end{align*}
を満たす.従ってもう一回$H$と交換関係をとると,係数$-\hbar\omega_i$がかかるだけで再び$a_i$が出てくることになる.
\begin{align*}
 \left[H,\left[H,a_i\right]\right]=-\hbar\omega_i\left[a_i,H\right]=\left(-\hbar\omega_i\right)^2a_i
\end{align*}
これらをbaker hausdolffの公式に代入すると
\begin{align*}
 a_i(t)=e^{iHt/\hbar}a_ie^{-iHt/\hbar}=\sum_{n=0}^{\infty}\frac{(-it\hbar\omega_i/\hbar)^n}{n!}a_i=e^{-i\omega_it}a_i
\end{align*}
となり,これはハイゼンベルグ方程式を解いて得られる結果と一致する.

\section{証明}
左辺を愚直に展開することでも証明することができるが,ここではもう少し綺麗な方法で証明しよう.関数
\begin{align*}
 f(t)=e^{tB}Ae^{-tB}
\end{align*}
をTaylor展開して,最後に$t=1$と置くという方法である.一回微分すると
\begin{align*}
 f'(t)&=Be^{tB}Ae^{-tB}-e^{tB}A(-B)e^{-tB} \\
 &=e^{tB}\left(BA-AB\right)e^{-tB} \\
 &=e^{tB}[B,A]e^{-tB}
\end{align*}
従って,もう一回微分すると,$A$を$[B,A]$に置き換えれば良いので
\begin{align*}
 f''(x)=e^{tB}\left[B,[B,A]\right]e^{-tB}
\end{align*}
となる.以下はこの繰り返しであるから,$f(t)$のTaylor展開は
\begin{align*}
 f(t)&=f(0)+f'(0)t+\frac{1}{2!}f''(t)t^2+\frac{1}{3!}f'''(0)t^3+\cdots \\
&=A+[B,A]t+\frac{1}{2!}\left[B,[B,A]\right]t^2+\frac{1}{3!}\left[B,\left[B,[B,A]\right]\right]t^3+\cdots 
\end{align*}
となる.ここに$t=1$を代入すればbaker hausdolffの公式
\begin{align*}
 e^{B}Ae^{-B}=A+[B,A]+\frac{1}{2}\left[B,[B,A]\right]+\cdots
\end{align*}
となる.
\end{document}


