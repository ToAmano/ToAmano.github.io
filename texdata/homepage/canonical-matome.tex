\documentclass[a4j]{jarticle}

\title{カノニカル分布の公式まとめ}
\usepackage{amsmath}	% required for `\align*' (yatex added)
\begin{document}
\maketitle

\section{カノニカル分布のまとめ}
今回は,カノニカル分布における分配関数$Z$と各種の熱力学量の関係をまとめよう.


\section{必要になる熱力学関係式}
必要になる熱力学の関係式を先にまとめて置く.まず,全微分の式
\begin{align}
 \mathrm{d}F=-S\mathrm{d}T-p\mathrm{d}V+\mu \mathrm{d}N\label{1}
\end{align}
および$U$からのルジャンドル変換
\begin{align}
 F=U-TS\label{2}
\end{align}
が基本的である.従って
\begin{align}
 S=-\frac{\partial F}{\partial T}=k_B\beta^2\frac{\partial F}{\partial \beta }\label{3}
\end{align}
\ref{2}および\ref{3}から,エネルギーは
\begin{align*}
 U&=F+TS \\
&=F+\frac{1}{k_B\beta }k_B\beta^2\frac{\partial F}{\partial \beta } \\
&=F+\beta\frac{\partial F}{\partial \beta } \\
&=\frac{\partial}{\partial \beta }\left(\beta F\right)
\end{align*}
と書ける事になる.


\section{カノニカル分布}
カノニカル分布では,密度行列が
\begin{align*}
 \rho =\frac{e^{-\beta H}}{Z}
\end{align*}
ただし,分配関数$Z$は$\rho$の規格化から求まる.
\begin{align*}
 Z=\mathrm{Tr}e^{-\beta H}=<e^{-\beta H}>=\sum e^{-\beta E_i}
\end{align*}

ある量$\hat{A}$の期待値
\begin{align*}
 <\hat{A}>=\mathrm{Tr}(\hat{\rho}\hat{A})=\frac{<e^{-\beta H}A>}{Z}
\end{align*}

特にエネルギーの期待値(これはすなわちエネルギーに他ならない)
\begin{align*}
 E=\frac{\sum E_ie^{-\beta E_i}}{Z}=-\frac{\partial }{\partial \beta}\log Z
\end{align*}
および,エネルギーの揺らぎは
\begin{align*}
 (\Delta E)^2=\frac{\sum (E_i-E)^2e^{-\beta EE_i}}{Z}=\frac{\sum E_i^2e^{-\beta E_i}}{Z}-\frac{\sum E^2e^{-\beta E_i}}{Z}=\frac{Z''-Z'Z'}{Z} \\
 =\frac{\partial^2}{\partial \beta^2}\log Z
\end{align*}
従ってエネルギーの揺らぎは十分に小さいということができる.


ヘルムホルツ自由エネルギーは
\begin{align*}
 E=\frac{\partial}{\partial \beta}(\beta\log F)
\end{align*}
より
\begin{align*}
 F=-\frac{1}{\beta}\log Z
\end{align*}
ヘルムホルツ自由エネルギーが求まれば,原理的に全ての熱力学量が求まる.ヘルムホルツ自由エネルギーは分配関数$Z$がわかれば定まるので,結局,分配関数$Z$から全ての熱力学量が求まることになる.


エントロピーは,ボルツマンの原理より分布関数の対数の平均値で
\begin{align*}
 S=&-k_B<\log \hat{\rho}> \\
 =&-k_B\sum \rho_i\log(e^{-\beta E_i}/Z) \\
 &=-k_B\sum( -\beta E_i\rho_i-\log(Z)\rho_i) \\
 &=\beta k_BE+k_B\log(Z) \\
 &=k_B\beta^2\frac{\partial F}{\partial \beta}
\end{align*}

熱容量は
\begin{align*}
 C=T\frac{dS}{dT}=-\beta\frac{\partial S}{\partial \beta}=k_B\beta^2\frac{\partial^2}{\partial \beta^2}\log Z
\end{align*}
従って,エネルギー揺らぎと熱容量には
\begin{align*}
 C=k_B\beta^2(\Delta E)^2
\end{align*}
の関係がある.

圧力P,化学ポテンシャル$\mu$は
\begin{align*}
 P=\frac{1}{\beta}\frac{\partial}{\partial V}\log Z\\
  \mu=-\frac{1}{\beta}\frac{\partial}{\partial N}\log Z\\
\end{align*}

\section{例1:古典理想気体}

\section{例2:}

\end{document}