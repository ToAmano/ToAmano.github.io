\documentclass[a4j]{jarticle}
\title{スピンモデルの常磁性}
\usepackage{amsmath}	% required for `\align*' (yatex added)
\begin{document}
\section{相互作用のないスピンモデル}
どのように常磁性が発現するかを見る簡単な例として,$N$個の相互作用しないスピン$S$を考えて,ハミルトニアンをゼーマン項
\begin{align*}
 H_{all}=-g\mu_B\sum_{i}B\dot S
\end{align*}
と書く.ただし定数$g\mu_B$は正とする.このように係数をおけば,スピンは磁場と平行になる方がエネルギー的に安定なので常磁性となる.
ただし$g$は$g$因子,$\mu_B$はボーア磁子
\begin{align*}
 \mu_B=
\end{align*}
さらにこのように書いた場合,スピン演算子は無次元になる.

%ここにスピンモデルの絵が欲しい.



統計和を考える際には,相互作用がないことから一体ハミルトニアンを考えれば良いので,以下では一体ハミルトニアン
\begin{align*}
 H=-g\mu_BB\dot S
\end{align*}
を考える.磁場は$z$方向にかかっているとすれば
\begin{align*}
 H=-g\mu_BBS_z
\end{align*}
となる.

\section{スピン1/2の場合}
まずはよく知られたスピン$1/2$の$2$準位系の計算をやってみよう.この場合$S_z=\pm 1/2$に注意して分配関数は
\begin{align*}
 Z&=\sum_{n}e^{-\beta E_n} \\
&=e^{-\beta g\mu_BB/2}+e^{\beta g\mu_BB/2} \\
&=2\cosh\left(\frac{\beta g\mu_BB}{2}\right)
\end{align*}
と計算できる.従って一スピンあたりの自由エネルギー$F$は
\begin{align*}
 F&=-\frac{1}{\beta}\log Z \\
&=-\frac{1}{\beta}\log\left(2\cosh\left(\frac{\beta g\mu_BB}{2}\right)\right)
\end{align*}
一スピンあたりの磁化$M$は
\begin{align*}
 M&=-\frac{\partial F}{\partial B} \\
&=\frac{g\mu_B}{2 }\tanh\left(\frac{\beta g\mu_BB}{2}\right)
\end{align*}
従って一スピンあたりの帯磁率$\chi$は(有限温度で)
\begin{align*}
 \chi&=\lim_{B to 0}\frac{\partial M}{\partial B} \\ 
 &=\frac{g^2\mu_B^2\beta}{2}
\end{align*}
であって,確かにこれは常磁性になっている.

簡単な検算として,低温度極限$\beta\to\infty$では$M=g\mu_B/2$となるが,これは低温では全てのスピンが上向きになって磁化が最大となることと対応している.逆に高温極限ではスピンの向きは完全にランダムになって,磁化は$0$になってしまうことになる.


\section{分配関数の計算とブリルアン関数}
スピン$J$の物体では$S_z$は$J$から$-J$まで$2J+1$個の値をとるから,分配関数は
\begin{align*}
 Z&=\sum_{s=-J}^{J}e^{\beta g\mu_BB s} \\
 &=\frac{e^{-\beta g\mu_BBJ}-e^{+\beta g\mu_BB(J+1)}}{1-e^{\beta g\mu_BB}} \\
 &=\frac{e^{-\beta g\mu_BB(J+1/2)}-e^{\beta g\mu_BB(J+1/2)}}{e^{-\beta g\mu_BB/2}-e^{\beta g\mu_BB/2}}
\end{align*}
ここでわかりやすさのため無次元定数$x$を
\begin{align*}
 x=\beta g\mu_BJB
\end{align*}
と置く.これはゼーマンエネルギーと逆温度の積である.分配関数を書き直して
\begin{align*}
 Z&=\frac{e^{x(1+1/2J)}-e^{-x(1+1/2J)}}{e^{x/2J}-e^{-x/2J}} \\
&=\frac{\cosh\left(x(1+1/2J)\right)}{\cosh\left(x/2J\right)}
\end{align*}

すると一スピンあたりの自由エネルギー$F$は
\begin{align*}
 F&=-\frac{1}{\beta}\log Z \\
  &=-\frac{1}{\beta}\left(\log\cosh\left(\frac{2J+1}{2J}x\right)-\log\cosh\left(\frac{x}{2J}\right)\right)
\end{align*}
ここで,ブリルアン関数$B_J(x)$を
\begin{align*}
 B_J(x)=\frac{2J+1}{2J}\coth\left(\frac{2J+1}{2J}x\right)-\frac{1}{2J}\coth\left(\frac{1}{2J}x\right)
\end{align*}
と定義する.ただし$\coth x$は$1/\tanh x$である.すると自由エネルギーとの間に
\begin{align*}
 \frac{\partial F}{\partial x}=-\frac{1}{\beta }B_J(x)
\end{align*}
の関係があることに注意しよう.一スピンあたりの磁化$M$は
\begin{align*}
 M&=-\frac{\partial F}{\partial B} \\
&=-\frac{\mathrm{d} x}{\mathrm{d}B}\frac{\partial F}{\partial x} \\
&=-\frac{-1}{\beta }\beta g\mu_BJ\left(\frac{2J+1}{2J}\coth\left(\frac{2J+1}{2J}x\right)-\frac{1}{2J}\coth\left(\frac{x}{2J}\right)\right)\\
&=g\mu_BJB_J(x)
\end{align*}
と,ブリルアン関数を用いて綺麗にかける.ここで$J=1/2$の場合に上でやった場合に一致していることを確認しよう.

\section{熱力学量}
次に,エントロピーを計算しよう.自由エネルギー$F$を用いて
\begin{align*}
 S&=k_B\beta^2\frac{\partial F}{\partial \beta} \\
  &=k_B\beta^2\frac{\partial x}{\partial \beta}\frac{\partial F}{\partial x} \\
  &=k_B\beta^2 g\mu_BJB
\end{align*}
と書ける.比熱は
\begin{align*}
 C=-\beta \frac{\partial S}{\partial \beta }
\end{align*}
となる.

以上のように,基本的な物理量は全てブリルアン関数で書けることになる.具体的にブリルアンを図示してみると図のようになっている.
%ブリルアン関数の図示


\section{断熱消磁}
エントロピーの式から,断熱消磁という有名な冷却方法を理解できる.


\section{ショットキー比熱}
比熱の式をいくつかのスピン$J$について図示したものが下図である.
%図

これを見ると,小さい$J$で


\section{古典的な極限}
次に古典的な極限を考えよう.古典論では軌道角運動量は離散化されず連続的な値を取ることができる.



\end{document}