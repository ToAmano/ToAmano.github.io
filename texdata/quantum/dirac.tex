\documentclass[a4j]{jarticle}
\title{ディラック方程式あれこれ}
\author{天野}
\date{}
\usepackage{amsmath,braket}	% required for `\align*' (yatex added)
\begin{document}
\maketitle
\tableofcontents
 \section{角運動量復習}
 \subsection{軌道角運動量とスピン角運動量}
まずは角運動量についての復習から始めるのが良いだろう.角運動量には軌道角運動量Lとスピン角運動量Sがあって,軌道角運動量は古典的な角運動量$L=r\times p$の量子論への拡張であり,スピン角運動量は粒子の持つスピンに起因する運動量である.これらの和を全角運動量と言って$J=L+S$で表す.

\subsection{軌道角運動量}
親しみやすい軌道角運動量について述べていく.すなわち古典的な角運動量$L=r\times p$を正準量子化$p\to-i\hbar\nabla$して
\begin{align*}
 \hat{L}=\hat{r}\times\hat{p}
\end{align*}
とするのである.以下この演算子から作られる$L^2$及び$L_z$の固有関数及び固有値を求めることにする.(なぜ$L^2$と$L_z$かは議論の成り行きを見ていけばわかると思う.)そのためには球座標に移った方がやりやすい.すると
\begin{align*}
 =\frac{\partial x}{}
\end{align*}
などから,計算を実行して(練習問題)


\subsection{全角運動量}
一般に球対称ポテンシャルの(従って自由粒子も含む)粒子の状態は,エネルギーE,及びL,$L_z$を指定すれば一意に定まる.これを波数を指定して定める平面波表示との対比で球面波表示という.後々の便宜のためL,Sの固有値を
\begin{align*}
 L=l(l+1)\hbar^2\\
 L_z=m\hbar
\end{align*}
とおいて,球面波表示を$\ket{E,l,m}$と書くことにする.以下で,演算子$\hat{L}$,$\hat{S}$の球面波表示ケットに対する作用を見ていくことにしよう.

角運動量の性質を定めるのに,以下の代数を用いる.(ここの議論は後で追加するかも)
すなわち,
\begin{align*}
 [L_i,L_j]=i\hbar\epsilon_{ijk}L_k 	   
\end{align*}
とする.これから以下のように全ての議論が出てきてうまくいく.まず,

\subsection{パウリ行列}
  スピン角運動量
\begin{align*}
 S=s(s+1)\hbar^2\\
 S_z=m\hbar
\end{align*}
に対して$s=1/2$の時は,全スピン$S=3\hbar^2/4 $及び$m=\pm 1/2$から
\begin{align*}
 S_z=\pm\frac{1}{2}\hbar
\end{align*}
である.ゆえに行列表示
\begin{align*}
 \braket{s',m'|S^2|s,m}=\\
 \braket{s',m'|S_z|s,m}=\\
 \braket{s',m'|S_{\pm}|s,m}=
\end{align*}
は,簡単に
\begin{align*}
 a
\end{align*}
とかけて,これから
\begin{align*}
 S_i=\frac{\hbar}{2}\rho_i
\end{align*}
によってパウリの行列$\rho_i$を定義すると
\begin{align*}
 \rho_1=
\end{align*}
を得る.以下このスピン行列のいくつかの性質を見ていくことにする.まず,一番わかりやすいのは
\begin{align*}
 \rho_i^2=1
\end{align*}
である.これは実際に計算すればすぐに確認できる.次に,添字の違うもの同士の掛け算では
\begin{align*}
 a
\end{align*}


\section{DIRAC方程式の導出}
\subsection{導出1}
まずは割と自然な方の導出から行く.日相対論では,vector potentialが存在している時,電荷$e$を持つ自由粒子のハミルトニアンが置き換え
\begin{align*}
 p\to p-\frac{e}{c}A
\end{align*}
で記述できる.これにうまくスピンを取り込むために,さらなる置き換え$p\to p\cdot\sigma$を考えてみよう.これは,vector potentialが存在しなければ,$\sigma$の公式
\begin{align*}
 (\sigma a)(\sigma b)=ab+i\sigma (a\times b)
\end{align*}
を用いて
\begin{align*}
 2mH=(p\sigma)^2=p^2+i\sigma (p\times p)=p^2
\end{align*}
と,今までどおりの自由粒子ハミルトニアンを与える一方,vector potentialが存在すれば簡単のため$q=e/c$とおいて
\begin{align*}
 2mH&=\left\{(p-qA)\sigma\right\}^2
 &=(p\sigma)^2-2q(p\sigma)(A\sigma)+(qA\sigma)^2\\ 
 &=p^2+(qA)^2-2q\left[pA+i\sigma(p\times A)\right]\\
 &=p^2+(qA)^2-2qpA-2qi\rho(-i\hbar B) \\
 &=(p-qA)^2-\frac{2\hbar e}{c}\sigma B 
\end{align*}
となり,スピン磁場相互作用
\begin{align*}
 H_{spin}=2\cdot\frac{\hbar e}{2mc}\sigma B
\end{align*}
を得る.これは$g$因子として電子の場合に正しい$g=2$を与えており,どうやら正しい表式に辿り着いていると考えて良さそうである.そこで,この置き換えを相対論の場合に実行してみる.すなわちエネルギーの式
\begin{align*}
 E-(cp)^2=(m_0c^2)^2
\end{align*}
に置き換えを施して
\begin{align*}
 i\hbar\frac{\partial}{\partial t}=c^2()
\end{align*}

\section{ガンマ行列の性質}
\section{dirac方程式の非相対論的極限}
次にdirac方程式のなじみやすい部分から考えて行くことにする.その一つに非相対論への極限があげられる.非相対論的な極限では,$E\sim mc^2$,すなわち$mc^2$が他に比べて十分大きいとすれば良い.


\section{Dirac方程式の共変性}
Dirac方程式の導出2で得られた表式とガンマ行列の各種の性質を用いてDirac方程式の共変性について考える.
  \subsection{ローレンツ変換}
  復習になるがローレンツ変換から行こう.



\section{自由粒子}
\subsection{平面波}
\subsection{球面波}

\section{一次元粒子の運動}
\subsection{概論}
ここでは,ポテンシャル$V(x)$の元で運動する一次元粒子について考える.ポテンシャル$V(x)$がある時の取り扱いはまだやっていなかったような気がするが,この時は粒子の運動エネルギーが$E-V$となることに注意して
\begin{align*}
 E-V=(cp)^2+(mc^2)^2
\end{align*}
に量子化手続きをすればよく,結局dirac方程式として
\begin{align*}
 i\hbar \phi =H\phi\\
 H=c\alpha_1p_1+mc^2\beta +V(x)
\end{align*}
を得る.とりあえずは定常状態を考えるので時間部分を
\begin{align*}
 \phi(t,x)=e^{-iEt/\hbar}\phi (x)
\end{align*}
と分離すれば,定常状態のdirac方程式
\begin{align*}
 \left\{c\alpha_1p_1+mc^2\beta+V(x)\right\}\phi (x)=E\phi (X)
\end{align*}
を得る.さらに波動関数を
\begin{align*}
 \phi(x)=
 \begin{pmatrix}
  \phi_u(x)\\
  \phi_d(x)
 \end{pmatrix}
\end{align*}
と二次元スピノールに分離すれば,二本の方程式
\begin{align*}
 a
\end{align*}
を得る.この演算子部分は二次元スピノールに関係していないので,さらにスピノールを定数として分離できて
\begin{align*}
 a
\end{align*}
最終的にはスカラー関数$F(x)$,$G(x)$の方程式が
\begin{align*}
 a
\end{align*}
と得られる.特に$V(x)$が$x$に寄らない定数であれば,$F$に関する方程式は簡単に
\begin{align*}
 F''(x)=\left\{(E-V)^2+m^2\right\}F
\end{align*}
とかける.これが基本となる方程式である.これは今まで用いてきた非相対論的なシュレーディンガー方程式
\begin{align*}
 \left(\frac{-\hbar^2}{2m}\Delta\right)
\end{align*}


\subsection{階段ポテンシャル}
Dirac方程式はポテンシャルが定数から構成されていない場合方程式が複雑になって解き難い.まずは一番簡単な階段ポテンシャル
\begin{align*}
 V(x)=
 \begin{cases}
  0&(x<0)\\
  V_0&(x>0) 
 \end{cases}
\end{align*}
を考える.ただし$V_0$は正定数とする.この場合の$F$についての方程式は
\begin{align*}
 a
\end{align*}
となる.


  \subsection{井戸型ポテンシャル}

\section{球対称ポテンシャル}
\subsection{概論}
\subsection{角井戸ポテンシャル}
\subsection{水素原子}





\end{document}