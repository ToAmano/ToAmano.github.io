\documentclass[a4j]{jarticle}
\title{}

\usepackage{amsmath}	% required for `\align*' (yatex added)
\usepackage{bm}	% required for `\bm' (yatex added)
\begin{document}
\section{ディラック方程式の非相対論的な極限}
ディラック方程式がどのような意味を持っているのかを調べるために,非相対論的な極限を考えましょう.簡単のため時間に依存しない電磁場を想定し,ハミルトニアン形式のディラック方程式
\begin{align*}
\left[\gamma^{\mu}\left(i\hbar\partial_{\mu}+eA_{\mu}\right)+mc\right]\phi=0
\end{align*}
\begin{align*}
 i\hbar\partial_t\phi=\left[-c\bm{\alpha}\cdot\left(-i\hbar\nabla-e\bm{A}\right)-eA_0+mc\bm{\beta}\right]
\end{align*}
に平面波解
\begin{align*}
 \phi(r,t)=u\exp\left[\frac{i}{\hbar}\left(\bm{p}\bm{r}-Et\right)\right]
\end{align*}
を代入し,スピノール$u$を上下の二成分毎に分解して$(u_A,u_B)$と書きます.すると$\bm{\alpha}$および$\bm{\beta}$の具体的な表式
\begin{align*}
 \bm{\alpha}=\left(
 

 \right)
\end{align*}
によって
\begin{align*}
 a
\end{align*}
となります.これから
\begin{align*}
 a
\end{align*}
を得ます.ここで非相対論的な近似
\begin{align*}
 E\neq mc^2
\end{align*}
を行うと,まず第二式より
\begin{align*}
 u_B\neq 2\frac{v}{c}u_A<<u_A
\end{align*}
となり,$u_B$は$u_A$よりも小さいことがわかります.このことから,スピノール$u$の上二成分を大きい成分,下二成分を小さい成分と言います.

変形を続けましょう.




\end{document}