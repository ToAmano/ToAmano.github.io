\documentclass[a4j]{jarticle}

\usepackage{amsmath}	% required for `\align*' (yatex added)
\usepackage{bm}	% required for `\bm' (yatex added)
\begin{document}

\section{相対論的な量子力学概論}

\section{単位系}
相対論的な量子力学では,単位系の取り方と計量の取り方(あるいは計量を採用しない流儀)によっていくつかの表記があり,慣れないうちは大変です.このページではSI単位系を用い,また計量として$g^{\mu\nu}=\diag(-1,1,1,1)$を採用します.素電荷は$e>0$としておきます.記事が進むにつれて計算が煩雑になってきたときに自然単位系に移行することを考えています.


\section{ディラック方程式の導出}
相対論的な粒子のうちスピン$1/2$をもつ粒子(電子など)はディラック方程式
\begin{align*}
 \left(i\hbar\gamma^{\mu}\partial_{\mu}+mc\right)\phi=0
\end{align*}
にしたがって時間発展します.式の具体的な意味は追って説明することにして,これがどのように導かれるかということが今日のお題です.
様々なやり方がありますが,ここではふた通り紹介することにします.一つはアインシュタインの関係式
\begin{align*}
 E^2=(pc)^2+(mc^2)^2
\end{align*}
に機械的な量子化規則を課すもので,かなり綺麗です.こちらでは初めからパウリ行列が組み込まれていて,スピン$1/2$を意識したものになっています.今回の記事ではこちらの方法について紹介します.

もう一つはディラックの元々のアイディアによるもので,アインシュタインの関係式を満たすような一階の方程式を作ろうというものです.
これは次の記事で紹介します.





何れにしても,準備としてシュレーディンガー方程式およびクラインゴルドン方程式に触れておきましょう.


\section{非相対論的な量子力学}
非相対論的な量子力学では,波動関数$\phi$の時間発展はシュレーディンガー方程式
\begin{align*}
 i\hbar\frac{\partial}{\partial t}\phi =-\frac{\hbar^2}{2m}\Delta\phi
\end{align*}
で記述されます.これはどのように導かれるかを思い返してみよう.古典的な粒子のエネルギーと運動量の関係式
\begin{align*}
 E=\frac{p^2}{2m}
\end{align*}
に置いて機械的な置き換え規則
\begin{align*}
 E\to i\hbar\partial_t \\
 p\to -i\hbar\nabla \\
\end{align*}
を採用すれば直ちにシュレーディンガー方程式を得ることができる.


\section{クラインゴルドン方程式}
シュレーディンガー方程式を参考にして,機械的な置き換えをアインシュタインの関係式に適用して
\begin{align*}
 \left(i\hbar\partial_t\right)^2=\left(-ic\hbar\nabla\right)^2+(mc^2)^2 \\
 -\hbar^2\partial_t^2=-c^2\hbar^2\nabla\Delta+(mc^2)^2 \\
 \therefore \left[\frac{\partial t^2}{\partial^2}+\Delta+\left(\frac{mc}{\hbar}\right)^2\right]\phi=0
\end{align*}
を得る.これは波動方程式に余分な項$(mc/\hbar)^2$が加わった形をしており,クラインゴルドン方程式という.$\phi$はスカラーであり,この方程式はスピン$0$の相対論的な粒子の時間発展を表す.大切なのは置き換え規則で容易に波動関数の時間発展を表す微分方程式が得られるということです.


\section{パウリ方程式}
次にスピン$1/2$の非相対論的な粒子を考えてみましょう.これは以前にもやっていますが,置き換え規則に
\begin{align*}
 \bm{p}\to \bm{p}\cdot \bm{\sigma} 
\end{align*}
を追加すれば良いです.実際自由粒子の場合にはこれはシュレーディンガー方程式になりますが,電磁場がある場合のいつもの置き換えと合わせて
 \begin{align*}
  \bm{p}\to \left(\bm{p}+q\bm{A}\right)\cdot \bm{\sigma} 
 \end{align*}
 とすることで,ハミルトニアンは
 \begin{align*}
  H=\frac{1}{2m}\left[\left(-i\hbar\nabla +q\bm{A}\right)\cdot \bm{\sigma}\right]^2
 \end{align*}
 となります.ここに有名公式
 \begin{align*}
  \left(\bm{a}\cdot\bm{\sigma}\right)\left(\bm{b}\cdot\bm{\sigma}\right)=\bm{a}\cdot\bm{b}+i\bm{\sigma}\cdot\left(a\times b\right)
 \end{align*}
 を用いて
 \begin{align*}
  H=-\frac{\hbar^2}{2m}\Delta+2\cdot\frac{2m}{\hbar q}\bm{\sigma}\cdot\bm{B}
 \end{align*}
とパウリの場合のハミルトニアンを得る.


\section{ディラック方程式の導出1}
以上のことをまとめよう.スピン$1/2$の場合,古典的な標識から出発して,まず運動量$\bm{p}$を$\bm{p}\cdot\bm{\sigma}$に置き換える.次に量子化の規則を採用すれば良い.

アインシュタインの関係式から





\end{document}