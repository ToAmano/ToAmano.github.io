\documentclass[a4j]{article}




\begin{document}



\section{勉強の方法について}
全体として,理解している問題は素早く解くことができ,個々の掛け算や割り算の計算速度も早いです.ただし,初見の問題へのアプローチや計算の正確性に課題を残していると思います.この問題点の解決のために必要だと思うことを述べておきます.

\subsection{図を描く}
どのような問題でも図をかこう.過不足算なら線分図がかけるし,立体図形なら展開図がかけるはず.図をかけば問題文がかんたんに理解できるし,


\subsection{初見の問題も15分は考える}
わからないことを考えるくせをつけよう.はじめてみる問題でも,










\section{計算について}
かけざんやわりざんの速度は早いと思います.だけど計算の正確さや,難しい計算を工夫してとく点が今後の宿題だと思います.
\subsection{計算はゆっくりでよい}
計算は,早さよりも正確さが大切です.とくに筆算で間違えが多いので,ゆっくり正しく計算するくせをつけよう.









\end{document}