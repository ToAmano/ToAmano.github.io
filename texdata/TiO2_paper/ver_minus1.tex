\documentclass[twocolumn,showpacs,prb,amsfonts,amsmath,amssymb,floatfix,groupedaddress]{revtex4-1}
\usepackage{color}
\usepackage{hhline}
\usepackage{mathrsfs}
\usepackage{graphicx}
\usepackage{dcolumn}
\usepackage{bm}% bold math
\usepackage{multirow}
\usepackage{booktabs}
\usepackage{afterpage}
\usepackage{amsmath}
\usepackage{ulem}
\usepackage{physics}


\newcommand{\red}[1]{\textcolor{red}{#1}}
\newcommand{\blue}[1]{\textcolor{blue}{#1}}
\arraycolsep=0.0em
\setlength{\abovecaptionskip}{0mm}
\setlength{\belowcaptionskip}{0mm}
%\setlength{\MidlineHeight}{2pt}


\newcommand{\ph}{\phantom{0}}

\renewcommand{\topfraction}{1.0}
\renewcommand{\bottomfraction}{1.0}
\renewcommand{\dbltopfraction}{1.0}
\renewcommand{\textfraction}{0.1}
\renewcommand{\floatpagefraction}{0.9}
\renewcommand{\dblfloatpagefraction}{0.9}
\begin{document}


\title{Lattice dielectric properties of rutile TiO2 from first principles}
\author{Tomohito Amano$^{1}$ }
\author{Tamio Yamazaki$^{2}$ }
\author{Ryosuke Akashi$^{13}$ }
\author{Terumasa Tadano$^{4}$ }
\author{Shinji Tsuneyuki$^{1}$ }
\thanks{akashi@cms.phys.s.u-tokyo.ac.jp}
\affiliation{$^1$Department of Physics, The University of Tokyo, Hongo, Bunkyo-ku, Tokyo 113-0033, Japan}
\affiliation{$^2$JSR coopration}
\affiliation{$^3$QST}





\date{\today}
\begin{abstract}
ルチルTiO2の格子関数を非調和フォノンの理論を用いて計算した.
\end{abstract}

\maketitle

\section{Introduction}
\begin{itemize}
 \item TiO2の重要性
 \item TiO2の研究の歴史
 \item フォノン非調和理論
\end{itemize}

\section{Theory}



Frequency shift

lifetime
\begin{align}
 \Im \Sigma(\omega) &=\Im \Sigma^{\mathrm{B}}[G_{\rm{NL}},\Phi_3](\omega)  \\
 \gamma^{4ph}_{0j}  &= \Im \Sigma^{4ph}[G_{\rm{NL}},\Phi_4](\omega=\Omega_{\rm{NL}})
\end{align}


dielectric function
\begin{align}
 \epsilon_{\alpha\beta}(\omega)=\epsilon^{\infty}_{\alpha\beta}+\frac{1}{\Omega_0}\sum_{(0,j)}\frac{S^{j}_{\alpha\beta}}{\left(\Omega^{\mathrm{NL}}_{0j}\right)^2-\omega^2+i\Omega^{\mathrm{NL}}_{0j} \Sigma^{\mathrm{B}}_{0j}(\omega)+i \omega \gamma^{4ph}_{0j} }
\end{align}

\section{Results and Discussion}

\subsection{Computational Details}

\subsection{band structure}
% http://coffee.guhaw.com/Entry/350/
% ----- 表:物理定数 ------
\renewcommand{\arraystretch}{1.8}
\begin{table}[h]
 \caption{Calculated and reference structural parameters}
 \label{table:SpeedOfLight}
 \centering
  \begin{tabular}{lccccc}
   \hline\hline
       & $a\,[\mathrm{\AA}]$ & $c\,[\mathrm{\AA}]$ & $u$ & $c/a$ & $V\,[\mathrm{\AA}^3]$ \\
   \hline 
    LDA        & $4.552$  & $2.922$  & $0.3038$  & $0.642$   & \\
    r2SCAN     & $4.602$  & $2.961$  & $0.3046$  & $0.643$   & \\ 
    exp 300K   & $4.5931$ & $2.9589$ & $0.30476$ & $0.644$   & \\ 
    X ray 298K & $4.5937$ & $2.9587$ & $0.30511$ & $0.644$   & \\ 
    exp 15K    & $4.5867$ & $2.9541$ & $0.30469$ & $0.644$   & \\ 
    \hline
   %1638 & Galileo & 二人が離れてランプの光を見る & (音速10倍以上) \\
   %1675 & Roemer & 木星の衛星の観測から & 2 \\
   %1728 & Bradley & 星の収差から & 3.01 \\
   %1849 & Fizeau & 高速に回転する歯車を通過する光を見る & 3.133 \\
   %1862 & Foucault & 高速に回転する鏡の光の角度変化 & 2.99796 \\
  % 今日 & (定義) & & 2.99792458 \\
   \hline
  \end{tabular}
\end{table}
\renewcommand{\arraystretch}{1}
% --------------------
% http://coffee.guhaw.com/Entry/350/
% ----- 表:物理定数 ------
\renewcommand{\arraystretch}{1.8}
\begin{table}[h]
 \caption{ Born Effective Charge}
 \label{table:SpeedOfLight}
 \centering
  \begin{tabular}{lccccc}
   \hline\hline
       & $a[\mathrm{\AA}]$ & $c[\mathrm{\AA}]$ & $u$ & $c/a$ & $V[\mathrm{\AA}^3]$ \\
   \hline 
    LDA        & $4.552$  & $2.922$  & $0.3038$  & $0.642$   & \\
    r2SCAN     & $4.602$  & $2.961$  & $0.3046$  & $0.643$   & \\ 
    exp 300K   & $4.5931$ & $2.9589$ & $0.30476$ & $0.644$   & \\ 
    X ray 298K & $4.5937$ & $2.9587$ & $0.30511$ & $0.644$   & \\ 
    exp 15K    & $4.5867$ & $2.9541$ & $0.30469$ & $0.644$   & \\ 
    \hline
   %1638 & Galileo & 二人が離れてランプの光を見る & (音速10倍以上) \\
   %1675 & Roemer & 木星の衛星の観測から & 2 \\
   %1728 & Bradley & 星の収差から & 3.01 \\
   %1849 & Fizeau & 高速に回転する歯車を通過する光を見る & 3.133 \\
   %1862 & Foucault & 高速に回転する鏡の光の角度変化 & 2.99796 \\
  % 今日 & (定義) & & 2.99792458 \\
   \hline
  \end{tabular}
\end{table}
\renewcommand{\arraystretch}{1}
% --------------------



\begin{itemize}
 \item バンド図r2SCAN, harmonic vs scph+bubble3
 \item バンド図LDA vs r2SCAN
\end{itemize}


\subsection{lifetime and thermal conductivity}

\begin{itemize}
 \item thermal conductivity
\end{itemize}


\subsection{dielectric function}
\begin{itemize}
 \item lifetime
 \item dielectric function
 \item Reflectivity
\end{itemize}


\section{Conclusions}


\section{Appendix}


% \begin{acknowledgments}

% \end{acknowledgments}

% \bibliography{reference}

\end{document}



